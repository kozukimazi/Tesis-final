% Template:     Tesis LaTeX
% Documento:    Archivo principal
% Versión:      3.4.0 (23/08/2024)
% Codificación: UTF-8
%
% Autor: Pablo Pizarro R.
%        pablo@ppizarror.com
%
% Manual template: [https://latex.ppizarror.com/tesis]
% Licencia MIT:    [https://opensource.org/licenses/MIT]

% CREACIÓN DEL DOCUMENTO
\documentclass[
	spanish, % Idioma: spanish, english, etc.	
	letterpaper, oneside
]{book}

% INFORMACIÓN DEL DOCUMENTO
\def\documenttitle {Demonio de Maxwell autónomo en un sistema de tres puntos cuánticos}
\def\documentsubtitle {}
\def\degreetitle {
	Tesis para optar al grado de magíster en ciencias, mención en física
}

\def\universityname {Universidad de Chile}
\def\universityfaculty {Facultad de Ciencias Físicas y Matemáticas}
\def\universitydepartment {Departamento de Física}
\def\universitydepartmentimage {departamentos/uchile2}
\def\universitydepartmentimagecfg {height=3cm}
\def\universitylocation {Santiago de Chile}

% INTEGRANTES, PROFESORES Y FECHAS
\def\documentauthor {Maximiliano Adolfo Bernal Santibañez}
\def\documentdate {\the\year}

\def\portrait {
	\begin{center}
	\vspace{1.5cm} ~ \\
	\MakeUppercase{\textbf{\documenttitle}} ~ \\
	\vspace{1.5cm}
	\MakeUppercase{\degreetitle} ~ \\
	\vfill
	\begin{tabular}{c}
		\MakeUppercase{\textbf{\documentauthor}} \\ \\
		\vspace{1.0cm} \\
		PROFESOR GUÍA: \\
		FELIPE BARRA DE LA GUARDA \\
		\vspace{0.5cm} \\
		MIEMBROS DE LA COMISIÓN: \\
		ALVARO NÚÑEZ VÁSQUEZ\\
		GONZALO GUTIÉRREZ GALLARDO \\
		\vspace{0.5cm} \\
		Este trabajo ha sido parcialmente financiado por: \\
		FONDECYT 1231210 \\
		\vspace{0.5cm} \\
		\MakeUppercase{\universitylocation} \\
		\MakeUppercase{\documentdate}
	\end{tabular}
	\end{center}
}
\def\abstracttable {
	\begin{tabular}{l}
		RESUMEN DE LA TESIS PARA OPTAR \\
		AL GRADO DE MAGÍSTER EN CIENCIAS, \\
		MENCIÓN EN FÍSICA \\
		POR: \MakeUppercase{\documentauthor} \\
		FECHA: \MakeUppercase{\documentdate} \\
		PROF. GUÍA: FELIPE BARRA DE LA GUARDA
	\end{tabular}
}

% IMPORTACIÓN DEL TEMPLATE
% Template:     Tesis LaTeX
% Documento:    Núcleo del template
% Versión:      3.4.0 (23/08/2024)
% Codificación: UTF-8
%
% Autor: Pablo Pizarro R.
%        pablo@ppizarror.com
%
% Manual template: [https://latex.ppizarror.com/tesis]
% Licencia MIT:    [https://opensource.org/licenses/MIT]

% -----------------------------------------------------------------------------
% CONFIGURACIONES
% -----------------------------------------------------------------------------
% Definiciones previas
\usepackage[dvipsnames,table]{xcolor} % Paquete de colores avanzado

% Definición de colores
\definecolor{cardinalred}{RGB}{140, 21, 21}
\definecolor{dkcyan}{RGB}{0, 123, 167}
\definecolor{dkgray}{RGB}{90, 90, 90}
\definecolor{dkgreen}{RGB}{0, 150, 0}
\definecolor{gray}{RGB}{127, 127, 127}
\definecolor{lbrown}{RGB}{255, 252, 249}
\definecolor{lgray}{RGB}{240, 240, 240}
\definecolor{mauve}{RGB}{150, 0, 210}
\definecolor{mitred}{RGB}{161, 0, 47}
\definecolor{ocre}{RGB}{243, 102, 25}

% Definiciones en configuraciones
\def\iitembcirc {\raisebox{0.55\height}{\scriptsize$\bullet$}}
\def\iitembsquare {\raisebox{0.3\height}{\tiny$\blacksquare$}}
\def\iitemdash {\raisebox{0.35\height}{\textendash}}
\def\iitemcirc {\raisebox{0.25\height}{\small$\circ$}}
\def\iitemdiamond {\raisebox{0.25\height}{\small$\diamond$}}

% Ajustes usuario
% Template:     Tesis LaTeX
% Documento:    Configuraciones del template
% Versión:      3.4.0 (23/08/2024)
% Codificación: UTF-8
%
% Autor: Pablo Pizarro R.
%        pablo@ppizarror.com
%
% Manual template: [https://latex.ppizarror.com/tesis]
% Licencia MIT:    [https://opensource.org/licenses/MIT]

% CONFIGURACIONES GENERALES
\def\documentfontsize {12}         % Tamaño de la fuente del documento [pt]
\def\documentinterline {1.0}       % Interlineado del documento [factor]
\def\documentparindent {15}        % Tamaño del indentado de párrafos [pt]
\def\documentparskip {0}           % Tamaño adicional entre párrafos (+/-) [pt]
\def\fontdocument {lmodern}        % Tipografía base, ver soportadas en manual
\def\fonttypewriter {tmodern}      % Tipografía de \texttt, ver manual
\def\fonturl {same}                % Tipo de fuente url {tt,sf,rm,same}
\def\graphicxdraft {false}         % En true no carga las imágenes (modo draft)
\def\pointdecimal {true}           % N° decimales con punto en vez de coma
\def\predocpageromanupper {false}  % Páginas en número romano en mayúsculas
\def\showlayoutlines {false}       % Muestra el layout de la página
\def\showlinenumbers {false}       % Muestra los números de línea del documento
\def\twopagesclearformat {blank}   % Formato nueva página twoside {blank,empty}

% ESTILO HEADER-FOOTER
\def\chapterstyle {style1}         % Estilo de los capítulos (12 estilos)
\def\disablehfrightmark {false}    % Desactiva el rightmark del header-footer
\def\hfstyle {style7}              % Estilo header-footer (16 estilos)
\def\hfwidthcourse {0.35}          % Tamaño máximo del curso en header-footer
\def\hfwidthtitle {0.6}            % Tamaño máximo del título en header-footer
\def\hfwidthwrap {false}           % Activa el tamaño máximo en header-footer

% CONFIGURACIÓN DE LAS LEYENDAS - CAPTION
\def\captionalignment {justified}  % Posición {centered,justified,left,right}
\def\captionfontsize {small}       % Tamaño de fuente de los caption
\def\captionlabelformat {simple}   % Formato leyenda {empty,simple,parens}
\def\captionlabelsep {colon}       % Sep. {none,colon,period,space,quad,newline}
\def\captionlessmarginimage {0.1}  % Margen sup/inf de figura sin leyenda [cm]
\def\captionlrmargin {2}           % Márgenes izq/der de la leyenda [cm]
\def\captionlrmarginmc {1}         % Margen izq/der leyenda dentro de cols. [cm]
\def\captionmarginimage {0}        % Margen vertical entre caption e imagen [cm]
\def\captionmarginimages {0}       % Margen vertical entre caption e images [cm]
\def\captionmarginimagesmc {0}     % Margen vert. entre caption e imagesmc [cm]
\def\captionmarginmultimg {0}      % Margen izq/der leyendas múltiple img [cm]
\def\captionnumcode {arabic}       % N° código {arabic,alph,Alph,roman,Roman}
\def\captionnumequation {arabic}   % N° ecuac. {arabic,alph,Alph,roman,Roman}
\def\captionnumfigure {arabic}     % N° figuras {arabic,alph,Alph,roman,Roman}
\def\captionnumsubfigure {alph}    % N° subfigs. {arabic,alph,Alph,roman,Roman}
\def\captionnumsubtable {alph}     % N° subtabla {arabic,alph,Alph,roman,Roman}
\def\captionnumtable {arabic}      % N° tabla {arabic,alph,Alph,roman,Roman}
\def\captionsubchar {.}            % Carácter entre N° objeto - subfigura/tabla
\def\captiontbmarginfigure {9.35}  % Margen sup/inf de leyenda en figuras [pt]
\def\captiontbmargintable {7}      % Margen sup/inf de la leyenda en tablas [pt]
\def\captiontextbold {false}       % Etiqueta (código,figura,tabla) en negrita
\def\captiontextsubnumbold {false} % N° subfigura/subtabla en negrita
\def\codecaptiontop {true}         % Leyenda arriba del código fuente
\def\equationcaptioncenter {true}  % Ecuaciones están centradas o justificadas
\def\figurecaptiontop {false}      % Leyenda arriba de las imágenes
\def\marginaligncaptbottom {0.1}   % Margen inferior caption en align [cm]
\def\marginaligncapttop {-0.75}    % Margen superior caption en align [cm]
\def\marginalignedcaptbottom {0.1} % Margen inferior caption en aligned [cm]
\def\marginalignedcapttop {-0.75}  % Margen superior caption en aligned [cm]
\def\margineqncaptionbottom {0}    % Margen inferior caption ecuación [cm]
\def\margineqncaptiontop {-0.7}    % Margen superior caption ecuación [cm]
\def\margingathercaptbottom {0.1}  % Margen inferior caption en gather [cm]
\def\margingathercapttop {-0.9}    % Margen superior caption en gather [cm]
\def\margingatheredcaptbottom{0.1} % Margen inferior caption en gathered [cm]
\def\margingatheredcapttop {-0.7}  % Margen superior caption en gathered [cm]
\def\sectioncaptiondelimiter {.}   % Carácter delimitador n° objeto y sección
\def\showsectioncaptioncode {chap} % N° sec código {none,chap,(s/ss/sss/ssss)ec}
\def\showsectioncaptioneqn {chap}  % N° sec ecuac. {none,chap,(s/ss/sss/ssss)ec}
\def\showsectioncaptionfig {chap}  % N° sec figs. {none,chap,(s/ss/sss/ssss)ec}
\def\showsectioncaptionmat {chap}  % N° matemático {none,chap,(s/ss/sss/ssss)ec}
\def\showsectioncaptiontab {chap}  % N° sec tablas {none,chap,(s/ss/sss/ssss)ec}
\def\subcaptionfsize{footnotesize} % Tamaño de la fuente de los subcaption
\def\subcaptionlabelformat{parens} % Formato leyenda sub. {empty,simple,parens}
\def\subcaptionlabelsep {space}    % Sep. {none,colon,period,space,quad,newline}
\def\tablecaptiontop {true}        % Leyenda arriba de las tablas

% CONFIGURACIÓN DEL ÍNDICE
\def\addabstracttobookmarks {true} % Añade el resumen a los marcadores del pdf
\def\addagradectobookmarks {true}  % Añade el agradecimiento a los marcadores
\def\addindexsubtobookmarks {true} % Agrega índice tabla,codigo,etc a marcadores
\def\addindextobookmarks {true}    % Añade el índice a los marcadores del pdf
\def\charafterobjectindex {.}      % Carácter después de n° figura,tabla,código
\def\charnumpageindex {.}          % Carácter número de página en índice
\def\indexdepth {4}                % Profundidad máxima del índice
\def\indexstyle {tf}               % Tipo {f:figura,t:tabla,c:código,e:ecuación}
\def\indextitlemargin {11.4}       % Margen título índice \insertindextitle [pt]
\def\objectchaptermargin {false}   % Activa margen de objetos entre capítulos
\def\objectindexindent {false}     % Indenta la lista de objetos
\def\showappendixsecindex {true}   % Título de la sección de anexos en el índice

% ANEXO, CITAS, REFERENCIAS
\def\apacitebothers {et al.}       % Etiqueta usada en (y otros) con \shortcite
\def\apaciterefcitecharclose {]}   % Carácter final cita apacite
\def\apaciterefcitecharopen {[}    % Carácter inicial cita apacite
\def\apaciterefnumber {false}      % Lista de referencias con números
\def\apaciterefsep {5}             % Separación entre refs. {apacite} [pt]
\def\apaciteshowurl {false}        % Muestra las url en las referencias
\def\apacitestyle {apacite}        % Formato refs. apacite {apa,ieeetr,etc..}
\def\appendixindepobjnum {true}    % Anexo usa n° objetos independientes
\def\backrefpagecite {false}       % Las citas en bibliografía poseen nº de pag.
\def\bibtexindexbibliography{true} % Función \bibliography se inserta en índice
\def\bibtexrefsep {5}              % Separación entre refs. {bibtex} [pt]
\def\bibtexstyle {ieeetr}          % Formato refs. bibtex {apa,ieeetr,etc...}
\def\bibtextextalign {justify}     % Alineac. bibtex {justify,left,right,center}
\def\fontsizerefbibl {\normalsize} % Tamaño fuente al usar \bibliography{file}
\def\natbibrefcitecharclose {]}    % Carácter final cita natbib
\def\natbibrefcitecharopen {[}     % Carácter inicial cita natbib
\def\natbibrefcitecompress {true}  % Comprime refererencias al citar
\def\natbibrefcitesepcomma {true}  % Separador en coma (,) o punto y coma (;)
\def\natbibrefcitetype {numbers}   % Tipo citación {authoryear,numbers,super}
\def\natbibrefsep {5}              % Separación entre referencia {natbib} [pt]
\def\natbibrefstyle {natnumurl}    % Formato refs. natbib {apa,ieeetr,etc...}
\def\stylecitereferences {natbib}  % Estilo refs. {apacite,bibtex,natbib,custom}
\def\twocolumnreferences {false}   % Referencias en dos columnas

% CONFIGURACIONES DE OBJETOS
\def\animatedimageautoplay {true}  % Autoplay en imágenes animadas
\def\animatedimagecontrols {false} % Muestra los controles en imágenes animadas
\def\animatedimageloop {true}      % Hace loops en imágenes animadas
\def\columnsepwidth {2.1}          % Separación entre columnas [em]
\def\defaultimagefolder {img/}     % Carpeta raíz de las imágenes
\def\equationleftalign {false}     % Ecuaciones alineadas a la izquierda
\def\equationrestart {none}        % Reinicio n° {none,chap,(s/ss/sss/ssss)ec}
\def\footnotelmargin {10}          % Margen entre footnote y el número [pt]
\def\footnoterestart {none}        % N° foot. {none,chap,page,(s/ss/sss/ssss)ec}
\def\footnoterulefigure {false}    % Footnote en figuras tienen línea superior
\def\footnoterulepage {true}       % Footnote en páginas tienen línea superior
\def\footnoteruletable {false}     % Footnote en tablas tienen línea superior
\def\footnotetopmargin {0.5}       % Margen superior de los footnote [cm]
\def\footnotetwocolumn {false}     % Footnote en dos columnas
\def\fpremovetopbottomcenter{true} % Elimina espacio vert. al centrar con b!,t!
\def\imagedefaultplacement {H}     % Posición por defecto de las imágenes
\def\marginalignbottom {-0.4}      % Margen inferior entorno align [cm]
\def\marginalignedbottom {-0.2}    % Margen inferior entorno aligned [cm]
\def\marginalignedtop {-0.4}       % Margen superior entorno aligned [cm]
\def\marginaligntop {-0.4}         % Margen superior entorno align [cm]
\def\margineqnindexbottom {-0.9}   % Margen inferior ecuaciones índice [cm]
\def\margineqnindextop {0}         % Margen superior ecuaciones índice [cm]
\def\marginequationbottom {-0.2}   % Margen inferior ecuaciones [cm]
\def\marginequationtop {0}         % Margen superior ecuaciones [cm]
\def\marginfloatimages {-13}       % Margen sup figs. insertimageleft/right [pt]
\def\margingatherbottom {-0.2}     % Margen inferior entorno gather [cm]
\def\margingatheredbottom {-0.1}   % Margen inferior entorno gathered [cm]
\def\margingatheredtop {-0.4}      % Margen superior entorno gathered [cm]
\def\margingathertop {-0.4}        % Margen superior entorno gather [cm]
\def\marginimagebottom {-0.15}     % Margen inferior figura [cm]
\def\marginimagemultbottom {0.25}  % Margen inferior imágenes múltiples [cm]
\def\marginimagemultright {0.5}    % Margen derecho imágenes múltiples [cm]
\def\marginimagemulttop {-0.3}     % Margen superior imágenes múltiples [cm]
\def\marginimagetop {0}            % Margen superior figuras [cm]
\def\marginlinenumbers {7.5}       % Margen izquierdo (\showlinenumbers) [pt]
\def\numberedequation {true}       % Ecuaciones con \insert... numeradas
\def\senumerti {\arabic{enumi}.}   % Estilo enumerate nivel 1
\def\senumertii {\alph{enumii})}   % Estilo enumerate nivel 2
\def\senumertiii{\roman{enumiii}.} % Estilo enumerate nivel 3
\def\senumertiv {\Alph{enumiv})}   % Estilo enumerate nivel 4
\def\sitemizei {\iitembcirc}       % Estilo itemize nivel 1
\def\sitemizeii {\iitemdash}       % Estilo itemize nivel 2
\def\sitemizeiii {\iitemcirc}      % Estilo itemize nivel 3
\def\sitemizeiv {\iitembsquare}    % Estilo itemize nivel 4
\def\sitemsmargini {25}            % Margen ítems nivel 1 [pt]
\def\sitemsmarginii {22}           % Margen ítems nivel 2 [pt]
\def\sitemsmarginiii {18.7}        % Margen ítems nivel 3 [pt]
\def\sitemsmarginiv {17}           % Margen ítems nivel 4 [pt]
\def\sourcecodebgmarginbottom {0}  % Margen inferior del bloque de color [pt]
\def\sourcecodebgmarginleft {0}    % Margen izquierdo del bloque de color [pt]
\def\sourcecodebgmarginright {0}   % Margen derecho del bloque de color [pt]
\def\sourcecodebgmargintop {0}     % Margen superior del bloque de color [pt]
\def\sourcecodefontf {\ttfamily}   % Tipo de letra código fuente
\def\sourcecodefonts {\small}      % Tamaño letra código fuente
\def\sourcecodeilfontf {\ttfamily} % Tipo de letra código fuente inline
\def\sourcecodeilfonts {\small}    % Tamaño letra código fuente inline
\def\sourcecodenumbersep {6}       % Separación entre número línea y código [pt]
\def\sourcecodenumbersize {\tiny}  % Tamaño fuente número línea
\def\sourcecodeskipabove {0.75}    % Espacio sobre recuadro código [em]
\def\sourcecodeskipbelow {0.95}    % Espacio bajo recuadro código [em]
\def\sourcecodetabsize {3}         % Tamaño tabulación código fuente
\def\tabledefaultplacement {H}     % Posición por defecto de las tablas
\def\tablenotesameline {true}      % Notas en tablas en una sola línea
\def\tablenotesfontsize {\small}   % Tamaño de fuente de las notas en tablas
\def\tablepaddingh {0.75}          % Espaciado horizontal de celda de las tablas
\def\tablepaddingv {1.15}          % Espaciado vertical de celda de las tablas
\def\tikzdefaultplacement {H}      % Posición por defecto de las figuras tikz

% CONFIGURACIÓN DE LOS TÍTULOS
\def\anumsecaddtocounter {false}   % Insertar títulos anum. aumenta n° de sec
\def\chapterfontsize {\huge}       % Tamaño fuente de los capítulos
\def\chapterfontstyle {\bfseries}  % Estilo fuente de los capítulos
\def\charaftersectionnum {.}       % Carácter después n° (s/ss/sss/ssss)ection
\def\charappendixsection {.}       % Carácter entre n° sección anexo y título
\def\charbetwchaptersection {.}    % Carácter entre nº capítulo y sección
\def\charbetwsectionsubsection {.} % Carácter entre nº sección y subsección
\def\charbetwssectionsssect {.}    % Carácter entre nº subsubsección y sssección
\def\charbetwsubsectionssect {.}   % Carácter entre nº subsección y ssección
\def\formatnumapchapter {\Alph}    % Formato nº capítulo en appendixd
\def\formatnumapsection {\arabic}  % Formato nº sección en appendixd
\def\formatnumapssection {\arabic} % Formato nº subsección en appendixd
\def\formatnumapsssection{\arabic} % Formato nº sub-subsección en appendixd
\def\formatnumapssssection{\arabic}% Formato nº sub-sub-subsección appendixd
\def\formatnumchapter {\arabic}    % Formato número capítulo
\def\formatnumsection {\arabic}    % Formato número sección
\def\formatnumssection {\arabic}   % Formato número subsección
\def\formatnumsssection {\arabic}  % Formato número sub-subsección
\def\formatnumssssection {\arabic} % Formato número sub-sub-subsección
\def\paragfontsize {\normalsize}   % Tamaño fuente paragraph
\def\paragfontstyle {\bfseries}    % Estilo fuente paragraph
\def\paragspacingbottom {4}        % Espaciado inferior en paragraph [pt]
\def\paragspacingleft {0}          % Espaciado izq. en paragraph [pt]
\def\paragspacingtop {8}           % Espaciado superior en paragraph [pt]
\def\paragsubfontsize{\normalsize} % Tamaño fuente subparagraph
\def\paragsubfontstyle {\bfseries} % Estilo fuente subparagraph
\def\paragsubspacingbottom {4}     % Espaciado inferior en subparagraph [pt]
\def\paragsubspacingleft {0}       % Espaciado izq. en subparagraph [pt]
\def\paragsubspacingtop {8}        % Espaciado superior en subparagraph [pt]
\def\sectionfontsize {\Large}      % Tamaño fuente section
\def\sectionfontstyle {\bfseries}  % Estilo fuente section
\def\sectionspacingbottom {10}     % Espaciado inferior en section [pt]
\def\sectionspacingleft {0}        % Espaciado izq. en section [pt]
\def\sectionspacingtop {20}        % Espaciado superior en section [pt]
\def\spacingaftersection {\quad}   % Espaciador después nº sección
\def\ssectionfontsize {\large}     % Tamaño fuente subtítulos
\def\ssectionfontstyle {\bfseries} % Estilo fuente subsection
\def\ssectionspacingbottom {10}    % Espaciado inferior en subsection [pt]
\def\ssectionspacingleft {0}       % Espaciado izq. en subsection [pt]
\def\ssectionspacingtop {12}       % Espaciado superior en subsection [pt]
\def\sssectionfontsize{\normalsize}% Tamaño fuente subsubsection
\def\sssectionfontstyle{\bfseries} % Estilo fuente subsubsection
\def\sssectionspacingbottom {8}    % Espaciado inferior en subsubsection [pt]
\def\sssectionspacingleft {0}      % Espaciado izq. en subsubsection [pt]
\def\sssectionspacingtop {10}      % Espaciado superior en subsubsection [pt]
\def\ssssectionfontstyle{\bfseries}% Estilo fuente subsubsubsection
\def\ssssectionfontsz{\normalsize} % Tamaño fuente subsubsubsection
\def\ssssectionspacingbottom {6}   % Espaciado inferior en subsubsubsection [pt]
\def\ssssectionspacingleft {0}     % Espaciado izq. en subsubsubsection [pt]
\def\ssssectionspacingtop {8}      % Espaciado superior en subsubsubsection [pt]

% CONFIGURACIÓN DE LOS COLORES DEL DOCUMENTO
\def\chaptercolor {black}          % Color de los capítulos
\def\captioncolor {black}          % Color nombre objeto (código,figura,tabla)
\def\captiontextcolor {black}      % Color de la leyenda
\def\enumerateitemcolor {black}    % Color de los enumerate por defecto
\def\highlightcolor {yellow}       % Color del subrayado con \hl
\def\indextitlecolor {black}       % Color de los títulos del índice
\def\itemizeitemcolor {black}      % Color de los ítems por defecto
\def\linenumbercolor {gray}        % Color del n° de línea (\showlinenumbers)
\def\linkcolor {black}             % Color de los links del documento
\def\maintextcolor {black}         % Color principal del texto
\def\numcitecolor {black}          % Color del n° de las referencias o citas
\def\pagescolor {white}            % Color de la página
\def\paragcolor {black}            % Color de los paragraph
\def\paragsubcolor {black}         % Color de los subparagraph
\def\sectioncolor {black}          % Color de los section
\def\showborderonlinks {false}     % Color de un link por un recuadro de color
\def\sourcecodebgcolor {lgray}     % Color de fondo del código fuente
\def\ssectioncolor {black}         % Color de los subsection
\def\sssectioncolor {black}        % Color de los subsubsection
\def\ssssectioncolor {black}       % Color de los subsubsubsection
\def\tablelinecolor {black}        % Color de las líneas de las tablas
\def\tablerowfirstcolor {none}     % Primer color de celda de las tablas
\def\tablerowsecondcolor {gray!20} % Segundo color de celda de las tablas
\def\urlcolor {magenta}            % Color de los enlaces web (\href,\url)

% MÁRGENES DE PÁGINA
\def\pagemarginbottom {2}          % Margen inferior página [cm]
\def\pagemarginleft {3}            % Margen izquierdo página [cm]
\def\pagemarginleftportrait {2.5}  % Margen izquierdo página portada [cm]
\def\pagemarginright {2}           % Margen derecho página [cm]
\def\pagemargintop {2}             % Margen superior página [cm]

% OPCIONES DEL PDF COMPILADO
\def\cfgbookmarksopenlevel {0}     % Nivel marcadores en pdf (1:secciones)
\def\cfgpdfbookmarkopen {true}     % Expande marcadores del nivel configurado
\def\cfgpdfcenterwindow {true}     % Centra ventana del lector al abrir el pdf
\def\cfgpdfcopyright {}            % Establece el copyright del documento
\def\cfgpdfdisplaydoctitle {true}  % Muestra título del informe en visor
\def\cfgpdffitwindow {false}       % Ajusta la ventana del lector tamaño pdf
\def\cfgpdfkeywords {}             % Palabras clave del pdf
\def\cfgpdflayout {OneColumn}      % Modo de página {OneColumn,SinglePage}
\def\cfgpdfmenubar {true}          % Muestra el menú del lector
\def\cfgpdfpageview {FitH}         % {Fit,FitH,FitV,FitR,FitB,FitBH,FitBV}
\def\cfgpdfsecnumbookmarks {true}  % Número de la sec. en marcadores del pdf
\def\cfgpdftoolbar {true}          % Muestra barra de herramientas lector pdf
\def\cfgshowbookmarkmenu {true}    % Muestra menú marcadores al abrir el pdf
\def\pdfcompilecompression {9}     % Factor de compresión del pdf (0-9)
\def\pdfcompileobjcompression {2}  % Nivel compresión objetos del pdf (0-3)
\def\pdfcompileversion {7}         % Versión mínima del pdf compilado
\def\usepdfmetadata {true}         % Añade metadatos al pdf compilado

% NOMBRE DE OBJETOS
\def\nameabstract {Resumen}           % Nombre del resumen-abstract
\def\nameagradec {Agradecimientos}    % Nombre del cap. de agradecimientos
\def\nameappendixsection {Anexos}     % Nombre de los anexos
\def\namechapter {Capítulo}           % Nombre de los capítulos
\def\nameltappendixsection {Anexo}    % Etiqueta sección en anexo/apéndices
\def\nameltcont{Tabla de Contenido}   % Nombre del índice de contenidos
\def\namelteqn {Índice de Ecuaciones} % Nombre de la lista de ecuaciones
\def\nameltfigure{Índice de Ilustraciones} % Nombre de la lista de figuras
\def\nameltsrc {Índice de Códigos}    % Nombre de la lista de código
\def\namelttable {Índice de Tablas}   % Nombre de la lista de tablas
\def\nameltwfigure {Figura}           % Etiqueta leyenda de las figuras
\def\nameltwsrc {Código}              % Etiqueta leyenda del código fuente
\def\nameltwtable {Tabla}             % Etiqueta leyenda de las tablas
\def\namemathcol {Corolario}          % Nombre de los colorarios
\def\namemathdefn {Definición}        % Nombre de las definiciones
\def\namemathej {Ejemplo}             % Nombre de los ejemplos
\def\namemathlem {Lema}               % Nombre de los lemas
\def\namemathobs {Observación}        % Nombre de las observaciones
\def\namemathprp {Proposición}        % Nombre de las proposiciones
\def\namemaththeorem {Teorema}        % Nombre de los teoremas
\def\namepageof { de }                % Etiqueta página # de #
\def\nameportraitpage {Cover}         % Etiqueta página de la portada
\def\namereferences {Bibliografía}    % Nombre de la sección de referencias


% -----------------------------------------------------------------------------
% IMPORTACIÓN DE LIBRERÍAS
% -----------------------------------------------------------------------------
% Se guardan variables antes de cargar librerías
\let\RE\Re
\let\IM\Im

% Parches de librerías
\let\counterwithout\relax
\let\counterwithin\relax
\let\underbar\relax
\let\underline\relax

% Si se desactiva el idioma
\def\unaccentedoperators {}
\def\decimalpoint {}
\def\bibname {}

% Parche de sectsty.sty
\makeatletter
\def\underline#1{\relax\ifmmode\@@underline{#1}\else $\@@underline{\hbox{#1}}\m@th$\relax\fi}
\def\underbar#1{\underline{\sbox\tw@{#1}\dp\tw@\z@\box\tw@}}
\makeatother

% -----------------------------------------------------------------------------
% Librerías del núcleo
% -----------------------------------------------------------------------------
% Manejo de condicionales
\usepackage{iftex}
\usepackage{ifthen}

% Verifica el tipo de compilador
\ifPDFTeX
	\def\compilertype {pdf2latex}
\else\ifXeTeX
	\def\compilertype {xelatex}
\else\ifLuaTeX
	\def\compilertype {lualatex}
\else
	\errmessage{Compilador no soportado}
	\stop
	\fi\fi
\fi

% Carga el idioma
\usepackage{tracklang}
\ifthenelse{\equal{\stylecitereferences}{natbib}}{% Formato citas natbib
	\usepackage[nottoc,notlof,notlot]{tocbibind}
	\ifthenelse{\equal{\natbibrefcitecompress}{true}}{
		\usepackage[sort&compress]{natbib}
	}{
		\usepackage{natbib}
	}
}{}% Fin carga natbib
\IfTrackedLanguage{spanish}{
	\usepackage[es-nosectiondot,es-lcroman,es-noquoting]{babel}
}{% english, otros
	\usepackage{babel}
}

% Cambia el estilo de los títulos
\usepackage{sectsty}

% Codificación
\ifthenelse{\equal{\compilertype}{pdf2latex}}{
	\usepackage[utf8]{inputenc}}{
}

% Lanza un mensaje de error indicando mala configuración
%	#1	Parámetros opcionales (nostop,noheader)
%	#2	Mensaje de error
% 	#3	Configuración usada
%	#4	Valores esperados
\newcommand{\throwbadconfig}[4][]{
	\ifthenelse{\equal{#1}{noheader}}{
		\errmessage{LaTeX Warning: #4}
	}{
		\ifthenelse{\equal{#1}{noheader-nostop}}{
			\errmessage{LaTeX Warning: #4}
		}{
			\errmessage{LaTeX Warning: #2 (\noexpand #3= #3). Valores esperados: #4}
		}
	}
	\ifthenelse{\equal{#1}{nostop}}{}{
		\ifthenelse{\equal{#1}{noheader-nostop}}{}{
			\stop
		}
	}
}

% Librerías matemáticas
\ifthenelse{\equal{\equationleftalign}{true}}{
	\usepackage[fleqn]{amsmath}
}{
	\usepackage{amsmath}
}

% Tamaño de la fuente del documento
\usepackage{scrextend}
\usepackage{anyfontsize}
\changefontsizes{\documentfontsize pt}

% Evita error "Too many alphabets used in version normal"
\newcommand\hmmax {0}
\newcommand\bmmax {0}

% -----------------------------------------------------------------------------
% Librerías independientes
% -----------------------------------------------------------------------------
\usepackage{amsbsy}        % Símbolos matemáticos en negrita
\usepackage{amssymb}       % Librerías matemáticas
\usepackage{amsthm}        % Definición de teoremas
\usepackage{animate}       % Imágenes animadas
\usepackage{array}         % Nuevas características a las tablas
\usepackage{bigstrut}      % Líneas horizontales en tablas
\usepackage{bm}            % Caracteres en negrita en ecuaciones
\usepackage{booktabs}      % Permite manejar elementos visuales en tablas
\usepackage{caption}       % Leyendas
\usepackage{chngcntr}      % Añade números a las leyendas
\usepackage{color}         % Colores
\usepackage{datetime}      % Fechas
\usepackage{floatpag}      % Maneja estilos de páginas introducidos por objetos flotantes
\usepackage{floatrow}      % Permite administrar posiciones en los caption
\usepackage{framed}        % Permite creación de recuadros
\usepackage{gensymb}       % Simbología común
\usepackage{graphicx}      % Propiedades extra para los gráficos
\usepackage{lipsum}        % Permite crear párrafos de prueba
\usepackage{listings}      % Permite añadir código fuente
\usepackage{longtable}     % Permite utilizar tablas en varias hojas
\usepackage{mathrsfs}      % Define más fuentes matemáticas
\usepackage{mathtools}     % Permite utilizar notaciones matemáticas
\usepackage{multicol}      % Múltiples columnas
\usepackage{needspace}     % Maneja los espacios en página
\usepackage{pdflscape}     % Modo página horizontal de página
\usepackage{pdfpages}      % Permite administrar páginas en pdf
\usepackage{physics}       % Paquete de matemáticas
\usepackage{realboxes}     % Permite inserción de recuadros
\usepackage{rotating}      % Permite rotación de objetos
\usepackage{selinput}      % Compatibilidad con acentos
\usepackage{setspace}      % Cambia el espacio entre líneas
\usepackage{soul}          % Permite subrayar texto
\usepackage{stfloats}      % Permite cambiar posición de flotantes con [b] y [t]
\usepackage{subcaption}    % Permite agrupar imágenes
\usepackage{textcomp}      % Simbología común
\usepackage{wrapfig}       % Posición de imágenes
\usepackage{xspace}        % Administra espacios en párrafos y líneas
\usepackage{xurl}          % Permite añadir enlaces

% -----------------------------------------------------------------------------
% Librerías con parámetros
% -----------------------------------------------------------------------------
\usepackage[export]{adjustbox} % Agrega nuevas etiquetas de posicionado
\usepackage[makeroom]{cancel} % Cancelar términos en fórmulas
\usepackage[inline]{enumitem} % Permite enumerar ítems
\usepackage[titles]{tocloft} % Maneja entradas en el índice
\usepackage[figure,table,lstlisting]{totalcount} % Contador de objetos
\usepackage[normalem]{ulem} % Permite tachar y subrayar
\usepackage[nointegrals]{wasysym} % Contiene caracteres misceláneos

% -----------------------------------------------------------------------------
% Librerías condicionales
% -----------------------------------------------------------------------------
% Imágenes en modo draft
\ifthenelse{\equal{\graphicxdraft}{true}}{
	\usepackage[
		allfiguresdraft,
		filename,
		size={scriptsize},
		style={tt}
	]{draftfigure}}{
}

% Acepta codificación UTF-8 en código fuente
\ifthenelse{\equal{\compilertype}{pdf2latex}}{
	\usepackage{listingsutf8}}{
}

% Footnotes en dos columnas
\ifthenelse{\equal{\footnotetwocolumn}{true}}{
	\usepackage{dblfnote}}{
}

% Regla superior
\ifthenelse{\equal{\footnoterulepage}{true}}{
	\usepackage[bottom,hang]{footmisc} % Estilo pie de página
}{
	\usepackage[bottom,norule,hang]{footmisc}
}

% Referencias
\ifthenelse{\equal{\backrefpagecite}{true}}{
	\usepackage[pdfencoding=auto,psdextra,backref=page]{hyperref} % Enlaces, referencias
}{
	\usepackage[pdfencoding=auto,psdextra]{hyperref} % Enlaces, referencias
}

% Anexos/Apéndices
\ifthenelse{\equal{\showappendixsecindex}{true}}{
	\usepackage[toc]{appendix} % Eliminado en Auxiliares/Controles, sin [toc]
}{
	\usepackage{appendix}
}

% Citado
\ifthenelse{\equal{\stylecitereferences}{apacite}}{% Formato citas apacite
	\usepackage[nottoc,notlof,notlot]{tocbibind}
	\usepackage[nosectionbib]{apacite}
}{
\ifthenelse{\equal{\stylecitereferences}{bibtex}}{% Formato citas bibtex
}{
\ifthenelse{\equal{\stylecitereferences}{custom}}{% Formato citas custom
}{}}}

% Dimensiones y geometría del documento
\ifthenelse{\equal{\compilertype}{lualatex}}{% En lualatex sólo se puede cambiar 1 vez el margen
	\usepackage[top=\pagemargintop cm,bottom=\pagemarginbottom cm,left=\pagemarginleft cm,right=\pagemarginright cm, footnotesep=\footnotetopmargin cm]{geometry}
}{% pdf2latex, xelatex
	\usepackage{geometry}
}

% Notas en tablas
\ifthenelse{\equal{\tablenotesameline}{true}}{
	\usepackage[para]{threeparttable}
}{
	\usepackage{threeparttable}
}

% -----------------------------------------------------------------------------
% Librerías dependientes
% -----------------------------------------------------------------------------
\usepackage{bookmark}      % Administración de marcadores en pdf
\usepackage{fancyhdr}      % Encabezados y pie de páginas
\usepackage{float}         % Administrador de posiciones de objetos
\usepackage{hyperxmp}      % Etiquetas opcionales para el pdf compilado
\usepackage{multirow}      % Agrega nuevas opciones a las tablas
\usepackage{notoccite}     % Desactiva las citas en el índice
\usepackage{titlesec}      % Administración de títulos

% -----------------------------------------------------------------------------
% Tipografía del documento
% -----------------------------------------------------------------------------
% Tipografías clásicas
\ifthenelse{\equal{\fontdocument}{lmodern}}{
	\usepackage{lmodern}
}{
\ifthenelse{\equal{\fontdocument}{arial}}{
	\usepackage{helvet}
	\renewcommand{\familydefault}{\sfdefault}
}{
\ifthenelse{\equal{\fontdocument}{arial2}}{
	\usepackage{arial}
}{
\ifthenelse{\equal{\fontdocument}{times}}{
	\usepackage{mathptmx}
}{
\ifthenelse{\equal{\fontdocument}{mathptmx}}{
	\usepackage{mathptmx}
}{
\ifthenelse{\equal{\fontdocument}{helvet}}{
	\renewcommand{\familydefault}{\sfdefault}
	\usepackage[scaled=0.95]{helvet}
	\usepackage[helvet]{sfmath}
}{
\ifthenelse{\equal{\fontdocument}{opensans}}{
	\usepackage[default,scale=0.95]{opensans}
}{
\ifthenelse{\equal{\fontdocument}{mathpazo}}{
	\usepackage{mathpazo}
}{
\ifthenelse{\equal{\fontdocument}{cambria}}{
	\usepackage{caladea}
}{
\ifthenelse{\equal{\fontdocument}{libertine}}{
	\usepackage[libertine]{newtxmath}
	\usepackage[tt=false]{libertine}
}{
\ifthenelse{\equal{\fontdocument}{custom}}{
}{

% Otros (último: fbb el 08/08/2021 - https://tug.org/FontCatalogue/seriffonts.html)
\ifthenelse{\equal{\fontdocument}{accanthis}}{
	\usepackage{accanthis}
}{
\ifthenelse{\equal{\fontdocument}{alegreya}}{
	\usepackage{Alegreya}
	\renewcommand*\oldstylenums[1]{{\AlegreyaOsF #1}}
}{
\ifthenelse{\equal{\fontdocument}{alegreyasans}}{
	\usepackage[sfdefault]{AlegreyaSans}
	\renewcommand*\oldstylenums[1]{{\AlegreyaSansOsF #1}}
}{
\ifthenelse{\equal{\fontdocument}{algolrevived}}{
	\usepackage{algolrevived}
}{
\ifthenelse{\equal{\fontdocument}{almendra}}{
	\usepackage{almendra}
}{
\ifthenelse{\equal{\fontdocument}{antpolt}}{
	\usepackage{antpolt}
}{
\ifthenelse{\equal{\fontdocument}{antpoltlight}}{
	\usepackage[light]{antpolt}
}{
\ifthenelse{\equal{\fontdocument}{anttor}}{
	\usepackage[math]{anttor}
}{
\ifthenelse{\equal{\fontdocument}{anttorcondensed}}{
	\usepackage[condensed,math]{anttor}
}{
\ifthenelse{\equal{\fontdocument}{anttorlight}}{
	\usepackage[light,math]{anttor}
}{
\ifthenelse{\equal{\fontdocument}{anttorlightcondensed}}{
	\usepackage[light,condensed,math]{anttor}
}{
\ifthenelse{\equal{\fontdocument}{arev}}{
	\let\quarternote\relax
	\let\eighthnote\relax
	\usepackage{arev}
}{
\ifthenelse{\equal{\fontdocument}{arimo}}{
	\usepackage[sfdefault]{arimo}
	\renewcommand*\familydefault{\sfdefault}
}{
\ifthenelse{\equal{\fontdocument}{arvo}}{
	\usepackage{Arvo}
}{
\ifthenelse{\equal{\fontdocument}{baskervald}}{
	\usepackage{baskervald}
}{
\ifthenelse{\equal{\fontdocument}{baskervaldx}}{
	\usepackage[lf]{Baskervaldx}
	\usepackage[bigdelims,vvarbb]{newtxmath}
	\usepackage[cal=boondoxo]{mathalfa}
	\renewcommand*\oldstylenums[1]{\textosf{#1}}
}{
\ifthenelse{\equal{\fontdocument}{berasans}}{
	\usepackage[scaled]{berasans}
	\renewcommand*\familydefault{\sfdefault}
}{
\ifthenelse{\equal{\fontdocument}{beraserif}}{
	\usepackage{bera}
}{
\ifthenelse{\equal{\fontdocument}{biolinum}}{
	\usepackage{libertine}
	\renewcommand*\familydefault{\sfdefault}
}{
\ifthenelse{\equal{\fontdocument}{bitter}}{
	\usepackage{bitter}
}{
\ifthenelse{\equal{\fontdocument}{boisik}}{
	\let\div\relax
	\usepackage{boisik}
}{
\ifthenelse{\equal{\fontdocument}{bookman}}{
	\usepackage{bookman}
}{
\ifthenelse{\equal{\fontdocument}{cabin}}{
	\usepackage[sfdefault]{cabin}
	\renewcommand*\familydefault{\sfdefault}
}{
\ifthenelse{\equal{\fontdocument}{cabincondensed}}{
	\usepackage[sfdefault,condensed]{cabin}
	\renewcommand*\familydefault{\sfdefault}
}{
\ifthenelse{\equal{\fontdocument}{caladea}}{
	\usepackage{caladea}
}{
\ifthenelse{\equal{\fontdocument}{cantarell}}{
	\usepackage[default]{cantarell}
}{
\ifthenelse{\equal{\fontdocument}{carlito}}{
	\usepackage[sfdefault]{carlito}
	\renewcommand*\familydefault{\sfdefault}
}{
\ifthenelse{\equal{\fontdocument}{charterbt}}{
	\usepackage[bitstream-charter]{mathdesign}
}{
\ifthenelse{\equal{\fontdocument}{chivolight}}{
	\usepackage[familydefault,light]{Chivo}
}{
\ifthenelse{\equal{\fontdocument}{chivoregular}}{
	\usepackage[familydefault,regular]{Chivo}
}{
\ifthenelse{\equal{\fontdocument}{clara}}{
	\usepackage{clara}
}{
\ifthenelse{\equal{\fontdocument}{clearsans}}{
	\usepackage[sfdefault]{ClearSans}
	\renewcommand*\familydefault{\sfdefault}
}{
\ifthenelse{\equal{\fontdocument}{cochineal}}{
	\usepackage{cochineal}
}{
\ifthenelse{\equal{\fontdocument}{coelacanth}}{
	\usepackage[nf]{coelacanth}
	\let\oldnormalfont\normalfont
	\def\normalfont {\oldnormalfont\mdseries}
}{
\ifthenelse{\equal{\fontdocument}{coelacanthextralight}}{
	\usepackage[el,nf]{coelacanth}
	\let\oldnormalfont\normalfont
	\def\normalfont {\oldnormalfont\mdseries}
}{
\ifthenelse{\equal{\fontdocument}{coelacanthlight}}{
	\usepackage[l,nf]{coelacanth}
	\let\oldnormalfont\normalfont
	\def\normalfont {\oldnormalfont\mdseries}
}{
\ifthenelse{\equal{\fontdocument}{comfortaa}}{
	\usepackage[default]{comfortaa}
}{
\ifthenelse{\equal{\fontdocument}{comicneue}}{
	\usepackage[default]{comicneue}
}{
\ifthenelse{\equal{\fontdocument}{comicneueangular}}{
	\usepackage[default,angular]{comicneue}
}{
\ifthenelse{\equal{\fontdocument}{computerconcrete}}{
	\usepackage{concmath}
}{
\ifthenelse{\equal{\fontdocument}{computerconcreteeuler}}{
	\let\Re\relax
	\let\Im\relax
	\usepackage{beton}
	\usepackage{euler}
}{
\ifthenelse{\equal{\fontdocument}{computermodern}}{
}{
\ifthenelse{\equal{\fontdocument}{computermodernbright}}{
	\usepackage{cmbright}
}{
\ifthenelse{\equal{\fontdocument}{crimson}}{
	\usepackage{crimson}
}{
\ifthenelse{\equal{\fontdocument}{crimsonpro}}{
	\usepackage{CrimsonPro}
	\let\oldnormalfont\normalfont
	\def\normalfont {\oldnormalfont\mdseries}
}{
\ifthenelse{\equal{\fontdocument}{crimsonproextralight}}{
	\usepackage[extralight]{CrimsonPro}
	\let\oldnormalfont\normalfont
	\def\normalfont {\oldnormalfont\mdseries}
}{
\ifthenelse{\equal{\fontdocument}{crimsonprolight}}{
	\usepackage[light]{CrimsonPro}
	\let\oldnormalfont\normalfont
	\def\normalfont {\oldnormalfont\mdseries}
}{
\ifthenelse{\equal{\fontdocument}{crimsonpromedium}}{
	\usepackage[medium]{CrimsonPro}
	\let\oldnormalfont\normalfont
	\def\normalfont {\oldnormalfont\mdseries}
}{
\ifthenelse{\equal{\fontdocument}{cyklop}}{
	\usepackage{cyklop}
}{
\ifthenelse{\equal{\fontdocument}{dejavusans}}{
	\usepackage{DejaVuSans}
	\renewcommand*\familydefault{\sfdefault}
}{
\ifthenelse{\equal{\fontdocument}{dejavusanscondensed}}{
	\usepackage{DejaVuSansCondensed}
	\renewcommand*\familydefault{\sfdefault}
}{
\ifthenelse{\equal{\fontdocument}{domitian}}{
	\usepackage{mathpazo}
	\usepackage{domitian}
	\let\oldstylenums\oldstyle
}{
\ifthenelse{\equal{\fontdocument}{droidsans}}{
	\usepackage[defaultsans]{droidsans}
	\renewcommand*\familydefault{\sfdefault}
}{
\ifthenelse{\equal{\fontdocument}{electrum}}{
	\usepackage[lf]{electrum}
}{	
\ifthenelse{\equal{\fontdocument}{erewhon}}{
	\usepackage[proportional,scaled=1.064]{erewhon}
	\usepackage[erewhon,vvarbb,bigdelims]{newtxmath}
	\renewcommand*\oldstylenums[1]{\textosf{#1}}
}{
\ifthenelse{\equal{\fontdocument}{fbb}}{
	\usepackage{fbb}
}{
\ifthenelse{\equal{\fontdocument}{fetamont}}{
	\usepackage{fetamont}
	\renewcommand*\familydefault{\sfdefault}
}{
\ifthenelse{\equal{\fontdocument}{firasans}}{
	\usepackage[sfdefault]{FiraSans}
	\renewcommand*\familydefault{\sfdefault}
}{
\ifthenelse{\equal{\fontdocument}{firasansnewtxsf}}{
	\usepackage[sfdefault]{FiraSans}
	\usepackage{newtxsf}
}{
\ifthenelse{\equal{\fontdocument}{fourier}}{
	\usepackage{fourier}
}{
\ifthenelse{\equal{\fontdocument}{fouriernc}}{
	\usepackage{fouriernc}
}{
\ifthenelse{\equal{\fontdocument}{gfsartemisia}}{
	\let\textlozenge\relax
	\usepackage{gfsartemisia}
}{
\ifthenelse{\equal{\fontdocument}{gfsartemisiaeuler}}{
	\let\textlozenge\relax
	\let\Re\relax
	\let\Im\relax
	\usepackage{gfsartemisia-euler}
}{
\ifthenelse{\equal{\fontdocument}{heuristica}}{
	\usepackage{heuristica}
	\usepackage[heuristica,vvarbb,bigdelims]{newtxmath}
	\renewcommand*\oldstylenums[1]{\textosf{#1}}
}{
\ifthenelse{\equal{\fontdocument}{iwona}}{
	\usepackage[math]{iwona}
}{
\ifthenelse{\equal{\fontdocument}{iwonacondensed}}{
	\usepackage[condensed,math]{iwona}
}{
\ifthenelse{\equal{\fontdocument}{iwonalight}}{
	\usepackage[light,math]{iwona}
}{
\ifthenelse{\equal{\fontdocument}{iwonalightcondensed}}{
	\usepackage[light,condensed,math]{iwona}
}{
\ifthenelse{\equal{\fontdocument}{kerkis}}{
	\usepackage{kmath,kerkis}
}{
\ifthenelse{\equal{\fontdocument}{kurier}}{
	\usepackage[math]{kurier}
}{
\ifthenelse{\equal{\fontdocument}{kuriercondensed}}{
	\usepackage[condensed,math]{kurier}
}{
\ifthenelse{\equal{\fontdocument}{kurierlight}}{
	\usepackage[light,math]{kurier}
}{
\ifthenelse{\equal{\fontdocument}{kurierlightcondensed}}{
	\usepackage[light,condensed,math]{kurier}
}{
\ifthenelse{\equal{\fontdocument}{lato}}{
	\usepackage[default]{lato}
}{
\ifthenelse{\equal{\fontdocument}{libertinus}}{
	\usepackage{libertinus}
}{
\ifthenelse{\equal{\fontdocument}{librebaskerville}}{
	\usepackage{librebaskerville}
}{
\ifthenelse{\equal{\fontdocument}{librebodoni}}{
	\usepackage{LibreBodoni}
}{
\ifthenelse{\equal{\fontdocument}{librecaslon}}{
	\usepackage{librecaslon}
}{
\ifthenelse{\equal{\fontdocument}{libris}}{
	\usepackage{libris}
	\renewcommand*\familydefault{\sfdefault}
}{
\ifthenelse{\equal{\fontdocument}{lxfonts}}{
	\usepackage{lxfonts}
}{
\ifthenelse{\equal{\fontdocument}{merriweather}}{
	\usepackage[sfdefault]{merriweather}
}{
\ifthenelse{\equal{\fontdocument}{merriweatherlight}}{
	\usepackage[sfdefault,light]{merriweather}
}{
\ifthenelse{\equal{\fontdocument}{mintspirit}}{
	\usepackage[default]{mintspirit}
}{
\ifthenelse{\equal{\fontdocument}{mlmodern}}{
	\usepackage{mlmodern}
}{
\ifthenelse{\equal{\fontdocument}{montserratalternatesextralight}}{
	\usepackage[defaultfam,extralight,tabular,lining,alternates]{montserrat}
	\renewcommand*\oldstylenums[1]{{\fontfamily{Montserrat-TOsF}\selectfont #1}}
}{
\ifthenelse{\equal{\fontdocument}{montserratalternatesregular}}{
	\usepackage[defaultfam,tabular,lining,alternates]{montserrat}
	\renewcommand*\oldstylenums[1]{{\fontfamily{Montserrat-TOsF}\selectfont #1}}
}{
\ifthenelse{\equal{\fontdocument}{montserratalternatesthin}}{
	\usepackage[defaultfam,thin,tabular,lining,alternates]{montserrat}
	\renewcommand*\oldstylenums[1]{{\fontfamily{Montserrat-TOsF}\selectfont #1}}
}{
\ifthenelse{\equal{\fontdocument}{montserratextralight}}{
	\usepackage[defaultfam,extralight,tabular,lining]{montserrat}
	\renewcommand*\oldstylenums[1]{{\fontfamily{Montserrat-TOsF}\selectfont #1}}
}{
\ifthenelse{\equal{\fontdocument}{montserratlight}}{
	\usepackage[defaultfam,light,tabular,lining]{montserrat}
	\renewcommand*\oldstylenums[1]{{\fontfamily{Montserrat-TOsF}\selectfont #1}}
}{
\ifthenelse{\equal{\fontdocument}{montserratregular}}{
	\usepackage[defaultfam,tabular,lining]{montserrat}
	\renewcommand*\oldstylenums[1]{{\fontfamily{Montserrat-TOsF}\selectfont #1}}
}{
\ifthenelse{\equal{\fontdocument}{montserratthin}}{
	\usepackage[defaultfam,thin,tabular,lining]{montserrat}
	\renewcommand*\oldstylenums[1]{{\fontfamily{Montserrat-TOsF}\selectfont #1}}
}{
\ifthenelse{\equal{\fontdocument}{newpx}}{
	\usepackage{newpxtext,newpxmath}
}{
\ifthenelse{\equal{\fontdocument}{nimbussans}}{
	\usepackage{nimbussans}
	\renewcommand*\familydefault{\sfdefault}
}{
\ifthenelse{\equal{\fontdocument}{noto}}{
	\usepackage[sfdefault]{noto}
	\renewcommand*\familydefault{\sfdefault}
}{
\ifthenelse{\equal{\fontdocument}{notoserif}}{
	\usepackage{notomath}
}{
\ifthenelse{\equal{\fontdocument}{opensansserif}}{
	\usepackage[default,oldstyle,scale=0.95]{opensans}
}{
\ifthenelse{\equal{\fontdocument}{overlock}}{
	\usepackage[sfdefault]{overlock}
	\renewcommand*\familydefault{\sfdefault}
}{
\ifthenelse{\equal{\fontdocument}{paratype}}{
	\usepackage{paratype}
	\renewcommand*\familydefault{\sfdefault}
}{
\ifthenelse{\equal{\fontdocument}{paratypesanscaption}}{
	\usepackage{PTSansCaption}
	\renewcommand*\familydefault{\sfdefault}
}{
\ifthenelse{\equal{\fontdocument}{paratypesansnarrow}}{
	\usepackage{PTSansNarrow}
	\renewcommand*\familydefault{\sfdefault}
}{
\ifthenelse{\equal{\fontdocument}{pxfonts}}{
	\usepackage{pxfonts}
}{
\ifthenelse{\equal{\fontdocument}{quattrocento}}{
	\usepackage[sfdefault]{quattrocento}
}{
\ifthenelse{\equal{\fontdocument}{raleway}}{
	\usepackage[default]{raleway}
}{
\ifthenelse{\equal{\fontdocument}{ralewayblack}}{
	\usepackage[black]{raleway}
}{
\ifthenelse{\equal{\fontdocument}{ralewayextralight}}{
	\usepackage[extralight]{raleway}
}{
\ifthenelse{\equal{\fontdocument}{ralewaymedium}}{
	\usepackage[medium]{raleway}
}{
\ifthenelse{\equal{\fontdocument}{ralewaylight}}{
	\usepackage[light]{raleway}
}{
\ifthenelse{\equal{\fontdocument}{ralewaythin}}{
	\usepackage[thin]{raleway}
}{
\ifthenelse{\equal{\fontdocument}{roboto}}{
	\usepackage[sfdefault]{roboto}
}{
\ifthenelse{\equal{\fontdocument}{robotocondensed}}{
	\usepackage[sfdefault,condensed]{roboto}
}{
\ifthenelse{\equal{\fontdocument}{robotolight}}{
	\usepackage[sfdefault,light]{roboto}
}{
\ifthenelse{\equal{\fontdocument}{robotolightcondensed}}{
	\usepackage[sfdefault,light,condensed]{roboto}
}{
\ifthenelse{\equal{\fontdocument}{robotothin}}{
	\usepackage[sfdefault,thin]{roboto}
}{
\ifthenelse{\equal{\fontdocument}{rosario}}{
	\usepackage[familydefault]{Rosario}
}{
\ifthenelse{\equal{\fontdocument}{sourcesanspro}}{
	\usepackage[default]{sourcesanspro}
}{
\ifthenelse{\equal{\fontdocument}{step}}{
	\usepackage[notext]{stix}
	\usepackage{step}
}{
\ifthenelse{\equal{\fontdocument}{stickstoo}}{
	\usepackage{stickstootext}
	\usepackage[stickstoo,vvarbb]{newtxmath}
}{
\ifthenelse{\equal{\fontdocument}{texgyrebonum}}{
	\usepackage{tgbonum}
}{
\ifthenelse{\equal{\fontdocument}{txfonts}}{
	\usepackage{txfonts}
}{
\ifthenelse{\equal{\fontdocument}{uarial}}{
	\usepackage{uarial}
	\renewcommand*\familydefault{\sfdefault}
}{
\ifthenelse{\equal{\fontdocument}{ugq}}{
	\renewcommand*\sfdefault{ugq}
	\renewcommand*\familydefault{\sfdefault}
}{
\ifthenelse{\equal{\fontdocument}{universalis}}{
	\usepackage[sfdefault]{universalis}
}{
\ifthenelse{\equal{\fontdocument}{universaliscondensed}}{
	\usepackage[condensed,sfdefault]{universalis}
}{
\ifthenelse{\equal{\fontdocument}{venturis}}{
	\usepackage[lf]{venturis}
	\renewcommand*\familydefault{\sfdefault}
}{
	\throwbadconfig[nostop]{Fuente desconocida}{\fontdocument}{(Fuentes recomendadas) lmodern,carial,arial2,times,mathptmx,helvet,opensans,mathpazo,cambria,libertine,custom}
	\throwbadconfig[noheader-nostop]{Fuente desconocida}{\fontdocument}{(Fuentes adicionales) accanthis,alegreya,alegreyasans,algolrevived,almendra,antpolt,antpoltlight,anttor,anttorcondensed,anttorlight,anttorlightcondensed,arev,arimo,arvo,baskervald,baskervaldx,berasans,beraserif,biolinum,bitter,boisik,bookman,cabin,cabincondensed,cantarell,caladea,carlito,charterbt,chivolight,chivoregular,clara,clearsans,cochineal,coelacanth,coelacanthextralight,coelacanthlight,comfortaa,comicneue,comicneueangular,computerconcrete,computerconcreteeuler,computermodern,computermodernbright,crimson,crimsonpro,crimsonproextralight,crimsonprolight,crimsonpromedium,cyklop}
	\throwbadconfig[noheader-nostop]{Fuente desconocida}{\fontdocument}{dejavusans,dejavusanscondensed,domitian,droidsans,electrum,erewhon,fbb,fetamont,firasans,firasansnewtxsf,fourier,fouriernc,gfsartemisia,gfsartemisiaeuler,heuristica,iwona,iwonacondensed,iwonalight,iwonalightcondensed,kerkis,kurier,kuriercondensed,kurierlight,kurierlightcondensed,lato,libertinus,librebaskerville,librebodoni,librecaslon,libris,lxfonts}
	\throwbadconfig[noheader]{Fuente desconocida}{\fontdocument}{merriweather,merriweatherlight,mintspirit,mlmodern,montserratalternatesextralight,montserratalternatesregular,montserratalternatesthin,montserratextralight,montserratlight,montserratregular,montserratthin,newpx,nimbussans,noto,notoserif,opensansserif,overlock,paratype,paratypesanscaption,paratypesansnarrow,pxfonts,quattrocento,raleway,ralewayblack,ralewayextralight,ralewaymedium,ralewaylight,ralewaythin,roboto,robotolight,robotolightcondensed,robotothin,rosario,sourcesanspro,step,stickstoo,uarial,texgyrebonum,txfonts,ugq,universalis,universaliscondensed,venturis}
	}}}}}}}}}}}}}}}}}}}}}}}}}}}}}}}}}}}}}}}}}}}}}}}}}}}}}}}}}}}}}}}}}}}}}}}}}}}}}}}}}}}}}}}}}}}}}}}}}}}}}}}}}}}}}}}}}}}}}}}}}}}}}}}}}}}}}}
}

% -----------------------------------------------------------------------------
% Tipografía typewriter
% -----------------------------------------------------------------------------
% https://tug.org/FontCatalogue/typewriterfonts.html
\ifthenelse{\equal{\fonttypewriter}{custom}}{
}{
\ifthenelse{\equal{\fonttypewriter}{tmodern}}{
	\renewcommand*\ttdefault{lmvtt}
}{
\ifthenelse{\equal{\fonttypewriter}{anonymouspro}}{
	\usepackage[ttdefault=true]{AnonymousPro}
}{
\ifthenelse{\equal{\fonttypewriter}{ascii}}{
	\usepackage{ascii}
	\let\SI\relax
}{
\ifthenelse{\equal{\fonttypewriter}{beramono}}{
	\usepackage[scaled]{beramono}
}{
\ifthenelse{\equal{\fonttypewriter}{cascadiacode}}{
	\usepackage{cascadia-code}
}{
\ifthenelse{\equal{\fonttypewriter}{cmpica}}{
	\usepackage{addfont}
	\addfont{OT1}{cmpica}{\pica}
	\addfont{OT1}{cmpicab}{\picab}
	\addfont{OT1}{cmpicati}{\picati}
	\renewcommand*\ttdefault{pica}
}{
\ifthenelse{\equal{\fonttypewriter}{cmodern}}{
}{
\ifthenelse{\equal{\fonttypewriter}{courier}}{
	\usepackage{courier}
}{
\ifthenelse{\equal{\fonttypewriter}{courier10}}{
	\usepackage{courierten}
}{
\ifthenelse{\equal{\fonttypewriter}{cmvtt}}{
	\renewcommand*\ttdefault{cmvtt}
}{
\ifthenelse{\equal{\fonttypewriter}{dejavusansmono}}{
	\usepackage[scaled]{DejaVuSansMono}
}{
\ifthenelse{\equal{\fonttypewriter}{droidsansmono}}{
	\usepackage[defaultmono]{droidsansmono}
}{
\ifthenelse{\equal{\fonttypewriter}{firamono}}{
	\usepackage[scale=0.85]{FiraMono}
}{
\ifthenelse{\equal{\fonttypewriter}{gomono}}{
	\usepackage[scale=0.85]{GoMono}
}{
\ifthenelse{\equal{\fonttypewriter}{inconsolata}}{
	\usepackage{inconsolata}
}{
\ifthenelse{\equal{\fonttypewriter}{nimbusmono}}{
	\usepackage{nimbusmono}
}{
\ifthenelse{\equal{\fonttypewriter}{newtxtt}}{
	\usepackage[zerostyle=d]{newtxtt}
}{
\ifthenelse{\equal{\fonttypewriter}{nimbusmono}}{
	\usepackage{nimbusmono}
}{
\ifthenelse{\equal{\fonttypewriter}{nimbusmononarrow}}{
	\usepackage{nimbusmononarrow}
}{
\ifthenelse{\equal{\fonttypewriter}{lcmtt}}{
	\renewcommand*\ttdefault{lcmtt}
}{
\ifthenelse{\equal{\fonttypewriter}{sourcecodepro}}{
	\usepackage[ttdefault=true,scale=0.85]{sourcecodepro}
}{
\ifthenelse{\equal{\fonttypewriter}{texgyrecursor}}{
	\usepackage{tgcursor}
}{
\ifthenelse{\equal{\fonttypewriter}{txtt}}{
	\renewcommand*\ttdefault{txtt}
}{
	\throwbadconfig{Fuente desconocida}{\fonttypewriter}{custom,anonymouspro,ascii,beramono,cascadiacode,cmpica,cmodern,courier,courier10,cvmtt,dejavusansmono,droidsansmono,firamono,gomono,inconsolata,kpmonospaced,lcmtt,newtxtt,nimbusmono,nimbusmononarrow,texgyrecursor,tmodern,txtt}
	}}}}}}}}}}}}}}}}}}}}}}}
}

% -----------------------------------------------------------------------------
% Finales
% -----------------------------------------------------------------------------
\usepackage[T1]{fontenc} % Caracteres acentuados
\ifthenelse{\equal{\showlayoutlines}{true}}{% Muestra las líneas del layout
	\usepackage{showframe}}{
}
\ifthenelse{\equal{\showlinenumbers}{true}}{% Muestra los números de línea
	\usepackage[switch,columnwise,running]{lineno}
	\newcommand*\linenomathpatch[1]{% Parcha entornos
		\cspreto{#1}{\linenomath}%
		\cspreto{#1*}{\linenomath}%
		\csappto{end#1}{\endlinenomath}%
		\csappto{end#1*}{\endlinenomath}}
	\linenomathpatch{equation}
	\linenomathpatch{gather}
	\linenomathpatch{multline}
	\linenomathpatch{align}
	\linenomathpatch{alignat}
	\linenomathpatch{flalign}}{
}
\usepackage{commonunicode} % Símbolos unicode
\usepackage{csquotes} % Citas y comillas
\ifthenelse{\equal{\compilertype}{pdf2latex}}{
	\inputencoding{utf8}}{
}

% -----------------------------------------------------------------------------
% IMPORTACIÓN DE FUNCIONES Y ENTORNOS
% -----------------------------------------------------------------------------
% Definición de variables globales
\global\def\GLOBALemptyvar {template:empty:var}   % Usado para indicar que una variable está vacía

\global\def\GLOBALcaptiondefn {\GLOBALemptyvar}   % Definición del caption
\global\def\GLOBALchapternumenabled {false}       % Numeración de capítulos empezó
\global\def\GLOBALenvappendix {false}             % Indica que el entorno anexo está activo
\global\def\GLOBALenvimageadded {false}           % Indica que una imagen ha sido añadida
\global\def\GLOBALenvimagecf {false}              % Indica que una imagen usa ContinuedFloat
\global\def\GLOBALenvimageinitialized {false}     % Entorno images activo
\global\def\GLOBALenvmulticol {false}             % Indica que el entorno multicol está activo
\global\def\GLOBALsectionanumenabled {false}      % Sección sin numeración
\global\def\GLOBALsubsectionanumenabled {false}   % Subsección sin numeración
\global\def\GLOBALsubsubsectionanumenabled{false} % Sub-subsección sin numeración
\global\def\GLOBALtablerowcolorindex {2}          % Índice tabla colores
\global\def\GLOBALtablerowcolorswitch {false}     % Tabla con colores cambiados
\global\def\GLOBALtwoside {false}                 % Indica que el documento es twoside

% Definición de formato de secciones
\global\def\GLOBALformatnumchapter {\formatnumchapter}
\global\def\GLOBALformatnumsection {\formatnumsection}
\global\def\GLOBALformatnumssection {\formatnumssection}
\global\def\GLOBALformatnumsssection {\formatnumsssection}
\global\def\GLOBALformatnumssssection {\formatnumssssection}

% Configura si el documento es twoside
\makeatletter
\if@twoside
\global\def\GLOBALtwoside {true}
\else
\fi 
\makeatother

% Signo porcentaje para archivos
\def\LOCALpercentchar#1{}
\edef\LOCALpercentchar{\expandafter\LOCALpercentchar\string\%}

% Contador global de objetos
\newcounter{templateEquations}      % Ecuaciones
\newcounter{templateFigures}        % Figuras
\newcounter{templateIndexEquations} % Ecuaciones en el índice
\newcounter{templateListings}       % Códigos fuente
\newcounter{templatePageCounter}    % Administra números de páginas
\newcounter{templateTables}         % Tablas

% Contador nivel de bookmarks marcadores
\newcounter{templateBookmarksLevelPrev}
\setcounter{templateBookmarksLevelPrev}{\cfgbookmarksopenlevel}
\addtocounter{templateBookmarksLevelPrev}{-1}

% Aumenta contador de páginas
\stepcounter{templatePageCounter}
\AtBeginShipout{\stepcounter{templatePageCounter}}

% Define latex para uso en referencias
\let\latex\LaTeX

% Nuevas dimensiones
\newlength{\coregluevarcm}
\setlength{\coregluevarcm}{0.25 cm}
\newlength{\corefontwidth}
\settowidth{\corefontwidth}{template}

% Lanza un mensaje de error
% 	#1	Función del error
%	#2	Mensaje
\newcommand{\throwerror}[2]{%
	\errmessage{LaTeX Error: \noexpand#1 #2 (linea \the\inputlineno)}%
	\stop
}

% Lanza un mensaje de advertencia
%	#1	Mensaje
\newcommand{\throwwarning}[1]{%
	\errmessage{LaTeX Warning: #1 (linea \the\inputlineno)}%
}

% Lanza un mensaje de error indicando mala configuración dentro de begin{document}
%	#1	Mensaje de error
% 	#2	Configuración usada
%	#3	Valores esperados
\newcommand{\throwbadconfigondoc}[3]{%
	\errmessage{#1 \noexpand #2=#2. Valores esperados: #3}%
	\stop%
}

% Chequea que un módulo no haya sido cargado antes de terminar el template
%	#1	Nombre del módulo
\makeatletter%
\newcommand{\checkmodulenotloaded}[1]{%
	\@ifpackageloaded{#1}{%
		\throwwarning{Template Error: No se pueden cargar paquetes (#1) antes de importar template.tex}%
		\stop%
	}{}%
}
\makeatother%

% Comprueba si una variable está definida
%	#1	Variable
\newcommand{\checkvardefined}[1]{%
	\ifthenelse{\isundefined{#1}}{%
		\errmessage{LaTeX Warning: Variable \noexpand#1 no definida}%
		\stop}{%
	}%
}

% Escribe un mensaje en la consola
%	#1	Mensaje
\newcommand{\coretemplatemessage}[1]{%
	\message{Template: #1}%
}

% Comprueba si una variable está definida
%	#1	Variable
%	#2	Mensaje
\newcommand{\checkextravarexist}[2]{%
	\ifthenelse{\isundefined{#1}}{%
		\errmessage{LaTeX Warning: Variable \noexpand#1 no definida}%
		\ifx\hfuzz#2\hfuzz%
			\errmessage{LaTeX Warning: Defina la variable en el bloque de INFORMACION DEL DOCUMENTO al comienzo del archivo principal del template}%
		\else%
			\errmessage{LaTeX Warning: #2}%
		\fi}{%
	}%
}

% Lanza un mensaje de error si una variable no ha sido definida
% 	#1	Función del error
%	#2	Variable
%	#3	Mensaje
\newcommand{\emptyvarerr}[3]{%
	\ifx\hfuzz#2\hfuzz%
		\errmessage{LaTeX Warning: \noexpand#1 #3 (linea \the\inputlineno)}%
	\fi
}

% Cambiar el margen de los caption
% 	#1	Margen en centímetros
\newcommand{\setcaptionmargincm}[1]{
	\captionsetup{margin=#1cm}
}

% Cambia márgenes de las páginas [cm]
% 	#1	Margen izquierdo
%	#2	Margen superior
%	#3	Margen derecho
%	#4	Margen inferior
\newcommand{\setpagemargincm}[4]{%
	\ifthenelse{\equal{\compilertype}{lualatex}}{%
		% Geometry no válido en lualatex
	}{%
		\newgeometry{left=#1cm, top=#2cm, right=#3cm, bottom=#4cm, footnotesep=\footnotetopmargin cm}
	}
}

% Define el caption del índice
% 	#1	Título del caption
\newcommand{\setindexcaption}[1]{%
	\global\def\GLOBALcaptiondefn {#1}%
}

% Resetea los caption
\newcommand{\resetindexcaption}{%
	\global\def\GLOBALcaptiondefn {\GLOBALemptyvar}%
	\hbadness=10000%
}

% Cambia los márgenes del documento
%	#1	Margen izquierdo
%	#2	Margen derecho
\newcommand{\changemargin}[2]{%
	\emptyvarerr{\changemargin}{#1}{Margen izquierdo no definido}%
	\emptyvarerr{\changemargin}{#2}{Margen derecho no definido}%
	\list{}{\rightmargin#2\leftmargin#1}\item[]%
}
\let\endchangemargin=\endlist

% Chequea que las funciones sólo puedan usarse en el entorno images
\newcommand{\checkonlyonenvimage}{%
	\ifthenelse{\equal{\GLOBALenvimageinitialized}{true}}{}{%
		\throwwarning{Funciones \noexpand\addimage o \noexpand\addimageboxed no pueden usarse fuera del entorno \noexpand\images}\stop%
	}%
}

% Chequea que las funciones sólo puedan usarse fuera del entorno images
\newcommand{\checkoutsideenvimage}{%
	\ifthenelse{\equal{\GLOBALenvimageinitialized}{true}}{%
		\throwwarning{Esta funcion solo puede usarse fuera del entorno \noexpand\images}%
		\stop}{%
	}%
}

% Chequea que las funciones puedan usarse solo en el entorno multicol
\newcommand{\checkinsidemulticol}{%
	\ifthenelse{\equal{\GLOBALenvmulticol}{false}}{%
		\throwwarning{Esta funcion solo puede usarse dentro de multicols}%
		\stop}{%
	}%
}

% Chequea que las funciones puedan usarse fuera del entorno anexo
\newcommand{\checkoutsideappendix}{%
	\ifthenelse{\equal{\GLOBALenvappendix}{true}}{%
		\throwwarning{Esta funcion solo puede usarse fuera de anexo}%
		\stop}{%
	}%
}

% Verifica que una variable sea del estilo "true" o "false"
\newcommand{\corecheckbooleanvar}[1]{%
	\emptyvarerr{\corecheckbooleanvar}{#1}{Variable no definida}%
	\ifthenelse{\equal{#1}{true}}{}{%
	\ifthenelse{\equal{#1}{false}}{}{%
		\throwwarning{Variable debe ser true o false}\stop%
	}}%
}

% Centra verticalmente un texto
%	#1	Texto a centrar
\newcommand{\verticallycentertext}[1]{%
	\emptyvarerr{\verticallycentertext}{#1}{Texto no definido}%
	\topskip0pt%
	\vspace*{\fill}%
	#1%
	\vspace*{\fill}%
}

% Inserta un espacio vertical en cm con una variación +/-
%	#1	Espacio (en cm)
\newcommand{\corevspacevarcm}[1]{%
	\ifthenelse{\equal{#1}{0}}{}{%
	\ifthenelse{\equal{#1}{0.0}}{}{%
		\vspace{\dimexpr#1 cm plus #1\coregluevarcm minus #1\coregluevarcm}%
	}}%
}

% Agrega una carpeta al path de imágenes
%	#1	Carpeta
\makeatletter
\newcommand\addpathimage[1]{%
	\gappto\Ginput@path{{#1}}%
}
\makeatother

% Verifica que un tamaño de fuente sea correcto
%	#1	Tamaño de fuente
\newcommand{\corecheckfontsize}[1]{%
	\ifthenelse{\equal{#1}{normalsize}}{}{%
	\ifthenelse{\equal{#1}{small}}{}{%
	\ifthenelse{\equal{#1}{large}}{}{%
	\ifthenelse{\equal{#1}{Large}}{}{%
	\ifthenelse{\equal{#1}{LARGE}}{}{%
	\ifthenelse{\equal{#1}{huge}}{}{%
	\ifthenelse{\equal{#1}{Huge}}{}{%
	\ifthenelse{\equal{#1}{HUGE}}{}{%
	\ifthenelse{\equal{#1}{footnotesize}}{}{%
	\ifthenelse{\equal{#1}{scriptsize}}{}{%
	\ifthenelse{\equal{#1}{tiny}}{}{%
		\errmessage{LaTeX Warning: Tamano de fuente incorrecto (\noexpand #1= #1). Valores esperados: tiny,scriptsize,footnotesize,small,normalisize,large,Large,LARGE,huge,Huge,HUGE}%
		\stop%
		}}}}}}}}}}%
	}%
}

% Insertar sub-índice, a_b
% 	#1	Elemento inferior (a)
%	#2	Elemento superior (b)
\newcommand{\lpow}[2]{%
	\ensuremath{{#1}_{#2}}
}

% Insertar elevado, a^b
% 	#1	Elemento inferior (a)
%	#2	Elemento superior (b)
\newcommand{\pow}[2]{%
	\ensuremath{{#1}^{#2}}
}

% Inserta inverso función seno, sin^-1
%	#1	Elemento
\newcommand{\aasin}[1][]{%
	\ifx\hfuzz#1\hfuzz%
		\ensuremath{\sin^{-1}#1}
	\else%
		\ensuremath{{\sin}^{-1}}
	\fi%
}

% Inserta inverso función coseno, cos^-1
%	#1	Elemento
\newcommand{\aacos}[1][]{%
	\ifx\hfuzz#1\hfuzz%
		\ensuremath{\cos^{-1}#1}
	\else%
		\ensuremath{\cos^{-1}}
	\fi%
}

% Inserta inverso función tangente, tan^-1
%	#1	Elemento
\newcommand{\aatan}[1][]{%
	\ifx\hfuzz#1\hfuzz%
		\ensuremath{\tan^{-1}#1}
	\else%
		\ensuremath{\tan^{-1}}
	\fi%
}

% Inserta inverso función cosecante, csc^-1
%	#1	Elemento
\newcommand{\aacsc}[1][]{%
	\ifx\hfuzz#1\hfuzz%
		\ensuremath{\csc^{-1}#1}
	\else%
		\ensuremath{\csc^{-1}}
	\fi%
}

% Inserta inverso función secante, sec^-1
%	#1	Elemento
\newcommand{\aasec}[1][]{%
	\ifx\hfuzz#1\hfuzz%
		\ensuremath{\sec^{-1}#1}
	\else%
		\ensuremath{\sec^{-1}}
	\fi%
}

% Inserta inverso función cotangente, cot^-1
%	#1	Elemento
\newcommand{\aacot}[1][]{%
	\ifx\hfuzz#1\hfuzz%
		\ensuremath{\cot^{-1}#1}
	\else%
		\ensuremath{\cot^{-1}}
	\fi%
}

% Fracción de derivadas parciales af/ax
% 	#1	Función a derivar (f)
%	#2	Variable a derivar (x)
\newcommand{\fracpartial}[2]{%
	\ensuremath{\pdv{#1}{#2}}
}

% Fracción de derivadas parciales dobles a^2f/ax^2
% 	#1	Función a derivar (f)
%	#2	Variable a derivar (x)
\newcommand{\fracdpartial}[2]{%
	\ensuremath{\pdv[2]{#1}{#2}}
}

% Fracción de derivadas parciales en n, a^nf/ax^n
% 	#1	Función a derivar (f)
%	#2	Variable a derivar (x)
%	#3	Orden (n)
\newcommand{\fracnpartial}[3]{%
	\ensuremath{\pdv[#3]{#1}{#2}}
}

% Fracción de derivadas df/dx
% 	#1	Función a derivar (f)
%	#2	Variable a derivar (x)
\newcommand{\fracderivat}[2]{%
	\ensuremath{\dv{#1}{#2}}
}

% Fracción de derivadas dobles d^2/dx^2
% 	#1	Función a derivar (f)
%	#2	Variable a derivar (x)
\newcommand{\fracdderivat}[2]{%
	\ensuremath{\dv[2]{#1}{#2}}
}

% Fracción de derivadas en n d^nf/dx^n
% 	#1	Función a derivar (f)
%	#2	Variable a derivar (x)
%	#3	Orden de la derivada (n)
\newcommand{\fracnderivat}[3]{%
	\ensuremath{\dv[#3]{#1}{#2}}
}

% Llave superior de equivalencia
% 	#1	Elemento a igualar
%	#2	Igualdad
\newcommand{\topequal}[2]{%
	\ensuremath{\overbrace{#1}^{\mathclap{#2}}}
}
\newcommand{\topequaltext}[2]{%
	\topequal{#1}{\text{#2}}
}

% Llave inferior de equivalencia
% 	#1	Elemento a igualar
%	#2	Igualdad
\newcommand{\underequal}[2]{%
	\ensuremath{\underbrace{#1}_{\mathclap{#2}}}
}
\newcommand{\underequaltext}[2]{%
	\underequal{#1}{\text{#2}}
}

% Rectángulo superior de equivalencia
% 	#1	Elemento a igualar
%	#2	Igualdad
\newcommand{\topsequal}[2]{%
	\ensuremath{\overbracket{#1}^{\mathclap{#2}}}
}
\newcommand{\topsequaltext}[2]{%
	\topsequal{#1}{\text{#2}}
}

% Rectángulo inferior de equivalencia
% 	#1	Elemento a igualar
%	#2	Igualdad
\newcommand{\undersequal}[2]{%
	\ensuremath{\underbracket{#1}_{\mathclap{#2}}}
}
\newcommand{\undersequaltext}[2]{%
	\undersequal{#1}{\text{#2}}
}

% Función piso
% 	#1	Elemento
\newcommand{\floorexp}[1]{%
	\ensuremath{\left\lfloor{#1}\right\rfloor}
}

% Función techo
% 	#1	Elemento
\newcommand{\ceilexp}[1]{%
	\ensuremath{\left\lceil{#1}\right\rceil}
}

% Función mod
%	#1	Elemento tal que (mod #1)
\newcommand{\Mod}[1]{%
	\ensuremath{\ (\mathrm{mod}\ #1)}
}

% Paréntesis grande
% 	#1	Expresión
\newcommand{\bigp}[1]{%
	\ensuremath{\big(#1\big)}
}

% Paréntesis g+grande
% 	#1	Expresión
\newcommand{\biggp}[1]{%
	\ensuremath{\bigg(#1\bigg)}
}

% Cajón grande
% 	#1	Expresión
\newcommand{\bigc}[1]{%
	\ensuremath{\big[#1\big]}
}

% Cajón g+grande
% 	#1	Expresión
\newcommand{\biggc}[1]{%
	\ensuremath{\bigg[#1\bigg]}
}

% Llave grande
% 	#1	Expresión
\newcommand{\bigb}[1]{%
	\ensuremath{\big\{#1\big\}}
}

% Llave g+grande
% 	#1	Expresión
\newcommand{\biggb}[1]{%
	\ensuremath{\bigg\{#1\bigg\}}
}

% Expresión divergencia
\newcommand{\divexp}{%
	\ensuremath{\rm{div}\ }
}

% Expresión automorfismo
\newcommand{\Autexp}{%
	\ensuremath{\rm{Aut}}
}

% Negrita introducida por word
% 	#1	Expresión
\newcommand{\mathbit}[1]{%
	\bm{#1}
}

% Expresión diff
\newcommand{\Diffexp}{%
	\ensuremath{\rm{Diff}}
}

% Expresión imaginario
\newcommand{\Imexp}{%
	\ensuremath{\rm{Im}}
}

% Expresión imaginario en z
\newcommand{\Imzexp}{%
	\ensuremath{\rm{Im}(z)}
}

% Expresión real
\newcommand{\Reexp}{%
	\ensuremath{\rm{Re}}
}

% Expresión real en z
\newcommand{\Rezexp}{%
	\ensuremath{\rm{Re}(z)}
}

% Barra superior en elemento
%	#1 	Elemento
\newcommand{\overbar}[1]{%
	\mkern 1.5mu\overline{\mkern-1.5mu#1\mkern-1.5mu}\mkern 1.5mu
}

% Función \tilde{} pero que encierra todo el texto
%	#1 	Elemento
\makeatletter
\def\longtilde#1{%
	\mathop{\vbox{\m@th\ialign{##\crcr\noalign{\kern3\p@}%
	\sortoftildefill\crcr\noalign{\kern3\p@\nointerlineskip}%
	$\hfil\displaystyle{#1}\hfil$\crcr}}}\limits%
}
\def\sortoftildefill {%
	$\m@th \setbox\z@\hbox{$\braceld$}%
	\braceld\leaders\vrule \@height\ht\z@ \@depth\z@\hfill\braceru$%
}
\makeatother

% Definición de letras
\newcommand{\A}{\ensuremath{\mathcal{A}}}

\newcommand{\B}{\ensuremath{\mathcal{B}}}

\ifthenelse{\isundefined{\C}}{\newcommand{\C}{C}}{\let\oldC=\C}
\renewcommand{\C}{\ensuremath{\mathbb{C}}}

\newcommand{\D}{\ensuremath{\mathbb{D}}}

\newcommand{\E}{\ensuremath{\mathbb{E}}}

\newcommand{\F}{\ensuremath{\mathcal{F}}}

\ifthenelse{\isundefined{\G}}{\newcommand{\G}{G}}{\let\oldG=\G}
\renewcommand{\G}{\ensuremath{\mathcal{G}}}

\ifthenelse{\isundefined{\H}}{\newcommand{\H}{H}}{\let\oldH=\H}
\renewcommand{\H}{\ensuremath{\mathcal{H}}}

\newcommand{\I}{\ensuremath{\mathbb{I}}}

\newcommand{\J}{\ensuremath{\mathcal{J}}}

\newcommand{\K}{\ensuremath{\mathcal{K}}}

\let\oldL=\L % L con una raya
\renewcommand{\L}{\ensuremath{\mathcal{L}}}

\newcommand{\M}{\ensuremath{\mathcal{M}}}

\newcommand{\N}{\ensuremath{\mathbb{N}}}

% \renewcommand{\O}{\ensuremath{\mathbb{O}}} % O equivale a o/oo

\let\oldP=\P % P negra
\renewcommand{\P}{\ensuremath{\mathbb{P}}}

\newcommand{\Q}{\ensuremath{\mathbb{Q}}}

\newcommand{\R}{\ensuremath{\mathbb{R}}}

\let\oldS=\S % Serpiente
\renewcommand{\S}{\ensuremath{\mathcal{S}}}

\newcommand{\T}{\ensuremath{\mathcal{T}}}

\ifthenelse{\isundefined{\U}}{\newcommand{\U}{U}}{\let\oldU=\U}
\renewcommand{\U}{\ensuremath{\mathcal{U}}}

\newcommand{\V}{\ensuremath{\mathcal{V}}}

\newcommand{\W}{\ensuremath{\mathcal{W}}}

\newcommand{\X}{\ensuremath{\mathcal{X}}}

\newcommand{\Y}{\ensuremath{\mathcal{Y}}}

\newcommand{\Z}{\ensuremath{\mathbb{Z}}}

% Definición de operadores matemáticos de asignación (Typeset assigments)
\ifthenelse{\equal{\fontdocument}{step}}{}{% Ya definidos en STEP
	\newcommand{\asteq}{\ensuremath{\mathrel{{*}{=}}}}
	\newcommand{\eqeq}{\ensuremath{\mathrel{{=}{=}}}}
}
\newcommand{\cdoteq}{\ensuremath{\mathrel{{\cdot}{=}}}}
\newcommand{\diveq}{\ensuremath{\mathrel{{/}{=}}}}
\newcommand{\eqast}{\ensuremath{\mathrel{{=}{*}}}}
\newcommand{\eqcdot}{\ensuremath{\mathrel{{=}{\cdot}}}}
\newcommand{\eqdiv}{\ensuremath{\mathrel{{=}{/}}}}
\newcommand{\eqminus}{\ensuremath{\mathrel{{=}{-}}}}
\newcommand{\eqnot}{\ensuremath{\mathrel{{=}{!}}}}
\newcommand{\eqplus}{\ensuremath{\mathrel{{=}{+}}}}
\newcommand{\eqtimes}{\ensuremath{\mathrel{{=}{\times}}}}
\newcommand{\minuseq}{\ensuremath{\mathrel{{-}{=}}}}
\newcommand{\minusminus}{\ensuremath{\mathrel{{-}{-}}}}
\newcommand{\noteq}{\ensuremath{\mathrel{{!}{=}}}}
\newcommand{\pluseq}{\ensuremath{\mathrel{{+}{=}}}}
\newcommand{\plusplus}{\ensuremath{\mathrel{{+}{+}}}}
\newcommand{\timeseq}{\ensuremath{\mathrel{{\times}{=}}}}

% Definición de teoremas y lemas
\makeatletter
	\renewenvironment{proof}[1][\proofname]{%
		\par\pushQED{\qed}%
		\normalfont\topsep6\p@\@plus6\p@\relax\trivlist%
		\item[\hskip\labelsep\scshape\footnotesize#1\@addpunct{.}]%
		\ignorespaces%
	}{%
		\popQED\endtrivlist\@endpefalse%
	}%
\makeatother

% Función que se ejecuta tras un equation
\newcommand{\coreafterequationfn}{%
	\hbadness=10000%
}

% Redimensiona una ecuación en linewidth
% 	#1	Tamaño del nuevo objeto (En linewidth)
%	#2	Ecuación a redimensionar
\newcommand{\equationresize}[2]{%
	\emptyvarerr{\equationresize}{#1}{Dimension no definida}%
	\emptyvarerr{\equationresize}{#2}{Ecuacion a redimensionar no definida}%
	\resizebox{#1\linewidth}{!}{$#2$}%
}

% Inserta el caption de un objeto tipo ecuación
%	#1	Texto del caption
\newcommand{\coreinsertequationcaption}[1]{%
	\begin{changemargin}{\captionlrmargin cm}{\captionlrmargin cm}%
		\ifthenelse{\equal{\equationcaptioncenter}{true}}{%
			\centering%
		}{%
			\justifying%
		}%
		\textcolor{\captiontextcolor}{%
			\linespread{0.5}\selectfont{%
				\begin{\captionfontsize}#1\end{\captionfontsize}%
			}%
		}%
	\end{changemargin}
}

% Insertar una ecuación
% 	#1	Label (opcional)
%	#2	Ecuación
\newcommand{\insertequation}[2][]{%
	\emptyvarerr{\insertequation}{#2}{Ecuacion no definida}%
	\ifthenelse{\equal{\numberedequation}{true}}{%
		\corevspacevarcm{\marginequationtop}%
		\begin{samepage}%
		\begin{equation}%
			\text{#1} #2
		\end{equation}
		\corevspacevarcm{\marginequationbottom}%
		\end{samepage}
		\coreafterequationfn%
	}{%
		\ifx\hfuzz#1\hfuzz%
		\else%
			\throwwarning{Label invalido en ecuacion sin numero}%
		\fi%
		\insertequationanum{#2}%
	}%
}

% Insertar una ecuación sin número
%	#1	Ecuación
\newcommand{\insertequationanum}[1]{%
	\emptyvarerr{\insertequationanum}{#1}{Ecuacion no definida}%
	\corevspacevarcm{\marginequationtop}%
	\begin{samepage}%
	\begin{equation*}%
		\ensuremath{#1}
	\end{equation*}
	\corevspacevarcm{\marginequationbottom}%
	\end{samepage}
	\coreafterequationfn%
}

% Insertar una ecuación en el índice
% 	#1	Label (opcional)
%	#2	Ecuación
%	#3	Leyenda de la ecuación
\newcommand{\insertindexequation}[3][]{%
	\emptyvarerr{\insertindexequation}{#2}{Ecuacion no definida}%
	\emptyvarerr{\insertindexequation}{#3}{Leyenda no definida}%
	\begin{equationindex}[#1]{#3}%
		#2
	\end{equationindex}
}

% Insertar una ecuación alineada a la izquierda
% 	#1	Label (opcional)
%	#2	Ecuación
\newcommand{\insertequationleft}[2][]{%
	\emptyvarerr{\insertequationleft}{#2}{Ecuacion no definida}%
	\ifthenelse{\equal{\numberedequation}{true}}{%
		\vspace{\dimexpr\marginequationtop cm - \baselineskip}%
		\begin{samepage}%
		\begin{equation}
			\hfilneg \text{#1} #2 \hspace{10000pt minus 1fil}
		\end{equation}
		\vspace{\dimexpr-0.2\baselineskip + \marginequationbottom cm}%
		\end{samepage}
		\coreafterequationfn%
	}{%
		\ifx\hfuzz#1\hfuzz%
		\else%
			\throwwarning{Label invalido en ecuacion sin numero}%
		\fi%
		\insertequationleftanum{#2}%
	}%
}

% Insertar una ecuación sin número alineada a la izquierda
%	#1	Ecuación
\newcommand{\insertequationleftanum}[1]{%
	\emptyvarerr{\insertequationleftanum}{#1}{Ecuacion no definida}%
	\vspace{\dimexpr\marginequationtop cm - \baselineskip}%
	\begin{samepage}%
	\begin{equation*}
		\hfilneg \ensuremath{#1} \hspace{10000pt minus 1fil}
	\end{equation*}
	\vspace{\dimexpr-0.2\baselineskip + \marginequationbottom cm}%
	\end{samepage}
	\coreafterequationfn%
}

% Insertar una ecuación alineada a la derecha
% 	#1	Label (opcional)
%	#2	Ecuación
\newcommand{\insertequationright}[2][]{%
	\emptyvarerr{\insertequationright}{#2}{Ecuacion no definida}%
	\ifthenelse{\equal{\numberedequation}{true}}{%
		\vspace{\dimexpr\marginequationtop cm - \baselineskip}%
		\begin{samepage}%
		\begin{equation}
			\hspace{10000pt minus 1fil} \text{#1} #2 \hfilneg
		\end{equation}
		\vspace{\dimexpr-0.2\baselineskip + \marginequationbottom cm}%
		\end{samepage}
		\coreafterequationfn%
	}{%
		\ifx\hfuzz#1\hfuzz%
		\else%
			\throwwarning{Label invalido en ecuacion sin numero}%
		\fi%
		\insertequationrightanum{#2}%
	}%
}

% Insertar una ecuación sin número alineada a la derecha
%	#1	Ecuación
\newcommand{\insertequationrightanum}[1]{%
	\emptyvarerr{\insertequationrightanum}{#1}{Ecuacion no definida}%
	\vspace{\dimexpr\marginequationtop cm - \baselineskip}%
	\begin{samepage}%
	\begin{equation*}
		\hspace{10000pt minus 1fil} \ensuremath{#1} \hfilneg
	\end{equation*}
	\vspace{\dimexpr-0.2\baselineskip + \marginequationbottom cm}%
	\end{samepage}
	\coreafterequationfn%
}

% Insertar una ecuación con leyenda
% 	#1	Label (opcional)
%	#2	Ecuación
%	#3	Leyenda
\newcommand{\insertequationcaptioned}[3][]{%
	\emptyvarerr{\insertequationcaptioned}{#2}{Ecuacion no definida}%
	\ifx\hfuzz#3\hfuzz%
		\insertequation[#1]{#2}%
	\else%
		\ifthenelse{\equal{\numberedequation}{true}}{%
			\corevspacevarcm{\marginequationtop}%
			\begin{samepage}%
			\begin{equation}
				\text{#1} #2
			\end{equation}
			\corevspacevarcm{\margineqncaptiontop}%
			\coreinsertequationcaption{#3}%
			\corevspacevarcm{\margineqncaptionbottom}%
			\end{samepage}
			\coreafterequationfn%
		}{%
			\ifx\hfuzz#1\hfuzz%
			\else%
				\throwwarning{Label invalido en ecuacion sin numero}%
			\fi%
			\insertequationcaptionedanum{#2}{#3}%
		}%
	\fi%
}

% Insertar una ecuación con leyenda sin número
%	#1	Ecuación
%	#2	Leyenda
\newcommand{\insertequationcaptionedanum}[2]{%
	\emptyvarerr{\insertequationcaptionedanum}{#1}{Ecuacion no definida}%
	\ifx\hfuzz#2\hfuzz%
		\insertequationanum{#1}%
	\else%
		\corevspacevarcm{\marginequationtop}%
		\begin{samepage}%
		\begin{equation*}
			\ensuremath{#1}%
		\end{equation*}
		\corevspacevarcm{\margineqncaptiontop}%
		\coreinsertequationcaption{#2}%
		\corevspacevarcm{\margineqncaptionbottom}%
		\end{samepage}
		\coreafterequationfn%
	\fi%
}

% Insertar una ecuación con el ambiente gather
%	#1	Ecuación
\newcommand{\insertgather}[1]{%
	\emptyvarerr{\insertgather}{#1}{Ecuacion no definida}%
	\ifthenelse{\equal{\numberedequation}{true}}{%
		\corevspacevarcm{\margingathertop}%
		\begin{samepage}%
		\begin{gather}%
			\ensuremath{#1}
		\end{gather}
		\corevspacevarcm{\margingatherbottom}%
		\end{samepage}
		\coreafterequationfn%
	}{%
		\insertgatheranum{#1}%
	}%
}

% Insertar una ecuación con el ambiente gather sin número
%	#1	Ecuación
\newcommand{\insertgatheranum}[1]{%
	\emptyvarerr{\insertgatheranum}{#1}{Ecuacion no definida}%
	\corevspacevarcm{\margingathertop}%
	\begin{samepage}%
	\begin{gather*}%
		\ensuremath{#1}
	\end{gather*}
	\corevspacevarcm{\margingatherbottom}%
	\end{samepage}
	\coreafterequationfn%
}

% Insertar una ecuación (gather) con leyenda
%	#1	Ecuación
%	#2	Leyenda
\newcommand{\insertgathercaptioned}[2]{%
	\emptyvarerr{\insertgathercaptioned}{#1}{Ecuacion no definida}%
	\ifx\hfuzz#2\hfuzz%
		\insertgather{#1}%
	\else%
		\ifthenelse{\equal{\numberedequation}{true}}{%
			\corevspacevarcm{\margingathertop}%
			\begin{samepage}%
			\begin{gather}%
				\ensuremath{#1}
			\end{gather}
			\corevspacevarcm{\margingathercapttop}%
			\coreinsertequationcaption{#2}%
			\corevspacevarcm{\margingathercaptbottom}%
			\end{samepage}
			\coreafterequationfn%
		}{%
			\insertgathercaptionedanum{#1}{#2}%
		}%
	\fi%
}

% Insertar una ecuación (gather) con leyenda sin número
%	#1	Ecuación
%	#2	Leyenda
\newcommand{\insertgathercaptionedanum}[2]{%
	\emptyvarerr{\insertgathercaptionedanum}{#1}{Ecuacion no definida}%
	\ifx\hfuzz#2\hfuzz%
		\insertgatheranum{#1}%
	\else%
		\corevspacevarcm{\margingathertop}%
		\begin{samepage}%
		\begin{gather*}%
			\ensuremath{#1}
		\end{gather*}
		\corevspacevarcm{\margingathercapttop}%
		\coreinsertequationcaption{#2}%
		\corevspacevarcm{\margingathercaptbottom}%
		\end{samepage}
		\coreafterequationfn%
	\fi%
}

% Insertar una ecuación con el ambiente gathered
% 	#1	Label (opcional)
%	#2	Ecuación
\newcommand{\insertgathered}[2][]{%
	\emptyvarerr{\insertgathered}{#2}{Ecuacion no definida}%
	\ifthenelse{\equal{\numberedequation}{true}}{%
		\corevspacevarcm{\marginequationtop}%
		\begin{samepage}%
		\begin{equation}
			\begin{gathered}
				\text{#1} \ensuremath{#2}
			\end{gathered}
		\end{equation}
		\corevspacevarcm{\margingatheredbottom}%
		\end{samepage}
	}{%
		\ifx\hfuzz#1\hfuzz%
		\else%
			\throwwarning{Label invalido en ecuacion (gathered) sin numero}%
		\fi%
		\corevspacevarcm{\margingatheredtop}%
		\begin{samepage}%
		\begin{gather*}%
			\ensuremath{#2}
		\end{gather*}
		\corevspacevarcm{\margingatheredbottom}%
		\end{samepage}
	}%
	\coreafterequationfn%
}

% Insertar una ecuación con el ambiente gathered sin número
%	#1	Ecuación
\newcommand{\insertgatheredanum}[1]{%
	\emptyvarerr{\insertgatheredanum}{#1}{Ecuacion no definida}%
	\corevspacevarcm{\margingatheredtop}%
	\begin{samepage}%
	\begin{gather*}
		\ensuremath{#1}
	\end{gather*}
	\vspace{\dimexpr-0.15cm + \margingatheredbottom cm}%
	\end{samepage}
	\coreafterequationfn%
}

% Insertar una ecuación (gathered) con leyenda
% 	#1	Label (opcional)
%	#2	Ecuación
%	#3	Leyenda
\newcommand{\insertgatheredcaptioned}[3][]{%
	\emptyvarerr{\insertgatheredcaptioned}{#2}{Ecuacion no definida}%
	\ifx\hfuzz#3\hfuzz%
		\insertgathered[#1]{#2}%
	\else%
		\ifthenelse{\equal{\numberedequation}{true}}{%
			\corevspacevarcm{\marginequationtop}%
			\begin{samepage}%
			\begin{equation}
				\begin{gathered}
					\text{#1} \ensuremath{#2}
				\end{gathered}
			\end{equation}
			\corevspacevarcm{\margingatheredcapttop}%
			\coreinsertequationcaption{#3}%
			\corevspacevarcm{\margingatheredcaptbottom}%
			\end{samepage}
			\coreafterequationfn%
		}{%
			\ifx\hfuzz#1\hfuzz%
			\else%
				\throwwarning{Label invalido en ecuacion (gathered) sin numero}
			\fi%
			\insertgatheredcaptionedanum{#2}{#3}%
		}%
	\fi%
}

% Insertar una ecuación (gathered) con leyenda sin número
%	#1	Ecuación
%	#2	Leyenda
\newcommand{\insertgatheredcaptionedanum}[2]{%
	\emptyvarerr{\insertgatheredcaptionedanum}{#1}{Ecuacion no definida}%
	\ifx\hfuzz#2\hfuzz%
		\insertgatheredanum{#1}%
	\else%
		\corevspacevarcm{\margingatheredtop}%
		\begin{samepage}%
		\begin{gather*}
			\ensuremath{#1}
		\end{gather*}
		\vspace{\dimexpr-0.2cm + \margingatheredcapttop cm}%
		\coreinsertequationcaption{#2}%
		\vspace{\dimexpr-0.05cm + \margingatheredcaptbottom cm}%
		\end{samepage}
		\coreafterequationfn%
	\fi%
}

% Insertar una ecuación con el ambiente align
%	#1	Ecuación
\newcommand{\insertalign}[1]{%
	\emptyvarerr{\insertalign}{#1}{Ecuacion no definida}%
	\ifthenelse{\equal{\numberedequation}{true}}{%
		\corevspacevarcm{\marginaligntop}%
		\begin{samepage}%
		\begin{align}
			\ensuremath{#1}
		\end{align}
		\corevspacevarcm{\marginalignbottom}%
		\end{samepage}
		\coreafterequationfn%
	}{%
		\insertalignanum{#1}%
	}%
}

% Insertar una ecuación con el ambiente align sin número
%	#1	Ecuación
\newcommand{\insertalignanum}[1]{%
	\emptyvarerr{\insertalignanum}{#1}{Ecuacion no definida}%
	\corevspacevarcm{\marginaligntop}%
	\begin{samepage}%
	\begin{align*}
		\ensuremath{#1}
	\end{align*}
	\corevspacevarcm{\marginalignbottom}%
	\end{samepage}
	\coreafterequationfn%
}

% Insertar una ecuación (align) con leyenda
%	#1	Ecuación
%	#2	Leyenda
\newcommand{\insertaligncaptioned}[2]{%
	\emptyvarerr{\insertaligncaptioned}{#1}{Ecuacion no definida}%
	\ifx\hfuzz#2\hfuzz%
		\insertalign{#1}%
	\else%
		\ifthenelse{\equal{\numberedequation}{true}}{%
			\corevspacevarcm{\marginaligntop}%
			\begin{samepage}%
			\begin{align}
				\ensuremath{#1}
			\end{align}
			\corevspacevarcm{\marginaligncapttop}%
			\coreinsertequationcaption{#2}%
			\corevspacevarcm{\marginaligncaptbottom}%
			\end{samepage}
			\coreafterequationfn%
		}{%
			\insertaligncaptionedanum{#1}{#2}%
		}%
	\fi%
}

% Insertar una ecuación (align) con leyenda sin número
%	#1	Ecuación
%	#2	Leyenda
\newcommand{\insertaligncaptionedanum}[2]{%
	\emptyvarerr{\insertaligncaptionedanum}{#1}{Ecuacion no definida}%
	\ifx\hfuzz#2\hfuzz%
		\insertalignanum{#1}%
	\else%
		\corevspacevarcm{\marginaligntop}%
		\begin{samepage}%
		\begin{align*}
			\ensuremath{#1}
		\end{align*}
		\corevspacevarcm{\marginaligncapttop}%
		\coreinsertequationcaption{#2}%
		\corevspacevarcm{\marginaligncaptbottom}%
		\end{samepage}
		\coreafterequationfn%
	\fi%
}

% Insertar una ecuación con el ambiente aligned
% 	#1	Label (opcional)
%	#2	Ecuación
\newcommand{\insertaligned}[2][]{%
	\emptyvarerr{\insertaligned}{#2}{Ecuacion no definida}%
	\ifthenelse{\equal{\numberedequation}{true}}{%
		\corevspacevarcm{\marginequationtop}%
		\begin{samepage}%
		\begin{equation}
			\begin{aligned}
				\text{#1} \ensuremath{#2}
			\end{aligned}
		\end{equation}
		\corevspacevarcm{\marginalignedbottom}%
		\end{samepage}
		\coreafterequationfn%
	}{%
		\ifx\hfuzz#1\hfuzz%
		\else%
			\throwwarning{Label invalido en ecuacion (aligned) sin numero}%
		\fi%
		\insertalignedanum{#2}%
	}%
}

% Insertar una ecuación con el ambiente aligned sin número
%	#1	Ecuación
\newcommand{\insertalignedanum}[1]{%
	\emptyvarerr{\insertalignedanum}{#1}{Ecuacion no definida}%
	\corevspacevarcm{\marginalignedtop}%
	\begin{samepage}%
	\begin{align*}
		\ensuremath{#1}
	\end{align*}
	\vspace{\dimexpr-0.2cm + \marginalignedbottom cm}%
	\end{samepage}
	\coreafterequationfn%
}

% Insertar una ecuación (aligned) con leyenda
% 	#1	Label (opcional)
%	#2	Ecuación
%	#3	Leyenda
\newcommand{\insertalignedcaptioned}[3][]{%
	\emptyvarerr{\insertalignedcaptioned}{#2}{Ecuacion no definida}%
	\ifx\hfuzz#3\hfuzz%
		\insertaligned[#1]{#2}%
	\else%
		\ifthenelse{\equal{\numberedequation}{true}}{%
			\corevspacevarcm{\marginequationtop}%
			\begin{samepage}%
			\begin{equation}
				\begin{aligned}
					\text{#1} \ensuremath{#2}
				\end{aligned}
			\end{equation}
			\corevspacevarcm{\marginalignedcapttop}%
			\coreinsertequationcaption{#3}%
			\corevspacevarcm{\marginalignedcaptbottom}%
			\end{samepage}
			\coreafterequationfn%
		}{%
			\ifx\hfuzz#1\hfuzz%
			\else%
				\throwwarning{Label invalido en ecuacion (aligned) sin numero}%
			\fi%
			\insertalignedcaptionedanum{#2}{#3}%
		}%
	\fi%
}

% Insertar una ecuación (aligned) con leyenda sin número
%	#1	Ecuación
%	#2	Leyenda
\newcommand{\insertalignedcaptionedanum}[2]{%
	\emptyvarerr{\insertalignedcaptionedanum}{#1}{Ecuacion no definida}%
	\ifx\hfuzz#2\hfuzz%
		\insertalignedanum{#1}%
	\else%
		\corevspacevarcm{\marginequationtop}%
		\begin{samepage}%
		\begin{equation}
			\begin{aligned}
				\ensuremath{#1}
			\end{aligned}
		\end{equation}
		\corevspacevarcm{\marginalignedcapttop}%
		\coreinsertequationcaption{#2}%
		\corevspacevarcm{\marginalignedcaptbottom}%
		\end{samepage}
		\coreafterequationfn%
	\fi%
}

\global\def\GLOBALimagelink {\GLOBALemptyvar} % Almacena el link de la imagen
\global\def\GLOBALimagenextmarginv {0 cm} % Almacena el margen vertical de las imágenes

% Calcula largo hspace
% Regresión entre 35,46446->9 y 52,68402->13,5
\newlength{\coreimageshspace}
\setlength{\coreimageshspace}{\dimexpr 9pt + 0.261330719\corefontwidth - 9.26795284pt}

% Añade una imagen en el entorno "images" con borde
% 	#1	Label (opcional)
%	#2	Dirección de la imagen
%	#3	Parámetros de la imagen
%	#4	Leyenda de la imagen (opcional)
\newcommand{\addimage}[4][]{%
	\addimageboxed[#1]{#2}{#3}{0}{#4}%
}

% Añade una imagen en el entorno "images" con borde
% 	#1	Label (opcional)
%	#2	Dirección de la imagen
%	#3	Parámetros de la imagen
%	#4	Ancho de la línea (en pt)
%	#5	Leyenda de la imagen (opcional)
\newcommand{\addimageboxed}[5][]{%
	\checkonlyonenvimage%
	\begingroup%
	\setlength{\fboxsep}{0 pt}%
	\setlength{\fboxrule}{#4 pt}%
	\ifthenelse{\equal{\GLOBALenvimageadded}{true}}{%
		\hspace{\dimexpr \marginimagemultright cm -\coreimageshspace}%
	}{} % Obs: No sacar el espacio en blanco, tampoco en sin/con label
	\ifthenelse{\equal{#5}{\GLOBALemptyvar}}{ % Sin label
		\ifthenelse{\equal{\GLOBALimagelink}{\GLOBALemptyvar}}{% Sin link
			\raisebox{\GLOBALimagenextmarginv}{%
				\fbox{\includegraphics[#3]{#2}}%
			}%
		}{% Con link
			\raisebox{\GLOBALimagenextmarginv}{%
				\fbox{\href{\GLOBALimagelink}{\includegraphics[#3]{#2}}}%
			}%
		}%
	}{ % Con label
		\ifthenelse{\equal{\GLOBALimagelink}{\GLOBALemptyvar}}{% Sin link
			\subfloat[#5#1]{%
				\raisebox{\GLOBALimagenextmarginv}{%
					\fbox{\includegraphics[#3]{#2}}%
				}%
			}%
		}{% Con link
			\subfloat[#5#1]{%
				\raisebox{\GLOBALimagenextmarginv}{%
					\fbox{\href{\GLOBALimagelink}{\includegraphics[#3]{#2}}}%
				}%
			}%
		}%
	}%
	\endgroup%
	\global\def\GLOBALenvimageadded {true}%
	\global\def\GLOBALimagenextmarginv {0 cm}%
}

% Añade una imagen en el entorno "images" con borde sin leyenda
%	#1	Dirección de la imagen
%	#2	Parámetros de la imagen
\newcommand{\addimageanum}[2]{%
	\addimageboxed{#1}{#2}{0}{\GLOBALemptyvar}%
}

% Añade una imagen en el entorno "images" con borde sin leyenda
%	#1	Dirección de la imagen
%	#2	Parámetros de la imagen
%	#3	Ancho de la línea (en pt)
\newcommand{\addimageanumboxed}[3]{%
	\addimageboxed{#1}{#2}{#3}{\GLOBALemptyvar}%
}

% Añade una imagen en el entorno "images" con borde animada
% 	#1	Label (opcional)
%	#2	Dirección de la imagen animada
%	#3	Parámetros de la imagen
%	#4	FPS de la imagen
%	#5	Total imágenes no definido
%	#6	Leyenda de la imagen (opcional)
\newcommand{\addimageanimated}[6][]{%
	\addimageanimatedboxed[#1]{#2}{#3}{#4}{#5}{0}{#6}%
}

% Añade una imagen en el entorno "images" con borde animada
% 	#1	Label (opcional)
%	#2	Dirección de la imagen animada
%	#3	Parámetros de la imagen
%	#4	FPS de la imagen
%	#5	Total imágenes no definido
%	#6	Ancho de la línea (en pt)
%	#7	Leyenda de la imagen (opcional)
\newcommand{\addimageanimatedboxed}[7][]{%
	\checkonlyonenvimage%
	\begingroup%
	\setlength{\fboxsep}{0 pt}%
	\setlength{\fboxrule}{#6 pt}%
	\ifthenelse{\equal{\GLOBALenvimageadded}{true}}{%
		\hspace{\dimexpr \marginimagemultright cm - \coreimageshspace}%
	}{}%
	\ifthenelse{\equal{#7}{\GLOBALemptyvar}}{% Sin label
		\ifthenelse{\equal{\animatedimageloop}{true}}{% Con loop
			\ifthenelse{\equal{\animatedimageautoplay}{true}}{% Con autoplay
				\raisebox{\GLOBALimagenextmarginv}{%
					\fbox{\animategraphics[loop,autoplay,#3]{#4}{#2-}{0}{#5}}%
				}%
			}{% Sin autoplay
				\raisebox{\GLOBALimagenextmarginv}{%
					\fbox{\animategraphics[loop,#3]{#4}{#2-}{0}{#5}}%
				}%
			}%
		}{% Sin loop
			\ifthenelse{\equal{\animatedimageautoplay}{true}}{% Con autoplay
				\raisebox{\GLOBALimagenextmarginv}{%
					\fbox{\animategraphics[autoplay,#3]{#4}{#2-}{0}{#5}}%
				}%
			}{% Sin autoplay
				\raisebox{\GLOBALimagenextmarginv}{%
					\fbox{\animategraphics[#3]{#4}{#2-}{0}{#5}}%
				}%
			}%
		}%
	}{% Con label
		\subfloat[#7#1]{%
			\ifthenelse{\equal{\animatedimageloop}{true}}{% Con loop
				\ifthenelse{\equal{\animatedimageautoplay}{true}}{% Con autoplay
					\raisebox{\GLOBALimagenextmarginv}{%
						\fbox{\animategraphics[loop,autoplay,#3]{#4}{#2-}{0}{#5}}%
					}%
				}{% Sin autoplay
					\raisebox{\GLOBALimagenextmarginv}{%
						\fbox{\animategraphics[loop,#3]{#4}{#2-}{0}{#5}}%
					}%
				}%
			}{% Sin loop
				\ifthenelse{\equal{\animatedimageautoplay}{true}}{% Con autoplay
					\raisebox{\GLOBALimagenextmarginv}{%
						\fbox{\animategraphics[autoplay,#3]{#4}{#2-}{0}{#5}}%
					}%
				}{% Sin autoplay
					\raisebox{\GLOBALimagenextmarginv}{%
						\fbox{\animategraphics[#3]{#4}{#2-}{0}{#5}}%
					}%
				}%
			}%
		}%
	}%
	\endgroup%
	\global\def\GLOBALenvimageadded {true}%
	\global\def\GLOBALimagenextmarginv {0 cm}%
}

% Añade una imagen en el entorno "images" con borde sin leyenda animada
%	#1	Dirección de la imagen animada
%	#2	Parámetros de la imagen
%	#3	FPS de la imagen
%	#4	Total imágenes no definido
\newcommand{\addimageanimatedanum}[4]{%
	\addimageanimatedboxed{#1}{#2}{#3}{#4}{0}{\GLOBALemptyvar}%
}

% Añade una imagen en el entorno "images" con borde sin leyenda animada
%	#1	Dirección de la imagen animada
%	#2	Parámetros de la imagen
%	#3	FPS de la imagen
%	#4	Total imágenes no definido
%	#5	Ancho de la línea (en pt)
\newcommand{\addimageanimatedanumboxed}[5]{%
	\addimageanimatedboxed{#1}{#2}{#3}{#4}{#5}{\GLOBALemptyvar}%
}

% Añade una imagen en el entorno "images" con borde y un link
% 	#1	Label (opcional)
%	#2	Dirección de la imagen
%	#3	Parámetros de la imagen
%	#4	Link de la imagen
%	#5	Leyenda de la imagen (opcional)
\newcommand{\addimagelink}[5][]{%
	\addimagelinkboxed[#1]{#2}{#3}{0}{#4}{#5}%
}

% Añade una imagen en el entorno "images" con borde
% 	#1	Label (opcional)
%	#2	Dirección de la imagen
%	#3	Parámetros de la imagen
%	#4	Ancho de la línea (en pt)
%	#5	Link de la imagen
%	#6	Leyenda de la imagen (opcional)
\newcommand{\addimagelinkboxed}[6][]{%
	\global\def\GLOBALimagelink {#5}%
	\addimageboxed[#1]{#2}{#3}{#4}{#6}%
	\global\def\GLOBALimagelink {\GLOBALemptyvar}%
}

% Añade una imagen en el entorno "images" con borde sin leyenda
%	#1	Dirección de la imagen
%	#2	Parámetros de la imagen
%	#3	Link de la imagen
\newcommand{\addimageanumlink}[3]{%
	\addimagelink{#1}{#2}{#3}{\GLOBALemptyvar}%
}

% Añade una imagen en el entorno "images" con borde sin leyenda
%	#1	Dirección de la imagen
%	#2	Parámetros de la imagen
%	#3	Ancho de la línea (en pt)
%	#4	Link de la imagen
\newcommand{\addimageanumlinkboxed}[4]{%
	\addimagelinkboxed{#1}{#2}{#3}{#4}{\GLOBALemptyvar}%
}

% Permite continuar la numeración en el entorno "images"
\newcommand{\imagescontinuenumbering}{%
	\checkonlyonenvimage%
	\global\def\GLOBALenvimagecf {true}%
}

% Agrega un espacio horizontal en el entorno "images"
% 	#1 Tamaño del espacio
\newcommand{\imageshspace}[1]{%
	\checkonlyonenvimage%
	\global\def\GLOBALenvimageadded {false}%
	\hspace{#1}%
}

% Añade un salto de línea en el entorno "images"
\newcommand{\imagesnewline}{%
	\checkonlyonenvimage%
	\global\def\GLOBALenvimageadded {false}%
	\corevspacevarcm{\marginimagemultbottom}%
	~\linebreak\noindent%
}

% Agrega un espacio vertical en el entorno "images"
% 	#1 Tamaño del espacio
\newcommand{\imagesvspace}[1]{%
	\checkonlyonenvimage%
	\global\def\GLOBALenvimageadded {false}%
	~ \\ \vspace*{#1}%
}

% Establece el margen vertical de la siguiente imagen en el entorno "images"
%	#1	Margen vertical
\newcommand{\setnextimagevmargin}[1]{%
	\checkonlyonenvimage%
	\emptyvarerr{\setimagesvmargin}{#1}{Tamaño del margen}%
	\global\def\GLOBALimagenextmarginv {#1}%
}

% Insertar una imagen
% 	#1	Label (opcional)
%	#2	Dirección de la imagen
%	#3	Parámetros de la imagen
%	#4	Leyenda de la imagen (opcional)
\newcommand{\insertimage}[4][]{%
	\insertimageboxed[#1]{#2}{#3}{0}{#4}%
}

% Insertar una imagen con recuadro
% 	#1	Label (opcional)
%	#2	Dirección de la imagen
%	#3	Parámetros de la imagen
%	#4	Ancho de la línea (en pt)
%	#5	Leyenda de la imagen (opcional)
\newcommand{\insertimageboxed}[5][]{%
	\emptyvarerr{\insertimageboxed}{#2}{Direccion de la imagen no definida}%
	\emptyvarerr{\insertimageboxed}{#3}{Parametros de la imagen no definidos}%
	\emptyvarerr{\insertimageboxed}{#4}{Ancho de la linea no definido}%
	\checkoutsideenvimage%
	\corevspacevarcm{\marginimagetop}%
	\begin{samepage}%
	\begin{figure}[H]%
		\begingroup%
			\setlength{\fboxsep}{0 pt}%
			\setlength{\fboxrule}{#4 pt}%
			\centering%
			\ifthenelse{\equal{\GLOBALimagelink}{\GLOBALemptyvar}}{% Sin link
				\fbox{\includegraphics[#3]{#2}}%
			}{% Con link
				\fbox{\href{\GLOBALimagelink}{\includegraphics[#3]{#2}}}%
			}%
		\endgroup%
		\ifx\hfuzz#5\hfuzz%
			\corevspacevarcm{\captionlessmarginimage}%
		\else%
			\hspace{0cm}%
			\corevspacevarcm{\captionmarginimage}%
			\ifthenelse{\equal{\GLOBALcaptiondefn}{\GLOBALemptyvar}}{\caption{#5 #1}}{\caption[\GLOBALcaptiondefn]{#5 #1}}%
		\fi%
	\end{figure}
	\corevspacevarcm{\marginimagebottom}%
	\end{samepage}
	\resetindexcaption%
}

% Insertar una imagen animada
% 	#1	Label (opcional)
%	#2	Dirección de la imagen animada
%	#3	Parámetros de la imagen
%	#4	FPS de la imagen
%	#5	Total imágenes no definido
%	#6	Leyenda de la imagen (opcional)
\newcommand{\insertanimatedimage}[6][]{%
	\insertanimatedimageboxed[#1]{#2}{#3}{#4}{#5}{0}{#6}%
}

% Insertar una imagen animada
% 	#1	Label (opcional)
%	#2	Dirección de la imagen animada
%	#3	Parámetros de la imagen
%	#4	FPS de la imagen
%	#5	Total imágenes no definido
%	#6	Ancho de la línea (en pt)
%	#7	Leyenda de la imagen (opcional)
\newcommand{\insertanimatedimageboxed}[7][]{%
	\emptyvarerr{\insertanimatedimage}{#2}{Direccion de la imagen no definida}%
	\emptyvarerr{\insertanimatedimage}{#3}{Parametros de la imagen no definidos}%
	\emptyvarerr{\insertanimatedimage}{#4}{FPS no definido}%
	\emptyvarerr{\insertanimatedimage}{#5}{Total imagenes no definido}%
	\emptyvarerr{\insertanimatedimage}{#6}{Ancho de la línea no definido}%
	\checkoutsideenvimage%
	\corevspacevarcm{\marginimagetop}%
	\begin{samepage}%
	\begin{figure}[H]%
		\begingroup%
			\setlength{\fboxsep}{0 pt}%
			\setlength{\fboxrule}{#6 pt}%
			\centering%
			\ifthenelse{\equal{\animatedimagecontrols}{true}}{% Muestra los controles
				\ifthenelse{\equal{\animatedimageloop}{true}}{% Con loop
					\ifthenelse{\equal{\animatedimageautoplay}{true}}{% Con autoplay
						\fbox{\animategraphics[controls,loop,autoplay,#3]{#4}{#2-}{0}{#5}}%	
					}{% Sin autoplay
						\fbox{\animategraphics[controls,loop,#3]{#4}{#2-}{0}{#5}}%
					}%
				}{% Sin loop
					\ifthenelse{\equal{\animatedimageautoplay}{true}}{% Con autoplay
						\fbox{\animategraphics[controls,autoplay,#3]{#4}{#2-}{0}{#5}}%
					}{% Sin autoplay
						\fbox{\animategraphics[controls,#3]{#4}{#2-}{0}{#5}}%
					}%
				}%
			}{% Sin controles
				\ifthenelse{\equal{\animatedimageloop}{true}}{% Con loop
					\ifthenelse{\equal{\animatedimageautoplay}{true}}{% Con autoplay
						\fbox{\animategraphics[loop,autoplay,#3]{#4}{#2-}{0}{#5}}%
					}{% Sin autoplay
						\fbox{\animategraphics[loop,#3]{#4}{#2-}{0}{#5}}%
					}%
				}{% Sin loop
					\ifthenelse{\equal{\animatedimageautoplay}{true}}{% Con autoplay
						\fbox{\animategraphics[autoplay,#3]{#4}{#2-}{0}{#5}}%
					}{% Sin autoplay
						\fbox{\animategraphics[#3]{#4}{#2-}{0}{#5}}%
					}%
				}%
			}%
		\endgroup%
		\ifx\hfuzz#7\hfuzz%
			\corevspacevarcm{\captionlessmarginimage}%
		\else%
			\hspace{0cm}%
			\corevspacevarcm{\captionmarginimage}%
			\ifthenelse{\equal{\GLOBALcaptiondefn}{\GLOBALemptyvar}}{\caption{#7 #1}}{\caption[\GLOBALcaptiondefn]{#7 #1}}%
		\fi%
	\end{figure}
	\corevspacevarcm{\marginimagebottom}%
	\end{samepage}
	\resetindexcaption%
}

% Insertar una imagen con link
% 	#1	Label (opcional)
%	#2	Dirección de la imagen
%	#3	Parámetros de la imagen
%	#4	Link de la imagen
%	#5	Leyenda de la imagen (opcional)
\newcommand{\insertimagelink}[5][]{%
	\insertimagelinkboxed[#1]{#2}{#3}{0}{#4}{#5}%
}

% Insertar una imagen con recuadro con link
% 	#1	Label (opcional)
%	#2	Dirección de la imagen
%	#3	Parámetros de la imagen
%	#4	Ancho de la línea (en pt)
%	#5	Link de la imagen
%	#6	Leyenda de la imagen (opcional)
\newcommand{\insertimagelinkboxed}[6][]{%
	\global\def\GLOBALimagelink {#5}%
	\insertimageboxed[#1]{#2}{#3}{#4}{#6}%
	\global\def\GLOBALimagelink {\GLOBALemptyvar}%
}

% Insertar una imagen completa en un entorno multicol
% 	#1	Label (opcional)
%	#2	Dirección de la imagen
%	#3	Parámetros de la imagen
%	#4	Posición, "bottom" o "top"
%	#5	Leyenda de la imagen (opcional)
\newcommand{\insertimagemc}[5][]{%
	\insertimageboxedmc[#1]{#2}{#3}{0}{#4}{#5}%
}

% Insertar una imagen completa con recuadro en un entorno multicol
% 	#1	Label (opcional)
%	#2	Dirección de la imagen
%	#3	Parámetros de la imagen
%	#4	Ancho de la línea (en pt)
%	#5	Posición, "bottom", "top", "fixed2", "fixed3", "fixed4"
%	#6	Leyenda de la imagen (opcional)
\newcommand{\insertimageboxedmc}[6][]{%
	\emptyvarerr{\insertimageboxedmc}{#2}{Direccion de la imagen no definida}%
	\emptyvarerr{\insertimageboxedmc}{#3}{Parametros de la imagen no definidos}%
	\emptyvarerr{\insertimageboxedmc}{#4}{Ancho de la linea no definido}%
	\emptyvarerr{\insertimageboxedmc}{#5}{Posicion de la imagen no definida}%
	\checkoutsideenvimage%
	\checkinsidemulticol%
	\checkoutsideappendix%
	\setcaptionmargincm{\captionlrmarginmc}%
	\ifthenelse{\equal{#5}{bottom}}{%
		\begin{samepage}%
		\begin{figure*}[!b]
	}{%
	\ifthenelse{\equal{#5}{top}}{%
		\begin{samepage}%
		\begin{figure*}[!t]
	}{%
	\ifthenelse{\equal{#5}{fixed2}}{%
		\end{multicols}
		\begin{samepage}%
		\begin{figure*}[!h]
	}{%
	\ifthenelse{\equal{#5}{fixed2b}}{%
		\end{multicols}
		\begin{samepage}%
		\begin{figure*}[!b]
	}{%
	\ifthenelse{\equal{#5}{fixed2t}}{%
		\end{multicols}
		\begin{samepage}%
		\begin{figure*}[!t]
	}{%
	\ifthenelse{\equal{#5}{fixed3}}{%
		\end{multicols}
		\begin{samepage}%
		\begin{figure*}[!h]
	}{%
	\ifthenelse{\equal{#5}{fixed3b}}{%
		\end{multicols}
		\begin{samepage}%
		\begin{figure*}[!b]
	}{%
	\ifthenelse{\equal{#5}{fixed3t}}{%
		\end{multicols}
		\begin{samepage}%
		\begin{figure*}[!t]
	}{%
	\ifthenelse{\equal{#5}{fixed4}}{%
		\end{multicols}
		\begin{samepage}%
		\begin{figure*}[!h]
	}{%
	\ifthenelse{\equal{#5}{fixed4b}}{%
		\end{multicols}
		\begin{samepage}%
		\begin{figure*}[!h]
	}{%
	\ifthenelse{\equal{#5}{fixed4t}}{%
		\end{multicols}
		\begin{samepage}%
		\begin{figure*}[!h]
	}{%
		\errmessage{LaTeX Warning: Posicion de imagen invalida, valores esperados: bottom,top,fixed2,fixed2b,fixed2t,fixed3,fixed3b,fixed3t,fixed4,fixed4b,fixed4t}
		\stop}}}}}}}}}}
	}%
		\begingroup%
			\setlength{\fboxsep}{0 pt}%
			\setlength{\fboxrule}{#4 pt}%
			\centering%
			\fbox{\includegraphics[#3]{#2}}%
		\endgroup%
		\ifx\hfuzz#6\hfuzz%
			\corevspacevarcm{\captionlessmarginimage}%
		\else%
			\hspace{0cm}%
			\corevspacevarcm{\captionmarginimage}%
			\ifthenelse{\equal{\GLOBALcaptiondefn}{\GLOBALemptyvar}}{\caption{#6 #1}}{\caption[\GLOBALcaptiondefn]{#6 #1}}%
		\fi%
	\end{figure*}
	\end{samepage}
	\ifthenelse{\equal{#5}{fixed2}}{%
		\begin{multicols}{2}%
	}{%
	\ifthenelse{\equal{#5}{fixed2b}}{%
		\begin{multicols}{2}%
	}{%
	\ifthenelse{\equal{#5}{fixed2t}}{%
		\begin{multicols}{2}%
	}{%
	\ifthenelse{\equal{#5}{fixed3}}{%
		\begin{multicols}{3}%
	}{%
	\ifthenelse{\equal{#5}{fixed3b}}{%
		\begin{multicols}{3}%
	}{%
	\ifthenelse{\equal{#5}{fixed3t}}{%
		\begin{multicols}{3}%
	}{%
	\ifthenelse{\equal{#5}{fixed4}}{%
		\begin{multicols}{4}%
	}{%
	\ifthenelse{\equal{#5}{fixed4b}}{%
		\begin{multicols}{4}%
	}{%
	\ifthenelse{\equal{#5}{fixed4t}}{%
		\begin{multicols}{4}%
	}{%
	}}}}}}}}}%
	\setcaptionmargincm{\captionlrmargin}%
	\resetindexcaption%
}

% Insertar una imagen dentro de una tabla
%	#1	Dirección de la imagen
%	#2	Parámetros de la imagen
\newcommand{\inserttableimage}[2]{%
	\inserttableimageboxed{#1}{#2}{0}%
}

% Insertar una imagen dentro de una tabla con recuadro
%	#1	Dirección de la imagen
%	#2	Parámetros de la imagen
%	#3	Ancho de la línea (en pt)
\newcommand{\inserttableimageboxed}[3]{%
	\emptyvarerr{\inserttableimageboxed}{#1}{Direccion de la imagen no definida}%
	\emptyvarerr{\inserttableimageboxed}{#2}{Parametros de la imagen no definidos}%
	\emptyvarerr{\inserttableimageboxed}{#3}{Ancho de la linea no definido}%
	\checkoutsideenvimage%
	\begingroup%
	\setlength{\fboxsep}{0 pt}%
	\setlength{\fboxrule}{#3 pt}%
	\raisebox{-1\totalheight}{\fbox{\includegraphics[#2]{#1}}}%
	\endgroup%
	\resetindexcaption%
}

% Insertar una imagen a la izquierda, escalada, ancho fijo
% 	#1	Label (opcional)
%	#2	Dirección de la imagen
%	#3	Ancho de la imagen (en linewidth)
%	#4	Leyenda de la imagen (opcional)
\newcommand{\insertimageleft}[4][]{%
	\insertimageleftboxed[#1]{#2}{#3}{0}{#4}%
}

% Insertar una imagen a la izquierda, escalada, ancho fijo
% 	#1	Label (opcional)
%	#2	Dirección de la imagen
%	#3	Ancho de la imagen (en linewidth)
%	#4	Ancho de la línea (en pt)
%	#5	Leyenda de la imagen (opcional)
\newcommand{\insertimageleftboxed}[5][]{%
	\emptyvarerr{\insertimageleftboxed}{#2}{Direccion de la imagen no definida}%
	\emptyvarerr{\insertimageleftboxed}{#3}{Ancho de la imagen no definido}%
	\emptyvarerr{\insertimageleftboxed}{#4}{Ancho de la linea no definido}%
	\checkoutsideenvimage%
	~%
	\vspace{-\baselineskip}%
	\par%
	\begin{wrapfigure}{l}{#3\linewidth}%
		\setcaptionmargincm{0}%
		\ifthenelse{\equal{\figurecaptiontop}{true}}{}{%
			\vspace{\marginfloatimages pt}%
		}%
		\begingroup%
			\setlength{\fboxsep}{0 pt}%
			\setlength{\fboxrule}{#4 pt}%
			\centering%
			\fbox{\includegraphics[width=\linewidth]{#2}}%
		\endgroup%
		\ifx\hfuzz#5\hfuzz%
			\corevspacevarcm{\captionlessmarginimage}%
		\else%
			\corevspacevarcm{\captionmarginimage}%
			\ifthenelse{\equal{\GLOBALcaptiondefn}{\GLOBALemptyvar}}{\caption{#5 #1}}{\caption[\GLOBALcaptiondefn]{#5 #1}}%
		\fi%
	\end{wrapfigure}
	\setcaptionmargincm{\captionlrmargin}%
	\resetindexcaption%
}

% Insertar una imagen a la izquierda, ajustada en un número de líneas, escalada, ancho fijo
% 	#1	Label (opcional)
%	#2	Dirección de la imagen
%	#3	Ancho de la imagen (en linewidth)
%	#4	Altura en líneas de la imagen
%	#5	Leyenda de la imagen (opcional)
\newcommand{\insertimageleftline}[5][]{%
	\insertimageleftlineboxed[#1]{#2}{#3}{0}{#4}{#5}%
}

% Insertar una imagen recuadrada a la izquierda, ajustada en un número de líneas, escalada, ancho fijo
% 	#1	Label (opcional)
%	#2	Dirección de la imagen
%	#3	Ancho de la imagen (en linewidth)
%	#4	Ancho de la línea (en pt)
%	#5	Altura en líneas de la imagen
%	#6	Leyenda de la imagen (opcional)
\newcommand{\insertimageleftlineboxed}[6][]{%
	\emptyvarerr{\insertimageleftlineboxed}{#2}{Direccion de la imagen no definida}%
	\emptyvarerr{\insertimageleftlineboxed}{#3}{Ancho de la imagen no definido}%
	\emptyvarerr{\insertimageleftlineboxed}{#4}{Ancho de la linea no definido}%
	\emptyvarerr{\insertimageleftlineboxed}{#5}{Altura en lineas de la imagen flotante izquierda no definida}
	\checkoutsideenvimage%
	~%
	\vspace{-\baselineskip}%
	\par%
	\begin{wrapfigure}[#5]{l}{#3\linewidth}%
		\setcaptionmargincm{0}%
		\ifthenelse{\equal{\figurecaptiontop}{true}}{}{%
			\vspace{\marginfloatimages pt}}%
		\begingroup%
			\setlength{\fboxsep}{0 pt}%
			\setlength{\fboxrule}{#4 pt}%
			\centering%
			\fbox{\includegraphics[width=\linewidth]{#2}}%
		\endgroup%
		\ifx\hfuzz#6\hfuzz%
			\corevspacevarcm{\captionlessmarginimage}%
		\else%
			\corevspacevarcm{\captionmarginimage}%
			\ifthenelse{\equal{\GLOBALcaptiondefn}{\GLOBALemptyvar}}{\caption{#6 #1}}{\caption[\GLOBALcaptiondefn]{#6 #1}}%
		\fi%
	\end{wrapfigure}
	\setcaptionmargincm{\captionlrmargin}%
	\resetindexcaption%
}

% Insertar una imagen a la derecha, escalada, ancho fijo
% 	#1	Label (opcional)
%	#2	Dirección de la imagen
%	#3	Ancho de la imagen (en linewidth)
%	#4	Leyenda de la imagen (opcional)
\newcommand{\insertimageright}[4][]{%
	\insertimagerightboxed[#1]{#2}{#3}{0}{#4}%
}

% Insertar una imagen recuadrada a la derecha, escalada, ancho fijo
% 	#1	Label (opcional)
%	#2	Dirección de la imagen
%	#3	Ancho de la imagen (en linewidth)
%	#4	Ancho de la línea (en pt)
%	#5	Leyenda de la imagen (opcional)
\newcommand{\insertimagerightboxed}[5][]{%
	\emptyvarerr{\insertimagerightboxed}{#2}{Direccion de la imagen no definida}%
	\emptyvarerr{\insertimagerightboxed}{#3}{Ancho de la imagen no defindo}%
	\emptyvarerr{\insertimagerightboxed}{#4}{Ancho de la linea no definido}%
	\checkoutsideenvimage%
	~%
	\vspace{-\baselineskip}%
	\par%
	\begin{wrapfigure}{r}{#3\linewidth}%
		\setcaptionmargincm{0}%
		\ifthenelse{\equal{\figurecaptiontop}{true}}{}{%
			\vspace{\marginfloatimages pt}%
		}%
		\begingroup%
			\setlength{\fboxsep}{0 pt}%
			\setlength{\fboxrule}{#4 pt}%
			\centering%
			\fbox{\includegraphics[width=\linewidth]{#2}}%
		\endgroup%
		\ifx\hfuzz#5\hfuzz%
			\corevspacevarcm{\captionlessmarginimage}%
		\else%
			\corevspacevarcm{\captionmarginimage}%
			\ifthenelse{\equal{\GLOBALcaptiondefn}{\GLOBALemptyvar}}{\caption{#5 #1}}{\caption[\GLOBALcaptiondefn]{#5 #1}}%
		\fi%
	\end{wrapfigure}
	\setcaptionmargincm{\captionlrmargin}%
	\resetindexcaption%
}

% Insertar una imagen a la derecha, ajustada en un número de líneas, escalada, ancho fijo
% 	#1	Label (opcional)
%	#2	Dirección de la imagen
%	#3	Ancho de la imagen (en linewidth)
%	#4	Altura en líneas de la imagen
%	#5	Leyenda de la imagen (opcional)
\newcommand{\insertimagerightline}[5][]{%
	\insertimagerightlineboxed[#1]{#2}{#3}{0}{#4}{#5}%
}

% Insertar una imagen recuadrada a la derecha, ajustada en un número de líneas, escalada, ancho fijo
% 	#1	Label (opcional)
%	#2	Dirección de la imagen
%	#3	Ancho de la imagen (en linewidth)
%	#4	Ancho de la línea (en pt)
%	#5	Altura en líneas de la imagen
%	#6	Leyenda de la imagen (opcional)
\newcommand{\insertimagerightlineboxed}[6][]{%
	\emptyvarerr{\insertimagerightlineboxed}{#2}{Direccion de la imagen no definida}%
	\emptyvarerr{\insertimagerightlineboxed}{#3}{Ancho de la imagen no defindo}%
	\emptyvarerr{\insertimagerightlineboxed}{#4}{Ancho de la linea no definido}%
	\emptyvarerr{\insertimagerightlineboxed}{#5}{Altura en lineas de la imagen flotante derecha no definida}%
	\checkoutsideenvimage%
	~%
	\vspace{-\baselineskip}%
	\par%
	\begin{wrapfigure}[#5]{r}{#3\linewidth}%
		\setcaptionmargincm{0}%
		\ifthenelse{\equal{\figurecaptiontop}{true}}{}{%
			\vspace{\marginfloatimages pt}%
		}%
		\begingroup%
			\setlength{\fboxsep}{0 pt}%
			\setlength{\fboxrule}{#4 pt}%
			\centering%
			\fbox{\includegraphics[width=\linewidth]{#2}}%
		\endgroup%
		\ifx\hfuzz#6\hfuzz%
			\corevspacevarcm{\captionlessmarginimage}%
		\else%
			\corevspacevarcm{\captionmarginimage}%
			\ifthenelse{\equal{\GLOBALcaptiondefn}{\GLOBALemptyvar}}{\caption{#6 #1}}{\caption[\GLOBALcaptiondefn]{#6 #1}}%
		\fi%
	\end{wrapfigure}
	\setcaptionmargincm{\captionlrmargin}%
	\resetindexcaption%
}

% Insertar una imagen a la izquierda, propiedades variables
% 	#1	Label (opcional)
%	#2	Dirección de la imagen
%	#3	Ancho del objeto
%	#4	Propiedades de la imagen
%	#5	Leyenda de la imagen (opcional)
\newcommand{\insertimageleftp}[5][]{%
	\xspace ~ \\%
	\vspace{-2\baselineskip}%
	\par%
	\insertimageleftboxedp[#1]{#2}{#3}{#4}{0}{#5}%
}

% Insertar una imagen a la izquierda, propiedades variables
% 	#1	Label (opcional)
%	#2	Dirección de la imagen
%	#3	Ancho del objeto
%	#4	Propiedades de la imagen
%	#5	Ancho de la línea (en pt)
%	#6	Leyenda de la imagen (opcional)
\newcommand{\insertimageleftboxedp}[6][]{%
	\emptyvarerr{\insertimageleftboxedp}{#2}{Direccion de la imagen no definida}%
	\emptyvarerr{\insertimageleftboxedp}{#3}{Ancho del objeto no definido}%
	\emptyvarerr{\insertimageleftboxedp}{#4}{Propiedades de la imagen no defindos}%
	\emptyvarerr{\insertimageleftboxedp}{#5}{Ancho de la linea no definido}%
	\checkoutsideenvimage%
	~%
	\vspace{-\baselineskip}%
	\par%
	\begin{wrapfigure}{l}{#3}%
		\setcaptionmargincm{0}%
		\ifthenelse{\equal{\figurecaptiontop}{true}}{}{%
			\vspace{\marginfloatimages pt}%
		}%
		\begingroup%
			\setlength{\fboxsep}{0 pt}%
			\setlength{\fboxrule}{#5 pt}%
			\centering%
			\fbox{\includegraphics[#4]{#2}}%
		\endgroup%
		\ifx\hfuzz#6\hfuzz%
			\corevspacevarcm{\captionlessmarginimage}%
		\else%
			\corevspacevarcm{\captionmarginimage}%
			\ifthenelse{\equal{\GLOBALcaptiondefn}{\GLOBALemptyvar}}{\caption{#6 #1}}{\caption[\GLOBALcaptiondefn]{#6 #1}}%
		\fi%
	\end{wrapfigure}
	\setcaptionmargincm{\captionlrmargin}%
	\resetindexcaption%
}

% Insertar una imagen a la izquierda, ajustada en un número de líneas, propiedades variables
% 	#1	Label (opcional)
%	#2	Dirección de la imagen
%	#3	Ancho del objeto
%	#4	Propiedades de la imagen
%	#5	Altura en líneas de la imagen
%	#6	Leyenda de la imagen (opcional)
\newcommand{\insertimageleftlinep}[6][]{%
	\insertimageleftlineboxedp[#1]{#2}{#3}{#4}{0}{#5}{#6}%
}

% Insertar una imagen recuadrada a la izquierda, ajustada en un número de líneas, propiedades variables
% 	#1	Label (opcional)
%	#2	Dirección de la imagen
%	#3	Ancho del objeto
%	#4	Propiedades de la imagen
%	#5	Ancho de la línea (en pt)
%	#6	Altura en líneas de la imagen
%	#7	Leyenda de la imagen (opcional)
\newcommand{\insertimageleftlineboxedp}[7][]{%
	\emptyvarerr{\insertimageleftlineboxedp}{#2}{Direccion de la imagen no definida}%
	\emptyvarerr{\insertimageleftlineboxedp}{#3}{Ancho del objeto no definido}%
	\emptyvarerr{\insertimageleftlineboxedp}{#4}{Propiedades de la imagen no definidos}%
	\emptyvarerr{\insertimageleftlineboxedp}{#5}{Ancho de la linea no definido}%
	\emptyvarerr{\insertimageleftlineboxedp}{#6}{Altura en lineas de la imagen flotante izquierda no definida}%
	\checkoutsideenvimage%
	~%
	\vspace{-\baselineskip}%
	\par%
	\begin{wrapfigure}[#6]{l}{#3}%
		\setcaptionmargincm{0}%
		\ifthenelse{\equal{\figurecaptiontop}{true}}{}{%
			\vspace{\marginfloatimages pt}%
		}%
		\begingroup%
			\setlength{\fboxsep}{0 pt}%
			\setlength{\fboxrule}{#5 pt}%
			\centering%
			\fbox{\includegraphics[#4]{#2}}%
		\endgroup%
		\ifx\hfuzz#7\hfuzz%
			\corevspacevarcm{\captionlessmarginimage}%
		\else%
			\corevspacevarcm{\captionmarginimage}%
			\ifthenelse{\equal{\GLOBALcaptiondefn}{\GLOBALemptyvar}}{\caption{#7 #1}}{\caption[\GLOBALcaptiondefn]{#7 #1}}%
		\fi%
	\end{wrapfigure}
	\setcaptionmargincm{\captionlrmargin}%
	\resetindexcaption%
}

% Insertar una imagen a la derecha, propiedades variables
% 	#1	Label (opcional)
%	#2	Dirección de la imagen
%	#3	Ancho del objeto (en cm)
%	#4	Propiedades de la imagen
%	#5	Leyenda de la imagen (opcional)
\newcommand{\insertimagerightp}[5][]{%
	\xspace ~ \\%
	\vspace{-2\baselineskip}%
	\par%
	\insertimagerightboxedp[#1]{#2}{#3}{#4}{0}{#5}%
}

% Insertar una imagen recuadrada a la derecha, propiedades variables
% 	#1	Label (opcional)
%	#2	Dirección de la imagen
%	#3	Ancho del objeto
%	#4	Propiedades de la imagen
%	#5	Ancho de la línea (en pt)
%	#6	Leyenda de la imagen (opcional)
\newcommand{\insertimagerightboxedp}[6][]{%
	\emptyvarerr{\insertimagerightboxedp}{#2}{Direccion de la imagen no definida}%
	\emptyvarerr{\insertimagerightboxedp}{#3}{Ancho del objeto no definido}%
	\emptyvarerr{\insertimagerightboxedp}{#4}{Propiedades de la imagen no definidos}%
	\emptyvarerr{\insertimagerightboxedp}{#5}{Ancho de la linea no definido}%
	\checkoutsideenvimage%
	~%
	\vspace{-\baselineskip}%
	\par%
	\begin{wrapfigure}{r}{#3}%
		\setcaptionmargincm{0}%
		\ifthenelse{\equal{\figurecaptiontop}{true}}{}{%
			\vspace{\marginfloatimages pt}%
		}%
		\begingroup%
			\setlength{\fboxsep}{0 pt}%
			\setlength{\fboxrule}{#5 pt}%
			\centering%
			\fbox{\includegraphics[#4]{#2}}%
		\endgroup%
		\ifx\hfuzz#6\hfuzz%
			\corevspacevarcm{\captionlessmarginimage}%
		\else%
			\corevspacevarcm{\captionmarginimage}%
			\ifthenelse{\equal{\GLOBALcaptiondefn}{\GLOBALemptyvar}}{\caption{#6 #1}}{\caption[\GLOBALcaptiondefn]{#6 #1}}%
		\fi%
	\end{wrapfigure}
	\setcaptionmargincm{\captionlrmargin}%
	\resetindexcaption%
}

% Insertar una imagen a la derecha, ajustada en un número de líneas, propiedades variables
% 	#1	Label (opcional)
%	#2	Dirección de la imagen
%	#3	Ancho del objeto (en cm)
%	#4	Propiedades de la imagen
%	#5	Altura en líneas de la imagen
%	#6	Leyenda de la imagen (opcional)
\newcommand{\insertimagerightlinep}[6][]{%
	\insertimagerightlineboxedp[#1]{#2}{#3}{#4}{0}{#5}{#6}%
}

% Insertar una imagen recuadrada a la derecha, ajustada en un número de líneas, propiedades variables
% 	#1	Label (opcional)
%	#2	Dirección de la imagen
%	#3	Ancho del objeto
%	#4	Propiedades de la imagen
%	#5	Ancho de la línea (en pt)
%	#6	Altura en líneas de la imagen
%	#7	Leyenda de la imagen (opcional)
\newcommand{\insertimagerightlineboxedp}[7][]{%
	\emptyvarerr{\insertimagerightlineboxedp}{#2}{Direccion de la imagen no definida}%
	\emptyvarerr{\insertimagerightlineboxedp}{#3}{Ancho del objeto no definido}%
	\emptyvarerr{\insertimagerightlineboxedp}{#4}{Propiedades de la imagen no definidos}%
	\emptyvarerr{\insertimagerightlineboxedp}{#5}{Ancho de la linea no definido}%
	\emptyvarerr{\insertimagerightlineboxedp}{#6}{Altura en lineas de la imagen flotante derecha no definida}%
	\checkoutsideenvimage%
	~%
	\vspace{-\baselineskip}%
	\par%
	\begin{wrapfigure}[#6]{r}{#3}%
		\setcaptionmargincm{0}%
		\ifthenelse{\equal{\figurecaptiontop}{true}}{}{%
			\vspace{\marginfloatimages pt}%
		}%
		\begingroup%
			\setlength{\fboxsep}{0 pt}%
			\setlength{\fboxrule}{#5 pt}%
			\centering%
			\fbox{\includegraphics[#4]{#2}}%
		\endgroup%
		\ifx\hfuzz#7\hfuzz%
			\corevspacevarcm{\captionlessmarginimage}%
		\else%
			\corevspacevarcm{\captionmarginimage}%
			\ifthenelse{\equal{\GLOBALcaptiondefn}{\GLOBALemptyvar}}{\caption{#7 #1}}{\caption[\GLOBALcaptiondefn]{#7 #1}}%
		\fi%
	\end{wrapfigure}
	\setcaptionmargincm{\captionlrmargin}%
	\resetindexcaption%
}

% Inserta una imagen con parametros keyvals almacenados en una variable
% 	#1	Parámetros (keyvals)
%	#2	Dirección de la imagen
\newcommand{\coreinsertkeyimage}[2]{%
	\expandafter\includegraphics\expandafter[#1]{\expandafter#2}%
}

% Define la clave resolution al insertar imágenes
\makeatletter
\define@key{Gin}{resolution}{\pdfimageresolution=#1\relax}
\makeatother

\global\def\GLOBALtitlerequirechapter {false}
\global\def\GLOBALtitleinitchapter {false}
\global\def\GLOBALtitleinitsection {false}
\global\def\GLOBALtitleinitsubsection {false}
\global\def\GLOBALtitleinitsubsubsection {false}
\global\def\GLOBALtitleinitsubsubsubsection {false}

% Configura textos a añadir antes de secciones
\global\def\GLOBALtitleprechapterstr {}
\global\def\GLOBALtitlepresectionstr {}
\global\def\GLOBALtitlepresubsectionstr {}
\global\def\GLOBALtitlepresubsubsectionstr {}
\global\def\GLOBALtitlepresubsubsubsectionstr {}

% Configura que entornos pueden funcionar
\global\def\GLOBALtitlechapterenabled {true}

% Activa la numeración en las secciones
\def\coreintializetitlenumbering {%
	% Capítulo
	\renewcommand{\thechapter}{\GLOBALformatnumchapter{chapter}}%
	% Section
	\ifthenelse{\equal{\GLOBALchapternumenabled}{false}}{%
		\renewcommand{\thesection}{%
			\GLOBALformatnumsection{section}%
		}%
	}{%
		\renewcommand{\thesection}{%
			\thechapter\charbetwchaptersection\GLOBALformatnumsection{section}%
		}%
	}%
	% Subsection
	\ifthenelse{\equal{\GLOBALsectionanumenabled}{true}}{%
		\renewcommand{\thesubsection}{%
			\GLOBALformatnumssection{subsection}%
		}%
	}{%
		\renewcommand{\thesubsection}{%
			\thesection\charbetwsectionsubsection\GLOBALformatnumssection{subsection}%
		}%
	}%
	% Subsubsection
	\ifthenelse{\equal{\GLOBALsubsectionanumenabled}{true}}{%
		\renewcommand{\thesubsubsection}{%
			\GLOBALformatnumsssection{subsubsection}%
		}%
	}{%
		\renewcommand{\thesubsubsection}{%
			\thesubsection\charbetwsubsectionssect\GLOBALformatnumsssection{subsubsection}%
		}%
	}%
	% Subsubsubsection
	\ifthenelse{\equal{\GLOBALsubsubsectionanumenabled}{true}}{%
		\renewcommand{\thesubsubsubsection}{%
			\GLOBALformatnumssssection{subsubsubsection}%
		}%
	}{%
		\renewcommand{\thesubsubsubsection}{%
			\thesubsubsection\charbetwssectionsssect\GLOBALformatnumssssection{subsubsubsection}%
		}%
	}%
	\hbadness=10000%
}

% Chequea si los capítulos están activados
\def\corecheckchapterenabled {%
	\ifthenelse{\equal{\GLOBALtitlechapterenabled}{false}}{% Verifica que el entorno esté activo
		\throwwarning{La insercion de capitulos esta desactivada}%
	}{}%
}

% Chequea si los capítulos han sido iniciados
\def\corecheckchapterinitialized {%
	\ifthenelse{\equal{\GLOBALtitlerequirechapter}{true}}{%
		\ifthenelse{\equal{\GLOBALtitleinitchapter}{false}}{%
			\throwwarning{Se requiere un nuevo capitulo}%
		}{}%
	}{}%
}

% Chequea si una sección han sido iniciada
\def\corechecksectioninitialized {%
	\ifthenelse{\equal{\GLOBALtitleinitsection}{false}}{%
		\throwwarning{Se requiere una nueva seccion}%
	}{}%
}

% Chequea si una subsección han sido iniciada
\def\corechecksubsectioninitialized {%
	\ifthenelse{\equal{\GLOBALtitleinitsubsection}{false}}{%
		\throwwarning{Se requiere una nueva subseccion}%
	}{}%
}

% Chequea si una subsubsección han sido iniciada
\def\corechecksubsubsectioninitialized {%
	\ifthenelse{\equal{\GLOBALtitleinitsubsubsection}{false}}{%
		\throwwarning{Se requiere una nueva subsubseccion}%
	}{}%
}

% Parcha el formato de capítulos
\pretocmd{\chapter}{%
	\corecheckchapterenabled%
	\ifthenelse{\equal{\showsectioncaptioncode}{chap}}{% Reinicia código fuente
		\addtocounter{templateListings}{\value{lstlisting}}%
		\setcounter{lstlisting}{0}%
	}{}%
	\ifthenelse{\equal{\showsectioncaptioneqn}{chap}}{% Reinicia ecuaciones
		\addtocounter{templateEquations}{\value{equation}}%
		\setcounter{equation}{0}%
	}{}%
	\ifthenelse{\equal{\equationrestart}{chap}}{% Reinicia ecuaciones
		\addtocounter{templateEquations}{\value{equation}}%
		\setcounter{equation}{0}%
	}{}%
	\ifthenelse{\equal{\showsectioncaptionfig}{chap}}{% Reinicia figuras
		\addtocounter{templateFigures}{\value{figure}}%
		\setcounter{figure}{0}%
	}{}%
	\ifthenelse{\equal{\showsectioncaptiontab}{chap}}{%Reinicia tablas
		\addtocounter{templateTables}{\value{table}}%
		\setcounter{table}{0}%
	}{}%
	\global\def\GLOBALchapternumenabled {true}%
	\global\def\GLOBALsectionanumenabled {false}%
	\global\def\GLOBALsubsectionanumenabled {false}%
	\global\def\GLOBALsubsubsectionanumenabled {false}%
	\global\def\GLOBALtitleinitchapter {true}%
	\global\def\GLOBALtitleinitsection {false}%
	\global\def\GLOBALtitleinitsubsection {false}%
	\global\def\GLOBALtitleinitsubsubsection {false}%
	\global\def\GLOBALtitleinitsubsubsubsection {false}%
	\coreintializetitlenumbering%
}{}{}

% Parcha el formato de secciones al pasar desde una anum, vuelve a activar número
% de la sección
\pretocmd{\section}{%
	\ifthenelse{\equal{\showsectioncaptioncode}{sec}}{% Reinicia código fuente
		\addtocounter{templateListings}{\value{lstlisting}}%
		\setcounter{lstlisting}{0}%
	}{}%
	\ifthenelse{\equal{\showsectioncaptioneqn}{sec}}{% Reinicia ecuaciones
		\addtocounter{templateEquations}{\value{equation}}%
		\setcounter{equation}{0}%
	}{}%
	\ifthenelse{\equal{\equationrestart}{sec}}{% Reinicia ecuaciones
		\addtocounter{templateEquations}{\value{equation}}%
		\setcounter{equation}{0}%
	}{}%
	\ifthenelse{\equal{\showsectioncaptionfig}{sec}}{% Reinicia figuras
		\addtocounter{templateFigures}{\value{figure}}%
		\setcounter{figure}{0}%
	}{}%
	\ifthenelse{\equal{\showsectioncaptiontab}{sec}}{% Reinicia tablas
		\addtocounter{templateTables}{\value{table}}%
		\setcounter{table}{0}%
	}{}%
	\global\def\GLOBALsectionanumenabled {false}%
	\global\def\GLOBALsubsectionanumenabled {false}%
	\global\def\GLOBALsubsubsectionanumenabled {false}%
	\global\def\GLOBALtitleinitsection {true}%
	\global\def\GLOBALtitleinitsubsection {false}%
	\global\def\GLOBALtitleinitsubsubsection {false}%
	\global\def\GLOBALtitleinitsubsubsubsection {false}%
	\corecheckchapterinitialized%
	\coreintializetitlenumbering%
}{}{}

% Comienza nueva subsección, si está dentro de una sectionanum entonces no dibuja el
% número de sección, si no entonces dibuja el número de forma normal
\pretocmd{\subsection}{%
	\ifthenelse{\equal{\showsectioncaptioncode}{ssec}}{% Reinicia código fuente
		\addtocounter{templateListings}{\value{lstlisting}}%
		\setcounter{lstlisting}{0}%
	}{}%
	\ifthenelse{\equal{\showsectioncaptioneqn}{ssec}}{% Reinicia ecuaciones
		\addtocounter{templateEquations}{\value{equation}}%
		\setcounter{equation}{0}%
	}{}%
	\ifthenelse{\equal{\equationrestart}{ssec}}{% Reinicia ecuaciones
		\addtocounter{templateEquations}{\value{equation}}%
		\setcounter{equation}{0}%
	}{}%
	\ifthenelse{\equal{\showsectioncaptionfig}{ssec}}{% Reinicia figuras
		\addtocounter{templateFigures}{\value{figure}}%
		\setcounter{figure}{0}%
	}{}%
	\ifthenelse{\equal{\showsectioncaptiontab}{ssec}}{% Reinicia tablas
		\addtocounter{templateTables}{\value{table}}%
		\setcounter{table}{0}%
	}{}%
	\global\def\GLOBALsubsectionanumenabled {false}%
	\global\def\GLOBALsubsubsectionanumenabled {false}%
	\global\def\GLOBALtitleinitsubsection {true}%
	\global\def\GLOBALtitleinitsubsubsection {false}%
	\global\def\GLOBALtitleinitsubsubsubsection {false}%
	\corecheckchapterinitialized%
	\corechecksectioninitialized%
	\coreintializetitlenumbering%
}{}{}

% Comienza nueva subsubsección, aquí hay varios casos:
%	- si está dentro de una subsección sin número ignora la sección
%	- si no, entonces puede estar dentro de una sección sin número o no, en ese caso
%	  debe evaluar ambas posibilidades
\pretocmd{\subsubsection}{%
	\ifthenelse{\equal{\showsectioncaptioncode}{sssec}}{% Reinicia código fuente
		\addtocounter{templateListings}{\value{lstlisting}}%
		\setcounter{lstlisting}{0}%
	}{}%
	\ifthenelse{\equal{\showsectioncaptioneqn}{sssec}}{% Reinicia ecuaciones
		\addtocounter{templateEquations}{\value{equation}}%
		\setcounter{equation}{0}%
	}{}%
	\ifthenelse{\equal{\equationrestart}{sssec}}{% Reinicia ecuaciones
		\addtocounter{templateEquations}{\value{equation}}%
		\setcounter{equation}{0}%
	}{}%
	\ifthenelse{\equal{\showsectioncaptionfig}{sssec}}{% Reinicia figuras
		\addtocounter{templateFigures}{\value{figure}}%
		\setcounter{figure}{0}%
	}{}%
	\ifthenelse{\equal{\showsectioncaptiontab}{sssec}}{% Reinicia tablas
		\addtocounter{templateTables}{\value{table}}%
		\setcounter{table}{0}%
	}{}%
	\global\def\GLOBALsubsubsectionanumenabled {false}%
	\global\def\GLOBALtitleinitsubsubsection {true}%
	\global\def\GLOBALtitleinitsubsubsubsection {false}%
	\corecheckchapterinitialized%
	\corechecksectioninitialized%
	\corechecksubsectioninitialized%
	\coreintializetitlenumbering%
}{}{}

% Parcha sub-sub-subsecciones
\def\corepatchaftersubsubsubsection {%
	\ifthenelse{\equal{\showsectioncaptioncode}{ssssec}}{% Reinicia código fuente
		\addtocounter{templateListings}{\value{lstlisting}}%
		\setcounter{lstlisting}{0}%
	}{}%
	\ifthenelse{\equal{\showsectioncaptioneqn}{ssssec}}{% Reinicia ecuaciones
		\addtocounter{templateEquations}{\value{equation}}%
		\setcounter{equation}{0}%
	}{}%
	\ifthenelse{\equal{\equationrestart}{ssssec}}{% Reinicia ecuaciones
		\addtocounter{templateEquations}{\value{equation}}%
		\setcounter{equation}{0}%
	}{}%
	\ifthenelse{\equal{\showsectioncaptionfig}{ssssec}}{% Reinicia figuras
		\addtocounter{templateFigures}{\value{figure}}%
		\setcounter{figure}{0}%
	}{}%
	\ifthenelse{\equal{\showsectioncaptiontab}{ssssec}}{% Reinicia tablas
		\addtocounter{templateTables}{\value{table}}%
		\setcounter{table}{0}%
	}{}%
	\global\def\GLOBALtitleinitsubsubsubsection {true}%
	\corecheckchapterinitialized%
	\corechecksectioninitialized%
	\corechecksubsectioninitialized%
	\corechecksubsubsectioninitialized%
}

% Entorno que permite desactivar los capítulos
\makeatletter
\newcommand*\coredisabledchapter{%
	\@ifstar{\coredisabledchapterstar}{\@dblarg\coredisabledchapternostar}}
\newcommand*\coredisabledchapterstar[1]{%
	\noindent\textcolor{red}{Error (chapter):} \newline#1%
	\throwwarning{La insercion de capitulos esta desactivada}%
}
\def\coredisabledchapternostar[#1]#2{%
	\noindent\textcolor{red}{Error (chapter):} #1%
	\throwwarning{La insercion de capitulos esta desactivada}%
}
\makeatother
\let\oldchapter\chapter

% Desactiva los capítulos
\newcommand{\disablechapter}{%
	\let\chapter\coredisabledchapter%
	\global\def\GLOBALtitlechapterenabled {false}%
}

% Activa los capítulos
\newcommand{\enablechapter}{%
	\let\chapter\oldchapter%
	\global\def\GLOBALtitlechapterenabled {true}%
}

% Insertar un capítulo sin número
% 	#1	Título
\newcommand{\chapteranum}[1]{%
	\corecheckchapterenabled%
	\emptyvarerr{\chapteranum}{#1}{Titulo no definido}%
	\phantomsection%
	\needspace{3\baselineskip}%
	\chapter*{#1}%
	\addcontentsline{toc}{chapter}{#1}%
	\ifthenelse{\equal{\anumsecaddtocounter}{true}}{\stepcounter{chapter}}{}%
	\changeheadertitle{#1}%
	\setcounter{section}{0}%
	\global\def\GLOBALchapternumenabled {false}%
	\coreintializetitlenumbering%
}

% Insertar un título sin número
% 	#1	Título
\newcommand{\sectionanum}[1]{%
	\emptyvarerr{\sectionanum}{#1}{Titulo no definido}%
	\phantomsection%
	\needspace{3\baselineskip}%
	\section*{#1}%
	\addcontentsline{toc}{section}{#1}%
	\ifthenelse{\equal{\anumsecaddtocounter}{true}}{\stepcounter{section}}{}%
	\changeheadertitle{#1}%
	\setcounter{subsection}{0}%
	\global\def\GLOBALsectionanumenabled {true}%
	\coreintializetitlenumbering%
}

% Insertar un título sin número y sin indexar
% 	#1	Título
\newcommand{\sectionanumnoi}[1]{%
	\emptyvarerr{\sectionanumnoi}{#1}{Titulo no definido}%
	\phantomsection%
	\needspace{3\baselineskip}%
	\section*{#1}%
	\ifthenelse{\equal{\anumsecaddtocounter}{true}}{\stepcounter{section}}{}%
	\changeheadertitle{#1}%
	\setcounter{subsection}{0}%
	\global\def\GLOBALsectionanumenabled {true}%
	\coreintializetitlenumbering%
}

% Insertar un título sin número sin cambiar el título del header
% 	#1	Título
\newcommand{\sectionanumheadless}[1]{%
	\emptyvarerr{\sectionanumnoheadless}{#1}{Titulo no definido}%
	\section*{#1}%
	\addcontentsline{toc}{section}{#1}%
	\ifthenelse{\equal{\anumsecaddtocounter}{true}}{\stepcounter{section}}{}%
	\setcounter{subsection}{0}%
	\global\def\GLOBALsectionanumenabled {true}%
	\coreintializetitlenumbering%
}

% Insertar un título sin número, sin indexar y sin cambiar el título del header
% 	#1	Título
\newcommand{\sectionanumnoiheadless}[1]{%
	\emptyvarerr{\sectionanumnoiheadless}{#1}{Titulo no definido}%
	\section*{#1}%
	\ifthenelse{\equal{\anumsecaddtocounter}{true}}{\stepcounter{section}}{}%
	\setcounter{subsection}{0}%
	\global\def\GLOBALsectionanumenabled {true}%
	\coreintializetitlenumbering%
}

% Insertar un subtítulo sin número
% 	#1	Subtítulo
\newcommand{\subsectionanum}[1]{%
	\emptyvarerr{\subsectionanum}{#1}{Subtitulo no definido}%
	\subsection*{#1}%
	\addcontentsline{toc}{subsection}{#1}
	\ifthenelse{\equal{\anumsecaddtocounter}{true}}{\stepcounter{subsection}}{}%
	\setcounter{subsubsection}{0}%
	\global\def\GLOBALsubsectionanumenabled {true}%
	\coreintializetitlenumbering%
}

% Insertar un subtítulo sin número y sin indexar
% 	#1	Subtítulo
\newcommand{\subsectionanumnoi}[1]{%
	\emptyvarerr{\subsectionanumnoi}{#1}{Subtitulo no definido}%
	\subsection*{#1}%
	\ifthenelse{\equal{\anumsecaddtocounter}{true}}{\stepcounter{subsection}}{}%
	\setcounter{subsubsection}{0}%
	\global\def\GLOBALsubsectionanumenabled {true}%
	\coreintializetitlenumbering%
}

% Insertar un sub-subtítulo sin número
% 	#1	Sub-subtítulo
\newcommand{\subsubsectionanum}[1]{%
	\emptyvarerr{\subsubsectionanum}{#1}{Sub-subtitulo no definido}%
	\subsubsection*{#1}%
	\addcontentsline{toc}{subsubsection}{#1}%
	\ifthenelse{\equal{\anumsecaddtocounter}{true}}{\stepcounter{subsubsection}}{}%
	\setcounter{subsubsubsection}{0}%
	\global\def\GLOBALsubsubsectionanumenabled {true}%
	\coreintializetitlenumbering%
}

% Insertar un sub-subtítulo sin número y sin indexar
% 	#1	Sub-subtítulo
\newcommand{\subsubsectionanumnoi}[1]{%
	\emptyvarerr{\subsubsectionanumnoi}{#1}{Sub-subtitulo no definido}%
	\subsubsection*{#1}%
	\ifthenelse{\equal{\anumsecaddtocounter}{true}}{\stepcounter{subsubsection}}{}%
	\setcounter{subsubsubsection}{0}%
	\global\def\GLOBALsubsubsectionanumenabled {true}%
	\coreintializetitlenumbering%
}

% Insertar un sub-sub-subtítulo sin número
% 	#1	Sub-sub-subtítulo
\newcommand{\subsubsubsectionanum}[1]{%
	\emptyvarerr{\subsubsubsectionanum}{#1}{Sub-sub-subtitulo no definido}%
	\subsubsubsection*{#1}%
	\addcontentsline{toc}{subsubsubsection}{#1}%
	\ifthenelse{\equal{\anumsecaddtocounter}{true}}{\stepcounter{subsubsubsection}}{}%
}

% Insertar un sub-sub-subtítulo sin número y sin indexar
% 	#1	Sub-sub-subtítulo
\newcommand{\subsubsubsectionanumnoi}[1]{%
	\emptyvarerr{\subsubsubsectionanumnoi}{#1}{Sub-sub-subtitulo no definido}%
	\subsubsection*{#1}%
	\ifthenelse{\equal{\anumsecaddtocounter}{true}}{\stepcounter{subsubsubsection}}{}%
}

% Cambia el título del encabezado (header)
%	#1	Título
\newcommand{\changeheadertitle}[1]{%
	\emptyvarerr{\changeheadertitle}{#1}{Titulo no definido}%
	\markboth{#1}{}%
}

% Elimina el título del encabezado (header)
\newcommand{\clearheadertitle}{%
	\markboth{}{}%
}

% Insertar un título en un índice, sin número de página
%	#1	Margen superior en pt. (opcional)
%	#2	Título
\newcommand{\insertindextitle}[2][]{%
	\emptyvarerr{\insertindextitle}{#2}{Titulo no definido}%
	\ifx\hfuzz#1\hfuzz%
		\addtocontents{toc}{\protect\addvspace{\indextitlemargin pt}}%
	\else%
		\addtocontents{toc}{\protect\addvspace{#1 pt}}%
	\fi%
	\addtocontents{toc}{\noindent\hyperref[swpn]{\textbf{#2}}}%
}

% Insertar un título en un índice, con número de página
%	#1	Margen superior en pt. (opcional)
%	#2	Título
\newcommand{\insertindextitlepage}[2][]{%
	\emptyvarerr{\insertindextitlepage}{#2}{Titulo no definido}%
	\ifx\hfuzz#1\hfuzz%
		\addtocontents{toc}{\protect\addvspace{\indextitlemargin pt}}%
	\else%
		\addtocontents{toc}{\protect\addvspace{#1 pt}}%
	\fi%
	\addcontentsline{toc}{section}{#2}%
}

% Crea una sección en el índice y en el header
%	#1	Margen superior en pt. (opcional)
%	#2	Título
\newcommand{\createhiddensection}[2][]{%
	\changeheadertitle{#2}%
	\insertindextitlepage[#1]{#2}%
}

% Crear un capítulo como una sección
%	#1	Título
\newcommand{\newchapter}[1]{%
	\emptyvarerr{\newchapter}{#1}{Titulo no definido}%
	\clearpage%
	\stepcounter{section}%
	\phantomsection%
	\needspace{3\baselineskip}%
	\vspace* {3cm}%
	\noindent {\huge{\textbf{\namechapter\ \thesection}}} \\%
	\vspace* {0.5cm} \\%
	\noindent {\Huge{\textbf{#1}}} \\%
	\vspace {0.5cm} \\%
	\addcontentsline{toc}{section}{\protect\numberline{\thesection}#1}%
	\markboth{#1}{}%
}

% Insertar párrafo
\newcommand{\newp}{%
	\hbadness=10000 \vspace{\baselinestretch\baselineskip} \par%
}

% Crea un salto de columna en el entorno multicol
\ifthenelse{\isundefined{\newcolumn}}{%
	\newcommand{\newcolumn}{%
		\checkinsidemulticol\vfill\null\columnbreak%
	}
}{%
	\renewcommand{\newcolumn}{%
		\checkinsidemulticol\vfill\null\columnbreak%
	}
}

% Salto de página en entorno multicol
\newcommand{\newpagemulticol}{%
	\newcolumn\newcolumn\clearpage%
}

% Redimensiona un ítem
% 	#1	Tamaño del nuevo objeto (En linewidth)
%	#2	Objeto a redimensionar
\newcommand{\itemresize}[2]{%
	\emptyvarerr{\itemresize}{#1}{Tamano del nuevo objeto no definido}%
	\emptyvarerr{\itemresize}{#2}{Objeto a redimensionar no definido}%
	\resizebox{#1\linewidth}{!}{#2}%
}

% Crea una página vacía sin header o footer
\newcommand{\insertemptypage}{%
	\clearpage%
	\thispagestyle{empty}%
	\null%
	\clearpage%
}

% Inserta una página vacía, aunque conserva header, footer y numeración
\newcommand{\insertblankpage}{%
	\clearpage%
	\null%
	\clearpage%
}

% Función personalizada \cleardoublepage
\def\corecleardoublepage {%
	\clearpage %
	\ifthenelse{\equal{\GLOBALtwoside}{true}}{%
		\ifodd\thepage %
		\else%
			\emptypagespredocformat%
		\fi%
	}{}%
}

% Ejecuta una función dependiendo si la página es par o impar
%	#1	Par
%	#2	Impar
\newcommand{\coretriggeronpage}[2]{%
	\ifthenelse{\isodd{\value{templatePageCounter}}}{%
		#2%
	}{%
		#1%
	}%
}

% Añade un archivo pdf con el header
%	#1	Parámetros (opcional)
%	#2	Nombre del archivo pdf
\newcommand{\includehfpdf}[2][]{%
	\includepdf[pagecommand={\pagestyle{fancy}},#1]{#2}%
}

% Añade un archivo pdf con el header
%	#1	Parámetros (opcional)
%	#2	Nombre del archivo pdf
\newcommand{\includefullhfpdf}[2][]{%
	\includepdf[pages=-,pagecommand={\pagestyle{fancy}},#1]{#2}%
}

% Inserta un texto entre comillas
%	#1 	Texto
\newcommand{\quotes}[1]{%
	\enquote*{#1}%
}

% Inserta un texto entre comillas y negrita
%	#1 	Texto
\newcommand{\quotesbf}[1]{%
	\quotes{\textbf{#1}}%
}

% Inserta un texto entre comillas e itálico
%	#1 	Texto
\newcommand{\quotesit}[1]{%
	\quotes{\textit{#1}}%
}

% Inserta un texto entre comillas y typewriter
%	#1 	Texto
\newcommand{\quotesttt}[1]{%
	\quotes{\texttt{#1}}%
}

% Inserta un texto entre comillas dobles
%	#1 	Texto
\newcommand{\doublequotes}[1]{%
	\enquote{#1}%
}

% Inserta una cita con texto elevado
%	#1	Cita
\newcommand{\scite}[1]{%
	\textsuperscript{\cite{#1}}%
}

% Fuerza la indentación
\newcommand{\forceindent}{%
	~ \\ %
	
	\vspace{-2\baselineskip}%
}

% Inserta un texto con el formato de enlace
% 	#1 	Enlace
\newcommand{\hreftext}[1]{%
	\ifthenelse{\equal{\fonturl}{same}}{%
		#1%
	}{%
	\ifthenelse{\equal{\fonturl}{tt}}{%
		\texttt{#1}%
	}{%
	\ifthenelse{\equal{\fonturl}{rm}}{%
		\textrm{#1}%
	}{%
	\ifthenelse{\equal{\fonturl}{sf}}{%
		\textsf{#1}%
	}{}}}}%
}

% Inserta un email con un link cliqueable
%	#1 	Dirección email
\newcommand{\insertemail}[1]{%
	\href{mailto:#1}{\hreftext{#1}}%
}

% Inserta un teléfono celular
%	#1	Teléfono celular
\newcommand{\insertphone}[1]{%
	\href{tel:#1}{\hreftext{#1}}%
}

% Reinicia el número de ecuaciones
\newcommand{\restartequation}{%
	\setcounter{equation}{0}%
}

% Desactiva el margen de las leyendas
\newcommand{\disablecaptionmargin}{%
	\setcaptionmargincm{0}%
}

% Reinicia el margen de las leyendas
\newcommand{\resetcaptionmargin}{%
	\setcaptionmargincm{\captionlrmargin}%
}

% Modifica el color de las tablas
%	#1	Posición inicial del inicio de colores
\newcommand{\settablerowcolors}[1]{%
	\emptyvarerr{\settablerowcolors}{#1}{Posicion de fila no definida}%
	\ifthenelse{\equal{\GLOBALtablerowcolorswitch}{false}}{% Usa colores normales
		\ifthenelse{\equal{\tablerowfirstcolor}{none}}{%
			\ifthenelse{\equal{\tablerowsecondcolor}{none}}{%
				\rowcolors{#1}{}{}%
			}{%
				\rowcolors{#1}{\tablerowsecondcolor}{}%
			}%
		}{%
			\ifthenelse{\equal{\tablerowsecondcolor}{none}}{%
				\rowcolors{#1}{}{\tablerowfirstcolor}%
			}{%
				\rowcolors{#1}{\tablerowsecondcolor}{\tablerowfirstcolor}%
			}%
		}%
	}{% Usa colores alternados
		\ifthenelse{\equal{\tablerowfirstcolor}{none}}{%
			\ifthenelse{\equal{\tablerowsecondcolor}{none}}{%
				\rowcolors{#1}{}{}%
			}{%
				\rowcolors{#1}{}{\tablerowsecondcolor}%
			}%
		}{%
			\ifthenelse{\equal{\tablerowsecondcolor}{none}}{%
				\rowcolors{#1}{\tablerowfirstcolor}{}%
			}{%
				\rowcolors{#1}{\tablerowfirstcolor}{\tablerowsecondcolor}%
			}%
		}%
	}%
	
	% Actualiza el índice previo
	\global\def\GLOBALtablerowcolorindex {#1}%
}

% Alterna los colores de las tablas a la última ejecución
\newcommand{\settablerowcolorslast}{%
	\ifthenelse{\equal{\GLOBALtablerowcolorswitch}{false}}{% Usa colores normales
		\ifthenelse{\equal{\tablerowfirstcolor}{none}}{%
			\ifthenelse{\equal{\tablerowsecondcolor}{none}}{%
				\rowcolors{\GLOBALtablerowcolorindex}{}{}%
			}{%
				\rowcolors{\GLOBALtablerowcolorindex}{\tablerowsecondcolor}{}%
			}%
		}{%
			\ifthenelse{\equal{\tablerowsecondcolor}{none}}{%
				\rowcolors{\GLOBALtablerowcolorindex}{}{\tablerowfirstcolor}%
			}{%
				\rowcolors{\GLOBALtablerowcolorindex}{\tablerowsecondcolor}{\tablerowfirstcolor}%
			}%
		}%
	}{% Usa colores alternados
		\ifthenelse{\equal{\tablerowfirstcolor}{none}}{%
			\ifthenelse{\equal{\tablerowsecondcolor}{none}}{%
				\rowcolors{\GLOBALtablerowcolorindex}{}{}%
			}{%
				\rowcolors{\GLOBALtablerowcolorindex}{}{\tablerowsecondcolor}%
			}%
		}{%
			\ifthenelse{\equal{\tablerowsecondcolor}{none}}{%
				\rowcolors{\GLOBALtablerowcolorindex}{\tablerowfirstcolor}{}%
			}{%
				\rowcolors{\GLOBALtablerowcolorindex}{\tablerowfirstcolor}{\tablerowsecondcolor}%
			}%
		}%
	}%
}

% Activa el color de las filas de las tablas
%	#1	Posición inicial del inicio de colores
\newcommand{\enabletablerowcolor}[1][]{%
	\ifx\hfuzz#1\hfuzz%
		\settablerowcolors{2}%
	\else%
		\settablerowcolors{#1}%
	\fi%
}

% Desactiva el color de las filas de las tablas
\newcommand{\disabletablerowcolor}{\rowcolors{2}{}{}}

% Alterna los colores de las filas de las tablas
\newcommand{\switchtablerowcolors}{%
	\ifthenelse{\equal{\GLOBALtablerowcolorswitch}{false}}{%
		\global\def\GLOBALtablerowcolorswitch {true}%
	}{%
		\global\def\GLOBALtablerowcolorswitch {false}%
	}%
	\settablerowcolorslast%
}

% Actualiza el padding de las celdas de las tablas
%	#1	Padding horizontal (em)
%	#2	Padding vertical (em)
\newcommand{\settablecellpadding}[2]{%
	\emptyvarerr{\settablecellpadding}{#1}{Padding horizontal no definido}%
	\emptyvarerr{\settablecellpadding}{#2}{Padding vertical no definido}%
	\setlength{\tabcolsep}{#1 em} % Horizontal
	\def\arraystretch {#2} % Vertical
}

% Resetea el padding de las celdas de las tablas
\newcommand{\resettablecellpadding}{%
	\settablecellpadding{\tablepaddingh}{\tablepaddingv}%
}

% Cambia el tamaño de la página
%	#1	Orientacion de la página, puede ser 0 o 90. Por defecto es cero
%	#2	Ancho de la página (cm)
%	#3	Alto de la página (cm)
\newcommand{\changepagesize}[3][]{%
	% \emptyvarerr{\changepagesize}{#1}{Orientacion de la pagina}
	\emptyvarerr{\changepagesize}{#2}{Ancho de la pagina no definida}%
	\emptyvarerr{\changepagesize}{#3}{Altura de la pagina no definida}%
	\ifthenelse{\equal{\compilertype}{lualatex}}{%
		\throwwarning{Funcion no valida en compilador lualatex}%
	}{%
		\clearpage%
		\ifthenelse{\equal{#1}{}}{%
			\newgeometry{left=\pagemarginleft cm, top=\pagemargintop cm, right=\pagemarginright cm, bottom=\pagemarginbottom cm, paperwidth=#2 cm, paperheight=#3 cm}%
		}{%
		\ifthenelse{\equal{#1}{0}}{%
			\newgeometry{left=\pagemarginleft cm, top=\pagemargintop cm, right=\pagemarginright cm, bottom=\pagemarginbottom cm, paperwidth=#2 cm, paperheight=#3 cm}%
		}{%
		\ifthenelse{\equal{#1}{90}}{%
			\newgeometry{left=\pagemarginleft cm, top=\pagemargintop cm, right=\pagemarginright cm, bottom=\pagemarginbottom cm, paperwidth=#3 cm, paperheight=#2 cm}%
		}{%
			\throwbadconfig{Orientacion de pagina no valido}{\changepagesize}{0,90}}}%
		}%
	}%
}

% Ofrece diferentes formatos de pagina
% https://www.prepressure.com/library/paper-size
%	#1	Indica la rotación, puede ser 0 o 90
%	#2	Formato de la pagina
\newcommand{\changepagesizeformat}[2][]{%
	\emptyvarerr{\changepagesizeformat}{#2}{Formato de pagina no definido}%
	\ifthenelse{\equal{#2}{4A0}}{%
		\changepagesize[#1]{168.2}{237.8}%
	}{%
	\ifthenelse{\equal{#2}{2A0}}{%
		\changepagesize[#1]{118.9}{168.2}%
	}{%
	\ifthenelse{\equal{#2}{A0}}{%
		\changepagesize[#1]{84.1}{118.9}%
	}{%
	\ifthenelse{\equal{#2}{A1}}{%
		\changepagesize[#1]{59.4}{84.1}%
	}{%
	\ifthenelse{\equal{#2}{A2}}{%
		\changepagesize[#1]{42.0}{84.1}%
	}{%
	\ifthenelse{\equal{#2}{A3}}{%
		\changepagesize[#1]{29.7}{42.0}%
	}{%
	\ifthenelse{\equal{#2}{A4}}{%
		\changepagesize[#1]{21.0}{29.7}%
	}{%
	\ifthenelse{\equal{#2}{A5}}{%
		\changepagesize[#1]{14.8}{21.0}%
	}{%
	\ifthenelse{\equal{#2}{A6}}{%
		\changepagesize[#1]{10.5}{14.8}%
	}{%
	\ifthenelse{\equal{#2}{letter}}{%
		\changepagesize[#1]{21.59}{27.94}%
	}{%
	\ifthenelse{\equal{#2}{legal}}{%
		\changepagesize[#1]{21.59}{35.6}%
	}{%
	\ifthenelse{\equal{#2}{foolscap}}{%
		\changepagesize[#1]{20.3}{33.0}%
	}{%
	\ifthenelse{\equal{#2}{executive}}{%
		\changepagesize[#1]{18.41}{26.67}%
	}{%
	\ifthenelse{\equal{#2}{ledger}}{%
		\changepagesize[#1]{27.94}{43.18}%
	}{%
	\ifthenelse{\equal{#2}{tabloid}}{%
		\changepagesize[#1]{43.18}{27.94}%
	}{%
	\ifthenelse{\equal{#2}{ANSIC}}{%
		\changepagesize[#1]{55.9}{43.2}%
	}{%
	\ifthenelse{\equal{#2}{ANSID}}{%
		\changepagesize[#1]{86.4}{55.9}%
	}{%
	\ifthenelse{\equal{#2}{ANSIE}}{%
		\changepagesize[#1]{111.8}{86.4}%
	}{%
	\ifthenelse{\equal{#2}{B0}}{%
		\changepagesize[#1]{100}{141.4}%
	}{%
	\ifthenelse{\equal{#2}{B1}}{%
		\changepagesize[#1]{70.7}{100}%
	}{%
	\ifthenelse{\equal{#2}{B2}}{%
		\changepagesize[#1]{50}{70.7}%
	}{%
	\ifthenelse{\equal{#2}{B3}}{%
		\changepagesize[#1]{35.3}{50}%
	}{%
	\ifthenelse{\equal{#2}{B4}}{%
		\changepagesize[#1]{25}{35.3}%
	}{%
	\ifthenelse{\equal{#2}{B5}}{%
		\changepagesize[#1]{17.6}{25}%
	}{%
	\ifthenelse{\equal{#2}{B6}}{%
		\changepagesize[#1]{12.5}{17.6}%
	}{%
		\throwbadconfig{Estilo de pagina no valido}{\changepagesizeformat}{4A0,2A0,A0,A1,A2,A3,A4,A5,A6,letter,legal,foolscap,executive,ledger,tabloid,ANSIC,ANSID,ANSIE,B0,B1,B2,B3,B4,B5,B6}}}}}}}}}}}}}}}}}}}}}}}}}%
	}%
}

% Crea variables para guardar configuraciones de columnas
\global\def\GLOBALtwocolumnap {l}
\global\def\GLOBALtwocolumnav {t}
\global\def\GLOBALtwocolumnbp {l}
\global\def\GLOBALtwocolumnbv {t}
\global\def\GLOBALthreecolumnap {l}
\global\def\GLOBALthreecolumnav {t}
\global\def\GLOBALthreecolumnbp {l}
\global\def\GLOBALthreecolumnbv {t}
\global\def\GLOBALthreecolumncp {l}
\global\def\GLOBALthreecolumncv {t}

% Chequea posición columna
%	#1	Valor posición (c, t, b)
\newcommand{\corecheckcolumnvvalue}[1]{%
	\ifthenelse{\equal{#1}{c}}{}{%
	\ifthenelse{\equal{#1}{t}}{}{%
	\ifthenelse{\equal{#1}{b}}{}{%
		\errmessage{LaTeX Warning: Posicion vertical columna invalido, valores esperados: c,t,b}%
	}}}%
}

% Chequea alineación columna
%	#1	Valor alineación (c, l, r)
\newcommand{\corecheckcolumnpvalue}[1]{%
	\ifthenelse{\equal{#1}{c}}{}{%
	\ifthenelse{\equal{#1}{l}}{}{%
	\ifthenelse{\equal{#1}{r}}{}{%
		\errmessage{LaTeX Warning: Alineacion columna invalida, valores esperados: c,l,r}%
	}}}%
}

% Configura las columnas dobles
%	#1	Posición vertical columna izquierda (c, t, b)
%	#2	Alineación horizontal columna izquierda (c, l, r)
%	#3	Posición vertical columna derecha (c, t, b)
%	#4	Alineación horizontal columna derecha (c, l, r)
\newcommand{\createtwocolumncfg}[4]{%
	\corecheckcolumnvvalue{#1}%
	\corecheckcolumnpvalue{#2}%
	\corecheckcolumnvvalue{#3}%
	\corecheckcolumnpvalue{#4}%
	\global\def\GLOBALtwocolumnav {#1}%
	\global\def\GLOBALtwocolumnap {#2}%
	\global\def\GLOBALtwocolumnbv {#3}%
	\global\def\GLOBALtwocolumnbp {#4}%
}

% Restaura la configuración de dos columnas
\newcommand{\resettwocolumncfg}{%
	\createtwocolumncfg{t}{l}{t}{l}%
}

% Configura las columnas triples
%	#1	Posición vertical columna izquierda (c, t, b)
%	#2	Alineación horizontal columna izquierda (c, l, r)
%	#3	Posición vertical columna central (c, t, b)
%	#4	Alineación horizontal columna central (c, l, r)
%	#5	Posición vertical columna derecha (c, t, b)
%	#6	Alineación horizontal columna derecha (c, l, r)
\newcommand{\createthreecolumncfg}[6]{%
	\corecheckcolumnvvalue{#1}%
	\corecheckcolumnpvalue{#2}%
	\corecheckcolumnvvalue{#3}%
	\corecheckcolumnpvalue{#4}%
	\corecheckcolumnvvalue{#5}%
	\corecheckcolumnpvalue{#6}%
	\global\def\GLOBALthreecolumnav {#1}%
	\global\def\GLOBALthreecolumnap {#2}%
	\global\def\GLOBALthreecolumnbv {#3}%
	\global\def\GLOBALthreecolumnbp {#4}%
	\global\def\GLOBALthreecolumncv {#3}%
	\global\def\GLOBALthreecolumncp {#4}%
}

% Restaura la configuración de tres columnas
\newcommand{\resetthreecolumncfg}{%
	\createthreecolumncfg{t}{l}{t}{l}{t}{l}%
}

% Crea dos columnas con contenido
%	#1	Altura de las columnas (opcional)
%	#2 	Dimensión de la columna izquierda (En linewidth)
%	#3	Dimensión de la columna derecha (En linewidth)
%	#4	Distancia entre columnas (En cm)
%	#5 	Contenido de la columna izquierda
%	#6	Contenido de la columna derecha
\newcommand{\createtwocolumn}[6][]{%
	\setcaptionmargincm{0}%
	\begin{samepage}%
	\begin{flushleft}%
		\vspace{-0.5\baselineskip}%
		\begin{minipage}{1\linewidth}%
			\begin{minipage}[t][#1][\GLOBALtwocolumnav]{#2\linewidth}%
				\ifthenelse{\equal{\GLOBALtwocolumnap}{c}}{%
					\begin{center}#5\end{center}%
				}{%
				\ifthenelse{\equal{\GLOBALtwocolumnap}{l}}{%
					\begin{raggedright}#5\end{raggedright}%
				}{%
				\ifthenelse{\equal{\GLOBALtwocolumnap}{r}}{%
					\hfill\begin{raggedleft}#5\end{raggedleft}%
				}{%
					\errmessage{LaTeX Warning: Alineacion columna izquierda incorrecta, valores esperados: c,l,r}%
				}}}%
			\end{minipage}%
			\hspace{#4 cm}%
			\begin{minipage}[t][#1][\GLOBALtwocolumnbv]{#3\linewidth}%
				\ifthenelse{\equal{\GLOBALtwocolumnbp}{c}}{%
					\begin{center}#6\end{center}%
				}{%
				\ifthenelse{\equal{\GLOBALtwocolumnbp}{l}}{%
					\begin{raggedright}#6\end{raggedright}%
				}{%
				\ifthenelse{\equal{\GLOBALtwocolumnbp}{r}}{%
					\hfill\begin{raggedleft}#6\end{raggedleft}%
				}{%
					\errmessage{LaTeX Warning: Alineacion columna derecha incorrecta, valores esperados: c,l,r}%
				}}}%
			\end{minipage}
		\end{minipage}
	\end{flushleft}
	~ \vspace{-0.5\baselineskip}%
	\end{samepage}
	\setcaptionmargincm{\captionlrmargin}%
}

% Crea dos columnas idénticas
%	#1	Altura de las columnas (opcional)
%	#2 	Contenido de la columna izquierda
%	#3	Contenido de la columna derecha
\newcommand{\createhalfcolumn}[3][]{%
	\createtwocolumn[#1]{0.5}{0.5}{0}{#2}{#3}%
}

% Crea tres columnas con contenido
%	#1	Altura de las columnas (opcional)
%	#2 	Dimensión de la columna izquierda (En linewidth)
%	#3	Dimensión de la columna central (En linewidth)
%	#4	Dimensión de la columna derecha (En linewidth)
%	#5	Distancia entre columna 1-2 (En cm)
%	#6	Distancia entre columna 2-3 (En cm)
%	#7 	Contenido de la columna izquierda
%	#8	Contenido de la columna central
%	#9	Contenido de la columna derecha
\newcommand{\createthreecolumn}[9][]{%
	\setcaptionmargincm{0}%
	\begin{samepage}%
	\begin{flushleft}%
		\vspace{-0.5\baselineskip}%
		\begin{minipage}{1\linewidth}%
			\begin{minipage}[t][#1][\GLOBALthreecolumnav]{#2\linewidth}%
				\ifthenelse{\equal{\GLOBALthreecolumnap}{c}}{%
					\begin{center}#7\end{center}%
				}{%
				\ifthenelse{\equal{\GLOBALthreecolumnap}{l}}{%
					\begin{raggedright}#7\end{raggedright}%
				}{%
				\ifthenelse{\equal{\GLOBALthreecolumnap}{r}}{%
					\hfill\begin{raggedleft}#7\end{raggedleft}%
				}{%
					\errmessage{LaTeX Warning: Alineacion columna izquierda incorrecta, valores esperados: c,l,r}%
				}}}%
			\end{minipage}
			\hspace{#5 cm}%
			\begin{minipage}[t][#1][\GLOBALthreecolumnbv]{#3\linewidth}%
				\ifthenelse{\equal{\GLOBALthreecolumnbp}{c}}{%
					\begin{center}#8\end{center}%
				}{%
				\ifthenelse{\equal{\GLOBALthreecolumnbp}{l}}{%
					\begin{raggedright}#8\end{raggedright}%
				}{%
				\ifthenelse{\equal{\GLOBALthreecolumnbp}{r}}{%
					\hfill\begin{raggedleft}#8\end{raggedleft}%
				}{%
					\errmessage{LaTeX Warning: Alineacion columna central incorrecta, valores esperados: c,l,r}%
				}}}%
			\end{minipage}
			\hspace{#6 cm}%
			\begin{minipage}[t][#1][\GLOBALthreecolumncv]{#4\linewidth}%
				\ifthenelse{\equal{\GLOBALthreecolumncp}{c}}{%
					\begin{center}#9\end{center}%
				}{%
				\ifthenelse{\equal{\GLOBALthreecolumncp}{l}}{%
					\begin{raggedright}#9\end{raggedright}%
				}{%
				\ifthenelse{\equal{\GLOBALthreecolumncp}{r}}{%
					\hfill\begin{raggedleft}#9\end{raggedleft}%
				}{%
					\errmessage{LaTeX Warning: Alineacion columna derecha incorrecta, valores esperados: c,l,r}%
				}}}%
			\end{minipage}
		\end{minipage}
	\end{flushleft}
	~ \vspace{-0.5\baselineskip}%
	\end{samepage}
	\setcaptionmargincm{\captionlrmargin}%
}

% Crea tres columnas idénticas
%	#1 	Contenido de la columna izquierda
%	#2	Contenido de la columna central
%	#3	Contenido de la columna derecha
\newcommand{\createthirdcolumn}[3]{%
	\createthreecolumn{0.3333}{0.3333}{0.3333}{0}{0}{#1}{#2}{#3}%
}

% Crea una sección de referencias solo para bibtex
\newenvironment{references}{%
	\ifthenelse{\equal{\stylecitereferences}{bibtex}}{% Verifica configuraciones
	}{%
		\throwerror{\references}{Solo se puede usar entorno references con estilo citas \noexpand\stylecitereferences=bibtex}%
	}%
	\phantomsection%
	\addcontentsline{toc}{chapter}{\namereferences}%
	\begin{thebibliography}{} % Inicia la bibliografía
		\ifthenelse{\equal{\bibtextextalign}{justify}}{% Formato ajuste de línea
		}{%
		\ifthenelse{\equal{\bibtextextalign}{left}}{%
			\raggedright%
		}{%
		\ifthenelse{\equal{\bibtextextalign}{right}}{%
			\raggedleft%
		}{%
		\ifthenelse{\equal{\bibtextextalign}{center}}{%
			\centering%
		}{%
			\throwbadconfig{Ajuste de linea referencias bibtex desconocido}{\bibtextextalign}{justified,left,right,center}}}}%
		}%
	}%
	{%
	\end{thebibliography}
}

% Crea un entorno para definir el tamaño de bloque
%	#1	Tamaño de fuente en pt
\newenvironment{fontsizeblock}[1][\documentfontsize]{%
	\changefontsizes{#1 pt}%
}{%
	\changefontsizes{\documentfontsize pt}%
}

% Crea una sección de anexos
\newenvironment{appendixd}{%
	\appendix%
	\global\def\GLOBALenvappendix {true}%
	\global\def\GLOBALtitlerequirechapter {true}%
	\begingroup%
	\phantomsection%
	\changeheadertitle{\nameltappendixsection} % Cambia el nombre del header
	% Define formato números para appendix
	\global\def\GLOBALformatnumchapter {\formatnumapchapter}%
	\global\def\GLOBALformatnumsection {\formatnumapsection}%
	\global\def\GLOBALformatnumssection {\formatnumapssection}%
	\global\def\GLOBALformatnumsssection {\formatnumapsssection}%
	\global\def\GLOBALformatnumssssection {\formatnumapssssection}%
	% Define estado de numeración
	\global\def\GLOBALtitleinitchapter {false}%
	\global\def\GLOBALtitleinitsection {false}%
	\global\def\GLOBALtitleinitsubsection {false}%
	\global\def\GLOBALtitleinitsubsubsection {false}%
	\global\def\GLOBALtitleinitsubsubsubsection {false}%
	\bookmarksetup{%
		numbered={true},
		openlevel={\thetemplateBookmarksLevelPrev}
	}%
	\appendixtitleon%
	\appendixtitletocon%
	\bookmarksetupnext{level=part}%
	\begin{appendices} % Crea la sección
		\ifthenelse{\equal{\showappendixsecindex}{true}}{}{%
			\pdfbookmark{\nameappendixsection}{appendix} % Si false
		}%
		% \setcounter{secnumdepth}{4}
		% \setcounter{tocdepth}{4}
		\ifthenelse{\equal{\appendixindepobjnum}{true}}{%
			\counterwithin{equation}{chapter}
			\counterwithin{figure}{chapter}
			\counterwithin{lstlisting}{chapter}
			\counterwithin{table}{chapter}
		}{}%
	}{%
	\end{appendices}
	% Restablece formato de números
	\global\def\GLOBALformatnumchapter {\formatnumchapter}%
	\global\def\GLOBALformatnumsection {\formatnumsection}%
	\global\def\GLOBALformatnumssection {\formatnumssection}%
	\global\def\GLOBALformatnumsssection {\formatnumsssection}%
	\global\def\GLOBALformatnumssssection {\formatnumssssection}%
	% Reestablece estado de numeración
	\global\def\GLOBALtitleinitchapter {false}%
	\global\def\GLOBALtitleinitsection {false}%
	\global\def\GLOBALtitleinitsubsection {false}%
	\global\def\GLOBALtitleinitsubsubsection {false}%
	\global\def\GLOBALtitleinitsubsubsubsection {false}%
	\bookmarksetupnext{level={\thetemplateBookmarksLevelPrev}} % Restablece índice marcador
	\bookmarksetup{%
		numbered={\cfgpdfsecnumbookmarks},
		openlevel={\cfgbookmarksopenlevel}
	}%
	\endgroup%
	\global\def\GLOBALenvappendix {false}%
	\global\def\GLOBALtitlerequirechapter {true}%
}

% Entorno simple de apéndices
\newenvironment{appendixs}{%
	\appendix%
	\global\def\GLOBALenvappendix {true}%
	\global\def\GLOBALtitlerequirechapter {false}%
	\begingroup%
	\chapteranum{\nameappendixsection}%
	% Define etiqueta secciones
	\global\def\GLOBALtitlepresectionstr {\nameltappendixsection~}%
	\changeheadertitle{\nameltappendixsection} % Cambia el nombre del header
	% Define formato números para appendix
	\global\def\GLOBALformatnumchapter {\formatnumapchapter}%
	\global\def\GLOBALformatnumsection {\formatnumapchapter}%
	\global\def\GLOBALformatnumssection {\formatnumapsection}%
	\global\def\GLOBALformatnumsssection {\formatnumapssection}%
	\global\def\GLOBALformatnumssssection {\formatnumapsssection}%
	% Define estado de numeración
	\global\def\GLOBALtitleinitchapter {false}%
	\global\def\GLOBALtitleinitsection {false}%
	\global\def\GLOBALtitleinitsubsection {false}%
	\global\def\GLOBALtitleinitsubsubsection {false}%
	\global\def\GLOBALtitleinitsubsubsubsection {false}%
	% Otras configuraciones
	\disablechapter%
	\ifthenelse{\equal{\appendixindepobjnum}{true}}{%
		\counterwithin{equation}{section}
		\counterwithin{figure}{section}
		\counterwithin{lstlisting}{section}
		\counterwithin{table}{section}
	}{}%
	}{%
	% Restablece formato de números
	\global\def\GLOBALformatnumchapter {\formatnumchapter}%
	\global\def\GLOBALformatnumsection {\formatnumsection}%
	\global\def\GLOBALformatnumssection {\formatnumssection}%
	\global\def\GLOBALformatnumsssection {\formatnumsssection}%
	\global\def\GLOBALformatnumssssection {\formatnumssssection}%
	% Reestablece estado de numeración
	\global\def\GLOBALtitleinitchapter {false}%
	\global\def\GLOBALtitleinitsection {false}%
	\global\def\GLOBALtitleinitsubsection {false}%
	\global\def\GLOBALtitleinitsubsubsection {false}%
	\global\def\GLOBALtitleinitsubsubsubsection {false}%
	% Resetea etiqueta secciones
	\global\def\GLOBALtitlepresectionstr {}%
	\enablechapter%
	\endgroup%
	\global\def\GLOBALenvappendix {false}%
}

% Entorno capítulos apéndices con título
\newenvironment{appendixdtitle}[1][style1]{%
	\chapter*{\nameappendixsection}%
	\let\clearpage\relax%
	\vspace{-1.75cm}%
	% Configura el tipo de capítulo
	\ifthenelse{\equal{#1}{style1}}{% Default
	}{%
	\ifthenelse{\equal{#1}{style2}}{%
		\titleformat{\chapter}[hang]{\huge\bfseries}{\thechapter.\hspace{20pt}}{0pt}{\huge\bfseries}%
	}{%
	\ifthenelse{\equal{#1}{style3}}{%
		\titleformat{\chapter}[hang]{\huge\bfseries}{\nameltappendixsection\ \thechapter.\hspace{20pt}}{0pt}{\huge\bfseries}%
	}{%
	\ifthenelse{\equal{#1}{style4}}{%
		\titleformat{\chapter}[hang]{\LARGE\bfseries}{\nameltappendixsection\ \thechapter.\hspace{20pt}}{0pt}{\LARGE\bfseries}%
	}{%
		\throwerror{appendixdtitle}{Estilo capitulo apendice incorrecto. Estilos validos style1..style4}%
	}{}}}}%
	\begin{appendixd}%
		}{%
	\end{appendixd}%
}

% Inicia código fuente con parámetros
%	#1	Label (opcional)
%	#2	Estilo de código
%	#3	Parámetros
%	#4	Caption
\newcommand{\coreinitsourcecodep}[4]{%
	\emptyvarerr{\coreinitsourcecodep}{#2}{Estilo de codigo no definido}%
	\checkvalidsourcecodestyle{#2}%
	\ifthenelse{\equal{\showlinenumbers}{true}}{%
		\rightlinenumbers}{%
	}%
	\lstset{%
		backgroundcolor=\color{\sourcecodebgcolor}
	}%
	\ifthenelse{\equal{\codecaptiontop}{true}}{%
		\ifx\hfuzz#4\hfuzz%
			\ifx\hfuzz#3\hfuzz%
				\lstset{%
					escapeinside={(*@}{@*)},
					style=#2
				}%
			\else%
				\lstset{%
					escapeinside={(*@}{@*)},
					style=#2,
					#3
				}%
			\fi%
		\else%
			\ifx\hfuzz#3\hfuzz%
				\lstset{%
					caption={#4 #1},
					captionpos=t,
					escapeinside={(*@}{@*)},
					style=#2
				}%
			\else%
				\lstset{%
					caption={#4 #1},
					captionpos=t,
					escapeinside={(*@}{@*)},
					style=#2,
					#3
				}%
			\fi%
		\fi%
	}{%
		\ifx\hfuzz#4\hfuzz%
			\ifx\hfuzz#3\hfuzz%
				\lstset{%
					escapeinside={(*@}{@*)},
					style=#2
				}%
			\else%
				\lstset{%
					escapeinside={(*@}{@*)},
					style=#2,
					#3
				}%
			\fi%
		\else%
			\ifx\hfuzz#3\hfuzz%
				\lstset{%
					caption={#4 #1},
					captionpos=b,
					style=#2
				}%
			\else%
				\lstset{%
					caption={#4 #1},
					captionpos=b,
					escapeinside={(*@}{@*)},
					style=#2,
					#3
				}%
			\fi%
		\fi%
	}%
}

% Inserta código fuente con parámetros
%	#1	Label (opcional)
%	#2	Estilo de código
%	#3	Parámetros
%	#4	Caption
\lstnewenvironment{sourcecodep}[4][]{%
	\coreinitsourcecodep{#1}{#2}{#3}{#4}%
}{%
	\ifthenelse{\equal{\showlinenumbers}{true}}{%
		\leftlinenumbers}{%
	}%
}

% Importa código fuente desde un archivo con parámetros
%	#1	Label (opcional)
%	#2	Estilo de código
%	#3	Parámetros
%	#4	Archivo de código fuente
%	#5	Caption
\newcommand{\importsourcecodep}[5][]{%
	\coreinitsourcecodep{#1}{#2}{#3}{#5}%
	\inputlisting{#4}%
	\ifthenelse{\equal{\showlinenumbers}{true}}{%
		\leftlinenumbers}{%
	}%
}

% Inicia código fuente sin parámetros
%	#1	Label (opcional)
%	#2	Estilo de código
%	#3	Caption
\newcommand{\coreinitsourcecode}[3]{%
	\emptyvarerr{\coreinitsourcecode}{#2}{Estilo de codigo no definido}%
	\checkvalidsourcecodestyle{#2}%
	\ifthenelse{\equal{\showlinenumbers}{true}}{%
		\rightlinenumbers}{%
	}%
	\lstset{%
		backgroundcolor=\color{\sourcecodebgcolor}
	}%
	\ifthenelse{\equal{\codecaptiontop}{true}}{%
		\ifx\hfuzz#3\hfuzz%
			\lstset{%
				escapeinside={(*@}{@*)},
				style=#2
			}%
		\else%
			\lstset{%
				escapeinside={(*@}{@*)},
				caption={#3 #1},
				captionpos=t,
				style=#2
			}%
		\fi%
	}{%
		\ifx\hfuzz#3\hfuzz%
			\lstset{%
				escapeinside={(*@}{@*)},
				style=#2
			}%
		\else%
			\lstset{%
				escapeinside={(*@}{@*)},
				caption={#3 #1},
				captionpos=b,
				style=#2
			}%
		\fi%
	}%
}

% Inserta código fuente sin parámetros
%	#1	Label (opcional)
%	#2	Estilo de código
%	#3	Caption
\lstnewenvironment{sourcecode}[3][]{%
	\coreinitsourcecode{#1}{#2}{#3}%
}{%
	\ifthenelse{\equal{\showlinenumbers}{true}}{%
		\leftlinenumbers}{%
	}%
}

% Importa código fuente desde un archivo sin parámetros
%	#1	Label (opcional)
%	#2	Estilo de código
%	#3	Archivo de código fuente
%	#4	Caption
\newcommand{\importsourcecode}[4][]{%
	\coreinitsourcecode{#1}{#2}{#4}%
	\lstinputlisting{#3}%
	\ifthenelse{\equal{\showlinenumbers}{true}}{%
		\leftlinenumbers}{%
	}%
}

% Itemize en negrita
%	#1	Parámetros opcionales
\newenvironment{itemizebf}[1][]{%
	\begin{itemize}[font=\bfseries,#1]%
	}{%
	\end{itemize}
}

% Enumerate en negrita
%	#1	Parámetros opcionales
\newenvironment{enumeratebf}[1][]{%
	\begin{enumerate}[font=\bfseries,#1]%
	}{%
	\end{enumerate}
}

% Crea una sección de resumen
%	#1	Tabla resumen
%	#2	Título de la tesis
%	#3	Título de la sección
%	#4	Etiqueta del marcador del pdf
\newenvironment{abstractenv}[4]{%
	\clearpage%
	\ifthenelse{\equal{\GLOBALtwoside}{true}}{%
		\coretriggeronpage{\emptypagespredocformat}{}%
	}{}%
	\emptyvarerr{\abstractenv}{#1}{Tabla resumen no definida}%
	\emptyvarerr{\abstractenv}{#2}{Titulo tesis no definido}%
	\emptyvarerr{\abstractenv}{#3}{Titulo seccion no definida}%
	\emptyvarerr{\abstractenv}{#4}{Etiqueta marcador del pdf}%
	% Añade a los marcadores
	\ifthenelse{\equal{\addabstracttobookmarks}{true}}{%
		\phantomsection%
		\pdfbookmark{#3}{#4}}{%
	}%
	% Inserta la tabla resumen
	\ifthenelse{\equal{#1}{}}{%
		\vspace*{0\baselineskip}%
	}{%
		\begin{flushright}%
			\small%
			#1%
		\end{flushright}%
		\vspace*{0.05\baselineskip}%
	}%
	% Título
	\begin{center}%
		\textcolor{\sectioncolor}{\MakeUppercase{\textbf{#2}}}%
	\end{center} \newp%
	\ifthenelse{\equal{#1}{}}{%
		\vspace{-0.5\baselineskip}%
	}{%
		\vspace{-\baselineskip}%
	}%
	}{%
}

% Llama al entorno de resumen
\newenvironment{abstractd}{%
	\ifthenelse{\isundefined{\abstracttable}}{%
		\def\abstracttable {}}{%
	}%
	\begin{abstractenv}{\abstracttable}{\documenttitle}{\nameabstract}{abstractbookmark}%
	}{%
	\end{abstractenv}%
}

% Crea una sección de dedicatoria
\newenvironment{dedicatory}{%
	\clearpage%
	\ifthenelse{\equal{\GLOBALtwoside}{true}}{%
		\coretriggeronpage{\emptypagespredocformat}{}%
	}{}%
	\null%
	\phantomsection%
	\vspace{\stretch{1}}%
	\begin{flushright}%
		\itshape}{%
	\end{flushright}%
	\vspace{\stretch{2}}%
	\null%
}

% Crea una sección de agradecimientos
\newenvironment{acknowledgments}{%
	\clearpage%
	\ifthenelse{\equal{\GLOBALtwoside}{true}}{%
		\coretriggeronpage{\emptypagespredocformat}{}%
	}{}%
	\chapter*{\nameagradec}%
	\ifthenelse{\equal{\addagradectobookmarks}{true}}{%
		\phantomsection%
		\pdfbookmark{\nameagradec}{acknowledgments}}{%
	}%
	\forceindent%
	}{%
}

% Crea una sección de imágenes múltiples
%	#1	Label (opcional)
%	#2	Caption
\newenvironment{images}[2][]{%
	% Modifica globales
	\def\envimageslabelvar {#1}%
	\def\envimagescaptioncf {false}%
	\def\envimagescaptionvar {#2}%
	\global\def\GLOBALenvimageadded {false}%
	\global\def\GLOBALenvimageinitialized {true}%
	% Configura caption y márgenes
	\corevspacevarcm{\marginimagetop}%
	\setcaptionmargincm{\captionmarginmultimg} % Eso es para los wrapfig
	% Inicia la figura
	\begin{samepage}%
	\begin{figure}[H] \centering%
		\ifthenelse{\equal{\GLOBALenvimagecf}{true}}{%
			\ContinuedFloat%
			\global\def\GLOBALenvimagecf {false}%
			\def\envimagescaptioncf {true}%
		}{}%
		\corevspacevarcm{\marginimagemulttop}%
		}{%
		\setcaptionmargincm{\captionlrmargin}%
		\ifthenelse{\equal{\envimagescaptionvar}{}}{%
			\corevspacevarcm{\captionlessmarginimage}%
		}{%
			\corevspacevarcm{\captionmarginimages}%
			\ifthenelse{\equal{\envimagescaptioncf}{true}}{%
				\caption[]{\envimagescaptionvar\envimageslabelvar}%
			}{%
				\caption{\envimagescaptionvar\envimageslabelvar}%
			}%
		}%
	\end{figure}%
	% Restablece caption y márgenes
	\setcaptionmargincm{\captionlrmargin}%
	\corevspacevarcm{\marginimagebottom}%
	\end{samepage}
	% Restablece globales
	\global\def\GLOBALenvimageinitialized {false}%
}

% Crea una sección de imágenes múltiples completa dentro de un multicol
%	#1	Label (opcional)
%	#2	Posición de la imagen, "bottom", "top"
%	#3	Caption
\newenvironment{imagesmc}[3][]{%
	% Modifica globales
	\def\envimageslabelvar {#1}%
	\def\envimagesmcpos {#2}%
	\def\envimagescaptioncf {false}%
	\def\envimagescaptionvar {#3}%
	\global\def\GLOBALenvimageadded {false}%
	\global\def\GLOBALenvimageinitialized {true}%
	\checkinsidemulticol%
	\checkoutsideappendix%
	% Configura caption y márgenes
	\setcaptionmargincm{\captionmarginmultimg} % Eso es para los wrapfig
	% Inicia la figura
	\ifthenelse{\equal{#2}{bottom}}{%
		\begin{figure*}[!b] \centering%
	}{%
	\ifthenelse{\equal{#2}{top}}{%
		\begin{figure*}[!t] \centering%
	}{%
		\errmessage{LaTeX Warning: Posicion de imagen invalida, valores esperados: bottom,top}
		\stop
	}}%
		\ifthenelse{\equal{\GLOBALenvimagecf}{true}}{%
			\ContinuedFloat%
			\global\def\GLOBALenvimagecf {false}%
			\def\envimagescaptioncf {true}%
		}{}%
		\corevspacevarcm{\marginimagemulttop}%
	}{%
		\setcaptionmargincm{\captionlrmargin}%
		\ifthenelse{\equal{\envimagescaptionvar}{}}{%
			\corevspacevarcm{\captionlessmarginimage}%
		}{%
			\corevspacevarcm{\captionmarginimagesmc}%
			\ifthenelse{\equal{\envimagescaptioncf}{true}}{%
				\caption[]{\envimagescaptionvar\envimageslabelvar}%
			}{%
				\caption{\envimagescaptionvar\envimageslabelvar}%
			}%
		}%
	\end{figure*}%
	% Restablece caption y márgenes
	\setcaptionmargincm{\captionlrmarginmc}%
	% Restablece globales
	\global\def\GLOBALenvimageinitialized {false}%
}

% Crea un entorno para insertar ecuaciones en el índice
% 	#1	Label (opcional)
%	#2	Leyenda de la ecuación
\newenvironment{equationindex}[2][]{%
	\def\coreequationindexcaption {#2}%
	\emptyvarerr{\coreequationindexcaption}{#2}{Leyenda no definida}%
	\corevspacevarcm{\margineqnindextop}%
	\begin{samepage}%
		\begin{equation}%
			\text{#1}%
		}{%
		\end{equation}
		\myindexequations{\coreequationindexcaption}%
		\corevspacevarcm{\margineqnindexbottom}%
	\end{samepage}
	\coreinsertequationcaption{\textit{\coreequationindexcaption}}%
	\addtocounter{templateIndexEquations}{1}%
	\coreafterequationfn%
}

% -----------------------------------------------------------------------------
% IMPORTACIÓN DE ESTILOS
% -----------------------------------------------------------------------------
% ABAP
\lstdefinestyle{abap}{
	language=ABAP
}

% Ada
\lstdefinestyle{ada}{
	language=[2005]Ada
}

% Assembler
\lstdefinelanguage[x64]{Assembler}[x86masm]{Assembler}{
	morekeywords={
		CDQE,CMPSQ,CMPXCHG16B,CQO,IRETQ,JRCXZ,LODSQ,MOVSXD,POPFQ,PUSHFQ,r8,r8b,r8d,r8w,r9,r9b,r9d,r9w,r10,r10b,r10d,r10w,r11,r11b,r11d,r11w,r12,r12b,r12d,r12w,r13,r13b,r13d,r13w,r14,r14b,r14d,r14w,r15,r15b,r15d,r15w,rax,rbp,rbx,rcx,rdi,RDTSCP,rdx,rsi,rsp,SCASQ,STOSQ,SWAPGS
	}
}
\lstdefinestyle{assemblerx64}{
	language=[x64]Assembler
}
\lstdefinestyle{assemblerx86}{
	language=[x86masm]Assembler
}

% Awk
\lstdefinestyle{awk}{
	language=[gnu]Awk
}

% Bash
\lstdefinestyle{bash}{
	language=bash,
	breakatwhitespace=false,
	morecomment=[l]{rem},
	morecomment=[s]{::}{::},
	morekeywords={
		call,cp,dig,gcc,git,grep,ls,mv,python,rm,sudo,vim
	},
	sensitive=false
}

% Basic
\lstdefinestyle{basic}{
	language=[Visual]Basic
}

% C
\lstdefinestyle{c}{
	language=C,
	breakatwhitespace=false,
	keepspaces=true
}

% Caml
\lstdefinestyle{caml}{
	language=[light]Caml
}

% CMake
\lstdefinestyle{cmake}{
	language=[gnu] make,
	keywordstyle=[2]\color{dkcyan},
	morekeywords=[1]{
		add_custom_command,add_custom_target,add_definitions,add_executable,add_library,add_subdirectory,cmake_minimum_required,cmake_policy,configure_file,cuda_add_library,cuda_include_directories,else,elseif,endforeach,endfunction,endif,endmacro,execute_process,file,find_library,find_package,find_path,find_program,foreach,function,get_directory_property,get_filename_component,get_filename_component,get_source_file_property,get_target_property,if,include,include_directories,install,link_directories,list,macro,mark_as_advanced,message,option,PKG_CHECK_MODULES,project,set,SET_CHECK_CXX_FLAGS,set_property,set_source_files_properties,set_target_properties,string,target_compile_options,target_include_directories,target_link_libraries,unset
	},
	morekeywords=[2]{
		AND,APPEND,APPLE,ARCHIVE,CACHE,CMAKE_CURRENT_LIST_DIR,CMAKE_CXX_STANDARD,CMAKE_MODULE_PATH,CMAKE_SYSTEM_NAME,COMMAND,COMMENT,COMPILE_DEFINITIONS,CONFIG,DEFINED,DEPENDS,DESTINATION,DIRECTORY,ENDIF,ENV,EQUAL,ERROR_QUIET,EXISTS,FATAL_ERROR,FILES,FILES_MATCHING,FIND,FIND,FIND_LIBRARY,FORCE,GLOB,GREATER,IF,INCLUDE_DIRECTORIES,IS_ABSOLUTE,LESS,LIBRARY,LINK_PRIVATE,LIST,MAIN_DEPENDENCY,MAKE_DIRECTORY,MARK_AS_ADVANCED,MATCHALL,MATCHES,NOT,OBJECT,OFF,ON,OPTIONAL,OR,OUTPUT,OUTPUT_STRIP_TRAILING_WHITESPACE,OUTPUT_VARIABLE,PARENT_SCOPE,PATTERN,PRE_BUILD_COMMAND,PRE_LINK,PRIVATE,PROJECT_NAME,PROPERTIES,PROPERTY,PUBLIC,REGEX,RELEASE,RENAME,REQUIRED,RUNTIME,SET,STATIC,STREQUAL,SYSTEM,TARGET,TARGETS,TOUPPER,UNIX,VERSION,VERSION_EQUAL,VERSION_LESS,WIN32,WORKING_DIRECTORY
	}
}

% Cobol
\lstdefinestyle{cobol}{
	language=Cobol
}

% C++
\lstdefinestyle{cpp}{
	language=C++,
	breakatwhitespace=false,
	morekeywords={NULL}
}

% C#
\lstdefinestyle{csharp}{
	language=csh,
	morecomment=[l]{//},
	morecomment=[s]{/*}{*/},
	morekeywords={
		abstract,as,base,bool,break,byte,case,catch,char,checked,class,const,continue,decimal,default,delegate,do,double,else,enum,event,explicit,extern,false,finally,fixed,float,for,foreach,goto,if,implicit,in,int,interface,internal,is,lock,long,namespace,new,null,object,operator,out,override,params,private,protected,public,readonly,ref,return,sbyte,sealed,short,sizeof,stackalloc,static,string,struct,switch,this,throw,true,try,typeof,uint,ulong,unchecked,unsafe,ushort,using,virtual,void,volatile,while
	}
}

% CSS
\lstdefinelanguage{CSS}{
	morecomment=[s]{/*}{*/},
	morekeywords={
		-moz-binding,-moz-border-bottom-colors,-moz-border-left-colors,-moz-border-radius,-moz-border-radius-bottomleft,-moz-border-radius-bottomright,-moz-border-radius-topleft,-moz-border-radius-topright,-moz-border-right-colors,-moz-border-top-colors,-moz-opacity,-moz-outline,-moz-outline-color,-moz-outline-style,-moz-outline-width,-moz-user-focus,-moz-user-input,-moz-user-modify,-moz-user-select,-replace,-set-link-source,-use-link-source,accelerator,azimuth,background,background-attachment,background-color,background-image,background-position,background-position-x,background-position-y,background-repeat,behavior,border,border-bottom,border-bottom-color,border-bottom-style,border-bottom-width,border-collapse,border-color,border-left,border-left-color,border-left-style,border-left-width,border-right,border-right-color,border-right-style,border-right-width,border-spacing,border-style,border-top,border-top-color,border-top-style,border-top-width,border-width,bottom,caption-side,clear,clip,color,content,counter-increment,counter-reset,cue,cue-after,cue-before,cursor,direction,display,elevation,empty-cells,filter,float,font,font-family,font-size,font-size-adjust,font-stretch,font-style,font-variant,font-weight,height,ime-mode,include-source,layer-background-color,layer-background-image,layout-flow,layout-grid,layout-grid-char,layout-grid-char-spacing,layout-grid-line,layout-grid-mode,layout-grid-type,left,letter-spacing,line-break,line-height,list-style,list-style-image,list-style-position,list-style-type,margin,margin-bottom,margin-left,margin-right,margin-top,marker-offset,marks,max-height,max-width,min-height,min-width,orphans,outline,outline-color,outline-style,outline-width,overflow,overflow-X,overflow-Y,padding,padding-bottom,padding-left,padding-right,padding-top,page,page-break-after,page-break-before,page-break-inside,pause,pause-after,pause-before,pitch,pitch-range,play-during,position,quotes,richness,right,ruby-align,ruby-overhang,ruby-position,scrollbar-3d-light-color,scrollbar-arrow-color,scrollbar-base-color,scrollbar-dark-shadow-color,scrollbar-face-color,scrollbar-highlight-color,scrollbar-shadow-color,scrollbar-track-color,size,speak,speak-header,speak-numeral,speak-punctuation,speech-rate,stress,table-layout,text-align,text-align-last,text-autospace,text-decoration,text-indent,text-justify,text-kashida-space,text-overflow,text-shadow,text-transform,text-underline-position,top,unicode-bidi,vertical-align,visibility,voice-family,volume,white-space,widows,width,word-break,word-spacing,word-wrap,writing-mode,z-index,zoom
	},
	morestring=[s]{:}{;},
	sensitive=true
}
\lstdefinestyle{css}{
	language=CSS,
	breakatwhitespace=true
}

% CSV
\lstdefinestyle{csv}{
	language={}
}

% CUDA
\lstdefinestyle{cuda}{
	language=C++,
	breakatwhitespace=false,
	emph={
		cudaFree,cudaMalloc,__device__,__global__,__host__,__shared__,__syncthreads
	},
	emphstyle=\color{dkcyan}\ttfamily,
	morecomment=[l][\color{magenta}]{\#},
	moredelim=[s][\ttfamily]{<<<}{>>>}
}

% Dart
\lstdefinestyle{dart}{
	language=Java,
	emph=[2]{
		findAllElements,findElements
	},
	morekeywords={
		*,get,library,List,num,set,String,var
	}
}

% Docker
\lstdefinelanguage{docker}{
	comment=[l]{\#},
	keywords={
		ADD,CMD,COPY,ENTRYPOINT,ENV,EXPOSE,FROM,LABEL,MAINTAINER,ONBUILD,RUN,STOPSIGNAL,USER,VOLUME,WORKDIR
	},
	morestring=[b]',
	morestring=[b]"
}
\lstdefinestyle{docker}{
	language=docker,
	breakatwhitespace=true
}

% Elisp
\lstdefinestyle{elisp}{
	language=elisp
}

% Elixir
\lstdefinestyle{elixir}{
	morekeywords={
		case,catch,def,do,else,false,use,alias,receive,timeout,defmacro,defp,for,if,import,defmodule,defprotocol,nil,defmacrop,defoverridable,defimpl,super,fn,raise,true,try,end,with,unless
	},
	otherkeywords={
		<-,->, |>, \%\{, \}, \{, \, (, )
	},
	morecomment=[l]{\#},
	morecomment=[n]{/*}{*/},
	morecomment=[s][\color{purple}]{:}{\ },
	morestring=[s][\color{mauve}]"",
	sensitive=true
}

% Erlang
\lstdefinestyle{erlang}{
	language=erlang
}

% Fortran-95
\lstdefinestyle{fortran}{
	language=[95]Fortran,
	breakatwhitespace=false
}

% F#
\lstdefinestyle{fsharp}{
	morecomment=[l][\color{dkgreen}]{///},
	morecomment=[l][\color{dkgreen}]{//},
	morecomment=[s][\color{dkgreen}]{{(*}{*)}},
	morestring=[b]",
	morekeywords={
		abstract,and,Application,Array,Async,async,begin,cloud,do,else,end,false,finally,for,fun,function,if,in,inherit,interface,let,List,match,member,module,mutable,namespace,new,of,open,rec,return,Seq,static,System,then,true,try,type,use,while,with,yield
	},
	otherkeywords={
		by,do!,from,let!,order,return!,select,use!,var,where,yield!
	},
	sensitive=true
}

% GLSL
\lstdefinelanguage{GLSL}{
	alsoletter={\#},
	morekeywords=[1]{
		attribute,bool,break,bvec2,bvec3,bvec4,case,centroid,const,continue,default,discard,do,else,false,flat,float,for,highp,if,in,inout,int,invariant,isampler1D,isampler1DArray,isampler2D,isampler2DArray,isampler2DMS,isampler2DMSArray,isampler2DRect,isampler3D,isamplerBuffer,isamplerCube,ivec2,ivec3,ivec4,layout,lowp,mat2,mat2x2,mat2x3,mat2x4,mat3,mat3x2,mat3x3,mat3x4,mat4,mat4x2,mat4x3,mat4x4,mediump,noperspective,out,precision,return,sampler1D,sampler1DArray,sampler1DArrayShadow,sampler1DShadow,sampler2D,sampler2DArray,sampler2DArrayShadow,sampler2DMS,sampler2DMSArray,sampler2DRect,sampler2DRectShadow,sampler2DShadow,sampler3D,samplerBuffer,samplerCube,samplerCubeShadow,smooth,struct,switch,true,uint,uniform,usampler1D,usampler1DArray,usampler2D,usampler2DArray,usampler2DMS,usampler2DMSArray,usampler2DRect,usampler3D,usamplerBuffer,usamplerCube,uvec2,uvec3,uvec4,varying,vec2,vec3,vec4,void,while
	},
	morekeywords=[2]{
		abs,acos,acosh,all,any,asin,asinh,atan,atan,atanh,ceil,clamp,cos,cosh,cross,degrees,determinant,dFdx,dFdy,distance,dot,EmitVertex,EndPrimitive,equal,exp,exp2,faceforward,floatBitsToInt,floatBitsToUint,floor,fract,fwidth,greaterThan,greaterThanEqual,intBitsToFloat,inverse,inversesqrt,isinf,isnan,length,lessThan,lessThanEqual,log,log2,matrixCompMult,max,min,mix,mod,modf,noise1,noise2,noise3,noise4,normalize,not,notEqual,outerProduct,pow,radians,reflect,refract,round,roundEven,shadow1D,shadow1DLod,shadow1DProj,shadow1DProjLod,shadow2D,shadow2DLod,shadow2DProj,shadow2DProjLod,sign,sin,sinh,smoothstep,sqrt,step,tan,tanh,texelFetch,texelFetchOffset,texture,texture1D,texture1DProj,texture1DProjLod,texture2D,texture2DLod,texture2DProj,texture2DProjLod,texture3D,texture3DLod,texture3DProj,texture3DProjLod,textureCube,textureCubeLod,textureGrad,textureGradOffset,textureLod,textureLodOffset,textureOffset,textureProj,textureProjGrad,textureProjGradOffset,textureProjLod,textureProjLodOffset,textureProjOffset,textureSize,transpose,trunc,uintBitsToFloat
	},
	morekeywords=[3]{
		\#version,core,gl_ClipDistance,gl_ClipDistance,gl_ClipVertex,gl_DepthRange,gl_FragColor,gl_FragCoord,gl_FragData,gl_FragDepth,gl_FrontFacing,gl_InstanceID,gl_Layer,gl_MaxClipDistances,gl_MaxCombinedTextureImageUnits,gl_MaxDrawBuffers,gl_MaxDrawBuffers,gl_MaxFragmentInputComponents,gl_MaxFragmentUniformComponents,gl_MaxGeometryInputComponents,gl_MaxGeometryOutputComponents,gl_MaxGeometryOutputVertices,gl_MaxGeometryOutputVertices,gl_MaxGeometryTextureImageUnits,gl_MaxGeometryTotalOutputComponents,gl_MaxGeometryUniformComponents,gl_MaxGeometryVaryingComponents,gl_MaxTextureImageUnits,gl_MaxVaryingComponents,gl_MaxVaryingFloats,gl_MaxVertexAttribs,gl_MaxVertexOutputComponents,gl_MaxVertexTextureImageUnits,gl_MaxVertexUniformComponents,gl_PerVertex,gl_PointCoord,gl_PointSize,gl_Position,gl_PrimitiveID,gl_VertexID
	},
	morecomment=[l]{//},
	morecomment=[s]{/*}{*/}
}
\lstdefinestyle{glsl}{
	language=GLSL,
	keywordstyle=[3]\color{dkcyan}\ttfamily,
	prebreak=\raisebox{0ex}[0ex][0ex]{\ensuremath{\hookleftarrow}},
	sensitive=true,
	upquote=true
}

% Gnuplot
\lstdefinestyle{gnuplot}{
	language=Gnuplot
}

% Go
\lstdefinestyle{go}{
	language=Go
}

% Haskell
\lstdefinestyle{haskell}{
	language=haskell,
	morecomment=[l]\%
}

% HTML5
\lstdefinelanguage{HTML5}{
	language=html,
	alsoletter={<>=-},
	morecomment=[s]{<!--}{-->},
	ndkeywords={
		% General
		=,
		% Atributos HTML
		accept-charset=,accept=,accesskey=,action=,align=,alt=,async=,autocomplete=,autofocus=,autoplay=,autosave=,bgcolor=,border=,buffered=,challenge=,charset=,checked=,cite=,class=,code=,codebase=,color=,cols=,colspan=,content=,contenteditable=,contextmenu=,controls=,coords=,data=,datetime=,default=,defer=,dir=,dirname=,disabled=,download=,draggable=,dropzone=,enctype=,for=,form=,formaction=,headers=,height=,hidden=,high=,href=,hreflang=,http-equiv=,icon=,id=,ismap=,itemprop=,keytype=,kind=,label=,lang=,language=,list=,loop=,low=,manifest=,max=,maxlength=,media=,method=,min=,multiple=,name=,novalidate=,open=,optimum=,pattern=,ping=,placeholder=,poster=,preload=,pubdate=,radiogroup=,readonly=,rel=,required=,reversed=,rows=,rowspan=,sandbox=,scope=,scoped=,seamless=,selected=,shape=,size=,sizes=,span=,spellcheck=,src=,srcdoc=,srclang=,start=,step=,style=,summary=,tabindex=,target=,title=,type=,usemap=,value=,width=,wrap=,
		% Propiedades CSS
		-moz-binding:,-moz-border-bottom-colors:,-moz-border-left-colors:,-moz-border-radius-bottomleft:,-moz-border-radius-bottomright:,-moz-border-radius-topleft:,-moz-border-radius-topright:,-moz-border-radius:,-moz-border-right-colors:,-moz-border-top-colors:,-moz-opacity:,-moz-outline-color:,-moz-outline-style:,-moz-outline-width:,-moz-outline:,-moz-transform:,-moz-user-focus:,-moz-user-input:,-moz-user-modify:,-moz-user-select:,-replace:,-set-link-source:,-use-link-source:,accelerator:,azimuth:,background-attachment:,background-color:,background-image:,background-position-x:,background-position-y:,background-position:,background-repeat:,background:,behavior:,border-bottom-color:,border-bottom-style:,border-bottom-width:,border-bottom:,border-collapse:,border-color:,border-left-color:,border-left-style:,border-left-width:,border-left:,border-right-color:,border-right-style:,border-right-width:,border-right:,border-spacing:,border-style:,border-top-color:,border-top-style:,border-top-width:,border-top:,border-width:,border:,bottom:,caption-side:,clear:,clip:,color:,content:,counter-increment:,counter-reset:,cue-after:,cue-before:,cue:,cursor:,direction:,display:,elevation:,empty-cells:,filter:,float:,font-family:,font-size-adjust:,font-size:,font-stretch:,font-style:,font-variant:,font-weight:,font:,height:,ime-mode:,include-source:,layer-background-color:,layer-background-image:,layout-flow:,layout-grid-char-spacing:,layout-grid-char:,layout-grid-line:,layout-grid-mode:,layout-grid-type:,layout-grid:,left:,letter-spacing:,line-break:,line-height:,list-style-image:,list-style-position:,list-style-type:,list-style:,margin-bottom:,margin-left:,margin-right:,margin-top:,margin:,marker-offset:,marks:,max-height:,max-width:,min-height:,min-width:,orphans:,outline-color:,outline-style:,outline-width:,outline:,overflow-X:,overflow-Y:,overflow:,padding-bottom:,padding-left:,padding-right:,padding-top:,padding:,page-break-after:,page-break-before:,page-break-inside:,page:,pause-after:,pause-before:,pause:,pitch-range:,pitch:,play-during:,position:,quotes:,richness:,right:,ruby-align:,ruby-overhang:,ruby-position:,scrollbar-3d-light-color:,scrollbar-arrow-color:,scrollbar-base-color:,scrollbar-dark-shadow-color:,scrollbar-face-color:,scrollbar-highlight-color:,scrollbar-shadow-color:,scrollbar-track-color:,size:,speak-header:,speak-numeral:,speak-punctuation:,speak:,speech-rate:,stress:,table-layout:,text-align-last:,text-align:,text-autospace:,text-decoration:,text-indent:,text-justify:,text-kashida-space:,text-overflow:,text-shadow:,text-transform:,text-underline-position:,top:,transform:,transition-duration:,transition-property:,transition-timing-function:,unicode-bidi:,vertical-align:,visibility:,voice-family:,volume:,white-space:,widows:,width:,word-break:,word-spacing:,word-wrap:,writing-mode:,z-index:,zoom:
	},
	otherkeywords={
		<,</,>,</a,<a,</a>,</abbr,<abbr,</abbr>,</address,<address,</address>,</area,<area,</area>,</area,<area,</area>,</article,<article,</article>,</aside,<aside,</aside>,</audio,<audio,</audio>,</audio,<audio,</audio>,</b,<b,</b>,</base,<base,</base>,</bdi,<bdi,</bdi>,</bdo,<bdo,</bdo>,</blockquote,<blockquote,</blockquote>,</body,<body,</body>,</br,<br,</br>,</button,<button,</button>,</canvas,<canvas,</canvas>,</caption,<caption,</caption>,</cite,<cite,</cite>,</code,<code,</code>,</col,<col,</col>,</colgroup,<colgroup,</colgroup>,</data,<data,</data>,</datalist,<datalist,</datalist>,</dd,<dd,</dd>,</del,<del,</del>,</details,<details,</details>,</dfn,<dfn,</dfn>,</div,<div,</div>,</dl,<dl,</dl>,</dt,<dt,</dt>,</em,<em,</em>,</embed,<embed,</embed>,</fieldset,<fieldset,</fieldset>,</figcaption,<figcaption,</figcaption>,</figure,<figure,</figure>,</footer,<footer,</footer>,</form,<form,</form>,</h1,<h1,</h1>,</h2,<h2,</h2>,</h3,<h3,</h3>,</h4,<h4,</h4>,</h5,<h5,</h5>,</h6,<h6,</h6>,</head,<head,</head>,</header,<header,</header>,</hr,<hr,</hr>,</html,<html,</html>,</i,<i,</i>,</iframe,<iframe,</iframe>,</img,<img,</img>,</input,<input,</input>,</ins,<ins,</ins>,</kbd,<kbd,</kbd>,</keygen,<keygen,</keygen>,</label,<label,</label>,</legend,<legend,</legend>,</li,<li,</li>,</link,<link,</link>,</main,<main,</main>,</map,<map,</map>,</mark,<mark,</mark>,</math,<math,</math>,</menu,<menu,</menu>,</menuitem,<menuitem,</menuitem>,</meta,<meta,</meta>,</meter,<meter,</meter>,</nav,<nav,</nav>,</noscript,<noscript,</noscript>,</object,<object,</object>,</ol,<ol,</ol>,</optgroup,<optgroup,</optgroup>,</option,<option,</option>,</output,<output,</output>,</p,<p,</p>,</param,<param,</param>,</pre,<pre,</pre>,</progress,<progress,</progress>,</q,<q,</q>,</rp,<rp,</rp>,</rt,<rt,</rt>,</ruby,<ruby,</ruby>,</s,<s,</s>,</samp,<samp,</samp>,</script,<script,</script>,</section,<section,</section>,</select,<select,</select>,</small,<small,</small>,</source,<source,</source>,</span,<span,</span>,</strong,<strong,</strong>,</style,<style,</style>,</summary,<summary,</summary>,</sup,<sup,</sup>,</svg,<svg,</svg>,</table,<table,</table>,</tbody,<tbody,</tbody>,</td,<td,</td>,</template,<template,</template>,</textarea,<textarea,</textarea>,</tfoot,<tfoot,</tfoot>,</th,<th,</th>,</thead,<thead,</thead>,</time,<time,</time>,</title,<title,</title>,</tr,<tr,</tr>,</track,<track,</track>,</u,<u,</u>,</ul,<ul,</ul>,</var,<var,</var>,</video,<video,</video>,</wbr,<wbr,</wbr>,/>,<!
	},
	sensitive=true,
	tag=[s]
}
\lstdefinestyle{html}{
	language=HTML5,
	alsodigit={.:;},
	alsolanguage=JavaScript,
	firstnumber=1,
	ndkeywordstyle=\color{dkgreen}\bfseries,
	numberfirstline=true
}

% INI, Archivos de configuraciones
\lstdefinestyle{ini}{
	language={},
	commentstyle=\color{gray}\ttfamily,
	keywordstyle={\color{black}\bfseries},
	morecomment=[l]{;},
	morecomment=[l]{\#},
	morecomment=[s][\color{dkgreen}\bfseries]{[}{]},
	morekeywords={},
	otherkeywords={=,:}
}

% Java
\lstdefinestyle{java}{
	language=Java,
	breakatwhitespace=true,
	keepspaces=true
}

% Javascript
\lstdefinelanguage{JavaScript}{
	comment=[l]{//},
	keepspaces=true,
	keywords={
		break,else,false,for,function,if,in,new,null,return,true,typeof,var,while
	},
	morecomment=[s]{/*}{*/},
	morestring=[b]',
	morestring=[b]",
	morestring=[b]`,
	ndkeywords={
		await,async,case,catch,class,const,default,do,enum,export,extends,finally,from,implements,import,instanceof,let,static,super,switch,then,this,throw,try
	},
	ndkeywordstyle=\color{blue}\bfseries,
	sensitive=false
}
\lstdefinestyle{javascript}{
	language=JavaScript
}

% JSON
\lstdefinestyle{json}{
	literate=*{0}{{{\color{cardinalred}0}}}{1}{1}{{{\color{cardinalred}1}}}{1}{2}
	{{{\color{cardinalred}2}}}{1}{3}{{{\color{cardinalred}3}}}{1}{4}{{{\color{cardinalred}4}}}
	{1}{5}{{{\color{cardinalred}5}}}{1}{6}{{{\color{cardinalred}6}}}{1}{7}{{{\color{cardinalred}7}}}
	{1}{8}{{{\color{cardinalred}8}}}{1}{9}{{{\color{cardinalred}9}}}{1}{:}
	{{{\color{dkcyan}{:}}}}{1}{,}{{{\color{dkcyan}{,}}}}{1}{\{}
	{{{\color{MidnightBlue}{\{}}}}{1}{\}}{{{\color{MidnightBlue}{\}}}}}
	{1}{[}{{{\color{MidnightBlue}{[}}}}{1}{]}{{{\color{MidnightBlue}{]}}}}{1},
	tabsize=2
}

% Julia
\lstdefinestyle{julia}{
	keywordsprefix=\@,
	morecomment=[l]{\#},
	morekeywords={
		abstract,Any,applicable,assert,baremodule,begin,bitstype,Bool,break,catch,ccall,Complex64,Complex128,const,continue,convert,dlopen,dlsym,do,edit,else,elseif,end,eps,error,exit,export,finalizer,Float32,Float64,for,function,global,hash,if,im,immutable,import,importall,in,Inf,Int,Int8,Int16,Int32,Int64,invoke,is,isa,isequal,let,load,local,macro,method_exists,module,Nan,new,None,Nothing,ntuple,pi,promote,promote_type,quote,realmax,realmin,return,sizeof,subtype,system,throw,try,tuple,type,typealias,typemax,typemin,typeof,uid,Uint,Uint8,Uint16,Uint32,Uint64,using,while,whos
	},
	morestring=[b]',
	morestring=[b]",
	sensitive=true
}

% Kotlin
\lstdefinestyle{kotlin}{
	comment=[l]{//},
	emph={delegate,filter,first,firstOrNull,forEach,lazy,map,mapNotNull,println,
		return@},
	emphstyle={\color{blue}},
	keywords={
		abstract,actual,as,as?,break,by,class,companion,continue,data,do,dynamic,else,enum,expect,false,final,for,fun,get,if,import,in,interface,internal,is,null,object,override,package,private,public,return,set,super,suspend,this,throw,true,try,typealias,val,var,vararg,when,where,while
	},
	morecomment=[s]{/*}{*/},
	morestring=[b]",
	morestring=[s]{"""*}{*"""},
	ndkeywords={
		@Deprecated,@JvmField,@JvmName,@JvmOverloads,@JvmStatic,@JvmSynthetic,Array,Byte,Double,Float,Int,Integer,Iterable,Long,Runnable,Short,String
	},
	ndkeywordstyle=\color{BurntOrange}\bfseries,
	sensitive=true
}

% LaTeX
\lstdefinestyle{latex}{
	language=TeX,
	morekeywords={
		aacos,aasin,aatan,acos,addimage,addimageanum,addimageboxed,align,asin,atan,begin,bibitem,bibliography,bigstrut,boldmath,bookmarksetup,boxed,cancelto,caption,changeheadertitle,checkmark,checkvardefined,cite,clearpage,dd,degree,eqref,equal,frac,fracnpartial,fullcite,hline,href,ifthenelse,imageshspace,imagesnewline,imagesvspace,includefullhfpdf,includehfpdf,insertalign,insertalignanum,insertaligncaptioned,insertaligncaptioned,insertaligncaptionedanum,insertaligned,insertalignedanum,insertalignedcaptioned,insertalignedcaptionedanum,insertemail,insertemptypage,inserteqimage,insertequation,insertequationanum,insertequationcaptioned,insertequationcaptionedanum,insertgather,insertgatheranum,insertgathercaptioned,insertgathercaptionedanum,insertgathered,insertgatheredanum,insertgatheredcaptioned,insertgatheredcaptionedanum,insertimage,insertimageleft,insertimageright,insertindextitle,insertindextitlepage,insertphone,isundefined,itemresize,label,LaTeX,lipsum,lpow,makeatletter,makeatother,newcommand,newcounter,newp,newpage,pow,quotes,ref,renewcommand,section,sectionanum,setcounter,setlength,shortcite,sourcecode,sourcecodep,subsection,subsectionanum,subsubsection,subsubsectionanum,subsubsubsection,subsubsubsection,subsubsubsectionanum,textbf,textit,textregistered,textsuperscript,texttt,throwbadconfig,unboldmath,url,xspace
	}
}

% Lisp
\lstdefinestyle{lisp}{
	language=Lisp,
	morekeywords={if}
}

% LLVM
\lstdefinestyle{llvm}{
	language=LLVM
}

% Lua
\lstdefinestyle{lua}{
	language={[5.3]Lua}
}

% Make
\lstdefinestyle{make}{
	language=[gnu] make
}

% Maple
\lstdefinelanguage{Maple}{
	morecomment=[l]\#,
	morekeywords={
		and,assuming,break,by,catch,description,do,done,elif,else,end,error,export,fi,finally,for,from,global,if,implies,in,intersect,local,minus,mod,module,next,not,o,option,options,or,proc,quit,read,restart,return,save,stop,subset,then,to,try,union,use,uses,with,while,xor
	},
	morestring=[b]",
	morestring=[d],
	sensitive=true
} 
\lstdefinestyle{maple}{
	language=Maple
}

% Mathematica
\lstdefinestyle{mathematica}{
	language=Mathematica
}

% Matlab
\lstdefinestyle{matlab}{
	language=Matlab,
	deletekeywords={fft},
	keepspaces=true,
	morecomment=[l]\%,
	morecomment=[n]{\%\{\^^M}{\%\}\^^M},
	morekeywords={
		addOptional,box,break,catch,cell,classdef,continue,deal,double,end,factorial,for,gradient,hessian,if,isa,ltitr,matlab2tikz,methods,minor,movegui,normcdf,normpdf,on,ones,parse,persistent,poissrnd,properties,repmat,solve,strcat,subs,syms,try,var,warning,xlim,ylim
	}
}

% Mercury
\lstdefinestyle{mercury}{
	language=Mercury
}

% Modula-2
\lstdefinestyle{modula2}{
	language=Modula-2
}

% Objective-C
\lstdefinestyle{objectivec}{
	language=[Objective]C,
	breakatwhitespace=false,
	keepspaces=true,
	moredirectives={
		import
	},
	morekeywords={
		@catch,@class,@dynamic,@encode,@end,@finally,@implementation,@interface,@package,@private,@property,@protected,@protocol,@public,@selector,@synchronized,@synthesize,@throw,@try,assign,BOOL,bycopy,byref,Class,copy,id,IMP,in,inout,Nil,nil,NO,nonatomic,oneway,out,readonly,readwrite,retain,SEL,self,super,YES,_cmd
	}
}

% Octave
\lstdefinestyle{octave}{
	language=Octave,
	keepspaces=true,
	morecomment=[l]\%,
	morecomment=[n]{\%\{\^^M}{\%\}\^^M}
}

% OpenCL
\lstdefinestyle{opencl}{
	language=C++,
	breakatwhitespace=false,
	emph={
		bool2,bool3,bool4,bool8,bool16,char2,char3,char4,char8,char16,complex,constant,event_t,float2,float3,float4,float8,float16,global,half2,half3,half4,half8,half16,image2d_t,image3d_t,imaginary,int2,int3,int4,int8,int16,kernel,local,long2,long3,long4,long8,long16,private,quad,quad2,quad3,quad4,quad8,quad16,sampler_t,short2,short3,short4,short8,short16,uchar2,uchar3,uchar4,uchar8,uchar16,uint2,uint3,uint4,uint8,uint16,ulong2,ulong3,ulong4,ulong8,ulong16,ushort2,ushort3,ushort4,ushort8,ushort16,__constant,__global,__kernel,__local,__private
	},
	emphstyle=\color{dkcyan}\ttfamily,
	morecomment=[l][\color{magenta}]{\#}
}

% OpenSees
\lstdefinestyle{opensees}{
	language=tcl,
	breakatwhitespace=false,
	emph=[1]{
		-accel,-beamUniform,-dir,-dof,-ele,-eleRange,-file,-height,-increment,-initial,-iNode,-integration,-iterate,-jNode,-kNode,-mass,-mat,-matConcrete,-matShear,-matSteel,-max,-maxDim,-maxEta,-maxIter,-min,-minEta,-ndf,-ndm,-node,-nodeRange,-numSublevels,-numSubSteps,-perpDirn,-region,-rho,-sections,-thick,-time,-tol,-type,-width
	},
	emphstyle=[1]\color{black}\bfseries\em,
	keepspaces=true,
	morecomment=[l]{\#},
	morekeywords={
		algorithm,analysis,analyze,constraints,deformation,disp,eleLoad,element,equalDOF,fix,fixX,fixY,fixZmodel,geomTransf,initialize,integrator,layer,loadConst,mass,model,node,numberer,patch,pattern,printA,PySimple1Gen,reaction,recorder,region,rigidDiaphragm,section,system,test,uniaxialMaterial,wipe,wipeAnalysis
	},
	ndkeywords={
		9_4_QuadUP,20_8_BrickUP,AC3D8,Aggregator,ArcLength,ASI3D8,AV3D4,AxialSp,AxialSpHD,BandGeneral,BARSLIP,BasicBuilder,bbarBrick,bbarBrickUP,bbarQuad,bbarQuadUP,BeamColumnJoint,BeamContact2D,BeamContact3D,BeamEndContact3D,BFGS,Bilin,BilinearOilDamper,Bond_SP01,BoucWen,Brick20N,brickUP,Broyden,BWBN,Cast,CatenaryCable,CentralDifference,CFSSSWP,CFSWSWP,Concrete01,Concrete01WithSITC,Concrete02,Concrete03,Concrete04,Concrete06,Concrete07,ConcreteCM,ConcreteD,ConfinedConcrete01,constraintsTypeGravity,Corotational,corotTruss,corotTrussSection,CoupledZeroLength,DeformedShape,dispBeamColumn,dispBeamColumnInt,DisplacementControl,Dodd_Restrepo,Drift,ECC01,Elastic,elasticBeamColumn,ElasticBilin,ElasticMultiLinear,ElasticPP,ElasticPPGap,ElasticTimoshenkoBeam,ElasticTubularJoint,elastomericBearingBoucWen,elastomericBearingPlasticity,ElastomericX,Element,EnergyIncr,enhancedQuad,ENT,Explicitdifference,Fatigue,flatSliderBearing,forceBeamColumn,forceBeamColumn,FourNodeTetrahedron,FPBearingPTV,FRPConfinedConcrete,GeneralizedAlpha,Hardening,HDR,HHT,HyperbolicGapMaterial,Hysteretic,ImpactMaterial,InitStrainMaterial,InitStressMaterial,Joint2D,KikuchiAikenHDR,KikuchiAikenLRB,KikuchiBearing,KrylovNewton,Lagrange,LeadRubberX,LimitState,Linear,LoadControl,LoadControl,MinMax,MinUnbalDispNorm,mkdir,ModElasticBeam2d,ModifiedNewton,ModIMKPeakOriented,ModIMKPinching,MultiLinear,multipleShearSpring,MVLEM,Newmark,Newton,NewtonLineSearch,Node,NodeNumbers,nonlinearBeamColumn,NormDispIncr,numberer,Parallel,PathIndependentMaterial,pattern,PDelta,Pinching4,PinchingLimitStateMaterial,Plain,PyLiq1,PySimple1,quad,quadr,quadUP,QzSimple1,RambergOsgoodSteel,rayleigh,RCM,rect,ReinforcingSteel,RJWatsonEqsBearing,SAWS,SecantNewton,SelfCentering,Series,SFI_MVLEM,ShallowFoundationGen,ShellDKGQ,ShellDKGT,ShellMITC4,ShellNL,ShellNLDKGQ,ShellNLDKGT,SimpleContact2D,SimpleContact3D,singleFPBearing,SparseGeneral,SSPbrick,SSPbrickUP,SSPquad,SSPquadUP,Static,stdBrick,Steel01,Steel01,Steel02,Steel4,SteelMPF,straight,SurfaceLoad,TFP,Transient,TRBDF2,tri31,TripleFrictionPendulum,truss,trussSection,twoNodeLink,TzLiq1,TzSimple1,UniformExcitation,ViewScale,Viscous,ViscousDamper,VS3D4,YamamotoBiaxialHDR,zeroLength,zeroLengthContact,zeroLengthContactNTS2D,zeroLengthImpact3D,zeroLengthImpact3D,zeroLengthInterface2D,zeroLengthND,zeroLengthSection
	},
	ndkeywordstyle=\color{dkcyan}\ttfamily
}

% Pascal
\lstdefinestyle{pascal}{
	language=Pascal,
	morecomment=[l]{//},
	sensitive=false
}

% Perl
\lstdefinestyle{perl}{
	language=Perl,
	alsoletter={\%},
	breakatwhitespace=false,
	keepspaces=true
}

% PHP
\lstdefinestyle{php}{
	language=php,
	emph=[1]{
		php
	},
	emph=[2]{
		if,and,or,else
	},
	emph=[3]{
		abstract,as,const,else,elseif,endfor,endforeach,endif,extends,final,for,foreach,global,if,implements,private,protected,public,static,var
	},
	emphstyle=[1]\color{black},
	emphstyle=[2]\color{blue},
	keywords={
		abstract,and,array,as,break,callable,case,catch,class,clone,const,continue,declare,default,die,do,echo,else,elseif,empty,enddeclare,endfor,endforeach,endif,endswitch,endwhile,eval,exit,extends,final,finally,for,foreach,function,global,goto,if,implements,include,include_once,instanceof,insteadof,interface,isset,list,namespace,new,or,print,private,protected,public,require,require_once,return,static,switch,throw,trait,try,unset,use,var,while,xor,yield,__halt_compiler
	},
	showlines=true,
	upquote=true
}

% Texto plano
\lstdefinestyle{plaintext}{
	language={},
	keepspaces=true,
	postbreak={},
	tabsize=4
}

% Postscript
\lstdefinestyle{postscript}{
	language=PostScript,
	keepspaces=true
}

% Powershell
% https://github.com/rmainer/latex-listings-powershell/blob/master/src/latex-listings-powershell.tex
\lstdefinestyle{powershell}{
	alsodigit={-},
	morecomment=[l]{\#},
	morecomment=[n]{<\#}{\#>},
	morekeywords={
		Add-Content,Add-PSSnapin,Clear-Content,Clear-History,Clear-Host,Clear-Item,Clear-ItemProperty,Clear-Variable,Compare-Object,Connect-PSSession,Convert-Path,ConvertFrom-String,Copy-Item,Copy-ItemProperty,Disable-PSBreakpoint,Disconnect-PSSession,Enable-PSBreakpoint,Enter-PSSession,Exit-PSSession,Export-Alias,Export-Csv,Export-PSSession,ForEach-Object,Format-Custom,Format-Hex,Format-List,Format-Table,Format-Wide,Get-Alias,Get-ChildItem,Get-Clipboard,Get-Command,Get-ComputerInfo,Get-Content,Get-History,Get-Item,Get-ItemProperty,Get-ItemPropertyValue,Get-Job,Get-Location,Get-Member,Get-Module,Get-Process,Get-PSBreakpoint,Get-PSCallStack,Get-PSDrive,Get-PSSession,Get-PSSnapin,Get-Service,Get-TimeZone,Get-Unique,Get-Variable,Get-WmiObject,Group-Object,help,Import-Alias,Import-Csv,Import-Module,Import-PSSession,Invoke-Command,Invoke-Expression,Invoke-History,Invoke-Item,Invoke-RestMethod,Invoke-WebRequest,Invoke-WmiMethod,Measure-Object,mkdir,Move-Item,Move-ItemProperty,New-Alias,New-Item,New-Module,New-PSDrive,New-PSSession,New-PSSessionConfigurationFile,New-Variable,Out-GridView,Out-Host,Out-Printer,Pop-Location,powershell_ise.exe,Push-Location,Receive-Job,Receive-PSSession,Remove-Item,Remove-ItemProperty,Remove-Job,Remove-Module,Remove-PSBreakpoint,Remove-PSDrive,Remove-PSSession,Remove-PSSnapin,Remove-Variable,Remove-WmiObject,Rename-Item,Rename-ItemProperty,Resolve-Path,Resume-Job,Select-Object,Select-String,Set-Alias,Set-Clipboard,Set-Content,Set-Item,Set-ItemProperty,Set-Location,Set-PSBreakpoint,Set-TimeZone,Set-Variable,Set-WmiInstance,Show-Command,Sort-Object,Start-Job,Start-Process,Start-Service,Start-Sleep,Stop-Job,Stop-Process,Stop-Service,Suspend-Job,Tee-Object,Trace-Command,Wait-Job,Where-Object,Write-Output
	},
	morekeywords={
		Do,Else,For,ForEach,Function,If,In,Until,While
	},
	morestring=[b]{"},
	morestring=[b]{'},
	morestring=[s]{@'}{'@},
	morestring=[s]{@"}{"@},
	sensitive=false
}

% Prolog
\lstdefinestyle{prolog}{
	language=Prolog
}

% Promela
\lstdefinestyle{promela}{
	language=Promela
}

% Pseudocódigo
\lstdefinelanguage{Pseudocode}{
	language={},
	breakatwhitespace=false,
	commentstyle=\color{gray}\upshape,
	keepspaces=true,
	keywords={
		and,be,begin,break,datatype,do,elif,else,end,for,foreach,fun,function,if,in,input,let,not,null,or,output,pop,procedure,push,repeat,return,swap,until,while,xor
	},
	keywordstyle=\color{black}\bfseries,
	mathescape=true,
	morecomment=[l]{//},
	morecomment=[l]{\#},
	morecomment=[s]{/*}{*/},
	morecomment=[s]{/**}{*/},
	sensitive=false,
	stringstyle=\color{dkgray}\bfseries\em
}
\lstdefinestyle{pseudocode}{
	language=Pseudocode,
	backgroundcolor=\color{white},
	frame=tb,
	numbers=none
}
\lstdefinestyle{pseudocodecolor}{
	language=Pseudocode
}

% Python
\lstdefinelanguage{pythonEXTENDED}{
	language=Python,
	breakatwhitespace=false,
	emph={
		AbstractSet,Any,AsyncContextManager,AsyncGenerator,AsyncIterable,AsyncIterator,Awaitable,AwaitableGenerator,BinaryIO,ByteString,Callable,Collection,Container,ContextManager,Coroutine,Dict,False,ForwardRef,Generator,GenericMeta,Hashable,IO,ItemsView,Iterable,Iterator,KeysView,List,Mapping,MappingView,Match,Meta,MutableMapping,MutableSequence,MutableSet,NamedTuple,None,Pattern,Reversible,Sequence,Sized,SupportInts,SupportsAbs,SupportsBytes,SupportsComplex,SupportsFloat,SupportsIndex,SupportsRound,TextIO,True,Tuple,TypeAlias,TYPE_CHECKING,Union,ValuesView,__add__,__and__,__eq__,__floordiv__,__ge__,__gt__,__init__,__le__,__lt__,__main__,__mod__,__mul__,__name__,__ne__,__or__,__pow__,__repr__,__str__,__sub__,__truediv__,__xor__
	},
	emphstyle=\color{dkcyan}\ttfamily,
	keepspaces=true,
	morecomment=[s][\color{BurntOrange}]{@}{\ },
	morekeywords={
		as,assert,close,listdir,self,sorted,split,strip,with
	}
}
\lstdefinestyle{python}{
	language=pythonEXTENDED
}

% Q#
\lstdefinestyle{qsharp}{
	mathescape=true,
	morecomment=[l]{//},
	morecomment=[l][\color{dkgreen}]{///},
	morekeywords={
		Adj,Adjoint,adjoint,and,apply,as,auto,BigInt,body,Bool,borrowing,Controlled,controlled,Ctl,distribute,Double,elif,else,fail,false,fixup,for,function,if,in,Int,intrinsic,invert,is,let,mutable,namespace,new,newtype,not,One,open,operation,or,Pauli,PauliI,PauliX,PauliY,PauliZ,Qubit,Range,repeat,Result,return,self,set,String,true,Unit,until,using,while,within,Zero
	},
	morekeywords=[2]{
		Assert,AssertProb,CCNOT,CNOT,Exp,ExpFrac,H,I,M,Measure,Message,R,R1,R1Frac,Random,Reset,ResetAll,RFrac,Rx,Ry,Rz,S,SWAP,T,X,Y,Z
	},
	sensitive=true
}

% R
\lstdefinestyle{r}{
	language=R,
	alsoletter={.<-},
	alsoother={._$},
	deletekeywords={
		df,data,frame,length,as,character
	},
	morecomment=[l]\#,
	morestring=[d]',
	morestring=[d]",
	otherkeywords={
		!,!=,~,$,*,\&,\%/\%,\%*\%,\%\%,<-,<<-,/
	}
}

% Racket
\lstdefinestyle{racket}{
	alsoletter={',`,-,/,>,<,\#,\%},
	morekeywords=[1]{
		define,define-macro,define-stream,define-syntax,lambda,stream-lambda
	},
	morekeywords=[2]{
		->,always_publish,and,\#',\#\%module-begin,\#lang,\#`,begin,begin-for-syntax,Boolean,call-with-current-continuation,call-with-input-file,call-with-output-file,callback,call/cc,case,cond,define-context,define-controller,define-struct/contract,define/contract,delay,do,else,environment,eval,fold,for,for-each,force,get,if,implement,in-range,Integer,label,let,let*,let*-values,let-syntax,let-values,letrec,letrec-syntax,map,maybe_publish,message-box,module,new,not,or,or/c,parent,provide,quasiquote,query,quote,rename-out,require,send,submod,syntax,syntax-case,syntax-rules,unquote,unquote-splicing,when,when-provided,when-required,with-syntax
	},
	morekeywords=[3]{
		export,import
	},
	morecomment=[l]{;},
	moredelim=**[is][\color{lgray}]{<<@<<}{>>@>>},
	moredelim=**[is][\itshape\color{mauve}]{<<;<<}{>>;>>},
	morecomment=[s]{\#|}{|\#},
	morestring=[s]{"}{"},
	sensitive=true
}

% Reil
\lstdefinestyle{reil}{
	comment=[l]{;},
	keywords=[1]{
		ADD,add,and,AND,BISZ,bisz,bsh,BSH,div,DIV,jcc,JCC,LDM,ldm,MOD,mod,mul,MUL,nop,NOP,or,OR,stm,STM,STR,str,sub,SUB,undef,UNDEF,unkn,UNKN,XOR,xor
	},
	keywords=[3]{
		ah,AH,al,AL,AX,ax,bh,BH,BL,bl,bp,BP,bpl,BPL,BX,bx,ch,CH,cl,CL,cx,CX,DH,dh,di,DI,dil,DIL,dl,DL,DX,dx,EAX,eax,EBP,ebp,ebx,EBX,ECX,ecx,EDI,edi,edx,EDX,esi,ESI,esp,ESP,r8,R8,r8b,R8B,r8d,R8D,r8w,R8W,r9,R9,R9B,r9b,R9D,r9d,r9w,R9W,r10,R10,R10B,r10b,R10D,r10d,r10w,R10W,r11,R11,r11b,R11B,r11d,R11D,R11W,r11w,R12,r12,R12B,r12b,r12d,R12D,r12w,R12W,R13,r13,r13b,R13B,R13D,r13d,R13W,r13w,r14,R14,R14B,r14b,r14d,R14D,R14W,r14w,r15,R15,r15b,R15B,r15d,R15D,R15W,r15w,RAX,rax,rbp,RBP,rbx,RBX,RCX,rcx,RDI,rdi,rdx,RDX,RSI,rsi,RSP,rsp,SI,si,SIL,sil,SP,sp,spl,SPL
	},
	sensitive=true
}

% Ruby
\lstdefinestyle{ruby}{
	language=Ruby,
	breakatwhitespace=true,
	morestring=[s][]{\#\{}{\}},
	morestring=*[d]{"},
	sensitive=true
}

% Rust
\lstdefinelanguage{Rust}{
	sensitive,
	alsodigit={},
	alsoletter={!},
	alsoother={},
	morecomment=[l]{//},
	morecomment=[s]{/*}{*/},
	moredelim=[s][{\itshape\color[rgb]{0,0,0.75}}]{\#[}{]},
	morekeywords=[2]{% Traits
		Add,AddAssign,Any,AsciiExt,AsInner,AsInnerMut,AsMut,AsRawFd,AsRawHandle,AsRawSocket,AsRef,Binary,BitAnd,BitAndAssign,Bitor,BitOr,BitOrAssign,BitXor,BitXorAssign,Borrow,BorrowMut,Boxed,BoxPlace,BufRead,BuildHasher,CastInto,CharExt,Clone,CoerceUnsized,CommandExt,Copy,Debug,DecodableFloat,Default,Deref,DerefMut,DirBuilderExt,DirEntryExt,Display,Div,DivAssign,DoubleEndedIterator,DoubleEndedSearcher,Drop,EnvKey,Eq,Error,ExactSizeIterator,ExitStatusExt,Extend,FileExt,FileTypeExt,Float,Fn,FnBox,FnMut,FnOnce,Freeze,From,FromInner,FromIterator,FromRawFd,FromRawHandle,FromRawSocket,FromStr,FullOps,FusedIterator,Generator,Hash,Hasher,Index,IndexMut,InPlace,Int,Into,IntoCow,IntoInner,IntoIterator,IntoRawFd,IntoRawHandle,IntoRawSocket,IsMinusOne,IsZero,Iterator,JoinHandleExt,LargeInt,LowerExp,LowerHex,MetadataExt,Mul,MulAssign,Neg,Not,Octal,OpenOptionsExt,Ord,OsStrExt,OsStringExt,Packet,PartialEq,PartialOrd,Pattern,PermissionsExt,Place,Placer,Pointer,Product,Put,RangeArgument,RawFloat,Read,Rem,RemAssign,Seek,Shl,ShlAssign,Shr,ShrAssign,Sized,SliceConcatExt,SliceExt,SliceIndex,Stats,Step,StrExt,Sub,SubAssign,Sum,Sync,TDynBenchFn,Terminal,Termination,ToOwned,ToSocketAddrs,ToString,Try,TryFrom,TryInto,UnicodeStr,Unsize,UpperExp,UpperHex,WideInt,Write
	},
	morekeywords=[2]{
		Send
	},
	morekeywords=[3]{% Primitivas
		bool,char,f32,f64,i8,i16,i32,i64,isize,str,u8,u16,u32,u64,unit,usize,i128,u128
	},
	morekeywords=[4]{% Valor y tipo de constructores
		Err,false,None,Ok,Some,true
	},
	morekeywords=[5]{% Identificadores
		assert!,assert_eq!,assert_ne!,cfg!,column!,compile_error!,concat!,concat_idents!,debug_assert!,debug_assert_eq!,debug_assert_ne!,env!,eprint!,eprintln!,file!,format!,format_args!,include!,include_bytes!,include_str!,line!,module_path!,option_env!,panic!,print!,println!,select!,stringify!,thread_local!,try!,unimplemented!,unreachable!,vec!,write!,writeln!
	},
	morekeywords={% Palabras reservadas
		abstract,alignof,become,box,do,final,macro,offsetof,override,priv, proc,pure,sizeof,typeof,unsized,virtual,yield
	},
	morekeywords={
		as,const,let,move,mut,ref,static
	},
	morekeywords={
		break,continue,else,for,if,in,loop,match,return,while
	},
	morekeywords={
		crate,extern,mod,pub,super
	},
	morekeywords={
		dyn,enum,fn,impl,Self,self,struct,trait,type,union,use,where
	},
	morekeywords={
		unsafe
	},
	morestring=[b]{"}
}
\lstdefinestyle{rust}{
	language=Rust,
	keywordstyle=[2]\color[rgb]{0.75,0,0}, % Traits
	keywordstyle=[3]\color[rgb]{0,0.5,0}, % Primitivas
	keywordstyle=[4]\color[rgb]{0,0.5,0}, % Valor y tipo de constructores
	keywordstyle=[5]\color[rgb]{0,0,0.75} % Macros
}

% Scala
\lstdefinestyle{scala}{
	language=scala,
	breakatwhitespace=true,
	morecomment=[l]{//},
	morecomment=[n]{/*}{*/},
	morekeywords={
		abstract,case,catch,class,def,do,else,extends,false,final,finally,for,if,implicit,import,match,mixin,new,null,object,override,package,private,protected,requires,return,sealed,super,this,throw,trait,true,try,type,val,var,while,with,yield
	},
	morestring=[b]',
	morestring=[b]",
	morestring=[b]""",
	otherkeywords={
		=>,<-,<\%,<:,>:,\#,@
	}
}

% Scheme
\lstdefinestyle{scheme}{
	language=Lisp,
	morecomment=[l]{;},
	morekeywords={
		and,begin,case,case-lambda,cond,cond-expand,define,delay,delay-force,do,else,force,guard,if,lambda,let,let*,let*-values,let-syntax,let-values,letrec,letrec*,letrec-syntax,make-parameter,make-promise,map,or,parameterize,promise?,quasiquote,quote,set!,syntax-rules,unless,when
	},
	morestring=[b]"
}

% Scilab
\lstdefinestyle{scilab}{
	language=Scilab
}

% Simula
\lstdefinestyle{simula}{
	language=Simula
}

% SPARQL
\lstdefinestyle{sparql}{
	language=SPARQL
}

% SQL
\lstdefinestyle{sql}{
	language=SQL,
	breakatwhitespace=true
}

% Swift
\lstdefinestyle{swift}{
	language=Swift
}

% TCL
\lstdefinestyle{tcl}{
	language=tcl,
	breakatwhitespace=false,
	keepspaces=true,
	morecomment=[l]{\#}
}

% Visual Basic
\lstdefinestyle{vbscript}{
	language=[Visual]Basic,
	extendedchars=true
}

% Verilog
\lstdefinestyle{verilog}{
	language=Verilog
}

% VDHL
\lstdefinelanguage{VHDL}{
	morekeywords=[1]{
		ALL,all,and,architecture,begin,downto,end,entity,in,is,library,Not,of,or,out,port,use
	},
	morekeywords=[2]{
		IEEE,NUMERIC_STD,STD_LOGIC,std_logic,STD_LOGIC_1164,STD_LOGIC_ARITH,STD_LOGIC_UNSIGNED,STD_LOGIC_VECTOR,std_logic_vector
	},
	morecomment=[l]--
}
\lstdefinestyle{vhdl}{
	language=VHDL
}

% XML
\lstdefinelanguage{XML}{
	morecomment=[s]{<?}{?>},
	morekeywords={
		encoding,type,version,xmlns
	},
	morestring=[b]",
	morestring=[s]{>}{<}
}
\lstdefinestyle{xml}{
	language=XML,
	tabsize=2
}

% -----------------------------------------------------------------------------
% Configuración de códigos fuente
% -----------------------------------------------------------------------------
\lstset{
	aboveskip=\sourcecodeskipabove em,
	basicstyle={\sourcecodefonts\sourcecodefontf\color{\maintextcolor}},
	belowskip=\sourcecodeskipbelow em,
	breaklines=true,
	columns=fullflexible,
	commentstyle=\color{dkgreen}\upshape,
	extendedchars=true,
	fontadjust=true,
	frame=ltb,
	framerule=0pt,
	framexbottommargin=\sourcecodebgmarginbottom pt,
	framexleftmargin=\sourcecodebgmarginleft pt,
	framexrightmargin=\sourcecodebgmarginright pt,
	framextopmargin=\sourcecodebgmargintop pt,
	identifierstyle=\color{\maintextcolor},
	keepspaces=true,
	keywordstyle=\color{blue},
	literate={á}{{\'a}}1 {é}{{\'e}}1 {í}{{\'i}}1 {ó}{{\'o}}1 {ú}{{\'u}}1
		{Á}{{\'A}}1 {É}{{\'E}}1 {Í}{{\'I}}1 {Ó}{{\'O}}1 {Ú}{{\'U}}1 {à}{{\`a}}1
		{è}{{\`e}}1 {ì}{{\`i}}1 {ò}{{\`o}}1 {ù}{{\`u}}1 {À}{{\`A}}1 {È}{{\'E}}1
		{Ì}{{\`I}}1 {Ò}{{\`O}}1 {Ù}{{\`U}}1 {ä}{{\"a}}1 {ë}{{\"e}}1 {ï}{{\"i}}1
		{ö}{{\"o}}1 {ü}{{\"u}}1 {Ä}{{\"A}}1 {Ë}{{\"E}}1 {Ï}{{\"I}}1 {Ö}{{\"O}}1
		{Ü}{{\"U}}1 {â}{{\^a}}1 {ê}{{\^e}}1 {î}{{\^i}}1 {ô}{{\^o}}1 {û}{{\^u}}1
		{Â}{{\^A}}1 {Ê}{{\^E}}1 {Î}{{\^I}}1 {Ô}{{\^O}}1 {Û}{{\^U}}1 {œ}{{\oe}}1
		{Œ}{{\OE}}1 {æ}{{\ae}}1 {Æ}{{\AE}}1 {ß}{{\ss}}1 {ű}{{\H{u}}}1
		{Ű}{{\H{U}}}1 {ő}{{\H{o}}}1 {Ő}{{\H{O}}}1 {ç}{{\c c}}1 {Ç}{{\c C}}1
		{ø}{{\o}}1 {å}{{\r a}}1 {Å}{{\r A}}1 {€}{{\EUR}}1 {£}{{\pounds}}1
		{ñ}{{\~n}}1 {Ñ}{{\~N}}1 {¿}{{?``}}1 {¡}{{!``}}1 {«}{{\guillemotleft}}1
		{»}{{\guillemotright}}1 {°}{{\textdegree}}1 {∢}{{$\sphericalangle$}}1
		{¬}{{$\neg$}}1 {¨}{{\textasciidieresis}}1 {ã}{{\~a}}1 {Ã}{{\~a}}1
		{õ}{{\~o}}1 {Õ}{{\~O}}1 {Ð}{{\DJ}}1 {Ø}{{\O}}1 {Ý}{{\'Y}}1
		{¹}{{\textsuperscript{1}}}1 {²}{{\textsuperscript{2}}}1
		{³}{{\textsuperscript{3}}}1 {⁴}{{\textsuperscript{4}}}1
		{⁵}{{\textsuperscript{5}}}1 {⁶}{{\textsuperscript{6}}}1
		{⁷}{{\textsuperscript{7}}}1 {⁸}{{\textsuperscript{8}}}1
		{⁹}{{\textsuperscript{9}}}1 {⁰}{{\textsuperscript{0}}}1
		{ᵃ}{{\textsuperscript{a}}}1 {ᵇ}{{\textsuperscript{b}}}1
		{ᶜ}{{\textsuperscript{c}}}1 {ᵈ}{{\textsuperscript{d}}}1
		{ᵉ}{{\textsuperscript{e}}}1 {ᶠ}{{\textsuperscript{f}}}1
		{ᵍ}{{\textsuperscript{g}}}1 {ʰ}{{\textsuperscript{h}}}1
		{ᶦ}{{\textsuperscript{i}}}1 {ʲ}{{\textsuperscript{j}}}1
		{ᵏ}{{\textsuperscript{k}}}1 {ˡ}{{\textsuperscript{l}}}1
		{ᵐ}{{\textsuperscript{m}}}1 {ⁿ}{{\textsuperscript{n}}}1
		{ᵒ}{{\textsuperscript{o}}}1 {ᵖ}{{\textsuperscript{p}}}1
		{ᵠ}{{\textsuperscript{q}}}1 {ʳ}{{\textsuperscript{r}}}1
		{ˢ}{{\textsuperscript{s}}}1 {ᵗ}{{\textsuperscript{t}}}1
		{ᵘ}{{\textsuperscript{u}}}1 {ᵛ}{{\textsuperscript{v}}}1
		{ʷ}{{\textsuperscript{w}}}1 {ˣ}{{\textsuperscript{x}}}1
		{ʸ}{{\textsuperscript{y}}}1 {ᶻ}{{\textsuperscript{z}}}1
		{α}{{$\alpha$}}1 {ά}{{$\dot \alpha$}}1 {Γ}{{$\Gamma$}}1 {Þ}{{$\Thorn$}}1
		{γ}{{$\gamma$}}1 {Δ}{{$\Delta$}}1 {δ}{{$\delta$}}1 {þ}{{$\thorn$}}1
		{ε}{{$\epsilon$}}1 {έ}{{$\dot \epsilon$}}1 {ζ}{{$\zeta$}}1
		{η}{{$\eta$}}1 {ή}{{$\dot \eta$}}1 {Θ}{{$\Theta$}}1
		{θ}{{$\theta$}}1 {ι}{{$\iota$}}1 {ί}{{$\dot \iota$}}1
		{Ϊ}{{$\ddot I$}}1 {ϊ}{{$\ddot \iota$}}1 {ΐ}{{$\dddot \iota$}}1
		{κ}{{$\kappa$}}1 {Λ}{{$\Lambda$}}1 {λ}{{$\lambda$}}1
		{μ}{{$\mu$}}1 {ν}{{$\nu$}}1 {Ξ}{{$\Xi$}}1 {ξ}{{$\xi$}}1
		{ό}{{$\dot o$}}1 {Π}{{$\Pi$}}1 {π}{{$\pi$}}1 {ρ}{{$\rho$}}1
		{Σ}{{$\Sigma$}}1 {σ}{{$\sigma$}}1 {ς}{{$\varsigma$}}1 {τ}{{$\tau$}}1
		{υ}{{$\upsilon$}}1 {ύ}{{$\dot \upsilon$}}1 {Ϋ}{{$\ddot Y$}}1
		{ϋ}{{$\ddot \upsilon$}}1 {ΰ}{{$\dddot \upsilon$}}1
		{Φ}{{$\Phi$}}1 {φ}{{$\phi$}}1 {Ψ}{{$\Psi$}}1 {ψ}{{$\psi$}}1
		{Ω}{{$\Omega$}}1 {ω}{{$\omega$}}1 {ώ}{{$\dot \omega$}}1
		{“}{{``}}1 {”}{{''}}1 {…}{{$\ldots$}}1 {`}{\`}1 {–}{{--}}1 { }{{ }}1,
	numbers=left,
	numbersep={\sourcecodenumbersep pt},
	numberstyle=\sourcecodenumbersize\color{dkgray},
	postbreak=\mbox{$\hookrightarrow$\space},
	showspaces=false,
	showstringspaces=false,
	showtabs=false,
	stepnumber=1,
	stringstyle=\color{mauve},
	tabsize={\sourcecodetabsize}
}

% -----------------------------------------------------------------------------
% Chequeo de estilos, cualquier nuevo estilo añadirlo a esta lista
% -----------------------------------------------------------------------------
\newcommand{\checkvalidsourcecodestyle}[1]{%
	\ifthenelse{\equal{#1}{abap}}{}{%
	\ifthenelse{\equal{#1}{ada}}{}{%
	\ifthenelse{\equal{#1}{assemblerx64}}{}{%
	\ifthenelse{\equal{#1}{assemblerx86}}{}{%
	\ifthenelse{\equal{#1}{awk}}{}{%
	\ifthenelse{\equal{#1}{bash}}{}{%
	\ifthenelse{\equal{#1}{basic}}{}{%
	\ifthenelse{\equal{#1}{c}}{}{%
	\ifthenelse{\equal{#1}{caml}}{}{%
	\ifthenelse{\equal{#1}{cmake}}{}{%
	\ifthenelse{\equal{#1}{cobol}}{}{%
	\ifthenelse{\equal{#1}{cpp}}{}{%
	\ifthenelse{\equal{#1}{csharp}}{}{%
	\ifthenelse{\equal{#1}{css}}{}{%
	\ifthenelse{\equal{#1}{csv}}{}{%
	\ifthenelse{\equal{#1}{cuda}}{}{%
	\ifthenelse{\equal{#1}{dart}}{}{%
	\ifthenelse{\equal{#1}{docker}}{}{%
	\ifthenelse{\equal{#1}{elisp}}{}{%
	\ifthenelse{\equal{#1}{elixir}}{}{%
	\ifthenelse{\equal{#1}{erlang}}{}{%
	\ifthenelse{\equal{#1}{fortran}}{}{%
	\ifthenelse{\equal{#1}{fsharp}}{}{%
	\ifthenelse{\equal{#1}{glsl}}{}{%
	\ifthenelse{\equal{#1}{gnuplot}}{}{%
	\ifthenelse{\equal{#1}{go}}{}{%
	\ifthenelse{\equal{#1}{haskell}}{}{%
	\ifthenelse{\equal{#1}{html}}{}{%
	\ifthenelse{\equal{#1}{ini}}{}{%
	\ifthenelse{\equal{#1}{java}}{}{%
	\ifthenelse{\equal{#1}{javascript}}{}{%
	\ifthenelse{\equal{#1}{json}}{}{%
	\ifthenelse{\equal{#1}{julia}}{}{%
	\ifthenelse{\equal{#1}{kotlin}}{}{%
	\ifthenelse{\equal{#1}{latex}}{}{%
	\ifthenelse{\equal{#1}{lisp}}{}{%
	\ifthenelse{\equal{#1}{llvm}}{}{%
	\ifthenelse{\equal{#1}{lua}}{}{%
	\ifthenelse{\equal{#1}{make}}{}{%
	\ifthenelse{\equal{#1}{maple}}{}{%
	\ifthenelse{\equal{#1}{mathematica}}{}{%
	\ifthenelse{\equal{#1}{matlab}}{}{%
	\ifthenelse{\equal{#1}{mercury}}{}{%
	\ifthenelse{\equal{#1}{modula2}}{}{%
	\ifthenelse{\equal{#1}{objectivec}}{}{%
	\ifthenelse{\equal{#1}{octave}}{}{%
	\ifthenelse{\equal{#1}{opencl}}{}{%
	\ifthenelse{\equal{#1}{opensees}}{}{%
	\ifthenelse{\equal{#1}{pascal}}{}{%
	\ifthenelse{\equal{#1}{perl}}{}{%
	\ifthenelse{\equal{#1}{php}}{}{%
	\ifthenelse{\equal{#1}{plaintext}}{}{%
	\ifthenelse{\equal{#1}{postscript}}{}{%
	\ifthenelse{\equal{#1}{powershell}}{}{%
	\ifthenelse{\equal{#1}{prolog}}{}{%
	\ifthenelse{\equal{#1}{promela}}{}{%
	\ifthenelse{\equal{#1}{pseudocode}}{}{%
	\ifthenelse{\equal{#1}{pseudocodecolor}}{}{%
	\ifthenelse{\equal{#1}{python}}{}{%
	\ifthenelse{\equal{#1}{qsharp}}{}{%
	\ifthenelse{\equal{#1}{r}}{}{%
	\ifthenelse{\equal{#1}{racket}}{}{%
	\ifthenelse{\equal{#1}{reil}}{}{%
	\ifthenelse{\equal{#1}{ruby}}{}{%
	\ifthenelse{\equal{#1}{rust}}{}{%
	\ifthenelse{\equal{#1}{scala}}{}{%
	\ifthenelse{\equal{#1}{scheme}}{}{%
	\ifthenelse{\equal{#1}{scilab}}{}{%
	\ifthenelse{\equal{#1}{simula}}{}{%
	\ifthenelse{\equal{#1}{sparql}}{}{%
	\ifthenelse{\equal{#1}{sql}}{}{%
	\ifthenelse{\equal{#1}{swift}}{}{%
	\ifthenelse{\equal{#1}{tcl}}{}{%
	\ifthenelse{\equal{#1}{vbscript}}{}{%
	\ifthenelse{\equal{#1}{verilog}}{}{%
	\ifthenelse{\equal{#1}{vhdl}}{}{%
	\ifthenelse{\equal{#1}{xml}}{}{%
		\errmessage{LaTeX Warning: Estilo de codigo desconocido. Valores esperados: abap,ada,assemblerx64,assemblerx86,awk,bash,basic,c,caml,cmake,cobol,cpp,csharp,css,csv,cuda,dart,docker,elisp,elixir,erlang,fortran,fsharp,glsl,gnuplot,go,haskell,html,ini,java,javascript,json,julia,kotlin,latex,lisp,llvm,lua,make,maple,mathematica,matlab,mercury,modula2,objectivec,octave,opencl,opensees,pascal,perl,php,plaintext,postscript,powershell,prolog,promela,pseudocode,pseudocodecolor,python,qsharp,r,racket,reil,ruby,rust,scala,scheme,scilab,simula,sparql,sql,swift,tcl,vbscript,verilog,vhdl,xml}%
		\stop%
	}}}}}}}}}}}}}}}}}}}}}}}}}}}}}}}}}}}}}}}}}}}}}}}}}}}}}}}}}}}}}}}}}}}}}}}}}}}}}%
}

% Crea un entorno de código inline
%	#1	Estilo de código
%	#2	Código a insertar
\newcommand{\inlinesourcecode}[2]{%
	\inlinesourcecodeboxed[NOCOLOR]{#1}{#2}%
}

% Crea un entorno de código inline dentro de un recuadro de color
%	#1	Color del recuadro
%	#2	Estilo de código
%	#3	Código a insertar
\newcommand{\inlinesourcecodeboxed}[3][]{%
	\emptyvarerr{\inlinesourcecodeboxed}{#2}{Estilo de codigo no definido}%
	\emptyvarerr{\inlinesourcecodeboxed}{#3}{Codigo no definido}%
	\lstset{%
		basicstyle={\sourcecodeilfonts\sourcecodeilfontf\color{\maintextcolor}}%
	}%
	\checkvalidsourcecodestyle{#2}%
	\ifthenelse{\equal{#1}{}}{%
		\Colorbox{\sourcecodebgcolor}{\lstinline[style=#2]!#3!}%
	}{%
	\ifthenelse{\equal{#1}{NOCOLOR}}{%
		\lstinline[style=#2]!#3!%
	}{%
		\Colorbox{#1}{\lstinline[style=#2]!#3!}%
	}}%
	\lstset{%
		basicstyle={\sourcecodefonts\sourcecodefontf\color{\maintextcolor}}%
	}%
}

% Inserta una referencia en un código fuente
% 	#1	Referencia
\newcommand{\coderef}[1]{%
	\ensuremath{\text{\ref{#1}}}%
}

% Inserta una referencia en un código fuente
% 	#1	Referencia
\newcommand{\codeeqref}[1]{%
	\ensuremath{\text{(\ref{#1})}}%
}

% -----------------------------------------------------------------------------
% Estilo de enumeración en griego
% -----------------------------------------------------------------------------
\makeatletter
\def\greek#1{\expandafter\@greek\csname c@#1\endcsname}
\def\Greek#1{\expandafter\@Greek\csname c@#1\endcsname}
\def\@greek#1{%
	\ifcase#1%
		\or $\alpha$%
		\or $\beta$%
		\or $\gamma$%
		\or $\delta$%
		\or $\epsilon$%
		\or $\zeta$%
		\or $\eta$%
		\or $\theta$%
		\or $\iota$%
		\or $\kappa$%
		\or $\lambda$%
		\or $\mu$%
		\or $\nu$%
		\or $\xi$%
		\or $o$%
		\or $\pi$%
		\or $\rho$%
		\or $\sigma$%
		\or $\tau$%
		\or $\upsilon$%
		\or $\phi$%
		\or $\chi$%
		\or $\psi$%
		\or $\omega$%
	\fi%
}
\def\@Greek#1{%
	\ifcase#1%
		\or $\mathrm{A}$%
		\or $\mathrm{B}$%
		\or $\Gamma$%
		\or $\Delta$%
		\or $\mathrm{E}$%
		\or $\mathrm{Z}$%
		\or $\mathrm{H}$%
		\or $\Theta$%
		\or $\mathrm{I}$%
		\or $\mathrm{K}$%
		\or $\Lambda$%
		\or $\mathrm{M}$%
		\or $\mathrm{N}$%
		\or $\Xi$%
		\or $\mathrm{O}$%
		\or $\Pi$%
		\or $\mathrm{P}$%
		\or $\Sigma$%
		\or $\mathrm{T}$%
		\or $\mathrm{Y}$%
		\or $\Phi$%
		\or $\mathrm{X}$%
		\or $\Psi$%
		\or $\Omega$%
	\fi%
}
\makeatother
\AddEnumerateCounter{\greek}{\@greek}{24}
\AddEnumerateCounter{\Greek}{\@Greek}{12}

% -----------------------------------------------------------------------------
% CONFIGURACIÓN INICIAL DEL DOCUMENTO
% -----------------------------------------------------------------------------
% Se revisa si las variables no han sido borradas
\def\documentsubject {}
\def\predocpageromannumber {true}
\def\predocresetpagenumber {true}
\def\indexnewpagec {false}
\def\indexnewpagef {false}
\def\indexnewpaget {false}
\def\indexnewpagee {false}
\def\showindex {true}
\def\showindexofcontents {true}
\def\coursecode {}
\def\coursename {}
\def\indexsectionfontsize {\sectionfontsize}
\def\indexsectionstyle {\sectionfontstyle}

\checkvardefined{\coursecode}
\checkvardefined{\documentauthor}
\checkvardefined{\documentsubject}
\checkvardefined{\documenttitle}
\checkvardefined{\universitydepartment}
\checkvardefined{\universitydepartmentimagecfg}
\checkvardefined{\universityfaculty}
\checkvardefined{\universitylocation}
\checkvardefined{\universityname}

% -----------------------------------------------------------------------------
% Se añade \xspace a las variables
% -----------------------------------------------------------------------------
\makeatletter
	\g@addto@macro\coursecode\xspace
	\g@addto@macro\coursename\xspace
	\g@addto@macro\documentauthor\xspace
	\g@addto@macro\documentsubject\xspace
	\g@addto@macro\documenttitle\xspace
	\g@addto@macro\universitydepartment\xspace
	\g@addto@macro\universityfaculty\xspace
	\g@addto@macro\universitylocation\xspace
	\g@addto@macro\universityname\xspace
\makeatother

% -----------------------------------------------------------------------------
% Se crean variables si se borraron
% -----------------------------------------------------------------------------
\ifthenelse{\isundefined{\documentsubtitle}}{
	\errmessage{LaTeX Warning: Se borro la variable \noexpand\documentsubtitle, creando una vacia}
	\def\documentsubtitle {}}{
}

\ifthenelse{\equal{\documentsubtitle}{}}{
	\def\documenttitlehf {\documenttitle}
}{
	\def\documenttitlehf {\documentsubtitle}
}

% -----------------------------------------------------------------------------
% Se activan números en menú marcadores del pdf
% -----------------------------------------------------------------------------
\ifthenelse{\equal{\cfgpdfsecnumbookmarks}{true}}{
	\bookmarksetup{numbered}}{
}

% -----------------------------------------------------------------------------
% Se define metadata del pdf
% -----------------------------------------------------------------------------
\ifthenelse{\equal{\cfgshowbookmarkmenu}{true}}{
	\def\cfgpdfpagemode {UseOutlines}
	}{
	\def\cfgpdfpagemode {UseNone}
}
\ifthenelse{\equal{\usepdfmetadata}{true}}{
	\def\pdfmetainfoauthor {\documentauthor}
	\def\pdfmetainfocoursecode {\coursecode}
	\def\pdfmetainfocoursename {\coursename}
	\def\pdfmetainfosubject {\documentsubject}
	\def\pdfmetainfotitle {\documenttitle}
	\def\pdfmetainfouniversity {\universityname}
	\def\pdfmetainfouniversitydepartment {\universitydepartment}
	\def\pdfmetainfouniversityfaculty {\universityfaculty}
	\def\pdfmetainfouniversitylocation {\universitylocation}
	\author{\pdfmetainfoauthor}
	\title{\pdfmetainfotitle}
}{
	\def\pdfmetainfoauthor {}
	\def\pdfmetainfocoursecode {}
	\def\pdfmetainfocoursename {}
	\def\pdfmetainfosubject {}
	\def\pdfmetainfotitle {}
	\def\pdfmetainfouniversity {}
	\def\pdfmetainfouniversitydepartment {}
	\def\pdfmetainfouniversityfaculty {}
	\def\pdfmetainfouniversitylocation {}
}
\hypersetup{
	keeppdfinfo,
	bookmarksopen={\cfgpdfbookmarkopen},
	bookmarksopenlevel={\cfgbookmarksopenlevel},
	bookmarkstype={toc},
	pdfauthor={\pdfmetainfoauthor},
	pdfcenterwindow={\cfgpdfcenterwindow},
	pdfcopyright={\cfgpdfcopyright},
	pdfcreator={LaTeX},
	pdfdisplaydoctitle={\cfgpdfdisplaydoctitle},
	pdfencoding={unicode},
	pdffitwindow={\cfgpdffitwindow},
	pdfinfo={
		Course.Code={\pdfmetainfocoursecode},
		Course.Name={\pdfmetainfocoursename},
		Document.Author={\pdfmetainfoauthor},
		Document.Subject={\pdfmetainfosubject},
		Document.Title={\pdfmetainfotitle},
		Template.Author.Alias={ppizarror},
		Template.Author.Email={pablo@ppizarror.com},
		Template.Author.Web={https://ppizarror.com},
		Template.Author={Pablo Pizarro R.},
		Template.Date={23/08/2024},
		Template.Encoding={UTF-8},
		Template.Latex.Compiler={pdflatex},
		Template.License.Type={MIT},
		Template.License.Web={https://opensource.org/licenses/MIT},
		Template.Name={Template-Tesis},
		Template.Type={Normal},
		Template.Version.Dev={3.4.0-THS},
		Template.Version.Hash={91A7B3E5AA124827D5092536AF382AE4},
		Template.Version.Release={3.4.0},
		Template.Web.Dev={https://github.com/Template-Latex/Template-Tesis},
		Template.Web.Manual={https://latex.ppizarror.com/tesis},
		University.Department={\pdfmetainfouniversitydepartment},
		University.Faculty={\pdfmetainfouniversityfaculty},
		University.Location={\pdfmetainfouniversitylocation},
		University.Name={\pdfmetainfouniversity}
	},
	pdfkeywords={\cfgpdfkeywords},
	pdfmenubar={\cfgpdfmenubar},
	pdfpagelayout={\cfgpdflayout},
	pdfpagemode={\cfgpdfpagemode},
	pdfproducer={Template-Tesis v3.4.0 | (Pablo Pizarro R.) ppizarror.com},
	pdfremotestartview={Fit},
	pdfstartpage={1},
	pdfstartview={\cfgpdfpageview},
	pdfsubject={\pdfmetainfosubject},
	pdftitle={\pdfmetainfotitle},
	pdftoolbar={\cfgpdftoolbar}
}

% -----------------------------------------------------------------------------
% Establece la carpeta de imágenes por defecto
% -----------------------------------------------------------------------------
\graphicspath{{./\defaultimagefolder}}

% -----------------------------------------------------------------------------
% Elimina el espacio vertical de los flotantes
% -----------------------------------------------------------------------------
\makeatletter
\ifthenelse{\equal{\fpremovetopbottomcenter}{true}}{
	\setlength{\@fptop}{0pt}
	\setlength{\@fpbot}{0pt}
}{}
\makeatother

% -----------------------------------------------------------------------------
% Definición de valores e dimensiones
% -----------------------------------------------------------------------------
\setstretch{\documentinterline} % Ajuste del entrelineado
\setlength{\headheight}{64 pt} % Tamaño de la cabecera sin fancyhdr
\setlength{\columnsep}{\columnsepwidth em} % Separación entre columnas
\ifthenelse{\equal{\showlinenumbers}{true}}{
	\setlength{\linenumbersep}{\marginlinenumbers pt}
	\renewcommand\linenumberfont{\normalfont\tiny\color{\linenumbercolor}}
	}{
}

% -----------------------------------------------------------------------------
% Posición inicial de los objetos
% -----------------------------------------------------------------------------
\floatplacement{figure}{\imagedefaultplacement}
\floatplacement{table}{\tabledefaultplacement}
\floatplacement{tikz}{\tikzdefaultplacement}

% -----------------------------------------------------------------------------
% Configuración de los colores
% -----------------------------------------------------------------------------
\color{\maintextcolor} % Color principal
\arrayrulecolor{\tablelinecolor} % Color de las líneas de las tablas
\sethlcolor{\highlightcolor} % Color del subrayado por defecto
\ifthenelse{\equal{\showborderonlinks}{true}}{
	% Color de links con borde
	\hypersetup{
		citebordercolor=\numcitecolor,
		linkbordercolor=\linkcolor,
		urlbordercolor=\urlcolor
	}
}{
	% Color de links sin borde
	\hypersetup{% No reorganizar
		hidelinks,
		colorlinks=true,
		citecolor=\numcitecolor,
		filecolor=\urlcolor,
		linkcolor=\linkcolor,
		urlcolor=\urlcolor
	}
}
\ifthenelse{\equal{\pagescolor}{white}}{}{
	\pagecolor{\pagescolor}
}

% -----------------------------------------------------------------------------
% Configuración de las leyendas
% -----------------------------------------------------------------------------
% Márgenes de las leyendas por defecto
\setcaptionmargincm{\captionlrmargin}
\ifthenelse{\equal{\captiontextbold}{true}}{% Texto en negrita en etiquetas
	\renewcommand{\captiontextbold}{bf}}{
	\renewcommand{\captiontextbold}{}
}
\ifthenelse{\equal{\captiontextsubnumbold}{true}}{% Número en negritas
	\renewcommand{\captiontextsubnumbold}{bf}}{
	\renewcommand{\captiontextsubnumbold}{}
}

% Se configura el texto de los caption
\corecheckfontsize{\captionfontsize}
\captionsetup{
	font={\captionfontsize},
	labelfont={color=\captioncolor, \captiontextbold},
	labelformat={\captionlabelformat},
	labelsep={\captionlabelsep},
	textfont={color=\captiontextcolor},
	singlelinecheck=on
}

% Configura texto de los subcaption
\corecheckfontsize{\subcaptionfsize}
\captionsetup*[subfigure]{
	font={\subcaptionfsize},
	labelfont={color=\captioncolor, \captiontextsubnumbold},
	labelformat={\subcaptionlabelformat},
	labelsep={\subcaptionlabelsep},
	lofdepth=1,
	textfont={color=\captiontextcolor},
	singlelinecheck=on
}
\captionsetup*[subtable]{
	font={\subcaptionfsize},
	labelfont={color=\captioncolor, \captiontextsubnumbold},
	labelformat={\subcaptionlabelformat},
	labelsep={\subcaptionlabelsep},
	lofdepth=1,
	textfont={color=\captiontextcolor},
	singlelinecheck=on
}

\makeatletter
\renewcommand\p@subfigure{\thefigure\captionsubchar}
\renewcommand\p@subtable{\thetable\captionsubchar}
\makeatother

% Configuración de márgenes en las figuras
\floatsetup[figure]{
	captionskip=\captiontbmarginfigure pt
}

% Configuración de márgenes en las tablas
\floatsetup[table]{
	captionskip=\captiontbmargintable pt
}

% Caption superior en figuras
\ifthenelse{\equal{\figurecaptiontop}{true}}{
	\floatsetup[figure]{position=above}}{
}

% Caption superior en tablas
\ifthenelse{\equal{\tablecaptiontop}{true}}{
	\floatsetup[table]{position=top}
	}{
	\floatsetup[table]{position=bottom}
}

% Alineado de leyendas
\ifthenelse{\equal{\captionalignment}{justified}}{% Leyenda justificada
	\captionsetup{
		format=plain,
		justification=justified
	}
}{
\ifthenelse{\equal{\captionalignment}{centered}}{% Leyenda centrada
	\captionsetup{
		justification=centering
	}
}{
\ifthenelse{\equal{\captionalignment}{left}}{% Leyenda alineada a la izquierda
	\captionsetup{
		justification=raggedright,
		singlelinecheck=false
	}
}{
\ifthenelse{\equal{\captionalignment}{right}}{% Leyenda alineada a la derecha
	\captionsetup{
		justification=raggedleft,
		singlelinecheck=false
	}
}{
	\throwbadconfig{Posicion de leyendas desconocida}{\captionalignment}{justified,centered,left,right}}}}
}

% -----------------------------------------------------------------------------
% Configuración de referencias y citas
% -----------------------------------------------------------------------------
\ifthenelse{\equal{\stylecitereferences}{natbib}}{
	\def\twocolumnreferencesmargin {-0.35cm}
	\bibliographystyle{\natbibrefstyle}
	\setlength{\bibsep}{\natbibrefsep pt}
	\newcommand{\shortcite}[1]{\citep{#1}}
	\newcommand{\fullcite}[1]{\citet{#1}}
	% Caracteres citas
	\setcitestyle{open={\natbibrefcitecharopen},close={\natbibrefcitecharclose}}
	% Separador citas
	\ifthenelse{\equal{\natbibrefcitesepcomma}{true}}{
		\setcitestyle{comma}
	}{
		\setcitestyle{semicolon}
	}
	% Tipo citas
	\ifthenelse{\equal{\natbibrefcitetype}{numbers}}{
		\setcitestyle{numbers}
	}{
	\ifthenelse{\equal{\natbibrefcitetype}{authoryear}}{
		\setcitestyle{authoryear}
	}{
	\ifthenelse{\equal{\natbibrefcitetype}{super}}{
		\setcitestyle{super}
	}{
		\throwbadconfig{Tipo cita natbib desconocido}{\natbibrefcitetype}{numbers,authoryear,super}}}
	}
}{
\ifthenelse{\equal{\stylecitereferences}{apacite}}{
	\def\twocolumnreferencesmargin {-0.39cm}
	\bibliographystyle{\apacitestyle}
	\setlength{\bibitemsep}{\apaciterefsep pt}
	\newcommand{\citep}[1]{\fullcite{#1}}
	\newcommand{\citet}[1]{\shortcite{#1}}
}{
\ifthenelse{\equal{\stylecitereferences}{bibtex}}{
	\def\twocolumnreferencesmargin {-0.35cm}
	\bibliographystyle{\bibtexstyle}
	\newlength{\bibitemsep}
	\setlength{\bibitemsep}{.2\baselineskip plus .05\baselineskip minus .05\baselineskip}
	\newlength{\bibparskip}\setlength{\bibparskip}{0pt}
	\ifthenelse{\equal{\bibtexindexbibliography}{true}}{
		\let\oldbibliography\bibliography
		\renewcommand{\bibliography}[1]{
			\clearpage
			\phantomsection
			\addcontentsline{toc}{chapter}{\namereferences} % bibtex tesis en chapter
			\oldbibliography{#1}}}{
	}
	\let\oldthebibliography\thebibliography
	\renewcommand\thebibliography[1]{
		\oldthebibliography{#1}
		\setlength{\parskip}{\bibitemsep}
		\setlength{\itemsep}{\bibparskip}
	}
	\setlength{\bibitemsep}{\bibtexrefsep pt}
}{
\ifthenelse{\equal{\stylecitereferences}{custom}}{
	\coretemplatemessage{Usando estilo citas referencias custom, importar librerias y configuraciones posterior al llamado de template.tex en archivo principal}
}{
	\throwbadconfig{Estilo citas desconocido}{\stylecitereferences}{bibtex,apacite,natbib,custom}}}}
}

% Crea referencias enumeradas en apacite
\makeatletter
\ifthenelse{\equal{\stylecitereferences}{apacite}}{
	\ifthenelse{\equal{\apaciterefnumber}{true}}{
		\newcounter{apaciteNumberCounter}
		\renewcommand{\theapaciteNumberCounter}{% Formato de número
			\apaciterefcitecharopen\arabic{apaciteNumberCounter}\apaciterefcitecharclose
		}
		\patchcmd{\@lbibitem}{\item[}{\item[\stepcounter{apaciteNumberCounter}{\hss\llap{\theapaciteNumberCounter}\quad}}{}{}
		\setlength{\bibleftmargin}{2.54em}
		\setlength{\bibindent}{-0.54em}
	}{}
}{}
\makeatother

% Desactiva la URL de apacite
\ifthenelse{\equal{\stylecitereferences}{apacite}}{
	\ifthenelse{\equal{\apaciteshowurl}{false}}{
		\renewenvironment{APACrefURL}[1][]{}{}
		\AtBeginEnvironment{APACrefURL}{\renewcommand{\url}[1]{}}
		\renewcommand{\doiprefix}{doi:~\kern-1pt}
	}{}
}{}

% Referencias en 2 columnas
\makeatletter
\ifthenelse{\equal{\twocolumnreferences}{true}}{
	\renewenvironment{thebibliography}[1]
	{\begin{multicols}{2}[\chapter*{\refname}]
		\@mkboth{\MakeUppercase\refname}{\MakeUppercase\refname}
		\list{\@biblabel{\@arabic\c@enumiv}}
		{\settowidth\labelwidth{\@biblabel{#1}}
			\leftmargin\labelwidth
			\advance\leftmargin\labelsep
			\@openbib@code
			\usecounter{enumiv}
			\let\p@enumiv\@empty
			\renewcommand\theenumiv{\@arabic\c@enumiv}}
		\sloppy
		\clubpenalty 4000
		\@clubpenalty \clubpenalty
		\widowpenalty 4000
		\sfcode`\.\@m}
		{\def\@noitemerr
		{\@latex@warning{Ambiente `thebibliography' no definido}}
		\endlist\end{multicols}}}{}
\makeatother

% -----------------------------------------------------------------------------
% Configuración anexo
% -----------------------------------------------------------------------------
\patchcmd{\appendices}{\quad}{\charappendixsection\spacingaftersection}{}{}

% -----------------------------------------------------------------------------
% Se añade listings (código fuente) a tocloft
% -----------------------------------------------------------------------------
\begingroup
	\makeatletter
	\let\newcounter\@gobble\let\setcounter\@gobbletwo
	\globaldefs\@ne\let\c@loldepth\@ne
	\newlistof{listings}{lol}{\lstlistlistingname}
	\newlistentry{lstlisting}{lol}{0}
	\makeatother
\endgroup

% -----------------------------------------------------------------------------
% Crea índice de ecuaciones
% -----------------------------------------------------------------------------
\newcommand{\listindexequationsname}{\namelteqn}
\newlistof{myindexequations}{equ}{\listindexequationsname}
\newcommand{\myindexequations}[1]{
	\addcontentsline{equ}{myindexequations}{\protect\numberline{\theequation}#1}
}
\setcounter{templateIndexEquations}{0}
\DeclareTotalCounter{templateIndexEquations}

% -----------------------------------------------------------------------------
% Reconfiguración de tamaño de páginas
% -----------------------------------------------------------------------------
\makeatletter
	\def\ifGm@preamble#1{\@firstofone}
	\appto\restoregeometry{
		\pdfpagewidth=\paperwidth
		\pdfpageheight=\paperheight}
	\apptocmd\newgeometry{
		\pdfpagewidth=\paperwidth
		\pdfpageheight=\paperheight}{}{}
\makeatother

% -----------------------------------------------------------------------------
% Configuración de hbox y vbox
% -----------------------------------------------------------------------------
\hfuzz=200pt
\vfuzz=200pt
\hbadness=\maxdimen
\vbadness=\maxdimen

% -----------------------------------------------------------------------------
% Configura las fuentes
% -----------------------------------------------------------------------------
\makeatletter
\def\Hv@scale {0.95}
\makeatother

% -----------------------------------------------------------------------------
% Configuraciones de las tablas
% -----------------------------------------------------------------------------
\makeatletter % Reinicia el número de cada fila en todas las tablas
\preto\tabular{\global\rownum=\z@}
\preto\tabularx{\global\rownum=\z@}
\makeatother

% -----------------------------------------------------------------------------
% Se activa el word-wrap para textos con \texttt{}
% -----------------------------------------------------------------------------
\ttfamily \hyphenchar\the\font=`\-

% -----------------------------------------------------------------------------
% Se define el tipo de texto de los url
% -----------------------------------------------------------------------------
\urlstyle{\fonturl}

% -----------------------------------------------------------------------------
% Configuraciones del motor de compilación
% -----------------------------------------------------------------------------
\ifthenelse{\equal{\compilertype}{pdf2latex}}{
	% Nivel de compresión
	\pdfcompresslevel=\pdfcompilecompression
	
	% El óptimo es 2, según
	% https://texdoc.org/serve/pdftex-a.pdf/0 p.20
	\pdfdecimaldigits=2
	
	% Inclusión de PDF
	\pdfinclusionerrorlevel=0
	
	% Versión
	\pdfminorversion=\pdfcompileversion
	
	% Compresión de objetos
	\pdfobjcompresslevel=\pdfcompileobjcompression
}{
\ifthenelse{\equal{\compilertype}{xelatex}}{
}{
\ifthenelse{\equal{\compilertype}{lualatex}}{
}{
	\throwbadconfig{Compilador desconocido}{\compilertype}{pdf2latex,xelatex,lualatex}}}
}

% -----------------------------------------------------------------------------
% Crea las sub-sub-sub-secciones
% -----------------------------------------------------------------------------
\newcounter{subsubsubsection}[subsubsection]

% Límite máximo profundidad
\setcounter{secnumdepth}{4}

% Agrega compatibilidad de sub-sub-sub-secciones al TOC
\makeatletter
	\def\toclevel@subsubsubsection {4}
	\def\toclevel@paragraph {5}
	\def\toclevel@subparagraph {6}
	\ifthenelse{\equal{\charaftersectionnum}{}}{% Sin caracter
		\def\l@subsubsubsection {\@dottedtocline{4}{6.97em}{4em}}
		\def\l@paragraph {\@dottedtocline{5}{10.97em}{5em}}
		\def\l@subparagraph {\@dottedtocline{6}{14em}{6em}}
	}{% Posee caracter, Incremento 0.77+3.35 a 3.35
		\def\l@subsubsubsection {\@dottedtocline{4}{7.83em}{4.15em}}
		\def\l@paragraph {\@dottedtocline{5}{11.98em}{4.92em}}
		\def\l@subparagraph {\@dottedtocline{6}{14.65em}{5.69em}}
	}
\makeatother

% -----------------------------------------------------------------------------
% Configura el número de las secciones
% -----------------------------------------------------------------------------
% Funciones de bajo nivel
\makeatletter
\newcommand\sectionpunct[2]{%
	\expandafter\def\csname @seccntfmt@#1\endcsname##1{%
		\csname the##1\endcsname#2%
	}%
}
\def\@seccntformat#1{\@ifundefined{#1@cntformat}%
	{\csname the#1\endcsname} % Default
	{\csname #1@cntformat\endcsname} % Control individual
}
% Configura secciones
\newcommand\section@cntformat{\GLOBALtitlepresectionstr\thesection\charaftersectionnum\spacingaftersection}
\newcommand\subsection@cntformat{\GLOBALtitlepresubsectionstr\thesubsection\charaftersectionnum\spacingaftersection}
\newcommand\subsubsection@cntformat{\GLOBALtitlepresubsubsectionstr\thesubsubsection\charaftersectionnum\spacingaftersection}
\makeatother

% -----------------------------------------------------------------------------
% Actualización margen títulos
% -----------------------------------------------------------------------------
\titlespacing*{\section}{\sectionspacingleft pt}{\sectionspacingtop pt plus 0pt minus 4pt}{\sectionspacingbottom pt plus 0pt minus 2pt}
\titlespacing*{\subsection}{\ssectionspacingleft pt}{\ssectionspacingtop pt plus 0pt minus 2pt}{\ssectionspacingbottom pt plus 0pt minus 2pt}
\titlespacing*{\subsubsection}{\sssectionspacingleft pt}{\sssectionspacingtop pt plus 0pt minus 2pt}{\sssectionspacingbottom pt plus 0pt minus 2pt}
\titlespacing*{\subsubsubsection}{\ssssectionspacingleft pt}{\ssssectionspacingtop pt plus 0pt minus 2pt}{\ssssectionspacingbottom pt plus 0pt minus 2pt}
\chaptertitlefont{\color{\chaptercolor} \chapterfontsize \chapterfontstyle \selectfont}
\makeatletter
\renewcommand\paragraph{\@startsection{paragraph}{5}{\paragspacingleft pt}
	{\paragspacingtop pt \@plus 0pt \@minus 2pt}
	{\paragspacingbottom pt \@plus 0pt \@minus 2pt}
	{\color{\paragcolor}\normalfont\paragfontsize\paragfontstyle}}
\renewcommand\subparagraph{\@startsection{subparagraph}{6}{\paragsubspacingleft pt}
	{\paragsubspacingtop pt \@plus 0pt \@minus 2pt}
	{\paragsubspacingbottom pt \@plus 0pt \@minus 2pt}
	{\color{\paragsubcolor}\normalfont\paragsubfontsize\paragsubfontstyle}}
\makeatother

% -----------------------------------------------------------------------------
% Profundidad del índice y bookmarks pdf
% -----------------------------------------------------------------------------
\setcounter{tocdepth}{\indexdepth}

% -----------------------------------------------------------------------------
% Configuración footnotes
% -----------------------------------------------------------------------------
% Restaura número
\ifthenelse{\equal{\footnoterestart}{none}}{
	\counterwithout*{footnote}{chapter}
}{
\ifthenelse{\equal{\footnoterestart}{sec}}{
	\counterwithin*{footnote}{section}
}{
\ifthenelse{\equal{\footnoterestart}{ssec}}{
	\counterwithin*{footnote}{subsection}
}{
\ifthenelse{\equal{\footnoterestart}{sssec}}{
	\counterwithin*{footnote}{subsubsection}
}{
\ifthenelse{\equal{\footnoterestart}{ssssec}}{
	\counterwithin*{footnote}{subsubsubsection}
}{
\ifthenelse{\equal{\footnoterestart}{page}}{
	\counterwithin*{footnote}{page}
}{
\ifthenelse{\equal{\footnoterestart}{chap}}{
	\counterwithin*{footnote}{chapter}
}{
	\throwbadconfig{Formato reinicio numero footnote desconocido}{\footnoterestart}{none,chap,page,sec,ssec,sssec,ssssec}}}}}}}
}

% Define el tamaño del margen
\setlength{\footnotemargin}{\footnotelmargin pt}

% Previene footnote en otras páginas
\interfootnotelinepenalty=10000

% Configura tablas y figuras
\ifthenelse{\equal{\footnoterulefigure}{false}}{
	\floatsetup[figure]{footnoterule=none}}{
}
\ifthenelse{\equal{\footnoteruletable}{false}}{
	\floatsetup[table]{footnoterule=none}}{
}

% -----------------------------------------------------------------------------
% Restauración número ecuación, NOTA: NO hace nada, sólo se modifica en title.tex
% -----------------------------------------------------------------------------
\ifthenelse{\equal{\equationrestart}{none}}{
}{
\ifthenelse{\equal{\equationrestart}{chap}}{
}{
\ifthenelse{\equal{\equationrestart}{sec}}{
}{
\ifthenelse{\equal{\equationrestart}{ssec}}{
}{
\ifthenelse{\equal{\equationrestart}{sssec}}{
}{
\ifthenelse{\equal{\equationrestart}{ssssec}}{
}{
	\throwbadconfig{Formato reinicio numero ecuacion desconocido}{\equationrestart}{none,chap,sec,ssec,sssec,ssssec}}}}}}
}

% -----------------------------------------------------------------------------
% Configuración elementos matemáticos
% -----------------------------------------------------------------------------
\newtheoremstyle{templatetheorem}{\baselineskip}{3pt}{\itshape}{}{\bfseries}{}{.5em}{}
\newtheoremstyle{templateobs}{\baselineskip}{3pt}{}{}{\bfseries}{}{.5em}{}
\theoremstyle{templatetheorem}

% Configura números
\ifthenelse{\equal{\showsectioncaptionmat}{none}}{
	\newtheorem{defn}{\namemathdefn}
	\newtheorem{teo}{\namemaththeorem}
	\newtheorem{cor}{\namemathcol}
	\newtheorem{lema}{\namemathlem}
	\newtheorem{prop}{\namemathprp}
}{
\ifthenelse{\equal{\showsectioncaptionmat}{chap}}{
	\newtheorem{defn}{\namemathdefn}[chapter]
	\newtheorem{teo}{\namemaththeorem}[chapter]
	\newtheorem{cor}{\namemathcol}[chapter]
	\newtheorem{lema}{\namemathlem}[chapter]
	\newtheorem{prop}{\namemathprp}[chapter]
}{
\ifthenelse{\equal{\showsectioncaptionmat}{sec}}{
	\newtheorem{defn}{\namemathdefn}[section]
	\newtheorem{teo}{\namemaththeorem}[section]
	\newtheorem{cor}{\namemathcol}[section]
	\newtheorem{lema}{\namemathlem}[section]
	\newtheorem{prop}{\namemathprp}[section]
}{
\ifthenelse{\equal{\showsectioncaptionmat}{ssec}}{
	\newtheorem{defn}{\namemathdefn}[subsection]
	\newtheorem{teo}{\namemaththeorem}[subsection]
	\newtheorem{cor}{\namemathcol}[subsection]
	\newtheorem{lema}{\namemathlem}[subsection]
	\newtheorem{prop}{\namemathprp}[subsection]
}{
\ifthenelse{\equal{\showsectioncaptionmat}{sssec}}{
	\newtheorem{defn}{\namemathdefn}[subsubsection]
	\newtheorem{teo}{\namemaththeorem}[subsubsection]
	\newtheorem{cor}{\namemathcol}[subsubsection]
	\newtheorem{lema}{\namemathlem}[subsubsection]
	\newtheorem{prop}{\namemathprp}[subsubsection]
}{
\ifthenelse{\equal{\showsectioncaptionmat}{ssssec}}{
	\newtheorem{defn}{\namemathdefn}[subsubsubsection]
	\newtheorem{teo}{\namemaththeorem}[subsubsubsection]
	\newtheorem{cor}{\namemathcol}[subsubsubsection]
	\newtheorem{lema}{\namemathlem}[subsubsubsection]
	\newtheorem{prop}{\namemathprp}[subsubsubsection]
}{
	\throwbadconfig{Valor configuracion incorrecto}{\showsectioncaptionmat}{none,chap,sec,ssec,sssec,ssssec}}}}}}
}
\theoremstyle{templateobs}
\newtheorem*{ej}{\namemathej}
\newtheorem*{obs}{\namemathobs}

% -----------------------------------------------------------------------------
% Configura el formato oneside/twoside
% -----------------------------------------------------------------------------
% Normaliza el formato de páginas
\raggedbottom

% Desactiva \cleardoublepage hasta el inicio del documento
\let\oldcleardoublepage\cleardoublepage
\let\cleardoublepage\clearpage

% Modifica el formato de nuevas páginas predoc y \cleardoublepage 
\ifthenelse{\equal{\twopagesclearformat}{blank}}{
	\let\emptypagespredocformat\insertblankpage
}{
\ifthenelse{\equal{\twopagesclearformat}{empty}}{
	\let\emptypagespredocformat\insertemptypage
}{
	\throwbadconfig{Valor configuracion incorrecto}{\twopagesclearformat}{blank,empty}}
}

% -----------------------------------------------------------------------------
% Configuraciones del idioma
% -----------------------------------------------------------------------------
% Desactiva caracteres acentuados en operaciones matemáticas
\unaccentedoperators

% -----------------------------------------------------------------------------
% Configura número de objetos en el final del documento
% -----------------------------------------------------------------------------
\AtEndDocument{
	\addtocounter{equation}{\value{templateEquations}}
	\addtocounter{figure}{\value{templateFigures}}
	\addtocounter{lstlisting}{\value{templateListings}}
	\addtocounter{table}{\value{templateTables}}
}

% -----------------------------------------------------------------------------
% Formato de columnas
% -----------------------------------------------------------------------------
% Centrado
\newcolumntype{C}[1]{>{\centering\let\newline\\\arraybackslash\hspace{0pt}}m{#1}}
\newcolumntype{\CColor}[2]{>{\columncolor{#1}\centering\let\newline\\\arraybackslash\hspace{0pt}}m{#2}}

\newcolumntype{P}[1]{>{\centering\let\newline\\\arraybackslash\hspace{0pt}}p{#1}}
\newcolumntype{\PColor}[2]{>{\columncolor{#1}\centering\let\newline\\\arraybackslash\hspace{0pt}}p{#2}}

\newcolumntype{B}[1]{>{\centering\let\newline\\\arraybackslash\hspace{0pt}}b{#1}}
\newcolumntype{\BColor}[2]{>{\columncolor{#1}\centering\let\newline\\\arraybackslash\hspace{0pt}}b{#2}}

% Izquierda
\newcolumntype{L}[1]{>{\raggedright\let\newline\\\arraybackslash\hspace{0pt}}m{#1}}
\newcolumntype{\LColor}[2]{>{\columncolor{#1}\raggedright\let\newline\\\arraybackslash\hspace{0pt}}m{#2}}
\newcolumntype{T}[1]{>{\raggedright\let\newline\\\arraybackslash\hspace{0pt}}p{#1}}
\newcolumntype{\TColor}[2]{>{\columncolor{#1}\raggedright\let\newline\\\arraybackslash\hspace{0pt}}p{#2}}
\newcolumntype{F}[1]{>{\raggedright\let\newline\\\arraybackslash\hspace{0pt}}b{#1}}
\newcolumntype{\FColor}[2]{>{\columncolor{#1}\raggedright\let\newline\\\arraybackslash\hspace{0pt}}b{#2}}

% Derecha
\newcolumntype{R}[1]{>{\raggedleft\let\newline\\\arraybackslash\hspace{0pt}}m{#1}}
\newcolumntype{\RColor}[2]{>{\columncolor{#1}\raggedleft\let\newline\\\arraybackslash\hspace{0pt}}m{#2}}
\newcolumntype{H}[1]{>{\raggedleft\let\newline\\\arraybackslash\hspace{0pt}}p{#1}}
\newcolumntype{\HColor}[2]{>{\columncolor{#1}\raggedleft\let\newline\\\arraybackslash\hspace{0pt}}p{#2}}
\newcolumntype{G}[1]{>{\raggedleft\let\newline\\\arraybackslash\hspace{0pt}}b{#1}}
\newcolumntype{\GColor}[2]{>{\columncolor{#1}\raggedleft\let\newline\\\arraybackslash\hspace{0pt}}b{#2}}

% -----------------------------------------------------------------------------
% Parcha el entorno tablenotes
% -----------------------------------------------------------------------------
\BeforeBeginEnvironment{tablenotes}{%
	\tablenotesfontsize\selectfont%
}
\AfterEndEnvironment{tablenotes}{%
	\normalsize\selectfont%
}

% -----------------------------------------------------------------------------
% Parcha el entorno multicols
% -----------------------------------------------------------------------------
\let\SOURCEcaptionlrmargin\captionlrmargin
\newcounter{multicoldepth}
\setcounter{multicoldepth}{0}
\BeforeBeginEnvironment{multicols}{%
	\def\captionlrmargin {\captionlrmarginmc}%
	\global\def\GLOBALenvmulticol {true}%
	\setcaptionmargincm{\captionlrmargin}%
	\addtocounter{multicoldepth}{1}%
}
\AfterEndEnvironment{multicols}{%
	\def\captionlrmargin {\SOURCEcaptionlrmargin}%
	\setcaptionmargincm{\captionlrmargin}%
	\addtocounter{multicoldepth}{-1}
	\ifnumequal{\number\value{multicoldepth}}{0}{%
		\global\def\GLOBALenvmulticol {false}
	}{}
}

% -----------------------------------------------------------------------------
% Configura estilos de listas
% -----------------------------------------------------------------------------
% Enumerate
\def\labelenumi {\textcolor{\enumerateitemcolor}{\senumerti}}
\def\labelenumii {\textcolor{\enumerateitemcolor}{\senumertii}}
\def\labelenumiii {\textcolor{\enumerateitemcolor}{\senumertiii}}
\def\labelenumiv {\textcolor{\enumerateitemcolor}{\senumertiv}}

% Itemize
\def\labelitemi {\textcolor{\itemizeitemcolor}{\sitemizei}}
\def\labelitemii {\textcolor{\itemizeitemcolor}{\sitemizeii}}
\def\labelitemiii {\textcolor{\itemizeitemcolor}{\sitemizeiii}}
\def\labelitemiv {\textcolor{\itemizeitemcolor}{\sitemizeiv}}

% Márgenes
\setlength\leftmargini{\sitemsmargini pt}
\setlength\leftmarginii{\sitemsmarginii pt}
\setlength\leftmarginiii{\sitemsmarginiii pt}
\setlength\leftmarginiv{\sitemsmarginiv pt}

% -----------------------------------------------------------------------------
% Chequea que ciertos módulos no hayan sido cargados antes del inicio del documento
% -----------------------------------------------------------------------------
\checkmodulenotloaded{tcolorbox}

% -----------------------------------------------------------------------------
% Da soporte a \hl{} del paquete soul
% -----------------------------------------------------------------------------
\soulregister\cite7
\soulregister\eqref7
\soulregister\eqref7
\soulregister\footnote7
\soulregister\href7
\soulregister\pageref7
\soulregister\quotes7
\soulregister\ref7
\soulregister\scite7

% -----------------------------------------------------------------------------
% Configura métodos aplicados al iniciar el documento
% -----------------------------------------------------------------------------
\AtBeginDocument{%
	\normalfont%
	\setlength{\parindent}{\documentparindent pt}%
	\setlength{\parskip}{\documentparskip pt}%
}

% -----------------------------------------------------------------------------
% Estilos de capítulos
% -----------------------------------------------------------------------------
\ifthenelse{\equal{\chapterstyle}{style1}}{
	% Default
}{
\ifthenelse{\equal{\chapterstyle}{style2}}{
	\definecolor{gray75}{gray}{0.75}
	\newcommand{\hsp}{\hspace{20pt}}
	\titleformat{\chapter}[hang]{\Huge\bfseries}{\thechapter\hsp\textcolor{gray75}{|}\hsp}{0pt}{\Huge\bfseries}
}{
\ifthenelse{\equal{\chapterstyle}{style3}}{
	\usepackage[Sonny]{fncychap}
	\ChNameVar{\Large}
	\ChTitleVar{\Large}
}{
\ifthenelse{\equal{\chapterstyle}{style4}}{
	\usepackage[Lenny]{fncychap}
	\ChNameVar{\Large}
	\ChTitleVar{\Large}
}{
\ifthenelse{\equal{\chapterstyle}{style5}}{
	\usepackage[Glenn]{fncychap}
	\ChNameVar{\Large}
	\ChTitleVar{\Large}
}{
\ifthenelse{\equal{\chapterstyle}{style6}}{
	\usepackage[Conny]{fncychap}
}{
\ifthenelse{\equal{\chapterstyle}{style7}}{
	\usepackage[Rejne]{fncychap}
}{
\ifthenelse{\equal{\chapterstyle}{style8}}{
	\usepackage[Bjarne]{fncychap}
}{
\ifthenelse{\equal{\chapterstyle}{style9}}{
	\usepackage[Bjornstrup]{fncychap}
}{
\ifthenelse{\equal{\chapterstyle}{style10}}{
	\titleformat{\chapter}[hang]{\Huge\bfseries}{\thechapter.\hspace{20pt}}{0pt}{\Huge\bfseries}
}{
\ifthenelse{\equal{\chapterstyle}{style11}}{
	\titleformat{\chapter}[hang]{\Huge\bfseries}{\thechapter\hspace{20pt}}{0pt}{\Huge\bfseries}
}{
\ifthenelse{\equal{\chapterstyle}{style12}}{
	\titleformat{\chapter}[hang]{\Huge\bfseries}{}{0pt}{\Huge\bfseries}
}{
	\throwbadconfigondoc{Estilo de capitulo incorrecto}{\chapterstyle}{style1 .. style12}}}}}}}}}}}}
}

% -----------------------------------------------------------------------------
% PORTADA
% -----------------------------------------------------------------------------
\newcommand{\templatePortrait}{%
	
	% Configura la página
	\clearpage
	\def\arraystretch {\tablepaddingv} % Ajusta espaciamiento de las tablas
	\renewcommand{\thepage}{\nameportraitpage}
	\setpagemargincm{\pagemarginleftportrait}{\pagemargintop}{\pagemarginright}{\pagemarginbottom}

	\pagestyle{fancy}
	\fancyhf{}
	\renewcommand{\headrulewidth}{0pt}
	\renewcommand{\footrulewidth}{0pt}

	% Logo de la universidad
	\hspace*{-0.5cm} % Necesario para centrar el logo con el título
	\coreinsertkeyimage{\universitydepartmentimagecfg}{\universitydepartmentimage}%
	\hspace*{0.05cm}%
	\begin{minipage}{0.8\linewidth}%
		\MakeUppercase \universityname ~ \\%
		\MakeUppercase \universityfaculty ~ \\%
		\MakeUppercase \universitydepartment%
		\vspace*{1.6cm}\mbox{}%
	\end{minipage}

	% Importada la portada
	\portrait%
	
	% Ajusta la fuente
	\normalfont%
	
}

% -----------------------------------------------------------------------------
% CONFIGURACIÓN DE PÁGINA Y ENCABEZADOS
% -----------------------------------------------------------------------------
\newcommand{\templatePagecfg}{%
	
	% -------------------------------------------------------------------------
	% Numeración de páginas
	% -------------------------------------------------------------------------
	\clearpage
	\ifthenelse{\equal{\predocpageromannumber}{true}}{% Si se usan números romanos en el pre-documento
		\ifthenelse{\equal{\predocpageromanupper}{true}}{%
			\pagenumbering{Roman}
		}{%
			\pagenumbering{roman}
		}}{%
		\pagenumbering{arabic}
	}
	\setcounter{page}{1}
	\setcounter{footnote}{0}
	
	% -------------------------------------------------------------------------
	% Márgenes de páginas y tablas
	% -------------------------------------------------------------------------
	\setpagemargincm{\pagemarginleft}{\pagemargintop}{\pagemarginright}{\pagemarginbottom}
	\resettablecellpadding
	
	% -------------------------------------------------------------------------
	% Se define el punto decimal
	% -------------------------------------------------------------------------
	\ifthenelse{\equal{\pointdecimal}{true}}{%
		\decimalpoint}{%
	}
	
	% -------------------------------------------------------------------------
	% Definición de nombres de objetos
	% -------------------------------------------------------------------------
	\renewcommand{\abstractname}{\nameabstract} % Nombre del abstract
	\renewcommand{\appendixname}{\nameltappendixsection} % Nombre del anexo (título)
	\renewcommand{\appendixpagename}{\nameappendixsection} % Nombre del anexo en índice
	\renewcommand{\appendixtocname}{\nameappendixsection} % Nombre del anexo en índice
	\renewcommand{\chaptername}{\namechapter}  % Nombre de los capítulos
	\renewcommand{\contentsname}{\nameltcont} % Nombre del índice
	\renewcommand{\figurename}{\nameltwfigure} % Nombre de la leyenda de las fig.
	\renewcommand{\listfigurename}{\nameltfigure} % Nombre del índice de figuras
	\renewcommand{\listtablename}{\namelttable} % Nombre del índice de tablas
	\renewcommand{\lstlistingname}{\nameltwsrc} % Nombre leyenda del código fuente
	\renewcommand{\lstlistlistingname}{\nameltsrc} % Nombre índice código fuente
	\renewcommand{\refname}{\namereferences} % Nombre de las referencias (bibtex)
	\renewcommand{\bibname}{\namereferences} % Nombre de las referencias (natbib)
	\renewcommand{\tablename}{\nameltwtable} % Nombre de la leyenda de tablas
	
	% -------------------------------------------------------------------------
	% Estilo de títulos
	% -------------------------------------------------------------------------
	\sectionfont{%
		\color{\sectioncolor} \sectionfontsize \sectionfontstyle \selectfont%
	}
	\subsectionfont{%
		\color{\ssectioncolor} \ssectionfontsize \ssectionfontstyle \selectfont%
	}
	\subsubsectionfont{%
		\color{\sssectioncolor} \sssectionfontsize \sssectionfontstyle \selectfont%
	}
	\titleformat{\subsubsubsection}{%
		\color{\ssssectioncolor} \ssssectionfontsz \ssssectionfontstyle%
	}{%
		\GLOBALtitlepresubsubsubsectionstr\thesubsubsubsection\charaftersectionnum\spacingaftersection%
		\corepatchaftersubsubsubsection%
	}{0em}{%
	}
	\def\bibfont {\fontsizerefbibl} % Tamaño de fuente de las referencias
	
	% -------------------------------------------------------------------------
	% Estilo citas
	% -------------------------------------------------------------------------
	\ifthenelse{\equal{\stylecitereferences}{apacite}}{%
		\renewcommand{\BOthers}[1]{\apacitebothers\hbox{}}%
	}{}
	
	% -------------------------------------------------------------------------
	% Se crean los header-footer
	% -------------------------------------------------------------------------
	\fancyheadoffset{0pt} % Desactiva el offset de los header-footer
	\def\hfheaderimageparamsA {height=\baselineskip} % Tamaño de las imágenes del encabezado estilo 3/13
	\ifthenelse{\equal{\hfstyle}{style1}}{%
		\pagestyle{fancy}
		\newcommand{\COREstyledefinition}{%
			\fancyhf{}
			\ifthenelse{\equal{\disablehfrightmark}{false}}{%
				\fancyhead[L]{\nouppercase{\rightmark}}
			}{}
			\fancyhead[R]{\small \thepage}
			\ifthenelse{\equal{\hfwidthwrap}{true}}{%
				\fancyfoot[L]{%
					\begin{minipage}[t]{\hfwidthtitle\linewidth}
						\begin{flushleft}
							\small \textit{\documenttitlehf}
						\end{flushleft}
					\end{minipage}
				}
				\fancyfoot[R]{%
					\begin{minipage}[t]{\hfwidthcourse\linewidth}
						\begin{flushright}
							\small \textit{\coursecode \coursename}
						\end{flushright}
					\end{minipage}
				}
			}{%
				\fancyfoot[L]{\small \textit{\documenttitlehf}}
				\fancyfoot[R]{\small \textit{\coursecode \coursename}}
			}
			\renewcommand{\headrulewidth}{0.5pt}
			\renewcommand{\footrulewidth}{0.5pt}
		}
		\renewcommand{\sectionmark}[1]{\markboth{##1}{}}
		\COREstyledefinition
	}{%
	\ifthenelse{\equal{\hfstyle}{style1-i}}{% Impar izquierdo
		\pagestyle{fancy}
		\newcommand{\COREstyledefinition}{%
			\fancyhf{}
			\ifthenelse{\equal{\disablehfrightmark}{false}}{%
				\fancyhead[LE,RO]{\nouppercase{\rightmark}}
			}{}
			\fancyhead[RE,LO]{\small \thepage}
			\ifthenelse{\equal{\hfwidthwrap}{true}}{%
				\fancyfoot[L]{%
					\begin{minipage}[t]{\hfwidthtitle\linewidth}
						\begin{flushleft}
							\small \textit{\documenttitlehf}
						\end{flushleft}
					\end{minipage}
				}
				\fancyfoot[R]{%
					\begin{minipage}[t]{\hfwidthcourse\linewidth}
						\begin{flushright}
							\small \textit{\coursecode \coursename}
						\end{flushright}
					\end{minipage}
				}
			}{%
				\fancyfoot[L]{\small \textit{\documenttitlehf}}
				\fancyfoot[R]{\small \textit{\coursecode \coursename}}
			}
			\renewcommand{\headrulewidth}{0.5pt}
			\renewcommand{\footrulewidth}{0.5pt}
		}
		\renewcommand{\sectionmark}[1]{\markboth{##1}{}}
		\COREstyledefinition
	}{%
	\ifthenelse{\equal{\hfstyle}{style1-d}}{% Impar derecho
		\pagestyle{fancy}
		\newcommand{\COREstyledefinition}{%
			\fancyhf{}
			\ifthenelse{\equal{\disablehfrightmark}{false}}{%
				\fancyhead[LO,RE]{\nouppercase{\rightmark}}
			}{}
			\fancyhead[RO,LE]{\small \thepage}
			\ifthenelse{\equal{\hfwidthwrap}{true}}{%
				\fancyfoot[L]{%
					\begin{minipage}[t]{\hfwidthtitle\linewidth}
						\begin{flushleft}
							\small \textit{\documenttitlehf}
						\end{flushleft}
					\end{minipage}
				}
				\fancyfoot[R]{%
					\begin{minipage}[t]{\hfwidthcourse\linewidth}
						\begin{flushright}
							\small \textit{\coursecode \coursename}
						\end{flushright}
					\end{minipage}
				}
			}{%
				\fancyfoot[L]{\small \textit{\documenttitlehf}}
				\fancyfoot[R]{\small \textit{\coursecode \coursename}}
			}
			\renewcommand{\headrulewidth}{0.5pt}
			\renewcommand{\footrulewidth}{0.5pt}
		}
		\renewcommand{\sectionmark}[1]{\markboth{##1}{}}
		\COREstyledefinition
	}{%
	\ifthenelse{\equal{\hfstyle}{style2}}{%
		\pagestyle{fancy}
		\newcommand{\COREstyledefinition}{%
			\fancyhf{}
			\ifthenelse{\equal{\disablehfrightmark}{false}}{%
				\fancyhead[L]{\nouppercase{\rightmark}}
			}{}
			\fancyhead[R]{\small \thepage}
			\ifthenelse{\equal{\hfwidthwrap}{true}}{%
				\fancyfoot[L]{%
					\begin{minipage}[t]{\hfwidthtitle\linewidth}
						\begin{flushleft}
							\small \textit{\documenttitlehf}
						\end{flushleft}
					\end{minipage}
				}
				\fancyfoot[R]{%
					\begin{minipage}[t]{\hfwidthcourse\linewidth}
						\begin{flushright}
							\small \textit{\coursecode \coursename}
						\end{flushright}
					\end{minipage}
				}
			}{%
				\fancyfoot[L]{\small \textit{\documenttitlehf}}
				\fancyfoot[R]{\small \textit{\coursecode \coursename}}
			}
			\renewcommand{\headrulewidth}{0.5pt}
			\renewcommand{\footrulewidth}{0pt}
		}
		\renewcommand{\sectionmark}[1]{\markboth{##1}{}}
		\COREstyledefinition
	}{%
	\ifthenelse{\equal{\hfstyle}{style2-i}}{% Impar izquierdo
		\pagestyle{fancy}
		\newcommand{\COREstyledefinition}{%
			\fancyhf{}
			\ifthenelse{\equal{\disablehfrightmark}{false}}{%
				\fancyhead[LE,RO]{\nouppercase{\rightmark}}
			}{}
			\fancyhead[RE,LO]{\small \thepage}
			\ifthenelse{\equal{\hfwidthwrap}{true}}{%
				\fancyfoot[L]{%
					\begin{minipage}[t]{\hfwidthtitle\linewidth}
						\begin{flushleft}
							\small \textit{\documenttitlehf}
						\end{flushleft}
					\end{minipage}
				}
				\fancyfoot[R]{%
					\begin{minipage}[t]{\hfwidthcourse\linewidth}
						\begin{flushright}
							\small \textit{\coursecode \coursename}
						\end{flushright}
					\end{minipage}
				}
			}{%
				\fancyfoot[L]{\small \textit{\documenttitlehf}}
				\fancyfoot[R]{\small \textit{\coursecode \coursename}}
			}
			\renewcommand{\headrulewidth}{0.5pt}
			\renewcommand{\footrulewidth}{0pt}
		}
		\renewcommand{\sectionmark}[1]{\markboth{##1}{}}
		\COREstyledefinition
	}{%
	\ifthenelse{\equal{\hfstyle}{style1-d}}{% Impar derecho
		\pagestyle{fancy}
		\newcommand{\COREstyledefinition}{%
			\fancyhf{}
			\ifthenelse{\equal{\disablehfrightmark}{false}}{%
				\fancyhead[LO,RE]{\nouppercase{\rightmark}}
			}{}
			\fancyhead[RO,LE]{\small \thepage}
			\ifthenelse{\equal{\hfwidthwrap}{true}}{%
				\fancyfoot[L]{%
					\begin{minipage}[t]{\hfwidthtitle\linewidth}
						\begin{flushleft}
							\small \textit{\documenttitlehf}
						\end{flushleft}
					\end{minipage}
				}
				\fancyfoot[R]{%
					\begin{minipage}[t]{\hfwidthcourse\linewidth}
						\begin{flushright}
							\small \textit{\coursecode \coursename}
						\end{flushright}
					\end{minipage}
				}
			}{%
				\fancyfoot[L]{\small \textit{\documenttitlehf}}
				\fancyfoot[R]{\small \textit{\coursecode \coursename}}
			}
			\renewcommand{\headrulewidth}{0.5pt}
			\renewcommand{\footrulewidth}{0pt}
		}
		\renewcommand{\sectionmark}[1]{\markboth{##1}{}}
		\COREstyledefinition
	}{%
	\ifthenelse{\equal{\hfstyle}{style3}}{%
		\pagestyle{fancy}
		\newcommand{\COREstyledefinition}{%
			\fancyhf{}
			\ifthenelse{\equal{\hfwidthwrap}{true}}{%
				\fancyhead[L]{%
					\begin{minipage}[t]{\hfwidthtitle\linewidth}
						\begin{flushleft}
							\small \textit{\coursecode \coursename}
						\end{flushleft}
					\end{minipage}
				}
			}{%
				\fancyhead[L]{\small \textit{\coursecode \coursename}}
			}
			\fancyhead[R]{%
				\coreinsertkeyimage{\hfheaderimageparamsA}{\universitydepartmentimage}%
				\vspace{-0.15cm}%
			}
			\fancyfoot[C]{\thepage}
			\renewcommand{\headrulewidth}{0.5pt}
			\renewcommand{\footrulewidth}{0pt}
		}
		\COREstyledefinition
	}{%
	\ifthenelse{\equal{\hfstyle}{style4}}{%
		\pagestyle{fancy}
		\newcommand{\COREstyledefinition}{%
			\fancyhf{}
			\ifthenelse{\equal{\disablehfrightmark}{false}}{%
				\fancyhead[L]{\nouppercase{\rightmark}}
			}{}
			\fancyhead[R]{}
			\fancyfoot[C]{\small \thepage}
			\renewcommand{\headrulewidth}{0.5pt}
			\renewcommand{\footrulewidth}{0pt}
		}
		\renewcommand{\sectionmark}[1]{\markboth{##1}{}}
		\COREstyledefinition
	}{%
	\ifthenelse{\equal{\hfstyle}{style5}}{%
		\pagestyle{fancy}
		\newcommand{\COREstyledefinition}{%
			\fancyhf{}
			\ifthenelse{\equal{\hfwidthwrap}{true}}{%
				\fancyhead[L]{%
					\begin{minipage}[t]{\hfwidthcourse\linewidth}
						\begin{flushleft}
							\coursecode \coursename
						\end{flushleft}
					\end{minipage}
				}
				\ifthenelse{\equal{\disablehfrightmark}{false}}{%
					\fancyhead[R]{%
						\begin{minipage}[t]{\hfwidthtitle\linewidth}
							\begin{flushright}
								\nouppercase{\rightmark}
							\end{flushright}
						\end{minipage}
					}
				}{}
			}{%
				\fancyhead[L]{\coursecode \coursename}
				\ifthenelse{\equal{\disablehfrightmark}{false}}{%
					\fancyhead[R]{\nouppercase{\rightmark}}
				}{}
			}
			\fancyfoot[L]{\universitydepartment, \universityname}
			\fancyfoot[R]{\small \thepage}
			\renewcommand{\headrulewidth}{0pt}
			\renewcommand{\footrulewidth}{0pt}
		}
		\renewcommand{\sectionmark}[1]{\markboth{##1}{}}
		\COREstyledefinition
	}{%
	\ifthenelse{\equal{\hfstyle}{style5-d}}{% Impar derecho
		\pagestyle{fancy}
		\newcommand{\COREstyledefinition}{%
			\fancyhf{}
			\ifthenelse{\equal{\hfwidthwrap}{true}}{%
				\fancyhead[L]{%
					\begin{minipage}[t]{\hfwidthcourse\linewidth}
						\begin{flushleft}
							\coursecode \coursename
						\end{flushleft}
					\end{minipage}
				}
				\ifthenelse{\equal{\disablehfrightmark}{false}}{%
					\fancyhead[R]{%
						\begin{minipage}[t]{\hfwidthtitle\linewidth}
							\begin{flushright}
								\nouppercase{\rightmark}
							\end{flushright}
						\end{minipage}
					}
				}{}
			}{%
				\fancyhead[L]{\coursecode \coursename}
				\ifthenelse{\equal{\disablehfrightmark}{false}}{%
					\fancyhead[R]{\nouppercase{\rightmark}}
				}{}
			}
			\fancyfoot[LO,RE]{\universitydepartment, \universityname}
			\fancyfoot[RO,LE]{\small \thepage}
			\renewcommand{\headrulewidth}{0pt}
			\renewcommand{\footrulewidth}{0pt}
		}
		\renewcommand{\sectionmark}[1]{\markboth{##1}{}}
		\COREstyledefinition
	}{%
	\ifthenelse{\equal{\hfstyle}{style5-i}}{% Impar izquierdo
		\pagestyle{fancy}
		\newcommand{\COREstyledefinition}{%
			\fancyhf{}
			\ifthenelse{\equal{\hfwidthwrap}{true}}{%
				\fancyhead[L]{%
					\begin{minipage}[t]{\hfwidthcourse\linewidth}
						\begin{flushleft}
							\coursecode \coursename
						\end{flushleft}
					\end{minipage}
				}
				\ifthenelse{\equal{\disablehfrightmark}{false}}{%
					\fancyhead[R]{%
						\begin{minipage}[t]{\hfwidthtitle\linewidth}
							\begin{flushright}
								\nouppercase{\rightmark}
							\end{flushright}
						\end{minipage}
					}
				}{}
			}{%
				\fancyhead[L]{\coursecode \coursename}
				\ifthenelse{\equal{\disablehfrightmark}{false}}{%
					\fancyhead[R]{\nouppercase{\rightmark}}
				}{}
			}
			\fancyfoot[LE,RO]{\universitydepartment, \universityname}
			\fancyfoot[RE,LO]{\small \thepage}
			\renewcommand{\headrulewidth}{0pt}
			\renewcommand{\footrulewidth}{0pt}
		}
		\renewcommand{\sectionmark}[1]{\markboth{##1}{}}
		\COREstyledefinition
	}{%
	\ifthenelse{\equal{\hfstyle}{style6}}{%
		\pagestyle{fancy}
		\newcommand{\COREstyledefinition}{%
			\fancyhf{}
			\fancyfoot[L]{\universitydepartment}
			\fancyfoot[C]{\thepage}
			\fancyfoot[R]{\universityname}
			\renewcommand{\headrulewidth}{0pt}
			\renewcommand{\footrulewidth}{0pt}
		}
		\setlength{\headheight}{49pt}
		\COREstyledefinition
	}{%
	\ifthenelse{\equal{\hfstyle}{style7}}{%
		\pagestyle{fancy}
		\newcommand{\COREstyledefinition}{%
			\fancyhf{}
			\fancyfoot[C]{\thepage}
			\renewcommand{\headrulewidth}{0pt}
			\renewcommand{\footrulewidth}{0pt}
		}
		\setlength{\headheight}{49pt}
		\COREstyledefinition
	}{%
	\ifthenelse{\equal{\hfstyle}{style8}}{%
		\pagestyle{fancy}
		\newcommand{\COREstyledefinition}{%
			\fancyhf{}
			\fancyfoot[R]{\thepage}
			\renewcommand{\headrulewidth}{0pt}
			\renewcommand{\footrulewidth}{0pt}
		}
		\setlength{\headheight}{49pt}
		\COREstyledefinition
	}{%
	\ifthenelse{\equal{\hfstyle}{style8-d}}{% Impar derecho
		\pagestyle{fancy}
		\newcommand{\COREstyledefinition}{%
			\fancyhf{}
			\fancyfoot[RO,LE]{\thepage}
			\renewcommand{\headrulewidth}{0pt}
			\renewcommand{\footrulewidth}{0pt}
		}
		\setlength{\headheight}{49pt}
		\COREstyledefinition
	}{%
	\ifthenelse{\equal{\hfstyle}{style8-i}}{% Impar izquierdo
		\pagestyle{fancy}
		\newcommand{\COREstyledefinition}{%
			\fancyhf{}
			\fancyfoot[RE,LO]{\thepage}
			\renewcommand{\headrulewidth}{0pt}
			\renewcommand{\footrulewidth}{0pt}
		}
		\setlength{\headheight}{49pt}
		\COREstyledefinition
	}{%
	\ifthenelse{\equal{\hfstyle}{style9}}{%
		\pagestyle{fancy}
		\newcommand{\COREstyledefinition}{%
			\fancyhf{}
			\ifthenelse{\equal{\disablehfrightmark}{false}}{%
				\fancyhead[L]{\nouppercase{\rightmark}}
			}{}
			\fancyhead[R]{}
			\fancyfoot[L]{\small \textit{\documenttitlehf}}
			\fancyfoot[R]{\small \thepage}
			\renewcommand{\headrulewidth}{0.5pt}
			\renewcommand{\footrulewidth}{0.5pt}
		}
		\renewcommand{\sectionmark}[1]{\markboth{##1}{}}
		\COREstyledefinition
	}{%
	\ifthenelse{\equal{\hfstyle}{style9-d}}{% Impar derecho
		\pagestyle{fancy}
		\newcommand{\COREstyledefinition}{%
			\fancyhf{}
			\ifthenelse{\equal{\disablehfrightmark}{false}}{%
				\fancyhead[L]{\nouppercase{\rightmark}}
			}{}
			\fancyhead[R]{}
			\fancyfoot[RE,LO]{\small \textit{\documenttitlehf}}
			\fancyfoot[RO,LE]{\small \thepage}
			\renewcommand{\headrulewidth}{0.5pt}
			\renewcommand{\footrulewidth}{0.5pt}
		}
		\renewcommand{\sectionmark}[1]{\markboth{##1}{}}
		\COREstyledefinition
	}{%
	\ifthenelse{\equal{\hfstyle}{style9-i}}{% Impar izquierdo
		\pagestyle{fancy}
		\newcommand{\COREstyledefinition}{%
			\fancyhf{}
			\ifthenelse{\equal{\disablehfrightmark}{false}}{%
				\fancyhead[L]{\nouppercase{\rightmark}}
			}{}
			\fancyhead[R]{}
			\fancyfoot[RO,LE]{\small \textit{\documenttitlehf}}
			\fancyfoot[RE,LO]{\small \thepage}
			\renewcommand{\headrulewidth}{0.5pt}
			\renewcommand{\footrulewidth}{0.5pt}
		}
		\renewcommand{\sectionmark}[1]{\markboth{##1}{}}
		\COREstyledefinition
	}{%
	\ifthenelse{\equal{\hfstyle}{style10}}{%
		\pagestyle{fancy}
		\newcommand{\COREstyledefinition}{%
			\fancyhf{}
			\ifthenelse{\equal{\hfwidthwrap}{true}}{%
				\ifthenelse{\equal{\disablehfrightmark}{false}}{%
					\fancyhead[L]{%
						\begin{minipage}[t]{\hfwidthtitle\linewidth}
							\begin{flushleft}
								\nouppercase{\rightmark}
							\end{flushleft}
						\end{minipage}
					}
				}{}
				\fancyhead[R]{%
					\begin{minipage}[t]{\hfwidthcourse\linewidth}
						\begin{flushright}
							\small \textit{\documenttitlehf}
						\end{flushright}
					\end{minipage}
				}
			}{%
				\ifthenelse{\equal{\disablehfrightmark}{false}}{%
					\fancyhead[L]{\nouppercase{\rightmark}}
				}{}
				\fancyhead[R]{\small \textit{\documenttitlehf}}
			}
			\fancyfoot[L]{}
			\fancyfoot[R]{\small \thepage}
			\renewcommand{\headrulewidth}{0.5pt}
			\renewcommand{\footrulewidth}{0.5pt}
		}
		\renewcommand{\sectionmark}[1]{\markboth{##1}{}}
		\COREstyledefinition
	}{%
	\ifthenelse{\equal{\hfstyle}{style10-i}}{% Impar izquierdo
		\pagestyle{fancy}
		\newcommand{\COREstyledefinition}{%
			\fancyhf{}
			\ifthenelse{\equal{\hfwidthwrap}{true}}{%
				\ifthenelse{\equal{\disablehfrightmark}{false}}{%
					\fancyhead[L]{%
						\begin{minipage}[t]{\hfwidthtitle\linewidth}
							\begin{flushleft}
								\nouppercase{\rightmark}
							\end{flushleft}
						\end{minipage}
					}
				}{}
				\fancyhead[R]{%
					\begin{minipage}[t]{\hfwidthcourse\linewidth}
						\begin{flushright}
							\small \textit{\documenttitlehf}
						\end{flushright}
					\end{minipage}
				}
			}{%
				\ifthenelse{\equal{\disablehfrightmark}{false}}{%
					\fancyhead[L]{\nouppercase{\rightmark}}
				}{}
				\fancyhead[R]{\small \textit{\documenttitlehf}}
			}
			\fancyfoot[L]{}
			\fancyfoot[RE,LO]{\small \thepage}
			\renewcommand{\headrulewidth}{0.5pt}
			\renewcommand{\footrulewidth}{0.5pt}
		}
		\renewcommand{\sectionmark}[1]{\markboth{##1}{}}
		\COREstyledefinition
	}{%
	\ifthenelse{\equal{\hfstyle}{style10-d}}{% Impar derecho
		\pagestyle{fancy}
		\newcommand{\COREstyledefinition}{%
			\fancyhf{}
			\ifthenelse{\equal{\hfwidthwrap}{true}}{%
				\ifthenelse{\equal{\disablehfrightmark}{false}}{%
					\fancyhead[L]{%
						\begin{minipage}[t]{\hfwidthtitle\linewidth}
							\begin{flushleft}
								\nouppercase{\rightmark}
							\end{flushleft}
						\end{minipage}
					}
				}{}
				\fancyhead[R]{%
					\begin{minipage}[t]{\hfwidthcourse\linewidth}
						\begin{flushright}
							\small \textit{\documenttitlehf}
						\end{flushright}
					\end{minipage}
				}
			}{%
				\ifthenelse{\equal{\disablehfrightmark}{false}}{%
					\fancyhead[L]{\nouppercase{\rightmark}}
				}{}
				\fancyhead[R]{\small \textit{\documenttitlehf}}
			}
			\fancyfoot[L]{}
			\fancyfoot[LE,RO]{\small \thepage}
			\renewcommand{\headrulewidth}{0.5pt}
			\renewcommand{\footrulewidth}{0.5pt}
		}
		\renewcommand{\sectionmark}[1]{\markboth{##1}{}}
		\COREstyledefinition
	}{%
	\ifthenelse{\equal{\hfstyle}{style11}}{% Similar a 1
		\pagestyle{fancy}
		\newcommand{\COREstyledefinition}{%
			\fancyhf{}
			\ifthenelse{\equal{\disablehfrightmark}{false}}{%
				\fancyhead[L]{\nouppercase{\rightmark}}
			}{}
			\fancyhead[R]{\small \thepage \namepageof \pageref{TotPages}}
			\ifthenelse{\equal{\hfwidthwrap}{true}}{%
				\fancyfoot[L]{%
					\begin{minipage}[t]{\hfwidthtitle\linewidth}
						\begin{flushleft}
							\small \textit{\documenttitlehf}
						\end{flushleft}
					\end{minipage}
				}
				\fancyfoot[R]{%
					\begin{minipage}[t]{\hfwidthcourse\linewidth}
						\begin{flushright}
							\small \textit{\coursecode \coursename}
						\end{flushright}
					\end{minipage}
				}
			}{%
				\fancyfoot[L]{\small \textit{\documenttitlehf}}
				\fancyfoot[R]{\small \textit{\coursecode \coursename}}
			}
			\renewcommand{\headrulewidth}{0.5pt}
			\renewcommand{\footrulewidth}{0.5pt}
		}
		\renewcommand{\sectionmark}[1]{\markboth{##1}{}}
		\COREstyledefinition
	}{%
	\ifthenelse{\equal{\hfstyle}{style12}}{% Similar a 6
		\pagestyle{fancy}
		\newcommand{\COREstyledefinition}{%
			\fancyhf{}
			\fancyfoot[L]{\universitydepartment}
			\fancyfoot[C]{\thepage \namepageof \pageref{TotPages}}
			\fancyfoot[R]{\universityname}
			\renewcommand{\headrulewidth}{0pt}
			\renewcommand{\footrulewidth}{0pt}
		}
		\setlength{\headheight}{49pt}
		\COREstyledefinition
	}{%
	\ifthenelse{\equal{\hfstyle}{style13}}{% Similar a 3
		\pagestyle{fancy}
		\newcommand{\COREstyledefinition}{%
			\fancyhf{}
			\ifthenelse{\equal{\hfwidthwrap}{true}}{%
				\fancyhead[L]{%
					\begin{minipage}[t]{\hfwidthtitle\linewidth}
						\begin{flushleft}
							\small \textit{\coursecode \coursename}
						\end{flushleft}
					\end{minipage}
				}
			}{%
				\fancyhead[L]{\small \textit{\coursecode \coursename}}
			}
			\fancyhead[R]{%
				\coreinsertkeyimage{\hfheaderimageparamsA}{\universitydepartmentimage}%
				\vspace{-0.15cm}%
			}
			\fancyfoot[C]{\thepage \namepageof \pageref{TotPages}}
			\renewcommand{\headrulewidth}{0.5pt}
			\renewcommand{\footrulewidth}{0pt}
		}
		\COREstyledefinition
	}{%
	\ifthenelse{\equal{\hfstyle}{style14}}{% Similar a 4
		\pagestyle{fancy}
		\newcommand{\COREstyledefinition}{%
			\fancyhf{}
			\ifthenelse{\equal{\disablehfrightmark}{false}}{%
				\fancyhead[L]{\nouppercase{\rightmark}}
			}{}
			\fancyhead[R]{}
			\fancyfoot[C]{\small \thepage \namepageof \pageref{TotPages}}
			\renewcommand{\headrulewidth}{0.5pt}
			\renewcommand{\footrulewidth}{0pt}
		}
		\renewcommand{\sectionmark}[1]{\markboth{##1}{}}
		\COREstyledefinition
	}{%
	\ifthenelse{\equal{\hfstyle}{style15}}{% Similar a 1
		\pagestyle{fancy}
		\newcommand{\COREstyledefinition}{%
			\fancyhf{}
			\ifthenelse{\equal{\disablehfrightmark}{false}}{%
				\fancyhead[L]{\nouppercase{\rightmark}}
			}{}
			\fancyhead[R]{}
			\fancyfoot[L]{%
				\small \coursecode \coursename
			}
			\fancyfoot[R]{%
				\small \thepage
			}
			\renewcommand{\headrulewidth}{0.5pt}
			\renewcommand{\footrulewidth}{0.5pt}
		}
		\renewcommand{\sectionmark}[1]{\markboth{##1}{}}
		\COREstyledefinition
	}{%
	\ifthenelse{\equal{\hfstyle}{style16}}{%
		\pagestyle{fancy}
		\newcommand{\COREstyledefinition}{%
			\fancyhf{}
			\renewcommand{\headrulewidth}{0pt}
			\renewcommand{\footrulewidth}{0pt}
		}
		\renewcommand{\sectionmark}[1]{\markboth{##1}{}}
		\COREstyledefinition
	}{%
		\throwbadconfigondoc{Estilo de header-footer incorrecto}{\hfstyle}{style1 .. style16}}}}}}}}}}}}}}}}}}}}}}}}}}}}
	}
	% Aplica el estilo de página
	\fancypagestyle{plain}{%
		\fancyheadoffset{0pt}
		\COREstyledefinition
	}
	% Define estilos por defecto en flotantes
	\floatpagestyle{plain}
	\rotfloatpagestyle{plain}
	
	% -------------------------------------------------------------------------
	% Muestra los números de línea
	% -------------------------------------------------------------------------
	\ifthenelse{\equal{\showlinenumbers}{true}}{%
		\linenumbers}{%
	}
	% Añade página en blanco
	\ifthenelse{\equal{\GLOBALtwoside}{true}}{%
		\insertemptypage%
		\addtocounter{page}{-1}}{%
	}
}

% -----------------------------------------------------------------------------
% TABLA DE CONTENIDOS - ÍNDICE
% -----------------------------------------------------------------------------
\newcommand{\templateIndex}{%

	% -------------------------------------------------------------------------
	% Inicio índice, desactiva espacio entre objetos
	% -------------------------------------------------------------------------
	\ifthenelse{\equal{\objectchaptermargin}{false}}{%
		\let\origaddvspace\addvspace
		\renewcommand{\addvspace}[1]{}
	}{}
	
	% -------------------------------------------------------------------------
	% Crea nueva página y establece estilo de títulos
	% -------------------------------------------------------------------------
	\clearpage%
	\begingroup%
	\sectionfont{\color{\indextitlecolor} \indexsectionfontsize \indexsectionstyle \selectfont}
	
	% -------------------------------------------------------------------------
	% Salta de página si está imprimiendo por ambas caras
	% -------------------------------------------------------------------------
	\ifthenelse{\equal{\GLOBALtwoside}{true}}{%
		\coretriggeronpage{\emptypagespredocformat}{}%
	}{}
	
	% -------------------------------------------------------------------------
	% Añade la entrada del índice a los marcadores del pdf
	% -------------------------------------------------------------------------
	\ifthenelse{\equal{\addindextobookmarks}{true}}{%
		\phantomsection
		\pdfbookmark{\nameltcont}{contents}}{%
	}
	\tocloftpagestyle{fancy}
	
	% -------------------------------------------------------------------------
	% Configuración del punto en índice
	% -------------------------------------------------------------------------
	\def\cftchapaftersnum {\charaftersectionnum}
	\def\cftsecaftersnum {\charaftersectionnum}
	\def\cftsubsecaftersnum {\charaftersectionnum}
	\def\cftsubsubsecaftersnum {\charaftersectionnum}
	\def\cftsubsubsubsecaftersnum {\charaftersectionnum}
	
	% -------------------------------------------------------------------------
	% Configuración carácter número de página
	% -------------------------------------------------------------------------
	\renewcommand{\cftdot}{\charnumpageindex}
	
	% -------------------------------------------------------------------------
	% Configuración del punto en número de objetos
	% -------------------------------------------------------------------------
	\def\cftfigaftersnum {\charafterobjectindex\enspace} % Figuras
	\def\cftsubfigaftersnum {\charafterobjectindex\enspace} % Subfiguras
	\def\cfttabaftersnum {\charafterobjectindex\enspace} % Tablas
	\def\cftlstlistingaftersnum {\charafterobjectindex\enspace} % Códigos fuente
	\def\cftmyindexequationsaftersnum {\charafterobjectindex\enspace} % Ecuaciones
	
	% -------------------------------------------------------------------------
	% Desactiva los números de línea
	% -------------------------------------------------------------------------
	\ifthenelse{\equal{\showlinenumbers}{true}}{%
		\nolinenumbers}{%
	}
	
	% -------------------------------------------------------------------------
	% Cambia tabulación índice de objetos para alinear con contenidos
	% -------------------------------------------------------------------------
	\ifthenelse{\equal{\objectindexindent}{true}}{%
		\setlength{\cfttabindent}{1.5em}
		\setlength{\cftfigindent}{1.5em}
		\def\cftlstlistingindent {1.5em}
	}{%
		\setlength{\cfttabindent}{0em}
		\setlength{\cftfigindent}{0em}
		\def\cftlstlistingindent {0em}
	}
	
	% -------------------------------------------------------------------------
	% Calcula tamaño del margen de los números en objetos del índice
	% -------------------------------------------------------------------------
	% Código fuente
	\ifthenelse{\equal{\showsectioncaptioncode}{none}}{%
		\def\cftdefautnumwidthcode {3.0em} % Añade +0.7em
		\def\cftdefaultnumwidthromancode {5.25em} % Añade +0.5em para no overflow
	}{%
	\ifthenelse{\equal{\showsectioncaptioncode}{sec}}{%
		\def\cftdefautnumwidthcode {3.7em}
		\def\cftdefaultnumwidthromancode {5.75em}
	}{%
	\ifthenelse{\equal{\showsectioncaptioncode}{ssec}}{%
		\def\cftdefautnumwidthcode {4.4em}
		\def\cftdefaultnumwidthromancode {6.25em}
	}{%
	\ifthenelse{\equal{\showsectioncaptioncode}{sssec}}{%
		\def\cftdefautnumwidthcode {5.1em}
		\def\cftdefaultnumwidthromancode {6.75em}
	}{%
	\ifthenelse{\equal{\showsectioncaptioncode}{ssssec}}{%
		\def\cftdefautnumwidthcode {5.8em}
		\def\cftdefaultnumwidthromancode {7.25em}
	}{%
	\ifthenelse{\equal{\showsectioncaptioncode}{chap}}{%
		\def\cftdefautnumwidthcode {3.0em}
		\def\cftdefaultnumwidthromancode {5.25em}
	}{%
		\throwbadconfig{Valor configuracion incorrecto}{\showsectioncaptioncode}{none,chap,sec,ssec,sssec,ssssec}}}}}}
	}
	
	% Código fuente
	\ifthenelse{\equal{\showsectioncaptioneqn}{none}}{%
		\def\cftdefautnumwidtheqn {3.0em} % Añade +0.7em
		\def\cftdefaultnumwidthromaneqn {5.25em} % Añade +0.5em para no overflow
	}{%
	\ifthenelse{\equal{\showsectioncaptioneqn}{sec}}{%
		\def\cftdefautnumwidtheqn {3.7em}
		\def\cftdefaultnumwidthromaneqn {5.75em}
	}{%
	\ifthenelse{\equal{\showsectioncaptioneqn}{ssec}}{%
		\def\cftdefautnumwidtheqn {4.4em}
		\def\cftdefaultnumwidthromaneqn {6.25em}
	}{%
	\ifthenelse{\equal{\showsectioncaptioneqn}{sssec}}{%
		\def\cftdefautnumwidtheqn {5.1em}
		\def\cftdefaultnumwidthromaneqn {6.75em}
	}{%
	\ifthenelse{\equal{\showsectioncaptioneqn}{ssssec}}{%
		\def\cftdefautnumwidtheqn {5.8em}
		\def\cftdefaultnumwidthromaneqn {7.25em}
	}{%
	\ifthenelse{\equal{\showsectioncaptioneqn}{chap}}{%
		\def\cftdefautnumwidtheqn {3.0em}
		\def\cftdefaultnumwidthromaneqn {5.25em}
	}{%
		\throwbadconfig{Valor configuracion incorrecto}{\showsectioncaptioneqn}{none,chap,sec,ssec,sssec,ssssec}}}}}}
	}
	
	% Figuras
	\ifthenelse{\equal{\showsectioncaptionfig}{none}}{%
		\def\cftdefautnumwidthfig {3.0em} % Añade +0.7em
		\def\cftdefaultnumwidthromanfig {5.25em} % Añade +0.5em
	}{%
	\ifthenelse{\equal{\showsectioncaptionfig}{sec}}{%
		\def\cftdefautnumwidthfig {3.7em}
		\def\cftdefaultnumwidthromanfig {5.75em}
	}{%
	\ifthenelse{\equal{\showsectioncaptionfig}{ssec}}{%
		\def\cftdefautnumwidthfig {4.4em}
		\def\cftdefaultnumwidthromanfig {6.25em}
	}{%
	\ifthenelse{\equal{\showsectioncaptionfig}{sssec}}{%
		\def\cftdefautnumwidthfig {5.1em}
		\def\cftdefaultnumwidthromanfig {6.75em}
	}{%
	\ifthenelse{\equal{\showsectioncaptionfig}{ssssec}}{%
		\def\cftdefautnumwidthfig {5.8em}
		\def\cftdefaultnumwidthromanfig {7.25em}
	}{%
	\ifthenelse{\equal{\showsectioncaptionfig}{chap}}{%
		\def\cftdefautnumwidthfig {3.0em}
		\def\cftdefaultnumwidthromanfig {5.25em}
	}{%
		\throwbadconfig{Valor configuracion incorrecto}{\showsectioncaptionfig}{none,chap,sec,ssec,sssec,ssssec}}}}}}
	}
	
	% Tablas
	\ifthenelse{\equal{\showsectioncaptiontab}{none}}{%
		\def\cftdefautnumwidthtab {3.0em} % Añade +0.7em
		\def\cftdefaultnumwidthromantab {5.25em} % Añade +0.5em
	}{%
	\ifthenelse{\equal{\showsectioncaptiontab}{sec}}{%
		\def\cftdefautnumwidthtab {3.7em}
		\def\cftdefaultnumwidthromantab {5.75em}
	}{%
	\ifthenelse{\equal{\showsectioncaptiontab}{ssec}}{%
		\def\cftdefautnumwidthtab {4.4em}
		\def\cftdefaultnumwidthromantab {6.25em}
	}{%
	\ifthenelse{\equal{\showsectioncaptiontab}{sssec}}{%
		\def\cftdefautnumwidthtab {5.1em}
		\def\cftdefaultnumwidthromantab {6.75em}
	}{%
	\ifthenelse{\equal{\showsectioncaptiontab}{ssssec}}{%
		\def\cftdefautnumwidthtab {5.8em}
		\def\cftdefaultnumwidthromantab {7.25em}
	}{%
	\ifthenelse{\equal{\showsectioncaptiontab}{chap}}{%
		\def\cftdefautnumwidthtab {3.0em}
		\def\cftdefaultnumwidthromantab {5.25em}
	}{%
		\throwbadconfig{Valor configuracion incorrecto}{\showsectioncaptiontab}{none,chap,sec,ssec,sssec,ssssec}}}}}}
	}
	
	% Configuración identado de títulos de objetos después del número
	\def\cftfignumwidth {\cftdefautnumwidth}
	% \def\cftsubfignumwidth {\cftdefautnumwidth}
	\def\cfttabnumwidth {\cftdefautnumwidth}
	\def\cftlstlistingnumwidth {\cftdefautnumwidth}
	
	% Código fuente
	\ifthenelse{\equal{\captionnumcode}{arabic}}{% No hace nada (default)
		\def\cftlstlistingnumwidth {\cftdefautnumwidthcode}
	}{%
		\ifthenelse{\equal{\captionnumcode}{roman}}{%
			\def\cftlstlistingnumwidth {\cftdefaultnumwidthromancode}
		}{%
		\ifthenelse{\equal{\captionnumcode}{Roman}}{%
			\def\cftlstlistingnumwidth {\cftdefaultnumwidthromancode}
		}{%
			\def\cftlstlistingnumwidth {\cftdefautnumwidthcode}
		}}
	}
	
	% Ecuaciones
	\ifthenelse{\equal{\captionnumequation}{arabic}}{% No hace nada (default)
		\def\cftmyindexequationsnumwidth {\cftdefautnumwidtheqn}
	}{%
		\ifthenelse{\equal{\captionnumequation}{roman}}{%
			\def\cftmyindexequationsnumwidth {\cftdefaultnumwidthromaneqn}
		}{%
		\ifthenelse{\equal{\captionnumequation}{Roman}}{%
			\def\cftmyindexequationsnumwidth {\cftdefaultnumwidthromaneqn}
		}{%
			\def\cftmyindexequationsnumwidth {\cftdefautnumwidtheqn}
		}}
	}
	
	% Figuras
	\ifthenelse{\equal{\captionnumfigure}{arabic}}{% No hace nada (default)
		\def\cftfignumwidth {\cftdefautnumwidthfig}
	}{%
		\ifthenelse{\equal{\captionnumfigure}{roman}}{%
			\def\cftfignumwidth {\cftdefaultnumwidthromanfig}
		}{%
			\ifthenelse{\equal{\captionnumfigure}{Roman}}{%
				\def\cftfignumwidth {\cftdefaultnumwidthromanfig}
			}{%
				\def\cftfignumwidth {\cftdefautnumwidthfig}
			}}
	}
	
	% Tablas
	\ifthenelse{\equal{\captionnumtable}{arabic}}{% No hace nada (default)
		\def\cfttabnumwidth {\cftdefautnumwidthtab}
	}{%
		\ifthenelse{\equal{\captionnumtable}{roman}}{%
			\def\cfttabnumwidth {\cftdefaultnumwidthromantab}
		}{%
			\ifthenelse{\equal{\captionnumtable}{Roman}}{%
				\def\cfttabnumwidth {\cftdefaultnumwidthromantab}
			}{%
				\def\cfttabnumwidth {\cftdefautnumwidthtab}
			}}
	}
	
	% -------------------------------------------------------------------------
	% Genera las funciones para los índices
	% -------------------------------------------------------------------------
	\newcommand{\LoIf}{% Lista de figuras
		\iftotalfigures
			\ifthenelse{\equal{\indexnewpagef}{true}}{\clearpage}{}
			\ifthenelse{\equal{\addindextobookmarks}{true}}{%
				\ifthenelse{\equal{\addindexsubtobookmarks}{true}}{%
					\phantomsection
					\belowpdfbookmark{\nameltfigure}{clof}}{}}{%
			}
			\listoffigures
		\fi
	}
	\newcommand{\LoIt}{% Tablas
		\iftotaltables
			\ifthenelse{\equal{\indexnewpaget}{true}}{\clearpage}{}
			\ifthenelse{\equal{\addindextobookmarks}{true}}{%
				\ifthenelse{\equal{\addindexsubtobookmarks}{true}}{%
					\phantomsection
					\belowpdfbookmark{\namelttable}{clot}}{}}{%
			}
			\listoftables
		\fi
	}
	\newcommand{\LoIc}{% Códigos fuente (listings)
		\iftotallstlistings
			\ifthenelse{\equal{\indexnewpagec}{true}}{\clearpage}{}
			\ifthenelse{\equal{\addindextobookmarks}{true}}{%
				\ifthenelse{\equal{\addindexsubtobookmarks}{true}}{%
					\phantomsection
					\belowpdfbookmark{\nameltsrc}{clsrc}}{}}{%
			}
			\lstlistoflistings
		\fi
	}
	\newcommand{\LoIe}{% Ecuaciones
		\iftotaltemplateIndexEquationss
			\ifthenelse{\equal{\indexnewpagee}{true}}{\clearpage}{}
			\ifthenelse{\equal{\addindextobookmarks}{true}}{%
				\ifthenelse{\equal{\addindexsubtobookmarks}{true}}{%
					\phantomsection
					\belowpdfbookmark{\namelteqn}{cleqn}}{}}{%
			}
			\listofmyindexequations
		\fi
	}
	
	% -------------------------------------------------------------------------
	% Índice de contenidos
	% -------------------------------------------------------------------------
	\ifthenelse{\equal{\showindexofcontents}{true}}{%
		\tableofcontents
	}{}
	
	% -------------------------------------------------------------------------
	% Índice de objetos
	% -------------------------------------------------------------------------
	\ifthenelse{\equal{\indexstyle}{ftc}}{%
		\LoIf\LoIt\LoIc
	}{%
	\ifthenelse{\equal{\indexstyle}{}}{%
	}{%
	\ifthenelse{\equal{\indexstyle}{e}}{%
		\LoIe
	}{%
	\ifthenelse{\equal{\indexstyle}{c}}{%
		\LoIc
	}{%
	\ifthenelse{\equal{\indexstyle}{f}}{%
		\LoIf
	}{%
	\ifthenelse{\equal{\indexstyle}{t}}{%
		\LoIt
	}{%
	\ifthenelse{\equal{\indexstyle}{ec}}{%
		\LoIe\LoIc
	}{%
	\ifthenelse{\equal{\indexstyle}{ce}}{%
		\LoIc\LoIe
	}{%
	\ifthenelse{\equal{\indexstyle}{ef}}{%
		\LoIe\LoIf
	}{%
	\ifthenelse{\equal{\indexstyle}{fe}}{%
		\LoIf\LoIe
	}{%
	\ifthenelse{\equal{\indexstyle}{et}}{%
		\LoIe\LoIt
	}{%
	\ifthenelse{\equal{\indexstyle}{te}}{%
		\LoIt\LoIe
	}{%
	\ifthenelse{\equal{\indexstyle}{cf}}{%
		\LoIc\LoIf
	}{%
	\ifthenelse{\equal{\indexstyle}{fc}}{%
		\LoIf\LoIc
	}{%
	\ifthenelse{\equal{\indexstyle}{ct}}{%
		\LoIc\LoIt
	}{%
	\ifthenelse{\equal{\indexstyle}{tc}}{%
		\LoIt\LoIc
	}{%
	\ifthenelse{\equal{\indexstyle}{ft}}{%
		\LoIf\LoIt
	}{%
	\ifthenelse{\equal{\indexstyle}{tf}}{%
		\LoIt\LoIf
	}{%
	\ifthenelse{\equal{\indexstyle}{ecf}}{%
		\LoIe\LoIc\LoIf
	}{%
	\ifthenelse{\equal{\indexstyle}{efc}}{%
		\LoIe\LoIf\LoIc
	}{%
	\ifthenelse{\equal{\indexstyle}{cef}}{%
		\LoIc\LoIe\LoIf
	}{%
	\ifthenelse{\equal{\indexstyle}{cfe}}{%
		\LoIc\LoIf\LoIe
	}{%
	\ifthenelse{\equal{\indexstyle}{fec}}{%
		\LoIf\LoIe\LoIc
	}{%
	\ifthenelse{\equal{\indexstyle}{fce}}{%
		\LoIf\LoIc\LoIe
	}{%
	\ifthenelse{\equal{\indexstyle}{ect}}{%
		\LoIe\LoIc\LoIt
	}{%
	\ifthenelse{\equal{\indexstyle}{etc}}{%
		\LoIe\LoIt\LoIc
	}{%
	\ifthenelse{\equal{\indexstyle}{cet}}{%
		\LoIc\LoIe\LoIt
	}{%
	\ifthenelse{\equal{\indexstyle}{cte}}{%
		\LoIc\LoIt\LoIe
	}{%
	\ifthenelse{\equal{\indexstyle}{tec}}{%
		\LoIt\LoIe\LoIc
	}{%
	\ifthenelse{\equal{\indexstyle}{tce}}{%
		\LoIt\LoIc\LoIe
	}{%
	\ifthenelse{\equal{\indexstyle}{eft}}{%
		\LoIe\LoIf\LoIt
	}{%
	\ifthenelse{\equal{\indexstyle}{etf}}{%
		\LoIe\LoIt\LoIf
	}{%
	\ifthenelse{\equal{\indexstyle}{fet}}{%
		\LoIf\LoIe\LoIt
	}{%
	\ifthenelse{\equal{\indexstyle}{fte}}{%
		\LoIf\LoIt\LoIe
	}{%
	\ifthenelse{\equal{\indexstyle}{tef}}{%
		\LoIt\LoIe\LoIf
	}{%
	\ifthenelse{\equal{\indexstyle}{tfe}}{%
		\LoIt\LoIf\LoIe
	}{%
	\ifthenelse{\equal{\indexstyle}{cft}}{%
		\LoIc\LoIf\LoIt
	}{%
	\ifthenelse{\equal{\indexstyle}{ctf}}{%
		\LoIc\LoIt\LoIf
	}{%
	\ifthenelse{\equal{\indexstyle}{fct}}{%
		\LoIf\LoIc\LoIt
	}{%
	\ifthenelse{\equal{\indexstyle}{tcf}}{%
		\LoIt\LoIc\LoIf
	}{%
	\ifthenelse{\equal{\indexstyle}{tfc}}{%
		\LoIt\LoIf\LoIc
	}{%
	\ifthenelse{\equal{\indexstyle}{ecft}}{%
		\LoIe\LoIc\LoIf\LoIt
	}{%
	\ifthenelse{\equal{\indexstyle}{ectf}}{%
		\LoIe\LoIc\LoIt\LoIf
	}{%
	\ifthenelse{\equal{\indexstyle}{efct}}{%
		\LoIe\LoIf\LoIc\LoIt
	}{%
	\ifthenelse{\equal{\indexstyle}{eftc}}{%
		\LoIe\LoIf\LoIt\LoIc
	}{%
	\ifthenelse{\equal{\indexstyle}{etcf}}{%
		\LoIe\LoIt\LoIc\LoIf
	}{%
	\ifthenelse{\equal{\indexstyle}{etfc}}{%
		\LoIe\LoIt\LoIf\LoIc
	}{%
	\ifthenelse{\equal{\indexstyle}{ceft}}{%
		\LoIc\LoIe\LoIf\LoIt
	}{%
	\ifthenelse{\equal{\indexstyle}{cetf}}{%
		\LoIc\LoIe\LoIt\LoIf
	}{%
	\ifthenelse{\equal{\indexstyle}{cfet}}{%
		\LoIc\LoIf\LoIe\LoIt
	}{%
	\ifthenelse{\equal{\indexstyle}{cfte}}{%
		\LoIc\LoIf\LoIt\LoIe
	}{%
	\ifthenelse{\equal{\indexstyle}{ctef}}{%
		\LoIc\LoIt\LoIe\LoIf
	}{%
	\ifthenelse{\equal{\indexstyle}{ctfe}}{%
		\LoIc\LoIt\LoIf\LoIe
	}{%
	\ifthenelse{\equal{\indexstyle}{fect}}{%
		\LoIf\LoIe\LoIc\LoIt
	}{%
	\ifthenelse{\equal{\indexstyle}{fetc}}{%
		\LoIf\LoIe\LoIt\LoIc
	}{%
	\ifthenelse{\equal{\indexstyle}{fcet}}{%
		\LoIf\LoIc\LoIe\LoIt
	}{%
	\ifthenelse{\equal{\indexstyle}{fcte}}{%
		\LoIf\LoIc\LoIt\LoIe
	}{%
	\ifthenelse{\equal{\indexstyle}{ftec}}{%
		\LoIf\LoIt\LoIe\LoIc
	}{%
	\ifthenelse{\equal{\indexstyle}{ftce}}{%
		\LoIf\LoIt\LoIc\LoIe
	}{%
	\ifthenelse{\equal{\indexstyle}{tecf}}{%
		\LoIt\LoIe\LoIc\LoIf
	}{%
	\ifthenelse{\equal{\indexstyle}{tefc}}{%
		\LoIt\LoIe\LoIf\LoIc
	}{%
	\ifthenelse{\equal{\indexstyle}{tcef}}{%
		\LoIt\LoIc\LoIe\LoIf
	}{%
	\ifthenelse{\equal{\indexstyle}{tcfe}}{%
		\LoIt\LoIc\LoIf\LoIe
	}{%
	\ifthenelse{\equal{\indexstyle}{tfec}}{%
		\LoIt\LoIf\LoIe\LoIc
	}{%
	\ifthenelse{\equal{\indexstyle}{tfce}}{%
		\LoIt\LoIf\LoIc\LoIe
	}{%
		\throwbadconfig{Estilo desconocido del indice}{\indexstyle}{ftc,,e,c,f,t,ec,ce,ef,fe,et,te,cf,fc,ct,tc,ft,tf,ecf,efc,cef,cfe,fec,fce,ect,etc,cet,cte,tec,tce,eft,etf,fet,fte,tef,tfe,cft,ctf,fct,tcf,tfc,ecft,ectf,efct,eftc,etcf,etfc,ceft,cetf,cfet,cfte,ctef,ctfe,fect,fetc,fcet,fcte,ftec,ftce,tecf,tefc,tcef,tcfe,tfec,tfce}}}}}}}}}}}}}}}}}}}}}}}}}}}}}}}}}}}}}}}}}}}}}}}}}}}}}}}}}}}}}}}}}
	}
	
	% -------------------------------------------------------------------------
	% Termina el bloque de índice
	% -------------------------------------------------------------------------
	\sectionfont{\color{\sectioncolor} \sectionfontsize \sectionfontstyle \selectfont}
	\endgroup
	
	% -------------------------------------------------------------------------
	% Final del índice, restablece el espacio
	% -------------------------------------------------------------------------
	\ifthenelse{\equal{\objectchaptermargin}{false}}{%
		\renewcommand{\addvspace}[1]{\origaddvspace{##1}}
	}{}
	
}

% -----------------------------------------------------------------------------
% CONFIGURACIONES FINALES
% -----------------------------------------------------------------------------
\newcommand{\templateFinalcfg}{%
	
	% -------------------------------------------------------------------------
	% Se restablecen headers y footers
	% -------------------------------------------------------------------------
	\markboth{}{}
	\clearpage
	% Actualiza headers
	\ifthenelse{\equal{\disablehfrightmark}{false}}{%
		% Define funciones generales
		\def\COREhfstyledefA {% 1, 2, 4, 9, 11, 14, 15
			\fancypagestyle{plain}{\fancyhead[L]{\nouppercase{\leftmark}}}
			\fancyhead[L]{\nouppercase{\leftmark}}
		}
		\def\COREhfstyledefB {% 5
			\fancypagestyle{plain}{%
				\ifthenelse{\equal{\hfwidthwrap}{true}}{%
					\fancyhead[R]{%
						\begin{minipage}[t]{\hfwidthtitle\linewidth}
							\begin{flushright}
								\nouppercase{\leftmark}
							\end{flushright}
						\end{minipage}
					}
				}{%
					\fancyhead[R]{\nouppercase{\leftmark}}
				}
			}
			\ifthenelse{\equal{\hfwidthwrap}{true}}{%
				\fancyhead[R]{%
					\begin{minipage}[t]{\hfwidthtitle\linewidth}
						\begin{flushright}
							\nouppercase{\leftmark}
						\end{flushright}
					\end{minipage}
				}
			}{%
				\fancyhead[R]{\nouppercase{\leftmark}}
			}
		}
		\def\COREhfstyledefC {% 10
			\fancypagestyle{plain}{%
				\ifthenelse{\equal{\hfwidthwrap}{true}}{%
					\fancyhead[L]{%
						\begin{minipage}[t]{\hfwidthtitle\linewidth}
							\begin{flushleft}
								\nouppercase{\leftmark}
							\end{flushleft}
						\end{minipage}
					}
				}{%
					\fancyhead[L]{\nouppercase{\leftmark}}
				}
			}
			\ifthenelse{\equal{\hfwidthwrap}{true}}{%
				\fancyhead[L]{%
					\begin{minipage}[t]{\hfwidthtitle\linewidth}
						\begin{flushleft}
							\nouppercase{\leftmark}
						\end{flushleft}
					\end{minipage}
				}
			}{%
				\fancyhead[L]{\nouppercase{\leftmark}}
			}
		}
		% Actualiza los header-footer
		\ifthenelse{\equal{\hfstyle}{style1}}{%
			\COREhfstyledefA
		}{%
		\ifthenelse{\equal{\hfstyle}{style1-i}}{% Impar izquierdo
			\fancypagestyle{plain}{\fancyhead[LE,RO]{\nouppercase{\leftmark}}}
			\fancyhead[LE,RO]{\nouppercase{\leftmark}}
		}{%
		\ifthenelse{\equal{\hfstyle}{style1-d}}{% Impar derecho
			\fancypagestyle{plain}{\fancyhead[LO,RE]{\nouppercase{\leftmark}}}
			\fancyhead[LO,RE]{\nouppercase{\leftmark}}
		}{%
		\ifthenelse{\equal{\hfstyle}{style2}}{%
			\COREhfstyledefA
		}{%
		\ifthenelse{\equal{\hfstyle}{style2-i}}{% Impar izquierdo
			\fancypagestyle{plain}{\fancyhead[LE,RO]{\nouppercase{\leftmark}}}
			\fancyhead[LE,RO]{\nouppercase{\leftmark}}
		}{%
		\ifthenelse{\equal{\hfstyle}{style2-d}}{% Impar derecho
			\fancypagestyle{plain}{\fancyhead[LO,RE]{\nouppercase{\leftmark}}}
			\fancyhead[LO,RE]{\nouppercase{\leftmark}}
		}{%
		\ifthenelse{\equal{\hfstyle}{style4}}{%
			\COREhfstyledefA
		}{%
		\ifthenelse{\equal{\hfstyle}{style5}}{%
			\COREhfstyledefB
		}{%
		\ifthenelse{\equal{\hfstyle}{style5-d}}{% Impar derecho
			\COREhfstyledefB
		}{%
		\ifthenelse{\equal{\hfstyle}{style5-i}}{% Impar izquierdo
			\COREhfstyledefB
		}{%
		\ifthenelse{\equal{\hfstyle}{style9}}{%
			\COREhfstyledefA
		}{%
		\ifthenelse{\equal{\hfstyle}{style9-d}}{% Impar derecho
			\COREhfstyledefA
		}{%
		\ifthenelse{\equal{\hfstyle}{style9-i}}{% Impar izquierdo
			\COREhfstyledefA
		}{%
		\ifthenelse{\equal{\hfstyle}{style10}}{%
			\COREhfstyledefC
		}{%
		\ifthenelse{\equal{\hfstyle}{style10-d}}{% Impar derecho
			\COREhfstyledefC
		}{%
		\ifthenelse{\equal{\hfstyle}{style10-i}}{% Impar izquierdo
			\COREhfstyledefC
		}{%
		\ifthenelse{\equal{\hfstyle}{style11}}{% Similar a 1
			\COREhfstyledefA
		}{%
		\ifthenelse{\equal{\hfstyle}{style14}}{% Similar a 4
			\COREhfstyledefA
		}{%
		\ifthenelse{\equal{\hfstyle}{style15}}{% Similar a 1
			\COREhfstyledefA
		}{%
			% No se encontró el header-footer, no hace nada
		}}}}}}}}}}}}}}}}}}}
	}{%
	}
	
	% -------------------------------------------------------------------------
	% Crea funciones para numerar objetos
	% -------------------------------------------------------------------------
	% Numeración de la sección en los objetos código fuente
	\ifthenelse{\equal{\showsectioncaptioncode}{none}}{%
		\def\sectionobjectnumcode {}
	}{%
	\ifthenelse{\equal{\showsectioncaptioncode}{sec}}{%
		\def\sectionobjectnumcode {\thesection\sectioncaptiondelimiter}
	}{%
	\ifthenelse{\equal{\showsectioncaptioncode}{ssec}}{%
		\def\sectionobjectnumcode {\thesubsection\sectioncaptiondelimiter}
	}{%
	\ifthenelse{\equal{\showsectioncaptioncode}{sssec}}{%
		\def\sectionobjectnumcode {\thesubsubsection\sectioncaptiondelimiter}
	}{%
	\ifthenelse{\equal{\showsectioncaptioncode}{ssssec}}{%
		\def\sectionobjectnumcode {\thesubsubsubsection\sectioncaptiondelimiter}
	}{%
	\ifthenelse{\equal{\showsectioncaptioncode}{chap}}{%
		\def\sectionobjectnumcode {\thechapter\sectioncaptiondelimiter}
	}{%
		\throwbadconfig{Valor configuracion incorrecto}{\showsectioncaptioncode}{none,chap,sec,ssec,sssec,ssssec}}}}}}
	}
	
	% Numeración de la sección en los objetos ecuaciones
	\ifthenelse{\equal{\showsectioncaptioneqn}{none}}{%
		\def\sectionobjectnumeqn {}
	}{%
	\ifthenelse{\equal{\showsectioncaptioneqn}{sec}}{%
		\def\sectionobjectnumeqn {\thesection\sectioncaptiondelimiter}
	}{%
	\ifthenelse{\equal{\showsectioncaptioneqn}{ssec}}{%
		\def\sectionobjectnumeqn {\thesubsection\sectioncaptiondelimiter}
	}{%
	\ifthenelse{\equal{\showsectioncaptioneqn}{sssec}}{%
		\def\sectionobjectnumeqn {\thesubsubsection\sectioncaptiondelimiter}
	}{%
	\ifthenelse{\equal{\showsectioncaptioneqn}{ssssec}}{%
		\def\sectionobjectnumeqn {\thesubsubsubsection\sectioncaptiondelimiter}
	}{%
	\ifthenelse{\equal{\showsectioncaptioneqn}{chap}}{%
		\def\sectionobjectnumeqn {\thechapter\sectioncaptiondelimiter}
	}{%
		\throwbadconfig{Valor configuracion incorrecto}{\showsectioncaptioneqn}{none,chap,sec,ssec,sssec,ssssec}}}}}}
	}
	
	% Numeración de la sección en los objetos figuras
	\ifthenelse{\equal{\showsectioncaptionfig}{none}}{%
		\def\sectionobjectnumfig {}
	}{%
	\ifthenelse{\equal{\showsectioncaptionfig}{sec}}{%
		\def\sectionobjectnumfig {\thesection\sectioncaptiondelimiter}
	}{%
	\ifthenelse{\equal{\showsectioncaptionfig}{ssec}}{%
		\def\sectionobjectnumfig {\thesubsection\sectioncaptiondelimiter}
	}{%
	\ifthenelse{\equal{\showsectioncaptionfig}{sssec}}{%
		\def\sectionobjectnumfig {\thesubsubsection\sectioncaptiondelimiter}
	}{%
	\ifthenelse{\equal{\showsectioncaptionfig}{ssssec}}{%
		\def\sectionobjectnumfig {\thesubsubsubsection\sectioncaptiondelimiter}
	}{%
	\ifthenelse{\equal{\showsectioncaptionfig}{chap}}{%
		\def\sectionobjectnumfig {\thechapter\sectioncaptiondelimiter}
	}{%
		\throwbadconfig{Valor configuracion incorrecto}{\showsectioncaptionfig}{none,chap,sec,ssec,sssec,ssssec}}}}}}
	}
	
	% Numeración de la sección en los objetos tablas
	\ifthenelse{\equal{\showsectioncaptiontab}{none}}{%
		\def\sectionobjectnumtab {}
	}{%
	\ifthenelse{\equal{\showsectioncaptiontab}{sec}}{%
		\def\sectionobjectnumtab {\thesection\sectioncaptiondelimiter}
	}{%
	\ifthenelse{\equal{\showsectioncaptiontab}{ssec}}{%
		\def\sectionobjectnumtab {\thesubsection\sectioncaptiondelimiter}
	}{%
	\ifthenelse{\equal{\showsectioncaptiontab}{sssec}}{%
		\def\sectionobjectnumtab {\thesubsubsection\sectioncaptiondelimiter}
	}{%
	\ifthenelse{\equal{\showsectioncaptiontab}{ssssec}}{%
		\def\sectionobjectnumtab {\thesubsubsubsection\sectioncaptiondelimiter}
	}{%
	\ifthenelse{\equal{\showsectioncaptiontab}{chap}}{%
		\def\sectionobjectnumtab {\thechapter\sectioncaptiondelimiter}
	}{%
		\throwbadconfig{Valor configuracion incorrecto}{\showsectioncaptiontab}{none,chap,sec,ssec,sssec,ssssec}}}}}}
	}
	
	% -------------------------------------------------------------------------
	% Modifica numeración de objetos
	% -------------------------------------------------------------------------
	% Código fuente, incluir sección
	\ifthenelse{\equal{\captionnumcode}{arabic}}{%
		\renewcommand{\thelstlisting}{\sectionobjectnumcode\arabic{lstlisting}}
	}{%
	\ifthenelse{\equal{\captionnumcode}{alph}}{%
		\renewcommand{\thelstlisting}{\sectionobjectnumcode\alph{lstlisting}}
	}{%
	\ifthenelse{\equal{\captionnumcode}{Alph}}{%
		\renewcommand{\thelstlisting}{\sectionobjectnumcode\Alph{lstlisting}}
	}{%
	\ifthenelse{\equal{\captionnumcode}{roman}}{%
		\renewcommand{\thelstlisting}{\sectionobjectnumcode\roman{lstlisting}}
	}{%
	\ifthenelse{\equal{\captionnumcode}{Roman}}{%
		\renewcommand{\thelstlisting}{\sectionobjectnumcode\Roman{lstlisting}}
	}{%
		\throwbadconfig{Tipo numero codigo fuente desconocido}{\captionnumcode}{arabic,alph,Alph,roman,Roman}}}}}
	}
	
	% Ecuaciones, incluir sección
	\ifthenelse{\equal{\captionnumequation}{arabic}}{%
		\renewcommand{\theequation}{\sectionobjectnumeqn\arabic{equation}}
	}{%
	\ifthenelse{\equal{\captionnumequation}{alph}}{%
		\renewcommand{\theequation}{\sectionobjectnumeqn\alph{equation}}
	}{%
	\ifthenelse{\equal{\captionnumequation}{Alph}}{%
		\renewcommand{\theequation}{\sectionobjectnumeqn\Alph{equation}}
	}{%
	\ifthenelse{\equal{\captionnumequation}{roman}}{%
		\renewcommand{\theequation}{\sectionobjectnumeqn\roman{equation}}
	}{%
	\ifthenelse{\equal{\captionnumequation}{Roman}}{%
		\renewcommand{\theequation}{\sectionobjectnumeqn\Roman{equation}}
	}{%
		\throwbadconfig{Tipo numero ecuacion desconocido}{\captionnumequation}{arabic,alph,Alph,roman,Roman}}}}}
	}
	
	% Figuras, incluir sección
	\ifthenelse{\equal{\captionnumfigure}{arabic}}{%
		\renewcommand{\thefigure}{\sectionobjectnumfig\arabic{figure}}
	}{%
	\ifthenelse{\equal{\captionnumfigure}{alph}}{%
		\renewcommand{\thefigure}{\sectionobjectnumfig\alph{figure}}
	}{%
	\ifthenelse{\equal{\captionnumfigure}{Alph}}{%
		\renewcommand{\thefigure}{\sectionobjectnumfig\Alph{figure}}
	}{%
	\ifthenelse{\equal{\captionnumfigure}{roman}}{%
		\renewcommand{\thefigure}{\sectionobjectnumfig\roman{figure}}
	}{%
	\ifthenelse{\equal{\captionnumfigure}{Roman}}{%
		\renewcommand{\thefigure}{\sectionobjectnumfig\Roman{figure}}
	}{%
		\throwbadconfig{Tipo numero figura desconocido}{\captionnumfigure}{arabic,alph,Alph,roman,Roman}}}}}
	}
	
	% Subfiguras, no usar secciones ya que son hijas de figura
	\ifthenelse{\equal{\captionnumsubfigure}{arabic}}{%
		\renewcommand{\thesubfigure}{\arabic{subfigure}}
	}{%
	\ifthenelse{\equal{\captionnumsubfigure}{alph}}{%
		\renewcommand{\thesubfigure}{\alph{subfigure}}
	}{%
	\ifthenelse{\equal{\captionnumsubfigure}{Alph}}{%
		\renewcommand{\thesubfigure}{\Alph{subfigure}}
	}{%
	\ifthenelse{\equal{\captionnumsubfigure}{roman}}{%
		\renewcommand{\thesubfigure}{\roman{subfigure}}
	}{%
	\ifthenelse{\equal{\captionnumsubfigure}{Roman}}{%
		\renewcommand{\thesubfigure}{\Roman{subfigure}}
	}{%
		\throwbadconfig{Tipo numero subfigura desconocido}{\captionnumsubfigure}{arabic,alph,Alph,roman,Roman}}}}}
	}
	
	% Tablas, incluir sección
	\ifthenelse{\equal{\captionnumtable}{arabic}}{%
		\renewcommand{\thetable}{\sectionobjectnumtab\arabic{table}}
	}{%
	\ifthenelse{\equal{\captionnumtable}{alph}}{%
		\renewcommand{\thetable}{\sectionobjectnumtab\alph{table}}
	}{%
	\ifthenelse{\equal{\captionnumtable}{Alph}}{%
		\renewcommand{\thetable}{\sectionobjectnumtab\Alph{table}}
	}{%
	\ifthenelse{\equal{\captionnumtable}{roman}}{%
		\renewcommand{\thetable}{\sectionobjectnumtab\roman{table}}
	}{%
	\ifthenelse{\equal{\captionnumtable}{Roman}}{%
		\renewcommand{\thetable}{\sectionobjectnumtab\Roman{table}}
	}{%
		\throwbadconfig{Tipo numero tabla desconocido}{\captionnumtable}{arabic,alph,Alph,roman,Roman}}}}}
	}
	
	% Subtablas, no incluir sección ya que son hijas de las tablas
	\ifthenelse{\equal{\captionnumsubtable}{arabic}}{%
		\renewcommand{\thesubtable}{\arabic{subtable}}
	}{%
	\ifthenelse{\equal{\captionnumsubtable}{alph}}{%
		\renewcommand{\thesubtable}{\alph{subtable}}
	}{%
	\ifthenelse{\equal{\captionnumsubtable}{Alph}}{%
		\renewcommand{\thesubtable}{\Alph{subtable}}
	}{%
	\ifthenelse{\equal{\captionnumsubtable}{roman}}{%
		\renewcommand{\thesubtable}{\roman{subtable}}
	}{%
	\ifthenelse{\equal{\captionnumsubtable}{Roman}}{%
		\renewcommand{\thesubtable}{\Roman{subtable}}
	}{%
		\throwbadconfig{Tipo numero subtabla desconocido}{\captionnumsubtable}{arabic,alph,Alph,roman,Roman}}}}}
	}
	
	% -------------------------------------------------------------------------
	% Agrega páginas dependiendo del formato
	% -------------------------------------------------------------------------
	\ifthenelse{\equal{\GLOBALtwoside}{true}}{%
		\coretriggeronpage{\emptypagespredocformat}{}}{%
	}
	
	% -------------------------------------------------------------------------
	% Reestablece \cleardoublepage
	% -------------------------------------------------------------------------
	% \let\cleardoublepage\oldcleardoublepage
	\let\cleardoublepage\corecleardoublepage
	
	% -------------------------------------------------------------------------
	% Se restablecen números de página y secciones
	% -------------------------------------------------------------------------
	% Se usa número de páginas en arábigo si es que se tenía activado los números romanos
	\ifthenelse{\equal{\predocpageromannumber}{true}}{%
		\renewcommand{\thepage}{\arabic{page}}}{%
	}
	
	% Reinicia número de página
	\ifthenelse{\equal{\predocresetpagenumber}{true}}{%
		\setcounter{page}{1}}{%
	}
	
	\setcounter{section}{0}
	\setcounter{footnote}{0}
	
	% -------------------------------------------------------------------------
	% Muestra los números de línea
	% -------------------------------------------------------------------------
	\ifthenelse{\equal{\showlinenumbers}{true}}{%
		\linenumbers}{%
	}
	
	% -------------------------------------------------------------------------
	% Establece el estilo de las sub-sub-sub-secciones
	% -------------------------------------------------------------------------
	\titleclass{\subsubsubsection}{straight}[\subsection]
	
	% -------------------------------------------------------------------------
	% Reestablece los valores del estado de los títulos
	% -------------------------------------------------------------------------
	\global\def\GLOBALtitlerequirechapter {true}
	\global\def\GLOBALtitleinitchapter {false}
	\global\def\GLOBALtitleinitsection {false}
	\global\def\GLOBALtitleinitsubsection {false}
	\global\def\GLOBALtitleinitsubsubsection {false}
	\global\def\GLOBALtitleinitsubsubsubsection {false}
	
}


% INICIO DE LAS PÁGINAS
\begin{document}

% PORTADA
\templatePortrait

% CONFIGURACIÓN DE PÁGINA Y ENCABEZADOS
\templatePagecfg

% RESUMEN O ABSTRACT
\begin{abstractd}
En esta tesis se estudian los flujos de información en un sistema compuesto por tres puntos cuánticos, denominados \( L \), \( R \) y \( D \), cada uno acoplado a un reservorio distinto. Se derivó una ecuación de tipo Lindblad semilocal para modelar la evolución del sistema, lo que permite incorporar efectos no seculares en la ecuación maestra. A partir de esta formulación, se simula numéricamente la dinámica bajo distintas condiciones iniciales y de acoplamiento, evaluando cantidades termodinámicas como el flujo de calor, la potencia y otros indicadores relevantes. El objetivo es identificar regímenes en los que el sistema actúe como un Demonio de Maxwell autónomo. Para caracterizar dicho comportamiento, se analiza el papel de la información en el proceso de transporte y cómo los flujos de información pueden verse afectados por el acoplamiento entre los sitios. Se demuestra que, para ciertas tasas de transición, el sistema exhibe un comportamiento coherente con el de un Demonio de Maxwell, siendo el flujo de información una fuente fundamental para la extracción de trabajo del sistema. Además, se calculan la concurrencia y la coherencia en el estado estacionario para distintos parámetros, encontrando que ambas alcanzan valores cercanos a sus máximos en la misma región en que los flujos de información asociados a los puntos \( L \) y \( R \) se vuelven idénticos. Por último, se construye un modelo semiclásico que considera únicamente las componentes diagonales de la matriz densidad, permitiendo comparar los flujos de información con el modelo plenamente cuántico. Se concluye que este último presenta una ventaja en la transferencia de información, lo que resalta el papel de las coherencias cuánticas como recurso termodinámico.
\end{abstractd}


% DEDICATORIA
\begin{dedicatory}
	Para todos los que han formado parte de este camino.\\
	~ \\
	\textbf{Every second counts.}
\end{dedicatory}

% AGRADECIMIENTOS
\begin{acknowledgments}
   Quiero agradecer a mi madre, por otorgarme su apoyo incondicional durante este período, a pesar de que tuvo que pasar por momentos difíciles en este proceso en el cual no pude estar presente físicamente. Aun así, siempre me otorgó el cariño y la motivación que me llevó a seguir adelante.  
   \\

   A mi padre, por ser un ejemplo de perseverancia y resiliencia,el cual me ha inspirado a seguir adelante en los momentos más difíciles. Especialmente su templanza, que me ha permitido enfrentar los desafíos de la vida con una mirada optimista. 
   \\

   A mis perritos Mukai, Beb, Blu y los que estan en el cielo Greg y Laika, por su cariño incondicional y su felicidad que me hace mirar con esperanza el futuro.
   \\

   También agradecer la presencia  y cariño de mi familia: A mi hermano Nelson, a mis primos hermanos Camila, Carlos Valentín, Benjamín, Rafa, a mis tíos Jorge, Elena, Marcela, mi tía Pepa, a mis abuela Pocha y a mi abuela Lila. A todos ellos por demostrarme su cariño y impulsarme a seguir adelante. 
   \\

   Agradecer a mis amigos de la universidad María, Trevor, Tom, Seba, Gabo y Medel por estar presente en esta dura etapa y mostrar su hermandad y apoyo mutuo en este proceso, tanto en lo personal como en lo académico, además agradecer a mis amigos del magister Guido y Mondaca.
   \\

   Agradecer a mis amigos del colegio, que me vieron y ayudaron a crecer, siempre se les guardará un cariño grande, Jorge GG, Jami, Matías, Pipe, Basti, Vanesa, Kevin, Almendra y Sofía.  
   \\

   Agradecer a los miembros de la comisión Alvaro Nuñez y Gonzalo Gutierrez por ser parte importante de mi formación como físico tanto en este proceso, como en el pregrado.
   \\

   A Claudia Urrutia, por su apoyo en todos los procesos burocráticos y administrativos, y por su buena disposición para ayudar en todo momento. 
   \\

   A los sensei de Judo de la Universidad, por entregar las palabras precisas justo en momentos en que lo necesitaba.
   \\

   Por último, agradecer a mi profesor guía Felipe Barra por su paciencia y disposición para poder afrontar las preguntas y desafíos de este proceso, y siempre inculcarme la curiosidad y el pensamiento crítico en este mundo de la ciencia. 

\end{acknowledgments}

% TABLA DE CONTENIDOS - ÍNDICE
\templateIndex

% CONFIGURACIONES FINALES
\templateFinalcfg

% ======================= INICIO DEL DOCUMENTO =======================

% Template:     Tesis LaTeX
% Documento:    Archivo de ejemplo
% Versión:      3.4.0 (23/08/2024)
% Codificación: UTF-8
%
% Autor: Pablo Pizarro R.
%        pablo@ppizarror.com
%
% Manual template: [https://latex.ppizarror.com/tesis]
% Licencia MIT:    [https://opensource.org/licenses/MIT]

% ------------------------------------------------------------------------------
% NUEVO CAPÍTULO
% ------------------------------------------------------------------------------
% A diferencia de Template-Informe, Template-Tesis requiere el uso de capítulos; las secciones, subsecciones, etc son parte de un capítulo. Se recomienda el uso de un capítulo en un archivo distinto
\chapteranum{Introducción}

Uno de los principales objetivos de la termodinámica contemporánea es esclarecer el carácter físico de la información. Este desafío fue anticipado por el célebre experimento mental propuesto por James Clerk Maxwell en 1871, el cual sugiere una aparente violación de la segunda ley de la termodinámica al introducir un "demonio" que manipula la dinámica de un gas ideal mediante información\cite{Maxwell_1871}. Esta paradoja motivó una profunda revisión conceptual, abordada en trabajos fundamentales de Szilard\cite{szilard1964decrease}, Brillouin\cite{brillouin1951maxwell}, Landauer\cite{Landauer_1961} y Bennett\cite{bennett1982thermodynamics}, quienes formalizaron el vínculo entre información y entropía, sentando las bases de la termodinámica de la información.

Los avances experimentales en nanotecnología han permitido explorar sistemas de pocas partículas donde las fluctuaciones térmicas no pueden ser despreciadas\cite{douarche2005experimental,wang2005experimental}. En este contexto, las formulaciones tradicionales de la termodinámica pierden validez, lo que exige un marco teórico más general. La termodinámica estocástica cumple este rol, al extender las leyes termodinámicas a regímenes fuera del equilibrio en sistemas clásicos de pequeña escala\cite{van2013stochastic,jarzynski1997nonequilibrium}.

Además del estudio de energía y entropía, resulta fundamental entender el rol de la información en estos sistemas. Se deben distinguir dos enfoques según el tipo de dinámica involucrada. El primero considera sistemas no autónomos, cuyo Hamiltoniano depende explícitamente del tiempo y es controlado por un agente externo. En estos casos, el demonio de Maxwell puede modelarse mediante procesos de medición y retroalimentación, como ha sido mostrado en diversos trabajos\cite{cao2009thermodynamics,sagawa2010generalized}. El segundo enfoque aborda sistemas autónomos, de especial relevancia en contextos biológicos\cite{ehrich2023energy}, donde el sistema evoluciona sin intervención externa bajo gradientes constantes de energía (como diferencias de potencial o temperatura). Aquí, la estructura bipartita del sistema permite identificar un subsistema que actúa como "demonio", acoplado al sistema de interés y modulando su dinámica dependiendo del estado en que se encuentre.

En ausencia de un agente externo, cuantificar el flujo de información entre el demonio y el sistema presenta una dificultad adicional, ya que no se dispone del resultado explícito de la medición ni del mecanismo de retroalimentación. Este desafío fue abordado por Horowitz y Esposito mediante el formalismo de la termodinámica estocástica\cite{horowitz2014thermodynamics}, quienes propusieron una definición operacional del flujo de información en sistemas clásicos autónomos.

Posteriormente, Ptaszynski y Esposito extendieron este marco al contexto cuántico, cuantificando flujos de información en sistemas abiertos gobernados por dinámicas markovianas\cite{ptaszynski2019thermodynamics}. Para ello, emplearon una ecuación maestra de tipo Lindblad en su forma estándar (GKLS)\cite{gorini1976completely,lindblad1976generators,breuer2002theory}, compatible con una descripción termodinámica consistente cuando se cumple la condición de balance detallado. Este enfoque permite incorporar coherencias cuánticas —elementos no diagonales de la matriz densidad—, las cuales pueden servir como recursos termodinámicos\cite{ptaszynski2023fermionic,streltsov2017colloquium}.

Sin embargo, una dificultad clave en la descripción cuántica es la elección de la base. Si se adopta una base global, como exige la aproximación secular, se pierde la posibilidad de distinguir entre flujos locales de información. Por otro lado, una formulación en la base local puede revelar la estructura bipartita del sistema, pero ha sido criticada por posibles inconsistencias termodinámicas\cite{levy2014local,novotny2002investigation}, dado que en general no respeta las leyes de la termodinámica.

Para resolver este dilema, Potts et al.\ propusieron una ecuación maestra termodinámicamente consistente que permite trabajar en la base local e incluir términos no seculares\cite{potts2021thermodynamically}. Su enfoque reconoce que una descripción markoviana implica una resolución energética finita, y por tanto introduce un Hamiltoniano rescalado en la definición de las cantidades termodinámicas.

Trabajos previos han empleado el formalismo desarrollado por Potts para analizar los efectos cuánticos en modelos compuestos por dos puntos cuánticos, cada uno acoplado a un reservorio distinto, y cómo estos efectos influyen en el flujo de partículas~\cite{prech2023entanglement}. Por otro lado, el trabajo de Esposito y Horowitz investigó los flujos de información en un sistema clásico constituido por dos puntos cuánticos, donde uno de ellos está acoplado a dos reservorios y el otro a un único baño térmico. En dicho estudio, se demostró que el sistema actúa como un Demonio de Maxwell autónomo~\cite{horowitz2014thermodynamics}.

En el presente trabajo se emplea el formalismo de Potts para describir la evolución temporal de un sistema cuántico formado por tres puntos cuánticos, cada uno acoplado a un reservorio distinto. Uno de estos puntos cuánticos desempeña el rol de Demonio de Maxwell, mientras que los otros dos constituyen el sistema físico, denotado como $LR$. Esta configuración permite estudiar los flujos de información y las demás magnitudes termodinámicas relevantes, en un régimen donde se manifiestan efectos cuánticos tales como la coherencia y el entrelazamiento entre los distintos sitios del sistema.

La tesis está estructurada de la siguiente manera: en el Capítulo 1 se introducen los conceptos fundamentales necesarios para describir la evolución de un sistema cuántico, ya sea como sistema cerrado —mediante la ecuación de Liouville— o como sistema abierto —a través de la ecuación de Lindblad en la aproximación secular. El Capítulo 2 está dedicado al desarrollo del Formalismo de Estadística de Conteo, el cual será empleado para derivar la ecuación de Lindblad termodinámicamente consistente propuesta por Potts. En el Capítulo 3 se introduce el concepto de flujo de información, discutiendo su influencia en la formulación de la segunda ley de la termodinámica tanto en contextos clásicos como cuánticos. Finalmente, en el Capítulo 4 se presenta la evolución dinámica de un sistema compuesto por tres puntos cuánticos, y se analizan los principales resultados obtenidos para dos configuraciones distintas del modelo.




% SUB-SECCIÓN
% Las sub-secciones se inician con \subsection, si se quiere una sub-sección
% sin número se pueden usar las funciones \subsectionanum (nuevo subtítulo sin
% numeración) o la función \subsectionanumnoi para crear el mismo subtítulo sin
% numerar y sin aparecer en el índice


% ------------------------------------------------------------------------------
% NUEVO CAPÍTULO
% ------------------------------------------------------------------------------

\chapter{Sistemas cuánticos abiertos}

En este capítulo se describen los conceptos básicos para poder describir la evolución de un sistema cuántico, en la sección \ref{sec:closedQM} se describe la evolución unitaria para un sistema cerrado.  Por otro lado en la sección \ref{sec:lindblad} se describe una de las ecuaciones utilizadas para describir el comportamiento de un sistema cuántico acoplado a un reservorio con infinitos grados de libertad.


\section{Sistemas cuánticos cerrados}
Un sistema cuántico cerrado puede ser descrito por la matriz densidad $\hat{\rho}$, la evolución de la matriz densidad dependerá del Hamiltoniano del sistema $\hat{H}(t)$, la cuál en el cuadro de Schrodinger consiste en la ecuación de Liouville-Von Neumman($\hbar = 1$)\cite{breuer2002theory}:

\begin{equation*}
    \frac{d}{dt}\hat{\rho}(t) = -i[\hat{H}(t),\hat{\rho}(t)].
\end{equation*}

Y la solución descrita por la evolución unitaria 

\begin{equation*}
    \hat{U}(t,t_{0}) = \hat{T}_{\leftarrow} \exp \left[ -i \int_{t_{0}}^{t}ds \hat{H}(s) \right] \implies \hat{\rho}(t) = \hat{U}(t,t_{0})\hat{\rho}(t_{0})\hat{U}^{\dagger}(t,t_{0}),
\end{equation*}
en donde $\hat{T}_{\leftarrow}$ consiste en el operador ordenación temporal cronológico que ordena los productos de operadores dependientes del tiempo, de tal manera que el tiempo en el que son evaluados los operadores va creciendo de derecha a izquierda.

\subsection{Cuadro de interacción}
Supongamos que el Hamiltoniano del sistema se puede separar en dos partes

\begin{equation*}
    \hat{H}(t) = \hat{H}_{0} + \hat{H}_{I}(t),
\end{equation*}
en teoría esto se puede hacer de varias formas, pero por lo general, si tenemos el caso de dos subsistemas, $\hat{H}_{0}$ contiene los Hamiltonianos de cada uno de ellos cuando no hay interacción, mientras que $\hat{H}_{I}$ representa la interacción entre ellos. Si introducimos los operadores unitarios

\begin{equation*}
    \hat{U}_{0}(t,t_{0}) \equiv \exp[-i\hat{H}_{0}(t-t_{0}) ]  \hspace{15mm} \hat{U}_{I}(t,t_0) \equiv \hat{U}^{\dagger}_{0}(t,t_{0})\hat{U}(t,t_{0}),
\end{equation*}
se puede describir la matriz densidad en el cuadro de interacción

\begin{equation*}
    \hat{\rho}_{I}(t) \equiv \hat{U}_{I}(t,t_{0})\hat{\rho}(t_{0})\hat{U}^{\dagger}_{I}(t,t_{0}).
\end{equation*}

Y la evolución en este cuadro de interacción es 

\begin{equation}
    \frac{d}{dt}\hat{\rho}_{I}(t) = -i[\tilde{H}_{I}(t), \hat{\rho}_{I}(t)],
    \label{sec11:interactionp}
\end{equation}
con 

\begin{equation*}
    \tilde{H}_{I}(t) = \hat{U}^{\dagger}_{0}(t,t_{0})\hat{H}_{I}(t)\hat{U}_{0}(t,t_{0}).
\end{equation*}

\label{sec:closedQM}



\section{Ecuación de Lindblad}
\label{SEClindblad}
\subsection{Dinámica de un sistema abierto}

Un sistema abierto consiste en un sistema cuántico $S$ denominado el sistema reducido, el cuál está acoplado a un sistema $B$ denominado el ambiente. Estos representan  subsistemas del sistema total $S+B$. Sea $\mathcal{H}_{s}$ el espacio de Hilbert del sistema y $\mathcal{H}_{B}$ el espacio de Hilbert del ambiente, el espacio de Hilbert del sistema total $S+B$ es  $\mathcal{H} = \mathcal{H}_{s} \otimes \mathcal{H}_{B}$, el Hamiltoniano total se constituye por el Hamiltoniano del sistema $\hat{H}_{S}$, el Hamiltoniano $\hat{H}_{B}$ y la interacción $\hat{H}_{I}(t)$

\begin{equation}
    \hat{H}(t) = \hat{H}_{S} \otimes \mathbf{I}_{B} + \mathbf{I}_{s} \otimes \hat{H}_{B} + \hat{H}_{I}(t).
    \label{sec2:sistemabierto}
\end{equation}

Para describir el sistema con la ecuación \ref{sec2:sistemabierto} es necesario separarlo en dos subsistemas, $S$ y $B$, permitiendo así que el espacio de Hilbert total sea descrito por $\mathcal{H} = \mathcal{H}_{s}\otimes \mathcal{H}_{B}$. Sin embargo, en el caso de partículas idénticas como los fermiones el identificar estos dos subsistemas no es obvio. Aún así, en el formalismo de segunda cuantización el sistema se puede describir mediante una base del espacio de Fock, que sí permite encontrar subsistemas en los cuáles el espacio de Hilbert total es un producto tensorial entre los dos subsistemas\cite{friis2013fermionic,vidal2021quantum}.

Los observables que se estudian en el sistema $S$ son de la forma $A\otimes \mathbf{I}_{B}$ y su valor de expectación puede ser calculado mediante la ecuación

\begin{equation*}
    \langle A \rangle = \text{Tr}_{S}\{A \hat{\rho}_{S} \},
\end{equation*}
donde 

\begin{equation*}
    \hat{\rho}_{S} = \text{Tr}_{B}\{ \hat{\rho} \},
\end{equation*}
es la matriz densidad del sistema reducido. La expresión $\text{Tr}_{S}$ hace referencia a la traza con respecto al espacio de Hilbert del sistema, mientras que $\text{Tr}_{B}$  
es la traza con respecto al espacio de Hilbert del ambiente. El objeto de estudio principalmente es $\hat{\rho}_{S}$ y la ecuación de Lindblad será utilizada para describir su evolución.

\subsection{Ecuación maestra}
Partiendo de un sistema $S$ acoplado débilmente a un reservorio $B$, en donde el Hamiltoniano total es descrito por 

\begin{equation*}
    \hat{H} = \hat{H}_{S} + \hat{H}_{B} + \hat{H}_{I},
\end{equation*}
la evolución en el cuadro de interacción del sistema total será

\begin{equation}
    \frac{d}{dt}\hat{\rho}_{I}(t) = -i[\hat{H}_{I}(t), \hat{\rho}_{I}(t)],
    \label{eqsec2:linbladpic}
\end{equation}
o en su forma integral

\begin{equation*}
    \hat{\rho}_{I}(t) = \hat{\rho}(0) - i \int_{0}^{t} ds[\hat{H}_{I}(s), \hat{\rho}_{I}(s)].
\end{equation*}

Reinsertando la forma integral en la ecuación \ref{eqsec2:linbladpic} y trazando los grados de libertad del reservorio, nos queda

\begin{equation}
    \frac{d}{dt}\hat{\rho}_{IS}(t) = -i \text{Tr}_{B}\{[\hat{H}_{I}(t),\hat{\rho}(0)] \}  -  \int_{0}^{t}ds \text{Tr}_{B}\{[\hat{H}_{I}(t), [\hat{H}_{I}(s),\hat{\rho}_{I}(s)]]\}.
\end{equation}

Ahora se asumirán dos cosas, primero que $\text{Tr}_{B}\{[\hat{H}_{I}(t),\rho(0)] \} = 0$. Segundo, debido a que todavía la ecuación depende de $\hat{\rho}(t)$ correspondiente al sistema total, tendremos que realizar la aproximación de Born, que consiste en que si se tiene acoplamiento débil, la influencia del sistema en el reservorio será pequeña, por ende se puede asumir que el estado del reservorio $\hat{\rho}_{B}$ permanece constante, y sólo el que se ve alterado es la evolución del sistema, es decir

\begin{equation*}
    \hat{\rho}(t) = \hat{\rho}_{S}(t)\otimes \hat{\rho}_{B},
\end{equation*}
con lo cuál la ecuación \ref{eqsec2:linbladpic} se vuelve

\begin{equation}
    \frac{d}{dt}\hat{\rho}_{IS}(t) = -  \int_{0}^{t}ds \text{Tr}_{B}\{[\hat{H}_{I}(t), [\hat{H}_{I}(s),\hat{\rho}_{IS}(s) \otimes \hat{\rho}_{B}]]\}.
\end{equation} 

Esta ecuación es lo que se denomina no Markoviana, ya que requiere conocer todo el pasado de $\hat{\rho}_{IS}(s)$. Para simplificar el problema, se aplicará la aproximación de Markov, con el objetivo de que la evolución de $\hat{\rho}_{IS}(t)$ dependa sólo del estado en que se encuentra en el instante de tiempo, es decir

\begin{equation*}
    \frac{d}{dt}\hat{\rho}_{IS}(t) = -  \int_{0}^{t}ds \text{Tr}_{B}\{[\hat{H}_{I}(t), [\hat{H}_{I}(s),\hat{\rho}_{IS}(t) \otimes \hat{\rho}_{B}]]\},
\end{equation*} 
esta ecuación es denominada la ecuación de Redfield. Se puede hacer el cambio de variable $s= x = t-s$, así el integral queda

\begin{equation}
    \frac{d}{dt}\hat{\rho}_{IS}(t) = -  \int_{0}^{t}ds \text{Tr}_{B}\{[\hat{H}_{I}(t), [\hat{H}_{I}(t-s),\hat{\rho}_{IS}(t) \otimes \hat{\rho}_{B}]]\}.
    \label{eq3sec2:markov}
\end{equation} 

Por último, se podrá hacer otra aproximación, ya que en esta expresión aparecen las funciones correlación del baño, las cuáles decaen en el tiempo con una escala característica dada por el denominado tiempo de correlación $\tau_{B}$, a partir del cual dichas funciones se vuelven despreciables. Mientras que el sistema tendrá su tiempo de relajación $\tau_{R}$ que consiste en el tiempo que demora en llegar a su estado estacionario. La aproximación markoviana requiere que $\tau_{R}\gg \tau_{B}$, ya que él sistema no debe ser capaz de percibir la dinámica del baño, esto permite tomar el límite del integral en \ref{eq3sec2:markov} a infinito, ya que para tiempos muy largos se anula el integral debido a que las funciones correlación se vuelven practicamente nulas. Y finalmente quedará la evolución

\begin{equation}
    \frac{d}{dt}\hat{\rho}_{IS}(t) = -  \int_{0}^{\infty}ds \text{Tr}_{B}\{[\hat{H}_{I}(t), [\hat{H}_{I}(t-s),\hat{\rho}_{IS}(t) \otimes \hat{\rho}_{B}]]\}.
    \label{eq3sec2:markov1}
\end{equation} 

La interacción en el cuadro de Schrodinger $\hat{H}_{I}$ será de la forma general

\begin{equation*}
    \hat{H}_{I} = \sum_{\alpha}A_{\alpha} \otimes B_{\alpha},
\end{equation*}
donde los operadores $A_{\alpha} = A^{\dagger}_{\alpha}$ actuan en el espacio de Hilbert del sistema, mientras que $B_{\alpha}=B^{\dagger}_{\alpha}$ actua en el espacio de Hilbert del reservorio. Es conveniente escribir la interacción en función de los autoestados de $\hat{H}_{S}$. Para ello, si tenemos los autovalores $\epsilon$ y sus respectivos operadores de proyección $\Pi(\epsilon) = |\epsilon\rangle \langle \epsilon|$, se podrán definir los operadores

\begin{equation*}
    A_{\alpha}(\omega) \equiv \sum_{\epsilon' - \epsilon}\Pi(\epsilon)A_{\alpha}\Pi(\epsilon'), 
\end{equation*}
estos operadores se denominan operadores globales, ya que utilizan los autoestados del sistema. Debido a esta definición, se cumplen las relaciones

\begin{align*}
    [\hat{H}_{S},A_{\alpha}(\omega)] & = - \omega A_{\alpha}(\omega) \\
    [\hat{H}_{S},A^{\dagger}_{\alpha}(\omega)] & = \omega A^{\dagger}_{\alpha}(\omega),
\end{align*}
posteriormente se deberá pasar al cuadro de interacción y calcular $U^{\dagger}_{s}(t)\hat{H}_{I}U_{s}(t)$. Para ello, se utilizará la relación de Baker Campbell

\begin{equation}
    e^{A}Be^{-A} = B  + [A,B] + \frac{1}{2}[A,[A,B]] +..,
    \label{sec2lind:baker}
\end{equation}
y se puede derivar las relaciones 

\begin{align*}
    e^{i\hat{H}_{S}t}A_{\alpha}(\omega) e^{-i\hat{H}_{S}t} & = e^{-i\omega t} A_{\alpha}(\omega) \\
    e^{i\hat{H}_{S}t}A^{\dagger}_{\alpha}(\omega) e^{-i\hat{H}_{S}t} & = e^{i\omega t} A^{\dagger}_{\alpha}(\omega).
\end{align*}

De la relación de completitud $\sum_{\epsilon}\Pi(\epsilon) = \mathbf{I}$ podremos notar que

\begin{align*}
   \sum_{\omega}A_{\alpha}(\omega) & =  \sum_{\omega,\epsilon'-\epsilon  = \omega} \Pi(\epsilon) A_{\alpha} \Pi(\epsilon') \\
   & = \sum_{\omega,\epsilon} \Pi(\epsilon) A_{\alpha} \Pi(\epsilon + \omega) \\
   & = A_{\alpha} = \sum_{\omega}A^{\dagger}_{\alpha}(\omega),
\end{align*}

con esto el Hamiltoniano de interacción en el cuadro de Schrodinger

\begin{equation*}
    \hat{H}_{I} = \sum_{\alpha,\omega}A_{\alpha}(\omega) \otimes B_{\alpha} = \sum_{\alpha,\omega}A^{\dagger}_{\alpha}(\omega) \otimes B^{\dagger}_{\alpha}.
\end{equation*}

Aplicando las relaciones en el cuadro de interacción

\begin{equation}
    \hat{H}_{I}(t) = \sum_{\alpha,\omega}e^{-i\omega t}A_{\alpha}(\omega) \otimes B_{\alpha}(t) = \sum_{\alpha,\omega}e^{i\omega t}A^{\dagger}_{\alpha}(\omega) \otimes B^{\dagger}_{\alpha}(t),
    \label{seclindbladinteraction1}
\end{equation}
donde $B_{\alpha}(t) = e^{i\hat{H}_{B}t}B_{\alpha}e^{-i\hat{H}_{B}t}$. La ecuación \ref{seclindbladinteraction1} se puede introducir en la ecuación \ref{eq3sec2:markov1} y se obtiene 

\begin{align*}
    \frac{d}{dt}\hat{\rho}_{IS}(t) = \int_{0}^{\infty} ds \text{Tr}_{B}\left[  \hat{H}_{I}(t-s)\hat{\rho}_{IS}(t)\hat{\rho}_{B}\hat{H}_{I}(t) - \hat{\rho}_{IS}(t)\hat{\rho}_{B}\hat{H}_{I}(t-s)\hat{H}_{I}(t)    \right.\\
    \left. + \hat{H}_{I}(t)\hat{\rho}_{IS}(t)\hat{\rho}_{B}\hat{H}_{I}(t-s) -  \hat{H}_{I}(t)\hat{H}_{I}(t-s)\hat{\rho}_{IS}(t)\hat{\rho}_{B}  \right],
\end{align*}
así, reemplazando la forma explícita de la interacción  

\begin{align*}
    \frac{d}{dt}\hat{\rho}_{IS}(t) & = \sum_{\omega,\omega'}\sum_{\alpha,\beta} \int_{0}^{\infty} ds  \text{Tr}_{B}[e^{i\omega s}B_{\beta}(t-s)\hat{\rho}_{B}B^{\dagger}_{\alpha}(t)]e^{i(\omega'- \omega)t}\left(A_{\beta}(\omega)\hat{\rho}_{IS}(t)A^{\dagger}_{\alpha}(\omega') - A^{\dagger}_{\alpha}(\omega')A_{\beta}(\omega) \hat{\rho}_{IS}(t) \right) \\
     + & \sum_{\omega,\omega'}\sum_{\alpha,\beta} \int_{0}^{\infty} ds  \text{Tr}_{B}[e^{-i\omega s}B_{\alpha}(t)\hat{\rho}_{B}B^{\dagger}_{\beta}(t-s)]e^{-i(\omega'- \omega)t}\left(A_{\alpha}(\omega')\hat{\rho}_{IS}(t)A^{\dagger}_{\beta}(\omega) - \hat{\rho}_{IS}(t)A^{\dagger}_{\beta}(\omega)A_{\alpha}(\omega') \right).
\end{align*}

Se pueden definir las funciones correlación del baño por

\begin{equation*}
    \Gamma_{\alpha\beta}(\omega) = \int_{0}^{\infty}ds e^{i\omega s}\text{Tr}_{B}[B^{\dagger}_{\alpha}(t)B_{\beta}(t-s)\hat{\rho}_{B}],
\end{equation*}
y aplicando estas funciones queda 

\begin{align*}
    \frac{d}{dt}\hat{\rho}_{IS}(t) & = \sum_{\omega,\omega'}\sum_{\alpha,\beta} e^{i(\omega'- \omega)t}\Gamma_{\alpha \beta}(\omega)\left(A_{\beta}(\omega)\hat{\rho}_{IS}(t)A^{\dagger}_{\alpha}(\omega') - A^{\dagger}_{\alpha}(\omega')A_{\beta}(\omega) \hat{\rho}_{IS}(t) \right) \\
    & + \sum_{\omega,\omega'}\sum_{\alpha,\beta} e^{-i(\omega'- \omega)t}\Gamma^{*}_{\beta \alpha}(\omega) \left(A_{\alpha}(\omega')\hat{\rho}_{IS}(t)A^{\dagger}_{\beta}(\omega) - \hat{\rho}_{IS}(t)A^{\dagger}_{\beta}(\omega)A_{\alpha}(\omega') \right).
\end{align*}

Si se considera el tiempo característico de evolución del sistema $S$ como $\tau_S = |\omega - \omega'|^{-1}$, y además se cumple que el tiempo de relajación $\tau_R \gg \tau_S$, entonces los términos no seculares (aquellos con $\omega \neq \omega'$) pueden despreciarse. Esto se debe a que los factores oscilatorios $e^{i(\omega - \omega')t}$ varían rápidamente en la escala de tiempo $\tau_R$, promediándose a cero.

Al tomar esta aproximación, la evolución del sistema queda
\begin{align*}
    \frac{d}{dt}\hat{\rho}_{IS}(t) & = \sum_{\omega}\sum_{\alpha,\beta} \Gamma_{\alpha \beta}(\omega)\left(A_{\beta}(\omega)\hat{\rho}_{IS}(t)A^{\dagger}_{\alpha}(\omega) - A^{\dagger}_{\alpha}(\omega)A_{\beta}(\omega) \hat{\rho}_{IS}(t) \right) \\
    & + \sum_{\omega}\sum_{\alpha,\beta} \Gamma^{*}_{\beta \alpha}(\omega) \left(A_{\alpha}(\omega)\hat{\rho}_{IS}(t)A^{\dagger}_{\beta}(\omega) - \hat{\rho}_{IS}(t)A^{\dagger}_{\beta}(\omega)A_{\alpha}(\omega) \right),
\end{align*}
finalmente, al separar la función correlación espectral en su parte real e imaginaria como

\begin{equation*}
\Gamma_{\alpha \beta}(\omega) = \gamma_{\alpha \beta}(\omega)/2 + iS_{\alpha \beta}(\omega).
\end{equation*}

Se obtiene la ecuación de Lindblad 

\begin{equation}
    \frac{d}{dt} \hat{\rho}_{IS}(t) = -i[\hat{H}_{LS},\hat{\rho}_{IS}(t)] + \mathcal{D}(\hat{\rho}_{IS}(t)),
    \label{seclindbladfinal}
\end{equation}
con $\hat{H}_{LS}$ el Hamiltoniano \textit{Lamb Shift}

\begin{equation*}
    \hat{H}_{LS} = \sum_{\omega} \sum_{\alpha,\beta} S_{\alpha,\beta}(\omega)A^{\dagger}_{\alpha}(\omega)A_{\beta}(\omega), 
\end{equation*}
y el Disipador

\begin{equation*}
    \mathcal{D}(\hat{\rho}_{IS}(t)) = \sum_{\omega}\sum_{\alpha,\beta} \gamma_{\alpha \beta}(\omega) \left[ A_{\beta}(\omega)\hat{\rho}_{IS}(t)A^{\dagger}_{\alpha}(\omega) - \frac{1}{2}\{A^{\dagger}_{\alpha}(\omega)A_{\beta}(\omega), \hat{\rho}_{IS}(t)  \} \right].
\end{equation*}

El escribir el disipador de esta manera es importante, ya que esta evolución se encuentra en la forma GKLS que permite preservar traza, hermiticidad y positividad de la matriz densidad\cite{manzano2020short}. 

\label{sec:lindblad}


\chapter{Estadística de conteo(\textit{Full Counting Statistics}) }
En este capítulo se presenta el formalismo de \textit{Full Counting Statistics}(FCS). La sección \ref{sec3workheat} esta dedicada a la  descripción de las cantidades termodinámicas de interés, y en la sección \ref{Leyestermo} se establecen las leyes de la termodinámica en el contexto del modelo. En la sección \ref{sec2:estadistica2puntos} se desarrolla la estadística de medición en 2 puntos, mientras que en la sección \ref{sec2:superop} se describe brevemente el formalismo de superoperadores. Finalmente, en la sección \ref{sec2:master} se desarrolla el cálculo de la ecuación maestra generalizada deducida en \cite{potts2021thermodynamically}. 

\section{Cantidades termodinámicas}
Se considera el sistema descrito por el Hamiltoniano

\begin{equation*}
    \hat{H}_{tot}(t) = \hat{H}_{S}(t) + \sum_{\alpha}(\hat{H}_{\alpha} + \hat{V}_{\alpha})  = \hat{H}_{S}(t) + \hat{H}_{B} + \hat{V},
\end{equation*}
donde el primer término describe el Hamiltoniano del sistema el cuál puede ser tiempo dependiente, la segunda parte describe los reservorios térmicos y por último la tercera parte constituye el acoplamiento entre sistema-baño.\\
El sistema intercambia energía y partículas con el reservorio, por ende el cambio de energía puede ser dividido por una contribución correspondiente al calor y otra correspondiente al trabajo. Así, se define el calor que libera el baño $\alpha$ durante el intervalo de tiempo $[0,t]$ como

\begin{equation}
    \langle Q_{\alpha}\rangle = - \text{Tr}\{(\hat{H}_{\alpha} - \mu_{\alpha}\hat{N}_{\alpha})\hat{\rho}_{tot}(t) \} + \text{Tr}\{(\hat{H}_{\alpha} - \mu_{\alpha}\hat{N}_{\alpha})\hat{\rho}_{tot}(0) \},
\label{sec3:calor}
\end{equation}
en la cuál $\hat{N}_{\alpha}$ corresponde al operador de número que cuantifica el número de partículas en el baño $\alpha$ y $\mu_{\alpha}$ es su potencial químico. El trabajo promedio que entrega el reservorio $\alpha$ se define como

\begin{equation}
    \langle W_{\alpha}\rangle = - \mu_{\alpha} (\text{Tr}\{\hat{N}_{\alpha} \hat{\rho}_{tot}(t) \} - \text{Tr}\{\hat{N}_{\alpha}\hat{\rho}_{tot}(0) \}  ).
    \label{sec3:trabajo}
\end{equation}
\label{sec3workheat}

\newpage

\section{Leyes de la termodinámica}
\label{Leyestermo}
\subsection{Ley cero}
Si se tiene un sistema total descrito por un sistema reducido y un baño, que se encuentran en equilibrio a temperatura inversa $\beta$ y potencial químico $\mu$. De tal manera que el estado del sistema en equilibrio es el equilibrio gran canónico

\begin{equation*}
    \hat{\rho}^{eq}_{tot} = \frac{e^{-\beta(\hat{H}_{tot} - \mu \hat{N}_{tot})}}{Z}  \hspace{28mm} Z = \text{Tr}\{e^{-\beta(\hat{H}_{tot} - \mu \hat{N}_{tot})} \},
\end{equation*}
y por lo tanto el estado de equilibrio del sistema reducido es

\begin{equation*}
    \hat{\rho}_{S} = \frac{1}{Z}\text{Tr}_{B}\{ e^{-\beta(\hat{H}_{tot} - \mu \hat{N}_{tot})} \},
\end{equation*}
que en el límite de acoplamiento débil entre el sistema y el baño, este equilibrio se convierte en \cite{geva2000second}
\begin{equation*}
    \hat{\rho}_{S} = \frac{e^{-\beta(\hat{H}_{S} - \mu\hat{N}_{S})}}{\text{Tr}_{S}\{e^{-\beta(\hat{H}_{S} - \mu \hat{N}_{S})} \} }.
\end{equation*}

\subsection{Primera Ley}
Para escribir la primera ley, primero se debe introducir la corriente de calor y la potencia entregadas por el baño $\alpha$

\begin{equation*}
    J_{\alpha}(t) = \partial_{t}\langle Q_{\alpha}\rangle \hspace{12mm}  P_{\alpha}(t) = \partial_{t}\langle W_{\alpha}\rangle.
\end{equation*}

En el límite de acoplamiento débil para un Hamiltoniano tiempo independiente la primera ley consiste en 
\begin{equation*}
    \partial_{t}E(t) = \sum_{\alpha}[J_{\alpha}(t) + P_{\alpha}(t) ] \hspace{12mm} E = \text{Tr}\{\hat{H}_{S}\hat{\rho}_{tot}(t) \},
\end{equation*}
donde $E(t)$ consiste en la energía interna del sistema, mientras que la corriente de calor y la potencia entregada por el baño $\alpha$ en función de la matriz densidad del sistema será 

\begin{equation*}
    J_{\alpha}(t) = \text{Tr}\{(\hat{H}_{S} - \mu_{\alpha}\hat{N}_S)\mathcal{L}_{\alpha}\hat{\rho}_{S}(t) \} \hspace{12mm} P_{\alpha} = \mu_{\alpha} \text{Tr}\{\hat{N}_{S}\mathcal{L}_{\alpha}\hat{\rho}_{S}(t) \}. 
\end{equation*}

\subsection{Segunda Ley}
Para introducir la segunda ley se parte de la condición inicial en la que el sistema y el reservorio son sistemas no correlacionados, es decir 

\begin{equation*}
    \hat{\rho}_{tot}(0) = \hat{\rho}_{S}(0)\otimes_{\alpha}\hat{\tau}_{\alpha} \hspace{12mm} \hat{\tau}_{\alpha} = \frac{e^{-\beta_{\alpha}(\hat{H}_{\alpha}-\mu_{\alpha}\hat{N}_{\alpha}) }}{ \text{Tr}\{e^{-\beta_{\alpha}(\hat{H}_{\alpha} - \mu_{\alpha}\hat{N}_{\alpha})}\} },
\end{equation*}
donde cada reservorio se encuentra en equilibrio gran canónico con inverso de la temperatura $\beta_{\alpha}$ y potencial químico $\mu_{\alpha}$. Con esta condición inicial, la segunda ley puede ser escrita en función de la producción de entropía $\sigma(t)$ como \cite{esposito2010entropy}

\begin{equation*}
    \sigma(t) \equiv \Delta S(t) - \sum_{\alpha}\beta_{\alpha}\langle Q_{\alpha}\rangle \geq 0.
\end{equation*}

$\Delta S$ denota el cambio en la entropía de Von Neumann del sistema

\begin{equation*}
    \Delta S(t) = -  \text{Tr}\{\hat{\rho}_{S}(t)\ln \hat{\rho}_{S}(t) \} +  \text{Tr}\{ \hat{\rho}_{S}(0)\ln \hat{\rho}_{S}(0) \},
\end{equation*}
es decir, la producción de entropía se separa en una parte que corresponde al cambio de entropía del sistema, mientras que la segunda parte corresponde a la contribución debido al calor que entrega el ambiente.\\
Como en este trabajo se hará análisis de flujos continuos tanto de energía como de partículas, es importante estudiar la tasa de producción de entropía. Si bien, la producción de entropía siempre es positiva, la taza de producción de entropía no siempre lo es, aún así para sistemas markovianos se cumple que \cite{strasberg2019non}

\begin{equation*}
    \dot{\sigma}(t) \equiv \partial_{t}\Delta S(t) - \sum_{\alpha}\beta_{\alpha}J_{\alpha} \geq 0.
\end{equation*}

Y la igualdad se cumple para procesos reversibles. Con esto se han definido las leyes de la termodinámica en función de flujos continuos, que se podrán calcular a través de la matriz densidad reducida. En esta tesis estas cantidades serán estudiadas principalmente en el estado estacionario.

\label{sec3sub:leyestermo}

\section{Estadística de medición en dos puntos}
A lo largo de este trabajo, se pretende calcular diversas cantidades termodinámicas asociadas a la dinámica del sistema, tales como la energía y el trabajo. Si bien los valores promedio de estas cantidades pueden obtenerse con las expresiones \ref{sec3:calor} y \ref{sec3:trabajo}, tanto el calor y el trabajo intercambiados con los reservorios deben considerarse como variables aleatorias, ya que su determinación implica medir sobre el estado de los reservorios. Debido a la naturaleza cuántica de la evolución, dichas mediciones pueden arrojar distintos resultados en diferentes repeticiones del experimento. Por lo tanto, para acceder a los valores promedio, es útil calcular la distribución de probabilidad del calor y el trabajo intercambiados por el reservorio.

Para  un sistema total descrito por la condición inicial $\hat{\rho}_{tot}(0) = \hat{\rho}_{S}(0) \otimes_{\alpha} \hat{\tau}_{\alpha}$ la distribución de probabilidad es   

\begin{align*}
    P(\textbf{Q},\textbf{W}) & = \sum_{\textbf{E},\textbf{E}',\textbf{N},\textbf{N}'} P_{t}(\textbf{E}',\textbf{N}'|\textbf{E},\textbf{N}) P_{0}(\textbf{E},\textbf{N})\\
                             & \times \Pi_{\alpha} \delta(W_{\alpha} - \mu_{\alpha}(N_{\alpha} - N'_{\alpha})) \delta(Q_{\alpha} + W_{\alpha}  - (E_{\alpha} - E'_{\alpha})).   
\end{align*}

La probabilidad conjunta de que cada baño $\alpha$ tenga energía $E_{\alpha}$ y número de partículas $N_{\alpha}$ a tiempo $t=0$ es

\begin{equation*}
    P_{0}(\textbf{E},\textbf{N}) = \Pi_{\alpha}  \frac{e^{-\beta_{\alpha}(E_{\alpha} - \mu_{\alpha}N_{\alpha} )  }}{ \text{Tr}\{e^{-\beta_{\alpha}(E_{\alpha} - \mu_{\alpha}N_{\alpha} )  }\} }.
\end{equation*}

Para obtener $P_{t}(\textbf{E}',\textbf{N}'|\textbf{E},\textbf{N})$ que corresponde a la probabilidad condicional de que los reservorios tengan energías $\textbf{E}'$ y números de partículas $\textbf{N}'$ a tiempo $t$, dado que inicialmente se midieron los valores $\textbf{E}$ y $\textbf{N}$, se debe partir del estado proyectado a $t=0$ correspondiente a dicha medición inicial, es decir

\begin{equation*}
    \hat{\rho}'(0) = \frac{\hat{P}_{\textbf{E}, \textbf{N} }\hat{\rho}(0) \hat{P}_{\textbf{E}, \textbf{N} } }{\text{Tr}\{\hat{P}_{\textbf{E}, \textbf{N} }\hat{\rho}(0) \hat{P}_{\textbf{E}, \textbf{N} } \} },
\end{equation*}
donde $\hat{P}_{\textbf{E}, \textbf{N} } = I_{S} \otimes \hat{P}^{B}_{\textbf{E}, \textbf{N} }$ es el proyector asociado a una medición de las energías y los números de partículas en los reservorios. A continuación, el sistema evoluciona de forma unitaria hasta un tiempo $t$

\begin{equation*}
    \hat{\rho}'(t) = \hat{U}(t)\hat{\rho}'(0)\hat{U}^{\dagger}(t),
\end{equation*}
finalmente se realiza una segunda medición a tiempo $t$, obteniéndose los valores $\textbf{E}'$ y $\textbf{N}'$. Así, la probabilidad condicional correspondiente es

\begin{equation*}
    P_{t}(\textbf{E}',\textbf{N}'|\textbf{E},\textbf{N}) = \text{Tr}\{\hat{P}_{\textbf{E}', \textbf{N}' }\hat{\rho}'(t) \hat{P}_{\textbf{E}', \textbf{N}' } \},
\end{equation*}
escribiendo los proyectores de manera explícita $\hat{P}^{B}_{\textbf{E}, \textbf{N} } = |\textbf{E}, \textbf{N} \rangle\langle \textbf{E}, \textbf{N}|$, se puede obtener 

\begin{align*} 
    P_{t}(\textbf{E}',\textbf{N}'|\textbf{E},\textbf{N}) & =  \text{Tr}\{\hat{U}(t)(\hat{\rho}_{S}(0)\otimes |\textbf{E}, \textbf{N}\rangle  \langle \textbf{E}, \textbf{N}| ) \hat{U}^{\dagger}(t)  |\textbf{E}', \textbf{N}'\rangle  \langle \textbf{E}', \textbf{N}'| \} \\
        & =  \text{Tr}\{|\textbf{E},\textbf{N} \rangle \langle \textbf{E}', \textbf{N}'| \hat{U}(t)\hat{\rho}_{S}(0) \langle \textbf{E}, \textbf{N}|\hat{U}^{\dagger}(t)|\textbf{E}', \textbf{N}'\rangle \}      \\ 
        & = \text{Tr}_{S}\{ \text{Tr}_{B}\{|\textbf{E},\textbf{N} \rangle \langle \textbf{E}', \textbf{N}'|\hat{U}(t) \}\hat{\rho}_{S}(0)\langle \textbf{E}, \textbf{N}|\hat{U}^{\dagger}(t)|\textbf{E}', \textbf{N}'\rangle      \} \\
        & = \text{Tr}_{s}\{ \text{Tr}_{B}\{|\textbf{E},\textbf{N}\rangle \langle \textbf{E}',\textbf{N}'|\hat{U}(t)  \} \hat{\rho}_{S}(0) \text{Tr}_{B}\{\hat{U}^{\dagger}(t) |\textbf{E}',\textbf{N}' \rangle \langle \textbf{E},\textbf{N}| \}     \}.
    \end{align*}    

Por ende, $P_{t}(\textbf{E}',\textbf{N}'|\textbf{E},\textbf{N})$ se puede expresar como
\begin{equation*}
    P_{t}(\textbf{E}',\textbf{N}'|\textbf{E},\textbf{N}) = \text{Tr}_{S}\{\hat{M} \hat{\rho}_{s}(0)\hat{M}^{\dagger} \}  \hspace{10mm} \hat{M} = \text{Tr}_{B}\{|\textbf{E},\textbf{N} \rangle \langle \textbf{E}',\textbf{N}' | \hat{U}(t)\}.
\end{equation*}

Al contar con una distribución de probabilidad, se puede construir la función generadora de momentos, a partir de la cuál pueden obtenerse cantidades como el promedio o la varianza. Esta función es

\begin{equation}
    \Lambda(\vec{\lambda},\vec{\chi}) \equiv \int d\textbf{Q} d\textbf{W}P(\textbf{Q},\textbf{W}) e^{-i\vec{\lambda}\cdot \textbf{Q} -i\vec{\chi}\cdot \textbf{W} },
\label{sec2funciongeneradora}
\end{equation}
que se puede escribir en función de la evolución de una matriz densidad auxiliar \ref{apendix:fcs1}

\begin{equation}
    \Lambda(\vec{\lambda},\vec{\chi}) = \text{Tr}\{\hat{\rho}_{tot}(\vec{\lambda},\vec{\chi};t) \}    \hspace{14mm} \hat{\rho}_{tot}(\vec{\lambda},\vec{\chi};t) = \hat{U}(\vec{\lambda},\vec{\chi};t) \hat{\rho}_{tot}(0) \hat{U}^{\dagger}(\vec{\lambda},\vec{\chi};t),
    \label{sec2:evolucionconteo}
\end{equation}
y 

\begin{equation*}
    \hat{U}(\vec{\lambda},\vec{\chi};t) = e^{\frac{i}{2}\sum_{\alpha}[\lambda_{\alpha}(\hat{H}_{\alpha} - \mu_{\alpha}\hat{N}_{\alpha} ) + \chi_{\alpha}\mu_{\alpha}\hat{N}_{\alpha} ]  } \hat{U}(t) e^{-\frac{i}{2}\sum_{\alpha}[ \lambda_{\alpha}(\hat{H}_{\alpha} - \mu_{\alpha}\hat{N}_{\alpha}) + \chi_{\alpha}\mu_{\alpha}\hat{N}_{\alpha} ]},
\end{equation*}
$\vec{\lambda}$ y $\vec{\chi}$ se denominan los parámetros de conteo o \textit{Counting Fields}, y $\hat{\rho}(\vec{\lambda},\vec{\chi};t)$ se denomina la matriz densidad generalizada. Conocer la evolución de esta matriz permite acceder a los momentos del calor, el trabajo y en general, del observable que se quiera estudiar. Sin embargo, dado que esta matriz se define a partir de la matriz densidad total, el siguiente paso consiste en derivar una ecuación maestra efectiva para los grados de libertad del sistema reducido.

\label{sec2:estadistica2puntos}



\section{Formalismo de Superoperadores y Espacio de Liouville}
Un operador en el espacio de Hilbert, representado por $\hat{\rho}$ y de dimensión $N\times N$, puede mapearse al espacio de Liouville mediante  un vector columna $|\rho \rangle \rangle$ de dimension $N^{2}\times 1$. De forma análoga, un superoperador \( L \), que actúa sobre \( \hat{\rho} \) en el espacio de Hilbert, se convierte en una matriz \( \check{L} \) de dimensión \( N^2 \times N^2 \) que actúa sobre \( |\rho\rangle\rangle \) en el espacio de Liouville. En este espacio se definen las siguientes operaciones

\begin{equation*}
    \langle \langle A|B\rangle \rangle  \equiv \text{Tr}\{\hat{A}^{\dagger}\hat{B}\}
\end{equation*}

\begin{equation*}
    \check{1}  \equiv \sum_{n,n'}|nn'\rangle \rangle \langle \langle nn'|
\end{equation*}

\begin{equation*}
     |nn'\rangle \rangle   \to |n\rangle \langle n'|  \hspace{10mm}  \langle \langle nn'| \to |n'\rangle \langle n|,
\end{equation*}
además, se cumple que 

\begin{equation*}
     \langle \langle nn'|mm'\rangle \rangle  = \delta_{nm}\delta_{n'm'}
\end{equation*}

\begin{equation*}
    \langle \langle nn'|A\rangle \rangle  = \langle n|\hat{A}|n'\rangle
\end{equation*}

\begin{equation*}
     \langle \langle 1|A\rangle \rangle  = \text{Tr}\{\hat{A}\}.
\end{equation*}


Si la evolución de la matriz densidad $|\hat{\rho}(t)\rangle \rangle$ esta gobernada por el superoperador $\check{\mathcal{L}}$, entonces la evolución temporal en el espacio de Liouville es 

\begin{equation*}
    \frac{d|\hat{\rho}(t) \rangle \rangle}{dt} = \check{\mathcal{L}}|\hat{\rho}(t) \rangle \rangle,
\end{equation*}
la solución formal de esta ecuación consiste en

\begin{equation}
    |\hat{\rho}(t)\rangle \rangle = e^{\check{\mathcal{L}}t}|\hat{\rho}(0)\rangle \rangle. 
    \label{sec2liouvilleformal}
\end{equation}

Además, existen los superoperadores de proyección de Nakajima-Zwanzig, que actúan sobre la matriz densidad total. En este contexto, se define el superoperador \( \check{\mathcal{P}} \), que proyecta sobre la parte relevante de la dinámica, mientras que \( \check{\mathcal{Q}} = 1 - \check{\mathcal{P}} \) proyecta sobre la parte irrelevante\cite{zwanzig1966statistical}. Que cumplen con las propiedades

\begin{align*}
    & \check{\mathcal{P}} + \check{\mathcal{Q}} = \check{1} \\
    & \check{\mathcal{P}}^{2} = \check{\mathcal{P}} \\
    & \check{\mathcal{Q}}^{2} = \check{\mathcal{Q}} \\
    & \check{\mathcal{P}}\check{\mathcal{Q}} = \check{\mathcal{Q}}\check{\mathcal{P}} = 0.
\end{align*}    

Finalmente, la evolución temporal de la matriz densidad a través de estos proyectores es 

\begin{align*}
    \frac{d}{dt}\check{\mathcal{P}}|\hat{\rho}(t)\rangle \rangle & = \check{\mathcal{P}}\check{\mathcal{L}}\check{\mathcal{P}}|\hat{\rho}(t)\rangle \rangle  + \check{\mathcal{P}}\check{\mathcal{L}}\check{\mathcal{Q}}|\hat{\rho}(t)\rangle \rangle \\
    \frac{d}{dt}\check{\mathcal{Q}}|\hat{\rho}(t)\rangle \rangle  & = \check{\mathcal{Q}}\check{\mathcal{L}}\check{\mathcal{Q}}|\hat{\rho}(t)\rangle \rangle  + \check{\mathcal{Q}}\check{\mathcal{L}}\check{\mathcal{P}}|\hat{\rho}(t)\rangle \rangle.
\end{align*}

\label{sec2:superop}

\newpage

\section{Ecuación maestra generalizada}
Si se considera un Hamiltoniano total $\hat{H} = \hat{H}_{S} + \hat{H}_{B} + \epsilon \hat{H}_{I} = \hat{H}_{0} + \epsilon \hat{H}_{I}$, con $\epsilon$ un parámetro adimensional que posteriormente permite aplicar acoplamiento débil. Y se introduce este Hamiltoniano en la ecuación \ref{sec2:evolucionconteo}, se obtiene 

\begin{equation}
    \frac{d}{dt}\hat{\rho}_{tot}(\vec{\lambda},\vec{\chi},t) = -i[\hat{H}_{0},\hat{\rho}_{tot}(\vec{\lambda},\vec{\chi},t)] - i\epsilon[\hat{V}_{\lambda} \hat{\rho}_{tot}(\vec{\lambda},\vec{\chi},t) - \hat{\rho}_{tot}(\vec{\lambda},\vec{\chi},t)\hat{V}_{-\lambda}],
    \label{sec3:ecgeneral}
\end{equation}
con el operador $\hat{V}_{\lambda}$ descrito por

\begin{equation*}
    \hat{V}_{\lambda} = e^{-\frac{i}{2}\hat{A}(\lambda,\chi)}\hat{H}_{I}e^{\frac{i}{2}\hat{A}(\lambda,\chi)}, \hspace{14mm}  
\end{equation*}
y $\hat{A}(\lambda,\chi) = -\sum_{\alpha}[\lambda_{\alpha}(\hat{H}_{\alpha} - \mu_{\alpha}\hat{N}_{\alpha}) + \chi_{\alpha}\mu_{\alpha}\hat{N}_{\alpha} ]$. 

En el formalismo de superoperadores, \ref{sec3:ecgeneral} se transforma en 

\begin{align*}
    \frac{d}{dt}|\hat{\rho}_{tot}(\vec{\lambda},\vec{\chi},t)\rangle \rangle  & = \check{\mathcal{L}}_{\lambda}|\hat{\rho}_{tot}(\vec{\lambda},\vec{\chi},t)\rangle \rangle  \\  
        & = (\check{\mathcal{L}}_{0} + \epsilon \check{\mathcal{L}}'_{\lambda} )|\hat{\rho}_{tot}(\vec{\lambda},\vec{\chi},t)\rangle \rangle. 
\end{align*}

Si se utiliza el cuadro de interacción, esta ecuación se convierte en  

\begin{align*}
    \hat{\rho}_{totI}(\vec{\lambda},\vec{\chi},t) & = e^{-\mathcal{L}_{0}t}\hat{\rho}_{tot}(\vec{\lambda},\vec{\chi},t) \\
    & = e^{i\hat{H}_{0}t}\hat{\rho}_{tot}(\vec{\lambda},\vec{\chi},t)e^{-i\hat{H}_{0}t}.
\end{align*}

En el cuadro de interacción, el superoperador de Liouville que incorpora el parámetro de conteo se transforma en 

\begin{equation*}
    \check{\mathcal{L}}_{\lambda}(t) = e^{-\check{\mathcal{L}}_{0}t}\check{\mathcal{L}}_{\lambda}e^{\check{\mathcal{L}}_{0}t},
\end{equation*}
y la evolución temporal en el cuadro de interacción es 

\begin{equation}
    \frac{d}{dt}|\hat{\rho}_{totI}(\vec{\lambda},\vec{\chi},t)\rangle \rangle  = \epsilon \check{\mathcal{L}}_{\lambda}(t)|\hat{\rho}_{totI}(\vec{\lambda},\vec{\chi},t)\rangle \rangle,
 \label{sec2FCS:evolution}
\end{equation}
de este modo, la evolución de los grados de libertad del sistema reducido en el espacio de Hilbert queda descrita por 

\begin{multline}
    \dot{\hat{\rho}}_{IS}(\vec{\lambda},\vec{\chi},t) =  \epsilon^{2}\int_{0}^{t}ds \left[- \text{Tr}_{B}\{\hat{V}_{\lambda}(t)\hat{V}_{\lambda}(t-s)\hat{\rho}_{IS}(\vec{\lambda},\vec{\chi},t)\hat{\rho}^{eq}_{R} \} - \text{Tr}_{B}\{\hat{\rho}_{IS}(\vec{\lambda},\vec{\chi},t)\hat{\rho}^{eq}_{R}\hat{V}_{-\lambda}(t-s)\hat{V}_{-\lambda}(t) \} \right.\\
    \left. + \text{Tr}_{B}\{\hat{V}_{\lambda}(t)\hat{\rho}_{IS}(\vec{\lambda},\vec{\chi},t)\hat{\rho}^{eq}_{R}\hat{V}_{-\lambda}(t-s) \} + \text{Tr}_{B}\{ \hat{V}_{\lambda}(t-s)\hat{\rho}_{IS}(\vec{\lambda},\vec{\chi},t)\hat{\rho}^{eq}_{R}\hat{V}_{-\lambda}(t) \}  \right],
\label{ecmaestraVlambda}
\end{multline}
la demostración de esta ecuación se encuentra en el apéndice \ref{apendixsubsectionmatriz}.

Escribiendo una interacción de la forma

\begin{align*}
    \hat{V} & = \sum_{\alpha,k}\hat{S}_{\alpha,k}\hat{B}_{\alpha,k} \\
    \hat{V}_{\lambda} & = \sum_{\alpha,k}\hat{S}_{\alpha,k}\hat{B}_{\alpha,k,\lambda} \\
    \hat{B}_{\alpha,k,\lambda} & \equiv e^{(i/2)[\lambda_{\alpha}(\hat{H}_{\alpha} - \mu_{\alpha}\hat{N}_{\alpha}) + \chi_{\alpha}\mu_{\alpha}\hat{N}_{\alpha}]}\hat{B}_{\alpha,k}e^{-(i/2)[\lambda_{\alpha}(\hat{H}_{\alpha} - \mu_{\alpha}\hat{N}_{\alpha}) + \chi_{\alpha}\mu_{\alpha}\hat{N}_{\alpha}]},  
\end{align*}
donde los operadores del baño cumplen  
\begin{equation*}
    [\hat{B}_{\alpha,k},\hat{N}_{\alpha}] = n_{\alpha,k}\hat{B}_{\alpha,k},
\end{equation*}
con $n_{\alpha,k}$ el número de part\'iculas que se intercambia en la interacción $k$.
 
Y los operadores del sistema cumplen

\begin{equation*}
    \hat{U}^{\dagger}_{S}(t)\hat{S}_{\alpha,k}\hat{U}_{S}(t) = \sum_{j}e^{-i\omega_{j}t}\hat{S}_{\alpha,k;j},
\end{equation*}
con $\hat{S}_{\alpha,k;j}$ los operadores de salto y $\omega_{j}$ las frecuencias de Bohr del Hamiltoniano del sistema. 

Cabe mencionar que los operadores $\hat{S}_{\alpha,k}$ no es necesitan ser autoadjuntos, a diferencia de los  operadores de interacción considerados en \ref{sec:lindblad}.

Si se definen las funciones correlación como $C^{\alpha}_{k,k'}(s) = \text{Tr}\{e^{is\hat{H}_{\alpha} }\hat{B}^{\dagger}_{\alpha,k}e^{-is\hat{H}_{\alpha} }\hat{B}_{\alpha,t}\hat{\tau}_{\alpha}\}$, se obtiene la ecuación maestra generalizada

\begin{equation}
    \frac{d}{dt}\hat{\rho}_{IS}(\vec{\lambda},\vec{\chi},t) = - \sum_{\alpha,k,k';j,j'}e^{i(\omega_{j}-\omega_{j'})t}\int_{0}^{t}ds \mathcal{I}(s,t) ,
\label{ecmaestrafinal}
\end{equation}
donde

\begin{multline*}
    \mathcal{I}(s,t) = e^{i\omega_{j'}s} C^{\alpha}_{k,k'}(s)\hat{S}^{\dagger}_{\alpha,k;j}\hat{S}_{\alpha,k',j'}\hat{\rho}_{IS}(\vec{\lambda},\vec{\chi},t) + e^{-i\omega_{j}s}C^{\alpha}_{k,k'}(-s)\hat{\rho}_{IS}(\vec{\lambda},\vec{\chi},t)\hat{S}^{\dagger}_{\alpha,k;j}\hat{S}_{\alpha,k';j'} \\
    - e^{-i\mu_{\alpha}n_{\alpha,k}(\lambda_{\alpha} - \chi_{\alpha})}\left[e^{i\omega_{j'}s}C^{\alpha}_{k,k'}(s-\lambda_{\alpha}) + e^{-i\omega_{j}s}C^{\alpha}_{k,k'}(-s-\lambda_{\alpha})  \right]  \hat{S}_{\alpha,k';j'}\hat{\rho}_{IS}(\vec{\lambda},\vec{\chi},t)\hat{S}^{\dagger}_{\alpha,k;j}.
\end{multline*}

Está ecuación fue deducida en el Apéndice \ref{finalequation}.

\label{sec2:master}

\subsection{Resolución finita de energía}
De forma análoga a lo presentado en la sección \ref{sec:lindblad}, uno de los requisitos que se busca en la evolución de la ecuación maestra es que sea Markoviana. Para ello, se debe tomar el límite superior del integral en el tiempo, presente en la ecuación \ref{ecmaestrafinal}, hacia infinito. Esta aproximación es válida siempre que el tiempo de correlación del baño $\tau_{B}$ sea mucho menor al tiempo de relajación del sistema $\tau_{R}$. Sin embargo, en presencia de parámetros de conteo $\lambda_{\alpha}$, las funciones de correlación del baño adquieren una dependencia modificada en el tiempo, del tipo $C^{\alpha}_{k,k'}(\pm \tau - \lambda_\alpha)$. Por lo tanto, para que la aproximación markoviana sea válida en presencia de parámetros de conteo, se requiere que 

\begin{equation*}
    C^{\alpha}_{k,k'}(\pm \tau - \lambda_\alpha) \approx 0 \quad \text{para } \tau > \tau_{R}.
\end{equation*}

Esto implica que el régimen de validez de la aproximación markoviana es 

\begin{equation*}
    \tau_{B} \ll \tau_{R} \hspace{10mm} |\lambda_{\alpha}| \ll \tau_{R}.
\end{equation*}

Esto tiene repercusiones importantes, ya que implica que la resolución de diferencias de energía en la medición del calor sea finita. Esta limitación se debe a que el parámetro $\lambda_{\alpha}$ y el calor medido en el baño $Q_{\alpha}$ son variables conjugadas en la distribución de probabilidad de calor y trabajo. Esto implica que ambas obedecen una relación de incertidumbre \cite{folland1997uncertainty}. Como consecuencia las diferencias de energía del orden de $1/\tau_{R}$ dejan de ser confiables, ya que en ese régimen el valor promedio del calor es comparable con su varianza, es decir

\begin{equation*}
    \langle \Delta \lambda^{2}_{\alpha} \rangle \langle (\Delta Q_{\alpha})^{2}\rangle \geq \gamma \implies \langle (\Delta Q_{\alpha})^{2}\rangle  \geq \frac{\gamma}{\tau^{2}_{R}},
\end{equation*}
con $\gamma$ alguna constante positiva. La profundidad de este resultado radica en que al aplicar una aproximación markoviana, la evolución del sistema sufre, de forma inherente, una resolución limitada respecto al calor intercambiado con los reservorios. Esta limitación puede conducir a inconsistencias termodinámicas, ya que se pierde información sobre fluctuaciones relevantes a escalas energéticas del orden $1/\tau_{R}$. Por lo tanto, para garantizar una evolución termodinámicamente consistente, es necesario redefinir las leyes de la termodinámica considerando explícitamente la resolución finita del calor impuesta por la dinámica.

\label{sec2:finiteresol}

\subsection{Agrupación de frecuencias}
%La ecuación de Redfield no siempre preserva positividad, lo que puede generar la aparición de probabilidades negativas en la matriz densidad del sistema reducido. La forma más común de asegurar la positividad es usar la aproximación secular vista en la sección \ref{sec:lindblad} para obtener la forma GKLS, el problema de aplicar esta aproximación es que requiere que las frecuencias de Bohr estén bien separadas con respecto a $1/\tau_{R}$, por lo tanto el aplicar esta aproximación necesita que no hayan frecuencias de Bohr casi degeneradas, eliminando una parte de los efectos cuánticos, ya que se pierden las coherencias entre niveles de energías cercanos\cite{trushechkin2021unified}. Se puede considerar un esquema diferente que asegure positividad, partiendo del punto que la aproximación de Markov asegura que para dos frecuencias de transición distintas, se cumple que $|\omega_{j} - \omega_{j'}|\ll 1/\tau_{B}$ o $|\omega_{j}-\omega_{j'}|\gg 1/\tau_{R}$. Incluso, se pueden cumplir las dos opciones. Dependiendo de cuál se cumpla, podemos agrupar las frecuencias de transición en conjuntos $x_{q}$, tal que si se cumple la primera o la segunda inecuación, estan en el mismo o en diferentes grupos, matemáticamente se traduce en
La ecuación de Redfield no garantiza en general la preservación de la positividad, lo que puede llevar a la aparición de probabilidades negativas en la matriz densidad del sistema reducido. La forma más común de asegurar dicha positividad es aplicar la aproximación secular, vista en la sección \ref{sec:lindblad}, la cual conduce a una ecuación maestra en forma (GKLS).

Sin embargo, esta aproximación requiere que las frecuencias de Bohr estén bien separadas en comparación con \( 1/\tau_R \), lo cual excluye casos con frecuencias casi degeneradas. Esta condición impone una pérdida de efectos cuánticos significativos, ya que elimina los términos no seculares en la evolución temporal\cite{trushechkin2021unified}. Para enfrentar esta limitación, puede considerarse un enfoque alternativo que conserve la positividad sin eliminar los términos no seculares. 

Partimos de que la aproximación de Markov garantiza que para dos frecuencias de transición distintas, se cumple al menos una de las siguientes condiciones:
\[
|\omega_j - \omega_{j'}| \ll 1/\tau_B \quad \text{o} \quad |\omega_j - \omega_{j'}| \gg 1/\tau_R.
\]
Incluso, ambas pueden cumplirse simultáneamente. Es posible agrupar las frecuencias de transición en subconjuntos \( x_q \), tal que frecuencias que cumplen la primera desigualdad pertenecen al mismo conjunto, mientras que aquellas que cumplen la segunda pertenecen a conjuntos distintos. Esto se traduce matemáticamente en


\begin{align*}
    |\omega_{j}-\omega_{j'}| \ll 1/\tau_{B}  &\hspace{10mm} \omega_{j} \in x_{q}, \omega_{j'} \in x_{q} \\
    |\omega_{j}-\omega_{j'}| \gg 1/\tau_{R}  &\hspace{10mm} \omega_{j} \in x_{q}, \omega_{j'} \in x_{q'}.
\end{align*}

Notemos que para frecuencias $\omega_{j}$, $\omega_{j'}$ que están en distintos grupos, se cumple la aproximación secular, es decir los términos $e^{i(\omega_{j} - \omega_{j'})t}$ oscilan rápidamente, por ende en  promedio se anulan.

La funcion correlación  espectral $\Gamma^{\alpha}(\omega) = \int_{-\infty}^{\infty} ds e^{i \omega s} C^{\alpha}(s)$, representa la razón de transición del sistema entre estados con diferencia de energía $\omega$, inducida por el baño $\alpha$. Dentro de un mismo conjunto $x_{q}$, los términos $e^{i\omega_{j}s},e^{i\omega_{j'}s}$ contribuyen, respectivamente, a  $\Gamma^{\alpha}(\omega_{j})$ y $\Gamma^{\alpha}(\omega_{j'})$, es decir,  describen transiciones inducidas por el baño con energías cercanas $\omega_{j}$ y $\omega_{j'}$. Sin embargo, debido a la resolución finita del calor intercambiado con el reservorio, no es posible distinguir entre transiciones con frecuencias dentro del mismo conjunto $x_q$. Por lo tanto, en las funciones correlación espectral se deben sustituir las frecuencias individuales $\omega_{j} \in x_{q}$ por una frecuencia auxiliar $\omega_{q}$, es decir 

\begin{equation*}
    e^{i\omega_{j}s},e^{i\omega_{j'}s} \to e^{i\omega_{q}s} \hspace{10mm} |\omega_{q} - \omega_{j}| \ll 1/\tau_{B} \hspace{10mm} \forall \omega_{j} \in x_{q}.
\end{equation*}

Esta sustitución refleja la incapacidad de distinguir transiciones con energías dentro del mismo conjunto \( x_q \), y permite construir una descripción coarse-grained de la dinámica que respeta tanto la positividad como las limitaciones de resolución energética\cite{chruscinski2017brief}. \\
Usando este esquema en la ecuación \ref{ecmaestrafinal}, se obtiene la ecuación en la forma GKLS

\begin{equation*}
    \frac{d}{dt}\hat{\rho}_{IS}(\vec{\lambda},\vec{\chi},t) = -i[\hat{H}_{LS},\hat{\rho}_{IS}(\vec{\lambda},\vec{\chi},t)] + \sum_{\alpha}\tilde{\mathcal{L}}^{\chi_{\alpha},\lambda_{\alpha}}_{\alpha} \hat{\rho}_{IS}(\vec{\lambda},\vec{\chi},t),
\end{equation*}
con

\begin{equation*}
    \tilde{\mathcal{L}}^{\chi_{\alpha},\lambda_{\alpha}}_{\alpha}\hat{\rho} = \sum_{k,q}\Gamma^{\alpha}_{k}(\omega_{q}) \left[e^{i\lambda_{\alpha}\omega_{q} + i(\chi_{\alpha}-\lambda_{\alpha})\mu_{\alpha}n_{\alpha,k}}\hat{S}_{\alpha,k;q}(t)\hat{\rho}\hat{S}^{\dagger}_{\alpha,k;q}(t) - \frac{1}{2}\{\hat{S}^{\dagger}_{\alpha,k;q}(t)\hat{S}_{\alpha,k;q}(t),\hat{\rho} \} \right].
\end{equation*}

Los operadores de salto consisten en

\begin{equation*}
    \hat{S}_{\alpha,k;q}(t) = \sum_{\{j|\omega_{j}\in x_{q} \} } e^{-i\omega_{j}t}\hat{S}_{\alpha,k;j}.
\end{equation*}

Y el Hamiltoniano de \textit{Lamb Shift}

\begin{equation*}
    \hat{H}_{LS} = \sum_{\alpha,k;q} \Delta^{\alpha}_{k}(\omega_{q}) \hat{S}^{\dagger}_{\alpha,k;q}(t)\hat{S}_{\alpha,k;q}(t),
\end{equation*}
con las cantidades

\begin{equation*}
    \Gamma_{k}^{\alpha}(\omega) = \int_{-\infty}^{\infty}ds e^{i\omega s}C^{\alpha}_{k,k}(s) \hspace{10mm} \Delta^{\alpha}_{k}(\omega) = - \frac{i}{2} \int^{\infty}_{-\infty}ds e^{i\omega s} \text{sign}(s)C^{\alpha}_{k,k}(s).
\end{equation*}

En donde se asume por simplicidad $C^{\alpha}_{k,k'} \propto \delta_{k,k'}$. La demostración de esta ecuación está incluida en el apéndice \ref{apendixGKLSgeneral}. 

En el límite en que los parámetros de conteo tienden a cero, se obtiene  

\begin{equation*}
    \frac{d}{dt}\hat{\rho}_{IS}(t) = - i[\hat{H}_{LS}(t),\hat{\rho}_{IS}(t)] + \sum_{\alpha}\tilde{\mathcal{L}}_{\alpha} \hat{\rho}_{IS}(t),
\end{equation*}
con
\begin{equation*}
    \tilde{\mathcal{L}}_{\alpha} = \sum_{\{q|\omega_{q}>0\}} \sum_{k}\Gamma^{\alpha}_{k,k}(\omega_{q}) \left[ \mathcal{D}[\hat{S}_{\alpha,k,q}(t)] + e^{-\beta_{\alpha}(\omega_{q} - \mu_{\alpha}n_{\alpha,k})}\mathcal{D}[\hat{S}^{\dagger}_{\alpha,k,q}(t)]  \right].
\end{equation*}

La demostración de esta ecuación se encuentra en \ref{apendixKMS}.

Para un Hamiltoniano tiempo independiente, la ecuación maestra en el cuadro de Schrodinger es

\begin{equation}
    \frac{d}{dt}\hat{\rho}_{S} = -i [\hat{H}_{S}+ \hat{H}_{LS},\hat{\rho}_{S}(t)] + \sum_{\alpha}\mathcal{L}_{\alpha}(\hat{\rho}_{S}(t)),
\label{sec2schrodingerthermo}
\end{equation}
con

\begin{equation}
    \mathcal{L}_{\alpha} = \sum_{\{q|\omega_{q}>0\}} \sum_{k}\Gamma^{\alpha}_{k,k}(\omega_{q}) \left[ \mathcal{D}[\hat{S}_{\alpha,k,q}] + e^{-\beta_{\alpha}(\omega_{q} - \mu_{\alpha}n_{\alpha,k})}\mathcal{D}[\hat{S}^{\dagger}_{\alpha,k,q}]  \right].
\label{sec2lindbladconsistency}
\end{equation}

Hay dos límites importantes a considerar. El primero corresponde al caso en que todas las frecuencias satisfacen $|\omega_{j}-\omega_{j'}| \gg 1/\tau_{S}$, ya que se cumple que $\mathcal{D}[\hat{S}_{\alpha,k,q}] = \mathcal{D}[\hat{S}_{\alpha,k,j}]$, lo que permite realizar la aproximación secular entre todas las frecuencias de Bohr y recuperar la ecuación de Lindblad. En el caso de que se cumpla que $|\omega_{j}-\omega_{j'}| \ll 1/\tau_{B}$, todas las frecuencias se agrupan en un sólo grupo. Por lo tanto, los operadores del sistema cumplen que $\hat{S}_{\alpha,k;q} = \hat{S}_{\alpha,k}$, y la ecuación maestra queda descrita por operadores locales, lo que corresponde a la ecuación maestra local\cite{wichterich2007modeling}.
\section{Consistencia termodinámica}
Debido a la resolución finita de energía impuesta por la dinámica markoviana, para asegurar la consistencia termodinámica es necesario redefinir las leyes de la termodinámica. Como primer paso, se introduce el Hamiltoniano termodinámico $\hat{H}_{TD}$, el cual satisface la relación de conmutación

\begin{equation*}
    [\hat{S}_{\alpha,k,j},\hat{H}_{TD}] = \omega_{q}\hat{S}_{\alpha,k,j},
\end{equation*}
para todas las frecuencias $\omega_{j} \in x_{q}$. Este Hamiltoniano puede construirse mediante el Hamiltoniano $\hat{H}_{S}$, modificando sus autovalores de modo que todas las transiciones dentro de un mismo conjunto $x_q$ compartan una frecuencia $\omega_{q}$ para $\omega_{j} \in x_{q}$.

Para las leyes de la termodinámica, se redefine la energía interna

\begin{equation*}
    E(t) = \text{Tr}\{\hat{H}_{TD}\hat{\rho}(t) \}.
\end{equation*}

Finalmente, la corriente de calor y el trabajo entregado por el baño $\alpha$ se redefinen por

\begin{equation*}
    J_{\alpha}(t) = \text{Tr}\{(\hat{H}_{TD} - \mu_{\alpha}\hat{N}_S)\mathcal{L}_{\alpha}\hat{\rho}_{S}(t) \} \hspace{12mm} P_{\alpha} = \mu_{\alpha} \text{Tr}\{\hat{N}_{S}\mathcal{L}_{\alpha}\hat{\rho}_{S}(t) \}.
\end{equation*}

Está definición automáticamente cumple con la primera ley de la termodinámica, ya que al derivar la energía interna $\partial_{t}E(t) = \text{Tr}\{ \hat{H}_{TD}\partial_{t}\hat{\rho}_{S}(t) \}$ y utilizar la relación de conmutación $[\hat{H}_{TD},\hat{H}_{S} + \hat{H}_{LS}] = 0$. Se obtiene la primera ley

\begin{equation*}
    \partial_{t}E(t) = \sum_{\alpha}[J_{\alpha} + P_{\alpha}].
\end{equation*}

\subsection{Ley cero}
Usando la ecuación maestra, con los superoperadores \ref{sec2lindbladconsistency} se cumple que

\begin{equation}
    \mathcal{L}_{\alpha}e^{-\beta_{\alpha}(\hat{H}_{TD} - \mu_{\alpha}\hat{N}_{S})} = 0,
\label{sec2cerolaw}
\end{equation}
además, si los reservorios tienen la misma temperatura inversa $\beta$ y el mismo potencial químico $\mu$, el estado de Gibbs corresponde a 

\begin{equation*}
    \hat{\rho}_G = \frac{e^{-\beta(\hat{H}_{TD} - \mu \hat{N}_{S})}}{\text{Tr}\{ e^{-\beta(\hat{H}_{TD} - \mu \hat{N}_{S})}\}}.
\end{equation*}

\subsection{Segunda ley}
La tasa de producción de entropía será 

\begin{equation}
    \dot{\sigma} = - \frac{d}{dt}\text{Tr}\{\hat{\rho}_{S}(t) \ln \hat{\rho}_{S}(t) \} - \sum_{\alpha} \beta_{\alpha} J_{\alpha}(t) = -\sum_{\alpha} \text{Tr}\{(\mathcal{L}_{\alpha}\hat{\rho}_{S}(t))[\ln \hat{\rho}_{S}(t) - \ln \hat{\rho}_{G}(\beta_{\alpha},\mu_{\alpha})] \} \geq 0.
\label{sec2secondlaw}
\end{equation}

Donde en la última parte se considera el hecho de que $\rho_{G}(\beta_{\alpha},\mu_{\alpha})$ es estado estacionario de $\mathcal{L}_{\alpha}$, para utilizar la desigualdad de Spohn\cite{spohn2007irreversible}. La demostración de \ref{sec2cerolaw} y de \ref{sec2secondlaw} se encuentra en el apéndice \ref{apendix:thermolaws}. 

Con esto finalmente se logra definir las leyes de la termodinámica usando la ecuación maestra \ref{sec2schrodingerthermo}.


% ------------------------------------------------------------------------------
% NUEVO CAPÍTULO
% ------------------------------------------------------------------------------
\chapter{Flujos de información}
En este capítulo se introduce el concepto de información termodinámica, comenzando en la sección \ref{sec4:Demon} con una revisión del paradigma del Demonio de Maxwell. En la sección \ref{sec4:autonomo} se presenta el Demonio de Maxwell autónomo. Finalmente, en las secciones \ref{sec4:flujos} y \ref{sec4:flujos0} se presenta el concepto de flujos de información, utilizando herramientas como la termodinámica estocástica en contexto clásico y la matriz densidad en contexto cuántico\cite{horowitz2014thermodynamics,ptaszynski2019thermodynamics}.


\section{Demonio de Maxwell}
El demonio de Maxwell es un experimento mental esbozado por James Clerk Maxwell en su obra\cite{Maxwell_1871}. Consiste en una caja dividida en dos compartimientos, A y B, cada uno de ellos lleno con un gas ideal a temperatura $T$ y presión $P$, como se muestra en la Figura \ref{img:demon}. Entre ambos compartimientos existe una compuerta sin masa, que permite el paso de partículas de un lado a otro. Esta puerta es controlada por una entidad denominada "Demonio", la cual tiene la capacidad de abrirla y cerrarla sin costo energético. La característica esencial del Demonio es que conoce la velocidad de cada partícula en ambos compartimientos. 

Dado que la temperatura se relaciona con la velocidad promedio de las partículas, el Demonio puede discriminar entre partículas rápidas y lentas. De este modo, permite pasar del compartimiento A al B solo aquellas partículas cuya velocidad es mayor que la velocidad promedio, mientras que deja pasar del compartimiento B al A las partículas más lentas. Si se mantiene constante el número de partículas en cada compartimiento, este proceso da lugar a un aumento de la temperatura en A y una disminución en B, es decir, una diferencia \( \Delta T \) entre ambos lados.

El cambio de entropía del sistema es
 
\begin{align*}
    \Delta S & = \Delta S_{A} + \Delta S_{B} = C_{V}\left( \log \frac{T-\Delta T}{T} + \log \frac{T+\Delta T}{T} \right) \\
       & =  C_{V} \log \left( 1 - \frac{\Delta T^{2}}{T^{2}}  \right) < 0,
\end{align*}
Donde \( C_V \) representa la capacidad calorífica a volumen constante.

Del cálculo de la entropía total se concluye que, sin realizar trabajo externo, el Demonio logra una disminución de la entropía, lo cual representa una aparente violación de la segunda ley de la termodinámica.

Sin embargo, esta paradoja fue resuelta posteriormente por Rolf Landauer, quien ``exorcizó'' al Demonio al señalar que, para que este pueda conocer la velocidad de las partículas, debe realizar mediciones. El proceso de medición, y en particular el borrado de la información asociada a dichas mediciones, implica una disipación de energía. Este resultado, conocido como el principio de Landauer, restablece la validez de la segunda ley\cite{Landauer_1961}.

En sistemas no autónomos, es decir sistemas manipulados por un agente externo a través del control de cantidades macroscópicas, la paradoja de Maxwell ha sido abordada extensamente. Un ejemplo clásico es el motor de Szilard\cite{szilard1964decrease}, donde se ha logrado describir y ``exorcizar'' al Demonio cuantificando tanto la energía requerida para realizar una medición,  como el costo energético asociado al borrado de la memoria del Demonio, y el trabajo máximo que puede extraerse mediante retroalimentación (feedback) \cite{maruyama2009colloquium, sagawa2008second}. No obstante, el caso de sistemas autónomos, en los cuales no existe intervención externa y tanto la dinámica como la retroalimentación son generadas internamente, presenta características fundamentales que merecen un análisis más detallado.

\insertimage[\label{img:demon}]{ejemplos/Maxwelldemon1}{scale=0.9}{Esquema que representa al Demonio de Maxwell. Primero, consiste en los dos compartimientos que poseen el gas ideal distribuido de manera homogénea. Por último, al trasladar las partículas de un lado a otro, queda el compartimiento A con partículas frías mientras que el compartimiento B con partículas calientes. Esta figura fue tomada de \cite{link1} .}

\label{sec4:Demon}

\section{Demonio de Maxwell autónomo}
Muchos procesos físicos requieren la interacción entre un conjunto de subsistemas que forman un sistema global. Esta interacción entre los subsistemas no sólo incluye un intercambio de energía o partículas, sino que también incluye un intercambio de información a medida que se correlacionan estos subsistemas entre sí de manera autónoma, es decir, sin un factor externo el cuál realize un feedback en él.

Comprender como son utilizados estos flujos de información para hacer tareas útiles es de gran relevancia. Un ejemplo claro se encuentra en los sistemas biólogicos, donde ocurre adaptación sensorial, que corresponde al monitoreo que realiza un organismo a su ambiente mientras cambia su respuesta a él\cite{lan2012energy}. Para caracterizar este tipo de procesos,se considera un Demonio de Maxwell autónomo como un sistema bipartito. Donde una de sus partes consiste en el sistema controlado, mientras que la otra parte en el sistema que actúa como detector, el cuál ejerce control mediante una interacción física que afecta al sistema controlado.

Es fundamental que el sistema sea autónomo, es decir que el Hamiltoniano total sea tiempo independiente e intervenciones externas, tales como mediciones y feedback no son consideradas.

\label{sec4:autonomo}
\section{Descripción clásica}
Para describir un demonio de Maxwell autónomo, es necesario poder calcular la evolución de un sistema acoplado a uno o más reservorios, y por lo tanto, sujeto a las leyes de la termodinámica. 

Consideremos el caso de dos sistemas independientes, \( X \) e \( Y \), cada uno con estados discretos denotados por \( x \) e \( y \), respectivamente. Ambos sistemas presentan una dinámica interna caracterizada por saltos aleatorios entre sus respectivos estados, cuya tasa de transición está determinada por los reservorios locales a los que cada subsistema se encuentra acoplado. Estas tasas cumplen la condición de \textit{balance detallado local} \cite{van2015ensemble}.

La evolución entre estados se modela como un proceso de Markov\cite{van1992stochastic}. Para estudiar la evolución conjunta del sistema, se asume un acoplamiento bipartito: en este contexto, esto significa que si el sistema total se encuentra en un estado conjunto $(x,y)$, sólo se permiten transiciones del tipo $(x,y) \to (x,y')$ o $(x,y) \to (x',y)$, pero no transiciones simultáneas del tipo $(x,y) \to (x',y')$.

Esta estructura garantiza que la dinámica conjunta del sistema \( XY \) siga siendo markoviana, y que la probabilidad \( p(x,y) \) de encontrar al sistema en el estado \( (x,y) \) esté gobernada por la siguiente ecuación maestra:

\begin{equation*}
    d_{t}p(x,y) = \sum_{x',y'} \left[ W_{x,x'}^{y,y'}p(x',y')  - W_{x',x}^{y',y}p(x,y) \right].
\end{equation*}

Donde \( W_{x,x'}^{y,y'} \) representa la razón de transición para un salto del estado \( (x',y') \) al estado \( (x,y) \). Esta razón obedece la condición de balance detallado local:

\[
\ln \left( \frac{W_{x,x'}^{y,y'}}{W_{x',x}^{y',y}} \right) = -\frac{\epsilon_{x,y} - \epsilon_{x',y'}}{T},
\]

la cual relaciona las razones de transición con el cambio de energía \( \Delta \epsilon = \epsilon_{x,y} - \epsilon_{x',y'} \) asociado al salto. Esta energía es intercambiada con el reservorio térmico local en forma de calor.

Debido a que el sistema es bipartito, la forma de las razones de transición corresponde a

\begin{equation*}
    W_{x,x'}^{y,y'} = \left\{ \begin{array}{lcc} w_{x,x'}^{y} & si & x \neq x'; y=y' \\ \\ w_{x}^{y,y'} & si & x=x';y\neq y'\\ \\ 0 & si & x \neq x'; y \neq y' \end{array} \right..
\end{equation*}

En términos de la corriente de probabilidad 

\begin{equation*}
    J_{x,x'}^{y,y'} = W_{x,x'}^{y,y'}p(x',y') - W_{x',x}^{y',y}p(x,y),
\end{equation*}
la ecuación maestra puede reescribirse como 

\begin{equation*}
    d_{t}p(x,y) = \sum_{x',y'}J_{x,x'}^{y,y'} = \sum_{x'}J_{x,x'}^{y} + \sum_{y'}J_{x}^{y,y'}.
\end{equation*}

Donde  
\[
J_{x}^{y,y'} = w_{x}^{y,y'}\,p(x,y') - w_{x}^{y',y}\,p(x,y)
\]
representa la corriente de probabilidad correspondiente a un salto entre los estados \( y' \to y \), manteniendo fijo el estado \( x \). Es decir, describe el flujo en la dirección \( Y \) para un valor dado de \( x \). De manera análoga, $J^{y}_{x,x'}$ describe el flujo en la dirección \( X\) para un valor fijo de \(y\).  

Este hecho es importante, ya que la estructura bipartita permite separar las corrientes de probabilidad en dos componentes: una correspondiente a transiciones en la dirección \( X \), y otra en la dirección \( Y \). Esta propiedad puede ser aprovechada para estudiar funcionales de la corriente, es decir, cantidades de la forma
\[
\mathcal{A}(J) = \sum_{x,x',y,y'} J_{x,x'}^{y,y'}\, A_{x,x'}^{y,y'},
\]
que pueden ser separados en dos contribuciones

\begin{equation}
    \mathcal{A}(J) = \sum_{x\geq x';y \geq y'} J_{x,x'}^{y}A_{x,x'}^{y,y'} + \sum_{x \geq x'; y \geq y'}J_{x}^{y,y'} A_{x,x'}^{y,y'}.
\label{sec4:functionalcurrent}
\end{equation}

De este modo, la variación de \( \mathcal{A} \) se puede descomponer en dos contribuciones: una asociada a las transiciones en la dirección \( X \), y otra correspondiente a las transiciones en la dirección \( Y \).


\label{sec4:flujos}
\section{Segunda ley de la termodinámica y flujos de información}
El sistema conjunto \( XY \), al estar en contacto con un baño, constituye un sistema abierto que debe satisfacer la segunda ley de la termodinámica. Esta ley exige que la tasa de producción de entropía sea siempre mayor o igual que cero, es decir 

\begin{equation*}
    \dot{\sigma} = \partial_{t}S^{XY} + \dot{S}_{r} \geq 0. 
\end{equation*}

Donde $\partial_{t}S^{XY}$ corresponde a la derivada temporal de la entropía del sistema, que se expresa como  

\begin{equation*}
    \partial_{t}S^{XY} = \sum_{x\geq x'; y\geq y'} J_{x,x'}^{y,y'} \ln \frac{p(x',y')}{p(x,y)}.
\end{equation*}

Y el cambio de entropía en el ambiente corresponde a

\begin{equation*}
    \dot{S}_{r} = \sum_{x\geq x'; y\geq y'} J_{x,x'}^{y,y'} \ln \frac{W_{x,x'}^{y,y'}}{W_{x',x}^{y',y}},
\end{equation*}
por ende

\begin{equation*}
    \dot{\sigma} = \sum_{x\geq x'; y\geq y'} J_{x,x'}^{y,y'} \ln \frac{ W_{x,x'}^{y,y'}p(x',y')  }{ W_{x',x}^{y',y}p(x,y) } \geq 0.
\end{equation*}

Estos resultados se demuestran en el apendice \ref{apendix4:secondlaw}.  

Una vez definida la tasa de producción de entropía, es necesario cuantificar la información compartida entre los dos subsistemas. La magnitud que permite medir las correlaciones entre ellos es la información mutua, la cual captura las correlaciones estadísticas entre las variables \( X \) e \( Y \). Se define por

\begin{equation*}
    I_{xy} = \sum_{x,y} p(x,y) \ln \frac{p(x,y)}{p(x)p(y)} \geq 0. 
\end{equation*}

De este modo, cuando \( I_{XY} \) es grande, los dos subsistemas están fuertemente correlacionados, mientras que $I_{xy}$ pequeña implica que los dos sistemas se conocen poco entre ellos, \( I_{XY} = 0 \) indica que son estadísticamente independientes.

Para definir los flujos de información, es necesario calcular la variación temporal de la información mutua. Esta puede descomponerse en dos contribuciones asociadas a las dinámicas de cada subsistema:
\[
\partial_t I_{XY} = \dot{I}^{X} + \dot{I}^{Y},
\]
donde \( \dot{I}^{X} \) representa la contribución debida a la evolución del subsistema \( X \) (con \( Y \) fijo), y \( \dot{I}^{Y} \) la correspondiente al subsistema \( Y \). Que explícitamente son

\begin{align*}
    \dot{I}^{X} & = \sum_{x\geq x'; y}J_{x,x'}^{y} \ln \frac{ p(y|x) }{p(y|x')} \\
    \dot{I}^{Y} & = \sum_{x;y\geq y'} J_{x}^{y,y'} \ln \frac{p(x|y)}{ p(x|y') }.
\end{align*}

 La demostración de este resultado también está en el apendice \ref{apendix4:secondlaw}.

Los términos \( \dot{I}^{X} \) y \( \dot{I}^{Y} \) cuantifican cómo fluye la información entre los dos subsistemas. Si \( \dot{I}^{X} > 0 \), significa que un salto en la dirección \( X \), en promedio, aumenta la información mutua \( I_{xy} \); es decir, el subsistema \( X \) está ``aprendiendo'' o midiendo al subsistema \( Y \).

En cambio, si \( \dot{I}^{X} < 0 \), los saltos en la dirección \( X \) reducen las correlaciones, lo que puede interpretarse como un consumo de información con el fin de extraer trabajo o energía.

Como la tasa de producción de entropía es un funcional de la corriente de probabilidad, se puede usar la ecuación \ref{sec4:functionalcurrent} para escribir:


 \begin{equation*}
    \dot{\sigma} = \dot{\sigma}^{X} + \dot{\sigma}^{Y},
 \end{equation*}
así

\begin{align*}
    \dot{\sigma}^{X} & = \sum_{x \geq x';y} J_{x,x'}^{y} \ln  \frac{w_{x,x'}^{y} p(x',y) }{w_{x',x}^{y} p(x,y) } \geq 0 \\
    \dot{\sigma}^{Y} & = \sum_{x;y\geq y'}J_{x}^{y,y'} \ln \frac{w_{x}^{y,y'} p(x,y') }{ w_{x}^{y',y} p(x,y) } \geq 0, 
\end{align*}
donde se pueden identificar las tasas de producción de entropía locales

\begin{align*}
    \dot{\sigma}^{X} &  = \sum_{x \geq x';y} J_{x,x'}^{y} \left[ \ln \frac{p(x')}{p(x)}  +\ln \frac{w_{x,x'}^{y}}{ w_{x',x}^{y} } + \ln \frac{p(y|x')}{p(y|x)} \right] \\
    \dot{\sigma}^{Y} &  = \sum_{x;y \geq y'} J_{x}^{y,y'} \left[ \ln \frac{p(y')}{p(y)}  + \ln \frac{w_{x}^{y,y'}}{ w_{x}^{y',y} } + \ln \frac{p(x|y')}{p(x|y)} \right].
\end{align*}

Y en función de los flujos de información son  

\begin{align*}
    \dot{\sigma}^{X} & = \partial_{t}S^{X} + \dot{S}_{r}^{X} - \dot{I}^{X} \geq 0 \\
    \dot{\sigma}^{Y} & = \partial_{t}S^{Y} + \dot{S}_{r}^{Y} - \dot{I}^{Y} \geq 0.
\end{align*}

Estas dos ecuaciones permiten visualizar cómo la contribución de la información influye en la producción de entropía local de cada subsistema.

Supongamos que se desconoce la existencia de la interacción entre el subsistema \( Y \) y el subsistema \( X \), y que únicamente se monitoriza la dinámica de \( X \). En ese caso, se le asignaría al sistema una tasa de producción de entropía dada por:
\[
\dot{\sigma}^{X}_0 = \partial_t S^X + \dot{S}_r^X.
\]

Si el subsistema \( X \) estuviera efectivamente aislado, se cumpliría \( \dot{\sigma}^{X}_0 \geq 0 \), en concordancia con la segunda ley de la termodinámica. Sin embargo, debido a su interacción con \( Y \), pueden darse situaciones en las que \( \dot{\sigma}^{X}_0 < 0 \), lo cual representa una violación aparente de la segunda ley. Esta puede interpretarse como el efecto de un Demonio de Maxwell, que utiliza la información que proviene de monitorear \(X \).

En la evolución de sistemas autónomos que alcanzan un estado estacionario, se cumple que \( \partial_t I_{xy} = 0 \). En ese régimen, el flujo de información se conserva:
\[
\dot{\mathcal{I}} = \dot{I}^{X} = -\dot{I}^{Y},
\]
y la producción de entropía local de cada subsistema se puede escribir como:


\begin{align*}
    \dot{\sigma}^{X} & = \dot{S}_{r}^{X} - \dot{\mathcal{I}} \geq 0 \\
    \dot{\sigma}^{Y} & =  \dot{S}_{r}^{Y} + \dot{\mathcal{I}} \geq 0 . 
\end{align*}

Supongamos el caso en que \( \dot{\mathcal{I}} > 0 \). En esta situación, se puede interpretar que el subsistema \( X \) está actuando como un sensor, al monitorear activamente al subsistema \( Y \). Para realizar esta función de monitoreo, el subsistema \( X \) debe disipar una cantidad mínima de energía tal que:
\[
\dot{S}_{r}^{X} \geq \dot{\mathcal{I}}.
\]

Por otro lado, el subsistema \( Y \) está proporcionando información, la cual puede ser aprovechada para extraer energía del entorno. En este caso, se cumple que:
\[
-\dot{S}_{r}^{Y} \leq \dot{\mathcal{I}}.
\]

Esta relación es importante, ya que permite utilizar información para realizar trabajo útil en sistemas fuera del equilibrio, o incluso para extraer calor de un reservorio caliente.

Un ejemplo característico de este fenómeno se encuentra en sistemas sometidos a un gradiente de potencial, donde el Demonio de Maxwell se manifiesta a través del surgimiento de una corriente de partículas en contra del gradiente. Este efecto será analizado en detalle en las secciones posteriores.


\section{Descripción cuántica}
Para describir los flujos de información en un contexto cuántico, es necesario conocer la dinámica de un sistema cuántico abierto acoplado a uno o varios reservorios. Para ello, se utilizará la matriz densidad, la cual describe la evolución del sistema reducido. El sistema total estará descrito por el Hamiltoniano $\hat{H} = \hat{H}_{S} + \hat{H}_{B} + \hat{H}_{I}$. Y la evolución markoviana de este sistema conduce a una ecuación maestra

\begin{equation*}
    \partial_{t}\hat{\rho}_{S}(t) = - i[\hat{H}_{S} + \hat{H}_{LS},\hat{\rho}_{S}] + \mathcal{L}(\hat{\rho}_{S}),
\end{equation*}

El acoplamiento del sistema con múltiples reservorios $\alpha$ se modela a través de un generador de Lindblad, que se descompone como
\begin{equation*}
    \mathcal{L} = \sum_{\alpha} \mathcal{L}_{\alpha},
\end{equation*}
donde cada superoperador $\mathcal{L}_{\alpha}$ representa la interacción del sistema con el reservorio $\alpha$.

Si se asume que cada reservorio induce un equilibrio local descrito por un estado de Gibbs
\begin{equation*}
    \hat{\rho}_{\text{eq}}^{\alpha} = \frac{1}{Z_{\beta_\alpha, \mu_\alpha}} e^{-\beta_\alpha (\hat{H}_S - \mu_\alpha \hat{N})},
\end{equation*}
entonces es posible aplicar la desigualdad de Spohn \cite{spohn1978entropy}, para obtener una formulación generalizada de la segunda ley de la termodinámica, en términos de una desigualdad de Clausius para la tasa de producción de entropía local asociada al reservorio $\alpha$, es decir 

\begin{equation}
    - \text{Tr}[ (\mathcal{L}_{\alpha} \hat{\rho}_{S})(\ln \hat{\rho}_{S} - \ln \hat{\rho}^{\alpha}_{eq} )  ] \geq 0.
\label{spohninfo}
\end{equation}

Cabe señalar, que este resultado se mantiene válido si el estado estacionario esta descrito por el Hamiltoniano termodinámico $\hat{H}_{\mathrm{TD}}$, en lugar del Hamiltoniano físico $\hat{H}_S$.

Si se define la cantidad
\begin{equation}
    \dot{S}^{\alpha} = - \mathrm{Tr} \{ (\mathcal{L}_{\alpha} \hat{\rho}_S) \ln \hat{\rho}_S \},
\end{equation}
como la tasa de cambio de la entropía de von Neumann del sistema inducida por el acoplamiento al reservorio $\alpha$, con la entropía definida por
\begin{equation}
    S = - \mathrm{Tr} \{ \hat{\rho}_S \ln \hat{\rho}_S \},
\end{equation}
y se identifica el flujo de calor en el segundo término de la desigualdad de Spohn (ver ecuación~\ref{spohninfo}), se obtiene una expresión para la tasa de producción de entropía local asociada al reservorio $\alpha$:
\begin{equation}
    \dot{\sigma}^{\alpha} = \dot{S}^{\alpha} - \beta_{\alpha} \dot{Q}^{\alpha} \geq 0.
\end{equation}
Esta desigualdad corresponde a una forma local de la segunda ley de la termodinámica, donde $\dot{Q}^{\alpha}$ es el flujo de calor hacia el sistema desde el reservorio $\alpha$, y $\beta_{\alpha}$ su inversa de temperatura. El término $\dot{\sigma}^{\alpha}$ representa la producción neta de entropía asociada a dicho contacto, y es 

\begin{equation}
    \dot{\sigma}^{\alpha} = \dot{S}^{\alpha} - \beta_{\alpha} J_{\alpha} \geq 0,
\label{sec4:localentropy}
\end{equation}
que actúa como una inecuación de Clausius parcial. Esto es importante, ya que localmente se puede separar la producción de entropía en cantidades mayor a cero, similar a como se hizo en la descripción clásica. De hecho si se suman todas las tasas de cambio de entropía, se obtiene la derivada total $\partial_{t} S = \sum_{\alpha} \dot{S}^{\alpha}$, por lo tanto al sumar todas las producciones de entropía locales

\begin{equation*}
    \sum_{\alpha}\dot{\sigma}^{\alpha} = \partial_{t}S - \sum_{\alpha}\beta_{\alpha}J_{\alpha} = \dot{\sigma} \geq 0,
\end{equation*}
que corresponde a la inecuación de Clausius estándar.

Se puede notar que para el caso estacionario $\partial_{t}S = 0$, no necesariamente $\dot{S}^{\alpha}$ es cero. Sino que depende del flujo de calor que entra localmente por el disipador $\alpha$. También se puede tratar la inecuación para la energía libre, primero definiendo la corriente de energía y calor correspondientes al baño $\alpha$

\begin{align*}
    \dot{E}_{\alpha} & = \text{Tr}[ (\mathcal{L}_{\alpha} \hat{\rho}_{S}) \hat{H}_{S}] \\
    \dot{W}_{\alpha} & = \mu_{\alpha}\text{Tr}[ (\mathcal{L}_{\alpha} \hat{\rho}_{S}) \hat{N}].
\end{align*}

En este caso se usa el Hamiltoniano $\hat{H}_{S}$ para escribir la tasa de cambio de la energía interna, sin embargo si el estado estacionario contiene $\hat{H}_{TD}$, entonces el flujo de energía es $ \dot{E}_{\alpha} = \text{Tr}[ (\mathcal{L}_{\alpha} \hat{\rho}_{S}) \hat{H}_{TD}]$.

Se cumple que la suma de las corrientes de energía provenientes de los distintos reservorios corresponde a la derivada temporal de la energía interna del sistema

\begin{equation*}
    \sum_{\alpha}\dot{E}_{\alpha} = \partial_{t}E.
\end{equation*}

En el caso de un Hamiltoniano independiente del tiempo, no se realiza trabajo mecánico, por lo que sólo se considera trabajo químico. En consecuencia, la energía intercambiada con cada reservorio se descompone como

\begin{equation*}
    \dot{E}_{\alpha} = J_{\alpha} + \dot{W}_{\alpha}.
\end{equation*}

Multiplicando la ecuación \ref{sec4:localentropy} por la temperatura del reservorio $T_{\alpha}$ y reemplazando $J_{\alpha}$ mediante la expresión anterior, se obtiene

\begin{equation}
    T_{\alpha} \dot{\sigma}^{\alpha} = \dot{W}_{\alpha} - \dot{\mathcal{F}}_{\alpha} \geq 0,
\label{sec4:localfreerate}
\end{equation}
donde se ha definido la tasa de variación de la energía libre asociado al reservorio $\alpha$ como 

\begin{equation*}
    \dot{\mathcal{F}}_{\alpha} = \dot{E}_{\alpha} - T_{\alpha}\dot{S}^{\alpha}.
\end{equation*}

La inecuación \ref{sec4:localfreerate} consiste en una inecuación local de energía libre. Si se suman las inecuaciones de todos los reservorios, se tendrá 

\begin{equation}
    \sum_{\alpha}T_{\alpha} \dot{\sigma}^{\alpha} = \dot{W} - \dot{\mathcal{F}} \geq 0,
\label{sec4:freeratefinal}
\end{equation}
donde $\dot{W} = \sum_{\alpha}\dot{W}_{\alpha}$ y la tasa de cambio de energía libre total es

\begin{equation*}
    \dot{\mathcal{F}} = \partial_{t}E - \sum_{\alpha}T_{\alpha} \dot{S}^{\alpha}.
\end{equation*}

Para el caso particular, en qué todas las temperaturas son iguales, la tasa de cambio de energía libre $\dot{\mathcal{F}}$ se vuelve equivalente a la derivada temporal de la energía libre $\partial_{t}F = \partial_{t}(E-TS)$. Sólo en este caso \ref{sec4:localentropy} y \ref{sec4:freeratefinal} son equivalentes.

En el caso isotérmico, se cumple en el estado estacionario que $\dot{\mathcal{F}} = \partial_{t}F = 0$, lo que implica $\dot{W}>0$. Por lo tanto, el sistema no puede realizar trabajo neto si sólo va a intercambiar  calor con reservorios a la misma temperatura. La única forma de obtener trabajo neto del sistema, o sea, lograr que $\dot{W}<0$, es mediante la presencia de reservorios con distinta temperatura.  En tal caso, la tasa de cambio de energía libre no se anula y es $\dot{\mathcal{F}} = - \sum_{\alpha} T_{\alpha} \dot{S}^{\alpha} \neq 0$.   

\section{Flujos de información en contexto cuántico}
Suponga un sistema compuesto por dos subsistemas acoplados, descrito por el Hamiltoniano total

\begin{equation*}
    \hat{H}_{S} = \hat{H}_{X} + \hat{H}_{Y} + \hat{H}_{XY}, 
\end{equation*}
donde $\hat{H}_{i}$ representa el Hamiltoniano del subsistema $i=X,Y$, y $\hat{H}_{XY}$ corresponde al término de interacción entre los dos subsistemas. Se asume, además, que cada subsistema está acoplado a un conjunto distinto de reservorios. En particular, los reservorios asociados al subsistema $i$ serán denotados por $\alpha_{i}$. Esto permite describir la tasa de producción de entropía local de cada subsistema como

\begin{equation*}
    \dot{\sigma}^{i} \equiv \sum_{\alpha_{i}} \dot{\sigma}^{\alpha_{i}} = \sum_{\alpha_{i}} \dot{S}^{\alpha_{i}} - \sum_{\alpha_{i}} \beta_{\alpha_{i}} J_{\alpha_{i}} \geq 0,
\end{equation*}
por lo tanto, la tasa de producción de entropía total del sistema corresponde a $\dot{\sigma} = \dot{\sigma}^{X} + \dot{\sigma}^{Y}$. 

De forma análoga a lo realizado en la descripción clásica, se busca establecer una relación entre las producciones locales de entropía y la información compartida entre subsistemas. Para ello, se utilizará la información mutua entre los dos subsistemas, definida como
 
\begin{equation*}
    I_{xy} = S_{X} + S_{Y} - S_{XY},
\end{equation*}
donde $S_{i} = - \text{Tr}\{\hat{\rho}_{i}\ln \hat{\rho_{i}}\}$ es la entropía de Von Neumann del subsistema $i=X,Y$, y $\hat{\rho}_{i}$ consiste en la matriz densidad del subsistema $i$. 

Se puede separar la derivada temporal de la información mutua

\begin{align*}
    \partial_{t}I_{xy} & = \partial_{t}S_{X} + \partial_{t}S_{Y} - \partial_{t}S_{XY} \\
        & = \partial_{t}S_{X} + \partial_{t}S_{Y} - \sum_{i=X,Y;\alpha_{i}}\dot{S}^{\alpha_{i}} \\
        & =  \partial_{t}S_{X} - \sum_{\alpha_{X}} \dot{S}^{\alpha_{X}} + \partial_{t}S_{Y} - \sum_{\alpha_{Y}} \dot{S}^{\alpha_{Y}} \\
        & = \dot{I}^{X} + \dot{I}^{Y},
\end{align*}
donde aparece el flujo de información correspondiente a cada subsistema $i=X,Y$, como la cantidad $\dot{I}^{i} = \partial_{t}S^{i} - \sum_{\alpha_{i}} \dot{S}^{\alpha_{i}} $. Luego, si se reemplaza $\sum_{\alpha_{i}}\dot{S}^{\alpha_{i}} \to \partial_{t}S_{i} - \dot{I}^{i}$, en las tasas de producción de entropía locales, quedan las inecuaciones  

\begin{align*}
    \dot{\sigma}^{X} = \partial_{t}S^{X} - \sum_{\alpha_{X}} \beta_{\alpha_{X}} J_{\alpha_{X}} - \dot{I}^{X} \geq 0 \\
    \dot{\sigma}^{Y} = \partial_{t}S^{Y} - \sum_{\alpha_{Y}} \beta_{\alpha_{Y}} J_{\alpha_{Y}} - \dot{I}^{Y} \geq 0,
\end{align*}
este resultado coincide con el obtenido en la descripción clásica. No obstante, su formulación en términos de entropía de von Neumann permite incorporar de forma natural los efectos cuánticos presentes en el sistema, lo cual resulta fundamental para caracterizar los flujos de información en regímenes donde las correlaciones cuánticas y las coherencias desempeñan un rol relevante.

Si se considera que el subsistema $i$ esta conectado a a un reservorio isotermico de temperatura $T_i$, la tasa de cambio de la energía libre será
\begin{align*}
    \dot{\mathcal{F}}_{i} & = \dot{E}_{i} - T_{i} \sum_{\alpha_{i}}\dot{S}^{\alpha_{i}}. 
\end{align*}

En el estado estacionario se cumple la relación

\begin{equation*}
    \dot{I}^{i} = - \sum_{\alpha_{i}}\dot{S}^{\alpha_{i}},
\end{equation*}
y la tasa de cambio de energía libre del subsistema $i$ es 

\begin{equation*}
    \dot{\mathcal{F}}_{i} = \dot{E}_{i} + T_{i} \dot{I}^{i}.
\end{equation*}

Así, la tasa de cambio de la energía libre posee dos contribuciones: una correspondiente al intercambio de energía y otra asociada al flujo de información. Si se considera el régimen estacionario, donde la energía total permanece constante, es decir
\begin{equation*}
    \partial_{t}E = \dot{E}^{X} + \dot{E}^{Y} = 0,
\end{equation*}
entonces los términos $\dot{E}^{i}$ pueden interpretarse como flujos de energía entre los dos subsistemas. En este contexto, la tasa de cambio de energía libre para cada subsistema se descompone en una parte asociada al intercambio energético y otra vinculada al flujo de información generado por las correlaciones entre los subsistemas.

En el caso $\dot{E}^{X} = \dot{E}^{Y} \approx 0 $, esta descomposición permite que el sistema realice trabajo a partir de una contribución dominante de la información, incluso en ausencia de diferencias térmicas significativas. Tal mecanismo constituye lo que se conoce como una máquina de la información, en la que los flujos de información entre subsistemas juegan un rol principal, permitiendo la extracción de trabajo en un régimen puramente informacional.


\label{sec4:flujos0}

\chapter{Dinámica de un sistema de 3 puntos cuánticos conectado al ambiente}
En este capítulo se estudia un sistema compuesto por tres puntos cuánticos acoplados a múltiples reservorios térmicos. La evolución del sistema se analiza mediante la ecuación de Lindblad desarrollada en la sección \ref{sec2lindbladconsistency}. En la sección \ref{sec5:modelo} se introduce el Hamiltoniano del sistema reducido y su entorno. Posteriormente, en la sección \ref{sec5:transporte}, se analiza el comportamiento del sistema en un escenario típico de transporte cuántico. En la sección \ref{sec5:demonio1}, se establecen las condiciones bajo las cuales la dinámica del sistema puede interpretarse como la de un Demonio de Maxwell autónomo. Finalmente, en la sección \ref{sec5:demonio}, se compara este comportamiento con su contraparte clásica, destacando las diferencias inducidas por los efectos cuánticos.

\section{Modelo teórico}
El sistema reducido está compuesto por tres puntos cuánticos. Uno de ellos posee un nivel de energía $\epsilon_d$ y se encuentra acoplado, mediante una interacción de Coulomb $U$, a los otros dos puntos cuánticos. Estos últimos tienen una energía de sitio $\epsilon$ y están conectados entre sí por un término de hopping $g$, además de una interacción de Coulomb $U_f$. Cada uno de los niveles de energía se encuentra acoplado a un baño distinto. 

La configuración del sistema reducido se ilustra en la Figura~\ref{img:sistema3puntos}.


\insertimage[\label{img:sistema3puntos}]{ejemplos/semilocal1}{scale=0.36}{Esquema que representa el sistema reducido compuesto por tres puntos cuánticos, modelados como tres niveles de energía. Estos niveles presentan interacción de Coulomb repulsiva entre ellos, y cada uno está acoplado a un baño térmico distinto.}

El Hamiltoniano del sistema reducido esta dado por 

\begin{align*}
    \hat{H}_{S} & = \epsilon_{d}\hat{d}^{\dagger}_{D}\hat{d}_{D} + \epsilon \hat{d}^{\dagger}_{L}\hat{d}_{L} + \epsilon \hat{d}^{\dagger}_{R}\hat{d}_{R} + g(\hat{d}^{\dagger}_{L}\hat{d}_{R} + \hat{d}^{\dagger}_{R}\hat{d}_{L} ) \\
          & + U(\hat{n}_{D}\hat{n}_{L} + \hat{n}_{D}\hat{n}_{R} )  + U_{f}\hat{n}_{R}\hat{n}_{L},
\end{align*}
donde $\hat{n}_{i} = \hat{d}^{\dagger}_{i} \hat{d}_{i}$ es el operador de número del sitio $i$, y los operadores $\hat{d}_{i}$ satisfacen las relaciones de anticonmutación

\begin{equation*}
    \{\hat{d}_{\alpha},\hat{d}_{\beta} \} = 0  \hspace{10mm} \{\hat{d}^{\dagger}_{\alpha}, \hat{d}_{\beta} \} = \delta_{\alpha \beta}.
\end{equation*}

Usando la base de Fock $|n_{L},n_{R},n_{D} \rangle = (\hat{d}^{\dagger}_{L})^{n_{L}}(\hat{d}^{\dagger}_{R})^{n_{R}}(\hat{d}^{\dagger}_{D})^{n_{D}}|0,0,0\rangle$, y escribiendo el Hamiltoniano en el orden de la base 

\begin{equation*}
     \{|0,0,0\rangle, |1,0,0\rangle, |0,1,0\rangle, |0,0,1\rangle, |1,1,0\rangle, |1,0,1\rangle, |0,1,1\rangle, |1,1,1\rangle\},
\end{equation*}
se puede representar el Hamiltoniano del sistema en forma matricial 

\begin{equation*}
    \hat{H}_{S} = 
    \begin{bmatrix}
        0 & 0 & 0 & 0 & 0 & 0 & 0 & 0 \\
        0 & \epsilon_{L} & g & 0 & 0 & 0 & 0 & 0 \\
        0 & g & \epsilon_{R} & 0 & 0 & 0 & 0 & 0 \\
        0 & 0 & 0 & \epsilon_{d} & 0 & 0 & 0 & 0 \\
        0 & 0 & 0 & 0 & \epsilon_{L} + \epsilon_{R}  + U_{f} & 0 & 0 & 0 \\
        0 & 0 & 0 & 0 & 0 & \epsilon_{L} + \epsilon_{d} + U & g & 0 \\
        0 & 0 & 0 & 0 & 0 & g & \epsilon_{R} + \epsilon_{d} + U & 0 \\
        0 & 0 & 0 & 0 & 0 & 0 & 0 & \epsilon_{L} + \epsilon_{R}  + \epsilon_{d} + 2U + U_{f} 
        \end{bmatrix}.
\end{equation*}

Por otro lado, el Hamiltoniano del baño y la interacción con el sistema se descomponen como $\hat{H}_{B} = \hat{H}_{L}+\hat{H}_{R}+\hat{H}_{D}$ y $\hat{V} = \hat{V}_{L}+\hat{V}_{R}+\hat{V}_{D}$ respectivamente. Cada término esta dado por 

\begin{equation*}
    \hat{H}_{\alpha} = \sum_{l} \epsilon_{\alpha l} \hat{c}^{\dagger}_{\alpha l}\hat{c}_{\alpha l} \hspace{10mm} \hat{V}_{\alpha} = \sum_{l} t_{\alpha l} ( \hat{d}^{\dagger}_{\alpha} \hat{c}_{\alpha l} + \hat{c}^{\dagger}_{\alpha l} \hat{d}_{\alpha} ),
\end{equation*}
donde $\alpha = L,R,D$ denota los tres reservorios acoplados a cada punto cuántico, $\hat{c}_{\alpha,l}$ son los operadores de aniquilación de los modos del reservorio $\alpha$, $\epsilon_{\alpha l}$ sus energías, y $t_{\alpha l}$ los parámetros de acoplamiento entre el sistema y baño. 

A partir del término de interacción $\hat{V}$, es posible identificar los operadores asociados al sistema y al baño

\begin{equation*}
    \hat{S}_{\alpha,1} = \hat{d}^{\dagger}_{\alpha} \hspace{10mm} \hat{S}_{\alpha,-1} = \hat{d}_{\alpha}
\end{equation*}

\begin{equation*}
    \hat{B}_{\alpha,1} = \sum_{l}t_{\alpha l} \hat{c}_{\alpha l} \hspace{10mm} \hat{B}_{\alpha,-1} = \sum_{l}t_{\alpha l} \hat{c}^{\dagger}_{\alpha l}.
\end{equation*}

La evolución no unitaria del sistema de tres puntos cuánticos, inducida por su acoplamiento a los baños, se describe mediante una ecuación maestra descrita por el operador de Lindblad

\begin{equation}
    \mathcal{L} = \mathcal{L}_{R} + \mathcal{L}_{L} + \mathcal{L}_{D},
    \label{Lindbladsec5}
\end{equation}
donde el operador de Lindblad $\mathcal{L}$ se descompone como una suma de contribuciones individuales, una por cada baño acoplado al sistema. Explícitamente cada contribución $\mathcal{L}_{i}$ corresponde a 

\begin{align*}
    \mathcal{L}_{R} & = \gamma_{R}(\epsilon)(f_{R}(\epsilon)\mathcal{D}_{0}[\hat{d}^{\dagger}_{R}(\textbf{1}-\hat{n}_{D})(\textbf{1}-\hat{n}_{L}) ]  + [1-f_{R}(\epsilon)]\mathcal{D}_{0}[\hat{d}_{R}(\textbf{1}-\hat{n}_{D})(\textbf{1}-\hat{n}_{L}) ]  )  \\
                    & + \gamma_{R}(\epsilon+U)(f_{R}(\epsilon+U)\mathcal{D}_{0}[\hat{d}^{\dagger}_{R}\hat{n}_{D}(\textbf{1}-\hat{n}_{L}) ]  + [1-f_{R}(\epsilon+U)]\mathcal{D}_{0}[\hat{d}_{R}\hat{n}_{D}(\textbf{1}-\hat{n}_{L}) ]  ) \\
                   & + \gamma_{R}(\epsilon+U_{f})(f_{R}(\epsilon+U_{f})\mathcal{D}_{0}[\hat{d}^{\dagger}_{R}(\textbf{1}-\hat{n}_{D})\hat{n}_{L} ]  + [1-f_{R}(\epsilon+U_{f})]\mathcal{D}_{0}[\hat{d}_{R}(\textbf{1}-\hat{n}_{D})\hat{n}_{L} ]  ) \\
                  & + \gamma_{R}(\epsilon+U+U_{f})(f_{R}(\epsilon+U+U_{f})\mathcal{D}_{0}[\hat{d}^{\dagger}_{R}\hat{n}_{D}\hat{n}_{L} ]  + [1-f_{R}(\epsilon+U+U_{f})]\mathcal{D}_{0}[\hat{d}_{R}\hat{n}_{D}\hat{n}_{L} ]  ) 
\end{align*}

\begin{align*}
    \mathcal{L}_{L} & = \gamma_{L}(\epsilon)(f_{L}(\epsilon)\mathcal{D}_{0}[\hat{d}^{\dagger}_{L}(\textbf{1}-\hat{n}_{D})(\textbf{1}-\hat{n}_{R}) ]  + [1-f_{L}(\epsilon)]\mathcal{D}_{0}[\hat{d}_{L}(\textbf{1}-\hat{n}_{D})(\textbf{1}-\hat{n}_{R}) ]  )  \\
                    & + \gamma_{L}(\epsilon+U)(f_{L}(\epsilon+U)\mathcal{D}_{0}[\hat{d}^{\dagger}_{L}\hat{n}_{D}(\textbf{1}-\hat{n}_{R}) ]  + [1-f_{L}(\epsilon+U)]\mathcal{D}_{0}[\hat{d}_{L}\hat{n}_{D}(\textbf{1}-\hat{n}_{R}) ]  ) \\
                   & + \gamma_{L}(\epsilon+U_{f})(f_{L}(\epsilon+U_{f})\mathcal{D}_{0}[\hat{d}^{\dagger}_{L}(\textbf{1}-\hat{n}_{D})\hat{n}_{R} ]  + [1-f_{L}(\epsilon+U_{f})]\mathcal{D}_{0}[\hat{d}_{L}(\textbf{1}-\hat{n}_{D})\hat{n}_{R} ]  ) \\
                  & + \gamma_{L}(\epsilon+U+U_{f})(f_{L}(\epsilon+U+U_{f})\mathcal{D}_{0}[\hat{d}^{\dagger}_{L}\hat{n}_{D}\hat{n}_{R} ]  + [1-f_{L}(\epsilon+U+U_{f})]\mathcal{D}_{0}[\hat{d}_{L}\hat{n}_{D}\hat{n}_{R} ]  ) 
\end{align*}

\begin{align*}
    \mathcal{L}_{D} & = \gamma_{D}(\epsilon_{d})(f_{D}(\epsilon_{d})\mathcal{D}_{0}[\hat{d}^{\dagger}_{D}(\textbf{1}-\hat{n}_{R})(\textbf{1}-\hat{n}_{L}) ]  + [1-f_{D}(\epsilon_{d})]\mathcal{D}_{0}[\hat{d}_{R}(\textbf{1}-\hat{n}_{D})(\textbf{1}-\hat{n}_{L}) ]  )  \\
                    & + \gamma_{D}(\epsilon_{d}+U)f_{D}(\epsilon_{d}+U)\mathcal{D}_{0}[\hat{d}^{\dagger}_{D}(\hat{n}_{R}(\textbf{1}-\hat{n}_{L}) + \hat{n}_{L}(\textbf{1}-\hat{n}_{R})) ]  \\
                    & + \gamma_{D}(\epsilon_{d}+U)[1-f_{D}(\epsilon_{d}+U)]\mathcal{D}_{0}[\hat{d}_{D}(\hat{n}_{R}(\textbf{1}-\hat{n}_{L}) + \hat{n}_{L}(\textbf{1}-\hat{n}_{R}))]   \\
                   & + \gamma_{D}(\epsilon_{d}+2U)(f_{D}(\epsilon_{d}+2U)\mathcal{D}_{0}[\hat{d}^{\dagger}_{D}\hat{n}_{R}\hat{n}_{L} ]  + [1-f_{D}(\epsilon_{d}+2U)]\mathcal{D}_{0}[\hat{d}_{D}\hat{n}_{R}\hat{n}_{L} ]  ).
\end{align*}

Con $f_{i}(\omega) = (\exp(\beta_{i}(\omega - \mu_{i})) + 1)^{-1}$ y el operador $\mathcal{D}_{0}$  

\begin{equation*}
    \mathcal{D}_{0}[\hat{A}]\hat{\rho} = \hat{A}\hat{\rho}\hat{A}^{\dagger}- \frac{1}{2} \{\hat{A }^{\dagger}\hat{A},\hat{\rho} \}.
\end{equation*}

La demostración de la ecuación \ref{Lindbladsec5} se presenta en los apéndices \ref{apendix5jumpop} y \ref{apendix5frecuencygroup}.

Despreciando el término de \textit{Lamb Shift} $\hat{H}_{LS}$, como se discute en~\cite{prech2023entanglement}, la evolución del sistema de tres puntos cuánticos está dada por
\begin{equation}
    \frac{d}{dt}\hat{\rho}_{S}(t) = -i[\hat{H}_{S},\hat{\rho}_{S}(t)] + \mathcal{L}(\hat{\rho}_{S}(t)),
\end{equation}
donde $\hat{\rho}_{S}(t)$ es la matriz densidad del sistema reducido. A partir de esta ecuación, es posible calcular numéricamente la evolución del sistema y obtener las magnitudes termodinámicas relevantes.

\label{sec5:modelo}

\newpage

\section{Estudio del transporte}
Se aborda inicialmente el caso en que las razones de túnel no dependen de la frecuencia, es decir, $\gamma_{i}(\omega) = \gamma_{i}$. Y se estudia el flujo de corriente generado por una diferencia de potencial $\mu_{L} - \mu_{R} = eV$, empleando las razones de túnel $\gamma_{L} = \gamma_{R} = 1/100$ y $\gamma_{d} = 1/50$. Los resultados se muestran en la Figura~\ref{img:transporte1}.

\insertimage[\label{img:transporte1}]{ejemplos/particletp.pdf}{scale=0.35}{Corriente de partículas en función de la razón \( eV/T \). Las temperaturas de los reservorios son \( T_{L} = T_{R} = 100 \) y \( T_{d} = 2 \). El parámetro de acoplamiento se fija en \( g = \frac{5}{1000} \). Las energías de sitio corresponden a \( \epsilon = 0 \) y \( \epsilon_{d} = \mu_{d} - \frac{U}{2} \), mientras que las interacciones de Coulomb se fijan en \( U_{F} = 500 \) y \( U = 40 \).}

La figura muestra que, a medida que aumenta la diferencia de potencial \( eV \), la corriente de partículas \( \dot{N}_L = -\dot{N}_R \) que fluye del baño \( L \) al baño \( R \) también aumenta. Este comportamiento es consistente con lo esperado, ya que un mayor potencial químico en el reservorio \( L \) impulsa el transporte hacia \( R \). Asimismo, se puede analizar la potencia suministrada al sistema en función de \( eV \).

\insertimage[\label{img:transporte2}]{ejemplos/worktp.pdf}{scale =0.35}{Potencia entregada por los reservorios en función de $eV/T$.}
    
La Figura \ref{img:transporte2} muestra que el conjunto de reservorios \( L \) y \( R \) entrega trabajo al sistema para mantener la corriente de partículas. Este comportamiento es consistente desde el punto de vista físico, ya que el paso de electrones desde un potencial químico mayor hacia uno menor implica una transferencia de energía al sistema, que se manifiesta como trabajo. Por otro lado, dado que el reservorio \( D \) no intercambia partículas en promedio, la potencia que este entrega es nula.

 Asimismo, pueden examinarse los flujos de calor que reciben los sistemas $LR$ y $D$


\insertimage[\label{img:transporte3}]{ejemplos/heatlrdtp.pdf}{scale =0.35}{Flujo de calor recibido por los sistemas $LR$ y $D$ en función de $eV/T$.}
    
Como uno puede ver, para $eV/T \lessapprox 2$ , el sistema $LR$ en total recibe calor de los reservorios $L$ y $R$. Para $eV/T \gtrapprox 2$ el sistema $LR$ en total entrega calor a los reservorios $L$ y $R$. 

Para completar el análisis termodinámico a partir de la primera ley $\dot{E}_{\alpha} = J_{\alpha} + \dot{W}_{\alpha}$, se grafican a continuación los flujos de energía involucrados

\insertimage[\label{img:transporte4}]{ejemplos/energytp.pdf}{scale =0.4}{Flujo de energía entregado por los reservorios en función de $eV/T$.}

En el estado estacionario, la condición \( \partial_t E = 0 \) implica que \( \dot{E}_{LR} = -\dot{E}_d \), lo cual se observa en la Figura \ref{img:transporte4}. Allí, el flujo de energía hacia el sistema \( LR \) es positivo, mientras que el flujo desde el subsistema \( D \) es negativo. Esto indica que el sistema \( LR \) recibe energía proveniente de los reservorios \( L \) y \( R \), la cual es posteriormente transferida al subsistema \( D \) y desde allí al reservorio \( D \). Este comportamiento es coherente con el hecho de que la temperatura del reservorio \( D \) es menor (\( T_{d} < T_{L} = T_{R} \)), y por lo tanto, el flujo de calor se dirige desde los reservorios más calientes hacia el más frío.

Para analizar la presencia de efectos cuánticos en el estado estacionario, se calcularán dos cantidades relevantes. La primera de ellas es la norma-\( l_{1} \), \( \mathcal{C}_{l_{1}} \), que permite cuantificar las coherencias en el sistema \( LR \).


\begin{equation*}
    \mathcal{C}_{l_{1}} = \sum_{i> j} |\hat{\rho}_{S;i,j}|,
\end{equation*}
otro efecto cuántico relevante de estudiar es el entrelazamiento entre modos específicos del sistema. En particular, se analiza el entrelazamiento entre los pares de estados \(|1,0,0\rangle, |0,1,0\rangle\) y \(|1,0,1\rangle, |0,1,1\rangle\). Para cuantificar este tipo de correlaciones no clásicas se utiliza la concurrencia \( \mathcal{C}_{\text{on}} \) \cite{hill1997entanglement, wootters1998entanglement}, definida como

\begin{equation*}
    \mathcal{C}_{\text{on}} = \max \left\{ 2|\alpha + \beta| - 2\sqrt{p_0 p_D},\, 0 \right\},
\end{equation*}
donde \( p_0 \) y \( p_D \) representan las probabilidades de que el sistema \( LR \) se encuentre vacío o doblemente ocupado, respectivamente. El término \( \alpha \) corresponde a la coherencia entre los estados \(|1,0,0\rangle\) y \(|0,1,0\rangle\), mientras que \( \beta \) representa la coherencia entre los estados \(|1,0,1\rangle\) y \(|0,1,1\rangle\). La deducción de esta fórmula se presenta en el Apéndice \ref{appendix5final}. 

Finalmente, en la Figura \ref{img:transporte5} se analiza cómo varían la coherencia y la concurrencia en función de la diferencia de potencial

\insertimage[\label{img:transporte5}]{ejemplos/quantumtp.pdf}{scale =0.4}{Coherencia(línea azul) y concurrencia(linea roja) en función de $eV/T$.}
    
En este caso, la coherencia comienza en cero y aumenta progresivamente a medida que se incrementa la diferencia de potencial. En cuanto al entrelazamiento, también parte desde cero —coincidiendo con el valor nulo de la coherencia— y se mantiene nulo hasta alcanzar un cierto umbral, a partir del cual la concurrencia comienza a crecer con el aumento de \( eV/T \). Esto indica la aparición de entrelazamiento en el sistema \( LR \). Este comportamiento es relevante, ya que sugiere que la presencia de una corriente en el sistema contribuye a que se desarrollen coherencia y entrelazamiento entre los sitios $L$ y $R$.  


\label{sec5:transporte}

\newpage

\section{Demonio de Maxwell en 3 puntos cuánticos}
Para que el sistema de tres puntos cuánticos se comporte como un Demonio de Maxwell autónomo, es fundamental tratarlo como un sistema bipartito compuesto por dos subsistemas: por un lado, el subsistema $LR$, que consiste en dos niveles de energía $\epsilon$; por otro, el nivel $\epsilon_D$, que desempeña el rol de demonio.

Para que este último actúe como demonio, debe ser capaz de detectar cambios en el subsistema $LR$, ejecutando una acción análoga a una medición. Gracias a la interacción de Coulomb $U$, el demonio puede distinguir si el subsistema $LR$ está vacío, ocupado o doblemente ocupado. A su vez, el subsistema $LR$ modifica su dinámica dependiendo del estado de ocupación del demonio.

Esta retroalimentación se implementa matemáticamente al modificar las razones de túnel $\gamma_{i}(\epsilon + U) \neq \gamma_{i}(\epsilon)$, con $i = L, R$. 

Para que el demonio pueda detectar eficazmente el estado del subsistema $LR$, deben cumplirse dos condiciones. Primero, que la energía de interacción supere las fluctuaciones térmicas del reservorio asociado al demonio, es decir, $\beta_{D} U \gg 1$. Segundo, que su dinámica disipativa sea más rápida que la del subsistema $LR$, lo que implica $\gamma_{D} > \max\{\gamma_{i}, \gamma^{U}_{i}\}$ con $i = L, R$.

La dinámica específica del demonio se describirá con mayor detalle en la Figura~\ref{img:dinamica1}.

\insertimage[\label{img:dinamica1}]{ejemplos/diseño1}{scale=0.6}{Primer paso en la dinámica, el sitio $D$ y el sitio $R$ poseen una mayor probabilidad de estar ocupados.}

La elección $\epsilon_D = \mu_D - U/2$ favorece la ocupación del nivel de energía del demonio. Para inducir una corriente en dirección opuesta al gradiente de potencial, es conveniente imponer que $\gamma^{U}_{R} > \gamma^{U}_{L}$, de modo que las transiciones sean más frecuentes hacia el baño derecho, aumentando así la probabilidad de que este nivel se pueble. Posteriormente, como se ilustra en la Figura \ref{img:dinamica2}, el acoplamiento $g$ permite el intercambio coherente del electrón entre los niveles $L$ y $R$, introduciendo así una dinámica esencialmente cuántica.


\insertimage[\label{img:dinamica2}]{ejemplos/diseño2}{scale=0.6}{Segundo paso en la dinámica, el hopping $g$ genera superposición entre los sitios $L$ y $R$.}

Finalmente, se produce una tercera etapa, ilustrada en la Figura \ref{img:dinamica3}. En esta fase, con el objetivo de generar un flujo de corriente en contra del gradiente de potencial, resulta conveniente imponer que $\gamma_{L} > \gamma_{R}$. Esta condición asegura que, cuando el demonio se encuentra desocupado, la probabilidad de que ocurra una transición hacia el nivel $L$ sea mayor que hacia el nivel $R$. Como consecuencia, se favorece el transporte de electrones desde el reservorio $R$ (de menor potencial) hacia el reservorio $L$ (de mayor potencial), evidenciando así un comportamiento característico de un Demonio de Maxwell autónomo.


\insertimage[\label{img:dinamica3}]{ejemplos/diseño3}{scale=0.6}{Tercer paso en la dinámica, el electrón se transporta al baño $L$.}

Teniendo definida la dinámica del sistema, es posible elegir de forma adecuada distintos candidatos para las razones de túnel con el objetivo de inducir un comportamiento tipo Demonio de Maxwell. Para ello, se estudiará el comportamiento de la matriz densidad, así como los flujos de corriente de partículas, calor, información y potencia, en función de la diferencia de potencial $\mu_{L} - \mu_{R} = eV$.

Se comenzará analizando las probabilidades de ocupación extraídas de la matriz densidad. Como se muestra en la Figura~\ref{img:densitymatrix}, las tres probabilidades más altas corresponden a los estados $\rho_{001}$, $\rho_{100}$ y $\rho_{010}$, lo cual es coherente con la presencia de interacción de Coulomb, que penaliza las configuraciones con ocupación múltiple.

Para valores de $eV/T \lessapprox 3.6$, la probabilidad más alta corresponde al estado $\rho_{001}$, lo que indica que el sitio $D$ es, en promedio, el más ocupado. Sin embargo, a medida que la diferencia de potencial aumenta ($eV/T \gtrapprox 3.6$), las probabilidades asociadas a los estados $\rho_{100}$ y $\rho_{010}$ —correspondientes a los sitios $L$ y $R$— superan a $\rho_{001}$, lo que refleja un cambio en la ocupación predominante del sistema.

Debido a que la interacción de Coulomb penaliza los estados con más de un electrón, cabría esperar que la probabilidad de vacío $\rho_{000}$ fuese del mismo orden de magnitud que $\rho_{100},\rho_{010},\rho_{001}$. Sin embargo, debido a la constante inyección de partículas inducida por la diferencia de potencial, esta probabilidad resulta ser significativamente menor. 

Se observa además un cruce entre las curvas de $\rho_{101}$ y $\rho_{011}$ en $eV/T \approx 2.4$. Para valores $eV/T \lessapprox 2.4$, la configuración con interacción Coulomb entre los puntos $R$ y $D$ (representada por $\rho_{011}$) es más probable. En cambio, para $eV/T \gtrapprox 2.4$, la ocupación $\rho_{101}$ domina, indicando una mayor interacción entre los puntos $L$ y $D$.


\insertimage[\label{img:densitymatrix}]{ejemplos/matrixdemon.pdf}{scale=0.4}{Elementos diagonales de la matriz densidad en función de $eV/T$.}

El comportamiento de un Demonio de Maxwell autónomo se refleja en las cantidades termodinámicas del sistema. En la Figura~\ref{img:fig1resultado} se muestran: (a) los flujos de calor correspondientes a cada reservorio, y (b) las corrientes de partículas asociadas. Cabe destacar que, dado que no existe un término de hopping entre el sistema y el demonio, la corriente de partículas en el demonio es nula, $\dot{N}_{d} = 0$. Además, por conservación de partículas, se cumple que $\dot{N}_{L} = -\dot{N}_{R}$, lo cual implica que la influencia del demonio sobre el sistema ocurre exclusivamente a través de flujos de energía o información.

Dado que $\mu_L - \mu_R = eV$, se esperaría que $\dot{N}_L > 0$ para todo valor de $eV$. No obstante, en el régimen $eV/T \lessapprox 2.4$, se observa que $\dot{N}_L < 0$, lo que indica un flujo de partículas en contra del gradiente de potencial. Este cambio de signo en la corriente coincide con el cruce entre las probabilidades $\rho_{101}$ y $\rho_{011}$, lo que es consistente con el comportamiento tipo Maxwell. En particular, como se ilustró en la Figura~\ref{img:dinamica1}, dicho régimen requiere una interacción simultánea entre los puntos $R$ y $D$. Sin embargo, al dominar $\rho_{101}$, esta dinámica se ve suprimida.

Si se analiza únicamente el subsistema $LR$, la existencia de una corriente contra el gradiente químico representa una violación aparente de la segunda ley de la termodinámica. Para este análisis, se considera la tasa de producción de entropía efectiva del subsistema, dada por $\dot{\sigma}^{o}_{LR} = d_{t}S_{LR} - \sum_{i=L,R} J_i/T$, cuya evolución en función de $eV/T$ se muestra en la Figura~\ref{img:fig2resultado}.


\insertimage[\label{img:fig1resultado}]{ejemplos/heatparticledemon.pdf}{scale=0.47}{(a) Flujos de calor liberados por cada reservorio $\alpha$. (b) Flujos de corriente de partículas correspondientes a cada reservorio. Los resultados se obtuvieron considerando temperaturas $T = 100$ y $T_d = 2$, niveles de energía $\epsilon = 0$ y $\epsilon_d = \mu_d - U/2$ con potencial químico $\mu_d = 2$, razones de túnel $\gamma_L = \gamma^{U}_{R} = 1/100$, $\gamma_R = \gamma^{U}_{L} = 1/600$, $\gamma_D = \gamma^{U}_{D} = 1/50$, acoplamiento entre los sitios $g = 5/1000$ e interacciones de Coulomb $U = 40$ y $U_F = 500$.}

\insertimage[\label{img:fig2resultado}]{ejemplos/entropyapparentdemon.pdf}{scale=0.45}{Tasa de producción de entropía aparente del sistema $LR$($\dot{\sigma}^{o}_{LR}$) en función de $eV/T$.}

En la Figura \ref{img:fig2resultado} se grafica la tasa de producción de entropía aparente, donde se observa que para $eV/T \lessapprox 2.4$, la tasa de producción de entropía aparente $\dot{\sigma}^{o}_{LR}$ es menor a cero, lo que implica una violación aparente de la segunda ley de la termodinámica. Sin embargo, al incorporar la contribución del flujo de información en el cálculo de la tasa de producción de entropía, se obtiene la Figura \ref{img:fig3resultado}, en la cual se verifica que la producción total de entropía en el sistema $LR$ permanece siempre positiva.

\insertimage[\label{img:fig3resultado}]{ejemplos/entropyrialdemon.pdf}{scale=0.45}{Tasa de producción de entropía del sistema $LR$($\dot{\sigma}_{LR}$) en función de $eV/T$.}

En el régimen de parámetros en que \( eV/T \lessapprox 2.4 \), el subsistema \( LR \) realiza trabajo. Para determinar si este trabajo es impulsado principalmente por el flujo de información, es necesario analizar el intercambio energético entre el sistema y el demonio. En el estado estacionario se cumple:

\[
\dot{E}_{L} + \dot{E}_{R} + \dot{E}_{d} = \partial_{t}E = \dot{E}_{LR} + \dot{E}_{d} = 0.
\]

Para que el sistema pueda interpretarse como un Demonio de Maxwell, el trabajo realizado por el subsistema \( LR \) debe estar dominado por la contribución informacional. Esta condición se puede expresar como:

\[
\dot{\mathcal{F}}_{LR} = \dot{E}_{LR} + T\dot{I}_{LR} = -\dot{E}_{d} + T\dot{I}_{LR} \approx T\dot{I}_{LR}, \hspace{10mm} \dot{W}_{LR} \geq \dot{\mathcal{F}}_{LR},
\]

donde \( \dot{\mathcal{F}}_{LR} \) representa la tasa de cambio de energía libre del subsistema \( LR \), incluyendo una contribución informacional \( T\dot{I}_{LR} \).

A fin de verificar este comportamiento, se grafican las cantidades termodinámicas relevantes tanto del sistema como del demonio.


\insertimage[\label{img:fig4resultado}]{ejemplos/thermoquantitiesdemon}{scale=0.53}{ Flujo de energía, tasa de cambio de energía libre, flujo de información, potencia y flujo de calor del sistema $LR$. (b)  Flujo de energía, tasa de cambio de energía libre, flujo de información y potencia del Demonio. }

La Figura \ref{img:fig4resultado} muestra que, en el régimen en el que el subsistema $LR$ realiza trabajo ($\dot{W}_{LR} < 0$ y $\dot{\mathcal{F}}_{LR} < 0$), la contribución informacional domina sobre el flujo energético, es decir, $\dot{\mathcal{F}}_{LR} \approx T \dot{I}_{LR}$ mientras que $\dot{E}_{LR} \approx 0$. Este resultado respalda la interpretación del dispositivo como un Demonio de Maxwell autónomo.

La potencia negativa $\dot{W}_{LR} < 0$ indica que el sistema está realizando trabajo en contra del gradiente de potencial, lo cual requiere absorber calor del entorno. En efecto, se verifica que $J_{LR} > 0$, señalando que el subsistema $LR$ extrae calor de sus respectivos reservorios.

Dado que no hay transporte neto de partículas a través del Demonio ($\dot{N}_{d} = 0$), el flujo de energía asociado se reduce a $J_{d} = \dot{E}_{d}$. Como se observa en la figura, $J_{d} < 0$, lo que implica que el Demonio disipa calor hacia su baño, caracterizado por una temperatura menor a los reservorios $L$ y $R$ ($T_d < T$). 

Este comportamiento puede entenderse también como el de una máquina térmica, donde el sistema total extrae calor del reservorio caliente ($J_{LR}$), transforma parte de él en trabajo útil ($\dot{W}_{LR}$), y disipa el resto en el baño frío ($J_{d}$), cumpliendo así $J_{LR} + J_{d} = -\dot{W}_{LR}$.

Al estar cada punto cuántico acoplado a un reservorio distinto, es posible descomponer el flujo de información total como $\dot{I}_{LR} = \dot{I}_{L} + \dot{I}_{R}$, lo que permite analizar la contribución informacional de cada punto cuántico por separado. Esta descomposición se ilustra en la Figura \ref{img:figinforesultado}.


\insertimage[\label{img:figinforesultado}]{ejemplos/infolrddemon.pdf}{scale=0.45}{Flujos de información de cada sitio, $L$,$R$ y $D$. }

Es interesante notar que, cuando \( eV/T \gtrapprox 3.6 \), el menor flujo de información proviene del punto cuántico \( R \), mientras que para \( eV/T \lessapprox 3.6 \), el sitio \( L \) es el menor, y por ende entrega más información a $D$. Esta transición se correlaciona con los resultados observados en la Figura~\ref{img:densitymatrix}, donde se muestra que, a medida que aumenta \( eV/T \), la probabilidad de ocupación del estado \( \rho_{110} \) supera a la de \( \rho_{011} \).

Este comportamiento es consistente con la intuición física: un mayor potencial favorece la ocupación del punto cuántico \( L \), mientras que la asimetría en las razones de túnel está diseñada para promover la ocupación de \( R \). Esta competencia reduce la eficacia de la dinámica asociada al mecanismo tipo Demonio de Maxwell, descrita en las Figuras~\ref{img:dinamica1},~\ref{img:dinamica2} y~\ref{img:dinamica3}.

La clave de este mecanismo es que el subsistema \( D \) debe ser capaz de monitorear el estado de ocupación del punto \( R \), para luego permitir una transición hacia \( L \) cuando se encuentra desocupado. Sin embargo, al aumentar la probabilidad de ocupación de \( L \), se interrumpe esta dinámica, lo que reduce el flujo de información desde ese sitio. Como consecuencia, el flujo informacional asociado a \( L \) crece con \( eV/T \), llegando a superar el de \( R \).

Un aspecto relevante de esta dinámica es la posible presencia de efectos cuánticos. Para examinar este fenómeno, se representan dos cantidades: la norma-$l_{1}$ de coherencia \( \mathcal{C}_{l_{1}} \), que permite cuantificar la coherencia cuántica en el subsistema \( LR \), y la concurrencia, que sirve como medida del entrelazamiento entre estados del subsistema $LR$ $(|1,0>,|0,1>)$.

\insertimage[\label{img:fig5resultado}]{ejemplos/quantumdemon.pdf}{scale=0.40}{En línea azul la norma-$l_{1}$ de la coherencia $\mathcal{C}_{l_{1}}$ y en línea rojas la concurrencia $\mathcal{C}_{on}$.}

Se observan dos comportamientos relevantes. Primero, para \( eV/T \lessapprox 1 \), la concurrencia \( \mathcal{C}_{\text{on}} > 0 \), lo que indica la presencia de entrelazamiento entre los sitios \( L \) y \( R \). Segundo, al analizar la coherencia, se aprecia que para \( eV/T \lessapprox 2.4 \) esta disminuye progresivamente hasta anularse, y luego comienza a incrementarse nuevamente a medida que crece \( eV/T \). Este comportamiento es particularmente significativo, ya que la coherencia se anula en un punto cercano al valor en el que la corriente de partículas invierte su dirección.

Dado el comportamiento tipo Demonio de Maxwell del sistema, resulta pertinente examinar cómo estas cantidades cuánticas se ven influenciadas por el parámetro de acoplamiento \( g \). Para ello, se comenzará analizando las cantidades termodinámicas.

\insertimage[\label{img:fig6resultado}]{ejemplos/thermoquantum.pdf}{scale=0.47}{Cantidades termodinámicas para el sistema $LR$ y el Demonio en función de $g/\gamma_{L}$, considerando $eV/T=1$ y el resto de parámetros corresponden a los de la Figura \ref{img:fig1resultado}. }

Se observa que cuando el acoplamiento \( g \) alcanza valores comparables con la tasa de túnel \( \gamma_{L} \), tanto la potencia generada como la contribución informacional del subsistema \( LR \) aumentan. Este resultado sugiere que un acoplamiento suficientemente fuerte entre el sistema y el demonio potencia la conversión de información en trabajo. 

Además, bajo esta misma condición \( g \sim \gamma_{L} \), el valor absoluto del flujo de energía del demonio \( \dot{E}_{d} \) también se incrementa, lo que indica mayor transporte energético entre el demonio y el sistema $LR$.

Por otra parte, el análisis de las contribuciones individuales a los flujos de información revela cómo el acoplamiento modifica la cantidad de información entregada por cada punto cuántico. Estos efectos se visualizan en la Figura \ref{img:fig7resultado}, donde se aprecia la dependencia de los flujos informacionales con respecto al parámetro \( g \).

\insertimage[\label{img:fig7resultado}]{ejemplos/concuinfodemon.pdf}{scale=0.51}{(a)Flujos de información de cada sitio. (b)Concurrencia y coherencia en función de $g/\gamma_{L}$. }

Un comportamiento notable se presenta cuando el acoplamiento alcanza valores comparables a la tasa de túnel, es decir, \( g \sim \gamma_{L} \). En este régimen, tanto la concurrencia como la coherencia alcanzan valores cercanos a sus máximos, lo que indica una presencia significativa de efectos cuánticos entre los puntos \( L \) y \( R \), resultado que se asemeja a \cite{prech2023entanglement}. 

Simultáneamente, los flujos de información individuales tienden a igualarse, es decir, \( \dot{I}_{L} \approx \dot{I}_{R} \), lo que sugiere que, en condiciones de entrelazamiento y coherencia elevadas, ambos puntos cuánticos contribuyen de manera equilibrada en el flujo de información del sistema $LR$. 


\label{sec5:demonio1}

\newpage 

\section{Comparación con caso clásico}
Una de las ventajas del formalismo presentado en \cite{potts2021thermodynamically} es que permite incluir términos no seculares en la dinámica del sistema. Esto posibilita la aparición de coherencias y entrelazamiento no nulos en el régimen \( g \sim \gamma_{L} \). La pregunta natural que surge es si estos efectos cuánticos representan una ventaja para el funcionamiento del demonio. Para abordar esta cuestión, se puede considerar una evolución puramente(semi) clásica, limitada a las componentes diagonales de la matriz de densidad del sistema, la cual está gobernada por

\begin{equation}
    \frac{d}{dt}\check{\mathcal{P}}|\hat{\rho}_{S}(t)\rangle \rangle = (\check{\mathcal{L}}_{0} - \check{\mathcal{P}}\check{\mathcal{V}}\check{\mathcal{Q}}\check{\mathcal{L}}^{-1}_{0}\check{\mathcal{Q}}\check{\mathcal{V}}\check{\mathcal{P}})\check{\mathcal{P}}|\hat{\rho}_{S}(t)\rangle \rangle, 
\label{ec5:classicalmodel}
\end{equation}

En este contexto, la evolución del sistema está gobernada por el superoperador total $\check{\mathcal{L}}_{f}$, definido como
\begin{equation*}
\check{\mathcal{L}}_{f}|\hat{\rho}\rangle\rangle = -i[\hat{H}_{S}, \hat{\rho}] + \mathcal{L}(\hat{\rho}),
\end{equation*}
que combina la dinámica coherente del sistema con la disipación inducida por los baños. Este superoperador puede descomponerse como $\check{\mathcal{L}}_{f} = \check{\mathcal{L}}_{0} + \check{\mathcal{V}}$, donde el término $\check{\mathcal{V}}$ representa el acoplamiento entre los sitios y actúa como
\begin{equation*}
\check{\mathcal{V}}|\hat{\rho}\rangle\rangle = -ig[\hat{d}_{L}^{\dagger} \hat{d}_{R} + \hat{d}_{R}^{\dagger} \hat{d}_{L}, \hat{\rho}].
\end{equation*}
Por otro lado, el operador inverso $\check{\mathcal{L}}_{0}^{-1}$ se define a través del inverso de Drazin del operador $\check{\mathcal{L}}_{0}$ como:
\begin{equation*}
\check{\mathcal{L}}_{0}^{-1} = -\int_{0}^{\infty} d\tau\, e^{\check{\mathcal{L}}_{0} \tau} \check{\mathcal{Q}},
\end{equation*}
según lo discutido en~\cite{landi2024current}. Finalmente, $\check{\mathcal{P}}$ y $\check{\mathcal{Q}}$ son los operadores de proyección de Nakajima-Zwanzig sobre las partes diagonal y no diagonal de la matriz densidad del sistema, respectivamente. La demostración detallada de esta ecuación se encuentra en el Apéndice~\ref{appendix5clasic}.

El hecho de disponer de una ecuación maestra que describe únicamente la evolución de las partes diagonales de la matriz densidad permite modelar el sistema de manera estocástica. Esto se logra mediante un vector de probabilidades de ocupación definido como:
\begin{equation*}
\mathbf{P} = [\rho_{000}, \rho_{100}, \rho_{010}, \rho_{001}, \rho_{110}, \rho_{101}, \rho_{011}, \rho_{111}]^{T},
\end{equation*}
cuya dinámica está gobernada por la ecuación maestra

\begin{equation}
    \frac{d}{dt}\textbf{P} = \textbf{W}\textbf{P},
    \label{sec5:classicsemi}
\end{equation}
donde la matriz de transición $\mathbf{W}$ puede obtenerse a partir de la ecuación~(\ref{ec5:classicalmodel}), y se descompone como $\mathbf{W} = \sum_{\alpha} \mathbf{W}_{\alpha}$, donde $\mathbf{W}_{\alpha}$ representa la contribución del baño $\alpha$. 

En lo que sigue, se estudiará el estado estacionario de la ecuación maestra presentada en la ecuación~\ref{sec5:classicsemi}. En particular, se compararán las probabilidades de ocupación obtenidas en este modelo clásico con aquellas resultantes del tratamiento cuántico, lo cual se muestra en las Figuras~\ref{img:rho100} y~\ref{img:rho011}. 

\insertimage[\label{img:rho100}]{ejemplos/rho1004.pdf}{scale=0.46}{Elementos diagonales $\rho_{100}$, $\rho_{010}$, $\rho_{111}$ y $\rho_{110}$ en función de $eV/T$. Donde el caso clásico corresponde a la línea roja y el caso cuántico a la línea azul.}
\insertimage[\label{img:rho011}]{ejemplos/rho1014.pdf}{scale=0.46}{Elementos diagonales $\rho_{101}$, $\rho_{000}$, $\rho_{011}$ y $\rho_{001}$ en función de $eV/T$.}

De las Figuras \ref{img:rho100} y \ref{img:rho011} se observa que en los dos casos, las probabilidades de ocupación dominantes corresponden a $\rho_{100},\rho_{010}$ y $\rho_{001}$. Por el contrario, la probabilidad de ocupación más pequeña es $\rho_{111}$.
 
Se puede notar que, cerca del punto $eV/T \approx 2.4$ donde la coherencia es casi nula, todas las probabilidades de ocupación clásicas se cruzan con sus contrapartes cuánticas. En cambio, para $eV/T \lessapprox 1$, los elementos $\rho_{100}$, $\rho_{010}$ y $\rho_{101}$ presentan una diferencia apreciable entre los comportamientos cuántico y clásico, lo cual se debe al efecto de la coherencia sobre estos elementos.


A partir de la solución estacionaria del sistema, es posible calcular los flujos de corriente de partículas e información asociados al baño $\alpha$ mediante las expresiones

\begin{equation}
    \dot{N}_{\alpha} = -\mathbf{N}^{T} \mathbf{W}_{\alpha}\mathbf{P},
    \label{sec5:currentsemi}
\end{equation}
\begin{equation}
    \dot{I}_{\alpha} = -(\log \mathbf{P})^{T} \mathbf{W}_{\alpha}\mathbf{P},
    \label{sec5:infosemi}
\end{equation}

donde $\mathbf{P}$ es el vector de probabilidades en la base de Fock utilizada, $\mathbf{N}$ contiene el número de partículas en cada estado, y $\log \mathbf{P}$ se refiere al logaritmo aplicado componente a componente.

En el modelo clásico, la expresión analítica para el flujo de corriente de partículas correspondiente al baño $L$ es idéntica al cuántico. Usando la expresión \ref{sec5:currentsemi}, la corriente de partículas esta dada por  

\begin{align*}
    \dot{N}_{L} & = \gamma_{L}(\epsilon)\big(f_{L}(\epsilon)\rho_{000} - [1-f_{L}(\epsilon)]\rho_{100} \big) \\
        & + \gamma_{L}(\epsilon + U) \big(f_{L}(\epsilon+U)\rho_{001} - [1-f_{L}(\epsilon+U)]\rho_{101} \big)  \\
        & + \gamma_{L}(\epsilon + U_{f}) \big(f_{L}(\epsilon+U_{f})\rho_{010} - [1-f_{L}(\epsilon+U_{f})]\rho_{110} \big)  \\  
        & + \gamma_{L}(\epsilon + U + U_{f}) \big(f_{L}(\epsilon+U + U_{f})\rho_{011} - [1-f_{L}(\epsilon+U+U_{f})]\rho_{111} \big).
\end{align*}

Por otro lado, mediante la expresión \ref{sec5:infosemi} se obtienen los flujos de información clásicos asociados a los baños $L$ y $R$, los cuales están descritos por

\begin{align*}
  \dot{I}_{L}  &  =  \gamma_{L}(\epsilon)(f_{L}(\epsilon)[\rho_{000}\ln \rho_{100} - \rho_{000}\ln \rho_{000}] + (1-f_{L}(\epsilon))[\rho_{100}\ln \rho_{000} - \rho_{100}\ln \rho_{100} ] )  \\ 
      & + \gamma_{L}(\epsilon + U)(f_{L}(\epsilon + U)[\rho_{001}\ln \rho_{101} - \rho_{001}\ln \rho_{001}] + (1-f_{L}(\epsilon + U))[\rho_{101}\ln \rho_{001} -\rho_{101}\ln \rho_{101} ]   ) \\  
      & + \gamma_{L}(\epsilon + U_{f})( f_{L}(\epsilon + U_{f})[\rho_{010}\ln \rho_{110} -\rho_{010}\ln \rho_{010} ]  + (1-f_{L}(\epsilon + U_{f}))[\rho_{110}\ln \rho_{010} - \rho_{110}\ln \rho_{110} ] ) \\  
      & + \gamma_{L}(\epsilon + U + U_{f})( f_{L}(\epsilon + U + U_{f})[\rho_{011}\ln \rho_{111}- \rho_{011}\ln \rho_{011}]) \\   
      & + \gamma_{L}(\epsilon + U + U_{f})( (1-f_{L}(\epsilon+U+U_{f}) )[\rho_{111}\ln \rho_{011} - \rho_{111}\ln \rho_{111}])   
\end{align*}

\begin{align*}
 \dot{I}_{R} & =  \gamma_{R}(\epsilon)(f_{R}(\epsilon)[\rho_{000}\ln \rho_{010} - \rho_{000}\ln \rho_{000}] + (1-f_{R}(\epsilon))[\rho_{010}\ln \rho_{000} - \rho_{010}\ln \rho_{010} ] )  \\ 
      & + \gamma_{R}(\epsilon + U)(f_{R}(\epsilon + U)[\rho_{001}\ln \rho_{011} - \rho_{001}\ln \rho_{001}] + (1-f_{R}(\epsilon + U))[\rho_{011}\ln \rho_{001} -\rho_{011}\ln \rho_{011} ]   ) \\  
      & + \gamma_{R}(\epsilon + U_{f})( f_{R}(\epsilon + U_{f})[\rho_{100}\ln \rho_{110} - \rho_{100}\ln \rho_{100} ]  + (1-f_{R}(\epsilon + U_{f}))[\rho_{110}\ln \rho_{100} - \rho_{110}\ln \rho_{110} ] ) \\  
      & + \gamma_{R}(\epsilon + U + U_{f})( f_{R}(\epsilon + U + U_{f})[\rho_{101}\ln \rho_{111}- \rho_{101}\ln \rho_{101}]) \\   
      & + \gamma_{R}(\epsilon + U + U_{f})( (1-f_{R}(\epsilon+U+U_{f}) )[\rho_{111}\ln \rho_{101} - \rho_{111}\ln \rho_{111}]).   
\end{align*}

La versión cuántica de los flujos de información se encuentra detallada en el apéndice~\ref{apendix5infoflow}. En la Figura~\ref{img:P4clqm} se muestra una comparación gráfica entre los casos clásico y cuántico para estas cantidades.

\insertimage[\label{img:P4clqm}]{ejemplos/currentinfosemi.pdf}{scale=0.54}{En línea azul el cálculo realizado con el modelo cuántico que permite coherencias no nulas, mientras que la línea roja corresponde al cálculo semiclásico.}

Como se muestra en la Figura \ref{img:P4clqm}, en el régimen $eV/T\lessapprox 2.4$ la corriente de partículas calculada en el caso cuántico es menor a la del caso clásico. Por otro lado, para $eV/T \gtrapprox 2.4$ la corriente de partículas en el caso cuántico es mayor que en el caso clásico. Es decir, el comportamiento cuántico mejora el transporte de partículas, ya sea en contra o a favor del gradiente de potencial. 

Como se observa en la Figura \ref{img:P4clqm}, el flujo de información asociado al sitio $D$ es mayor en el caso cuántico, mostrando una diferencia apreciable en las regiones $eV/T \lessapprox 1$ y $eV/T \gtrapprox 6$. Este resultado indica que la dinámica cuántica permite una mejora en el intercambio de información entre el sistema $D$ y el subsistema $LR$. Dicha mejora se explica porque el sitio $D$ no solo establece correlaciones clásicas a través de los elementos diagonales del estado, sino que, debido a la presencia de coherencia, también genera correlaciones cuánticas con el sistema $LR$, incluso cuando la interacción entre ellos es exclusivamente de tipo clásico, mediada por la energía de Coulomb. Este comportamiento se corrobora con los flujos de información cuánticos obtenidos analíticamente, los cuales presentan contribuciones provenientes de los elementos no diagonales de la matriz densidad (ver Apéndice~\ref{apendix5infoflow}).
 

Finalmente, debido a este aumento en el flujo de información en el régimen cuántico, la tasa de cambio de la energía libre del subsistema $LR$, dada por $\dot{\mathcal{F}}_{LR} \approx T\dot{I}_{LR}$, disminuye. Como consecuencia, se incrementa la capacidad del sistema para realizar trabajo, lo cual evidencia el rol beneficioso de la coherencia cuántica como recurso termodinámico.


\label{sec5:demonio}


% ------------------------------------------------------------------------------
% NUEVO CAPÍTULO
% ------------------------------------------------------------------------------
\chapteranum{Conclusiones}



% ------------------------------------------------------------------------------
% REFERENCIAS, revisar configuración \stylecitereferences
% ------------------------------------------------------------------------------
\bibliography{library}


% ------------------------------------------------------------------------------
% ANEXO
% Existe adicionalmente el entorno \begin{appendixd} que permite insertar
% \chapter y el entorno \begin{appendixdtitle}[style1] (4 estilos diferentes),
% el cual acepta \chapter y escribe el título de anexos encima
% ------------------------------------------------------------------------------
\begin{appendixs}
	
	\section{Cálculos realizados sección 3}

    \subsection{Matriz de densidad en función del campo de conteo}

    \label{apendix:fcs1}
    Al sustituir la distribución de probabilidad en la función generadora \ref{sec2funciongeneradora}, se obtiene la expresión 

    \begin{align*}
        \Lambda(\vec{\lambda},\vec{\chi}) & = \sum_{\textbf{E},\textbf{E}',\textbf{N},\textbf{N}'} \int d\textbf{Q}d\textbf{W} P_{t}(\textbf{E}',\textbf{N}'|\textbf{E},\textbf{N}) P_{0}(\textbf{E},\textbf{N})  \\
        & \times \Pi_{\alpha} \delta(W_{\alpha} - \mu_{\alpha}(N_{\alpha} - N'_{\alpha}) ) \delta(Q_{\alpha} + W_{\alpha} -(E_{\alpha} - E'_{\alpha})) e^{-i\vec{\lambda}\cdot \textbf{Q}} e^{-i\vec{\chi}\cdot \textbf{W}} \\
        & = \sum_{\textbf{E},\textbf{E}',\textbf{N},\textbf{N}'}P_{t}(\textbf{E}',\textbf{N}'|\textbf{E},\textbf{N})P_{0}(\textbf{E},\textbf{N}) \Pi_{\alpha}e^{-i\chi_{\alpha}\mu_{\alpha}(N_{\alpha} - N'_{\alpha})}e^{-i\lambda_{\alpha}((E_{\alpha} -\mu_{\alpha}N_{\alpha}) - (E'_{\alpha} - \mu_{\alpha}N'_{\alpha}) ) }  \\
        & = \sum_{\textbf{E},\textbf{E}',\textbf{N},\textbf{N}'} \text{Tr}\{\hat{P}_{\textbf{E}',\textbf{N}'}\hat{U}(t)\hat{P}_{\textbf{E},\textbf{N}}(\hat{\rho}_{S}(0) \otimes \Pi_{\alpha}\hat{\tau}_{\alpha} )\hat{P}_{\textbf{E},\textbf{N}}\hat{U}^{\dagger}(t)\hat{P}_{\textbf{E}',\textbf{N}'}   \} \\
        & \times \Pi_{\alpha} e^{-i\chi_{\alpha}\mu_{\alpha}(N_{\alpha} - N'_{\alpha})}e^{-i\lambda_{\alpha}((E_{\alpha} -\mu_{\alpha}N_{\alpha}) - (E'_{\alpha} - \mu_{\alpha}N'_{\alpha}) ) }.
    \end{align*}

Considerando un observable $\hat{A}(0)$ con descomposición espectral asociada a los proyectores $\hat{P}_{a_0}$, al aplicarlos sobre un estado diagonal de densidad $\hat{\rho}_{\text{diag}}$, se obtiene la identidad\cite{esposito2009nonequilibrium}

\begin{equation}
    \sum_{a_0} e^{-i\lambda a_0} \hat{P}_{a_0} \hat{\rho}_{\text{diag}} \hat{P}_{a_0} 
    = e^{-i(\lambda/2)\hat{A}(0)} \hat{\rho}_{\text{diag}} e^{-i(\lambda/2)\hat{A}(0)}.
    \label{apendixobservable}
\end{equation}

Esta relación permite reescribir la función generadora en términos de un observable $\hat{A}(0)$ y el estado inicial diagonal. Así, si se utiliza \ref{apendixobservable} con $\hat{A}(0) = \sum_{\alpha}[\lambda_{\alpha}(\hat{H}_{\alpha}-\mu_{\alpha}\hat{N}_{\alpha}) + \chi_{\alpha}\mu_{\alpha}\hat{N}_{\alpha}] $ se obtiene 
\begin{align*}
    \Lambda(\vec{\lambda},\vec{\chi}) & = \sum_{\textbf{E}',\textbf{N}'}\text{Tr}\{ \hat{P}_{\textbf{E}',\textbf{N}'} \hat{U}(t) e^{-\frac{i}{2}\sum_{\alpha}[\lambda_{\alpha}(\hat{H}_{\alpha} - \mu_{\alpha}\hat{N}_{\alpha}) + \chi_{\alpha}\mu_{\alpha}\hat{N}_{\alpha}  ]}  \hat{\rho}_{tot}(0) e^{-\frac{i}{2}\sum_{\alpha}[\lambda_{\alpha}(\hat{H}_{\alpha} - \mu_{\alpha}\hat{N}_{\alpha}) + \chi_{\alpha}\mu_{\alpha}\hat{N}_{\alpha} ] }\hat{U}^{\dagger}(t) \hat{P}_{\textbf{E}',\textbf{N}'}  \\
    & \times \Pi_{\alpha}e^{i\chi_{\alpha}\mu_{\alpha}\hat{N}_{\alpha}}e^{i\lambda_{\alpha}(\hat{H}_{\alpha} -\mu_{\alpha}\hat{N}_{\alpha})} \} \\
    & = \text{Tr}\{ \hat{U}(\vec{\lambda},\vec{\chi};t)\hat{\rho}_{tot}(0)\hat{U}^{\dagger}(-\vec{\lambda},-\vec{\chi};t)  \}.
\end{align*}

\newpage 

    \subsection{Matriz de densidad generalizada}

La ecuación \ref{sec2FCS:evolution} puede expandirse en serie hasta segundo orden con el objetivo de obtener una expresión aproximada para la evolución de la matriz de densidad generalizada

    \begin{equation*}
        |\hat{\rho}_{totI}(\vec{\lambda},\vec{\chi},t)\rangle \rangle  = \left( \textbf{1} + \epsilon \int_{0}^{t}\check{\mathcal{L}}'_{\lambda}(t_{1})dt_{1} + \epsilon^{2}\int_{0}^{t}dt_{1}\int_{0}^{t_{1}}\check{\mathcal{L}}'_{\lambda}(t_{1})\check{\mathcal{L}}'_{\lambda}(t_{2})dt_{2} \right) |\hat{\rho}_{tot}(0)\rangle \rangle, 
    \end{equation*}
Al introducir el cambio de variables \( t_1 = T \), \( t_2 = T - s \), se transforma la integral doble en una forma más conveniente para su evaluación

\begin{align*}
    |\hat{\rho}_{totI}(\vec{\lambda},\vec{\chi},t)\rangle \rangle  & = \left( \textbf{1} + \epsilon \int_{0}^{t}\check{\mathcal{L}}'_{\lambda}(T)dT + \epsilon^{2}\int_{0}^{t}dT\int_{0}^{T}ds \check{\mathcal{L}}'_{\lambda}(T)\check{\mathcal{L}}'_{\lambda}(T-s) \right)|\hat{\rho}_{tot}(0)\rangle \rangle  \\
    & = \check{\mathcal{W}}(\vec{\lambda},\vec{\chi},t)|\hat{\rho}_{tot}(0)\rangle \rangle \\
    & =  [\check{\mathcal{W}}_{0}(\vec{\lambda},\vec{\chi},t) + \epsilon \check{\mathcal{W}}_{1}(\vec{\lambda},\vec{\chi},t) + \epsilon^{2}\check{\mathcal{W}}_{2}(\vec{\lambda},\vec{\chi},t)] |\hat{\rho}_{tot}(0)\rangle \rangle,
\end{align*}
con los superoperadores definidos por 

\begin{align*}
    \check{\mathcal{W}}_{0}(\vec{\lambda},\vec{\chi},t) &  = \textbf{1}\\
    \check{\mathcal{W}}_{1}(\vec{\lambda},\vec{\chi},t) & = \int_{0}^{t}dT \check{\mathcal{L}'}_{\lambda}(T) \\
    \check{\mathcal{W}}_{2}(\vec{\lambda},\vec{\chi},t) & = \int_{0}^{t}dT \int_{0}^{T}ds \check{\mathcal{L}'}_{\lambda}(T)\check{\mathcal{L}'}_{\lambda}(T-s).
\end{align*}

Considerando la expansión hasta segundo orden de la inversa del superoperador $\check{\mathcal{W}}$

\begin{equation}
    \check{\mathcal{W}}^{-1}(\vec{\lambda},\vec{\chi},t) =  \check{\mathcal{W}}_{0}(\vec{\lambda},\vec{\chi},t) - \epsilon  \check{\mathcal{W}}_{1}(\vec{\lambda},\vec{\chi},t) +  \epsilon^{2}[\check{\mathcal{W}}^{2}_{1}(\vec{\lambda},\vec{\chi},t) -  \check{\mathcal{W}}_{2}(\vec{\lambda},\vec{\chi},t) ],
\label{apendix2inverseW}
\end{equation}
se cumple la propiedad

\begin{equation}
    \dot{\check{\mathcal{W}}}(\vec{\lambda},\vec{\chi},t)\check{A}\check{\mathcal{W}}^{-1}(\vec{\lambda},\vec{\chi},t) = \epsilon \dot{\check{\mathcal{W}}}_{1}(\vec{\lambda},\vec{\chi},t)\check{A} + \epsilon^{2}[\dot{\check{\mathcal{W}}}_{2}(\vec{\lambda},\vec{\chi},t)\check{A} - \dot{\check{\mathcal{W}}}_{1}(\vec{\lambda},\vec{\chi},t)\check{A}\check{\mathcal{W}}_{1}(\vec{\lambda},\vec{\chi},t) ].
    \label{apendix2Wproperty}
\end{equation}

La obtención de la matriz de densidad generalizada requiere proyectar la matriz de densidad total sobre el subespacio del sistema, lo que equivale a trazar los grados de libertad del reservorio. Para ello, se introduce el siguiente operador de proyección

\begin{equation*}
    \check{\mathcal{P}} = \sum_{r}|\rho_{R}^{eq} \rangle \rangle \langle \langle rr|,
\end{equation*}
donde \(\otimes_{\alpha} \hat{\tau}_{\alpha} \to |\hat{\rho}_{R}^{\mathrm{eq}} \rangle\rangle\) representa el estado de equilibrio del reservorio, expresado como un vector en el espacio de Liouville. Al aplicar este proyector sobre la matriz de densidad total, se obtiene

\begin{equation*}
    \check{\mathcal{P}}|\hat{\rho}(\vec{\lambda},\vec{\chi},t) \rangle \rangle = |\hat{\rho}_{S}(\vec{\lambda},\vec{\chi},t)\rangle \rangle \otimes |\hat{\rho}^{eq}_{R}\rangle \rangle .
\end{equation*}

La evolución de la matriz de densidad generalizada, siguiendo el formalismo de Nakajima–Zwanzig, se expresa como

\begin{align}
    \check{\mathcal{P}}|\hat{\rho}_{totI}(\vec{\lambda},\vec{\chi},t)\rangle \rangle & =  \check{\mathcal{P}} \check{\mathcal{W}}(t)( \check{\mathcal{P}} +  \check{\mathcal{Q}})|\hat{\rho}_{tot}(0)\rangle \rangle  \label{apendix2proyectionev1} \\
    \check{\mathcal{Q}}|\hat{\rho}_{totI}(\vec{\lambda},\vec{\chi},t)\rangle \rangle & = \check{\mathcal{Q}} \check{\mathcal{W}}(t)( \check{\mathcal{P}} +  \check{\mathcal{Q}})|\hat{\rho}_{tot}(0)\rangle \rangle. 
\label{apendix2proyectionev}
\end{align}

Existen dos consideraciones fundamentales para continuar con el desarrollo. 

Primero, como se asume que la condición inicial del reservorio es diagonal, se cumple que
\[
\check{\mathcal{Q}}|\hat{\rho}_{\mathrm{tot}}(0)\rangle\rangle = 0.
\]

Segundo, se tiene que el estado total inicial puede expresarse como
\[
|\hat{\rho}_{\mathrm{tot}}(0)\rangle\rangle = \check{\mathcal{W}}^{-1}(\vec{\lambda}, \vec{\chi}, t)\, |\hat{\rho}_{\mathrm{tot}}(\vec{\lambda}, \vec{\chi}, t)\rangle\rangle,
\]
lo cual permite reformular la evolución proyectada en función del estado actual. Es decir, 
\begin{align*}
    |\hat{\rho}_{totI}(0)\rangle \rangle & = (\check{\mathcal{P}} + \check{\mathcal{Q}} )\check{\mathcal{W}}^{-1}(\vec{\lambda},\vec{\chi},t)(\check{\mathcal{P}} + \check{\mathcal{Q}})|\hat{\rho}_{totI}(\vec{\lambda},\vec{\chi},t)\rangle \rangle \\
        & = \check{\mathcal{P}}\check{\mathcal{W}}^{-1}(\vec{\lambda},\vec{\chi},t)(\check{\mathcal{P}} + \check{\mathcal{Q}})|\hat{\rho}_{totI}(\vec{\lambda},\vec{\chi},t)\rangle \rangle.
\end{align*}

Al tomar la derivada respecto al tiempo de las ecuaciones \ref{apendix2proyectionev1} y \ref{apendix2proyectionev}, se obtiene la siguiente expresión para la evolución proyectada

\begin{align*}
    \check{\mathcal{P}}|\dot{\hat{\rho}}_{totI}(\vec{\lambda},\vec{\chi},t) \rangle \rangle & = \check{\mathcal{P}}\dot{\check{\mathcal{W}}}(\vec{\lambda},\vec{\chi},t)\check{\mathcal{P}}\check{\mathcal{W}}^{-1}(\vec{\lambda},\vec{\chi},t)\check{\mathcal{P}}|\hat{\rho}_{totI}(\vec{\lambda},\vec{\chi},t)\rangle \rangle \\
     & = \check{\mathcal{P}}\dot{\check{\mathcal{W}}}(\vec{\lambda},\vec{\chi},t)\check{\mathcal{P}}\check{\mathcal{W}}^{-1}(\vec{\lambda},\vec{\chi},t)\check{\mathcal{Q}}|\hat{\rho}_{totI}(\vec{\lambda},\vec{\chi},t)\rangle \rangle 
\end{align*}

\begin{align*}
    \check{\mathcal{Q}}|\dot{\hat{\rho}}_{totI}(\vec{\lambda},\vec{\chi},t) \rangle \rangle & = \check{\mathcal{Q}}\dot{\check{\mathcal{W}}}(\vec{\lambda},\vec{\chi},t)\check{\mathcal{P}}\check{\mathcal{W}}^{-1}(\vec{\lambda},\vec{\chi},t)\check{\mathcal{P}}|\hat{\rho}_{totI}(\vec{\lambda},\vec{\chi},t)\rangle \rangle \\
     & = \check{\mathcal{Q}}\dot{\check{\mathcal{W}}}(\vec{\lambda},\vec{\chi},t)\check{\mathcal{P}}\check{\mathcal{W}}^{-1}(\vec{\lambda},\vec{\chi},t)\check{\mathcal{Q}}|\hat{\rho}_{totI}(\vec{\lambda},\vec{\chi},t)\rangle \rangle, 
\end{align*}
cabe destacar que las ecuaciones obtenidas son exactas. 

Para incorporar la hipótesis de acoplamiento débil entre el sistema y el reservorio, se comenzará utilizando la relación \ref{apendix2Wproperty}, con lo cual se obtiene

\begin{align*}
    \check{\mathcal{P}}\dot{\check{\mathcal{W}}}(\vec{\lambda},\vec{\chi},t)\check{\mathcal{P}}\check{\mathcal{W}}^{-1}(\vec{\lambda},\vec{\chi},t)\check{\mathcal{Q}} & = \epsilon \check{\mathcal{P}}\dot{\check{\mathcal{W}}}_{1}(\vec{\lambda},\vec{\chi},t) \check{\mathcal{P}}\check{\mathcal{Q}} \\
     & + \epsilon^{2} \check{\mathcal{P}}\dot{\check{\mathcal{W}}}_{2}(\vec{\lambda},\vec{\chi},t) \check{\mathcal{P}}\check{\mathcal{Q}} \\
     & - \epsilon^{2}\check{\mathcal{P}}\dot{\check{\mathcal{W}}}_{1}(\vec{\lambda},\vec{\chi},t)\check{\mathcal{P}} \check{\mathcal{W}}_{1}(\vec{\lambda},\vec{\chi},t)\check{\mathcal{Q}},
\end{align*}
tanto el primer como el segundo término se anulan debido a que \(\check{\mathcal{P}} \check{\mathcal{Q}} = 0\). Por otro lado, 

\begin{align*}
    \check{\mathcal{P}}\dot{\check{\mathcal{W}}}_{1}(\vec{\lambda},\vec{\chi},t)\check{\mathcal{P}} = \sum_{r,r'}|\hat{\rho}^{eq}_{R} \rangle \rangle \langle \langle rr| \check{\mathcal{L}}'_{\lambda}(t)|\hat{\rho}_{R}^{eq}\rangle \rangle \langle \langle r'r'|,
\end{align*}
donde el término \(\langle\langle rr|\check{\mathcal{L}}'_{\lambda}(t)|\hat{\rho}_{R}^{\mathrm{eq}}\rangle\rangle\) equivale a

\begin{equation}
    \langle \langle rr|\check{\mathcal{L}}'_{\lambda}(t)|\hat{\rho}_{R}^{eq}\rangle \rangle  = \text{Tr}_{B}\{\hat{\rho}^{eq}_{R}\hat{V}_{\lambda}(t)\} - \text{Tr}_{B}\{\hat{V}_{-\lambda}(t)\hat{\rho}^{eq}_{R}\}.
    \label{sec3:ecrandom}
\end{equation}

Dado que \(\hat{\rho}_{R}^{\mathrm{eq}}\) conmuta con \(\hat{H}_{R}\), el estado de equilibrio también conmuta con \(\hat{A}(\lambda, \chi)\). Por lo tanto, el término correspondiente a la ecuación \ref{sec3:ecrandom} se anula, quedando únicamente el siguiente término

\begin{align*}
    \check{\mathcal{P}}\dot{\check{\mathcal{W}}}(\vec{\lambda},\vec{\chi},t)\check{\mathcal{P}}\check{\mathcal{W}}^{-1}(\vec{\lambda},\vec{\chi},t)\check{\mathcal{P}} & = \epsilon \check{\mathcal{P}}\dot{\check{\mathcal{W}}}_{1}(\vec{\lambda},\vec{\chi},t)\check{\mathcal{P}} \\
    & + \epsilon^{2}\check{\mathcal{P}}\dot{\check{\mathcal{W}}}_{2}(\vec{\lambda},\vec{\chi},t)\check{\mathcal{P}} \\
    & - \epsilon^{2}\check{\mathcal{P}}\dot{\check{\mathcal{W}}}_{1}(\vec{\lambda},\vec{\chi},t)\check{\mathcal{P}}  \check{\mathcal{W}}_{1}(\vec{\lambda},\vec{\chi},t) \check{\mathcal{P}}.
\end{align*}

De forma explícita, esto corresponde a 

\begin{equation*}
    \check{\mathcal{P}}\dot{\check{\mathcal{W}}}(\vec{\lambda},\vec{\chi},t)\check{\mathcal{P}}\check{\mathcal{W}}^{-1}(\vec{\lambda},\vec{\chi},t)\check{\mathcal{P}} = \epsilon^{2}\check{\mathcal{P}}\int_{0}^{t}ds \check{\mathcal{L}}'_{\lambda}(t)\check{\mathcal{L}}'_{\lambda}(t-s)\check{\mathcal{P}}. 
\end{equation*}

Si se utiliza la relación \(\check{\mathcal{P}}|\dot{\hat{\rho}}_{\mathrm{tot},I}(\vec{\lambda},\vec{\chi},t)\rangle\rangle = |\dot{\hat{\rho}}_{I,S}(\vec{\lambda},\vec{\chi},t)\rangle\rangle \otimes |\hat{\rho}_{R}^{\mathrm{eq}}\rangle\rangle\) y se aplica por la izquierda el operador \(\sum_{r} \langle\langle rr|\), con el objetivo de eliminar los grados de libertad del reservorio, se obtiene

\begin{equation*}
    \dot{\hat{\rho}}_{IS}(\vec{\lambda},\vec{\chi},t) = \epsilon^{2} \sum_{r}\langle \langle rr|\int_{0}^{t}ds \check{\mathcal{L}}'_{\lambda}(t)\check{\mathcal{L}}'_{\lambda}(t-s)|\hat{\rho}_{R}^{eq}\rangle \rangle \hat{\rho}_{IS}(\vec{\lambda},\vec{\chi},t),
\end{equation*}
a partir de esta expresión, es posible derivar las funciones de correlación mediante el cálculo del producto dado por:

\begin{align*}
    \sum_{r}\langle \langle rr| \check{\mathcal{L}}'_{\lambda}(t)\check{\mathcal{L}}'_{\lambda}(t-s)|\hat{\rho}_{R}^{eq}\rangle \rangle \hat{\rho}_{IS}(\vec{\lambda},\vec{\chi},t) = \text{Tr}_{B}\{\mathcal{L}'_{\lambda}(t)\mathcal{L}'_{\lambda}(t-s)\hat{\rho}_{R}^{eq}\hat{\rho}_{IS}(\vec{\lambda},\vec{\chi},t) \},
\end{align*}
desarrollando de manera explícita, los términos de la ecuación 

\begin{equation*}
    \mathcal{L}'_{\lambda}(t-s)\hat{\rho}_{IS}(\vec{\lambda},\vec{\chi},t)\hat{\rho}_{R}^{eq} = -i[\hat{V}_{\lambda}(t-s)\hat{\rho}_{IS}(\vec{\lambda},\vec{\chi},t)\hat{\rho}^{eq}_{R} -  \hat{\rho}_{IS}(\vec{\lambda},\vec{\chi},t)\hat{\rho}^{eq}_{R}\hat{V}_{-\lambda}(t-s)],
\end{equation*}
así, se obtiene 

\begin{align*}
    \mathcal{L}'_{\lambda}(t)\mathcal{L}'_{\lambda}(t-s)\hat{\rho}_{IS}(\vec{\lambda},\vec{\chi},t)\hat{\rho}_{R}^{eq}  = &  -\hat{V}_{\lambda}(t)\hat{V}_{\lambda}(t-s)\hat{\rho}_{IS}(\vec{\lambda},\vec{\chi},t)\hat{\rho}_{R}^{eq} + \hat{V}_{\lambda}(t)\hat{\rho}_{IS}(\vec{\lambda},\vec{\chi},t)\hat{\rho}_{R}^{eq}\hat{V}_{-\lambda}(t-s) \\
    & + \hat{V}_{\lambda}(t-s)\hat{\rho}_{IS}(\vec{\lambda},\vec{\chi},t)\hat{\rho}_{R}^{eq}\hat{V}_{-\lambda}(t) - \hat{\rho}_{IS}(\vec{\lambda},\vec{\chi},t)\hat{\rho}_{R}^{eq}\hat{V}_{-\lambda}(t-s)\hat{V}_{-\lambda}(t).
\end{align*}

Finalmente, al aplicar la traza parcial sobre los grados de libertad del reservorio en esta ecuación, se obtiene la ecuación \ref{ecmaestraVlambda}.


\label{apendixsubsectionmatriz}
    
\newpage
%%%%%%%%%%%%%%%%%%%%%%%%%%%%%%%%%%%%%%%%%%%%%%%%%%%%%%%%
%%%%%%%seccion%%%%%%%%%%%%%%%%%%%%%%%%%%%%%%%%%%%%%%%%%%
%%%%%%%%%%%%%%%%%%%%%%%%%%%%%%%%%%%%%%%%%%%%%%%%%%%%%%%%

\subsection{Funciones correlación}
Para derivar la ecuación maestra generalizada en términos de las funciones de correlación, se partirá de la siguiente expresión

\begin{align*}
& \text{Tr}_{B}\{ \hat{V}_{\lambda}(t)\hat{V}_{\lambda}(t-s) \hat{\rho}_{IS}(\vec{\lambda},\vec{\chi},t)\hat{\rho}^{eq}_{R} \}  =\\
&  - \sum_{\alpha,k,k';j,j'}e^{i(\omega_{j}-\omega_{j'})t}e^{i\omega_{j'}s}\hat{S}^{\dagger}_{\alpha k;j}\hat{S}_{\alpha,k';j'}\hat{\rho}_{IS}(\vec{\lambda},\vec{\chi},t)\text{Tr}_{B}\{e^{-(i/2)\hat{A}(\lambda,\chi)}\hat{V}_{\lambda}(t)\hat{V}_{\lambda}(t-s)e^{(i/2)\hat{A}(\lambda,\chi)}\hat{\rho}^{eq}_{R}\},
\end{align*}    
bajo homogeneidad temporal $\langle \hat{V}_{\lambda}(t)\hat{V}_{\lambda}(t-s) \rangle = \langle \hat{V}_{\lambda}(s)\hat{V}_{\lambda}(0) \rangle$, la ecuación se reduce a 
\begin{align*}
    \text{Tr}_{B}\{ \hat{V}_{\lambda}(t)\hat{V}_{\lambda}(t-s) \hat{\rho}_{IS}(\vec{\lambda},\vec{\chi},t)\hat{\rho}^{eq}_{R} \} & = \\
    & - \sum_{\alpha,k,k';j,j'}e^{i(\omega_{j}-\omega_{j'})t}e^{i\omega_{j'}s}\hat{S}^{\dagger}_{\alpha k;j}\hat{S}_{\alpha,k';j'}\hat{\rho}_{IS}(\vec{\lambda},\vec{\chi},t)\text{Tr}_{B}\{\hat{B}^{\dagger}_{\alpha,k}(s)\hat{B}_{\alpha,k}\hat{\tau}_{\alpha} \} \\
    \text{Tr}_{B}\{ \hat{\rho}_{IS}(\vec{\lambda},\vec{\chi},t)\hat{\rho}^{eq}_{R} \hat{V}_{-\lambda}(t-s)\hat{V}_{-\lambda}(t) \} & = \\
    & - \sum_{\alpha,k,k';j,j'}e^{i(\omega_{j}-\omega_{j'})t}e^{-i\omega_{j}s}\hat{\rho}_{IS}(\vec{\lambda},\vec{\chi},t)\hat{S}^{\dagger}_{\alpha k;j}\hat{S}_{\alpha,k';j'} \text{Tr}_{B}\{ \hat{B}^{\dagger}_{\alpha,k}(s)\hat{B}_{\alpha,k}\hat{\tau}_{\alpha} \}.   
\end{align*}    

Para analizar el término

\begin{equation*}
    \text{Tr}_{B}\{ \hat{V}_{\lambda}(t)\hat{\rho}_{IS}(\vec{\lambda},\vec{\chi},t)\hat{\rho}_{R}^{eq}\hat{V}_{-\lambda}(t-s) \},
\end{equation*}

se emplea la relación de conmutación (conservación global de partículas)

\begin{equation*}
    [\hat{B}_{\alpha,k},\hat{N}_{\alpha}] = n_{\alpha,k}\hat{B}_{\alpha,k},
\end{equation*}
que implica la relación 

\begin{equation}
    e^{C\hat{N}_{\alpha}}\hat{B}_{\alpha,k}e^{-C\hat{N}_{\alpha}} = e^{-Cn_{\alpha,k}}\hat{B}_{\alpha,k}, 
    \label{apendix:conservationparticle}
\end{equation}
con $C$ alguna constante. Se podrá utilizar la relación \ref{apendix:conservationparticle} para obtener

\begin{align*}
   & \text{Tr}_{B}\{ \hat{V}_{\lambda}(t)\hat{\rho}_{IS}(\vec{\lambda},\vec{\chi},t)\hat{\rho}_{R}^{eq}\hat{V}_{-\lambda}(t-s) \} \\
   & = \sum_{\alpha,k,k';j,j'}e^{i(\omega_{j}-\omega_{j'})t}e^{-i\omega_{j}s}\hat{S}_{\alpha,k';j'}\hat{\rho}_{IS}(\vec{\lambda},\vec{\chi},t)\hat{S}^{\dagger}_{\alpha,k;j} \text{Tr}_{B}\{e^{-i(\hat{A}(\lambda,\chi))} \hat{B}_{\alpha,k}(t)e^{i\hat{A}(\lambda,\chi)}\hat{\rho}_{R}^{eq}\hat{B}^{\dagger}_{\alpha,k}(t-s) \}.
\end{align*}

Por otro lado, para los operadores del baño se cumple

\begin{align}
    e^{-i(\hat{A}(\lambda,\chi))} \hat{B}_{\alpha,k}(t)e^{i\hat{A}(\lambda,\chi)} & = e^{-i\mu_{\alpha}n_{\alpha,k}(\lambda_{\alpha}-\chi_{\alpha})} \hat{B}_{\alpha,k}(t+\lambda_{\alpha}), 
    \label{apendix:correlationlambda}
\end{align}
utilizando \ref{apendix:correlationlambda}, se obtiene

\begin{align*}
 & \text{Tr}_{B}\{ \hat{V}_{\lambda}(t)\hat{\rho}_{IS}(\vec{\lambda},\vec{\chi},t)\hat{\rho}_{R}^{eq}\hat{V}_{-\lambda}(t-s) \} = \\
 & \sum_{\alpha,k,k';j,j'}e^{i(\omega_{j}-\omega_{j'})t}e^{-i\omega_{j}s}\hat{S}_{\alpha,k';j'}\hat{\rho}_{IS}(\vec{\lambda},\vec{\chi},t)\hat{S}^{\dagger}_{\alpha,k;j} C^{\alpha}_{k,k'}(-s-\lambda_{\alpha})e^{-i\mu_{\alpha}n_{\alpha,k}(\lambda_{\alpha}-\chi_{\alpha})}
 \end{align*}

 \begin{align*}
    & \text{Tr}_{B}\{ \hat{V}_{\lambda}(t-s)\hat{\rho}_{IS}(\vec{\lambda},\vec{\chi},t)\hat{\rho}_{R}^{eq}\hat{V}_{-\lambda}(t) \} = \\
    & \sum_{\alpha,k,k';j,j'}e^{i(\omega_{j}-\omega_{j'})t}e^{i\omega_{j'}s}\hat{S}_{\alpha,k';j'}\hat{\rho}_{IS}(\vec{\lambda},\vec{\chi},t)\hat{S}^{\dagger}_{\alpha,k;j} C^{\alpha}_{k,k'}(s-\lambda_{\alpha})e^{-i\mu_{\alpha}n_{\alpha,k}(\lambda_{\alpha}-\chi_{\alpha})}.
    \end{align*}
   
Finalmente, al sustituir las expresiones desarrolladas para los términos de correlación en la ecuación \ref{ecmaestraVlambda}, se obtiene la ecuación maestra generalizada para el sistema.
\label{finalequation}

\newpage
%%%%%%%%%%%%%%%%%%%%%%%%%%%%%%%%%%%%%%%%%%%%%%%%%%%%%%
%%%%%%%%%%%%%%%%%seccion%%%%%%%%%%%%%%%%%%%%%%%%%%%%%%
%%%%%%%%%%%%%%%%%%%%%%%%%%%%%%%%%%%%%%%%%%%%%%%%%%%%%%

\subsection{Ecuación de Lindblad generalizada}
Se puede desarrollar la ecuación \ref{ecmaestrafinal} para escribir

\begin{multline*}
    \frac{d}{dt}\hat{\rho}_{IS}(\vec{\lambda},\vec{\chi},t) = - \sum_{\alpha,k,k';q} \int_{0}^{\infty}ds \left[e^{i\omega_{q}s}C^{\alpha}_{k,k'}(s) \left(\sum_{j}e^{i\omega_{j}t}\hat{S}^{\dagger}_{\alpha,k;j} \right)\left(\sum_{j'}e^{-i\omega_{j'}t}\hat{S}_{\alpha,k';j'} \right)\hat{\rho}_{IS}(\vec{\lambda},\vec{\chi},t)  \right. \\
    \left. +  e^{-i\omega_{q}s}C^{\alpha}_{k,k'}(-s) \hat{\rho}_{IS}(\vec{\lambda},\vec{\chi},t) \left(\sum_{j}e^{i\omega_{j}t}\hat{S}^{\dagger}_{\alpha,k;j} \right)\left(\sum_{j'}e^{-i\omega_{j'}t}\hat{S}_{\alpha,k';j'} \right)\right. \\
    \left. - e^{-i\mu_{\alpha}n_{\alpha,k}(\lambda_{\alpha}-\chi_{\alpha})}(e^{i\omega_{q}s}C^{\alpha}_{k,k'}(s-\lambda_{\alpha}) + e^{-i\omega_{q}s}C^{\alpha}_{k,k'}(-s-\lambda_{\alpha}) ) \right.\\
    \left.\times \left(\sum_{j'}e^{-i\omega_{j'}t}\hat{S}_{\alpha,k';j'} \right)\hat{\rho}_{IS}(\vec{\lambda},\vec{\chi},t) \left(\sum_{j}e^{i\omega_{j}t}\hat{S}^{\dagger}_{\alpha,k;j} \right)    \right],   
\end{multline*}
Por simplicidad se asumirá que $C_{k,k'}^{\alpha} \propto \delta_{k,k'}$, así se obtiene 

\begin{multline*}
    \frac{d}{dt}\hat{\rho}_{IS}(\vec{\lambda},\vec{\chi},t) = \\
     - \sum_{\alpha,k;q} \int_{0}^{\infty}ds \left[e^{i\omega_{q}s}C^{\alpha}_{k,k}(s) \hat{S}^{\dagger}_{\alpha,k;q}(t)\hat{S}_{\alpha,k;q}(t)\hat{\rho}_{IS}(\vec{\lambda},\vec{\chi},t)  +  e^{-i\omega_{q}s}C^{\alpha}_{k,k}(-s) \hat{\rho}_{IS}(\vec{\lambda},\vec{\chi},t) \hat{S}^{\dagger}_{\alpha,k;q}(t) \hat{S}_{\alpha,k;q}(t) \right. \\
    \left. - e^{-i\mu_{\alpha}n_{\alpha,k}(\lambda_{\alpha}-\chi_{\alpha})}(e^{i\omega_{q}s}C^{\alpha}_{k,k}(s-\lambda_{\alpha}) + e^{-i\omega_{q}s}C^{\alpha}_{k,k}(-s-\lambda_{\alpha}) ) \hat{S}_{\alpha,k;q}(t) \hat{\rho}_{IS}(\vec{\lambda},\vec{\chi},t) \hat{S}^{\dagger}_{\alpha,k;q}(t)    \right].  
\end{multline*}

Si se usan las identidades

\begin{align*}
    \int_{0}^{\infty}ds e^{i\omega_{q}s}C^{\alpha}_{k,k}(s) & = \int_{-\infty}^{\infty}ds e^{i\omega_{q}s}(1+\text{sgn}(s))C^{\alpha}_{k,k}(s)/2 \\
    & = \int_{-\infty}^{\infty}ds e^{i\omega_{q}s}C^{\alpha}_{k,k}(s)/2 + i \left(-\frac{i}{2} \right) \int_{-\infty}^{\infty}ds e^{i\omega_{q}s} \text{sgn}(s)C^{\alpha}_{k,k}(s) \\
    & = \frac{1}{2}\Gamma_{k,k}^{\alpha}(\omega_{q}) + i \Delta^{\alpha}_{k}(\omega_{q}),
\end{align*}

y 

\begin{equation*}
    \int_{0}^{\infty}dse^{i\omega_{q}s}C^{\alpha}_{k,k}(s-\lambda_{\alpha}) = e^{i\lambda_{\alpha}\omega_{q}}\int_{0}^{\infty}e^{i(s-\lambda_{\alpha})\omega_{q}}C^{\alpha}_{k,k}(s-\lambda_{\alpha}), 
\end{equation*}
en la ecuación maestra, finalmente queda la ecuación 

\begin{multline*}
    \frac{d}{dt}\hat{\rho}_{IS}(\vec{\lambda},\vec{\chi},t) = - i \sum_{\alpha,k;q}\Delta^{\alpha}_{k,k}(\omega_{q})\left[\hat{S}^{\dagger}_{\alpha,k;q}(t)\hat{S}_{\alpha,k;q}(t)\hat{\rho}_{IS}(\vec{\lambda},\vec{\chi},t) - \hat{\rho}_{IS}(\vec{\lambda},\vec{\chi},t)\hat{S}^{\dagger}_{\alpha,k;q}(t)\hat{S}_{\alpha,k;q}(t) \right] \\
    + \sum_{\alpha,k;q} \Gamma_{k,k}^{\alpha}(\omega_{q})\left[ e^{i\lambda_{\alpha}\omega_{q}+ i(\chi_{\alpha} - \lambda_{\alpha})\mu_{\alpha}n_{\alpha,k}}\hat{S}_{\alpha,k;q}(t)\hat{\rho}_{IS}(\vec{\lambda},\vec{\chi},t)\hat{S}^{\dagger}_{\alpha,k;q}(t) - \frac{1}{2}\{\hat{S}^{\dagger}_{\alpha,k;q}(t)\hat{S}_{\alpha,k;q}(t),\hat{\rho}_{IS}(\vec{\lambda},\vec{\chi},t) \} \right].
\end{multline*}

Que es la ecuación que se quiere demostrar.

\label{apendixGKLSgeneral}

\newpage
%%%%%%%%%%%%%%%%%%%%%%%%%%%%%%%%%%%%%%%%%%%%%%%%%%%%%%%
%%%%%%%%%%%%%%%%seccion%%%%%%%%%%%%%%%%%%%%%%%%%%%%%%%%
%%%%%%%%%%%%%%%%%%%%%%%%%%%%%%%%%%%%%%%%%%%%%%%%%%%%%%%


\subsection{Condición KMS y funciones correlación espectral}
Para empezar, se definirá una función correlación auxiliar que consiste en 

\begin{align*}
    C^{\alpha N}_{kk}(s) & =  \langle \hat{B}^{\dagger}_{\alpha,k}(s)\hat{B}_{\alpha,k} \rangle_{N} \\
   &  = \text{Tr}\{e^{is(\hat{H}_{\alpha} - \mu_{\alpha}\hat{N}_{\alpha})}\hat{B}^{\dagger}_{\alpha,k}e^{-is(\hat{H}_{\alpha} - \mu_{\alpha}\hat{N}_{\alpha})}\hat{B}_{\alpha,k}\hat{\tau}_{\alpha}  \} \\
   & = \frac{1}{Z}\text{Tr}\{\hat{B}_{\alpha,k} e^{-\beta_{\alpha}(\hat{H}_{\alpha} - \mu_{\alpha}\hat{N}_{\alpha})} e^{is(\hat{H}_{\alpha} - \mu_{\alpha}\hat{N}_{\alpha})}\hat{B}^{\dagger}_{\alpha,k}e^{-is(\hat{H}_{\alpha} - \mu_{\alpha}\hat{N}_{\alpha})}\} \\
   & = \text{Tr}\{\hat{B}_{\alpha,k}e^{i(s+i\beta_{\alpha})(\hat{H}_{\alpha} - \mu_{\alpha}\hat{N}_{\alpha})}\hat{B}^{\dagger}_{\alpha,k}e^{-i(s+i\beta_{\alpha})(\hat{H}_{\alpha} - \mu_{\alpha}\hat{N}_{\alpha})}\hat{\tau}_{\alpha} \} \\
   & = \langle \hat{B}_{\alpha,k} \hat{B}^{\dagger}_{\alpha,k}(s+i\beta_{\alpha})\rangle_{N},
\end{align*}
que se relaciona con la función correlación, de la siguiente manera 

\begin{align*}
    C^{\alpha N}_{kk}(s) & = \text{Tr}\{e^{-is\mu_{\alpha}\hat{N}_{\alpha} }\hat{B}^{\dagger}_{\alpha,k}(s)e^{is\mu_{\alpha}\hat{N}_{\alpha}} \hat{B}_{\alpha,k}e^{-is\mu_{\alpha}\hat{N}_{\alpha} }\hat{\tau}_{\alpha}   \} \\
 & = \text{Tr}\{\hat{B}^{\dagger}_{\alpha,k}(s)e^{is\mu_{\alpha}\hat{N}_{\alpha}} \hat{B}_{\alpha,k}e^{-is\mu_{\alpha}\hat{N}_{\alpha}}\hat{\tau}_{\alpha}   \}\\
 & = e^{-i\mu_{\alpha}n_{\alpha,k}s} C^{\alpha}_{kk}(s).
\end{align*}

Las funciones correlación espectral, pueden expresarse mediante las funciones correlación auxiliares

\begin{align*}
    \Gamma^{\alpha}_{k,k}(\omega) & = \int_{-\infty}^{\infty}ds e^{i\mu_{\alpha}n_{\alpha,k}s} e^{i\omega s}C^{\alpha N}_{k,k}(s) \\
    & = \int_{-\infty}^{\infty}ds e^{i\mu_{\alpha}n_{\alpha,k}s} e^{i\omega s}\langle \hat{B}_{\alpha,k} \hat{B}^{\dagger}_{\alpha,k}(s+i\beta_{\alpha}) \rangle_{N} \\
    & = \int_{-\infty}^{\infty}ds e^{-i(i\beta \mu_{\alpha}n_{\alpha,k})} e^{i\omega s} \langle \hat{B}_{\alpha,k}\hat{B}^{\dagger}_{\alpha,k}(s+i\beta_{\alpha}) \rangle \\
    & = e^{\beta_{\alpha}\mu_{\alpha}n_{\alpha,k}}e^{\beta \omega} \int_{-\infty}^{\infty} ds e^{i\omega(s+i\beta_{\alpha})} \langle \hat{B}_{\alpha,k}\hat{B}^{\dagger}_{\alpha,k}(s+i\beta_{\alpha}) \rangle \\
    & = e^{\beta_{\alpha}(\omega - \mu_{\alpha}n_{\alpha,k})} \Gamma^{\alpha}_{k,k}(-\omega).
\end{align*}


Esta relación es importante, pues garantiza el cumplimiento de la condición de balance detallado local. 

\label{apendixKMS}


\newpage
%%%%%%%%%%%%%%%%%%%%%%%%%%%%%%%%%%%%%%%%%%%%%%%%%%%%%%%%
%%%%%%%%%%%%%%%%%%%%%%%%%%%%Leyestermo%%%%%%%%%%%%%%%%%%
%%%%%%%%%%%%%%%%%%%%%%%%%%%%%%%%%%%%%%%%%%%%%%%%%%%%%%%%

\subsection{Ley cero}
Para demostrar la ley cero, se evaluarán los disipadores $\mathcal{D}[\hat{S}_{\alpha,k;q}]$ y $\mathcal{D}[\hat{S}^{\dagger}_{\alpha,k;q}]$ en el estado de Gibbs. Para ello, primero se debe considerar la relación de conmutación

\begin{align*}
    [\hat{S}_{\alpha,k;q},(\hat{H}_{TD} - \mu_{\alpha}\hat{N}_{S})] & = (\omega_{q} - \mu_{\alpha}n_{\alpha,k})\hat{S}_{\alpha,k;q}  \\
    [\hat{S}^{\dagger}_{\alpha,k;q},(\hat{H}_{TD} - \mu_{\alpha}\hat{N}_{S})] & = -(\omega_{q} - \mu_{\alpha}n_{\alpha,k})\hat{S}^{\dagger}_{\alpha,k;q},
\end{align*}
así, encontramos que 

 \begin{align*}
    e^{\beta_{\alpha}(\hat{H}_{TD} - \mu_{\alpha}\hat{N}_{S})}\hat{S}_{\alpha,k;q} e^{-\beta_{\alpha}(\hat{H}_{TD} - \mu_{\alpha}\hat{N}_{S})} & = \hat{S}_{\alpha,k;q}e^{-\beta_{\alpha}(\omega_{q} - \mu_{\alpha}n_{\alpha,k})} \\
    e^{\beta_{\alpha}(\hat{H}_{TD} - \mu_{\alpha}\hat{N}_{S})}\hat{S}^{\dagger}_{\alpha,k;q} e^{-\beta_{\alpha}(\hat{H}_{TD} - \mu_{\alpha}\hat{N}_{S})} & = \hat{S}^{\dagger}_{\alpha,k;q}e^{\beta_{\alpha}(\omega_{q} - \mu_{\alpha}n_{\alpha,k})}.
 \end{align*}

Por lo tanto, al aplicar el estado de Gibbs en los disipadores 

\begin{align*}
    \mathcal{D}[\hat{S}_{\alpha,k;q}]e^{-\beta_{\alpha}(\hat{H}_{TD} - \mu_{\alpha}\hat{N}_{S})} & =  e^{-\beta_{\alpha}(\hat{H}_{TD} - \mu_{\alpha}\hat{N}_{s})} \hat{S}_{\alpha,k;q}\hat{S}^{\dagger}_{\alpha,k;q} e^{-\beta_{\alpha(\omega_{q} - \mu_{\alpha}n_{\alpha,k})}} - e^{-\beta_{\alpha}(\hat{H}_{TD} - \mu_{\alpha}\hat{N}_{S})} \hat{S}^{\dagger}_{\alpha,k;q}\hat{S}_{\alpha,k;q} \\
    \mathcal{D}[\hat{S}^{\dagger}_{\alpha,k;q}]e^{-\beta_{\alpha}(\hat{H}_{TD} - \mu_{\alpha}\hat{N}_{S})} & = e^{-\beta_{\alpha}(\hat{H}_{TD} - \mu_{\alpha}\hat{N}_{s})} \hat{S}^{\dagger}_{\alpha,k;q}\hat{S}_{\alpha,k;q} e^{\beta_{\alpha(\omega_{q} - \mu_{\alpha}n_{\alpha,k})}} - e^{-\beta_{\alpha}(\hat{H}_{TD} - \mu_{\alpha}\hat{N}_{S})} \hat{S}_{\alpha,k;q}\hat{S}^{\dagger}_{\alpha,k;q}, 
\end{align*}
Al sumar las contribuciones de ambos disipadores en el superoperador de Lindblad $\mathcal{L}_{\alpha}$, se obtiene 

\begin{equation*}
    \mathcal{L}_{\alpha}e^{-\beta_{\alpha}(\hat{H}_{TD} - \mu_{\alpha}\hat{N}_{S})} \propto  \mathcal{D}[\hat{S}_{\alpha,k;q}]e^{-\beta_{\alpha}(\hat{H}_{TD} - \mu_{\alpha}\hat{N}_{S})} + e^{-\beta_{\alpha}(\omega_{q} - \mu_{\alpha}n_{\alpha,k})}\mathcal{D}[\hat{S}^{\dagger}_{\alpha,k;q}]e^{-\beta_{\alpha}(\hat{H}_{TD} - \mu_{\alpha}\hat{N}_{S})} = 0.
\end{equation*}

Con lo que se prueba la ley cero.

\newpage

%%%%%%%%%%%%%%%%%%%%%%%%%%%%%%%%%%%%%%%%%%%%%%%%%%%%%%%
%%%%%%%%%%%%%%%%%%%seccion%%%%%%%%%%%%%%%%%%%%%%%%%%%
%%%%%%%%%%%%%%%%%%%%%%%%%%%%%%%%%%%%%%%%%%%%%%%%%%%%%%%

\subsection{Segunda Ley}
Para derivar la segunda Ley de la termodinámica, empezamos calculando al derivada temporal de la entropía de Von Neumann

\begin{align*}
    - \frac{d}{dt}\text{Tr}\{ \hat{\rho}_{S}(t)\ln \hat{\rho}_{S}(t) \} & =  -\text{Tr}\Big\{ \frac{d}{dt}\hat{\rho}_{S}(t)\ln \hat{\rho}_{S}(t) \Big\} - \frac{d}{dt}\text{Tr}\{\hat{\rho}_{S}(t) \}\\
  & = - i \text{Tr}\{[\hat{H}_{S}+\hat{H}_{LS},\hat{\rho}_{S}(t)]\ln \hat{\rho}_{S}(t)  \} - \sum_{\alpha} \text{Tr}\{(\mathcal{L}_{\alpha}\hat{\rho}_{S}(t)) \ln \hat{\rho}_{S}(t) \}  \\
  & = -\text{Tr}\{(\mathcal{L}_{\alpha}\hat{\rho}_{S}(t)) \ln \hat{\rho}_{S}(t) \},
\end{align*}
por otro lado, el flujo de calor consiste en 

\begin{align*}
    J_{\alpha} & = \text{Tr}\{ (\hat{H}_{TD} - \mu_{\alpha}\hat{N}_{s})\mathcal{L}_{\alpha}(\hat{\rho}_{S}(t)) \} \\
    & = -\frac{1}{\beta_{\alpha}} \text{Tr}\{(\mathcal{L}_{\alpha}\hat{\rho}_{S}(t)) \ln \hat{\rho}_{G}(\beta_{\alpha},\mu_{\alpha})  \} + \frac{\text{Tr}\{e^{-\beta_{\alpha}(\hat{H}_{TD} - \mu_{\alpha}\hat{N}_{S})} \} }{\beta_{\alpha}} \text{Tr}\{(\mathcal{L}_{\alpha}\hat{\rho}_{S}(t)) \} \\
    & = -\frac{1}{\beta_{\alpha}} \text{Tr}\{(\mathcal{L}_{\alpha}\hat{\rho}_{S}(t))\ln \hat{\rho}_{G}(\beta_{\alpha},\mu_{\alpha})  \},
\end{align*}
Así, utilizando la desigualdad de Spohn se redefine la segunda ley de la termodinámica como

\begin{equation*}
    \dot{\sigma} = - \sum_{\alpha} \text{Tr}\{(\mathcal{L}_{\alpha}\hat{\rho}_{S}(t)) [\ln \hat{\rho}_{S}(t) -\ln \hat{\rho}_{G}(\beta_{\alpha},\mu_{\alpha}) ] \} \geq 0.
\end{equation*}

Finalmente, se obtiene que la tasa de producción de entropía es siempre mayor igual a cero.

\label{apendix:thermolaws}
	% Imagen, se numerará automáticamente con la letra del anexo según
	% la configuración \appendixindepobjnum

\newpage 

    \section{Cálculos realizados sección 4}
    \subsection{Producción de entropía y información}
    Partiendo de la definición de la entropía conjunta para dos variables aleatorias discretas \( X \) e \( Y \)

\begin{equation}
    S^{XY} = - \sum_{x,y} p(x,y) \ln p(x,y),
    \label{apendixSxy}
\end{equation}

donde \( p(x,y) \) denota la distribución de probabilidad conjunta de \( X \) e \( Y \). Al derivar temporalmente la expresión \ref{apendixSxy} y utilizar la propiedad de antisimetría de la corriente de probabilidades, \( J_{x,x'}^{y,y'} = - J_{x',x}^{y',y} \), se obtiene


    \begin{align*}
        \partial_{t}S^{XY} & = - \sum_{x,y} \dot{p}(x,y) \ln p(x,y) - \sum_{x,y} \dot{p}(x,y) \\
                           & = - \sum_{x,x';y,y'} J_{x,x'}^{y,y'} \ln p(x,y)  \\
                           & = \sum_{x \geq x'; y\geq y'} J_{x,x'}^{y,y'} \ln \frac{p(x',y')}{p(x,y)} \\
                           & = \sum_{x \geq x'; y\geq y'} J_{x,x'}^{y,y'} \ln \frac{W_{x,x'}^{y,y'} p(x',y')}{W_{x',x}^{y',y} p(x,y)} +  \sum_{x \geq x'; y\geq y'} J_{x,x'}^{y,y'} \ln \frac{W_{x',x}^{y',y} }{W_{x,x'}^{y,y'} } \\
                           & = \dot{\sigma} - \dot{S}_{r}.
    \end{align*}

Para demostrar que la producción de entropía es no negativa, obsérvese que si \( J_{x,x'}^{y,y'} > 0 \), entonces se cumple que 
\[
W_{x,x'}^{y,y'} p(x',y') > W_{x',x}^{y',y} p(x,y),
\]
y, por lo tanto, el cociente
\[
\frac{W_{x,x'}^{y,y'} p(x',y')}{W_{x',x}^{y',y} p(x,y)} > 1,
\]
lo que implica que su logaritmo es positivo. Como el término correspondiente a la producción de entropía es proporcional a \( J_{x,x'}^{y,y'} \ln \left( \frac{W_{x,x'}^{y,y'} p(x',y')}{W_{x',x}^{y',y} p(x,y)} \right) \), cada uno de estos términos contribuye positivamente, y por simetría de los índices se concluye que la suma total es mayor o igual a cero.

Por su parte, la derivada temporal de la información mutua se expresa como

\begin{equation*}
    \partial_{t} I_{XY} = \sum_{x,y} \dot{p}(x,y) \ln \frac{p(x,y)}{p(x)p(y)} + \sum_{x,y} p(x,y) \frac{\partial}{\partial t} \left( \ln \frac{1}{p(x)p(y)} \right),
\end{equation*}

donde, al derivar el logaritmo, se obtiene

\begin{align*}
    \frac{\partial}{\partial t} \left( \ln \frac{1}{p(x)p(y)} \right) 
    &= - \left( \frac{\dot{p}(x)}{p(x)} + \frac{\dot{p}(y)}{p(y)} \right),
\end{align*}

por lo que

\begin{align*}
    \partial_{t} I_{XY} &= \sum_{x,y} \dot{p}(x,y) \ln \frac{p(x,y)}{p(x)p(y)} 
    - \sum_{x,y} p(x,y) \left( \frac{\dot{p}(x)}{p(x)} + \frac{\dot{p}(y)}{p(y)} \right).
\end{align*}

Reescribiendo los últimos términos con ayuda de la identidad \(\sum_{y} p(x,y) = p(x)\) y la conservación de la probabilidad (\(\sum_{x} \dot{p}(x) = 0\)), se llega a

\begin{align}
    \partial_{t} I_{XY} 
    &= \sum_{x,y} \dot{p}(x,y) \ln \frac{p(x,y)}{p(x)p(y)}.
\end{align}

Utilizando la definición de corriente de probabilidad \(J_{x,x'}^{y,y'}\), se puede reescribir como

\begin{align}
    \partial_{t} I_{XY} &= \sum_{x,x';y,y'} J_{x,x'}^{y,y'} \ln \frac{p(x,y)}{p(x)p(y)}.
\end{align}

Separando la dinámica de \(X\) y de \(Y\), se obtiene

\begin{align*}
    \partial_{t} I_{XY} &= \sum_{x,x';y} J_{x,x'}^{y} \ln \frac{p(x,y)}{p(x)p(y)} 
    + \sum_{x;y,y'} J_{x}^{y,y'} \ln \frac{p(x,y)}{p(x)p(y)}.
\end{align*}

Agrupando términos simétricos mediante pares ordenados \(x \geq x'\) y \(y \geq y'\), se llega a

\begin{align*}
    \partial_{t} I_{XY} &= \sum_{x \geq x';y} J_{x,x'}^{y} \left[ \ln \frac{p(x,y)}{p(x)p(y)} - \ln \frac{p(x',y)}{p(x')p(y)} \right] \nonumber \\
    &\quad + \sum_{x;y \geq y'} J_{x}^{y,y'} \left[ \ln \frac{p(x,y)}{p(x)p(y)} - \ln \frac{p(x,y')}{p(x)p(y')} \right].
\end{align*}

Utilizando la regla de Bayes \(p(x,y) = p(x)p(y|x) = p(y)p(x|y)\), se simplifica como

\begin{align*}
    \partial_{t} I_{XY} 
    &= \sum_{x \geq x';y} J_{x,x'}^{y} \ln \frac{p(y|x)}{p(y|x')} 
     + \sum_{x;y \geq y'} J_{x}^{y,y'} \ln \frac{p(x|y)}{p(x|y')}.
\end{align*}

Esta expresión permite interpretar la variación de la información mutua como la suma de dos flujos:

\begin{equation*}
    \partial_{t} I_{XY} = \dot{I}^{X} + \dot{I}^{Y},
\end{equation*}

donde \(\dot{I}^{X}\) e \(\dot{I}^{Y}\) representan las contribuciones individuales debidas a la dinámica en los grados de libertad \(X\) y \(Y\), respectivamente.


    \label{apendix4:secondlaw}

	 % Desactiva el color de celda

\newpage

\section{Cálculos realizados seccion 5}
\subsection{Funciones de correlación para un baño de fermiones libres}
Para obtener las funciones de correlación espectral, se comenzará escribiendo las funciones correlación temporal 

\begin{equation*}
    C_{1}^{\alpha}(s) = \text{Tr}[e^{is\hat{H}_{\alpha}}\hat{B}^{\dagger}_{\alpha,1}e^{-is\hat{H}_{\alpha}}\hat{B}_{\alpha,1}\hat{\tau}_{\alpha}  ]  \hspace{10mm} C_{-1}^{\alpha}(s) = \text{Tr}[e^{is\hat{H}_{\alpha}}\hat{B}^{\dagger}_{\alpha,-1}e^{-is\hat{H}_{\alpha}}\hat{B}_{\alpha,-1}\hat{\tau}_{\alpha}  ], 
\end{equation*}
Reescribiendo las funciones correlación, en función de los operadores fermionicos

\begin{equation*}
    C_{1}^{\alpha}(s) = \sum_{l,l'}t_{\alpha,l}t_{\alpha,l'}e^{i\epsilon_{\alpha,l}s} \langle \hat{c}^{\dagger}_{\alpha,l}\hat{c}_{\alpha,l'} \rangle \hspace{10mm} C
    ^{\alpha}_{-1}(s) = \sum_{l,l'}t_{\alpha,l}t_{\alpha,l'}e^{-i\epsilon_{\alpha,l}s} \langle \hat{c}_{\alpha,l}\hat{c}^{\dagger}_{\alpha,l'} \rangle.
\end{equation*}

En el equilibrio gran canónico, el valor de expectación $\langle \hat{c}^{\dagger}_{\alpha,l}\hat{c}_{\alpha,l}\rangle$ es la distribución de Fermi $f_{\alpha}(\epsilon_{\alpha,l})$. Por lo tanto  

\begin{equation*}
    C_{1}^{\alpha}(s) = \sum_{l}t^{2}_{\alpha,l}e^{i\epsilon_{\alpha,l}s} f_{\alpha}(\epsilon_{\alpha,l}) \hspace{10mm} C
    ^{\alpha}_{-1}(s) = \sum_{l}t^{2}_{\alpha,l}e^{-i\epsilon_{\alpha,l}s} [1-f_{\alpha}(\epsilon_{\alpha,l})],
\end{equation*}
luego, si se define la tasa de túnel correspondiente al baño \( \alpha \) como

\begin{equation*}
    \gamma_{\alpha}(\omega) = 2\pi \sum_{l} |t_{\alpha,l}|^{2} \delta(\omega - \epsilon_{\alpha,l}),
\end{equation*}

se pueden reescribir las funciones de correlación en forma integral en términos de dicha tasa y de la distribución de ocupación del reservorio \( f_\alpha(\omega) \)

\begin{align*}
    C_{1}^{\alpha}(s) &= \frac{1}{2\pi} \int_{-\infty}^{\infty} d\omega\, e^{i\omega s}\, \gamma_{\alpha}(\omega) f_{\alpha}(\omega), \\
    C_{-1}^{\alpha}(s) &= \frac{1}{2\pi} \int_{-\infty}^{\infty} d\omega\, e^{-i\omega s}\, \gamma_{\alpha}(\omega)\big[1 - f_{\alpha}(\omega)\big].
\end{align*}

A partir de estas expresiones temporales, se puede obtener la forma espectral de las funciones de correlación a través de la transformada de Fourier

\begin{align*}
    \Gamma_{1}^{\alpha}(\omega) &= \int_{-\infty}^{\infty} ds\, e^{i \omega s} C_{1}^{\alpha}(s) = \gamma_{\alpha}(-\omega) f_{\alpha}(-\omega), \\
    \Gamma_{-1}^{\alpha}(\omega) &= \int_{-\infty}^{\infty} ds\, e^{i \omega s} C_{-1}^{\alpha}(s) = \gamma_{\alpha}(\omega)\big[1 - f_{\alpha}(\omega)\big].
\end{align*}

Estas funciones espectrales codifican tanto la densidad de estados accesibles del entorno como su grado de ocupación.

Más adelante, se abordará en detalle el régimen Markoviano, discutiendo bajo qué condiciones las funciones de correlación se aproximan por distribuciones localizadas temporalmente, y cómo esto permite simplificar el tratamiento dinámico del sistema.

\label{apendix5bathcorre}

\subsection{ Operadores de salto de sistema de 3 puntos cuánticos}
Para identificar los operadores de salto, primeramente se elimina el término de acoplamiento del Hamiltoniano del sistema mediante la transformación unitaria

\begin{align*}
    \hat{d}_{-} & = \cos(\theta/2)\hat{d}_{R} - \sin(\theta/2)\hat{d}_{L} \\
    \hat{d}_{+} & = \sin(\theta/2)\hat{d}_{R} + \cos(\theta/2)\hat{d}_{L},
\end{align*}
con 

\begin{equation*}
    \cos \theta = \Delta/\sqrt{ \Delta^{2} + g^{2} } \hspace{10mm} \Delta = \frac{(\epsilon_{L}-\epsilon_{R})}{2}.
\end{equation*}

Si se invierte la transformación unitaria, se pueden escribir los operadores locales en función de los operadores globales 

\begin{align*}
    \hat{d}_{R} & = \cos(\theta/2)\hat{d}_{-} + \sin(\theta/2)\hat{d}_{+} \\
    \hat{d}_{L} & = -\sin(\theta/2)\hat{d}_{-} + \cos(\theta/2)\hat{d}_{+}.
\end{align*}

Por otro lado

\begin{align*}
    \hat{d}^{\dagger}_{R}\hat{d}_{R} & = \cos^{2}(\theta/2) \hat{d}^{\dagger}_{-}\hat{d}_{-} + \sin^{2}(\theta/2) \hat{d}^{\dagger}_{+}\hat{d}_{+} + \cos(\theta/2)\sin(\theta/2)[\hat{d}^{\dagger}_{+}\hat{d}_{-} + \hat{d}^{\dagger}_{-}\hat{d}_{+} ] \\
    \hat{d}^{\dagger}_{L}\hat{d}_{L} & = \sin^{2}(\theta/2) \hat{d}^{\dagger}_{-}\hat{d}_{-} + \cos^{2}(\theta/2) \hat{d}^{\dagger}_{+}\hat{d}_{+} - \cos(\theta/2)\sin(\theta/2)[\hat{d}^{\dagger}_{+}\hat{d}_{-} + \hat{d}^{\dagger}_{-}\hat{d}_{+} ],
\end{align*}
lo que implica

\begin{equation}
    \hat{n}_{L} + \hat{n}_{R} = \hat{n}_{+} + \hat{n}_{-},
    \label{apendixnumber}
\end{equation}
Usando la relación \ref{apendixnumber} y la cantidad definida por $\bar{\epsilon} = (\epsilon_{R} + \epsilon_{L})/2$, se obtiene 

\begin{equation}
    \epsilon_{R} \hat{n}_{R} + \epsilon_{L} \hat{n}_{L}  = \bar{\epsilon}( \hat{n}_{+} + \hat{n}_{-} ) - \Delta \sin(\theta) [\hat{d}^{\dagger}_{+}\hat{d}_{-} + \hat{d}^{\dagger}_{-}\hat{d}_{+}] - \Delta \cos(\theta) [\hat{d}^{\dagger}_{-}\hat{d}_{-} - \hat{d}^{\dagger}_{+}\hat{d}_{+}].
\label{apendix5:ec1}
\end{equation}

Complementariamente

\begin{equation}
    \hat{d}^{\dagger}_{R}\hat{d}_{L} = \cos^{2}(\theta/2)\hat{d}^{\dagger}_{+}\hat{d}_{-} - \sin^{2}(\theta/2) \hat{d}^{\dagger}_{-}\hat{d}_{+}  + \sin(\theta/2)\cos(\theta/2)[ \hat{d}^{\dagger}_{+}\hat{d}_{+} - \hat{d}^{\dagger}_{-}\hat{d}_{-} ]
    \label{apendix5:ec2}
\end{equation}

\begin{equation}
    \hat{d}^{\dagger}_{L}\hat{d}_{R} = \cos^{2}(\theta/2)\hat{d}^{\dagger}_{-}\hat{d}_{+} - \sin^{2}(\theta/2) \hat{d}^{\dagger}_{+}\hat{d}_{-}  + \sin(\theta/2)\cos(\theta/2)[ \hat{d}^{\dagger}_{+}\hat{d}_{+} - \hat{d}^{\dagger}_{-}\hat{d}_{-} ].
    \label{apendix5:ec3}
\end{equation}

De la combinación de \ref{apendix5:ec1}, \ref{apendix5:ec2} y \ref{apendix5:ec3} se puede mostrar que 

\begin{equation*}
    \epsilon_{R} \hat{n}_{R} + \epsilon_{L} \hat{n}_{L} = (\bar{\epsilon} + \sqrt{\Delta^{2} + g^{2}})\hat{n}_{+} +  (\bar{\epsilon} - \sqrt{\Delta^{2} + g^{2}})\hat{n}_{-}.
\end{equation*}

Para trabajar el término de Coulomb en el Hamiltoniano, se aplicará que los operadores de Fermi cumplen $\hat{n}^{2}_{i} = \hat{n}_{i}$, así el término de Coulomb 

\begin{align*}
    2 \hat{n}_{R}\hat{n}_{L} & = (\hat{n}_{R} +\hat{n}_{L})(\hat{n}_{R} +\hat{n}_{L}) - (\hat{n}_{R} + \hat{n}_{L}) \\
    & = (\hat{n}_{+} +\hat{n}_{-})(\hat{n}_{+} +\hat{n}_{-}) - (\hat{n}_{+} + \hat{n}_{-})  = 2\hat{n}_{+}\hat{n}_{-}.
\end{align*}

Finalmente, se definen las energías $\epsilon_{\pm} = \bar{\epsilon} \pm \sqrt{\Delta^{2}+g^{2}}$ para escribir el Hamiltoniano en la base global, dado por 

\begin{equation}
    \hat{H}_{S} = \epsilon_{D}\hat{n}_{D} + \epsilon_{+}\hat{n}_{+} + \epsilon_{-}\hat{n}_{-} + U\hat{n}_{D}(\hat{n}_{+} + \hat{n}_{-}) + U_{f}\hat{n}_{+}\hat{n}_{-}.
    \label{apendix5:ec4}
\end{equation}

El escribir el Hamiltoniano en función de la baase global, permite calcular los operadores de salto, que están dados por 

\begin{align*}
    e^{i \hat{H}_{S}t}\hat{d}_{D}e^{-i\hat{H}_{S}t} & = \hat{d}_{D} + it[\hat{H}_{S},\hat{d}_{D}] + \frac{(it)^{2}}{2} [\hat{H}_{S},[\hat{H}_{S},\hat{d}_{D}]] +... \\
    e^{i \hat{H}_{S}t}\hat{d}_{+}e^{-i\hat{H}_{S}t} & = \hat{d}_{+} + it[\hat{H}_{S},\hat{d}_{+}] + \frac{(it)^{2}}{2} [\hat{H}_{S},[\hat{H}_{S},\hat{d}_{+}]] +... \\
    e^{i \hat{H}_{S}t}\hat{d}_{-}e^{-i\hat{H}_{S}t} & = \hat{d}_{-} + it[\hat{H}_{S},\hat{d}_{-}] + \frac{(it)^{2}}{2} [\hat{H}_{S},[\hat{H}_{S},\hat{d}_{-}]] +...,
\end{align*}
para calcular los operadores de salto correspondientes al operador $\hat{d}_{D}$, se usará la identidad

\begin{equation*}
    \textbf{1} = (\textbf{1} - \hat{n}_{+})(\textbf{1}-\hat{n}_{-}) + (\textbf{1} - \hat{n}_{+})\hat{n}_{-} + (\textbf{1} - \hat{n}_{-})\hat{n}_{+} + \hat{n}_{+}\hat{n}_{-},
\end{equation*}
al utilizar esta identidad y la relación $(\textbf{1} - \hat{n}_{i})\hat{n}_{i} = \textbf{0}$. El conmutador $[\hat{H}_{S},\hat{d}_{D}]$ se puede separar en los términos

\begin{align*}
    [\hat{H}_{S},\hat{d}_{D}(\textbf{1}-\hat{n}_{+})(\textbf{1} - \hat{n}_{-})] & = - \epsilon_{D}\hat{d}_{D}(\textbf{1}-\hat{n}_{+})(\textbf{1} - \hat{n}_{-}) \\
    [\hat{H}_{S},\hat{d}_{D}(\textbf{1}-\hat{n}_{+})\hat{n}_{-}] & = - (\epsilon_{D} + U)\hat{d}_{D}(\textbf{1} - \hat{n}_{+})\hat{n}_{-} \\
    [\hat{H}_{S},\hat{d}_{D}(\textbf{1}-\hat{n}_{-})\hat{n}_{+}] & = - (\epsilon_{D} + U)\hat{d}_{D}(\textbf{1} - \hat{n}_{-})\hat{n}_{+} \\
    [\hat{H}_{S},\hat{d}_{D}\hat{n}_{+}\hat{n}_{-}] & = - (\epsilon_{D} + 2U)\hat{d}_{D}\hat{n}_{+}\hat{n}_{-}, 
\end{align*}
al aplicar nuevamente el conmutador

\begin{align*}
    [\hat{H}_{S},[\hat{H}_{S},\hat{d}_{D}]] & = (\epsilon_{D})^{2}\hat{d}_{D}(\textbf{1} - \hat{n}_{+}) (\textbf{1} - \hat{n}_{-}) + (\epsilon_{D} + U)^{2}\hat{d}_{D}(\textbf{1} - \hat{n}_{+})\hat{n}_{-} \\
        & + (\epsilon_{D}+U)^{2}\hat{d}_{D}(\textbf{1} - \hat{n}_{-})\hat{n}_{+} + (\epsilon_{D} + 2U)^{2}\hat{d}_{D}\hat{n}_{+}\hat{n}_{-}.
\end{align*}

Finalmente 

\begin{align*}
    e^{i \hat{H}_{S}t}\hat{d}_{D}e^{-i\hat{H}_{S}t}  & = e^{-i\epsilon_{D}t} \hat{d}_{D}(\textbf{1} - \hat{n}_{+}) (\textbf{1} - \hat{n}_{-}) +  e^{-i(\epsilon_{D}+U)t} \hat{d}_{D}[(\textbf{1} - \hat{n}_{+})\hat{n}_{-} + (\textbf{1} - \hat{n}_{-})\hat{n}_{+}] \\
    & + e^{-i(\epsilon_{D} + 2U)t}\hat{d}_{D} \hat{n}_{+}\hat{n}_{-}.
\end{align*}

Para calcular los operadores de salto de $\hat{d}_{+}$, se usará la identidad 

\begin{equation*}
    \textbf{1} = (\textbf{1}-\hat{n}_{D})(\textbf{1}-\hat{n}_{-}) + (\textbf{1}-\hat{n}_{-})\hat{n}_{D} + (\textbf{1}-\hat{n}_{D})\hat{n}_{-} + \hat{n}_{D}\hat{n}_{-},
\end{equation*}
así, los conmutadores que incluyen $\hat{d}_{+}$ se podrán separar en 

\begin{align*}
    [\hat{H}_{S},\hat{d}_{+}(\textbf{1}-\hat{n}_{D})(\textbf{1} - \hat{n}_{-})] & = - \epsilon_{+}\hat{d}_{+}(\textbf{1}-\hat{n}_{D})(\textbf{1} - \hat{n}_{-}) \\
    [\hat{H}_{S},\hat{d}_{+}(\textbf{1}-\hat{n}_{D})\hat{n}_{-}] & = - (\epsilon_{+} + U)\hat{d}_{+}(\textbf{1} - \hat{n}_{-})\hat{n}_{D} \\
    [\hat{H}_{S},\hat{d}_{+}(\textbf{1}-\hat{n}_{-})\hat{n}_{D}] & = - (\epsilon_{+} + U_{f})\hat{d}_{+}(\textbf{1} - \hat{n}_{D})\hat{n}_{-} \\
    [\hat{H}_{S},\hat{d}_{+}\hat{n}_{D}\hat{n}_{-}] & = - (\epsilon_{+} + U + U_{f})\hat{d}_{+}\hat{n}_{D}\hat{n}_{-}, 
\end{align*}
aplicando recursivamente estos conmutadores, se obtiene la expresión para los operadores de salto 

\begin{align*}
    e^{i \hat{H}_{S}t}\hat{d}_{+}e^{-i\hat{H}_{S}t} & = e^{-i\epsilon_{+}t}\hat{d}_{+}(\textbf{1}-\hat{n}_{D})(\textbf{1}-\hat{n}_{-}) + e^{-i(\epsilon_{+}+U_{f})t} \hat{d}_{+}(\textbf{1}-\hat{n}_{D})\hat{n}_{-} \\
    & + e^{-i(\epsilon_{+}+U)t}\hat{d}_{+}(\textbf{1}-\hat{n}_{-})\hat{n}_{D} + e^{-i(\epsilon_{+}+U+U_{f})t}\hat{d}_{+}\hat{n}_{-}\hat{n}_{D},
\end{align*}
debido a la simetría entre los operadores de la base global $\hat{d}_{+}$ y $\hat{d}_{-}$, los operadores de salto de $\hat{d}_{-}$ son

\begin{align*}
    e^{i \hat{H}_{S}t}\hat{d}_{-}e^{-i\hat{H}_{S}t} & = e^{-i\epsilon_{-}t}\hat{d}_{-}(\textbf{1}-\hat{n}_{D})(\textbf{1}-\hat{n}_{+}) + e^{-i(\epsilon_{-}+U_{f})t} \hat{d}_{-}(\textbf{1}-\hat{n}_{D})\hat{n}_{+} \\
    & + e^{-i(\epsilon_{-}+U)t}\hat{d}_{-}(\textbf{1}-\hat{n}_{+})\hat{n}_{D} + e^{-i(\epsilon_{-}+U+U_{f})t}\hat{d}_{-}\hat{n}_{+}\hat{n}_{D}.
\end{align*}

Por su parte, se pueden expresar los operadores de salto en la base local $\hat{d}_{L}$ y $\hat{d}_{R}$ como 
\begin{align*}
    e^{i\hat{H}_{S}t}\hat{d}_{R}e^{-i\hat{H}_{S}t} & = \cos(\theta/2)\, e^{i\hat{H}_{S}t}\hat{d}_{-}e^{-i\hat{H}_{S}t} + \sin(\theta/2)\, e^{i\hat{H}_{S}t}\hat{d}_{+}e^{-i\hat{H}_{S}t}, \\
    e^{i\hat{H}_{S}t}\hat{d}_{L}e^{-i\hat{H}_{S}t} & = -\sin(\theta/2)\, e^{i\hat{H}_{S}t}\hat{d}_{-}e^{-i\hat{H}_{S}t} + \cos(\theta/2)\, e^{i\hat{H}_{S}t}\hat{d}_{+}e^{-i\hat{H}_{S}t}.
\end{align*}

A partir de esto, se identifican las frecuencias de Bohr asociadas a los operadores de creación y destrucción en el sistema

\begin{align*}
    \hat{d}_{D} & \rightarrow \{\epsilon_{D}, \epsilon_{D}+U, \epsilon_{D}+2U\}, \\
    \hat{d}_{L}, \hat{d}_{R} & \rightarrow \{\epsilon_{\pm}, \epsilon_{\pm}+U, \epsilon_{\pm}+U_{f}, \epsilon_{\pm}+U+U_{f}\}.
\end{align*}



\label{apendix5jumpop}

\subsection{Agrupación de frecuencias para 2 puntos cuánticos}
En el caso degenerado $\epsilon_{L} = \epsilon_{R}$ se tiene que $\epsilon_{\pm} = \epsilon \pm g$. En este régimen, si el parámetro de acoplamiento $g$ es pequeño en comparación con el inverso del tiempo de correlación del baño, es posible agrupar estas dos frecuencias. La agrupación de frecuencias estará dada por 

\begin{align*}
    &(\epsilon_{+},\epsilon_{-})  \to \epsilon \\
    &(\epsilon_{+}+U,\epsilon_{-}+U)  \to \epsilon+U \\
    &(\epsilon_{+}+U_{f},\epsilon_{-}+U_{f})  \to \epsilon+U_{f} \\
    &(\epsilon_{+}+U+U_{f},\epsilon_{-}+U+U_{f})  \to \epsilon+U+U_{f}.
\end{align*}

Para determinar los operadores de salto asociados al operador $\hat{d}_{L}$ en el caso degenerado, se debe considerar la contribución de las transiciones con frecuencias $\epsilon_{+}$ y $\epsilon_{-}$. Es decir, se deben sumar las contribuciones correspondientes a los operadores $\hat{d}_{+}$ y $\hat{d}_{-}$, para formar un único operador efectivo de salto que posee la frecuencia agrupada. Por ende se debe calcular 

\begin{align*}
    \frac{\hat{d}_{+}(\textbf{1}-\hat{n}_{D})(\textbf{1}-\hat{n}_{-})}{\sqrt{2}} - \frac{\hat{d}_{-}(\textbf{1}-\hat{n}_{D})(\textbf{1}-\hat{n}_{+})}{\sqrt{2}} & = \frac{(\textbf{1}-\hat{n}_{D})}{\sqrt{2}} [\hat{d}_{+}(\textbf{1}-\hat{n}_{-}) - \hat{d}_{-}(\textbf{1}-\hat{n}_{+}) ],
\end{align*}
además 

\begin{align*}
    \hat{d}_{+}(\textbf{1}-\hat{n}_{-}) - \hat{d}_{-}(\textbf{1}-\hat{n}_{+}) & = \frac{\hat{d}_{L} + \hat{d}_{R}}{\sqrt{2}} (\textbf{1} - \hat{n}_{-}) - \frac{\hat{d}_{R} - \hat{d}_{L}}{\sqrt{2}}(\textbf{1}-\hat{n}_{+}) \\
    & = \frac{\hat{d}_{L}}{\sqrt{2}}(\textbf{2} - \hat{n}_{+} -\hat{n}_{-} ) + \frac{\hat{d}_{R}}{\sqrt{2}}(\hat{n}_{+} -\hat{n}_{-}) \\
    & = \frac{\hat{d}_{L}}{\sqrt{2}}(\textbf{2}-\hat{n}_{L} -\hat{n}_{R}) + \frac{\hat{d}_{R}}{\sqrt{2}}(\hat{d}^{\dagger}_{R}\hat{d}_{L}+\hat{d}^{\dagger}_{L}\hat{d}_{R}),
\end{align*}
Dado que $\hat{d}_{L}(\mathbf{1} - \hat{n}_{L}) = \hat{d}_{L}\hat{d}_{L}\hat{d}^{\dagger}_{L} = 0$ y que 

\[
\hat{d}_{R}(\hat{d}^{\dagger}_{R}\hat{d}_{L} + \hat{d}^{\dagger}_{L}\hat{d}_{R}) = \hat{d}_{L}(\mathbf{1} - \hat{n}_{R}),
\]

se obtiene la identidad

\[
\hat{d}_{+}(\mathbf{1} - \hat{n}_{-}) - \hat{d}_{-}(\mathbf{1} - \hat{n}_{+}) = \hat{d}_{L}(\mathbf{1} - \hat{n}_{R}),
\]

lo que permite asociar la frecuencia $\epsilon$ al operador $\hat{d}_{L}(\mathbf{1} - \hat{n}_{D})(\mathbf{1} - \hat{n}_{R})$.

Del mismo modo, para las frecuencias agrupadas $(\epsilon_{-} + U,\, \epsilon_{+} + U) \to \epsilon + U$, el operador de salto efectivo queda determinado por

\[
\frac{\hat{d}_{+}\hat{n}_{D}(\mathbf{1} - \hat{n}_{-})}{\sqrt{2}} - \frac{\hat{d}_{-}\hat{n}_{D}(\mathbf{1} - \hat{n}_{+})}{\sqrt{2}} = \hat{d}_{L}\hat{n}_{D}(\mathbf{1} - \hat{n}_{R}).
\]

Finalmente, para el caso de frecuencias $(\epsilon_{-} + U_{f},\, \epsilon_{+} + U_{f}) \to \epsilon + U_{f}$, los operadores de salto agrupados son

\[
\frac{\hat{d}_{+}\hat{n}_{-}(\mathbf{1} - \hat{n}_{D})}{\sqrt{2}} - \frac{\hat{d}_{-}\hat{n}_{+}(\mathbf{1} - \hat{n}_{D})}{\sqrt{2}} = \frac{(\mathbf{1} - \hat{n}_{D})}{\sqrt{2}}\left[\hat{d}_{+}\hat{n}_{-} - \hat{d}_{-}\hat{n}_{+}\right],
\]
donde la cantidad $\left[\hat{d}_{+}\hat{n}_{-} - \hat{d}_{-}\hat{n}_{+}\right]$ es 

\begin{align*}
    \hat{d}_{+}\hat{n}_{-} -\hat{d}_{-}\hat{n}_{+} & = \frac{\hat{d}_{R}+\hat{d}_{L}}{\sqrt{2}}\hat{n}_{-} - \frac{\hat{d}_{R}-\hat{d}_{L}}{\sqrt{2}}\hat{n}_{+} \\
    & = - \frac{\hat{d}_{R}}{\sqrt{2}}(\hat{d}^{\dagger}_{R}\hat{d}_{L}+ \hat{d}^{\dagger}_{L}\hat{d}_{R}) + \frac{\hat{d}_{L}}{\sqrt{2}}(\hat{n}_{L}+\hat{n}_{R}) \\
    & = \frac{2\hat{d}_{L}}{\sqrt{2}}\hat{n}_{R} - \frac{\hat{d}_{L}}{\sqrt{2}}(\textbf{1}-\hat{n}_{L}) \\
    & = \frac{2\hat{d}_{L}}{\sqrt{2}}\hat{n}_{R}.
\end{align*}

Lo que permite asociar de forma coherente la frecuencia $\epsilon + U_{f}$ al operador $\hat{d}_{L}\hat{n}_{R}(\mathbf{1}-\hat{n}_{D})$. En recapitulación, al agrupar las frecuencias los operadores de salto estan dados por 

\begin{align*}
    &\epsilon \to \hat{d}_{L}(\textbf{1}-\hat{n}_{R})(\textbf{1}-\hat{n}_{D})\\
   &\epsilon + U \to \hat{d}_{L}\hat{n}_{D}(\textbf{1}-\hat{n}_{R})\\
    &\epsilon +U_{f}\to \hat{d}_{L}\hat{n}_{R}(\textbf{1}-\hat{n}_{D})\\
    &\epsilon +U+U_{f}\to \hat{d}_{L}\hat{n}_{D}\hat{n}_{R},
\end{align*}
Análogamente, en el caso del operador local $\hat{d}_{R}$, se procede agrupando las energías $(\epsilon{+}, \epsilon_{-})$, así calcular  
\begin{align*}
    \frac{\hat{d}_{+}(\textbf{1}-\hat{n}_{D})(\textbf{1}-\hat{n}_{-}) }{\sqrt{2}} + \frac{\hat{d}_{-}(\textbf{1}-\hat{n}_{D})(\textbf{1}-\hat{n}_{+}) }{\sqrt{2}} & = \frac{(\textbf{1} - \hat{n}_{D})}{\sqrt{2}}[\hat{d}_{+}(\textbf{1}-\hat{n}_{-}) + \hat{d}_{-}(\textbf{1}-\hat{n}_{+})],
\end{align*}
además  

\begin{align*}
    \hat{d}_{+}(\textbf{1}-\hat{n}_{-}) + \hat{d}_{-}(\textbf{1}-\hat{n}_{+}) & =  \frac{\hat{d}_{R} + \hat{d}_{L}}{\sqrt{2}}(\textbf{1} - \hat{n}_{-}) + \frac{\hat{d}_{R} - \hat{d}_{L}}{\sqrt{2}}(\textbf{1} - \hat{n}_{+})   \\
    & = \frac{\hat{d}_{R}}{\sqrt{2}}(\textbf{2} - (\hat{n}_{L} + \hat{n}_{R})) + \frac{\hat{d}_{L}}{\sqrt{2}} (\hat{n}_{+}-\hat{n}_{-}) \\
    & = \frac{\hat{d}_{R}}{\sqrt{2}}(\textbf{1} -  \hat{n}_{R}) + \frac{\hat{d}_{L}}{\sqrt{2}} (\hat{d}^{\dagger}_{R}\hat{d}_{L} + \hat{d}^{\dagger}_{L}\hat{d}_{R}) \\
    & = \sqrt{2}\hat{d}_{R}(\textbf{1}-\hat{n}_{R}),
\end{align*}
por lo tanto, el operador de salto asociado a la agrupación $(\epsilon_{+},\epsilon_{-}) \to \epsilon$ es $\hat{d}_{R}(\textbf{1} - \hat{n}_{D})(\textbf{1} - \hat{n}_{L})$. De manera análoga, al calcular el operador correspondiente a la agrupación de frecuencias $(\epsilon_{+}+U, \epsilon_{-}+U)$

\begin{equation*}
    \frac{\hat{d}_{+}(\textbf{1}-\hat{n}_{-})\hat{n}_{D}}{\sqrt{2}} + \frac{\hat{d}_{-}(\textbf{1}-\hat{n}_{+})\hat{n}_{D}}{\sqrt{2}}  = \hat{d}_{R}\hat{n}_{D}(\textbf{1}-\hat{n}_{L}).
\end{equation*}

Para las frecuencias $(\epsilon_{+}+U_{f},\epsilon_{-}+U_{f})\to \epsilon + U_{f}$, se tendrán que sumar los operadores 

\begin{equation*}
    \frac{\hat{d}_{+}(\textbf{1}-\hat{n}_{D})\hat{n}_{-} }{\sqrt{2}} + \frac{\hat{d}_{-}(\textbf{1}-\hat{n}_{D})\hat{n}_{+} }{\sqrt{2}} = \frac{(\textbf{1}-\hat{n}_{D})}{\sqrt{2}} [\hat{d}_{+}\hat{n}_{-} + \hat{d}_{-}\hat{n}_{+}],
\end{equation*}
donde, el operador $\hat{d}_{+}\hat{n}_{-} + \hat{d}_{-}\hat{n}_{+}$ es 

\begin{align*}
    \hat{d}_{+}\hat{n}_{-} + \hat{d}_{-}\hat{n}_{+} & = \frac{\hat{d}_{R}+\hat{d}_{L}}{\sqrt{2}}\hat{n}_{-} + \frac{\hat{d}_{R}-\hat{d}_{L}}{\sqrt{2}}\hat{n}_{+}  \\
    & = \frac{\hat{d}_{R}}{\sqrt{2}}(\hat{n}_{L}+\hat{n}_{R}) - \frac{\hat{d}_{L}}{\sqrt{2}}(\hat{d}^{\dagger}_{L}\hat{d}_{R}+\hat{d}^{\dagger}_{R}\hat{d}_{L}) \\
    & = \sqrt{2}\hat{d}_{R}\hat{n}_{L},
\end{align*}
por lo tanto, el operador de salto de $\hat{d}_{R}$ al agrupar las frecuencias $(\epsilon_{+}+U_{f}, \epsilon_{-}+U_{f})$ es $\hat{d}_{R}(\textbf{1} - \hat{n}_{D})\hat{n}_{L}$; realizando el mismo procedimiento, se encuentra que para las frecuencias $(\epsilon_{+}+U+U_{f}, \epsilon_{-}+U+U_{f})$ el operador de salto corresponde a $\hat{d}_{R}\hat{n}_{D}\hat{n}_{L}$. Por ende, recapitulando, para el operador $\hat{d}_{R}$ se tiene

\begin{align*}
    &\epsilon \to \hat{d}_{R}(\textbf{1}-\hat{n}_{L})(\textbf{1}-\hat{n}_{D})\\
   &\epsilon + U \to \hat{d}_{R}\hat{n}_{D}(\textbf{1}-\hat{n}_{L})\\
    &\epsilon +U_{f}\to \hat{d}_{R}\hat{n}_{L}(\textbf{1}-\hat{n}_{D})\\
    &\epsilon +U+U_{f}\to \hat{d}_{R}\hat{n}_{D}\hat{n}_{L}.
\end{align*}

\label{apendix5frecuencygroup}

%%%%%%%%%%%%%%%%%%%%%%%%%ojooooaqui%%%%%%%%%%%%%
\subsection{Límitaciones de la aproximación semilocal}

\label{apendix5límites}

\subsection{Aspecto Markoviano de las funciones correlación}
Para que el sistema descrito en la sección \ref{sec5:modelo} exhiba un comportamiento análogo al de un Demonio de Maxwell autónomo, es necesario que las tasas de túnel $\gamma_{i}(\omega)$ respondan de manera diferenciada dependiendo de si el sitio $D$ se encuentra ocupado o desocupado. Sin embargo, no toda dependencia funcional de $\gamma_{i}(\omega)$ garantiza un comportamiento Markoviano del sistema. Esta condición puede evaluarse mediante el cálculo explícito de las funciones de correlación del reservorio, ya que el carácter Markoviano se encuentra fuertemente ligado al tiempo de correlación característico del entorno, denotado por $\tau_{B}$. Para analizar este aspecto, se requiere calcular las funciones de correlación correspondientes a un entorno fermiónico, las cuales, en general, adoptan la forma

\begin{equation*}
    C^{\sigma}_{i}(t) = \frac{1}{2\pi} \int_{-\infty}^{\infty} d\omega e^{i\sigma \omega t} \gamma_{i}(\omega) f_{F}(\sigma \beta(\omega-\mu)),
\end{equation*}
donde $f_{F}(x) = (\exp(x)+1)^{-1}$, $i= L,R$ y $\sigma = \pm$. 

La distribución de Fermi puede ser aproximada mediante desarrollos de Padé \cite{hu2011pade,schinabeck2019hierarchical}, lo que permite expresarla como una sumatoria 

\begin{equation}
    f_{F}(x) \approx \frac{1}{2} - \sum_{l=0}^{N} \frac{ 2 \kappa_{l}x }{ x^{2} + \xi^{2}_{l} },
    \label{apendix5:pade}
\end{equation}
los coeficientes $\kappa_{l}$ y $\xi_{l}$ pueden ser calculados numéricamente y se encuentran tabulados en \cite{hu2011pade}. Para evaluar numéricamente la función de correlación, es necesario especificar una forma explícita para $\gamma_{i}(\omega)$, la cual debe presentar un máximo en cierta frecuencia característica $\epsilon_{0}$. Una elección común consiste en considerar un perfil lorentziano dado por

\begin{equation*}
    \gamma_{i}(\omega) = \gamma_{0} + \frac{\gamma_{f} W^{2}}{(\omega - \epsilon_{0})^{2} + W^{2}},
\end{equation*}
donde $W$ representa el ancho de la lorentziana, $\gamma_{0}$ y $\gamma_{f}$ son constantes que controlan la amplitud. En el estudio del Demonio de Maxwell autónomo, es fundamental distinguir dos casos relevantes: cuando el punto cuántico $D$ se encuentra desocupado o bien ocupado. Esta distinción se refleja en la evaluación de las tasas de túnel en dos frecuencias distintas, $\epsilon_{0}$ y $\epsilon_{0} + U$, tal como se ilustra en la Figura \ref{img:gammas} para las tasas de túnel correspondientes a los baños $L$ y $R$.

\insertimage[\label{img:gammas}]{ejemplos/gammmas.pdf}{scale=0.65}{Tasas de túnel en función de las frecuencias $\omega$.}

Para obtener tasas de túnel que satisfagan dichas condiciones, se puede plantear el siguiente sistema de ecuaciones 

\begin{align*}
    \gamma_{i}(\epsilon_{0}) & = \gamma_{0} + \gamma_{f} \\
    \gamma_{i}(\epsilon_{0} + U) & = \gamma_{0} + \frac{\gamma_{f} W^{2} }{U^{2} + W^{2}}.
\end{align*}

A partir de estas igualdades, es posible determinar los parámetros $\gamma_{0}$, $\gamma_{f}$ y $W$. Sustituyendo la expresión de $\gamma_{i}(\omega)$ en la función de correlación, se obtiene

\begin{align*}
    C^{\sigma}_{i}(t) & = \frac{\gamma_{0}}{2\pi} \int_{-\infty}^{\infty}d\omega e^{i\sigma \omega t} f_{F}(\sigma \beta (\omega-\mu)) + \frac{\gamma_{f}}{2\pi} \int_{-\infty}^{\infty}d\omega e^{i\sigma \omega t}\left[ \frac{W^{2}}{(\omega-\epsilon_{0})^{2} + W^{2}} \right] f_{F}(\sigma \beta (\omega-\mu)),
\end{align*}
la función de correlación puede descomponerse en dos contribuciones

\begin{equation*}
    C^{\sigma}_{0}(t) = \frac{\gamma_{0}}{2\pi} \int_{-\infty}^{\infty}d\omega e^{i\sigma \omega t} f_{F}(\sigma \beta (\omega-\mu)) \hspace{12mm} C^{\sigma}_{f}(t) = \frac{\gamma_{f}}{2\pi} \int_{-\infty}^{\infty}d\omega e^{i\sigma \omega t}\left[ \frac{W^{2}}{(\omega-\epsilon_{0})^{2} + W^{2}} \right] f_{F}(\sigma \beta (\omega-\mu)),
\end{equation*}
y se podrá determinar el tiempo de correlación del baño como $\tau_{B} = \max\{\tau_{B0}, \tau_{Bf}\}$. Se comenzará calculando una expresión analítica aproximada para la función de correlación $C^{\sigma}_{f}(t)$. Para resolver esta integral, será necesario localizar los polos en el plano complejo. Según \ref{apendix5:pade}, los polos asociados a la aproximación de Padé se encuentran en $\omega = \pm i \xi_{l}/\beta + \mu$, mientras que los polos correspondientes a la forma Lorentziana se ubican en $\omega = \pm i W + \omega_{0}$. Para realizar esta integración, se utilizará el teorema del residuo \cite{riley2006mathematical}, considerando un contorno de integración en forma de semicírculo en el hemisferio superior o inferior del plano complejo, dependiendo del signo de $\sigma$. El residuo asociado a la Lorentziana es

\begin{align*}
   \text{Res} \left[ \frac{e^{ i\sigma \omega t} f_{F}[\sigma \beta (\omega-\mu)] }{ (\omega-\omega_{0} + iW)(\omega -\omega_{0} -iW)}\right]_{\omega = \pm i W + \omega_{0}} = \frac{1}{\pm 2iW} ( e^{\mp \sigma Wt} f_{F}[\sigma \beta(\pm iW +\omega_{0}-\mu)]e^{i\sigma \omega_{0}t}). 
\end{align*}

Si se cumple que $\sigma > 0$, se toma como contorno de integración el hemisferio inferior del plano complejo; en cambio, si $\sigma < 0$, se considera el hemisferio superior. De esta manera, se obtiene

\begin{align*}
    \text{Res} \left[ \frac{e^{ i\sigma \omega t} f_{F}[\sigma \beta (\omega-\mu)] }{ (\omega-\omega_{0} + iW)(\omega -\omega_{0} -iW)}\right]_{\omega = \pm i W + \omega_{0}} = \frac{1}{ 2iW} ( e^{i\sigma \omega_{0}t}e^{- Wt} f_{F}[i\beta W + \sigma \beta(\omega_{0}-\mu)]). 
 \end{align*}

Para encontrar el residuo asociado a los polos restantes, se deberá calcular

\begin{align*}
    \text{Res} \left[ \frac{-e^{i\sigma \omega t}}{(\omega - \omega_{0})^{2} + W^{2} } \frac{ 2\kappa_{l}[\sigma \beta (\omega-\mu)] }{ (\beta(\omega-\mu) + i \xi_{l} )(\beta(\omega-\mu) - i \xi_{l})} \right]_{\omega = \pm i \xi_{l}/\beta + \mu} & = \frac{-1}{\beta} \frac{e^{- \frac{\sigma \xi_{l}}{\beta}t} e^{i\sigma \mu t}(\pm \kappa_{l}\sigma )}{ [ \frac{ \pm i\xi_{l}}{\beta} + (\mu - \omega_{0}) ]^{2} + W^{2} } ,
\end{align*}
seleccionando la región de integración en función del signo de $\sigma$, se concluye que

\begin{align*}
    \text{Res} \left[ \frac{-e^{i\sigma \omega t}}{(\omega - \omega_{0})^{2} + W^{2} } \frac{ 2\kappa_{l}[\sigma \beta (\omega-\mu)] }{ (\beta(\omega-\mu) + i \xi_{l} )(\beta(\omega-\mu) - i \xi_{l})} \right]_{\omega = \pm i \xi_{l}/\beta + \mu} & = \frac{-1}{\beta} \frac{e^{- \frac{ \xi_{l}}{\beta}t} e^{i\sigma \mu t} \kappa_{l} }{ [ \frac{ \sigma i\xi_{l}}{\beta} + (\mu - \omega_{0}) ]^{2} + W^{2} }. 
\end{align*}

Al encontrar los residuos explícitamente, estos se podrán usar para escribir la función correlación 

\begin{equation}
    C^{\sigma}_{f}(t) \approx \sum_{l=0}^{N} \eta^{\sigma,l} e^{-\gamma_{\sigma,l} t},
    \label{correlationf}
\end{equation}
donde 

\begin{equation*}
    \eta^{\sigma,l} = \left\{ \begin{array}{lc} \frac{\gamma_{f}W}{2} f_{F}[i\beta W + \sigma \beta (\omega_{0}-\mu)]  & l = 0 \\ \\ - \frac{i\kappa_{l}}{\beta} \left(\frac{\gamma_{f}W^{2}}{ (\frac{i\sigma \xi_{l}}{\beta} + (\mu-\omega_{0}))^{2} + W^{2} } \right) &  l \neq 0 \end{array} \right.
\end{equation*}

\begin{equation*}
    \gamma_{\sigma,l} =  \left\{ \begin{array}{lc} W- \sigma i \omega_{0}  & l = 0 \\ \\ \frac{\xi_{l}}{\beta} - \sigma i \mu &  l \neq 0 \end{array} \right..
\end{equation*}

Esto permite obtener una expresión numérica para la función de correlación $C^{\sigma}_{f}(t)$, y con ello evaluar su tiempo de correlación característico, $\tau_{Bf}$. Por otro lado, para la otra función de correlación es posible calcular la integral de forma analítica, es decir, evaluar

\begin{align*}
       C_{0}^{\sigma}(t) = \frac{\gamma_{0}}{2\pi} \int_{-\infty}^{\infty} d\omega e^{i\sigma \omega t }f_{F}(\sigma \beta (\omega -  \mu)).
\end{align*}    

Se puede comenzar observando que este cálculo equivale a obtener la transformada de Fourier 

\[
\mathcal{F}\big(f[\sigma (\omega - \mu)]\big)(-\sigma t),
\]

de la función 

\[
f(\sigma(\omega-\mu)) = f_{F}(\sigma \beta (\omega-\mu)),
\]

empleando la propiedad

\begin{equation*}
\mathcal{F}(f[\sigma (\omega - \mu)])(-\sigma t) = e^{-i \sigma \mu t}\mathcal{F}(f[\sigma \omega])(-\sigma t),
\end{equation*}
es decir, podemos centrarnos en calcular la transformación

\begin{align*}
    \mathcal{F}(f[\sigma \omega])(-\sigma t) &= \frac{\gamma_{0}}{2\pi} \int_{-\infty}^{\infty} d\omega \frac{e^{i\sigma \omega t}}{e^{\sigma \beta \omega} +1 } \\
        & = \frac{\gamma_{0}}{4\pi} \left[\int_{-\infty}^{\infty}d \omega e^{i\sigma \omega t} - \int_{-\infty}^{\infty}d\omega e^{i\sigma \omega t} \tanh \left(\frac{ \sigma \beta \omega }{2} \right)   \right] \\
        & =  \frac{\gamma_{0}}{2} \left[\delta(t) - \frac{i}{\beta \sigma \sinh(\pi t/\beta \sigma)} \right],
\end{align*}
así

\begin{equation*}
    C^{\sigma}_{0}(t) = \frac{\gamma_{0}}{2}e^{-i\sigma \mu t} \left[\delta(t) - \frac{i}{\beta \sigma \sinh(\pi t/\beta \sigma)} \right].
\end{equation*}

Gracias a que se dispone de una expresión analítica para $C^{\sigma}_{0}(t)$, es posible determinar el tiempo de correlación $\tau_{B0}$. El análisis se centra principalmente en el término que involucra $\sinh(\pi t/\beta \sigma)$, ya que, dado que $\sigma = \pm 1$, el parámetro relevante es $\pi t/\beta$. Para tiempos $t > \beta$, se puede aproximar que 

\[
\frac{1}{\sinh(\pi t/\beta \sigma)} \propto e^{-t/\beta},
\]

lo que implica que el tiempo de correlación del baño está dado por el inverso de la temperatura, es decir, $\tau_{B0} = \beta$. Por consiguiente, el tiempo de correlación efectivo asociado a un único reservorio queda definido como

\[
\tau_{B} = \max \{\beta, \tau_{Bf}\}.
\]

\label{appendix5correlation}

\subsection{Cálculo numérico para los tiempos de correlación}
Dado que es posible reconstruir las tasas de túnel $\gamma_{L}(\omega)$ y $\gamma_{R}(\omega)$ mediante una función de tipo Lorentziana, cumpliendo las condiciones $\gamma_{L}(\epsilon) = \gamma_{R}(\epsilon+U) = 1/100$ y $\gamma_{L}(\epsilon+U) = \gamma_{R}(\epsilon) = 1/600$, se pueden calcular numéricamente las funciones correlación asociadas. 

Para determinar los tiempos de correlación $\tau_{Bf}$ correspondientes a los baños $L$ y $R$, se utilizará la expresión \ref{correlationf} de las funciones correlación. La estimación de $\tau_{Bf}$ se realiza identificando el punto donde la función correlación se anula. Posteriormente, para evaluar si el comportamiento del sistema es efectivamente markoviano, se debe verificar que $\tau_{R} \gg \tau_{Bf}$. Con este criterio, se procede al cálculo de las partes real e imaginaria de las funciones de correlación correspondientes a los baños $L$ y $R$.

\insertimage[\label{img:correlacionminus}]{ejemplos/cminusreal.pdf}{scale=0.5}{(a)Parte real de las funciones correlación $\sigma=-$ del baño $L$ en función del tiempo.(b) Parte real de las funciones correlación $\sigma=-$ del baño $R$ en función del tiempo.}
\insertimage[\label{img:correlacionminusimag}]{ejemplos/cminusimag.pdf}{scale=0.5}{(a)Parte imaginaria de las funciones correlación $\sigma=-$ del baño $L$ en función del tiempo.(b) Parte imaginaria de las funciones correlación $\sigma=-$ del baño $R$ en función del tiempo.}

\insertimage[\label{img:correlacionplus}]{ejemplos/cplusreal.pdf}{scale=0.5}{(a)Parte real de las funciones correlación $\sigma=+$ del baño $L$ en función del tiempo.(b) Parte real de las funciones correlación $\sigma=+$ del baño $R$ en función del tiempo.}
\insertimage[\label{img:correlacionplusimag}]{ejemplos/cplusimag.pdf}{scale=0.5}{(a)Parte imaginaria de las funciones correlación $\sigma=+$ del baño $L$ en función del tiempo.(b) Parte imaginaria de las funciones correlación $\sigma=+$ del baño $R$ en función del tiempo.}



A partir de las Figuras \ref{img:correlacionminus}, \ref{img:correlacionminusimag}, \ref{img:correlacionplus} y \ref{img:correlacionplusimag}, se observa que, para $t > 0.4$, las funciones de correlación correspondientes a ambos baños se anulan prácticamente. 

Para analizar el carácter markoviano del sistema, se puede comparar este tiempo de correlación con la dinámica de relajación de la matriz densidad hacia el estado estacionario. Para ello, en la Figura \ref{img:probas} se grafica la evolución temporal de las componentes de la matriz densidad y se analiza el tiempo característico en que estas alcanzan su régimen estacionario.


\insertimage[\label{img:probas}]{ejemplos/rhotiempo.pdf}{scale=0.56}{Componentes $\rho_{100}$ y $\rho_{010}$ de la matriz densidad en función del tiempo, para distintos valores de $eV/T$.}

A partir de la Figura \ref{img:probas}, se observa que el tiempo de relajación del sistema es del orden de $\tau_{R} > 2000$. Esto permite corroborar que, para distintos valores de $eV/T$, se cumple la condición $\tau_{Bf}/\tau_{R} < 0.0002$. Lo que implica que es adecuado realizar la aproximación markoviana para las funciones correlación de los baños $L$ y $R$.

\label{appendix5tauf}

\newpage 

\subsection{Concurrencia}
En esta sección se presenta la demostración de la fórmula de la concurrencia. Para ello, se debe calcular la concurrencia en el subsistema $\hat{\rho}_{LR}$, lo que requiere, como primer paso, el cálculo de

\begin{equation*}
    \hat{\rho}_{LR} = \text{Tr}_{D}\{ \hat{\rho_{S}} \}.
\end{equation*}

Para escribir la matriz densidad $\hat{\rho}_{S}$, se utilizará la base  
\[
\left\{ |0,0,0\rangle,\ |1,0,0\rangle,\ |0,1,0\rangle,\ |0,0,1\rangle,\ |1,1,0\rangle,\ |1,0,1\rangle,\ |0,1,1\rangle,\ |1,1,1\rangle \right\}.
\]
En esta representación, la matriz densidad del sistema se expresa como

\begin{equation}
    \hat{\rho}_{S} = 
    \begin{bmatrix}
        \rho_{000} & 0 & 0 & 0 & 0 & 0 & 0 & 0 \\
        0 & \rho_{100} & \alpha & 0 & 0 & 0 & 0 & 0 \\
        0 & \alpha^{*} & \rho_{010} & 0 & 0 & 0 & 0 & 0 \\
        0 & 0 & 0 & \rho_{001} & 0 & 0 & 0 & 0 \\
        0 & 0 & 0 & 0 & \rho_{110} & 0 & 0 & 0 \\
        0 & 0 & 0 & 0 & 0 & \rho_{101} & \beta & 0 \\
        0 & 0 & 0 & 0 & 0 & \beta^{*} & \rho_{011} & 0 \\
        0 & 0 & 0 & 0 & 0 & 0 & 0 & \rho_{111} 
        \end{bmatrix}.
        \label{appendix5rhoconcu}
\end{equation}

En esta matriz densidad no se observan coherencias entre estados con distinto número total de partículas, lo cual se debe al principio de superselección de carga \cite{bartlett2007reference,wick1997intrinsic}. Además, dado que el punto cuántico $D$ sólo interactúa con los puntos $L$ y $R$ a través de una interacción de tipo Coulomb, no se generan coherencias del tipo $L$-$D$ ni $R$-$D$. 

Es posible obtener la matriz densidad reducida del subsistema $LR$ al proyectar sobre la base  $\{|00\rangle, |10\rangle, |01\rangle, |11\rangle\}$

\begin{equation*}
    \hat{\rho}_{LR} = 
    \begin{bmatrix}
        \rho_{000}+\rho_{001} & 0 & 0 & 0  \\
        0 & \rho_{100} + \rho_{101} & \alpha + \beta & 0  \\
        0 & \alpha^{*} +\beta^{*} & \rho_{010} + \rho_{011} & 0  \\
        0 & 0 & 0 & \rho_{110} + \rho_{111} 
        \end{bmatrix},
\end{equation*}
de este modo, se podrá calcular la concurrencia 

\begin{equation*}
    \mathcal{C}_{on} = \max \{ 0,\lambda_{1} - \lambda_{2} - \lambda_{3} - \lambda_{4} \},
\end{equation*}
donde los $\lambda_{i}$ son la raíz cuadrada de los autovalores ordenados en forma decreciente, de la matriz dada por 

\begin{equation}
    B = \hat{\rho}_{LR} \tilde{\rho}_{LR} \hspace{10mm} \tilde{\rho}_{LR} = (\sigma_{y} \otimes \sigma_{y}) \hat{\rho}_{LR} (\sigma_{y} \otimes \sigma_{y}),
    \label{apendixconcu}
\end{equation}
en la cual $\sigma_{y}$ pertenece a las matrices de Pauli. Al aplicar \ref{apendixconcu} se obtiene 

\begin{equation*}
    \tilde{\rho}_{LR} = 
    \begin{bmatrix}
        p_{D} & 0 & 0 & 0  \\
        0 & p_{R} & \alpha + \beta & 0  \\
        0 & \alpha^{*} +\beta^{*} & p_{L} & 0  \\
        0 & 0 & 0 & p_{0} 
        \end{bmatrix},
\end{equation*}
con  $p_{0}=\rho_{000}+\rho_{001}$, $p_{L}=\rho_{100} + \rho_{101}$, $p_{R}=\rho_{010}+\rho_{011}$, $p_{D}=\rho_{110}+\rho_{111}$  y la matriz $B$ es

\begin{equation*}
    B = 
    \begin{bmatrix}
        p_{0}p_{D} & 0 & 0 & 0  \\
        0 & p_{L}p_{R}+|\alpha+\beta|^{2} & 2(\alpha + \beta)p_{L} & 0  \\
        0 & 2(\alpha^{*} +\beta^{*})p_{R} & p_{L}p_{R}+|\alpha+\beta|^{2}  & 0  \\
        0 & 0 & 0 & p_{0}p_{D}
        \end{bmatrix}.
\end{equation*}

Suponiendo que $p_{D}$ y $p_{0}$ son pequeños, el orden decreciente de los autovalores corresponderá a

\begin{equation*}
    \lambda_{1} = p_{L}p_{R} + |\alpha + \beta| \hspace{10mm}  \lambda_{2} = p_{L}p_{R} - |\alpha + \beta| \hspace{10mm} \lambda_{3}=\lambda_{4} = \sqrt{p_{0}p_{D}},
\end{equation*}
entonces se obtiene

\begin{equation*}
    \mathcal{C}_{on} = \max\{ 2|\alpha+\beta| - 2\sqrt{p_{0}p_{D}},0\}.
\end{equation*}

\label{appendix5final}

\subsection{Modelo clásico}
Para comenzar con la derivación del modelo clásico que describe las componentes diagonales de la matriz densidad $\rho_{ijk}$, con $i,j,k \in {0,1}$, se parte considerando la ecuación de evolución para los operadores Nakajima-Zwanzig

\begin{equation*}
    \frac{d}{dt}\check{\mathcal{Q}}|\hat{\rho}(t)\rangle \rangle = \check{\mathcal{Q}}\check{\mathcal{L}}_{f}\check{\mathcal{Q}}|\hat{\rho}(t)\rangle \rangle + \check{\mathcal{Q}}\check{\mathcal{L}}_{f}\check{\mathcal{P}}|\hat{\rho}(t)\rangle \rangle \to \frac{d}{dt}\check{\mathcal{Q}}|\hat{\rho}(t)\rangle \rangle - \check{\mathcal{Q}}\check{\mathcal{L}}_{f}\check{\mathcal{Q}}|\hat{\rho}(t)\rangle \rangle = \check{\mathcal{Q}}\check{\mathcal{L}}_{f}\check{\mathcal{P}}|\hat{\rho}(t)\rangle \rangle ,
\end{equation*}
que es equivalente a una ecuación no homogenea lineal de la forma

\begin{equation*}
    \frac{dy}{dt} - A(t)y = B(t),
\end{equation*}
por lo tanto, se puede solucionar utilizando factor integrante

\begin{equation*}
    \check{\mu}(t) = \exp\left( - \int_{0}^{t}\check{\mathcal{Q}}\check{\mathcal{L}}_{f}ds \right),
\end{equation*}
multiplicando por el factor integrante la ecuacion diferencial para $\mu(t)$, se obtiene 

\begin{equation*}
    \check{\mu}(t)\frac{d}{dt}\check{\mathcal{Q}}|\hat{\rho}(t)\rangle \rangle - \check{\mu}(t)\check{\mathcal{Q}}\check{\mathcal{L}}_{f}\check{\mathcal{Q}}|\hat{\rho}(t)\rangle \rangle  = \check{\mu}(t)\check{\mathcal{Q}}\check{\mathcal{L}}_{f}\check{\mathcal{P}}|\hat{\rho}(t)\rangle \rangle  \to \frac{d}{dt}(\check{\mu}(t)\check{\mathcal{Q}}|\hat{\rho}(t))\rangle \rangle = \check{\mu}(t)\check{\mathcal{Q}}\check{\mathcal{L}}_{f}\check{\mathcal{P}}|\hat{\rho}(t)\rangle \rangle ,
\end{equation*}
con solución

\begin{equation*}
    \check{\mu}(t)\check{\mathcal{Q}}|\hat{\rho}(t)\rangle \rangle  = \check{\mu}(0)\check{\mathcal{Q}}|\hat{\rho}(0)\rangle \rangle  + \int_{0}^{t}\check{\mu}(\tau)\check{\mathcal{Q}}\check{\mathcal{L}}_{f} \check{\mathcal{P}}|\hat{\rho}(\tau)\rangle \rangle d\tau,   
\end{equation*}
multiplicando por $\check{\mu}^{-1}(t)$ ambos lados de la ecuación, se obtiene

\begin{equation*}
    \check{\mathcal{Q}}|\hat{\rho}(t)\rangle \rangle  = \check{\mu}^{-1}(t)\check{\mathcal{Q}}|\hat{\rho}(0)\rangle \rangle  + \int_{0}^{t}\check{\mu}^{-1}(t)\check{\mu}(\tau)\check{\mathcal{Q}}\check{\mathcal{L}}_{f} \check{\mathcal{P}}|\hat{\rho}(\tau)\rangle \rangle d\tau,    
\end{equation*}
luego, se define el operador 
\begin{equation*}
    \check{\mathcal{G}}(t,s) = \exp\left(  \int_{s}^{t}\check{\mathcal{Q}}\check{\mathcal{L}}_{f}dt' \right),
\end{equation*}
y queda la ecuación 

\begin{equation*}
    \check{\mathcal{Q}}|\hat{\rho}(t)\rangle \rangle = \check{\mathcal{G}}(t,0)\check{\mathcal{Q}}|\hat{\rho}(0)\rangle \rangle  + \int_{0}^{t}\check{\mathcal{G}}(t,\tau)\check{\mathcal{Q}}\check{\mathcal{L}}_{f} \check{\mathcal{P}}|\hat{\rho}(\tau)\rangle \rangle d\tau.    
\end{equation*}

Esto se puede reinsertar en la ecuación principal, obteniéndose así

\begin{equation}
    \frac{d}{dt}\check{\mathcal{P}}|\hat{\rho}(t)\rangle \rangle  = \check{\mathcal{P}}\check{\mathcal{L}}_{f}\check{\mathcal{P}}|\hat{\rho}(t)\rangle \rangle  + \check{\mathcal{P}}\check{\mathcal{L}}_{f}\check{\mathcal{G}}(t,0)\check{\mathcal{Q}}|\hat{\rho}(0)\rangle \rangle  + \check{\mathcal{P}}\check{\mathcal{L}}_{f}\int_{0}^{t}\check{\mathcal{G}}(t,\tau)\check{\mathcal{Q}}\check{\mathcal{L}}_{f} \check{\mathcal{P}}|\hat{\rho}(\tau)\rangle \rangle d\tau.    
    \label{apendix5exactp}
\end{equation}

Estas ecuaciones son exactas, ya que hasta este punto no se ha introducido ninguna aproximación. El primer término del lado derecho de la ecuación \ref{apendix5exactp} representa la evolución markoviana del sistema, mientras que el segundo y tercer término corresponden, respectivamente, a un corrimiento inicial y a los efectos de memoria inducidos por el acoplamiento con el entorno. Para continuar con la deducción, se utilizarán las siguientes relaciones

\begin{equation*}
    \check{\mathcal{P}}\check{\mathcal{V}}\check{\mathcal{P}} = [\check{\mathcal{L}}_0,\check{\mathcal{P}}] = [\check{\mathcal{L}}_0,\check{\mathcal{Q}}] = 0, 
\end{equation*}
estas relaciones pueden entenderse de manera intuitiva; dado que $\check{\mathcal{V}}$ representa la perturbación responsable del túnel cuántico entre los puntos $L$ y $R$, su acción introduce coherencias en la matriz densidad. En particular, al aplicar $\check{\mathcal{V}}$ sobre un estado diagonal, se obtiene una componente no diagonal. Por lo tanto, al aplicar nuevamente el operador de proyección sobre la diagonal $\check{\mathcal{P}}$, esta contribución se anula, es decir $\check{\mathcal{P}} \check{\mathcal{V}} \check{\mathcal{P}} = 0$.

Para continuar con el análisis de la ecuación \ref{apendix5exactp}, se harán dos suposiciones, en primer lugar, que el estado inicial no presenta coherencias, es decir, $\check{\mathcal{Q}}\hat{\rho}(0) = 0$; en segundo lugar, se sustituirá explícitamente el operador $\check{\mathcal{L}}_f$ en la ecuación \ref{apendix5exactp}. Con esto, se procederá a evaluar el término integral, utilizando las propiedades de los operadores de proyección de Nakajima-Zwanzig, comenzando por

\begin{align*}
    \check{\mathcal{G}}(t,\tau) & = \exp\left(  \int_{\tau}^{t}\check{\mathcal{Q}}(\check{\mathcal{L}}_{0} + \check{\mathcal{V}})dt' \right) \\
        & = \exp\left(  \int_{\tau}^{t}\check{\mathcal{Q}}(\check{\mathcal{L}}_{0}+\check{\mathcal{V}})(\check{\mathcal{P}}+\check{\mathcal{Q}}) dt' \right) \\
            & = \exp\left(  \int_{\tau}^{t}\left[\check{\mathcal{Q}}\check{\mathcal{L}}_{0}\check{\mathcal{Q}}+ \check{\mathcal{Q}}\check{\mathcal{V}}\check{\mathcal{P}}+\check{\mathcal{Q}}\check{\mathcal{V}}\check{\mathcal{Q}}\right] dt' \right),       
\end{align*}
(modificar aqui)en donde se uso que $\check{\mathcal{Q}}\check{\mathcal{L}}_{0}\check{\mathcal{P}} = \check{\mathcal{Q}}\check{\mathcal{P}}\check{\mathcal{L}}_{0}=0$. Continuando con la ecuación principal

\begin{equation*}
    \frac{d}{dt}\check{\mathcal{P}}|\hat{\rho}(t)\rangle \rangle = \check{\mathcal{P}}\check{\mathcal{L}}_{0}\check{\mathcal{P}}|\hat{\rho}(t)\rangle \rangle + \check{\mathcal{P}}(\check{\mathcal{L}}_{0} + \check{\mathcal{V}})(\check{\mathcal{P}}+\check{\mathcal{Q}})\int_{0}^{t}\check{\mathcal{G}}(t,\tau)\check{\mathcal{Q}}\check{\mathcal{L}}_{f} \check{\mathcal{P}}|\hat{\rho}(\tau)\rangle \rangle d\tau.       
\end{equation*}

Debido a la forma del exponencial en $\check{\mathcal{G}}$ al aplicar $\check{\mathcal{P}}\check{\mathcal{L}}_{0}\check{\mathcal{G}}(t,\tau)$ se obtiene $\check{\mathcal{P}}\check{\mathcal{L}}_{0}(t-\tau)$, ya que si extendemos el exponencial

\begin{equation*}
    \check{\mathcal{P}}\check{\mathcal{L}}_{0}\exp\left[\int_{\tau}^{t}ds\check{\mathcal{Q}}\check{\mathcal{L}}_{0}\check{\mathcal{Q}}+ \check{\mathcal{Q}}\check{\mathcal{V}}\check{\mathcal{P}}+\check{\mathcal{Q}}\check{\mathcal{V}}\check{\mathcal{Q}} \right] = \check{\mathcal{L}}_{0}\check{\mathcal{P}}\int_{\tau}^{t}\sum_{k=0}^{\infty}\left(\frac{1}{k!} \right)\left[\check{\mathcal{Q}}\check{\mathcal{L}}_{0}\check{\mathcal{Q}}+ \check{\mathcal{Q}}\check{\mathcal{V}}\check{\mathcal{P}}+\check{\mathcal{Q}}\check{\mathcal{V}}\check{\mathcal{Q}} \right]^{k}ds = \check{\mathcal{P}}\check{\mathcal{L}}_{0}(t-\tau), 
\end{equation*}
posteriormente esta contribución se anula y la ecuación principal se reduce a

\begin{equation*}
    \frac{d}{dt}\check{\mathcal{P}}|\hat{\rho}(t)\rangle \rangle = \check{\mathcal{P}}\check{\mathcal{L}}_{0}\check{\mathcal{P}}|\hat{\rho}(t)\rangle \rangle + \check{\mathcal{P}}\check{\mathcal{V}}\check{\mathcal{Q}}\int_{0}^{t}\check{\mathcal{G}}(t,\tau)\check{\mathcal{Q}}\check{\mathcal{V}} \check{\mathcal{P}}|\hat{\rho}(\tau)\rangle \rangle d\tau,        
\end{equation*}
de manera similar se puede notar que el término de la exponencial $\check{\mathcal{Q}}\check{\mathcal{V}}\check{\mathcal{P}}$ no actua en la ecuación, por ende queda que 

\begin{equation*}
    \frac{d}{dt}\check{\mathcal{P}}|\hat{\rho}(t)\rangle \rangle  = \check{\mathcal{P}}\check{\mathcal{L}}_{0}\check{\mathcal{P}}|\hat{\rho}(t)\rangle \rangle  + \check{\mathcal{P}}\check{\mathcal{V}}\check{\mathcal{Q}}\int_{0}^{t}\exp \left[\left(\check{\mathcal{Q}}\check{\mathcal{V}}\check{\mathcal{Q}} + \check{\mathcal{Q}}\check{\mathcal{L}}_{0}\check{\mathcal{Q}}\right) (t-\tau) \right]\check{\mathcal{Q}}\check{\mathcal{V}} \check{\mathcal{P}}|\hat{\rho}(\tau)\rangle \rangle d\tau,        
\end{equation*}
haciendo el cambio de variable $\tau \to t-\tau$

\begin{equation*}
    \frac{d}{dt}\check{\mathcal{P}}|\hat{\rho}(t)\rangle \rangle = \check{\mathcal{P}}\check{\mathcal{L}}_{0}\check{\mathcal{P}}|\hat{\rho}(t)\rangle \rangle  + \check{\mathcal{P}}\check{\mathcal{V}}\check{\mathcal{Q}}\int_{0}^{t}\exp \left[\left(\check{\mathcal{Q}}\check{\mathcal{V}}\check{\mathcal{Q}} + \check{\mathcal{Q}}\check{\mathcal{L}}_{0}\check{\mathcal{Q}}\right) (\tau) \right]\check{\mathcal{Q}}\check{\mathcal{V}} \check{\mathcal{P}}|\hat{\rho}(t-\tau)\rangle \rangle d\tau,        
\end{equation*}
en este momento es donde se hacen aproximaciones, ya que si consideramos el acoplamiento $g$ pequeño, es decir $g \ll \gamma_{i}$. Sabiendo que los autovalores de los superoperadores $\check{\mathcal{V}}$ y $\check{\mathcal{L}}_{0}$ serán proporcionales a $g$ y $\gamma_{i}$ respectivamente,  podremos despreciar la contribución de $\check{\mathcal{V}}$ en la exponencial, así queda

\begin{equation*}
    \frac{d}{dt}\check{\mathcal{P}}|\hat{\rho}(t)\rangle \rangle = \check{\mathcal{P}}\check{\mathcal{L}}_{0}\check{\mathcal{P}}|\hat{\rho}(t)\rangle \rangle + \check{\mathcal{P}}\check{\mathcal{V}}\check{\mathcal{Q}}\int_{0}^{t}\exp \left[\left( \check{\mathcal{Q}}\check{\mathcal{L}}_{0}\check{\mathcal{Q}}\right)\tau \right]\check{\mathcal{Q}}\check{\mathcal{V}} \check{\mathcal{P}}|\hat{\rho}(t-\tau)\rangle \rangle d\tau,        
\end{equation*}
finalmente, para aplicar aproximación de Markov suponemos que $\check{\mathcal{P}}\hat{\rho}(t-\tau)$ no varia mucho en escalas de tiempo de $1/\gamma_{i}$, así poder considerar $\check{\mathcal{P}}\hat{\rho}(t-\tau) = \check{\mathcal{P}}\hat{\rho}(t)$ y el integral de $0$ a el infinito con lo que se obtiene

\begin{equation*}
    \frac{d}{dt}\check{\mathcal{P}}|\hat{\rho}(t)\rangle \rangle = \check{\mathcal{P}}\check{\mathcal{L}}_{0}\check{\mathcal{P}}|\hat{\rho}(t)\rangle \rangle + \check{\mathcal{P}}\check{\mathcal{V}}\check{\mathcal{Q}}\int_{0}^{\infty}\exp \left[\left( \check{\mathcal{Q}}\check{\mathcal{L}}_{0}\check{\mathcal{Q}}\right)\tau \right]d\tau \check{\mathcal{Q}}\check{\mathcal{V}} \check{\mathcal{P}}|\hat{\rho}(t)\rangle \rangle ,        
\end{equation*}
y finalmente se recupera la ecuación en formalismo de superoperadores  

\begin{equation*}
    \frac{d}{dt}\check{\mathcal{P}}|\hat{\rho}_{s}(t)\rangle \rangle = (\check{\mathcal{L}}_{0} - \check{\mathcal{P}}\check{\mathcal{V}}\check{\mathcal{Q}}\check{\mathcal{L}}^{-1}_{0}\check{\mathcal{Q}}\check{\mathcal{V}}\check{\mathcal{P}})\check{\mathcal{P}}|\hat{\rho}_{s}(t)\rangle \rangle. 
\end{equation*}



\label{appendix5clasic}

\subsection{ Inverso de Drazin }
El inverso de Drazin es un tipo de inverso generalizado que consiste en la única matriz que  dado la matriz $\check{\mathcal{L}}_{0}$ cumple que 

\begin{align*}
    \check{\mathcal{L}}_{0}\check{\mathcal{L}}_{0}^{+}\check{\mathcal{L}}_{0} & = \check{\mathcal{L}}_{0} \\
    \check{\mathcal{L}}_{0}^{+}\check{\mathcal{L}}_{0}\check{\mathcal{L}}_{0}^{+} & = \check{\mathcal{L}}_{0}^{+} \\
    \check{\mathcal{L}}_{0}^{+}\check{\mathcal{L}}_{0} & = \check{\mathcal{L}}_{0}\check{\mathcal{L}}_{0}^{+}.
\end{align*}

Considerando la ecuación \ref{ec5:classicalmodel} se puede probar que 

\begin{equation*}
    \check{\mathcal{L}}_{0}\check{\mathcal{L}}_{0}^{+} = - \int_{0}^{\infty}d\tau \check{\mathcal{L}}_{0}\exp[ \check{\mathcal{L}}_{0}\tau] \check{Q}= - \int_{0}^{\infty}d\tau \exp[ \check{\mathcal{L}}_{0}\tau]\check{\mathcal{L}}_{0} \check{Q} = - \int_{0}^{\infty}d\tau \exp[ \check{\mathcal{L}}_{0}\tau] \check{Q} \check{\mathcal{L}}_{0} = \check{\mathcal{L}}_{0}^{+}\check{\mathcal{L}}_{0},
\end{equation*}
de manera similar se pueden probar las otras 2 ecuaciones.\\
 Finalmente para poder ocupar el inverso de Drazin, este se puede calcular numéricamente mediante descomposición espectral usando los autovalores por la izquierda y derecha dados por 

\begin{equation*}
    \check{\mathcal{L}}_{0}|x_{j}\rangle \rangle = \lambda_{j} |x_{j}\rangle \rangle \hspace{10mm}  \langle \langle y_{j}|\check{\mathcal{L}}_{0} = \langle \langle y_{j}|\lambda_{j},
\end{equation*}
en donde $\langle \langle y_{i}|x_{j}\rangle \rangle = \delta_{ij}$. Si se toma $\lambda_{0}$ como el único autovalor cero del operador $\check{\mathcal{L}}_{0}$, se podrá escribir la descomposición espectral como

\begin{equation*}
    \check{\mathcal{L}}_{0} = \sum_{j\neq 0}\lambda_{j}|x_{j}\rangle \rangle \langle \langle y_{j}|,
\end{equation*}
y así poder calcular el inverso de Drazin usando

\begin{equation*}
    \check{\mathcal{L}}^{+}_{0} = \sum_{j\neq 0}\frac{1}{\lambda_{j}}|x_{j}\rangle \rangle \langle \langle y_{j}|.
\end{equation*}

\label{appendix5drazin}

\subsection{Expresión para los flujos de información}
Para poder calcular los flujos de información, primero se debe ser capaces de calcular $\ln \hat{\rho}_{s}$. Se puede partir de la matriz densidad descrita en la ecuación \ref{appendix5rhoconcu} la cuál ya viene diagonal, excepto por los bloques

\begin{equation*}
    \begin{bmatrix}
        \rho_{100} & \alpha \\
        \alpha^{*} & \rho_{010}
    \end{bmatrix}
    \hspace{10mm}
    \begin{bmatrix}
        \rho_{101} & \beta \\
        \beta^{*} & \rho_{011}
    \end{bmatrix}.
\end{equation*}

Los autovalores estarán dados por

\begin{equation*}
    \lambda_{0\pm} = \frac{ \rho_{100} +\rho_{010} }{2} \pm \frac{( [\rho_{100} - \rho_{010}]^{2} + 4|\alpha|^{2} )^{1/2} }{ 2 }  \hspace{10mm}      \lambda_{1\pm} = \frac{ \rho_{101} +\rho_{011} }{2} \pm \frac{( [\rho_{101} - \rho_{011}]^{2} + 4|\beta|^{2} )^{1/2} }{ 2 }, 
\end{equation*}
si se considera $\alpha = |\alpha|e^{i\phi_{0}}$ y $\beta = |\beta|e^{i\phi_{1}}$ podremos escribir los autovectores como

\begin{align*}
    v_{0+} &  = [0,0,\cos(\theta_{0}/2)e^{i\phi_{0}/2},\sen(\theta_{0}/2)e^{-i\phi_{0}/2},0,0,0,0]^{T} \\
    v_{0-} &  = [0,0,-\sen(\theta_{0}/2)e^{i\phi_{0}/2},\cos(\theta_{0}/2)e^{-i\phi_{0}/2},0,0,0,0]^{T} \\
    v_{1+} &  = [0,0,0,0,\cos(\theta_{1}/2)e^{i\phi_{1}/2},\sen(\theta_{1}/2)e^{-i\phi_{1}/2},0,0]^{T} \\
    v_{1-} &  = [0,0,0,0,-\sen(\theta_{1}/2)e^{i\phi_{1}/2},\cos(\theta_{1}/2)e^{-i\phi_{1}/2},0,0]^{T},                   
\end{align*}
con $\sen(\theta_{0}) = |\alpha|/\sqrt{\Delta^{2}_{0} + |\alpha|^{2}}$, $\sen(\theta_{1}) = |\beta|/\sqrt{\Delta^{2}_{1} + |\beta|^{2}}$ y

\begin{equation*}
    \Delta_{i} = \frac{(\rho_{10i} - \rho_{01i})}{2},
\end{equation*}
con esto se podrá obtener el logaritmo natural de la matriz mediante

\begin{equation*}
    \ln \hat{\rho}_{S} = \hat{V}\ln \hat{\rho}_{D} \hat{V}^{-1}.
\end{equation*}
    
En donde $\ln \hat{\rho}_{D}$ es la matriz diagonal 

\begin{equation*}
    \ln \hat{\rho}_{D} = 
    \begin{bmatrix}
        \ln \rho_{000} & 0 & 0 & 0 & 0 & 0 & 0 & 0 \\
        0 & \ln \lambda_{0+} & 0 & 0 & 0 & 0 & 0 & 0 \\
        0 & 0 & \ln \lambda_{0-} & 0 & 0 & 0 & 0 & 0 \\
        0 & 0 & 0 & \ln \rho_{001} & 0 & 0 & 0 & 0 \\
        0 & 0 & 0 & 0 & \ln \rho_{110} & 0 & 0 & 0 \\
        0 & 0 & 0 & 0 & 0 & \ln \lambda_{1+} & 0 & 0 \\
        0 & 0 & 0 & 0 & 0 & 0 & \ln \lambda_{1-} & 0 \\
        0 & 0 & 0 & 0 & 0 & 0 & 0 & \ln \rho_{111} 
        \end{bmatrix},
\end{equation*}
y 

\begin{equation*}
    \hat{V} = 
    \begin{bmatrix}
        1 & 0 & 0 & 0 & 0 & 0 & 0 & 0 \\
        0 & \cos(\theta_{0}/2)e^{i\phi_{0}/2} & -\sen(\theta_{0}/2)e^{i\phi_{0}/2} & 0 & 0 & 0 & 0 & 0 \\
        0 & \sen(\theta_{0}/2)e^{-i\phi_{0}/2} & \cos(\theta_{0}/2)e^{-i\phi_{0}/2} & 0 & 0 & 0 & 0 & 0 \\
        0 & 0 & 0 & 1 & 0 & 0 & 0 & 0 \\
        0 & 0 & 0 & 0 & 1 & 0 & 0 & 0 \\
        0 & 0 & 0 & 0 & 0 &  \cos(\theta_{1}/2)e^{i\phi_{1}/2} & -\sen(\theta_{1}/2)e^{i\phi_{1}/2} & 0 \\
        0 & 0 & 0 & 0 & 0 & \sen(\theta_{1}/2)e^{-i\phi_{1}/2} & \cos(\theta_{1}/2)e^{-i\phi_{1}/2} & 0 \\
        0 & 0 & 0 & 0 & 0 & 0 & 0 & 1 
        \end{bmatrix}.
\end{equation*}

Finalmente se obtiene

\begin{equation*}
    \ln \hat{\rho}_{S} = 
    \begin{bmatrix}
        \ln \rho_{000} & 0 & 0 & 0 & 0 & 0 & 0 & 0 \\
        0 & a_{0} & c_{0} & 0 & 0 & 0 & 0 & 0 \\
        0 & c^{*}_{0} & b_{0} & 0 & 0 & 0 & 0 & 0 \\
        0 & 0 & 0 & \ln \rho_{001} & 0 & 0 & 0 & 0 \\
        0 & 0 & 0 & 0 & \ln \rho_{110} & 0 & 0 & 0 \\
        0 & 0 & 0 & 0 & 0 & a_{1} & c_{1} & 0 \\
        0 & 0 & 0 & 0 & 0 & c^{*}_{1} & b_{1} & 0 \\
        0 & 0 & 0 & 0 & 0 & 0 & 0 & \ln \rho_{111} 
        \end{bmatrix},
\end{equation*}
con las cantidades 

\begin{align*}
    a_{i} & = \cos^{2}(\theta_{i}/2) \ln \lambda_{i+} + \sen^{2}(\theta_{i}/2) \ln \lambda_{i-} \\
    b_{i} & = \sen^{2}(\theta_{i}/2) \ln \lambda_{i+} + \cos^{2}(\theta_{i}/2) \ln \lambda_{i-}  \\
    c_{i} & = \frac{\sin(\theta_{i})}{2}e^{i\phi_{i}} (\ln \lambda_{i+} - \ln \lambda_{i-}),    
\end{align*}
así el logaritmo de la matriz densidad es 

\begin{align*}
    \ln \hat{\rho}_{S} & = \ln \rho_{111}|111\rangle \langle 111| + \ln \rho_{110}|110\rangle \langle 110| + \ln \rho_{001}|001\rangle \langle 001| \\ 
     & + \ln \rho_{000}|000\rangle \langle 000| + a_{0}|100\rangle \langle 100| + b_{0}|010\rangle \langle 010|  \\  
     & + a_{1}|101\rangle \langle 101| + b_{1}|011\rangle\langle 011| + c_{0}|100\rangle \langle 010| \\ 
     & + c^{*}_{0}|010\rangle \langle 100| + c_{1}|101\rangle \langle 011| + c^{*}_{1}|011\rangle \langle 101|.
\end{align*}

Por otro lado, los disipadores que actuan en los operadores de Lindblad $\mathcal{L}_{L}$ y $\mathcal{L}_{R}$ se pueden expresar 

\begin{align*}
    \mathcal{D}[\hat{d}^{\dagger}_{L}(\textbf{1}-\hat{n}_{D})(\textbf{1}-\hat{n}_{R})\hat{\rho}_{S}] & = \rho_{000}|100\rangle \langle 100| - \rho_{000}|000\rangle \langle 000| \\ 
    \mathcal{D}[\hat{d}^{\dagger}_{R}(\textbf{1}-\hat{n}_{D})(\textbf{1}-\hat{n}_{L})\hat{\rho}_{S}] & = \rho_{000}|010\rangle \langle 010| - \rho_{000}|000\rangle \langle 000| \\  
    \mathcal{D}[\hat{d}_{L}(\textbf{1}-\hat{n}_{D})(\textbf{1}-\hat{n}_{R})\hat{\rho}_{S}] & = \rho_{100}|000\rangle \langle 000| - \rho_{100}|100\rangle \langle 100|\\ 
    \mathcal{D}[\hat{d}_{R}(\textbf{1}-\hat{n}_{D})(\textbf{1}-\hat{n}_{L})\hat{\rho}_{S}] & = \rho_{010}|000\rangle \langle 000| - \rho_{010}|010\rangle \langle 010| \\ 
    \mathcal{D}[\hat{d}^{\dagger}_{L}\hat{n}_{D} \hat{n}_{R}\hat{\rho}_{S}] & = \rho_{011}|111\rangle \langle 111| - \rho_{011}|011\rangle \langle 011| \\  
    \mathcal{D}[\hat{d}^{\dagger}_{R} \hat{n}_{D}\hat{n}_{L}\hat{\rho}_{S}] & = \rho_{101}|111\rangle \langle 111| - \rho_{101}|101\rangle \langle 101| \\ 
     \mathcal{D}[\hat{d}_{L}\hat{n}_{D}\hat{n}_{R}\hat{\rho}_{S}] & = \rho_{111}|011\rangle \langle 011| - \rho_{111}|111\rangle \langle 111|  \\    
      \mathcal{D}[\hat{d}_{R}\hat{n}_{D}\hat{n}_{L}\hat{\rho}_{S}] & = \rho_{111}|101\rangle \langle 101| - \rho_{111}|111\rangle \langle 111|  \\ 
    \mathcal{D}[\hat{d}^{\dagger}_{L}(\textbf{1}-\hat{n}_{D})\hat{n}_{R}\hat{\rho}_{S}] & = \rho_{010}|110\rangle \langle 110| - \rho_{010}|010\rangle \langle 010| - \frac{1}{2}( \alpha|100\rangle \langle 010| + \alpha^{*}|010\rangle \langle 100|) \\ 
    \mathcal{D}[\hat{d}^{\dagger}_{R}(\textbf{1}-\hat{n}_{D})\hat{n}_{L}\hat{\rho}_{S}] & = \rho_{100}|110\rangle \langle 110| - \rho_{100}|100\rangle \langle 100| - \frac{1}{2}( \alpha |100\rangle \langle 010| + \alpha^{*}|010\rangle \langle 100| )  \\  
    \mathcal{D}[\hat{d}_{L}(\textbf{1}-\hat{n}_{D})\hat{n}_{R}\hat{\rho}_{S}] & = \rho_{110}|010\rangle \langle 010| - \rho_{110}|110\rangle \langle 110| \\ 
    \mathcal{D}[\hat{d}_{R}(\textbf{1}-\hat{n}_{D})\hat{n}_{L}\hat{\rho}_{S}] & = \rho_{110}|100\rangle \langle 100| - \rho_{110}|110\rangle \langle 110| \\  
    \mathcal{D}[\hat{d}^{\dagger}_{L}\hat{n}_{D}(\textbf{1}-\hat{n}_{R})\hat{\rho}_{S}] & = \rho_{001}|101\rangle \langle 101| - \rho_{001}|001\rangle \langle 001| \\ 
    \mathcal{D}[\hat{d}^{\dagger}_{R}\hat{n}_{D}(\textbf{1}-\hat{n}_{L})\hat{\rho}_{S}] & = \rho_{001}|011\rangle \langle 011| - \rho_{001}|001\rangle \langle 001| \\  
    \mathcal{D}[\hat{d}_{L}\hat{n}_{D}(\textbf{1}-\hat{n}_{R})\hat{\rho}_{S}] & = \rho_{101}|001\rangle \langle 001| - \rho_{101}|101\rangle \langle 101| - \frac{1}{2}(\beta |101\rangle \langle 011| + \beta^{*}|011\rangle \langle 101| ) \\ 
    \mathcal{D}[\hat{d}_{R}\hat{n}_{D}(\textbf{1}-\hat{n}_{L})\hat{\rho}_{S}] & = \rho_{011}|001\rangle \langle 001| - \rho_{011}|011\rangle \langle 011| - \frac{1}{2}(\beta^{*} |011\rangle \langle 101| + \beta|101\rangle \langle 011|  ),   
\end{align*}
con estas cantidades se podrán obtener los flujos de información en el estado estacionario $\dot{I}_{i} = \text{Tr}[(\mathcal{L}_{i}\hat{\rho}_{S} \ln \hat{\rho_{S}}) ]$ lo que permite obtener

\begin{align*}
    \dot{I}_{L} &  =  \gamma_{L}(\epsilon)(f_{L}(\epsilon)[a_{0}\rho_{000} - \rho_{000}\ln \rho_{000}] + (1-f_{L}(\epsilon))[\rho_{100}\ln \rho_{000} - a_{0}\rho_{100} ] )  \\ 
      & + \gamma_{L}(\epsilon + U)(f_{L}(\epsilon + U)[a_{1}\rho_{001} - \rho_{001}\ln \rho_{001}] + (1-f_{L}(\epsilon + U))[\rho_{101}\ln \rho_{001} -a_{1}\rho_{101} - Re(b^{*}_{1}\beta ) ]   ) \\  
      & + \gamma_{L}(\epsilon + U_{f})( f_{L}(\epsilon + U_{f})[\rho_{010}\ln \rho_{110} -b_{0}\rho_{010} - Re(\alpha c^{*}_{0}) ]  + (1-f_{L}(\epsilon + U_{f}))[b_{0}\rho_{110} - \rho_{110}\ln \rho_{110} ] ) \\  
      & + \gamma_{L}(\epsilon + U + U_{f})( f_{L}(\epsilon + U + U_{f})[\rho_{011}\ln \rho_{111}- b_{1}\rho_{011}] + (1-f_{L}(\epsilon+U+U_{f}) )[b_{1}\rho_{111} - \rho_{111}\ln \rho_{111}]  ) 
\end{align*}

\begin{align*}
    \dot{I}_{R} &  =  \gamma_{R}(\epsilon)(f_{L}(\epsilon)[b_{0}\rho_{000} - \rho_{000}\ln \rho_{000}] + (1-f_{R}(\epsilon))[\rho_{010}\ln \rho_{000} - b_{0}\rho_{010} ] )  \\ 
      & + \gamma_{R}(\epsilon + U)(f_{R}(\epsilon + U)[b_{1} \rho_{001} - \rho_{001}\ln \rho_{001}] + (1-f_{R}(\epsilon + U))[\rho_{011}\ln \rho_{001} -a_{1}\rho_{011} - Re(b^{*}_{1}\beta ) ]   ) \\  
      & + \gamma_{R}(\epsilon + U_{f})( f_{R}(\epsilon + U_{f})[\rho_{100}\ln \rho_{110} -a_{0}\rho_{100} - Re(\alpha c^{*}_{0}) ]  + (1-f_{R}(\epsilon + U_{f}))[a_{0}\rho_{110} - \rho_{110}\ln \rho_{110} ] ) \\  
      & + \gamma_{R}(\epsilon + U + U_{f})( f_{R}(\epsilon + U + U_{f})[\rho_{101}\ln \rho_{111}- a_{1}\rho_{101}] + (1-f_{R}(\epsilon+U+U_{f}) )[a_{1}\rho_{111} - \rho_{111}\ln \rho_{111}]  ), 
\end{align*}
si bien el cálculo de los flujos de información se puede hacer númerico, escribir estas expresiones analíticas nos permite determinar que existe contribución de las coherencias en la información, tanto en los parámetros $a_{i},b_{i}$ como en los términos $Re(\alpha c^{*}_{0}),Re(b^{*}_{1} \beta)$ . Es importante destacar que en el  límite en que las coherencias se anulan se recupera el resultado para un sistema clásico

\begin{align*}
 \lim_{\alpha,\beta \to 0}  & \dot{I}_{L}  = \dot{I}_{Lclassic}   =  \gamma_{L}(\epsilon)(f_{L}(\epsilon)[\rho_{000}\ln \rho_{100} - \rho_{000}\ln \rho_{000}] + (1-f_{L}(\epsilon))[\rho_{100}\ln \rho_{000} - \rho_{100}\ln \rho_{100} ] )  \\ 
      & + \gamma_{L}(\epsilon + U)(f_{L}(\epsilon + U)[\rho_{001}\ln \rho_{101} - \rho_{001}\ln \rho_{001}] + (1-f_{L}(\epsilon + U))[\rho_{101}\ln \rho_{001} -\rho_{101}\ln \rho_{101} ]   ) \\  
      & + \gamma_{L}(\epsilon + U_{f})( f_{L}(\epsilon + U_{f})[\rho_{010}\ln \rho_{110} -\rho_{010}\ln \rho_{010} ]  + (1-f_{L}(\epsilon + U_{f}))[\rho_{110}\ln \rho_{010} - \rho_{110}\ln \rho_{110} ] ) \\  
      & + \gamma_{L}(\epsilon + U + U_{f})( f_{L}(\epsilon + U + U_{f})[\rho_{011}\ln \rho_{111}- \rho_{011}\ln \rho_{011}]) \\   
      & + \gamma_{L}(\epsilon + U + U_{f})( (1-f_{L}(\epsilon+U+U_{f}) )[\rho_{111}\ln \rho_{011} - \rho_{111}\ln \rho_{111}])   
\end{align*}

\begin{align*}
 \lim_{\alpha,\beta \to 0}  & \dot{I}_{R}  = \dot{I}_{Rclassic}   =  \gamma_{R}(\epsilon)(f_{R}(\epsilon)[\rho_{000}\ln \rho_{010} - \rho_{000}\ln \rho_{000}] + (1-f_{R}(\epsilon))[\rho_{010}\ln \rho_{000} - \rho_{010}\ln \rho_{010} ] )  \\ 
      & + \gamma_{R}(\epsilon + U)(f_{R}(\epsilon + U)[\rho_{001}\ln \rho_{011} - \rho_{001}\ln \rho_{001}] + (1-f_{R}(\epsilon + U))[\rho_{011}\ln \rho_{001} -\rho_{011}\ln \rho_{011} ]   ) \\  
      & + \gamma_{R}(\epsilon + U_{f})( f_{R}(\epsilon + U_{f})[\rho_{100}\ln \rho_{110} -\rho_{100}\ln \rho_{100} ]  + (1-f_{R}(\epsilon + U_{f}))[\rho_{110}\ln \rho_{100} - \rho_{110}\ln \rho_{110} ] ) \\  
      & + \gamma_{R}(\epsilon + U + U_{f})( f_{R}(\epsilon + U + U_{f})[\rho_{101}\ln \rho_{111}- \rho_{101}\ln \rho_{101}]) \\   
      & + \gamma_{R}(\epsilon + U + U_{f})( (1-f_{R}(\epsilon+U+U_{f}) )[\rho_{111}\ln \rho_{101} - \rho_{111}\ln \rho_{111}]).   
\end{align*}

\label{apendix5infoflow}

\end{appendixs} % Ejemplo, se puede borrar

% FIN DEL DOCUMENTO
\end{document}