% Template:     Tesis LaTeX
% Documento:    Archivo de ejemplo
% Versión:      3.4.0 (23/08/2024)
% Codificación: UTF-8
%
% Autor: Pablo Pizarro R.
%        pablo@ppizarror.com
%
% Manual template: [https://latex.ppizarror.com/tesis]
% Licencia MIT:    [https://opensource.org/licenses/MIT]

% ------------------------------------------------------------------------------
% NUEVO CAPÍTULO
% ------------------------------------------------------------------------------
% A diferencia de Template-Informe, Template-Tesis requiere el uso de capítulos; las secciones, subsecciones, etc son parte de un capítulo. Se recomienda el uso de un capítulo en un archivo distinto
\chapteranum{Introducción}

Uno de los principales objetivos de la termodinámica contemporánea es esclarecer el carácter físico de la información. Este desafío fue anticipado por el célebre experimento mental propuesto por James Clerk Maxwell en 1871, el cual sugiere una aparente violación de la segunda ley de la termodinámica al introducir un "demonio" que manipula la dinámica de un gas ideal mediante información\cite{Maxwell_1871}. Esta paradoja motivó una profunda revisión conceptual, abordada en trabajos fundamentales de Szilard\cite{szilard1964decrease}, Brillouin\cite{brillouin1951maxwell}, Landauer\cite{Landauer_1961} y Bennett\cite{bennett1982thermodynamics}, quienes formalizaron el vínculo entre información y entropía, sentando las bases de la termodinámica de la información.

Los avances experimentales en nanotecnología han permitido explorar sistemas de pocas partículas donde las fluctuaciones térmicas no pueden ser despreciadas\cite{douarche2005experimental,wang2005experimental}. En este contexto, las formulaciones tradicionales de la termodinámica pierden validez, lo que exige un marco teórico más general. La termodinámica estocástica cumple este rol, al extender las leyes termodinámicas a regímenes fuera del equilibrio en sistemas clásicos de pequeña escala\cite{van2013stochastic,jarzynski1997nonequilibrium}.

Además del estudio de energía y entropía, resulta fundamental entender el rol de la información en estos sistemas. Se deben distinguir dos enfoques según el tipo de dinámica involucrada. El primero considera sistemas no autónomos, cuyo Hamiltoniano depende explícitamente del tiempo y es controlado por un agente externo. En estos casos, el demonio de Maxwell puede modelarse mediante procesos de medición y retroalimentación, como ha sido mostrado en diversos trabajos\cite{cao2009thermodynamics,sagawa2010generalized}. El segundo enfoque aborda sistemas autónomos, de especial relevancia en contextos biológicos\cite{ehrich2023energy}, donde el sistema evoluciona sin intervención externa bajo gradientes constantes de energía (como diferencias de potencial o temperatura). Aquí, la estructura bipartita del sistema permite identificar un subsistema que actúa como "demonio", acoplado al sistema de interés y modulando su dinámica dependiendo del estado en que se encuentre.

En ausencia de un agente externo, cuantificar el flujo de información entre el demonio y el sistema presenta una dificultad adicional, ya que no se dispone del resultado explícito de la medición ni del mecanismo de retroalimentación. Este desafío fue abordado por Horowitz y Esposito mediante el formalismo de la termodinámica estocástica\cite{horowitz2014thermodynamics}, quienes propusieron una definición operacional del flujo de información en sistemas clásicos autónomos.

Posteriormente, Ptaszynski y Esposito extendieron este marco al contexto cuántico, cuantificando flujos de información en sistemas abiertos gobernados por dinámicas markovianas\cite{ptaszynski2019thermodynamics}. Para ello, emplearon una ecuación maestra de tipo Lindblad en su forma estándar (GKLS)\cite{gorini1976completely,lindblad1976generators,breuer2002theory}, compatible con una descripción termodinámica consistente cuando se cumple la condición de balance detallado local. Este enfoque permite incorporar coherencias cuánticas —elementos no diagonales de la matriz densidad—, las cuales pueden servir como recursos termodinámicos\cite{ptaszynski2023fermionic,streltsov2017colloquium}.

Sin embargo, una dificultad clave en la descripción cuántica es la elección de la base. Si se adopta una base global, como exige la aproximación secular, no siempre existe la posibilidad de distinguir entre flujos locales de información, como se realiza en su contraparte clásica\cite{horowitz2014thermodynamics}. Por otro lado, una formulación en la base local puede revelar la estructura bipartita del sistema, pero ha sido criticada por posibles inconsistencias termodinámicas\cite{levy2014local,novotny2002investigation}.

Potts et al.\ propusieron una ecuación maestra termodinámicamente consistente que permite trabajar en la base local e incluir términos no seculares\cite{potts2021thermodynamically}. Su enfoque reconoce que una descripción markoviana implica una resolución energética finita, y por tanto introduce un Hamiltoniano rescalado en la definición de las cantidades termodinámicas.

Trabajos previos han empleado el formalismo desarrollado por Potts para analizar los efectos cuánticos en modelos compuestos por dos puntos cuánticos, cada uno acoplado a un reservorio distinto, y cómo estos efectos influyen en el flujo de partículas~\cite{prech2023entanglement}. Por otro lado, el trabajo de Esposito y Horowitz investigó los flujos de información en un sistema clásico constituido por dos puntos cuánticos, donde uno de ellos está acoplado a dos reservorios y el otro a un único baño térmico. En dicho estudio, se demostró que el sistema actúa como un Demonio de Maxwell autónomo~\cite{horowitz2014thermodynamics}.

En el presente trabajo se emplea el formalismo de Potts para describir la evolución temporal de un sistema cuántico formado por tres puntos cuánticos, cada uno acoplado a un reservorio distinto. Uno de estos puntos cuánticos desempeña el rol de Demonio de Maxwell, mientras que los otros dos constituyen el sistema físico, denotado como $LR$. Esta configuración permite estudiar los flujos de información y las demás magnitudes termodinámicas relevantes, en un régimen donde se manifiestan efectos cuánticos tales como la coherencia y el entrelazamiento entre los distintos sitios del sistema.

La tesis está estructurada de la siguiente manera: en el Capítulo 1 se introducen los conceptos fundamentales necesarios para describir la evolución de un sistema cuántico, ya sea como sistema cerrado —mediante la ecuación de Liouville— o como sistema abierto —a través de la ecuación de Lindblad en la aproximación secular. El Capítulo 2 está dedicado al desarrollo del Formalismo de Estadística de Conteo, el cual será empleado para derivar la ecuación de Lindblad termodinámicamente consistente propuesta por Potts. En el Capítulo 3 se introduce el concepto de flujo de información, discutiendo su influencia en la formulación de la segunda ley de la termodinámica tanto en contextos clásicos como cuánticos. Finalmente, en el Capítulo 4 se presenta la dinámica de un sistema compuesto por tres puntos cuánticos, y se analizan los principales resultados obtenidos para dos configuraciones distintas del modelo.




% SUB-SECCIÓN
% Las sub-secciones se inician con \subsection, si se quiere una sub-sección
% sin número se pueden usar las funciones \subsectionanum (nuevo subtítulo sin
% numeración) o la función \subsectionanumnoi para crear el mismo subtítulo sin
% numerar y sin aparecer en el índice


% ------------------------------------------------------------------------------
% NUEVO CAPÍTULO
% ------------------------------------------------------------------------------

\chapter{Sistemas cuánticos abiertos}

En este capítulo se describen los conceptos básicos para poder describir la evolución de un sistema cuántico, en la sección \ref{sec:closedQM} se describe la evolución unitaria para un sistema cerrado.  Por otro lado en la sección \ref{sec:lindblad} se describe una de las ecuaciones utilizadas para describir el comportamiento de un sistema cuántico acoplado a un reservorio con infinitos grados de libertad.


\section{Sistemas cuánticos cerrados}
Un sistema cuántico cerrado puede ser descrito por la matriz densidad $\hat{\rho}$, la evolución de la matriz densidad dependerá del Hamiltoniano del sistema $\hat{H}(t)$, la cuál en el cuadro de Schrodinger consiste en la ecuación de Liouville-Von Neumman($\hbar = 1$)\cite{breuer2002theory}:

\begin{equation*}
    \frac{d}{dt}\hat{\rho}(t) = -i[\hat{H}(t),\hat{\rho}(t)].
\end{equation*}

Y la solución descrita por la evolución unitaria 

\begin{equation*}
    \hat{U}(t,t_{0}) = \hat{T}_{\leftarrow} \exp \left[ -i \int_{t_{0}}^{t}ds \hat{H}(s) \right] \implies \hat{\rho}(t) = \hat{U}(t,t_{0})\hat{\rho}(t_{0})\hat{U}^{\dagger}(t,t_{0}),
\end{equation*}
en donde $\hat{T}_{\leftarrow}$ consiste en el operador ordenación temporal cronológico que ordena los productos de operadores dependientes del tiempo, de tal manera que el tiempo en el que son evaluados los operadores va creciendo de derecha a izquierda.

\subsection{Cuadro de interacción}
Supongamos que el Hamiltoniano del sistema se puede separar en dos partes

\begin{equation*}
    \hat{H}(t) = \hat{H}_{0} + \hat{H}_{I}(t),
\end{equation*}
en teoría esto se puede hacer de varias formas, pero por lo general, si tenemos el caso de dos subsistemas, $\hat{H}_{0}$ contiene los Hamiltonianos de cada uno de ellos cuando no hay interacción, mientras que $\hat{H}_{I}$ representa la interacción entre ellos. Si introducimos los operadores unitarios

\begin{equation*}
    \hat{U}_{0}(t,t_{0}) \equiv \exp[-i\hat{H}_{0}(t-t_{0}) ]  \hspace{15mm} \hat{U}_{I}(t,t_0) \equiv \hat{U}^{\dagger}_{0}(t,t_{0})\hat{U}(t,t_{0}),
\end{equation*}
se puede describir la matriz densidad en el cuadro de interacción

\begin{equation*}
    \hat{\rho}_{I}(t) \equiv \hat{U}_{I}(t,t_{0})\hat{\rho}(t_{0})\hat{U}^{\dagger}_{I}(t,t_{0}).
\end{equation*}

Y la evolución en este cuadro de interacción es 

\begin{equation}
    \frac{d}{dt}\hat{\rho}_{I}(t) = -i[\tilde{H}_{I}(t), \hat{\rho}_{I}(t)],
    \label{sec11:interactionp}
\end{equation}
con 

\begin{equation*}
    \tilde{H}_{I}(t) = \hat{U}^{\dagger}_{0}(t,t_{0})\hat{H}_{I}(t)\hat{U}_{0}(t,t_{0}).
\end{equation*}

\label{sec:closedQM}



\section{Ecuación de Lindblad}
\label{SEClindblad}
\subsection{Dinámica de un sistema abierto}

Un sistema abierto consiste en un sistema cuántico $S$ denominado el sistema reducido, el cuál está acoplado a un sistema $B$ denominado el ambiente. Estos representan  subsistemas del sistema total $S+B$. Sea $\mathcal{H}_{S}$ el espacio de Hilbert del sistema y $\mathcal{H}_{B}$ el espacio de Hilbert del ambiente, el espacio de Hilbert del sistema total $S+B$ es  $\mathcal{H} = \mathcal{H}_{S} \otimes \mathcal{H}_{B}$, el Hamiltoniano total se constituye por el Hamiltoniano del sistema $\hat{H}_{S}$, el Hamiltoniano $\hat{H}_{B}$ y la interacción $\hat{H}_{I}(t)$

\begin{equation}
    \hat{H}(t) = \hat{H}_{S} \otimes \mathbf{I}_{B} + \mathbf{I}_{S} \otimes \hat{H}_{B} + \hat{H}_{I}(t).
    \label{sec2:sistemabierto}
\end{equation}


Para describir el sistema con la ecuación \ref{sec2:sistemabierto} es necesario separarlo en dos subsistemas, $S$ y $B$, permitiendo así que el espacio de Hilbert total sea descrito por $\mathcal{H} = \mathcal{H}_{S}\otimes \mathcal{H}_{B}$. Sin embargo, en el caso de partículas idénticas como los fermiones el identificar estos dos subsistemas no es obvio. Aún así, en el formalismo de segunda cuantización el sistema se puede describir mediante una base del espacio de Fock, que sí permite encontrar subsistemas en los cuáles el espacio de Hilbert total es un producto tensorial antisimétrico entre los dos subsistemas\cite{friis2013fermionic,vidal2021quantum}.

Los observables que se estudian en el sistema $S$ son de la forma $A\otimes \mathbf{I}_{B}$ y su valor de expectación puede ser calculado mediante la ecuación

\begin{equation*}
    \langle A \rangle = \text{Tr}_{S}\{A \hat{\rho}_{S} \},
\end{equation*}
donde 

\begin{equation*}
    \hat{\rho}_{S} = \text{Tr}_{B}\{ \hat{\rho} \},
\end{equation*}
es la matriz densidad del sistema reducido. La expresión $\text{Tr}_{S}$ hace referencia a la traza con respecto al espacio de Hilbert del sistema, mientras que $\text{Tr}_{B}$  
es la traza con respecto al espacio de Hilbert del ambiente. El objeto de estudio principalmente es $\hat{\rho}_{S}$ y la ecuación de Lindblad será utilizada para describir su evolución.

\subsection{Ecuación maestra}
Partiendo de un sistema $S$ acoplado débilmente a un reservorio $B$, donde el Hamiltoniano total es descrito por 

\begin{equation*}
    \hat{H} = \hat{H}_{S} + \hat{H}_{B} + \hat{H}_{I},
\end{equation*}
la evolución en el cuadro de interacción del sistema total será

\begin{equation}
    \frac{d}{dt}\hat{\rho}_{I}(t) = -i[\hat{H}_{I}(t), \hat{\rho}_{I}(t)],
    \label{eqsec2:linbladpic}
\end{equation}
o en su forma integral

\begin{equation*}
    \hat{\rho}_{I}(t) = \hat{\rho}(0) - i  \int_{0}^{t} ds[\hat{H}_{I}(s), \hat{\rho}_{I}(s)].
\end{equation*}

Reinsertando la forma integral en la ecuación \ref{eqsec2:linbladpic} y trazando los grados de libertad del reservorio, nos queda

\begin{equation}
    \frac{d}{dt}\hat{\rho}_{IS}(t) = -i  \text{Tr}_{B}\{[\hat{H}_{I}(t),\hat{\rho}(0)] \}  -  \int_{0}^{t}ds \text{Tr}_{B}\{[\hat{H}_{I}(t), [\hat{H}_{I}(s),\hat{\rho}_{I}(s)]]\}.
\end{equation}

Ahora se asumirán dos cosas, primero $\text{Tr}_{B}\{[\hat{H}_{I}(t),\hat{\rho}(0)] \} = 0$ sin pérdida de generalidad\cite{breuer2002theory}. Segundo, tendremos que realizar la aproximación de Born, que consiste en que si se tiene acoplamiento débil, la influencia del sistema en el reservorio será pequeña, por ende se puede asumir que el estado del reservorio $\hat{\rho}_{B}$ permanece constante, y sólo el que se ve alterado es la evolución del sistema, es decir,

\begin{equation*}
    \hat{\rho}(t) = \hat{\rho}_{S}(t)\otimes \hat{\rho}_{B},
\end{equation*}
con lo cuál la ecuación \ref{eqsec2:linbladpic} se vuelve

\begin{equation}
    \frac{d}{dt}\hat{\rho}_{IS}(t) = -  \int_{0}^{t}ds \text{Tr}_{B}\{[\hat{H}_{I}(t), [\hat{H}_{I}(s),\hat{\rho}_{IS}(s) \otimes \hat{\rho}_{B}]]\}.
\end{equation} 

Esta ecuación es lo que se denomina no Markoviana, ya que requiere conocer todo el pasado de $\hat{\rho}_{IS}(s)$. Para simplificar el problema, se aplicará la aproximación de Markov, con el objetivo de que la evolución de $\hat{\rho}_{IS}(t)$ dependa sólo del estado en que se encuentra en el instante de tiempo, es decir,

\begin{equation*}
    \frac{d}{dt}\hat{\rho}_{IS}(t) = -  \int_{0}^{t}ds \text{Tr}_{B}\{[\hat{H}_{I}(t), [\hat{H}_{I}(s),\hat{\rho}_{IS}(t) \otimes \hat{\rho}_{B}]]\},
\end{equation*} 
esta ecuación es denominada la ecuación de Redfield. Se puede hacer el cambio de variable $s= x = t-s$, así el integral queda

\begin{equation}
    \frac{d}{dt}\hat{\rho}_{IS}(t) = -  \int_{0}^{t}ds \text{Tr}_{B}\{[\hat{H}_{I}(t), [\hat{H}_{I}(t-s),\hat{\rho}_{IS}(t) \otimes \hat{\rho}_{B}]]\}.
    \label{eq3sec2:markov}
\end{equation} 

Por último, se podrá hacer otra aproximación, ya que en esta expresión aparecen las funciones correlación del baño, estas funciones decaen en el tiempo con una escala característica dada por el denominado tiempo de correlación $\tau_{B}$, a partir del cual dichas funciones se vuelven despreciables. Mientras que el sistema tendrá su tiempo de relajación $\tau_{R}$ que consiste en el tiempo que demora en llegar a su estado estacionario. La aproximación markoviana requiere que $\tau_{R}\gg \tau_{B}$, pues él sistema no debe ser capaz de percibir la dinámica del baño, esto permite tomar el límite del integral en \ref{eq3sec2:markov} a infinito, ya que para tiempos muy largos se anula el integrando debido a que las funciones correlación se vuelven practicamente nulas. Y finalmente quedará la evolución

\begin{equation}
    \frac{d}{dt}\hat{\rho}_{IS}(t) = -  \int_{0}^{\infty}ds \text{Tr}_{B}\{[\hat{H}_{I}(t), [\hat{H}_{I}(t-s),\hat{\rho}_{IS}(t) \otimes \hat{\rho}_{B}]]\}.
    \label{eq3sec2:markov1}
\end{equation} 

La interacción en el cuadro de Schrodinger $\hat{H}_{I}$ será de la forma general

\begin{equation*}
    \hat{H}_{I} = \sum_{\alpha}A_{\alpha} \otimes B_{\alpha},
\end{equation*}
donde los operadores $A_{\alpha} = A^{\dagger}_{\alpha}$ actuan en el espacio de Hilbert del sistema, mientras que $B_{\alpha}=B^{\dagger}_{\alpha}$ actua en el espacio de Hilbert del reservorio. Es conveniente escribir la interacción en función de los autoestados de $\hat{H}_{S}$. Para ello, si tenemos los autovalores $\epsilon$ y sus respectivos operadores de proyección $\Pi(\epsilon) = |\epsilon\rangle \langle \epsilon|$, se podrán definir los operadores

\begin{equation}
    A_{\alpha}(\omega) \equiv \sum_{\epsilon' - \epsilon = \omega}\Pi(\epsilon)A_{\alpha}\Pi(\epsilon'), 
    \label{sec1:globalA}
\end{equation}
estos operadores se denominan operadores globales, ya que la suma esta descrita por los autoestados del sistema que cumplen $\epsilon'-\epsilon=\omega$. 

De la relación de completitud $\sum_{\epsilon}\Pi(\epsilon) = \mathbf{I}$ podremos notar que

\begin{align*}
   \sum_{\omega}A_{\alpha}(\omega) & =  \sum_{\omega,\epsilon'-\epsilon  = \omega} \Pi(\epsilon) A_{\alpha} \Pi(\epsilon') \\
   & = \sum_{\omega,\epsilon} \Pi(\epsilon) A_{\alpha} \Pi(\epsilon + \omega) \\
   & = A_{\alpha} = \sum_{\omega}A^{\dagger}_{\alpha}(\omega),
\end{align*}

con esto el Hamiltoniano de interacción en el cuadro de Schrodinger

\begin{equation*}
    \hat{H}_{I} = \sum_{\alpha,\omega}A_{\alpha}(\omega) \otimes B_{\alpha} = \sum_{\alpha,\omega}A^{\dagger}_{\alpha}(\omega) \otimes B^{\dagger}_{\alpha}.
\end{equation*}

Debido a \ref{sec1:globalA}, el operador $A_{\alpha}(\omega)$ cumple las relaciones

\begin{align*}
    [\hat{H}_{S},A_{\alpha}(\omega)] & = - \omega A_{\alpha}(\omega) \\
    [\hat{H}_{S},A^{\dagger}_{\alpha}(\omega)] & = \omega A^{\dagger}_{\alpha}(\omega),
\end{align*}
posteriormente se deberá pasar al cuadro de interacción y calcular $U^{\dagger}_{s}(t)\hat{H}_{I}U_{s}(t)$. Para ello, se utilizará la relación de Baker Campbell

\begin{equation}
    e^{A}Be^{-A} = B  + [A,B] + \frac{1}{2}[A,[A,B]] +..,
    \label{sec2lind:baker}
\end{equation}
y se puede derivar las relaciones 

\begin{align*}
    e^{i\hat{H}_{S}t}A_{\alpha}(\omega) e^{-i\hat{H}_{S}t} & = e^{-i\omega t} A_{\alpha}(\omega) \\
    e^{i\hat{H}_{S}t}A^{\dagger}_{\alpha}(\omega) e^{-i\hat{H}_{S}t} & = e^{i\omega t} A^{\dagger}_{\alpha}(\omega).
\end{align*}

Con estas relaciones, $\hat{H}_{I}(t)$ en el cuadro de interacción

\begin{equation}
    \hat{H}_{I}(t) = \sum_{\alpha,\omega}e^{-i\omega t}A_{\alpha}(\omega) \otimes B_{\alpha}(t) = \sum_{\alpha,\omega}e^{i\omega t}A^{\dagger}_{\alpha}(\omega) \otimes B^{\dagger}_{\alpha}(t),
    \label{seclindbladinteraction1}
\end{equation}
donde $B_{\alpha}(t) = e^{i\hat{H}_{B}t}B_{\alpha}e^{-i\hat{H}_{B}t}$. La ecuación \ref{seclindbladinteraction1} se puede introducir en la ecuación \ref{eq3sec2:markov1} y se obtiene 

\begin{align*}
    \frac{d}{dt}\hat{\rho}_{IS}(t) = \int_{0}^{\infty} ds \text{Tr}_{B}\left[  \hat{H}_{I}(t-s)\hat{\rho}_{IS}(t)\hat{\rho}_{B}\hat{H}_{I}(t) - \hat{\rho}_{IS}(t)\hat{\rho}_{B}\hat{H}_{I}(t-s)\hat{H}_{I}(t)    \right.\\
    \left. + \hat{H}_{I}(t)\hat{\rho}_{IS}(t)\hat{\rho}_{B}\hat{H}_{I}(t-s) -  \hat{H}_{I}(t)\hat{H}_{I}(t-s)\hat{\rho}_{IS}(t)\hat{\rho}_{B}  \right],
\end{align*}
así, reemplazando la forma explícita de la interacción  

\begin{align*}
    \frac{d}{dt}\hat{\rho}_{IS}(t) & = \sum_{\omega,\omega'}\sum_{\alpha,\beta} \int_{0}^{\infty} ds  \text{Tr}_{B}[e^{i\omega s}B_{\beta}(t-s)\hat{\rho}_{B}B^{\dagger}_{\alpha}(t)]e^{i(\omega'- \omega)t}\left(A_{\beta}(\omega)\hat{\rho}_{IS}(t)A^{\dagger}_{\alpha}(\omega') - A^{\dagger}_{\alpha}(\omega')A_{\beta}(\omega) \hat{\rho}_{IS}(t) \right) \\
     + & \sum_{\omega,\omega'}\sum_{\alpha,\beta} \int_{0}^{\infty} ds  \text{Tr}_{B}[e^{-i\omega s}B_{\alpha}(t)\hat{\rho}_{B}B^{\dagger}_{\beta}(t-s)]e^{-i(\omega'- \omega)t}\left(A_{\alpha}(\omega')\hat{\rho}_{IS}(t)A^{\dagger}_{\beta}(\omega) - \hat{\rho}_{IS}(t)A^{\dagger}_{\beta}(\omega)A_{\alpha}(\omega') \right).
\end{align*}

Dado que las funciones correlación son homogéneas en el tiempo, es posible definir funciones correlación espectral del baño, independientes del tiempo, dadas por

\begin{equation*}
    \Gamma_{\alpha\beta}(\omega) = \int_{0}^{\infty}ds e^{i\omega s}\text{Tr}_{B}[B^{\dagger}_{\alpha}(t)B_{\beta}(t-s)\hat{\rho}_{B}],
\end{equation*}
y aplicando estas funciones en la ecuación maestra, queda 

\begin{align*}
    \frac{d}{dt}\hat{\rho}_{IS}(t) & = \sum_{\omega,\omega'}\sum_{\alpha,\beta} e^{i(\omega'- \omega)t}\Gamma_{\alpha \beta}(\omega)\left(A_{\beta}(\omega)\hat{\rho}_{IS}(t)A^{\dagger}_{\alpha}(\omega') - A^{\dagger}_{\alpha}(\omega')A_{\beta}(\omega) \hat{\rho}_{IS}(t) \right) \\
    & + \sum_{\omega,\omega'}\sum_{\alpha,\beta} e^{-i(\omega'- \omega)t}\Gamma^{*}_{\beta \alpha}(\omega) \left(A_{\alpha}(\omega')\hat{\rho}_{IS}(t)A^{\dagger}_{\beta}(\omega) - \hat{\rho}_{IS}(t)A^{\dagger}_{\beta}(\omega)A_{\alpha}(\omega') \right).
\end{align*}

Si se considera el tiempo característico de evolución del sistema $S$ como $\tau_S = |\omega - \omega'|^{-1}$, y además se cumple que el tiempo de relajación $\tau_R \gg \tau_S$, entonces los términos no seculares (aquellos con $\omega \neq \omega'$) pueden despreciarse. Esto se debe a que los factores oscilatorios $e^{i(\omega - \omega')t}$ varían rápidamente en la escala de tiempo $\tau_R$, promediándose a cero. Está aproximación, se denomina secular \cite{breuer2002theory}, y tiene repercusiones en la dinámica del sistema, ya que al suprimir los términos no seculares, se pierde la información de las partes no diagonales de la matriz densidad. Posteriormente, en la sección \ref{sec2:master} se discutirá una ecuación maestra que preserva estos efectos. 

Al realizar la aproximación secular, la evolución del sistema queda
\begin{align*}
    \frac{d}{dt}\hat{\rho}_{IS}(t) & = \sum_{\omega}\sum_{\alpha,\beta} \Gamma_{\alpha \beta}(\omega)\left(A_{\beta}(\omega)\hat{\rho}_{IS}(t)A^{\dagger}_{\alpha}(\omega) - A^{\dagger}_{\alpha}(\omega)A_{\beta}(\omega) \hat{\rho}_{IS}(t) \right) \\
    & + \sum_{\omega}\sum_{\alpha,\beta} \Gamma^{*}_{\beta \alpha}(\omega) \left(A_{\alpha}(\omega)\hat{\rho}_{IS}(t)A^{\dagger}_{\beta}(\omega) - \hat{\rho}_{IS}(t)A^{\dagger}_{\beta}(\omega)A_{\alpha}(\omega) \right),
\end{align*}
finalmente, al separar la función correlación espectral en su parte real e imaginaria como

\begin{equation*}
\Gamma_{\alpha \beta}(\omega) = \gamma_{\alpha \beta}(\omega)/2 + iS_{\alpha \beta}(\omega).
\end{equation*}

Se obtiene la ecuación de Lindblad 

\begin{equation}
    \frac{d}{dt} \hat{\rho}_{IS}(t) = -i[\hat{H}_{LS},\hat{\rho}_{IS}(t)] + \mathcal{D}(\hat{\rho}_{IS}(t)),
    \label{seclindbladfinal}
\end{equation}
con $\hat{H}_{LS}$ el Hamiltoniano \textit{Lamb Shift}

\begin{equation*}
    \hat{H}_{LS} = \sum_{\omega} \sum_{\alpha,\beta} S_{\alpha,\beta}(\omega)A^{\dagger}_{\alpha}(\omega)A_{\beta}(\omega), 
\end{equation*}
y el Disipador

\begin{equation*}
    \mathcal{L}_{g}(\hat{\rho}_{IS}(t)) = \sum_{\omega}\sum_{\alpha,\beta} \gamma_{\alpha \beta}(\omega) \left[ A_{\beta}(\omega)\hat{\rho}_{IS}(t)A^{\dagger}_{\alpha}(\omega) - \frac{1}{2}\{A^{\dagger}_{\alpha}(\omega)A_{\beta}(\omega), \hat{\rho}_{IS}(t)  \} \right].
\end{equation*}

El escribir el disipador de esta manera es importante, ya que esta evolución se encuentra en la forma GKLS que permite preservar traza, hermiticidad y positividad de la matriz densidad\cite{manzano2020short}. Finalmente, en el cuadro de Schrodinger la ecuación de Lindblad queda

\begin{equation}
    \frac{d}{dt} \hat{\rho}_{S}(t) = -i[\hat{H}_{S}+\hat{H}_{LS},\hat{\rho}_{S}(t)] + \mathcal{L}_{g}(\hat{\rho}_{S}(t)).
    \label{eqsec2:lindbladS}
\end{equation}

Si el sistema $S$ interactua con múltiples reservorios, el disipador se puede escribir como una suma de los disipadores de cada uno de los reservorios, es decir

\begin{equation*}
    \mathcal{L}_{g}(\hat{\rho}_{S}(t)) = \sum_{\alpha}\mathcal{L}_{\alpha,g}(\hat{\rho}_{S}(t)).
\end{equation*}

\label{sec:lindblad}


\chapter{Estadística de conteo(\textit{Full Counting Statistics}) }
En este capítulo se presenta el formalismo de \textit{Full Counting Statistics}(FCS). La sección \ref{sec3workheat} esta dedicada a la  descripción de las cantidades termodinámicas de interés, y en la sección \ref{Leyestermo} se establecen las leyes de la termodinámica. En la sección \ref{sec2:estadistica2puntos} se desarrolla la estadística de medición en 2 puntos, mientras que en la sección \ref{sec2:superop} se describe brevemente el formalismo de superoperadores. En la sección \ref{sec2:master} se desarrolla el cálculo de la ecuación maestra generalizada deducida en \cite{potts2021thermodynamically}. Finalmente, en la sección \ref{sec2:thermolaws} se analiza la consistencia termodinámica, tanto para la ecuación maestra global, como para la ecuación maestra semilocal presentada en este capítulo. 

\section{Cantidades termodinámicas}
Se considera el sistema descrito por el Hamiltoniano

\begin{equation*}
    \hat{H}_{tot}(t) = \hat{H}_{S}(t) + \sum_{\alpha}(\hat{H}_{\alpha} + \hat{V}_{\alpha})  = \hat{H}_{S}(t) + \hat{H}_{B} + \hat{V},
\end{equation*}
donde el primer término describe el Hamiltoniano del sistema el cual puede ser tiempo dependiente, la segunda parte describe los reservorios térmicos y por último la tercera parte constituye el acoplamiento entre sistema-baño.

El sistema intercambia energía y partículas con el reservorio, por ende el cambio de energía puede ser dividido por una contribución correspondiente al calor y otra correspondiente al trabajo. Así, se define el calor que libera el baño $\alpha$ durante el intervalo de tiempo $[0,t]$ como

\begin{equation}
    \langle Q_{\alpha}\rangle = - \text{Tr}\{(\hat{H}_{\alpha} - \mu_{\alpha}\hat{N}_{\alpha})\hat{\rho}_{tot}(t) \} + \text{Tr}\{(\hat{H}_{\alpha} - \mu_{\alpha}\hat{N}_{\alpha})\hat{\rho}_{tot}(0) \},
\label{sec3:calor}
\end{equation}
en la cual $\hat{N}_{\alpha}$ corresponde al operador de número que cuantifica el número de partículas en el baño $\alpha$ y $\mu_{\alpha}$ es su potencial químico. El trabajo promedio que entrega el reservorio $\alpha$ se define como

\begin{equation}
    \langle W_{\alpha}\rangle = - \mu_{\alpha} (\text{Tr}\{\hat{N}_{\alpha} \hat{\rho}_{tot}(t) \} - \text{Tr}\{\hat{N}_{\alpha}\hat{\rho}_{tot}(0) \}  ).
    \label{sec3:trabajo}
\end{equation}
\label{sec3workheat}

\newpage

\section{Leyes de la termodinámica}
\label{Leyestermo}
\subsection{Ley cero}
Si se tiene un sistema total descrito por un sistema reducido y un baño, que se encuentran en equilibrio a temperatura inversa $\beta$ y potencial químico $\mu$. De tal manera que el estado del sistema en equilibrio es el equilibrio gran canónico

\begin{equation*}
    \hat{\rho}^{eq}_{tot} = \frac{e^{-\beta(\hat{H}_{tot} - \mu \hat{N}_{tot})}}{Z}  \hspace{28mm} Z = \text{Tr}\{e^{-\beta(\hat{H}_{tot} - \mu \hat{N}_{tot})} \},
\end{equation*}
y por lo tanto el estado de equilibrio del sistema reducido es

\begin{equation*}
    \hat{\rho}_{S} = \frac{1}{Z}\text{Tr}_{B}\{ e^{-\beta(\hat{H}_{tot} - \mu \hat{N}_{tot})} \},
\end{equation*}
que en el límite de acoplamiento débil entre el sistema y el baño, este equilibrio se convierte en \cite{geva2000second}
\begin{equation*}
    \hat{\rho}_{S} = \frac{e^{-\beta(\hat{H}_{S} - \mu\hat{N}_{S})}}{\text{Tr}_{S}\{e^{-\beta(\hat{H}_{S} - \mu \hat{N}_{S})} \} }.
\end{equation*}

\subsection{Primera Ley}
Para escribir la primera ley, primero se puede definir la corriente de calor y la potencia entregadas por el baño $\alpha$

\begin{equation*}
    J_{\alpha}(t) = \partial_{t}\langle Q_{\alpha}\rangle \hspace{12mm}  \dot{W}_{\alpha}(t) = \partial_{t}\langle W_{\alpha}\rangle.
\end{equation*}

Así, en el límite de acoplamiento débil para un Hamiltoniano tiempo independiente, la primera ley consiste en 
\begin{equation*}
    \partial_{t}E(t) = \sum_{\alpha}[J_{\alpha}(t) + \dot{W}_{\alpha}(t) ] \hspace{12mm} E = \text{Tr}\{\hat{H}_{S}\hat{\rho}_{tot}(t) \},
\end{equation*}
donde $E(t)$ es la energía interna del sistema.

%\begin{equation*}
 %   J_{\alpha}(t) = \text{Tr}\{(\hat{H}_{S} - \mu_{\alpha}\hat{N}_S)\mathcal{L}_{\alpha}\hat{\rho}_{S}(t) \} \hspace{12mm} P_{\alpha} = \mu_{\alpha} \text{Tr}\{\hat{N}_{S}\mathcal{L}_{\alpha}\hat{\rho}_{S}(t) \}. 
%\end{equation*}

\subsection{Segunda Ley}
Para introducir la segunda ley, se parte de la condición inicial en que el sistema y el reservorio son sistemas no correlacionados, es decir, 

\begin{equation*}
    \hat{\rho}_{tot}(0) = \hat{\rho}_{S}(0)\otimes_{\alpha}\hat{\tau}_{\alpha} \hspace{12mm} \hat{\tau}_{\alpha} = \frac{e^{-\beta_{\alpha}(\hat{H}_{\alpha}-\mu_{\alpha}\hat{N}_{\alpha}) }}{ \text{Tr}\{e^{-\beta_{\alpha}(\hat{H}_{\alpha} - \mu_{\alpha}\hat{N}_{\alpha})}\} },
\end{equation*}
donde cada reservorio se encuentra en equilibrio gran canónico con inverso de la temperatura $\beta_{\alpha}$ y potencial químico $\mu_{\alpha}$. Con esta condición inicial, la segunda ley puede ser escrita en función de la producción de entropía $\sigma(t)$ como \cite{esposito2010entropy}

\begin{equation*}
    \sigma(t) \equiv \Delta S(t) - \sum_{\alpha}\beta_{\alpha}\langle Q_{\alpha}\rangle \geq 0.
\end{equation*}

$\Delta S$ denota el cambio en la entropía de Von Neumann del sistema

\begin{equation*}
    \Delta S(t) = -  \text{Tr}\{\hat{\rho}_{S}(t)\ln \hat{\rho}_{S}(t) \} +  \text{Tr}\{ \hat{\rho}_{S}(0)\ln \hat{\rho}_{S}(0) \},
\end{equation*}
es decir, la producción de entropía se separa en una parte que corresponde al cambio de entropía del sistema, mientras que la segunda parte corresponde a la contribución debido al calor que entrega el ambiente.\\
Como en este trabajo se hará análisis de flujos continuos tanto de energía como de partículas, es importante estudiar la tasa de producción de entropía. Si bien, la producción de entropía siempre es positiva, la tasa de producción de entropía no siempre lo es, aún así para sistemas markovianos se cumple que \cite{strasberg2019non}

\begin{equation*}
    \dot{\sigma}(t) \equiv \partial_{t}\Delta S(t) - \sum_{\alpha}\beta_{\alpha}J_{\alpha} \geq 0.
\end{equation*}

Y la igualdad se cumple para procesos reversibles. Con esto se han definido las leyes de la termodinámica en función de flujos continuos, que se podrán calcular a través de la matriz densidad reducida. En esta tesis estas cantidades serán estudiadas principalmente en el estado estacionario.

\label{sec3sub:leyestermo}

\section{Estadística de medición en dos puntos}
A lo largo de este trabajo, se pretende calcular diversas cantidades termodinámicas asociadas a la dinámica del sistema, tales como la energía y el trabajo. Si bien los valores promedio de estas cantidades pueden obtenerse con las expresiones \ref{sec3:calor} y \ref{sec3:trabajo}, tanto el calor y el trabajo intercambiados con los reservorios deben considerarse como variables aleatorias, ya que su determinación implica medir sobre el estado de los reservorios. Debido a la naturaleza cuántica de la evolución, dichas mediciones pueden arrojar distintos resultados en diferentes repeticiones del experimento. Por lo tanto, es útil calcular la distribución de probabilidad del calor y el trabajo intercambiados con el reservorio.

Para  un sistema total descrito por la condición inicial $\hat{\rho}_{tot}(0) = \hat{\rho}_{S}(0) \otimes_{\alpha} \hat{\tau}_{\alpha}$ la distribución de probabilidad es   

\begin{align*}
    P(\textbf{Q},\textbf{W}) & = \sum_{\textbf{E},\textbf{E}',\textbf{N},\textbf{N}'} P_{t}(\textbf{E}',\textbf{N}'|\textbf{E},\textbf{N}) P_{0}(\textbf{E},\textbf{N})\\
                             & \times \Pi_{\alpha} \delta(W_{\alpha} - \mu_{\alpha}(N_{\alpha} - N'_{\alpha})) \delta(Q_{\alpha} + W_{\alpha}  - (E_{\alpha} - E'_{\alpha})).   
\end{align*}

La probabilidad conjunta de que cada baño $\alpha$ tenga energía $E_{\alpha}$, y número de partículas $N_{\alpha}$ a tiempo $t=0$ es

\begin{equation*}
    P_{0}(\textbf{E},\textbf{N}) = \Pi_{\alpha}  \frac{e^{-\beta_{\alpha}(E_{\alpha} - \mu_{\alpha}N_{\alpha} )  }}{ \text{Tr}\{e^{-\beta_{\alpha}(E_{\alpha} - \mu_{\alpha}N_{\alpha} )  }\} }.
\end{equation*}

Para obtener $P_{t}(\textbf{E}',\textbf{N}'|\textbf{E},\textbf{N})$ que corresponde a la probabilidad condicional de que los reservorios tengan energías $\textbf{E}'$ y números de partículas $\textbf{N}'$ a tiempo $t$, dado que inicialmente se midieron los valores $\textbf{E}$ y $\textbf{N}$, se debe partir del estado proyectado a $t=0$ correspondiente a dicha medición inicial, es decir,

\begin{equation*}
    \hat{\rho}'(0) = \frac{\hat{P}_{\textbf{E}, \textbf{N} }\hat{\rho}(0) \hat{P}_{\textbf{E}, \textbf{N} } }{\text{Tr}\{\hat{P}_{\textbf{E}, \textbf{N} }\hat{\rho}(0) \hat{P}_{\textbf{E}, \textbf{N} } \} },
\end{equation*}
donde $\hat{P}_{\textbf{E}, \textbf{N} } = I_{S} \otimes \hat{P}^{B}_{\textbf{E}, \textbf{N} }$ es el proyector asociado a una medición de las energías y los números de partículas en los reservorios. A continuación, el sistema evoluciona de forma unitaria hasta un tiempo $t$

\begin{equation*}
    \hat{\rho}'(t) = \hat{U}(t)\hat{\rho}'(0)\hat{U}^{\dagger}(t),
\end{equation*}
finalmente se realiza una segunda medición a tiempo $t$, obteniéndose los valores $\textbf{E}'$ y $\textbf{N}'$. De este modo, la probabilidad condicional correspondiente es

\begin{equation*}
    P_{t}(\textbf{E}',\textbf{N}'|\textbf{E},\textbf{N}) = \text{Tr}\{\hat{P}_{\textbf{E}', \textbf{N}' }\hat{\rho}'(t) \hat{P}_{\textbf{E}', \textbf{N}' } \},
\end{equation*}
escribiendo los proyectores de manera explícita $\hat{P}^{B}_{\textbf{E}, \textbf{N} } = |\textbf{E}, \textbf{N} \rangle\langle \textbf{E}, \textbf{N}|$, se puede obtener 

\begin{align*} 
    P_{t}(\textbf{E}',\textbf{N}'|\textbf{E},\textbf{N}) & =  \text{Tr}\{\hat{U}(t)(\hat{\rho}_{S}(0)\otimes |\textbf{E}, \textbf{N}\rangle  \langle \textbf{E}, \textbf{N}| ) \hat{U}^{\dagger}(t)  |\textbf{E}', \textbf{N}'\rangle  \langle \textbf{E}', \textbf{N}'| \} \\
        & =  \text{Tr}\{|\textbf{E},\textbf{N} \rangle \langle \textbf{E}', \textbf{N}'| \hat{U}(t)\hat{\rho}_{S}(0) \langle \textbf{E}, \textbf{N}|\hat{U}^{\dagger}(t)|\textbf{E}', \textbf{N}'\rangle \}      \\ 
        & = \text{Tr}_{S}\{ \text{Tr}_{B}\{|\textbf{E},\textbf{N} \rangle \langle \textbf{E}', \textbf{N}'|\hat{U}(t) \}\hat{\rho}_{S}(0)\langle \textbf{E}, \textbf{N}|\hat{U}^{\dagger}(t)|\textbf{E}', \textbf{N}'\rangle      \} \\
        & = \text{Tr}_{S}\{ \text{Tr}_{B}\{|\textbf{E},\textbf{N}\rangle \langle \textbf{E}',\textbf{N}'|\hat{U}(t)  \} \hat{\rho}_{S}(0) \text{Tr}_{B}\{\hat{U}^{\dagger}(t) |\textbf{E}',\textbf{N}' \rangle \langle \textbf{E},\textbf{N}| \}     \}.
    \end{align*}    

Por ende, $P_{t}(\textbf{E}',\textbf{N}'|\textbf{E},\textbf{N})$ se puede expresar como
\begin{equation*}
    P_{t}(\textbf{E}',\textbf{N}'|\textbf{E},\textbf{N}) = \text{Tr}_{S}\{\hat{M} \hat{\rho}_{S}(0)\hat{M}^{\dagger} \}  \hspace{10mm} \hat{M} = \text{Tr}_{B}\{|\textbf{E},\textbf{N} \rangle \langle \textbf{E}',\textbf{N}' | \hat{U}(t)\}.
\end{equation*}

Al contar con una distribución de probabilidad, se puede construir la función generadora de momentos, a partir de la cual pueden obtenerse cantidades como el promedio o la varianza. Esta función es

\begin{equation}
    \Lambda(\vec{\lambda},\vec{\chi}) \equiv \int d\textbf{Q} d\textbf{W}P(\textbf{Q},\textbf{W}) e^{-i\vec{\lambda}\cdot \textbf{Q} -i\vec{\chi}\cdot \textbf{W} },
\label{sec2funciongeneradora}
\end{equation}
que se puede escribir en función de la evolución de una matriz densidad auxiliar \ref{apendix:fcs1}

\begin{equation}
    \Lambda(\vec{\lambda},\vec{\chi}) = \text{Tr}\{\hat{\rho}_{tot}(\vec{\lambda},\vec{\chi};t) \}    \hspace{14mm} \hat{\rho}_{tot}(\vec{\lambda},\vec{\chi};t) = \hat{U}(\vec{\lambda},\vec{\chi};t) \hat{\rho}_{tot}(0) \hat{U}^{\dagger}(\vec{\lambda},\vec{\chi};t),
    \label{sec2:evolucionconteo}
\end{equation}
y 

\begin{equation*}
    \hat{U}(\vec{\lambda},\vec{\chi};t) = e^{\frac{i}{2}\sum_{\alpha}[\lambda_{\alpha}(\hat{H}_{\alpha} - \mu_{\alpha}\hat{N}_{\alpha} ) + \chi_{\alpha}\mu_{\alpha}\hat{N}_{\alpha} ]  } \hat{U}(t) e^{-\frac{i}{2}\sum_{\alpha}[ \lambda_{\alpha}(\hat{H}_{\alpha} - \mu_{\alpha}\hat{N}_{\alpha}) + \chi_{\alpha}\mu_{\alpha}\hat{N}_{\alpha} ]},
\end{equation*}
$\vec{\lambda}$ y $\vec{\chi}$ se denominan los parámetros de conteo o \textit{Counting Fields}, y $\hat{\rho}(\vec{\lambda},\vec{\chi};t)$ se denomina la matriz densidad generalizada. Conocer la evolución de esta matriz permite acceder a los momentos del calor, el trabajo y en general, del observable que se quiera estudiar. Sin embargo, dado que esta matriz se describe a partir de la evolución unitaria de la matriz densidad total, el siguiente paso consiste en derivar una ecuación maestra efectiva para los grados de libertad del sistema reducido.

\label{sec2:estadistica2puntos}



\section{Formalismo de Superoperadores y Espacio de Liouville}
Un operador en el espacio de Hilbert, representado por $\hat{\rho}$ y de dimensión $N\times N$, puede mapearse al espacio de Liouville mediante  un vector columna $|\rho \rangle \rangle$ de dimension $N^{2}\times 1$. De forma análoga, un superoperador \( L \), que actúa sobre \( \hat{\rho} \) en el espacio de Hilbert, se convierte en una matriz \( \check{L} \) de dimensión \( N^2 \times N^2 \) que actúa sobre \( |\rho\rangle\rangle \) en el espacio de Liouville. En este espacio se definen las siguientes operaciones

\begin{equation*}
    \langle \langle A|B\rangle \rangle  \equiv \text{Tr}\{\hat{A}^{\dagger}\hat{B}\}
\end{equation*}

\begin{equation*}
    \check{1}  \equiv \sum_{n,n'}|nn'\rangle \rangle \langle \langle nn'|
\end{equation*}

\begin{equation*}
     |nn'\rangle \rangle   \to |n\rangle \langle n'|  \hspace{10mm}  \langle \langle nn'| \to |n'\rangle \langle n|,
\end{equation*}
además, se cumple que 

\begin{equation*}
     \langle \langle nn'|mm'\rangle \rangle  = \delta_{nm}\delta_{n'm'}
\end{equation*}

\begin{equation*}
    \langle \langle nn'|A\rangle \rangle  = \langle n|\hat{A}|n'\rangle
\end{equation*}

\begin{equation*}
     \langle \langle 1|A\rangle \rangle  = \text{Tr}\{\hat{A}\}.
\end{equation*}


Si la evolución de la matriz densidad $|\hat{\rho}(t)\rangle \rangle$ esta gobernada por el superoperador $\check{\mathcal{L}}$, entonces la evolución temporal en el espacio de Liouville es 

\begin{equation*}
    \frac{d|\hat{\rho}(t) \rangle \rangle}{dt} = \check{\mathcal{L}}|\hat{\rho}(t) \rangle \rangle,
\end{equation*}
la solución formal de esta ecuación consiste en

\begin{equation}
    |\hat{\rho}(t)\rangle \rangle = e^{\check{\mathcal{L}}t}|\hat{\rho}(0)\rangle \rangle. 
    \label{sec2liouvilleformal}
\end{equation}

Además, en este espacio se pueden definir los superoperadores de proyección de Nakajima-Zwanzig, los cuales actúan sobre la matriz densidad total. En este contexto, se define el superoperador \( \check{\mathcal{P}} \), que proyecta sobre la parte relevante de la dinámica, mientras que \( \check{\mathcal{Q}} = 1 - \check{\mathcal{P}} \) proyecta sobre la parte irrelevante\cite{zwanzig1966statistical}. Estos operadores cumplen con las propiedades

\begin{align*}
    & \check{\mathcal{P}} + \check{\mathcal{Q}} = \check{1} \\
    & \check{\mathcal{P}}^{2} = \check{\mathcal{P}} \\
    & \check{\mathcal{Q}}^{2} = \check{\mathcal{Q}} \\
    & \check{\mathcal{P}}\check{\mathcal{Q}} = \check{\mathcal{Q}}\check{\mathcal{P}} = 0.
\end{align*}    

Finalmente, la evolución temporal de la matriz densidad a través de estos proyectores es 

\begin{align*}
    \frac{d}{dt}\check{\mathcal{P}}|\hat{\rho}(t)\rangle \rangle & = \check{\mathcal{P}}\check{\mathcal{L}}\check{\mathcal{P}}|\hat{\rho}(t)\rangle \rangle  + \check{\mathcal{P}}\check{\mathcal{L}}\check{\mathcal{Q}}|\hat{\rho}(t)\rangle \rangle \\
    \frac{d}{dt}\check{\mathcal{Q}}|\hat{\rho}(t)\rangle \rangle  & = \check{\mathcal{Q}}\check{\mathcal{L}}\check{\mathcal{Q}}|\hat{\rho}(t)\rangle \rangle  + \check{\mathcal{Q}}\check{\mathcal{L}}\check{\mathcal{P}}|\hat{\rho}(t)\rangle \rangle.
\end{align*}

\label{sec2:superop}

\newpage

\section{Ecuación maestra generalizada}
Si se considera un Hamiltoniano total $\hat{H} = \hat{H}_{S} + \hat{H}_{B} + \epsilon \hat{H}_{I} = \hat{H}_{0} + \epsilon \hat{H}_{I}$, con $\epsilon$ un parámetro adimensional que posteriormente permite aplicar acoplamiento débil. Y se utiliza este Hamiltoniano para la evolución unitaria $\hat{U}(t)$ en la ecuación \ref{sec2:evolucionconteo}, se obtiene la ecuación 

\begin{equation}
    \frac{d}{dt}\hat{\rho}_{tot}(\vec{\lambda},\vec{\chi},t) = -i[\hat{H}_{0},\hat{\rho}_{tot}(\vec{\lambda},\vec{\chi},t)] - i\epsilon[\hat{V}_{\lambda} \hat{\rho}_{tot}(\vec{\lambda},\vec{\chi},t) - \hat{\rho}_{tot}(\vec{\lambda},\vec{\chi},t)\hat{V}_{-\lambda}],
    \label{sec3:ecgeneral}
\end{equation}
con el operador $\hat{V}_{\lambda}$ descrito por

\begin{equation*}
    \hat{V}_{\lambda} = e^{-\frac{i}{2}\hat{A}(\lambda,\chi)}\hat{H}_{I}e^{\frac{i}{2}\hat{A}(\lambda,\chi)}, \hspace{14mm}  
\end{equation*}
y $\hat{A}(\lambda,\chi) = -\sum_{\alpha}[\lambda_{\alpha}(\hat{H}_{\alpha} - \mu_{\alpha}\hat{N}_{\alpha}) + \chi_{\alpha}\mu_{\alpha}\hat{N}_{\alpha} ]$. 

En el formalismo de superoperadores, \ref{sec3:ecgeneral} se transforma en 

\begin{align*}
    \frac{d}{dt}|\hat{\rho}_{tot}(\vec{\lambda},\vec{\chi},t)\rangle \rangle  & = \check{\mathcal{L}}_{\lambda}|\hat{\rho}_{tot}(\vec{\lambda},\vec{\chi},t)\rangle \rangle  \\  
        & = (\check{\mathcal{L}}_{0} + \epsilon \check{\mathcal{L}}'_{\lambda} )|\hat{\rho}_{tot}(\vec{\lambda},\vec{\chi},t)\rangle \rangle. 
\end{align*}

Si se utiliza el cuadro de interacción, esta ecuación se convierte en  

\begin{align*}
    \hat{\rho}_{totI}(\vec{\lambda},\vec{\chi},t) & = e^{-\mathcal{L}_{0}t}\hat{\rho}_{tot}(\vec{\lambda},\vec{\chi},t) \\
    & = e^{i\hat{H}_{0}t}\hat{\rho}_{tot}(\vec{\lambda},\vec{\chi},t)e^{-i\hat{H}_{0}t}.
\end{align*}

En este cuadro, el superoperador de Liouville que incorpora el parámetro de conteo se transforma en 

\begin{equation*}
    \check{\mathcal{L}}_{\lambda}(t) = e^{-\check{\mathcal{L}}_{0}t}\check{\mathcal{L}}_{\lambda}e^{\check{\mathcal{L}}_{0}t},
\end{equation*}
y la evolución temporal en el cuadro de interacción es 

\begin{equation}
    \frac{d}{dt}|\hat{\rho}_{totI}(\vec{\lambda},\vec{\chi},t)\rangle \rangle  = \epsilon \check{\mathcal{L}}_{\lambda}(t)|\hat{\rho}_{totI}(\vec{\lambda},\vec{\chi},t)\rangle \rangle,
 \label{sec2FCS:evolution}
\end{equation}
de este modo, la evolución de los grados de libertad del sistema reducido en el espacio de Hilbert queda descrita por 

\begin{multline}
    \dot{\hat{\rho}}_{IS}(\vec{\lambda},\vec{\chi},t) =  \epsilon^{2}\int_{0}^{t}ds \left[- \text{Tr}_{B}\{\hat{V}_{\lambda}(t)\hat{V}_{\lambda}(t-s)\hat{\rho}_{IS}(\vec{\lambda},\vec{\chi},t)\hat{\rho}^{eq}_{R} \} - \text{Tr}_{B}\{\hat{\rho}_{IS}(\vec{\lambda},\vec{\chi},t)\hat{\rho}^{eq}_{R}\hat{V}_{-\lambda}(t-s)\hat{V}_{-\lambda}(t) \} \right.\\
    \left. + \text{Tr}_{B}\{\hat{V}_{\lambda}(t)\hat{\rho}_{IS}(\vec{\lambda},\vec{\chi},t)\hat{\rho}^{eq}_{R}\hat{V}_{-\lambda}(t-s) \} + \text{Tr}_{B}\{ \hat{V}_{\lambda}(t-s)\hat{\rho}_{IS}(\vec{\lambda},\vec{\chi},t)\hat{\rho}^{eq}_{R}\hat{V}_{-\lambda}(t) \}  \right],
\label{ecmaestraVlambda}
\end{multline}
la demostración de esta ecuación se encuentra en el apéndice \ref{apendixsubsectionmatriz}.

Escribiendo una interacción de la forma

\begin{align*}
    \hat{V} & = \sum_{\alpha,k}\hat{S}_{\alpha,k}\hat{B}_{\alpha,k} \\
    \hat{V}_{\lambda} & = \sum_{\alpha,k}\hat{S}_{\alpha,k}\hat{B}_{\alpha,k,\lambda} \\
    \hat{B}_{\alpha,k,\lambda} & \equiv e^{(i/2)[\lambda_{\alpha}(\hat{H}_{\alpha} - \mu_{\alpha}\hat{N}_{\alpha}) + \chi_{\alpha}\mu_{\alpha}\hat{N}_{\alpha}]}\hat{B}_{\alpha,k}e^{-(i/2)[\lambda_{\alpha}(\hat{H}_{\alpha} - \mu_{\alpha}\hat{N}_{\alpha}) + \chi_{\alpha}\mu_{\alpha}\hat{N}_{\alpha}]},  
\end{align*}
donde los operadores del baño representan operadores creación o destrucción, que cumplen  
\begin{equation*}
    [\hat{B}_{\alpha,k},\hat{N}_{\alpha}] = n_{\alpha,k}\hat{B}_{\alpha,k},
\end{equation*}
con $n_{\alpha,k}$ el número de part\'iculas que se intercambia en la interacción $k$.
 
Y los operadores del sistema satisfacen

\begin{equation*}
    \hat{U}^{\dagger}_{S}(t)\hat{S}_{\alpha,k}\hat{U}_{S}(t) = \sum_{j}e^{-i\omega_{j}t}\hat{S}_{\alpha,k;j},
\end{equation*}
con $\hat{S}_{\alpha,k;j}$ los operadores de salto y $\omega_{j}$ las frecuencias de Bohr del Hamiltoniano del sistema. 

Cabe mencionar que los operadores $\hat{S}_{\alpha,k}$ no necesitan ser autoadjuntos, a diferencia de los  operadores de interacción considerados en \ref{sec:lindblad}.

Si se definen las funciones correlación como $C^{\alpha}_{k,k'}(s) = \text{Tr}\{e^{is\hat{H}_{\alpha} }\hat{B}^{\dagger}_{\alpha,k}e^{-is\hat{H}_{\alpha} }\hat{B}_{\alpha,k'}\hat{\tau}_{\alpha}\}$, se obtiene la ecuación maestra generalizada

\begin{equation}
    \frac{d}{dt}\hat{\rho}_{IS}(\vec{\lambda},\vec{\chi},t) = - \sum_{\alpha,k,k';j,j'}e^{i(\omega_{j}-\omega_{j'})t}\int_{0}^{t}ds \mathcal{I}(s,t) ,
\label{ecmaestrafinal}
\end{equation}
donde

\begin{multline}
    \mathcal{I}(s,t) = e^{i\omega_{j'}s} C^{\alpha}_{k,k'}(s)\hat{S}^{\dagger}_{\alpha,k;j}\hat{S}_{\alpha,k',j'}\hat{\rho}_{IS}(\vec{\lambda},\vec{\chi},t) + e^{-i\omega_{j}s}C^{\alpha}_{k,k'}(-s)\hat{\rho}_{IS}(\vec{\lambda},\vec{\chi},t)\hat{S}^{\dagger}_{\alpha,k;j}\hat{S}_{\alpha,k';j'} \\
    - e^{-i\mu_{\alpha}n_{\alpha,k}(\lambda_{\alpha} - \chi_{\alpha})}\left[e^{i\omega_{j'}s}C^{\alpha}_{k,k'}(s-\lambda_{\alpha}) + e^{-i\omega_{j}s}C^{\alpha}_{k,k'}(-s-\lambda_{\alpha})  \right]  \hat{S}_{\alpha,k';j'}\hat{\rho}_{IS}(\vec{\lambda},\vec{\chi},t)\hat{S}^{\dagger}_{\alpha,k;j}.
    \label{ecmaestrafinalI}
\end{multline}

Está ecuación fue deducida en el Apéndice \ref{finalequation}.

\label{sec2:master}

\subsection{Resolución finita de energía}
De forma análoga a lo presentado en la sección \ref{sec:lindblad}, uno de los requisitos que se busca en la evolución de la ecuación maestra es que sea Markoviana. Para ello, se debe tomar el límite superior del integral en el tiempo, presente en la ecuación \ref{ecmaestrafinal}, hacia infinito. Esta aproximación es válida siempre que el tiempo de correlación del baño $\tau_{B}$ sea mucho menor al tiempo de relajación del sistema $\tau_{R}$. Sin embargo, en presencia de parámetros de conteo $\lambda_{\alpha}$, las funciones de correlación del baño adquieren una dependencia modificada en el tiempo, del tipo $C^{\alpha}_{k,k'}(\pm \tau - \lambda_\alpha)$. Por lo tanto, para que la aproximación markoviana sea válida en presencia de parámetros de conteo, se requiere que 

\begin{equation*}
    C^{\alpha}_{k,k'}(\pm \tau - \lambda_\alpha) \approx 0 \quad \text{para } \tau > \tau_{B}.
\end{equation*}

Esto implica que el régimen de validez de la aproximación markoviana es 

\begin{equation*}
    \tau_{B} \ll \tau_{R} \hspace{10mm} |\lambda_{\alpha}| \ll \tau_{R}.
\end{equation*}

Esto tiene repercusiones importantes, ya que implica que la resolución de diferencias de energía en la medición del calor sea finita. Esta limitación se debe a que el parámetro $\lambda_{\alpha}$ y el calor medido en el baño $Q_{\alpha}$ son variables conjugadas en la distribución de probabilidad de calor y trabajo, por ende ambas obedecen una relación de incertidumbre \cite{folland1997uncertainty}. Como consecuencia las diferencias de energía del orden de $1/\tau_{R}$ dejan de ser confiables, ya que en ese régimen el valor promedio del calor es comparable con su varianza, es decir,

\begin{equation*}
    \langle \Delta \lambda^{2}_{\alpha} \rangle \langle (\Delta Q_{\alpha})^{2}\rangle \geq \gamma \implies \langle (\Delta Q_{\alpha})^{2}\rangle  \geq \frac{\gamma}{\tau^{2}_{R}},
\end{equation*}
con $\gamma$ alguna constante positiva. La profundidad de este resultado radica en que al aplicar una aproximación markoviana, la evolución del sistema sufre, de forma inherente, una resolución limitada respecto al calor intercambiado con los reservorios. Esta limitación puede conducir a inconsistencias termodinámicas, ya que se pierde información sobre fluctuaciones relevantes a escalas energéticas del orden $1/\tau_{R}$. Por lo tanto, para garantizar una evolución termodinámicamente consistente, es necesario redefinir las leyes de la termodinámica considerando explícitamente la resolución finita del calor impuesta por la dinámica.

\label{sec2:finiteresol}

\subsection{Agrupación de frecuencias}
%La ecuación de Redfield no siempre preserva positividad, lo que puede generar la aparición de probabilidades negativas en la matriz densidad del sistema reducido. La forma más común de asegurar la positividad es usar la aproximación secular vista en la sección \ref{sec:lindblad} para obtener la forma GKLS, el problema de aplicar esta aproximación es que requiere que las frecuencias de Bohr estén bien separadas con respecto a $1/\tau_{R}$, por lo tanto el aplicar esta aproximación necesita que no hayan frecuencias de Bohr casi degeneradas, eliminando una parte de los efectos cuánticos, ya que se pierden las coherencias entre niveles de energías cercanos\cite{trushechkin2021unified}. Se puede considerar un esquema diferente que asegure positividad, partiendo del punto que la aproximación de Markov asegura que para dos frecuencias de transición distintas, se cumple que $|\omega_{j} - \omega_{j'}|\ll 1/\tau_{B}$ o $|\omega_{j}-\omega_{j'}|\gg 1/\tau_{R}$. Incluso, se pueden cumplir las dos opciones. Dependiendo de cuál se cumpla, podemos agrupar las frecuencias de transición en conjuntos $x_{q}$, tal que si se cumple la primera o la segunda inecuación, estan en el mismo o en diferentes grupos, matemáticamente se traduce en
La ecuación de Redfield no garantiza en general la preservación de la positividad, lo que puede llevar a la aparición de probabilidades negativas en la matriz densidad del sistema reducido. La forma más común de asegurar dicha positividad es aplicar la aproximación secular, vista en la sección \ref{sec:lindblad}, la cual conduce a una ecuación maestra en forma (GKLS).

Sin embargo, esta aproximación requiere que las frecuencias de Bohr estén bien separadas en comparación con \( 1/\tau_R \), lo cual excluye casos con frecuencias casi degeneradas. Esta condición impone una pérdida de efectos cuánticos significativos, ya que elimina los términos no seculares en la evolución temporal\cite{trushechkin2021unified}. Para enfrentar esta limitación, puede considerarse un enfoque alternativo que conserve la positividad sin eliminar los términos no seculares. 

Partimos de que la aproximación de Markov($\tau_{B} \ll \tau_{R}$) garantiza que para dos frecuencias de transición distintas, se cumple al menos una de las siguientes condiciones:
\[
|\omega_j - \omega_{j'}| \ll 1/\tau_B \quad \text{o} \quad |\omega_j - \omega_{j'}| \gg 1/\tau_R.
\]
Incluso, ambas pueden cumplirse simultáneamente. Es posible agrupar las frecuencias de transición en subconjuntos \( x_q \), tal que frecuencias que cumplen la primera desigualdad pertenecen al mismo conjunto, mientras que aquellas que cumplen la segunda pertenecen a conjuntos distintos. Esto se traduce matemáticamente en


\begin{align*}
    |\omega_{j}-\omega_{j'}| \ll 1/\tau_{B}  &\hspace{10mm} \omega_{j} \in x_{q}, \omega_{j'} \in x_{q} \\
    |\omega_{j}-\omega_{j'}| \gg 1/\tau_{R}  &\hspace{10mm} \omega_{j} \in x_{q}, \omega_{j'} \in x_{q'}.
\end{align*}

Notemos que para frecuencias $\omega_{j}$, $\omega_{j'}$ que están en distintos grupos, se cumple la aproximación secular, es decir, los términos $e^{i(\omega_{j} - \omega_{j'})t}$ oscilan rápidamente, por ende en  promedio se anulan.

La funcion correlación  espectral $\gamma^{\alpha}(\omega) = \int_{-\infty}^{\infty} ds e^{i \omega s} C^{\alpha}(s)$, describe la influencia del baño $\alpha$ en transiciones de estados con diferencia de energía $\omega$, donde los $\omega$ corresponden a las frecuencias de Bohr del sistema. Dentro de un mismo conjunto $x_{q}$, los términos $e^{i\omega_{j}s},e^{i\omega_{j'}s}$ contribuyen, respectivamente, a  $\gamma^{\alpha}(\omega_{j})$ y $\gamma^{\alpha}(\omega_{j'})$, es decir,  describen transiciones inducidas por el baño con frecuencias cercanas $\omega_{j}$ y $\omega_{j'}$. Sin embargo, debido a la resolución finita del calor intercambiado con el reservorio, no es posible distinguir entre transiciones con frecuencias dentro del mismo conjunto $x_q$. Por lo tanto, en las funciones correlación espectral se deben sustituir las frecuencias individuales $\omega_{j} \in x_{q}$ por una frecuencia auxiliar $\omega_{q}$, es decir, en la ecuación \ref{ecmaestrafinalI} sustituir los términos  

\begin{equation*}
    e^{i\omega_{j}s},e^{i\omega_{j'}s} \to e^{i\omega_{q}s} \hspace{10mm} |\omega_{q} - \omega_{j}| \ll 1/\tau_{B} \hspace{10mm} \forall \omega_{j} \in x_{q}.
\end{equation*}

Esta sustitución refleja la incapacidad de distinguir transiciones con energías dentro del mismo conjunto \( x_q \), y permite construir una descripción coarse-grained de la dinámica que respeta tanto la positividad como las limitaciones de resolución energética\cite{chruscinski2017brief}. 

Usando este esquema en la ecuación \ref{ecmaestrafinal} y extendiendo los límites del integral de $\mathcal{I}$ de 0 a $\infty$, se obtiene la ecuación en la forma GKLS

\begin{equation*}
    \frac{d}{dt}\hat{\rho}_{IS}(\vec{\lambda},\vec{\chi},t) = -i[\hat{H}_{LS},\hat{\rho}_{IS}(\vec{\lambda},\vec{\chi},t)] + \sum_{\alpha}\tilde{\mathcal{L}}^{\chi_{\alpha},\lambda_{\alpha}}_{\alpha} \hat{\rho}_{IS}(\vec{\lambda},\vec{\chi},t),
\end{equation*}
con

\begin{equation*}
    \tilde{\mathcal{L}}^{\chi_{\alpha},\lambda_{\alpha}}_{\alpha}\hat{\rho} = \sum_{k,q}\gamma^{\alpha}_{k}(\omega_{q}) \left[e^{i\lambda_{\alpha}\omega_{q} + i(\chi_{\alpha}-\lambda_{\alpha})\mu_{\alpha}n_{\alpha,k}}\hat{S}_{\alpha,k;q}(t)\hat{\rho}\hat{S}^{\dagger}_{\alpha,k;q}(t) - \frac{1}{2}\{\hat{S}^{\dagger}_{\alpha,k;q}(t)\hat{S}_{\alpha,k;q}(t),\hat{\rho} \} \right].
\end{equation*}

Los operadores de salto consisten en

\begin{equation*}
    \hat{S}_{\alpha,k;q}(t) = \sum_{\{j|\omega_{j}\in x_{q} \} } e^{-i\omega_{j}t}\hat{S}_{\alpha,k;j}.
\end{equation*}

Y el Hamiltoniano de \textit{Lamb Shift}

\begin{equation*}
    \hat{H}_{LS} = \sum_{\alpha,k;q} \Delta^{\alpha}_{k}(\omega_{q}) \hat{S}^{\dagger}_{\alpha,k;q}(t)\hat{S}_{\alpha,k;q}(t),
\end{equation*}
con las cantidades

\begin{equation}
    \gamma_{k}^{\alpha}(\omega) = \int_{-\infty}^{\infty}ds e^{i\omega s}C^{\alpha}_{k,k}(s) \hspace{10mm} \Delta^{\alpha}_{k}(\omega) = - \frac{i}{2} \int^{\infty}_{-\infty}ds e^{i\omega s} \text{sign}(s)C^{\alpha}_{k,k}(s).
    \label{sec2:gammaanddelta}
\end{equation}

En donde se asume por simplicidad $C^{\alpha}_{k,k'} \propto \delta_{k,k'}$. La demostración de esta ecuación está incluida en el apéndice \ref{apendixGKLSgeneral}. 

En el límite en que los parámetros de conteo tienden a cero, se obtiene  

\begin{equation*}
    \frac{d}{dt}\hat{\rho}_{IS}(t) = - i[\hat{H}_{LS}(t),\hat{\rho}_{IS}(t)] + \sum_{\alpha}\tilde{\mathcal{L}}_{\alpha} \hat{\rho}_{IS}(t),
\end{equation*}
con
\begin{equation*}
    \tilde{\mathcal{L}}_{\alpha} = \sum_{\{q|\omega_{q}>0\}} \sum_{k}\gamma^{\alpha}_{k,k}(\omega_{q}) \left[ \mathcal{D}[\hat{S}_{\alpha,k,q}(t)] + e^{-\beta_{\alpha}(\omega_{q} - \mu_{\alpha}n_{\alpha,k})}\mathcal{D}[\hat{S}^{\dagger}_{\alpha,k,q}(t)]  \right].
\end{equation*}

La demostración de esta ecuación se encuentra en \ref{apendixKMS}.

Para un Hamiltoniano tiempo independiente, la ecuación maestra en el cuadro de Schrodinger es

\begin{equation}
    \frac{d}{dt}\hat{\rho}_{S}(t) = -i [\hat{H}_{S}+ \hat{H}_{LS},\hat{\rho}_{S}(t)] + \sum_{\alpha}\mathcal{L}_{\alpha}(\hat{\rho}_{S}(t)),
\label{sec2schrodingerthermo}
\end{equation}
con

\begin{equation}
    \mathcal{L}_{\alpha} = \sum_{\{q|\omega_{q}>0\}} \sum_{k}\gamma^{\alpha}_{k,k}(\omega_{q}) \left[ \mathcal{D}[\hat{S}_{\alpha,k,q}] + e^{-\beta_{\alpha}(\omega_{q} - \mu_{\alpha}n_{\alpha,k})}\mathcal{D}[\hat{S}^{\dagger}_{\alpha,k,q}]  \right].
\label{sec2lindbladconsistency}
\end{equation}

Hay dos límites importantes a considerar, el primero corresponde al caso en que todas las frecuencias satisfacen $|\omega_{j}-\omega_{j'}| \gg 1/\tau_{R}$, pues se cumple que $\mathcal{D}[\hat{S}_{\alpha,k,q}] = \mathcal{D}[\hat{S}_{\alpha,k,j}]$, lo que permite realizar la aproximación secular entre todas las frecuencias de Bohr y recuperar la ecuación de Lindblad \ref{eqsec2:lindbladS}. En el caso de que se cumpla que $|\omega_{j}-\omega_{j'}| \ll 1/\tau_{B}$, todas las frecuencias se agrupan en un sólo grupo. Por lo tanto, los operadores del sistema cumplen que $\hat{S}_{\alpha,k;q} = \hat{S}_{\alpha,k}$, y la ecuación maestra queda descrita por operadores locales, lo que corresponde a la ecuación maestra local\cite{wichterich2007modeling}.
\section{Consistencia termodinámica}
Existen dos casos a diferenciar para derivar las leyes de la termodinámica, primero la ecuación maestra global definida en la sección \ref{sec:lindblad} y segundo la ecuación maestra semilocal, que presenta un mayor desafío; debido a la resolución finita de energía impuesta por la dinámica markoviana en la ecuación semilocal, para asegurar la consistencia termodinámica es necesario redefinir las leyes de la termodinámica. Como primer paso, se introduce el Hamiltoniano termodinámico $\hat{H}_{TD}$, el cual satisface la relación de conmutación
\label{sec2:thermolaws}
\begin{equation*}
    [\hat{S}_{\alpha,k,j},\hat{H}_{TD}] = \omega_{q}\hat{S}_{\alpha,k,j},
\end{equation*}
para todas las frecuencias $\omega_{j} \in x_{q}$. Este Hamiltoniano puede construirse mediante el Hamiltoniano $\hat{H}_{S}$, modificando sus autovalores de modo que todas las transiciones dentro de un mismo conjunto $x_q$ compartan una frecuencia $\omega_{q}$ para $\omega_{j} \in x_{q}$.

Para las leyes de la termodinámica, se debe hacer la distinción para la ecuación maestra global y la semilocal, para el caso global la energía interna corresponde a 

\begin{equation*}
    E_{g}(t) = \text{Tr}\{\hat{H}_{S}\hat{\rho}(t) \}.
\end{equation*}

Por otro lado, para la ecuación maestra semilocal, la energía interna se redefine con el Hamiltoniano termodinámico mencionado anteriormente 

\begin{equation}
    E(t) = \text{Tr}\{\hat{H}_{TD}\hat{\rho}(t) \}.
\label{sec2:energyeq}
\end{equation}

La corriente de calor y el trabajo entregado por el baño $\alpha$ para la ecuación global será

\begin{equation*}
    J_{\alpha,g}(t) = \text{Tr}\{(\hat{H}_{S} - \mu_{\alpha}\hat{N}_S)\mathcal{L}_{\alpha,g}\hat{\rho}_{S}(t) \} \hspace{12mm} \dot{W}_{\alpha,g} = \mu_{\alpha} \text{Tr}\{\hat{N}_{S}\mathcal{L}_{\alpha,g}\hat{\rho}_{S}(t) \}.
\end{equation*}

Donde $\mathcal{L}_{\alpha,g}$ es el superoperador de Lindblad asociado al baño $\alpha$, correspondiente a la ecuación maestra global \ref{eqsec2:lindbladS}.

Finalmente, la corriente de calor y el trabajo entregado por el baño $\alpha$ para el caso semilocal será

\begin{equation}
    J_{\alpha}(t) = \text{Tr}\{(\hat{H}_{TD} - \mu_{\alpha}\hat{N}_S)\mathcal{L}_{\alpha}\hat{\rho}_{S}(t) \} \hspace{12mm} \dot{W}_{\alpha} = \mu_{\alpha} \text{Tr}\{\hat{N}_{S}\mathcal{L}_{\alpha}\hat{\rho}_{S}(t) \}.
\label{sec2:heatworkeq}
\end{equation}

Estas definiciones automáticamente cumplen con la primera ley de la termodinámica, ya que si se deriva la energía interna para la ecuación global 

\begin{equation*}
    \partial_{t}E_{g}(t) = \text{Tr}\{ \hat{H}_{S}\partial_{t}\hat{\rho}_{S}(t) \} = \sum_{\alpha}[J_{\alpha,g} + \dot{W}_{\alpha,g}].
\end{equation*}

También, si se aplica la relación de conmutación $[\hat{H}_{TD},\hat{H}_{S} + \hat{H}_{LS}] = 0$, se obtiene la primera ley para el caso semilocal 

\begin{equation*}
    \partial_{t}E(t) = \text{Tr}\{ \hat{H}_{TD}\partial_{t}\hat{\rho}_{S}(t) \} = \sum_{\alpha}[J_{\alpha} + \dot{W}_{\alpha}].
\end{equation*}

\subsection{Ley cero}
Para la ecuación global, se cumple que el estado de equilibrio para el baño $\alpha$ es 

\begin{equation*}
    \mathcal{L}_{\alpha,g}e^{-\beta_{\alpha}(\hat{H}_{S} - \mu_{\alpha}\hat{N}_{S})} = 0,
\end{equation*}
además, si los reservorios poseen la misma temperatura inversa $\beta$ y el mismo potencial químico $\mu$, el estado de Gibbs corresponde a

\begin{equation*}
    \hat{\rho}_{G,g} = \frac{e^{-\beta(\hat{H}_{S} - \mu \hat{N}_{S})}}{\text{Tr}\{ e^{-\beta(\hat{H}_{S} - \mu \hat{N}_{S})}\}}.
\end{equation*}

Por el contrario, para la ecuación semilocal se cumple que

\begin{equation}
    \mathcal{L}_{\alpha}e^{-\beta_{\alpha}(\hat{H}_{TD} - \mu_{\alpha}\hat{N}_{S})} = 0,
\label{sec2cerolaw}
\end{equation}
y el estado de Gibbs corresponde a 
\begin{equation*}
    \hat{\rho}_{G} = \frac{e^{-\beta(\hat{H}_{TD} - \mu \hat{N}_{S})}}{\text{Tr}\{ e^{-\beta(\hat{H}_{TD} - \mu \hat{N}_{S})}\}}.
\end{equation*}

\subsection{Segunda ley}
Para definir la tasa de producción de entropía, se debe considerar la desigualdad de Spohn\cite{spohn2007irreversible}, que permite escribir las tasas de producción de entropía para el caso global y el semilocal\cite{potts2021thermodynamically}. Para el primero se cumple que 

\begin{equation*}
    \dot{\sigma}_{g} = - \frac{d}{dt}\text{Tr}\{\hat{\rho}_{S}(t) \ln \hat{\rho}_{S}(t) \} - \sum_{\alpha} \beta_{\alpha} J_{\alpha,g}(t) = -\sum_{\alpha} \text{Tr}\{(\mathcal{L}_{\alpha,g}\hat{\rho}_{S}(t))[\ln \hat{\rho}_{S}(t) - \ln \hat{\rho}_{G,g}(\beta_{\alpha},\mu_{\alpha})] \} \geq 0.
\end{equation*}

Mientras que para el segundo, la tasa de producción de entropía será 

\begin{equation}
    \dot{\sigma} = - \frac{d}{dt}\text{Tr}\{\hat{\rho}_{S}(t) \ln \hat{\rho}_{S}(t) \} - \sum_{\alpha} \beta_{\alpha} J_{\alpha}(t) = -\sum_{\alpha} \text{Tr}\{(\mathcal{L}_{\alpha}\hat{\rho}_{S}(t))[\ln \hat{\rho}_{S}(t) - \ln \hat{\rho}_{G}(\beta_{\alpha},\mu_{\alpha})] \} \geq 0.
\label{sec2secondlaw}
\end{equation}

La demostración de \ref{sec2cerolaw} y de \ref{sec2secondlaw} se encuentra en el apéndice \ref{apendix:thermolaws}. 

De este modo, se definen las leyes de la termodinámica para las ecuaciones maestras global y semilocal, descritas por \ref{eqsec2:lindbladS} y \ref{sec2schrodingerthermo}, respectivamente.



% ------------------------------------------------------------------------------
% NUEVO CAPÍTULO
% ------------------------------------------------------------------------------
\chapter{Flujos de información}
En este capítulo se introduce el concepto de información termodinámica, comenzando en la sección \ref{sec4:Demon} con una revisión del paradigma del Demonio de Maxwell. En la sección \ref{sec4:autonomo} se presenta el Demonio de Maxwell autónomo. Finalmente, en las secciones \ref{sec4:flujos} y \ref{sec4:flujos0} se define el concepto de flujos de información, utilizando herramientas como la termodinámica estocástica en contexto clásico y la matriz densidad en contexto cuántico\cite{horowitz2014thermodynamics,ptaszynski2019thermodynamics}.


\section{Demonio de Maxwell}
El demonio de Maxwell es un experimento mental esbozado por James Clerk Maxwell en su obra\cite{Maxwell_1871}. Consiste en una caja dividida en dos compartimientos, A y B, cada uno de ellos lleno con un gas ideal a temperatura $T$ y presión $P$, como se muestra en la Figura \ref{img:demon}. Entre ambos compartimientos existe una compuerta sin masa, que permite el paso de partículas de un lado a otro. Esta puerta es controlada por un ente denominado "Demonio", el cual tiene la capacidad de manipular la puerta sin costo energético. La característica esencial del Demonio es que conoce la velocidad de cada partícula en ambos compartimientos.

Dado que la temperatura se relaciona con la velocidad promedio de las partículas, el Demonio puede discriminar entre partículas rápidas y lentas, las cuales influirán en la temperatura del compartimiento que se dirigen. De este modo, permite pasar del compartimiento A al B solo aquellas partículas cuya velocidad es mayor que la velocidad promedio, mientras que deja pasar del compartimiento B al A las partículas más lentas. Si se mantiene constante el número de partículas en cada compartimiento, este proceso da lugar a un aumento de la temperatura en A y una disminución en B, es decir, una diferencia \( \Delta T \) entre ambos lados.

El cambio de entropía del sistema es
 
\begin{align*}
    \Delta S & = \Delta S_{A} + \Delta S_{B} = C_{V}\left( \log \frac{T-\Delta T}{T} + \log \frac{T+\Delta T}{T} \right) \\
       & =  C_{V} \log \left( 1 - \frac{\Delta T^{2}}{T^{2}}  \right) < 0,
\end{align*}
donde \( C_V \) representa la capacidad calorífica a volumen constante.

Del cálculo de la entropía total se concluye que, sin realizar trabajo externo, el Demonio logra una disminución de la entropía, lo cual representa una aparente violación de la segunda ley de la termodinámica.

Sin embargo, esta paradoja fue resuelta posteriormente por Rolf Landauer, quien ``exorcizó'' al Demonio al señalar que, para que este pueda conocer la velocidad de las partículas, debe realizar mediciones. El proceso de medición, y en particular el borrado de la información asociada a dichas mediciones, implica una disipación de energía. Este resultado, conocido como el principio de Landauer, restablece la validez de la segunda ley\cite{Landauer_1961}.

En sistemas no autónomos, es decir, sistemas manipulados por un agente externo a través del control de cantidades macroscópicas, la paradoja de Maxwell ha sido abordada extensamente. Un ejemplo clásico es el motor de Szilard\cite{szilard1964decrease}, donde se ha logrado describir y ``exorcizar'' al Demonio cuantificando tanto la energía requerida para realizar una medición,  como el costo energético asociado al borrado de la memoria del Demonio, y el trabajo máximo que puede extraerse mediante retroalimentación (feedback) \cite{maruyama2009colloquium, sagawa2008second}. No obstante, el caso de sistemas autónomos, en los cuales no existe intervención externa y tanto la dinámica como la retroalimentación son generadas internamente, presenta características fundamentales que merecen un análisis más detallado.

\insertimage[\label{img:demon}]{ejemplos/Maxwelldemon1}{scale=0.9}{Esquema que representa al Demonio de Maxwell. Primero, consiste en los dos compartimientos que poseen el gas ideal distribuido de manera homogénea. Por último, al trasladar las partículas de un lado a otro, queda el compartimiento A con partículas frías mientras que el compartimiento B con partículas calientes. Esta figura fue tomada de \cite{link1} .}

\label{sec4:Demon}

\section{Demonio de Maxwell autónomo}
Muchos procesos físicos requieren la interacción entre un conjunto de subsistemas que forman un sistema global. Esta interacción entre los subsistemas no sólo incluye un intercambio de energía o partículas, sino que también incluye un intercambio de información a medida que se correlacionan estos subsistemas entre sí de manera autónoma, es decir, sin un factor externo el cuál realize un feedback en él.

Comprender como son utilizados estos flujos de información para hacer tareas útiles es de gran relevancia. Un ejemplo claro se encuentra en los sistemas biólogicos, donde ocurre adaptación sensorial, que corresponde al monitoreo que realiza un organismo a su ambiente mientras cambia su respuesta a él\cite{lan2012energy}. Para caracterizar este tipo de procesos, se considera un Demonio de Maxwell autónomo como un sistema bipartito. Donde una de sus partes consiste en el sistema controlado, mientras que la otra parte en el sistema actúa como detector, el cuál ejerce control mediante una interacción física que afecta al sistema controlado.

Es fundamental que el sistema sea autónomo, es decir, que el Hamiltoniano total sea tiempo independiente e intervenciones externas, tales como mediciones y feedback no sean consideradas.

\label{sec4:autonomo}
\section{Descripción clásica}
Para describir un demonio de Maxwell autónomo, es necesario poder calcular la evolución de un sistema acoplado a uno o más reservorios, y por lo tanto, sujeto a las leyes de la termodinámica. 

Consideremos el caso de dos sistemas independientes, \( X \) e \( Y \), cada uno con estados discretos denotados por \( x \) e \( y \), respectivamente. Ambos sistemas presentan una dinámica interna caracterizada por saltos aleatorios entre sus respectivos estados, cuya tasa de transición está determinada por los reservorios locales a los que cada subsistema se encuentra acoplado. Estas tasas cumplen la condición de \textit{balance detallado local} \cite{van2015ensemble}.

La evolución entre estados se modela como un proceso de Markov\cite{van1992stochastic}. Para estudiar la evolución conjunta del sistema, se asume un acoplamiento bipartito: en este contexto, esto significa que si el sistema total se encuentra en un estado conjunto $(x,y)$, sólo se permiten transiciones del tipo $(x,y) \to (x,y')$ o $(x,y) \to (x',y)$, pero no transiciones simultáneas del tipo $(x,y) \to (x',y')$.

La probabilidad \( p(x,y) \) de encontrar al sistema en el estado \( (x,y) \) está gobernada por la siguiente ecuación maestra:

\begin{equation*}
    d_{t}p(x,y) = \sum_{x',y'} \left[ W_{x,x'}^{y,y'}p(x',y')  - W_{x',x}^{y',y}p(x,y) \right].
\end{equation*}

La tasa de transición \( W_{x,x'}^{y,y'} \) para un salto del estado \( (x',y') \) al estado \( (x,y) \), obedece la condición de balance detallado local:

\[
\ln \left( \frac{W_{x,x'}^{y,y'}}{W_{x',x}^{y',y}} \right) = -\frac{\epsilon_{x,y} - \epsilon_{x',y'}}{T},
\]

la cual relaciona las tasas de transición con el cambio de energía \( \Delta \epsilon = \epsilon_{x,y} - \epsilon_{x',y'} \) asociado al salto respectivo. Esta energía es intercambiada con el reservorio térmico local en forma de calor.

Debido a que el sistema es bipartito, la forma de las tasas de transición corresponde a

\begin{equation*}
    W_{x,x'}^{y,y'} = \left\{ \begin{array}{lcc} w_{x,x'}^{y} & si & x \neq x'; y=y' \\ \\ w_{x}^{y,y'} & si & x=x';y\neq y'\\ \\ 0 & si & x \neq x'; y \neq y' \end{array} \right..
\end{equation*}

En términos de la corriente de probabilidad 

\begin{equation*}
    J_{x,x'}^{y,y'} = W_{x,x'}^{y,y'}p(x',y') - W_{x',x}^{y',y}p(x,y),
\end{equation*}
la ecuación maestra puede reescribirse como 

\begin{equation*}
    d_{t}p(x,y) = \sum_{x',y'}J_{x,x'}^{y,y'} = \sum_{x'}J_{x,x'}^{y} + \sum_{y'}J_{x}^{y,y'}.
\end{equation*}

Donde  
\[
J_{x}^{y,y'} = w_{x}^{y,y'}\,p(x,y') - w_{x}^{y',y}\,p(x,y),
\]
representa la corriente de probabilidad correspondiente a un salto entre los estados \( y' \to y \), manteniendo fijo el estado \( x \). Es decir, describe el flujo en la dirección \( Y \) para un valor dado de \( x \). De manera análoga, $J^{y}_{x,x'}$ describe el flujo en la dirección \( X\) para un valor fijo de \(y\).  

Este hecho es importante, ya que la estructura bipartita permite separar las corrientes de probabilidad en dos componentes: una correspondiente a transiciones en la dirección \( X \), y otra en la dirección \( Y \). Esta propiedad puede ser aprovechada para estudiar funcionales de la corriente, es decir, cantidades de la forma
\[
\mathcal{A}(J) = \sum_{x,x',y,y'} J_{x,x'}^{y,y'}\, A_{x,x'}^{y,y'},
\]
que pueden ser separados en dos contribuciones

\begin{equation}
    \mathcal{A}(J) = \sum_{x\geq x';y \geq y'} J_{x,x'}^{y}A_{x,x'}^{y,y'} + \sum_{x \geq x'; y \geq y'}J_{x}^{y,y'} A_{x,x'}^{y,y'}.
\label{sec4:functionalcurrent}
\end{equation}

De este modo, la variación de \( \mathcal{A} \) se puede descomponer en dos contribuciones: una asociada a las transiciones en la dirección \( X \), y otra correspondiente a las transiciones en la dirección \( Y \).


\label{sec4:flujos}
\section{Segunda ley de la termodinámica y flujos de información}
El sistema conjunto \( XY \), al estar en contacto con un baño, constituye un sistema abierto que debe satisfacer la segunda ley de la termodinámica. Esta ley exige que la tasa de producción de entropía sea siempre mayor o igual que cero, es decir, 

\begin{equation*}
    \dot{\sigma} = \partial_{t}S^{XY} + \dot{S}_{r} \geq 0. 
\end{equation*}

Donde $\partial_{t}S^{XY}$ corresponde a la derivada temporal de la entropía del sistema, que se expresa como  

\begin{equation*}
    \partial_{t}S^{XY} = \sum_{x\geq x'; y\geq y'} J_{x,x'}^{y,y'} \ln \frac{p(x',y')}{p(x,y)}.
\end{equation*}

Y el cambio de entropía en el ambiente corresponde a

\begin{equation*}
    \dot{S}_{r} = \sum_{x\geq x'; y\geq y'} J_{x,x'}^{y,y'} \ln \frac{W_{x,x'}^{y,y'}}{W_{x',x}^{y',y}},
\end{equation*}
por ende

\begin{equation*}
    \dot{\sigma} = \sum_{x\geq x'; y\geq y'} J_{x,x'}^{y,y'} \ln \frac{ W_{x,x'}^{y,y'}p(x',y')  }{ W_{x',x}^{y',y}p(x,y) } \geq 0.
\end{equation*}

Estos resultados se demuestran en el apendice \ref{apendix4:secondlaw}.  

Una vez definida la tasa de producción de entropía, es necesario cuantificar la información compartida entre los dos subsistemas. La magnitud que permite medir las correlaciones entre ellos es la información mutua, la cual captura las correlaciones estadísticas entre las variables \( X \) e \( Y \). Se define por

\begin{equation*}
    I_{XY} = \sum_{x,y} p(x,y) \ln \frac{p(x,y)}{p(x)p(y)} \geq 0. 
\end{equation*}

De este modo, cuando \( I_{XY} \) es grande, los dos subsistemas están fuertemente correlacionados, mientras que $I_{XY}$ pequeña implica que los dos sistemas se conocen poco entre ellos, \( I_{XY} = 0 \) indica que son estadísticamente independientes.

Para definir los flujos de información, es necesario calcular la variación temporal de la información mutua. Esta puede descomponerse en dos contribuciones asociadas a las dinámicas de cada subsistema:
\[
\partial_t I_{XY} = \dot{I}^{X} + \dot{I}^{Y},
\]
donde \( \dot{I}^{X} \) representa la contribución debida a la evolución del subsistema \( X \) (con \( Y \) fijo), e \( \dot{I}^{Y} \) la correspondiente al subsistema \( Y \). Que explícitamente son

\begin{equation}
    \dot{I}^{X}  = \sum_{x\geq x'; y}J_{x,x'}^{y} \ln \frac{ p(y|x) }{p(y|x')}
    \label{sec4:clasicalflows0}
\end{equation}

\begin{equation}    
    \dot{I}^{Y}  = \sum_{x;y\geq y'} J_{x}^{y,y'} \ln \frac{p(x|y)}{ p(x|y') }.
    \label{sec4:clasicalflows}
\end{equation}

 La demostración de este resultado también está en el apendice \ref{apendix4:secondlaw}.

Los términos \( \dot{I}^{X} \) e \( \dot{I}^{Y} \) cuantifican cómo fluye la información entre los dos subsistemas. Si \( \dot{I}^{X} > 0 \), significa que un salto en la dirección \( X \), en promedio, aumenta la información mutua \( I_{XY} \); es decir, el subsistema \( X \) está ``aprendiendo'' o midiendo al subsistema \( Y \).

En cambio, si \( \dot{I}^{X} < 0 \), los saltos en la dirección \( X \) reducen las correlaciones, lo que puede interpretarse como un consumo de información con el fin de extraer trabajo o energía.

Como la tasa de producción de entropía es un funcional de la corriente de probabilidad, se puede usar la ecuación \ref{sec4:functionalcurrent} para escribir:


 \begin{equation*}
    \dot{\sigma} = \dot{\sigma}^{X} + \dot{\sigma}^{Y},
 \end{equation*}
así

\begin{align*}
    \dot{\sigma}^{X} & = \sum_{x \geq x';y} J_{x,x'}^{y} \ln  \frac{w_{x,x'}^{y} p(x',y) }{w_{x',x}^{y} p(x,y) } \geq 0 \\
    \dot{\sigma}^{Y} & = \sum_{x;y\geq y'}J_{x}^{y,y'} \ln \frac{w_{x}^{y,y'} p(x,y') }{ w_{x}^{y',y} p(x,y) } \geq 0, 
\end{align*}
donde se pueden identificar las tasas de producción de entropía locales

\begin{align*}
    \dot{\sigma}^{X} &  = \sum_{x \geq x';y} J_{x,x'}^{y} \left[ \ln \frac{p(x')}{p(x)}  +\ln \frac{w_{x,x'}^{y}}{ w_{x',x}^{y} } + \ln \frac{p(y|x')}{p(y|x)} \right] \\
    \dot{\sigma}^{Y} &  = \sum_{x;y \geq y'} J_{x}^{y,y'} \left[ \ln \frac{p(y')}{p(y)}  + \ln \frac{w_{x}^{y,y'}}{ w_{x}^{y',y} } + \ln \frac{p(x|y')}{p(x|y)} \right].
\end{align*}

Y en función de los flujos de información son  

\begin{align*}
    \dot{\sigma}^{X} & = \partial_{t}S^{X} + \dot{S}_{r}^{X} - \dot{I}^{X} \geq 0 \\
    \dot{\sigma}^{Y} & = \partial_{t}S^{Y} + \dot{S}_{r}^{Y} - \dot{I}^{Y} \geq 0.
\end{align*}

Estas dos ecuaciones permiten visualizar cómo la contribución de la información influye en la producción de entropía local de cada subsistema.

Supongamos que se desconoce la existencia de la interacción entre el subsistema \( Y \) y el subsistema \( X \), y que únicamente se monitorea la dinámica de \( X \). En ese caso, se le asignaría al sistema una tasa de producción de entropía dada por:
\[
\dot{\sigma}^{X}_0 = \partial_t S^X + \dot{S}_r^X.
\]

Si el subsistema \( X \) estuviera efectivamente aislado, se cumpliría \( \dot{\sigma}^{X}_0 \geq 0 \), en concordancia con la segunda ley de la termodinámica. Sin embargo, debido a su interacción con \( Y \), pueden darse situaciones en las que \( \dot{\sigma}^{X}_0 < 0 \), lo cual representa una violación aparente de la segunda ley. Esta puede interpretarse como el efecto de un Demonio de Maxwell, que utiliza la información que proviene de monitorear \(X \).

En la evolución de sistemas autónomos que alcanzan un estado estacionario, se cumple que \( \partial_t I_{XY} = 0 \). En ese régimen, el flujo de información se conserva:
\[
\dot{\mathcal{I}} = \dot{I}^{X} = -\dot{I}^{Y},
\]
y la producción de entropía local de cada subsistema se puede escribir como:


\begin{align*}
    \dot{\sigma}^{X} & = \dot{S}_{r}^{X} - \dot{\mathcal{I}} \geq 0 \\
    \dot{\sigma}^{Y} & =  \dot{S}_{r}^{Y} + \dot{\mathcal{I}} \geq 0 . 
\end{align*}

Supongamos el caso en que \( \dot{\mathcal{I}} > 0 \). En esta situación, se puede interpretar que el subsistema \( X \) está actuando como un sensor, al monitorear activamente al subsistema \( Y \). Para realizar esta función de monitoreo, el subsistema \( X \) debe disipar una cantidad mínima de energía tal que:
\[
\dot{S}_{r}^{X} \geq \dot{\mathcal{I}}.
\]

Por otro lado, el subsistema \( Y \) está proporcionando información, la cual puede ser aprovechada para extraer energía del entorno. En este caso, se cumple que:
\[
-\dot{S}_{r}^{Y} \leq \dot{\mathcal{I}}.
\]

Por lo tanto, esta desigualdad permite que bajo ciertos parámetros $-\dot{S}^{Y}_{r}>0$, es decir, que fluya calor del baño a $Y$ y así extraer trabajo. De esta manera, $\dot{\mathcal{I}}$ corresponde a una cota del trabajo que puede ser extraído, mediante el procesamiento de información.

En sistemas fuera del equilibrio, se puede utilizar información para realizar trabajo útil, o incluso para extraer calor de un reservorio caliente.

Un ejemplo característico de este fenómeno se encuentra en sistemas sometidos a un gradiente de potencial, donde el Demonio de Maxwell se manifiesta a través del surgimiento de una corriente de partículas en contra del gradiente. Este efecto será analizado en detalle en las secciones posteriores.


\section{Descripción cuántica}
Para describir los flujos de información en un contexto cuántico, es necesario conocer la dinámica de un sistema cuántico abierto acoplado a uno o varios reservorios. Para ello, se utilizará la matriz densidad, la cual describe la evolución del sistema reducido. El sistema total estará descrito por el Hamiltoniano $\hat{H} = \hat{H}_{S} + \hat{H}_{B} + \hat{H}_{I}$. Y la evolución markoviana de este sistema conduce a una ecuación maestra

\begin{equation*}
    \partial_{t}\hat{\rho}_{S}(t) = - i[\hat{H}_{S} + \hat{H}_{LS},\hat{\rho}_{S}] + \mathcal{L}(\hat{\rho}_{S}),
\end{equation*}

la cual en el caso de esta tesis, estará descrita por \ref{eqsec2:lindbladS} o \ref{sec2schrodingerthermo}. El acoplamiento del sistema con múltiples reservorios $\alpha$ se modela a través de un generador de Lindblad, que dependiendo si la ecuación maestra es global o semilocal, este se descompone como
\begin{equation*}
    \mathcal{L} = \sum_{\alpha} \mathcal{L}_{\alpha,g},
\end{equation*}

o bien

\begin{equation*}
     \mathcal{L} = \sum_{\alpha} \mathcal{L}_{\alpha},
\end{equation*}
donde cada superoperador $\mathcal{L}_{\alpha,g}$ o $\mathcal{L}_{\alpha}$, representa la interacción del sistema con el reservorio $\alpha$. 



Si se asume que cada reservorio induce un equilibrio local, este puede describirse mediante un estado de Gibbs. En los casos de acoplamientos globales o semilocales, dichos estados están dados, respectivamente, por:
\begin{equation*}
    \hat{\rho}_{\text{eq},g}^{\alpha} = \frac{1}{Z_{\beta_\alpha, \mu_\alpha;g}} e^{-\beta_\alpha (\hat{H}_S - \mu_\alpha \hat{N}_{S})} \hspace{18mm}  \hat{\rho}_{\text{eq}}^{\alpha} = \frac{1}{Z_{\beta_\alpha, \mu_\alpha}} e^{-\beta_\alpha (\hat{H}_{TD} - \mu_\alpha \hat{N}_{S})}, 
\end{equation*}
utilizando el estado de Gibbs correspondiente, es posible aplicar la desigualdad de Spohn \cite{spohn1978entropy}, para obtener una formulación generalizada de la segunda ley de la termodinámica, en términos de una desigualdad de Clausius para la tasa de producción de entropía local asociada al reservorio $\alpha$, es decir, para el caso global

\begin{equation}
    - \text{Tr}[ (\mathcal{L}_{\alpha,g} \hat{\rho}_{S})(\ln \hat{\rho}_{S} - \ln \hat{\rho}^{\alpha}_{eq,g} )  ] \geq 0.
\label{spohninfo}
\end{equation}

Mientras que, para el caso semilocal

\begin{equation}
    - \text{Tr}[ (\mathcal{L}_{\alpha} \hat{\rho}_{S})(\ln \hat{\rho}_{S} - \ln \hat{\rho}^{\alpha}_{eq} )  ] \geq 0.
\label{spohninfolocal}
\end{equation}

En los casos de ecuación maestra global o semilocal, se puede definir la siguiente cantidad:

\begin{equation}
    \dot{S}^{\alpha,g} = - \mathrm{Tr} \left\{ (\mathcal{L}_{\alpha,g} \hat{\rho}_S) \ln \hat{\rho}_S \right\}, \hspace{10mm}
    \dot{S}^{\alpha} = - \mathrm{Tr} \left\{ (\mathcal{L}_{\alpha} \hat{\rho}_S) \ln \hat{\rho}_S \right\},
\end{equation}
la cual representa la tasa de cambio de la entropía de von Neumann del sistema inducida por el acoplamiento al reservorio $\alpha$. La entropía de von Neumann está dada por:

\begin{equation}
    S = - \mathrm{Tr} \left\{ \hat{\rho}_S \ln \hat{\rho}_S \right\}.
\end{equation}

De las desigualdades de Spohn (ver ecuación~\ref{spohninfo} y \ref{spohninfolocal}) se identifica el flujo de calor inducido por el reservorio $\alpha$ como:

\begin{equation*}
    J_{\alpha,g} =  \mathrm{Tr} \left\{ (\mathcal{L}_{\alpha,g} \hat{\rho}_S) (\hat{H}_{S}-\mu_{\alpha}\hat{N}_{S}) \right\} \hspace{15mm}
    J_{\alpha} = \mathrm{Tr} \left\{ (\mathcal{L}_{\alpha} \hat{\rho}_S) (\hat{H}_{TD}-\mu_{\alpha}\hat{N}_{S}) \right\}. 
\end{equation*}

De este modo, se obtienen expresiones diferenciadas para la tasa de producción de entropía local asociada al reservorio $\alpha$, correspondientes a los casos de acoplamiento global y semilocal, respectivamente:

\begin{equation}
    \dot{\sigma}^{\alpha,g} = \dot{S}^{\alpha,g} - \beta_{\alpha} J_{\alpha,g} \geq 0 \hspace{15mm}
    \dot{\sigma}^{\alpha} = \dot{S}^{\alpha} - \beta_{\alpha} J_{\alpha} \geq 0. 
    \label{sec4:localentropy}
\end{equation}

Esta desigualdad representa una forma local de la segunda ley de la termodinámica, en la que el término $J_\alpha$ (o $J_{\alpha,g}$ en el caso de acoplamiento global) corresponde al flujo de calor desde el reservorio $\alpha$ hacia el sistema, y $\beta_\alpha$ es la inversa de su temperatura. Esta formulación permite descomponer localmente la producción de entropía en contribuciones individualmente positivas, de manera análoga a lo que ocurre en el marco clásico. De hecho, al sumar todas las tasas de cambio de entropía, se recupera la derivada temporal total de la entropía, que para el caso global y semilocal es, respectivamente: 

\begin{equation*}
    \partial_t S_{g} = \sum_\alpha \dot{S}^{\alpha,g} \hspace{15mm}
    \partial_t S = \sum_\alpha \dot{S}^\alpha.
\end{equation*}    

Esto implica que la suma de las producciones locales de entropía es

\begin{equation*}
    \sum_{\alpha}\dot{\sigma}^{\alpha,g} = \partial_{t}S_{g} - \sum_{\alpha}\beta_{\alpha}J_{\alpha,g} = \dot{\sigma}_{g} \geq 0 \hspace{15mm}
    \sum_{\alpha}\dot{\sigma}^{\alpha} =\partial_{t}S  -\sum_{\alpha}\beta_{\alpha}J_{\alpha} = \dot{\sigma} \geq 0, 
\end{equation*}
que corresponde a la inecuación de Clausius estándar.

Se puede notar que para el caso estacionario $\partial_{t}S = \partial_{t}S_{g} = 0$, no necesariamente $\dot{S}^{\alpha}$ y $\dot{S}^{\alpha,g}$ son cero. Sino que dependen de los flujos de calor que entran localmente por los disipadores $\alpha$. También se puede tratar la inecuación para la energía libre, primero definiendo la corriente de energía y potencia correspondientes al baño $\alpha$

\begin{align*}
    \dot{E}_{\alpha,g} & = \text{Tr}[ (\mathcal{L}_{\alpha,g} \hat{\rho}_{S}) \hat{H}_{S}]   & \dot{E}_{\alpha} = \text{Tr}[ (\mathcal{L}_{\alpha} \hat{\rho}_{S}) \hat{H}_{TD}]   \\
    \dot{W}_{\alpha,g} & = \mu_{\alpha}\text{Tr}[ (\mathcal{L}_{\alpha,g} \hat{\rho}_{S}) \hat{N}_{S}] &  \dot{W}_{\alpha} = \mu_{\alpha}\text{Tr}[ (\mathcal{L}_{\alpha} \hat{\rho}_{S}) \hat{N}_{S}].
\end{align*}

Para ambas ecuaciones maestras(global y semilocal), se cumple que la suma de las corrientes de energía provenientes de los distintos reservorios, corresponden a la derivada temporal de la energía interna del sistema

\begin{equation*}
    \sum_{\alpha}\dot{E}_{\alpha,g} = \partial_{t}E_{g} \hspace{15mm} \sum_{\alpha}\dot{E}_{\alpha} = \partial_{t}E.
\end{equation*}

En el caso de un Hamiltoniano independiente del tiempo, no se realiza trabajo mecánico, por lo que sólo se considera trabajo químico. En consecuencia, la energía intercambiada con cada reservorio se descompone como

\begin{equation*}
    \dot{E}_{\alpha,g} = J_{\alpha,g} + \dot{W}_{\alpha,g} \hspace{15mm} \dot{E}_{\alpha} = J_{\alpha} + \dot{W}_{\alpha}. 
\end{equation*}

Multiplicando las ecuaciones en \ref{sec4:localentropy} por la temperatura del reservorio $T_{\alpha}$ y reemplazando los flujos de calor mediante la expresiones anteriores, se obtiene

\begin{equation}
    T_{\alpha} \dot{\sigma}^{\alpha,g} = \dot{W}_{\alpha,g} - \dot{\mathcal{F}}_{\alpha,g} \geq 0 \hspace{10mm}  
    T_{\alpha} \dot{\sigma}^{\alpha} = \dot{W}_{\alpha} - \dot{\mathcal{F}}_{\alpha} \geq 0,              
\label{sec4:localfreerate}
\end{equation}
donde se ha definido la tasa de variación de la energía libre asociado al reservorio $\alpha$ para el caso global y semilocal, como

\begin{equation*}
    \dot{\mathcal{F}}_{\alpha,g} = \dot{E}_{\alpha,g} - T_{\alpha}\dot{S}^{\alpha,g} \hspace{10mm}
    \dot{\mathcal{F}}_{\alpha} = \dot{E}_{\alpha} - T_{\alpha}\dot{S}^{\alpha}.
\end{equation*}

La desigualdades en \ref{sec4:localfreerate} consisten en una inecuaciones locales de energía libre correspondientes al reservorio $\alpha$. Si se suman las inecuaciones de todos los reservorios, para el caso global y semilocal se tendrá 

\begin{equation}
    \sum_{\alpha}T_{\alpha} \dot{\sigma}^{\alpha,g} = \dot{W}_{g} - \dot{\mathcal{F}}_{g} \geq 0 \hspace{10mm}
    \sum_{\alpha}T_{\alpha} \dot{\sigma}^{\alpha} = \dot{W} - \dot{\mathcal{F}} \geq 0,
\label{sec4:freeratefinal}
\end{equation}

donde $\dot{W}_{g} = \sum_{\alpha}\dot{W}_{\alpha,g}$, $\dot{W} = \sum_{\alpha}\dot{W}_{\alpha}$  y las tasas de cambio de energía libre total son

\begin{equation*}
    \dot{\mathcal{F}}_{g} = \partial_{t}E_{g} - \sum_{\alpha}T_{\alpha} \dot{S}^{\alpha,g} \hspace{10mm}
    \dot{\mathcal{F}} = \partial_{t}E - \sum_{\alpha}T_{\alpha} \dot{S}^{\alpha}.
\end{equation*}

Para el caso particular, en qué todas las temperaturas son iguales, las tasas de cambio de energía libre $\dot{\mathcal{F}}$ y $\dot{\mathcal{F}}_{g}$ se vuelven equivalentes a la derivadas temporales de la energía libre $\partial_{t}F = \partial_{t}(E-TS)$ y $\partial_{t}F_{g} = \partial_{t}(E_{g}-TS_{g})$. Sólo en este caso \ref{sec4:localentropy} y \ref{sec4:freeratefinal} son equivalentes.

En el caso isotérmico, se cumple que en el estado estacionario $\dot{\mathcal{F}} = \partial_{t}F = 0$ y $\dot{\mathcal{F}}_{g} = \partial_{t}F_{g} = 0$, lo cual implica que $\dot{W} > 0$ y $\dot{W}_{g} > 0$. Por lo tanto, el sistema no puede realizar trabajo neto si solo intercambia calor con reservorios a la misma temperatura, independientemente de si la evolución es global o semilocal. La única forma de obtener trabajo neto del sistema —es decir, lograr que $\dot{W} < 0$ en el caso semilocal, o $\dot{W}_{g} < 0$ en el caso global— es mediante la presencia de reservorios a distintas temperaturas. En tal situación, las tasas de cambio de la energía libre ya no se anulan, es decir, $\dot{\mathcal{F}} = - \sum_{\alpha} T_{\alpha} \dot{S}^{\alpha} \neq 0$ y $\dot{\mathcal{F}}_{g} = - \sum_{\alpha} T_{\alpha} \dot{S}^{\alpha,g} \neq 0$.

\section{Flujos de información en contexto cuántico}
En esta sección, el cálculo de los flujos de información se enfocará principalmente en el caso de evolución semilocal, ya que es el marco que se desarrollará en el resto de esta tesis. Para el caso de evolución global, los flujos de información pueden definirse de manera análoga, empleando las cantidades termodinámicas correspondientes a dicha evolución.

Supóngase que el sistema $S$ está compuesto por dos subsistemas acoplados. En tal caso, su dinámica estará gobernada por un Hamiltoniano de la forma:

\begin{equation*}
    \hat{H}_{S} = \hat{H}_{X} + \hat{H}_{Y} + \hat{H}_{XY}, 
\end{equation*}
donde $\hat{H}_{i}$ representa el Hamiltoniano del subsistema $i=X,Y$, y $\hat{H}_{XY}$ corresponde al término de interacción entre los dos subsistemas. Se asume, además, que cada subsistema está acoplado a un conjunto distinto de reservorios. En particular, los reservorios asociados al subsistema $i$ serán denotados por $\alpha_{i}$. Esto permite describir la tasa de producción de entropía local de cada subsistema como

\begin{equation*}
    \dot{\sigma}^{i} \equiv \sum_{\alpha_{i}} \dot{\sigma}^{\alpha_{i}} = \sum_{\alpha_{i}} \dot{S}^{\alpha_{i}} - \sum_{\alpha_{i}} \beta_{\alpha_{i}} J_{\alpha_{i}} \geq 0,
\end{equation*}
por lo tanto, la tasa de producción de entropía total del sistema corresponde a $\dot{\sigma} = \dot{\sigma}^{X} + \dot{\sigma}^{Y}$. 

De forma análoga a lo realizado en la descripción clásica, se busca establecer una relación entre las producciones locales de entropía y la información compartida entre subsistemas. Para ello, se utilizará la información mutua entre los dos subsistemas, definida como
 
\begin{equation*}
    I_{XY} = S_{X} + S_{Y} - S_{XY},
\end{equation*}
donde $S_{XY}$ es la entropía de von Neumann total, dada por $S_{XY} = -\mathrm{Tr}\{ \hat{\rho}_{S} \ln \hat{\rho}_{S} \}$, y $S_{i} = - \mathrm{Tr}\{ \hat{\rho}_{i} \ln \hat{\rho}_{i} \}$ corresponde a la entropía de von Neumann del subsistema $i = X, Y$. Aquí, $\hat{\rho}_{i}$ representa la matriz densidad reducida del subsistema $i$.

Se puede separar la derivada temporal de la información mutua

\begin{align*}
    \partial_{t}I_{XY} & = \partial_{t}S_{X} + \partial_{t}S_{Y} - \partial_{t}S_{XY} \\
        & = \partial_{t}S_{X} + \partial_{t}S_{Y} - \sum_{i=X,Y;\alpha_{i}}\dot{S}^{\alpha_{i}} \\
        & =  \partial_{t}S_{X} - \sum_{\alpha_{X}} \dot{S}^{\alpha_{X}} + \partial_{t}S_{Y} - \sum_{\alpha_{Y}} \dot{S}^{\alpha_{Y}} \\
        & = \dot{I}^{X} + \dot{I}^{Y},
\end{align*}
donde aparece el flujo de información correspondiente a cada subsistema $i = X, Y$, definido como la cantidad $\dot{I}^{i} = \partial_{t} S^{i} - \sum_{\alpha_{i}} \dot{S}^{\alpha_{i}}$. Si se reemplaza $\sum_{\alpha_{i}} \dot{S}^{\alpha_{i}}$ por $\partial_{t} S^{i} - \dot{I}^{i}$ en las expresiones para las tasas de producción de entropía locales, se obtienen las siguientes inecuaciones:

\begin{align}
    \dot{\sigma}^{X} = \partial_{t}S^{X} - \sum_{\alpha_{X}} \beta_{\alpha_{X}} J_{\alpha_{X}} - \dot{I}^{X} \geq 0 \\
    \dot{\sigma}^{Y} = \partial_{t}S^{Y} - \sum_{\alpha_{Y}} \beta_{\alpha_{Y}} J_{\alpha_{Y}} - \dot{I}^{Y} \geq 0,
    \label{sec4:localentropyquantum}
\end{align}
este resultado coincide con el obtenido en la descripción clásica. No obstante, su formulación en términos de entropía de von Neumann permite incorporar de forma natural los efectos cuánticos presentes en el sistema, lo cual resulta fundamental para caracterizar los flujos de información en regímenes donde las correlaciones cuánticas y las coherencias desempeñan un rol relevante.

Si se considera que el subsistema $i$ esta conectado a a un reservorio isotermico de temperatura $T_i$, la tasa de cambio de energía libre será
\begin{align*}
    \dot{\mathcal{F}}_{i} & = \dot{E}_{i} - T_{i} \sum_{\alpha_{i}}\dot{S}^{\alpha_{i}}. 
\end{align*}

En el estado estacionario se cumple la relación

\begin{equation*}
    \dot{I}^{i} = - \sum_{\alpha_{i}}\dot{S}^{\alpha_{i}},
\end{equation*}
y la tasa de cambio de energía libre del subsistema $i$ es 

\begin{equation*}
    \dot{\mathcal{F}}_{i} = \dot{E}_{i} + T_{i} \dot{I}^{i}.
\end{equation*}

De modo que, la tasa de cambio de la energía libre posee dos contribuciones: una correspondiente al intercambio de energía y otra asociada al flujo de información. Por ende, la potencia correspondiente al subsistema $i$ se encuentra acotada por

\begin{equation*}
    \dot{W}_{i} \geq \dot{E}_{i} + T_{i}\dot{I}_{i}.
\end{equation*}

La importancia de esta cota radica, en que la capacidad de que el subsistema $i$ realize un trabajo externo depende del flujo de información que entrega el sistema.

Si se considera el régimen estacionario, donde la energía total permanece constante, es decir,
\begin{equation*}
    \partial_{t}E = \dot{E}^{X} + \dot{E}^{Y} = 0,
\end{equation*}
entonces los términos $\dot{E}^{i}$ pueden interpretarse como flujos de energía entre los dos subsistemas. En este contexto, la tasa de cambio de energía libre para cada subsistema se descompone en una parte asociada al intercambio energético y otra vinculada al flujo de información generado por las correlaciones entre los subsistemas.

En el caso $\dot{E}^{X} = \dot{E}^{Y} \approx 0 $, la cota para la potencia $\dot{W}_{i}$ se reduce a 

\begin{equation*}
     \dot{W}_{i} \geq T_{i}\dot{I}_{i}.
\end{equation*}

Si se tiene $\dot{I}_{i}<0$, esta cota permite que el sistema pueda realizar trabajo a partir de una contribución dominante de la información, incluso en ausencia de diferencias térmicas significativas. Tal mecanismo constituye lo que se conoce como una máquina de la información, en la que los flujos de información entre subsistemas juegan un rol principal, permitiendo la extracción de trabajo en un régimen puramente informacional.


\label{sec4:flujos0}

\chapter{Dinámica de un sistema de 3 puntos cuánticos conectado al ambiente}
En este capítulo se estudia un sistema compuesto por tres puntos cuánticos acoplados a múltiples reservorios térmicos. La evolución del sistema se analiza mediante la ecuación de Lindblad desarrollada en la sección \ref{sec2lindbladconsistency}. En la sección \ref{sec5:modelo} se introduce el Hamiltoniano del sistema reducido y su entorno. Posteriormente, en la sección \ref{sec5:transporte}, se analiza el comportamiento del sistema en un escenario típico de transporte cuántico. En la sección \ref{sec5:demonio1}, se establecen las condiciones bajo las cuales la dinámica del sistema puede interpretarse como la de un Demonio de Maxwell autónomo. Finalmente, en la sección \ref{sec5:demonio}, se compara este comportamiento con su contraparte clásica, destacando las diferencias inducidas por los efectos cuánticos.

\section{Modelo teórico}
El sistema reducido está compuesto por tres puntos cuánticos. Uno de ellos posee un nivel de energía $\epsilon_D$ y se encuentra acoplado, mediante una interacción de Coulomb $U$, a los otros dos puntos cuánticos. Estos últimos tienen una energía de sitio $\epsilon$ y están conectados entre sí por un término de \textit{hopping} $g$, además de una interacción de Coulomb $U_f$. Cada uno de los niveles de energía se encuentra acoplado a un baño distinto. 

La configuración del sistema reducido se ilustra en la Figura~\ref{img:sistema3puntos}.


\insertimage[\label{img:sistema3puntos}]{ejemplos/semilocal.pdf}{scale=0.36}{Esquema que representa el sistema reducido compuesto por tres puntos cuánticos, modelados como tres niveles de energía. Estos niveles presentan interacción de Coulomb repulsiva entre ellos, y cada uno está acoplado a un baño térmico distinto.}

El Hamiltoniano del sistema reducido esta dado por 

\begin{align*}
    \hat{H}_{S} & = \epsilon_{D}\hat{d}^{\dagger}_{D}\hat{d}_{D} + \epsilon \hat{d}^{\dagger}_{L}\hat{d}_{L} + \epsilon \hat{d}^{\dagger}_{R}\hat{d}_{R} + g(\hat{d}^{\dagger}_{L}\hat{d}_{R} + \hat{d}^{\dagger}_{R}\hat{d}_{L} ) \\
          & + U(\hat{n}_{D}\hat{n}_{L} + \hat{n}_{D}\hat{n}_{R} )  + U_{f}\hat{n}_{R}\hat{n}_{L},
\end{align*}
donde $\hat{n}_{i} = \hat{d}^{\dagger}_{i} \hat{d}_{i}$ es el operador de número del sitio $i$, y los operadores $\hat{d}_{i}$ satisfacen las relaciones de anticonmutación

\begin{equation*}
    \{\hat{d}_{\alpha},\hat{d}_{\beta} \} = 0  \hspace{10mm} \{\hat{d}^{\dagger}_{\alpha}, \hat{d}_{\beta} \} = \delta_{\alpha \beta}.
\end{equation*}

Usando la base de Fock $|n_{L},n_{R},n_{D} \rangle = (\hat{d}^{\dagger}_{L})^{n_{L}}(\hat{d}^{\dagger}_{R})^{n_{R}}(\hat{d}^{\dagger}_{D})^{n_{D}}|0,0,0\rangle$, y escribiendo el Hamiltoniano en el orden de la base 

\begin{equation*}
     \{|0,0,0\rangle, |1,0,0\rangle, |0,1,0\rangle, |0,0,1\rangle, |1,1,0\rangle, |1,0,1\rangle, |0,1,1\rangle, |1,1,1\rangle\},
\end{equation*}
se puede representar el Hamiltoniano del sistema en forma matricial 

\begin{equation*}
    \hat{H}_{S} = 
    \begin{bmatrix}
        0 & 0 & 0 & 0 & 0 & 0 & 0 & 0 \\
        0 & \epsilon_{L} & g & 0 & 0 & 0 & 0 & 0 \\
        0 & g & \epsilon_{R} & 0 & 0 & 0 & 0 & 0 \\
        0 & 0 & 0 & \epsilon_{D} & 0 & 0 & 0 & 0 \\
        0 & 0 & 0 & 0 & \epsilon_{L} + \epsilon_{R}  + U_{f} & 0 & 0 & 0 \\
        0 & 0 & 0 & 0 & 0 & \epsilon_{L} + \epsilon_{D} + U & g & 0 \\
        0 & 0 & 0 & 0 & 0 & g & \epsilon_{R} + \epsilon_{D} + U & 0 \\
        0 & 0 & 0 & 0 & 0 & 0 & 0 & \epsilon_{L} + \epsilon_{R}  + \epsilon_{D} + 2U + U_{f} 
        \end{bmatrix}.
\end{equation*}

Por otro lado, el Hamiltoniano del baño y la interacción con el sistema se descomponen como $\hat{H}_{B} = \hat{H}_{L}+\hat{H}_{R}+\hat{H}_{D}$ y $\hat{V} = \hat{V}_{L}+\hat{V}_{R}+\hat{V}_{D}$ respectivamente. Cada término esta dado por 

\begin{equation*}
    \hat{H}_{\alpha} = \sum_{l} \epsilon_{\alpha l} \hat{c}^{\dagger}_{\alpha l}\hat{c}_{\alpha l} \hspace{10mm} \hat{V}_{\alpha} = \sum_{l} t_{\alpha l} ( \hat{d}^{\dagger}_{\alpha} \hat{c}_{\alpha l} + \hat{c}^{\dagger}_{\alpha l} \hat{d}_{\alpha} ),
\end{equation*}
donde $\alpha = L,R,D$ denota los tres reservorios acoplados a cada punto cuántico, $\hat{c}_{\alpha,l}$ son los operadores de aniquilación de los modos del reservorio $\alpha$, $\epsilon_{\alpha l}$ sus energías, y $t_{\alpha l}$ los parámetros reales que describen el acoplamiento entre el sistema y baño. 

A partir del término de interacción $\hat{V}$, es posible identificar los operadores asociados al sistema y al baño

\begin{equation*}
    \hat{S}_{\alpha,1} = \hat{d}^{\dagger}_{\alpha} \hspace{10mm} \hat{S}_{\alpha,-1} = \hat{d}_{\alpha}
\end{equation*}

\begin{equation*}
    \hat{B}_{\alpha,1} = \sum_{l}t_{\alpha l} \hat{c}_{\alpha l} \hspace{10mm} \hat{B}_{\alpha,-1} = \sum_{l}t_{\alpha l} \hat{c}^{\dagger}_{\alpha l}.
\end{equation*}

La evolución no unitaria del sistema de tres puntos cuánticos, inducida por su acoplamiento a los baños, se describe mediante una ecuación maestra descrita por el operador de Lindblad

\begin{equation}
    \mathcal{L} = \mathcal{L}_{R} + \mathcal{L}_{L} + \mathcal{L}_{D},
    \label{Lindbladsec5}
\end{equation}
donde el operador de Lindblad $\mathcal{L}$ se descompone como una suma de contribuciones individuales, una por cada baño acoplado al sistema. Explícitamente cada contribución $\mathcal{L}_{i}$ corresponde a 

\begin{align*}
    \mathcal{L}_{R} & = \kappa_{R}(\epsilon)(f_{R}(\epsilon)\mathcal{D}_{0}[\hat{d}^{\dagger}_{R}(\textbf{1}-\hat{n}_{D})(\textbf{1}-\hat{n}_{L}) ]  + [1-f_{R}(\epsilon)]\mathcal{D}_{0}[\hat{d}_{R}(\textbf{1}-\hat{n}_{D})(\textbf{1}-\hat{n}_{L}) ]  )  \\
                    & + \kappa_{R}(\epsilon+U)(f_{R}(\epsilon+U)\mathcal{D}_{0}[\hat{d}^{\dagger}_{R}\hat{n}_{D}(\textbf{1}-\hat{n}_{L}) ]  + [1-f_{R}(\epsilon+U)]\mathcal{D}_{0}[\hat{d}_{R}\hat{n}_{D}(\textbf{1}-\hat{n}_{L}) ]  ) \\
                   & + \kappa_{R}(\epsilon+U_{f})(f_{R}(\epsilon+U_{f})\mathcal{D}_{0}[\hat{d}^{\dagger}_{R}(\textbf{1}-\hat{n}_{D})\hat{n}_{L} ]  + [1-f_{R}(\epsilon+U_{f})]\mathcal{D}_{0}[\hat{d}_{R}(\textbf{1}-\hat{n}_{D})\hat{n}_{L} ]  ) \\
                  & + \kappa_{R}(\epsilon+U+U_{f})(f_{R}(\epsilon+U+U_{f})\mathcal{D}_{0}[\hat{d}^{\dagger}_{R}\hat{n}_{D}\hat{n}_{L} ]  + [1-f_{R}(\epsilon+U+U_{f})]\mathcal{D}_{0}[\hat{d}_{R}\hat{n}_{D}\hat{n}_{L} ]  ) 
\end{align*}

\begin{align*}
    \mathcal{L}_{L} & = \kappa_{L}(\epsilon)(f_{L}(\epsilon)\mathcal{D}_{0}[\hat{d}^{\dagger}_{L}(\textbf{1}-\hat{n}_{D})(\textbf{1}-\hat{n}_{R}) ]  + [1-f_{L}(\epsilon)]\mathcal{D}_{0}[\hat{d}_{L}(\textbf{1}-\hat{n}_{D})(\textbf{1}-\hat{n}_{R}) ]  )  \\
                    & + \kappa_{L}(\epsilon+U)(f_{L}(\epsilon+U)\mathcal{D}_{0}[\hat{d}^{\dagger}_{L}\hat{n}_{D}(\textbf{1}-\hat{n}_{R}) ]  + [1-f_{L}(\epsilon+U)]\mathcal{D}_{0}[\hat{d}_{L}\hat{n}_{D}(\textbf{1}-\hat{n}_{R}) ]  ) \\
                   & + \kappa_{L}(\epsilon+U_{f})(f_{L}(\epsilon+U_{f})\mathcal{D}_{0}[\hat{d}^{\dagger}_{L}(\textbf{1}-\hat{n}_{D})\hat{n}_{R} ]  + [1-f_{L}(\epsilon+U_{f})]\mathcal{D}_{0}[\hat{d}_{L}(\textbf{1}-\hat{n}_{D})\hat{n}_{R} ]  ) \\
                  & + \kappa_{L}(\epsilon+U+U_{f})(f_{L}(\epsilon+U+U_{f})\mathcal{D}_{0}[\hat{d}^{\dagger}_{L}\hat{n}_{D}\hat{n}_{R} ]  + [1-f_{L}(\epsilon+U+U_{f})]\mathcal{D}_{0}[\hat{d}_{L}\hat{n}_{D}\hat{n}_{R} ]  ) 
\end{align*}

\begin{align*}
    \mathcal{L}_{D} & = \kappa_{D}(\epsilon_{D})(f_{D}(\epsilon_{D})\mathcal{D}_{0}[\hat{d}^{\dagger}_{D}(\textbf{1}-\hat{n}_{R})(\textbf{1}-\hat{n}_{L}) ]  + [1-f_{D}(\epsilon_{D})]\mathcal{D}_{0}[\hat{d}_{R}(\textbf{1}-\hat{n}_{D})(\textbf{1}-\hat{n}_{L}) ]  )  \\
                    & + \kappa_{D}(\epsilon_{D}+U)f_{D}(\epsilon_{D}+U)\mathcal{D}_{0}[\hat{d}^{\dagger}_{D}(\hat{n}_{R}(\textbf{1}-\hat{n}_{L}) + \hat{n}_{L}(\textbf{1}-\hat{n}_{R})) ]  \\
                    & + \kappa_{D}(\epsilon_{D}+U)[1-f_{D}(\epsilon_{D}+U)]\mathcal{D}_{0}[\hat{d}_{D}(\hat{n}_{R}(\textbf{1}-\hat{n}_{L}) + \hat{n}_{L}(\textbf{1}-\hat{n}_{R}))]   \\
                   & + \kappa_{D}(\epsilon_{D}+2U)(f_{D}(\epsilon_{D}+2U)\mathcal{D}_{0}[\hat{d}^{\dagger}_{D}\hat{n}_{R}\hat{n}_{L} ]  + [1-f_{D}(\epsilon_{D}+2U)]\mathcal{D}_{0}[\hat{d}_{D}\hat{n}_{R}\hat{n}_{L} ]  ).
\end{align*}

Con $f_{i}(\omega) = (\exp(\beta_{i}(\omega - \mu_{i})) + 1)^{-1}$ la distribución de Fermi y  las tasas de túnel $\kappa_{\alpha}(\omega)= 2\pi \sum_{l}t^{2}_{\alpha,l}\delta(\omega-\epsilon_{\alpha,l})$, las cuales provienen de calcular las funciones correlación espectral $\gamma$ en \ref{sec2:gammaanddelta}. Además, el operador $\mathcal{D}_{0}$ es 

\begin{equation*}
    \mathcal{D}_{0}[\hat{A}]\hat{\rho} = \hat{A}\hat{\rho}\hat{A}^{\dagger}- \frac{1}{2} \{\hat{A }^{\dagger}\hat{A},\hat{\rho} \}.
\end{equation*}

La demostración de la ecuación \ref{Lindbladsec5} se presenta en los apéndices \ref{apendix5jumpop} y \ref{apendix5frecuencygroup}.

Despreciando el término de \textit{Lamb Shift} $\hat{H}_{LS}$, como se discute en~\cite{prech2023entanglement}, la evolución del sistema de tres puntos cuánticos está dada por
\begin{equation}
    \frac{d}{dt}\hat{\rho}_{S}(t) = -i[\hat{H}_{S},\hat{\rho}_{S}(t)] + \mathcal{L}(\hat{\rho}_{S}(t)),
\end{equation}
donde $\hat{\rho}_{S}(t)$ es la matriz densidad del sistema reducido. A partir de esta ecuación, es posible calcular numéricamente la evolución del sistema y obtener las magnitudes termodinámicas relevantes.

Un aspecto importante a destacar es que los disipadores $\mathcal{L}_i$ en la ecuación maestra semilocal están formulados en función de operadores locales del sitio, lo que permite describir la evolución del sistema a partir de dinámicas locales. En contraste, la ecuación maestra global describe la evolución del sistema utilizando operadores globales, lo que dificulta el análisis detallado de la dinámica. En este sentido, la descripción semilocal resulta más conveniente, ya que permite distinguir las cantidades termodinámicas asociadas a cada subsistema, en particular los flujos de información. Es de especial interés determinar bajo qué condiciones un Hamiltoniano $\hat{H}_S$ permite que los operadores locales se mantengan como tales tras la agrupación de frecuencias. Un caso en el que dicha agrupación no conduce a disipadores locales ocurre, por ejemplo, cuando las interacciones de tipo Coulomb entre los diferentes sitios son todas distintas.

\label{sec5:modelo}

\newpage

\section{Estudio del transporte}
Se aborda inicialmente el caso en que las tasas de túnel no dependen de la frecuencia, es decir, $\kappa_{i}(\omega) = \kappa_{i}$. Y se estudia el flujo de corriente $\dot{N}_{i} = \text{Tr}\{(\mathcal{L}_{i}\hat{\rho}_{S})\hat{N}_{S}\}$ generado por una diferencia de potencial $\mu_{L} - \mu_{R} = eV$, empleando las tasas de túnel $\kappa_{L} = \kappa_{R} = 1/100$ y $\kappa_{D} = 1/50$. Los resultados se muestran en la Figura~\ref{img:transporte1}.

\insertimage[\label{img:transporte1}]{ejemplos/particletp.pdf}{scale=0.40}{Corriente de partículas en función de la razón \( eV/T \). Las temperaturas de los reservorios son \( T_{L} = T_{R} = 100 \) y \( T_{D} = 2 \). El parámetro de acoplamiento se fija en \( g = \frac{5}{1000} \). Las energías de sitio corresponden a \( \epsilon = 0 \) y \( \epsilon_{D} = \mu_{D} - \frac{U}{2} \), mientras que las interacciones de Coulomb se fijan en \( U_{f} = 500 \) y \( U = 40 \).}

La figura muestra que, a medida que aumenta la diferencia de potencial \( eV \), la corriente de partículas \( \dot{N}_L = -\dot{N}_R \) que fluye del baño \( L \) al baño \( R \) también aumenta. Este comportamiento es consistente con lo esperado, ya que un mayor potencial químico en el reservorio \( L \) impulsa el transporte hacia \( R \). Asimismo, se puede analizar la potencia suministrada al sistema, dada por \ref{sec2:heatworkeq}, en función de \( eV \).

\insertimage[\label{img:transporte2}]{ejemplos/worktp.pdf}{scale =0.40}{Potencia entregada por los reservorios en función de $eV/T$.}
    
La Figura \ref{img:transporte2} muestra que el conjunto de reservorios \( L \) y \( R \) entrega trabajo al sistema para mantener la corriente de partículas. Este comportamiento es consistente desde el punto de vista físico, ya que el paso de electrones desde un potencial químico mayor hacia uno menor implica una transferencia de energía al sistema, que se manifiesta como trabajo. Por otro lado, dado que el reservorio \( D \) no intercambia partículas en promedio, la potencia que este entrega es nula.

 A continuación, pueden examinarse los flujos de calor \ref{sec2:heatworkeq} que reciben los sistemas $LR$ y $D$

\insertimage[\label{img:transporte3}]{ejemplos/heatlrdtp.pdf}{scale =0.40}{Flujo de calor recibido por los sistemas $LR$ y $D$ en función de $eV/T$.}
    
Como uno puede ver, para $eV/T \lessapprox 2$ , el sistema $LR$ en total recibe calor de los reservorios $L$ y $R$. Para $eV/T \gtrapprox 2$ el sistema $LR$ en total entrega calor a los reservorios $L$ y $R$. 

Para completar el análisis termodinámico a partir de la primera ley $\dot{E}_{\alpha} = J_{\alpha} + \dot{W}_{\alpha}$, se grafican a continuación los flujos de energía involucrados, descritos por \ref{sec2:energyeq}

\insertimage[\label{img:transporte4}]{ejemplos/energytp.pdf}{scale =0.43}{Flujo de energía entregado por los reservorios en función de $eV/T$.}

En el estado estacionario, la condición \( \partial_t E = 0 \) implica que \( \dot{E}_{LR} = -\dot{E}_D \), lo cual se corrobora en la Figura \ref{img:transporte4}. En ella, el flujo de energía \( \dot{E}_{LR} \) es positivo, mientras que \( \dot{E}_D \) resulta negativo. Dado que \( \dot{W}_{D} = 0 \), se concluye que el flujo de calor hacia el demonio está dado por \( J_D = \dot{E}_D < 0 \); es decir, el reservorio \( D \) absorbe calor. Por su parte, el sistema \( LR \) recibe energía de los reservorios \( L \) y \( R \), la cual es posteriormente transferida al subsistema \( D \), para finalmente ser absorbida por su respectivo baño térmico. Este comportamiento concuerda con la primera ley de la termodinámica, ya que la energía suministrada por los reservorios de mayor temperatura (\( T_L = T_R > T_D \)) se conserva y es disipada hacia el reservorio de menor temperatura.

Para analizar la presencia de efectos cuánticos en el estado estacionario, se calcularán dos cantidades relevantes. La primera de ellas es la norma-\( l_{1} \), \( \mathcal{C}_{l_{1}} \), que permite cuantificar las coherencias en el sistema \( LR \), y está dada por


\begin{equation*}
    \mathcal{C}_{l_{1}} = \sum_{i> j} |\hat{\rho}_{S;i,j}|.
\end{equation*}

Otro efecto cuántico relevante de estudiar es el entrelazamiento entre modos específicos del sistema. En particular, se analiza el entrelazamiento entre los pares de estados \(|1,0,0\rangle, |0,1,0\rangle\) y \(|1,0,1\rangle, |0,1,1\rangle\). Para cuantificar este tipo de correlaciones no clásicas se utiliza la concurrencia \( \mathcal{C}_{\text{on}} \) \cite{hill1997entanglement, wootters1998entanglement}, definida como

\begin{equation*}
    \mathcal{C}_{\text{on}} = \max \left\{ 2|\alpha + \beta| - 2\sqrt{p_0 p_D},\, 0 \right\},
\end{equation*}
donde \( p_0 \) y \( p_D \) representan las probabilidades de que el sistema \( LR \) se encuentre vacío o doblemente ocupado, respectivamente. El término \( \alpha \) corresponde a la coherencia entre los estados \(|1,0,0\rangle\) y \(|0,1,0\rangle\), mientras que \( \beta \) representa la coherencia entre los estados \(|1,0,1\rangle\) y \(|0,1,1\rangle\). La deducción de esta fórmula se presenta en el Apéndice \ref{appendix5final}. 

Finalmente, en la Figura \ref{img:transporte5} se analiza cómo varían la coherencia y la concurrencia en función de la diferencia de potencial

\insertimage[\label{img:transporte5}]{ejemplos/quantumtp.pdf}{scale =0.4}{Coherencia(línea azul) y concurrencia(linea roja) en función de $eV/T$.}
    
En este caso, la coherencia comienza en cero y aumenta progresivamente a medida que se incrementa la diferencia de potencial. En cuanto al entrelazamiento, también parte desde cero —coincidiendo con el valor nulo de la coherencia— y se mantiene nulo hasta alcanzar un cierto umbral, a partir del cual la concurrencia comienza a crecer con el aumento de \( eV/T \). Esto indica la aparición de entrelazamiento en el sistema \( LR \). Este comportamiento es relevante, ya que sugiere que la presencia de una corriente en el sistema contribuye a que se desarrollen coherencia y entrelazamiento entre los sitios $L$ y $R$.  


\label{sec5:transporte}

\newpage

\section{Demonio de Maxwell en 3 puntos cuánticos}
Para que el sistema de tres puntos cuánticos se comporte como un Demonio de Maxwell autónomo, es fundamental tratarlo como un sistema bipartito compuesto por dos subsistemas: por un lado, el subsistema $LR$, que consiste en dos niveles de energía $\epsilon$; por otro, el nivel $\epsilon_D$, que desempeña el rol de demonio.

Para que este último actúe como demonio, debe ser capaz de detectar cambios en el subsistema $LR$, ejecutando una acción análoga a una medición. Gracias a la interacción de Coulomb $U$, el demonio puede distinguir si el subsistema $LR$ está vacío, ocupado o doblemente ocupado. A su vez, el subsistema $LR$ modifica su dinámica dependiendo del estado de ocupación del demonio.

Esta retroalimentación se implementa matemáticamente permitiendo que las tasas de túnel $\kappa_{i}(\epsilon + U) \neq \kappa_{i}(\epsilon)$, con $i = L, R$. En lo que sigue, se utilizará la notación $\kappa^{U}_{i} \equiv \kappa_{i}(\epsilon+U)$ y $\kappa_{i} \equiv \kappa_{i}(\epsilon)=\kappa_{i}(\epsilon+U_{f})=\kappa_{i}(\epsilon+U+U_{f})$. Por el contrario, las tasas de túnel del demonio no dependerán de la ocupación, por lo que se escribirá simplemente $\kappa_{D}\equiv \kappa_{D}(\epsilon')$.

Para que el demonio pueda detectar eficazmente el estado del subsistema $LR$, deben cumplirse dos condiciones. Primero, que la energía de interacción supere las fluctuaciones térmicas del reservorio asociado al demonio, es decir, $\beta_{D} U \gg 1$. Segundo, que su dinámica disipativa sea más rápida que la del subsistema $LR$, lo que implica $\kappa_{D} > \max\{\kappa_{i}, \kappa^{U}_{i}\}$ con $i = L, R$.

La dinámica específica del demonio se describirá con mayor detalle en la Figura~\ref{img:dinamica1}.

\insertimage[\label{img:dinamica1}]{ejemplos/diseño1}{scale=0.6}{Primer paso en la dinámica, el sitio $D$ y el sitio $R$ poseen una mayor probabilidad de estar ocupados.}

La elección $\epsilon_D = \mu_D - U/2$ favorece la ocupación del nivel de energía del demonio. Para inducir una corriente en dirección opuesta al gradiente de potencial, es conveniente imponer que $\kappa^{U}_{R} > \kappa^{U}_{L}$, de modo que las transiciones inducidas por el baño derecho sean más frecuentes, aumentando así la probabilidad de que este nivel se pueble. Posteriormente, como se ilustra en la Figura \ref{img:dinamica2}, el acoplamiento $g$ permite el intercambio coherente del electrón entre los niveles $L$ y $R$, introduciendo así una dinámica esencialmente cuántica.


\insertimage[\label{img:dinamica2}]{ejemplos/diseño2}{scale=0.6}{Segundo paso en la dinámica, el hopping $g$ genera superposición entre los sitios $L$ y $R$.}

Finalmente, se produce una tercera etapa, ilustrada en la Figura \ref{img:dinamica3}. En esta fase, con el objetivo de generar un flujo de corriente en contra del gradiente de potencial, resulta conveniente imponer que $\kappa_{L} > \kappa_{R}$. Esta condición asegura que, cuando el demonio se encuentra desocupado, la probabilidad de que ocurra una transición hacia el baño $L$ sea mayor que hacia el baño $R$. Como consecuencia, se favorece el transporte de electrones desde el reservorio $R$ (de menor potencial) hacia el reservorio $L$ (de mayor potencial), evidenciando así un comportamiento característico de un Demonio de Maxwell autónomo.

\insertimage[\label{img:dinamica3}]{ejemplos/diseño3}{scale=0.6}{Tercer paso en la dinámica, el electrón se transporta al baño $L$.}

Teniendo definida la dinámica del sistema, es posible elegir de forma adecuada distintos candidatos para las tasas de túnel con el objetivo de inducir un comportamiento tipo Demonio de Maxwell. Para ello, se estudiará el comportamiento de la matriz densidad, así como los flujos de corriente de partículas, calor, información y potencia, en función de la diferencia de potencial $\mu_{L} - \mu_{R} = eV$.

Se comenzará analizando las probabilidades de ocupación, extraídas de la matriz densidad en el estado estacionario. Como se muestra en la Figura~\ref{img:densitymatrix}, las tres probabilidades más altas corresponden a los estados $\rho_{001}$, $\rho_{100}$ y $\rho_{010}$, lo cual es coherente con la presencia de interacción de Coulomb, que penaliza las configuraciones con ocupación múltiple.

Para valores de $eV/T \lessapprox 3.6$, la probabilidad más alta corresponde al estado $\rho_{001}$, lo que indica que el sitio $D$ es, en promedio, el más ocupado. Sin embargo, a medida que la diferencia de potencial aumenta ($eV/T \gtrapprox 3.6$), las probabilidades asociadas a los estados $\rho_{100}$ y $\rho_{010}$ —correspondientes a los sitios $L$ y $R$— superan a $\rho_{001}$, lo que refleja un cambio en la ocupación predominante del sistema.

Debido a que la interacción de Coulomb penaliza los estados con más de un electrón, cabría esperar que la probabilidad de vacío $\rho_{000}$ fuese del mismo orden de magnitud que $\rho_{100},\rho_{010},\rho_{001}$. Sin embargo, debido a la constante inyección de partículas inducida por la diferencia de potencial, esta probabilidad resulta ser significativamente menor. 

Se observa además un cruce entre las curvas de $\rho_{101}$ y $\rho_{011}$ en $eV/T \approx 2.4$. Para valores $eV/T \lessapprox 2.4$, la configuración con interacción Coulomb entre los puntos $R$ y $D$ (representada por $\rho_{011}$) es más probable, esto favorece a que el sistema se encuentre en la configuración de la Figura \ref{img:dinamica1}, y por lo tanto es esperable que favorezca el comportamiento tipo Demonio. En cambio, para $eV/T \gtrapprox 2.4$, la ocupación $\rho_{011}$ se hace menos probable, indicando una menor interacción entre los sitios $R$ y $D$, por lo tanto el sistema deja de comportarse como Demonio.


\insertimage[\label{img:densitymatrix}]{ejemplos/matrixdemon.pdf}{scale=0.45}{Elementos diagonales de la matriz densidad en función de $eV/T$.}

El comportamiento de un Demonio de Maxwell autónomo se puede corrobar analizando las cantidades termodinámicas del sistema. En la Figura~\ref{img:fig1resultado} se muestran: (a) los flujos de calor correspondientes a cada reservorio, y (b) las corrientes de partículas asociadas. Cabe destacar que, dado que no existe un término de hopping entre el sistema y el demonio, la corriente de partículas en el demonio es nula, $\dot{N}_{d} = 0$. Además, por conservación de partículas, se cumple que $\dot{N}_{L} = -\dot{N}_{R}$, lo cual implica que la influencia del demonio sobre el sistema ocurre exclusivamente a través de flujos de energía o información.

Dado que $\mu_L - \mu_R = eV$, se esperaría que $\dot{N}_L > 0$ para todo valor de $eV$. No obstante, en el régimen $eV/T \lessapprox 2.4$, se observa que $\dot{N}_L < 0$, lo que indica un flujo de partículas en contra del gradiente de potencial. Este cambio de signo en la corriente coincide con el cruce entre las probabilidades $\rho_{101}$ y $\rho_{011}$, lo que es consistente con el comportamiento de Demonio de Maxwell. En particular, como se ilustró en la Figura~\ref{img:dinamica1}, dicho régimen requiere una interacción simultánea entre los puntos $R$ y $D$.

Si se analiza únicamente el subsistema $LR$, la existencia de una corriente contra el gradiente químico representa una violación aparente de la segunda ley de la termodinámica. Para este análisis, se considera la tasa de producción de entropía efectiva del subsistema en el estado estacionario, dada por $\dot{\sigma}^{o}_{LR} = - \sum_{i=L,R} J_i/T$, cuya evolución en función de $eV/T$ se muestra en la Figura~\ref{img:fig2resultado}.


\insertimage[\label{img:fig1resultado}]{ejemplos/heatparticledemon.pdf}{scale=0.54}{(a) Flujos de calor liberados por cada reservorio $\alpha$. (b) Flujos de corriente de partículas correspondientes a cada reservorio. Los resultados se obtuvieron considerando temperaturas $T = 100$ y $T_D = 2$, niveles de energía $\epsilon = 0$ y $\epsilon_D = \mu_D - U/2$ con potencial químico $\mu_D = 2$, tasas de túnel $\kappa_L = \kappa^{U}_{R} = 1/100$, $\kappa_R = \kappa^{U}_{L} = 1/600$, $\kappa_D =  1/50$, acoplamiento entre los sitios $g = 5/1000$ e interacciones de Coulomb $U = 40$ y $U_F = 500$.}

\insertimage[\label{img:fig2resultado}]{ejemplos/entropyapparentdemon.pdf}{scale=0.45}{Tasa de producción de entropía aparente del sistema $LR$($\dot{\sigma}^{o}_{LR}$) en función de $eV/T$.}

Donde se observa que para $eV/T \lessapprox 3$, la tasa de producción de entropía aparente $\dot{\sigma}^{o}_{LR}$ es menor a cero, lo que implica una violación aparente de la segunda ley de la termodinámica. Sin embargo, al incorporar la contribución del flujo de información en el cálculo de la tasa de producción de entropía \ref{sec4:localentropyquantum}, se obtiene la Figura \ref{img:fig3resultado}, en la cual se verifica que la producción total de entropía en el sistema $LR$ permanece siempre positiva.

\insertimage[\label{img:fig3resultado}]{ejemplos/entropyrialdemon.pdf}{scale=0.47}{Tasa de producción de entropía del sistema $LR$($\dot{\sigma}_{LR}$) en función de $eV/T$.}

En el régimen de parámetros en que \( eV/T \lessapprox 2.4 \), el subsistema \( LR \) realiza trabajo. Para determinar si este trabajo es impulsado principalmente por el flujo de información, es necesario analizar el intercambio energético entre el sistema y el demonio. En el estado estacionario, los flujos de energía cumplen $\dot{E}_{LR}= -\dot{E}_{D}$. Para que el sistema pueda interpretarse como un Demonio de Maxwell, el trabajo realizado por el subsistema \( LR \) debe estar dominado por la contribución informacional. Esta condición se puede expresar como:

\[
\dot{\mathcal{F}}_{LR} = \dot{E}_{LR} + T\dot{I}_{LR} = -\dot{E}_{d} + T\dot{I}_{LR} \approx T\dot{I}_{LR}, \hspace{10mm} \dot{W}_{LR} \geq \dot{\mathcal{F}}_{LR},
\]

donde \( \dot{\mathcal{F}}_{LR} \) representa la tasa de cambio de energía libre del subsistema \( LR \), incluyendo una contribución informacional \( T\dot{I}_{LR} \).

A fin de verificar este comportamiento, se grafican las cantidades termodinámicas relevantes tanto del sistema como del demonio.


\insertimage[\label{img:fig4resultado}]{ejemplos/thermoquantitiesdemon}{scale=0.60}{ Flujo de energía, tasa de cambio de energía libre, flujo de información, potencia y flujo de calor del sistema $LR$. (b)  Flujo de energía, tasa de cambio de energía libre, flujo de información y potencia del Demonio. }

La Figura \ref{img:fig4resultado} muestra que, en el régimen en el que el subsistema $LR$ realiza trabajo ($\dot{W}_{LR} < 0$ y $\dot{\mathcal{F}}_{LR} < 0$), la contribución informacional domina sobre el flujo energético, es decir, $\dot{\mathcal{F}}_{LR} \approx T \dot{I}_{LR}$ mientras que $\dot{E}_{LR} \approx 0$. Este resultado respalda la interpretación del dispositivo como un Demonio de Maxwell autónomo.

La potencia negativa $\dot{W}_{LR} < 0$ indica que el sistema está realizando trabajo en contra del gradiente de potencial, lo cual requiere absorber calor del entorno. En efecto, se verifica que $J_{LR} > 0$, señalando que el subsistema $LR$ extrae calor de sus respectivos reservorios.

Dado que no hay transporte neto de partículas a través del Demonio ($\dot{N}_{D} = 0$), el flujo de energía asociado se reduce a $J_{D} = \dot{E}_{D}$. Como se observa en la figura, $J_{D} < 0$, lo que implica que el Demonio disipa calor hacia su baño, caracterizado por una temperatura menor a los reservorios $L$ y $R$ ($T_D < T$). 

Si se adopta una perspectiva global del sistema, es decir, considerando la termodinámica de los tres puntos cuánticos como un conjunto, este comportamiento puede interpretarse como el de una máquina térmica. En este escenario, el sistema total extrae calor del reservorio caliente (\( J_{LR} \)), convierte parte de esa energía en trabajo útil (\( \dot{W}_{LR} \)) y disipa el resto hacia el baño frío (\( J_D \)), cumpliendo así la relación \( J_{LR} + J_D = -\dot{W}_{LR} \).

Al estar cada punto cuántico acoplado a un reservorio distinto, es posible descomponer el flujo de información total como $\dot{I}_{LR} = \dot{I}_{L} + \dot{I}_{R}$, lo que permite analizar la contribución informacional de cada punto cuántico por separado. Esta descomposición se ilustra en la Figura \ref{img:figinforesultado}.


\insertimage[\label{img:figinforesultado}]{ejemplos/infolrddemon.pdf}{scale=0.45}{Flujos de información de cada sitio, $L$,$R$ y $D$. }

Es interesante notar que, cuando \( eV/T \gtrapprox 3.6 \), el menor flujo de información corresponde al punto cuántico \( R \), lo que implica que este entrega más información al sistema. En cambio, para \( 1 \lessapprox eV/T \lessapprox 3.6 \), el flujo de información menor proviene del sitio \( L \), de modo que este es el que aporta más información en ese rango. Esta transición se correlaciona con lo observado en la Figura~\ref{img:densitymatrix}: para \( eV/T \gtrapprox 3.6 \) se cumple que \(\rho_{001} < \rho_{010} < \rho_{100}\), lo que indica una mayor probabilidad de ocupación del sitio \( L \). Además, en este mismo rango de parámetros  ($eV/T \gtrapprox 3.6$) la probabilidad del estado \( \rho_{110} \) supera a la de \( \rho_{011} \), lo que implica que la ocupación simultánea de \( L \) y \( R \) ---y por ende su monitoreo mutuo--- es más probable que la de \( R \) y \( D \). Lo anterior puede interpretarse como que, en este régimen, las correlaciones entre \( L \) y \( R \) se intensifican. Pues, en el estado \( \rho_{110} \) los sitios \( L \) y \( R \) interactúan mediante energía de Coulomb, siendo está una interacción mayor al \textit{hopping} ($U_{f}>g$), lo que facilita que \( L \) obtenga información de \( R \) y, en consecuencia, contribuya positivamente a su flujo de información, de modo que \( \dot{I}_{L} > \dot{I}_{R} \). Para diferencias de potencial pequeñas (\( eV/T \lessapprox 1 \)), ambos flujos de información son prácticamente idénticos (\( \dot{I}_{L} \approx \dot{I}_{R} \)).

%Este comportamiento es consistente con la intuición física: un mayor potencial favorece la ocupación del punto cuántico \( L \), mientras que la asimetría en las tasas de túnel está diseñada para promover la ocupación de \( R \). Al pasar cierto umbral de $eV/T$, se inhibe por completo la eficacia de la dinámica asociada al mecanismo tipo Demonio de Maxwell, descrita en las Figuras~\ref{img:dinamica1},~\ref{img:dinamica2} y~\ref{img:dinamica3}.

%La clave de este mecanismo es que el subsistema \( D \) debe ser capaz de monitorear el estado de ocupación del punto \( R \), para luego permitir una transición hacia \( L \) cuando se encuentra desocupado. Sin embargo, al aumentar la probabilidad de ocupación de \( L \), se interrumpe esta dinámica, lo que reduce el flujo de información desde el sitio $L$ y por lo tanto pasa a entregar más información. Como consecuencia, el flujo informacional asociado a \( L \) crece con \( eV/T \), llegando a superar el de \( R \).

Un aspecto relevante de esta dinámica es la posible presencia de efectos cuánticos. Para examinar este fenómeno, se representan dos cantidades: la norma-$l_{1}$ de coherencia \( \mathcal{C}_{l_{1}} \), que permite cuantificar la coherencia cuántica en el subsistema \( LR \), y la concurrencia, que sirve como medida del entrelazamiento entre estados del subsistema $LR$ $(|1,0>,|0,1>)$.

\insertimage[\label{img:fig5resultado}]{ejemplos/quantumdemon.pdf}{scale=0.40}{En línea azul la norma-$l_{1}$ de la coherencia $\mathcal{C}_{l_{1}}$ y en línea rojas la concurrencia $\mathcal{C}_{on}$.}

Se observan dos comportamientos relevantes. Primero, para \( eV/T \lessapprox 1 \), la concurrencia \( \mathcal{C}_{\text{on}} > 0 \), lo que indica la presencia de entrelazamiento entre los sitios \( L \) y \( R \). Segundo, al analizar la coherencia, se aprecia que para \( eV/T \lessapprox 2.4 \) esta disminuye progresivamente hasta anularse, y luego comienza a incrementarse, a partir de \( eV/T \gtrapprox 2.4 \). Este comportamiento es particularmente significativo, ya que la coherencia se anula en un punto cercano al valor en el que la corriente de partículas invierte su dirección. 

%Otra característica importante de está dinámica, es que si se compara \ref{img:fig5resultado} con \ref{img:transporte5}, la coherencia y concurrencia alcanzan valores más alto para el caso en que las tasas de túnel se mantienen constantes, esto es esperable, pues el transporte de electrones ocurre de una forma más coherente, que en el caso actual.

Dado el comportamiento tipo Demonio de Maxwell del sistema, resulta pertinente examinar cómo las cantidades termodinámicas se ven influenciadas por la coherencia, el parámetro del Hamiltoniano $\hat{H}_{S}$ que se encarga de inducir coherencia en el sistema, es el acoplamiento \( g \). Por ende, en la Figura \ref{img:fig6resultado} se muestran las cantidades termodinámicas en función del parámetro \( g/\kappa_{L} \) 

\insertimage[\label{img:fig6resultado}]{ejemplos/thermoquantum.pdf}{scale=0.57}{Cantidades termodinámicas para el sistema $LR$ y el Demonio en función de $g/\kappa_{L}$, considerando $eV/T=1$ y el resto de parámetros corresponden a los de la Figura \ref{img:fig1resultado}. }

Se observa que cuando el acoplamiento se encuentra en un régimen \( g/\kappa_{L} \gtrapprox 10^{-1}\), la potencia generada del subsistema $LR$ ($\dot{W}_{LR}$) y la contribución informacional ($T\dot{I}_{LR}$) decrecen, por ende, el subsistema $LR$ realiza más trabajo, y entrega más información al sitio $D$. Este resultado sugiere que un acoplamiento suficientemente fuerte entre los sitios $L$ y $R$ potencia la conversión de información en trabajo. 

Además, bajo esta misma condición \( g/\kappa_{L} \gtrapprox 10^{-1} \), el flujo de energía del demonio \( \dot{E}_{D} \) se vuelve más negativo, lo que indica mayor transporte energético entre el demonio y el sistema $LR$.

Por otra parte, el análisis de las contribuciones individuales a los flujos de información revela cómo el acoplamiento modifica la cantidad de información entregada por cada punto cuántico. Estos efectos se visualizan en la Figura \ref{img:fig7resultado}, donde se aprecia la dependencia de los flujos informacionales con respecto al parámetro \( g \).

\insertimage[\label{img:fig7resultado}]{ejemplos/concuinfodemon.pdf}{scale=0.56}{(a)Flujos de información de cada sitio. (b)Concurrencia y coherencia en función de $g/\kappa_{L}$. }

Un comportamiento notable se presenta cuando el acoplamiento alcanza valores comparables a la tasa de túnel, es decir, \( 10^{-1}< g/\kappa_{L}< 10^{0} \). En este régimen, tanto la concurrencia como la coherencia alcanzan valores cercanos a sus máximos, lo que indica una presencia significativa de efectos cuánticos entre los puntos \( L \) y \( R \), resultado que se asemeja a \cite{prech2023entanglement}. 

Simultáneamente, los flujos de información individuales tienden a igualarse, es decir, \( \dot{I}_{L} \approx \dot{I}_{R} \), lo que sugiere que, en condiciones de entrelazamiento y coherencia elevadas, ambos puntos cuánticos contribuyen de manera equilibrada en el flujo de información del sistema $LR$. 


\label{sec5:demonio1}

\newpage 

\section{Comparación con caso clásico}
Una de las ventajas del formalismo presentado en \cite{potts2021thermodynamically} es que permite incluir términos no seculares en la dinámica del sistema. Esto posibilita la aparición de coherencias y entrelazamiento no nulos en el régimen \( g \sim \kappa_{L} \). La pregunta natural que surge es si estos efectos cuánticos representan una ventaja para el funcionamiento del demonio. Para abordar esta cuestión, se puede considerar una evolución semiclásica, limitada a las componentes diagonales de la matriz de densidad del sistema. Inicialmente se considera una evolución descrita por el superoperador

\begin{equation*}
\check{\mathcal{L}}_{f}|\hat{\rho}\rangle\rangle = -i[\hat{H}_{S}, \hat{\rho}] + \mathcal{L}(\hat{\rho}),
\end{equation*}
que combina la dinámica coherente del sistema con la disipación inducida por los baños, como por ejemplo \ref{sec2schrodingerthermo}. Este superoperador puede descomponerse como $\check{\mathcal{L}}_{f} = \check{\mathcal{L}}_{0} + \check{\mathcal{V}}$, donde el término $\check{\mathcal{V}}$ representa el acoplamiento entre los sitios y actúa como

\begin{equation*}
\check{\mathcal{V}}|\hat{\rho}\rangle\rangle = -ig[\hat{d}_{L}^{\dagger} \hat{d}_{R} + \hat{d}_{R}^{\dagger} \hat{d}_{L}, \hat{\rho}].
\end{equation*}

Para obtener una ecuación maestra que describa únicamente la dinámica de las partes diagonales de la matriz densidad, se aplica el formalismo de Nakajima-Zwanzig \cite{zwanzig1966statistical}, con los operadores $\check{\mathcal{P}}$ y $\check{\mathcal{Q}}$ que distinguen la parte relevante(diagonal) y la parte no relevante(no diagonal) de la matriz densidad $\hat{\rho}_{S}$, lo que conduce a la ecuación maestra

\begin{equation}
    \frac{d}{dt}\check{\mathcal{P}}|\hat{\rho}_{S}(t)\rangle \rangle = (\check{\mathcal{L}}_{0} - \check{\mathcal{P}}\check{\mathcal{V}}\check{\mathcal{Q}}\check{\mathcal{L}}^{-1}_{0}\check{\mathcal{Q}}\check{\mathcal{V}}\check{\mathcal{P}})\check{\mathcal{P}}|\hat{\rho}_{S}(t)\rangle \rangle. 
\label{ec5:classicalmodel}
\end{equation}

Con el operador inverso $\check{\mathcal{L}}_{0}^{-1}$ que consiste en el inverso de Drazin del operador $\check{\mathcal{L}}_{0}$, dado por 
\begin{equation*}
\check{\mathcal{L}}_{0}^{-1} = -\int_{0}^{\infty} d\tau\, e^{\check{\mathcal{L}}_{0} \tau} \check{\mathcal{Q}},
\end{equation*}
según lo discutido en~\cite{landi2024current}. La demostración detallada de la ecuación \ref{ec5:classicalmodel} se encuentra en el Apéndice~\ref{appendix5clasic}.

El hecho de disponer de una ecuación maestra que describe únicamente la evolución de las partes diagonales de la matriz densidad, permite modelar el sistema de manera estocástica. Esto se logra mediante un vector de probabilidades de ocupación definido como:
\begin{equation*}
\mathbf{P} = [\rho_{000}, \rho_{100}, \rho_{010}, \rho_{001}, \rho_{110}, \rho_{101}, \rho_{011}, \rho_{111}]^{T},
\end{equation*}
cuya dinámica está gobernada por la ecuación maestra

\begin{equation}
    \frac{d}{dt}\textbf{P} = \textbf{W}\textbf{P},
    \label{sec5:classicsemi}
\end{equation}
donde la matriz de transición $\mathbf{W}$ puede obtenerse a partir de la ecuación~(\ref{ec5:classicalmodel}).

En lo que sigue, se estudiará el estado estacionario de la ecuación maestra presentada en la ecuación~\ref{sec5:classicsemi}. En particular, se compararán las probabilidades de ocupación obtenidas en este modelo clásico con aquellas resultantes del tratamiento cuántico, lo cual se muestra en las Figuras~\ref{img:rho100} y~\ref{img:rho011}. 

\insertimage[\label{img:rho100}]{ejemplos/rho1004.pdf}{scale=0.46}{Elementos diagonales $\rho_{100}$, $\rho_{010}$, $\rho_{111}$ y $\rho_{110}$ en función de $eV/T$. Donde el caso clásico corresponde a la línea roja y el caso cuántico a la línea azul.}
\insertimage[\label{img:rho011}]{ejemplos/rho1014.pdf}{scale=0.46}{Elementos diagonales $\rho_{101}$, $\rho_{000}$, $\rho_{011}$ y $\rho_{001}$ en función de $eV/T$.}

De las Figuras \ref{img:rho100} y \ref{img:rho011} se observa que en los dos casos, las probabilidades de ocupación dominantes corresponden a $\rho_{100},\rho_{010}$ y $\rho_{001}$. Por el contrario, la probabilidad de ocupación más pequeña es $\rho_{111}$.
 
Se puede notar que, cerca del punto $eV/T \approx 2.4$ donde la coherencia es casi nula, todas las probabilidades de ocupación clásicas se cruzan con sus contrapartes cuánticas. En cambio, para $eV/T \lessapprox 1$, los elementos $\rho_{100}$, $\rho_{010}$ y $\rho_{101}$ presentan una diferencia apreciable entre los comportamientos cuántico y clásico, lo cual se puede interpretar como un efecto de la coherencia sobre estos elementos.


A partir de la solución estacionaria del sistema, es posible calcular los flujos de corriente de partículas e información asociados al baño $\alpha$ mediante las expresiones

\begin{equation}
    \dot{N}_{\alpha} = -\mathbf{N}^{T} \mathbf{W}_{\alpha}\mathbf{P},
    \label{sec5:currentsemi}
\end{equation}
\begin{equation}
    \dot{I}_{\alpha} = -(\log \mathbf{P})^{T} \mathbf{W}_{\alpha}\mathbf{P},
    \label{sec5:infosemi}
\end{equation}

donde $\mathbf{P}$ es el vector de probabilidades en la base de Fock utilizada,$\mathbf{W}_{\alpha}$ es la contribución del baño $\alpha$ en la evolución $\mathbf{W}$, $\mathbf{N}$ contiene el número de partículas en cada estado, y $\log \mathbf{P}$ se refiere al logaritmo aplicado componente a componente.

En el modelo clásico, la expresión analítica para el flujo de corriente de partículas correspondiente al baño $L$ es idéntica al cuántico. Usando la expresión \ref{sec5:currentsemi}, la corriente de partículas esta dada por  

\begin{align*}
    \dot{N}_{L} & = \kappa_{L}\big(f_{L}(\epsilon)\rho_{000} - [1-f_{L}(\epsilon)]\rho_{100} \big) \\
        & + \kappa^{U}_{L} \big(f_{L}(\epsilon+U)\rho_{001} - [1-f_{L}(\epsilon+U)]\rho_{101} \big)  \\
        & + \kappa_{L} \big(f_{L}(\epsilon+U_{f})\rho_{010} - [1-f_{L}(\epsilon+U_{f})]\rho_{110} \big)  \\  
        & + \kappa_{L} \big(f_{L}(\epsilon+U + U_{f})\rho_{011} - [1-f_{L}(\epsilon+U+U_{f})]\rho_{111} \big).
\end{align*}

Por otro lado, mediante la expresión \ref{sec5:infosemi} se obtienen los flujos de información clásicos asociados a los baños $L$ y $R$, los cuales están descritos por

\begin{align*}
  \dot{I}_{L}  &  =  \kappa_{L}(f_{L}(\epsilon)[\rho_{000}\ln \rho_{100} - \rho_{000}\ln \rho_{000}] + (1-f_{L}(\epsilon))[\rho_{100}\ln \rho_{000} - \rho_{100}\ln \rho_{100} ] )  \\ 
      & + \kappa^{U}_{L}(f_{L}(\epsilon + U)[\rho_{001}\ln \rho_{101} - \rho_{001}\ln \rho_{001}] + (1-f_{L}(\epsilon + U))[\rho_{101}\ln \rho_{001} -\rho_{101}\ln \rho_{101} ]   ) \\  
      & + \kappa_{L}( f_{L}(\epsilon + U_{f})[\rho_{010}\ln \rho_{110} -\rho_{010}\ln \rho_{010} ]  + (1-f_{L}(\epsilon + U_{f}))[\rho_{110}\ln \rho_{010} - \rho_{110}\ln \rho_{110} ] ) \\  
      & + \kappa_{L}( f_{L}(\epsilon + U + U_{f})[\rho_{011}\ln \rho_{111}- \rho_{011}\ln \rho_{011}]) \\   
      & + \kappa_{L}( (1-f_{L}(\epsilon+U+U_{f}) )[\rho_{111}\ln \rho_{011} - \rho_{111}\ln \rho_{111}])   
\end{align*}

\begin{align*}
 \dot{I}_{R} & =  \kappa_{R}(f_{R}(\epsilon)[\rho_{000}\ln \rho_{010} - \rho_{000}\ln \rho_{000}] + (1-f_{R}(\epsilon))[\rho_{010}\ln \rho_{000} - \rho_{010}\ln \rho_{010} ] )  \\ 
      & + \kappa^{U}_{R}(f_{R}(\epsilon + U)[\rho_{001}\ln \rho_{011} - \rho_{001}\ln \rho_{001}] + (1-f_{R}(\epsilon + U))[\rho_{011}\ln \rho_{001} -\rho_{011}\ln \rho_{011} ]   ) \\  
      & + \kappa_{R}( f_{R}(\epsilon + U_{f})[\rho_{100}\ln \rho_{110} - \rho_{100}\ln \rho_{100} ]  + (1-f_{R}(\epsilon + U_{f}))[\rho_{110}\ln \rho_{100} - \rho_{110}\ln \rho_{110} ] ) \\  
      & + \kappa_{R}( f_{R}(\epsilon + U + U_{f})[\rho_{101}\ln \rho_{111}- \rho_{101}\ln \rho_{101}]) \\   
      & + \kappa_{R}( (1-f_{R}(\epsilon+U+U_{f}) )[\rho_{111}\ln \rho_{101} - \rho_{111}\ln \rho_{111}]).   
\end{align*}

La versión cuántica de los flujos de información se encuentra detallada en el apéndice~\ref{apendix5infoflow}. En la Figura~\ref{img:P4clqm} se muestra una comparación gráfica entre los casos clásico y cuántico para estas cantidades.

\insertimage[\label{img:P4clqm}]{ejemplos/currentinfosemi.pdf}{scale=0.54}{Corriente de partículas y flujos de información para el caso clásico(rojo) y cuántico(azul) en función de $eV/T$.}

Como se muestra en la Figura \ref{img:P4clqm}, en el régimen $eV/T\lessapprox 2.4$ la corriente de partículas calculada en el caso cuántico es menor a la del caso clásico, lo que representa una ventaja del caso cuántico, pues al ser menor la corriente, esto significa que hay mayor transporte de electrones en contra del gradiente de potencial. Por otro lado, para $eV/T \gtrapprox 2.4$ la corriente de partículas en el caso cuántico es mayor que en el caso clásico. 

Como se observa en la Figura \ref{img:P4clqm}, el flujo de información asociado al sitio $D$ es mayor en el caso cuántico, mostrando una diferencia apreciable en las regiones $eV/T \lessapprox 1$ y $eV/T \gtrapprox 6$. Este resultado indica que la dinámica cuántica permite una mejora en el intercambio de información entre el sistema $D$ y el subsistema $LR$. Dicha mejora se explica porque el sitio $D$ no solo establece correlaciones clásicas a través de los elementos diagonales del estado, sino que, debido a la presencia de coherencia, también genera correlaciones cuánticas con el sistema $LR$, incluso cuando la interacción entre ellos es exclusivamente de tipo clásico, mediada por la energía de Coulomb. Este comportamiento se corrobora con los flujos de información cuánticos obtenidos analíticamente, los cuales presentan contribuciones provenientes de los elementos no diagonales de la matriz densidad (ver Apéndice~\ref{apendix5infoflow}).
 

Finalmente, debido a este aumento en el flujo de información en el régimen cuántico, la tasa de cambio de la energía libre del subsistema $LR$, dada por $\dot{\mathcal{F}}_{LR} \approx T\dot{I}_{LR}$, disminuye. Como consecuencia, se incrementa la capacidad del sistema para realizar trabajo, lo cual evidencia el rol beneficioso de la coherencia cuántica como recurso termodinámico.


\label{sec5:demonio}


% ------------------------------------------------------------------------------
% NUEVO CAPÍTULO
% ------------------------------------------------------------------------------
\chapteranum{Conclusiones}
En esta tesis se derivó una ecuación maestra en la base local para un sistema compuesto por tres puntos cuánticos, la cual permite incorporar y analizar términos no seculares en su evolución. Se observó que, dependiendo de las tasas de túnel asociadas a los reservorios $L$ y $R$, existen ciertas regiones del parámetro $eV/T$ en los cuales el sistema exhibe un comportamiento análogo al de un Demonio de Maxwell autónomo. En estas regiones, los flujos de información entre el punto cuántico $D$ y el subsistema $LR$ permiten la realización de trabajo, manifestado como un transporte neto de partículas en contra del gradiente de potencial. Este fenómeno implica una aparente violación de la segunda ley de la termodinámica, en concordancia con resultados previos obtenidos para sistemas de dos puntos cuánticos, donde también se identifica un comportamiento tipo Demonio de Maxwell \cite{horowitz2014thermodynamics}.

Se obtuvo que los flujos de información asociados a los sitios $L$ y $R$ presentan una dependencia explícita en el parámetro de acoplamiento $g$. En particular, dichos flujos se igualan en torno a $g \sim \kappa_L$, lo que coincide con un régimen en el cual tanto la coherencia cuántica como la concurrencia alcanzan valores cercanos a su máximo. Este resultado evidencia que la coherencia y el entrelazamiento impactan directamente en la dinámica de los flujos de información.

Asimismo, se comparó la evolución cuántica completa del sistema con una descripción semiclásica que considera únicamente las partes diagonales de la matriz densidad. Se observó que, para valores de $eV/T \lesssim 2.4$, la corriente de partículas es menor en el caso cuántico. Además, los flujos de información que recibe el sitio $D$ (el Demonio) resultan ser mayores en la descripción cuántica que en la semiclásica. Esta observación tiene dos implicancias relevantes. Primero, confirma que la coherencia cuántica puede actuar como un recurso termodinámico, pues al aumentar el flujo de información hacia el demonio, se incrementa la cota superior para el trabajo que el subsistema $LR$ puede realizar. Segundo, muestra que incluso cuando la interacción entre el demonio y el sistema es puramente clásica (de tipo Coulomb), los efectos cuánticos —como la coherencia— continúan desempeñando un papel en la transferencia de información.

Finalmente, el modelo propuesto admite diversas extensiones. Una dirección posible consiste en derivar una ecuación maestra para el caso en que la interacción de Coulomb entre el sitio \( D \) y los sitios \( L \) y \( R \) sea asimétrica. Este tipo de asimetría podría inducir un comportamiento de tipo Demonio de Maxwell autónomo, análogo al que se ha reportado recientemente en modelos clásicos \cite{monsel2025autonomous}. Por otro lado, desde el punto de vista del transporte, sería interesante incorporar más puntos cuánticos al sistema, lo que permitiría incluir más \textit{hopping} en el modelo y encontrar efectos adicionales de coherencia cuántica, como se estudia en \cite{khandelwal2025maximal}.

%%%se obtuvo una ecuación maestra para un sistema de 3 puntos cuánticos, a través de la aproximación semilocal
%%%se obtuvo que bajo ciertas condiciones el sistema se comporta como demonio de Maxwell
%%%los flujos de información por separados poseen una dependecia de g y la coherencia y concurrencia
%%%se comparo el caso clásico y el cuántico, en donde en el caso cuántico se presenta una mejoría de los flujos de información  




% ------------------------------------------------------------------------------
% REFERENCIAS, revisar configuración \stylecitereferences
% ------------------------------------------------------------------------------
\bibliography{library}


% ------------------------------------------------------------------------------
% ANEXO
% Existe adicionalmente el entorno \begin{appendixd} que permite insertar
% \chapter y el entorno \begin{appendixdtitle}[style1] (4 estilos diferentes),
% el cual acepta \chapter y escribe el título de anexos encima
% ------------------------------------------------------------------------------
\begin{appendixs}
	
	\section{Cálculos realizados sección 2}

    \subsection{Matriz de densidad en función del campo de conteo}

    \label{apendix:fcs1}
    Al sustituir la distribución de probabilidad en la función generadora \ref{sec2funciongeneradora}, se obtiene la expresión 

    \begin{align*}
        \Lambda(\vec{\lambda},\vec{\chi}) & = \sum_{\textbf{E},\textbf{E}',\textbf{N},\textbf{N}'} \int d\textbf{Q}d\textbf{W} P_{t}(\textbf{E}',\textbf{N}'|\textbf{E},\textbf{N}) P_{0}(\textbf{E},\textbf{N})  \\
        & \times \Pi_{\alpha} \delta(W_{\alpha} - \mu_{\alpha}(N_{\alpha} - N'_{\alpha}) ) \delta(Q_{\alpha} + W_{\alpha} -(E_{\alpha} - E'_{\alpha})) e^{-i\vec{\lambda}\cdot \textbf{Q}} e^{-i\vec{\chi}\cdot \textbf{W}} \\
        & = \sum_{\textbf{E},\textbf{E}',\textbf{N},\textbf{N}'}P_{t}(\textbf{E}',\textbf{N}'|\textbf{E},\textbf{N})P_{0}(\textbf{E},\textbf{N}) \Pi_{\alpha}e^{-i\chi_{\alpha}\mu_{\alpha}(N_{\alpha} - N'_{\alpha})}e^{-i\lambda_{\alpha}((E_{\alpha} -\mu_{\alpha}N_{\alpha}) - (E'_{\alpha} - \mu_{\alpha}N'_{\alpha}) ) }  \\
        & = \sum_{\textbf{E},\textbf{E}',\textbf{N},\textbf{N}'} \text{Tr}\{\hat{P}_{\textbf{E}',\textbf{N}'}\hat{U}(t)\hat{P}_{\textbf{E},\textbf{N}}(\hat{\rho}_{S}(0) \otimes \Pi_{\alpha}\hat{\tau}_{\alpha} )\hat{P}_{\textbf{E},\textbf{N}}\hat{U}^{\dagger}(t)\hat{P}_{\textbf{E}',\textbf{N}'}   \} \\
        & \times \Pi_{\alpha} e^{-i\chi_{\alpha}\mu_{\alpha}(N_{\alpha} - N'_{\alpha})}e^{-i\lambda_{\alpha}((E_{\alpha} -\mu_{\alpha}N_{\alpha}) - (E'_{\alpha} - \mu_{\alpha}N'_{\alpha}) ) }.
    \end{align*}

Considerando un observable $\hat{A}(0)$ con descomposición espectral asociada a los proyectores $\hat{P}_{a_0}$, al aplicarlos sobre un estado diagonal de densidad $\hat{\rho}_{\text{diag}}$, se obtiene la identidad\cite{esposito2009nonequilibrium}

\begin{equation}
    \sum_{a_0} e^{-i\lambda a_0} \hat{P}_{a_0} \hat{\rho}_{\text{diag}} \hat{P}_{a_0} 
    = e^{-i(\lambda/2)\hat{A}(0)} \hat{\rho}_{\text{diag}} e^{-i(\lambda/2)\hat{A}(0)}.
    \label{apendixobservable}
\end{equation}

Esta relación permite reescribir la función generadora en términos de un observable $\hat{A}(0)$ y el estado inicial diagonal. Así, si se utiliza \ref{apendixobservable} con $\hat{A}(0) = \sum_{\alpha}[\lambda_{\alpha}(\hat{H}_{\alpha}-\mu_{\alpha}\hat{N}_{\alpha}) + \chi_{\alpha}\mu_{\alpha}\hat{N}_{\alpha}] $ se obtiene 
\begin{align*}
    \Lambda(\vec{\lambda},\vec{\chi}) & = \sum_{\textbf{E}',\textbf{N}'}\text{Tr}\{ \hat{P}_{\textbf{E}',\textbf{N}'} \hat{U}(t) e^{-\frac{i}{2}\sum_{\alpha}[\lambda_{\alpha}(\hat{H}_{\alpha} - \mu_{\alpha}\hat{N}_{\alpha}) + \chi_{\alpha}\mu_{\alpha}\hat{N}_{\alpha}  ]}  \hat{\rho}_{tot}(0) e^{-\frac{i}{2}\sum_{\alpha}[\lambda_{\alpha}(\hat{H}_{\alpha} - \mu_{\alpha}\hat{N}_{\alpha}) + \chi_{\alpha}\mu_{\alpha}\hat{N}_{\alpha} ] }\hat{U}^{\dagger}(t) \hat{P}_{\textbf{E}',\textbf{N}'}  \\
    & \times \Pi_{\alpha}e^{i\chi_{\alpha}\mu_{\alpha}\hat{N}_{\alpha}}e^{i\lambda_{\alpha}(\hat{H}_{\alpha} -\mu_{\alpha}\hat{N}_{\alpha})} \} \\
    & = \text{Tr}\{ \hat{U}(\vec{\lambda},\vec{\chi};t)\hat{\rho}_{tot}(0)\hat{U}^{\dagger}(-\vec{\lambda},-\vec{\chi};t)  \}.
\end{align*}

\newpage 

    \subsection{Matriz de densidad generalizada}

La ecuación \ref{sec2FCS:evolution} puede expandirse en serie hasta segundo orden, con el objetivo de obtener una expresión aproximada para la evolución de la matriz de densidad generalizada

    \begin{equation*}
        |\hat{\rho}_{totI}(\vec{\lambda},\vec{\chi},t)\rangle \rangle  = \left( \textbf{1} + \epsilon \int_{0}^{t}\check{\mathcal{L}}'_{\lambda}(t_{1})dt_{1} + \epsilon^{2}\int_{0}^{t}dt_{1}\int_{0}^{t_{1}}\check{\mathcal{L}}'_{\lambda}(t_{1})\check{\mathcal{L}}'_{\lambda}(t_{2})dt_{2} \right) |\hat{\rho}_{tot}(0)\rangle \rangle, 
    \end{equation*}
Al introducir el cambio de variables \( t_1 = T \), \( t_2 = T - s \), se transforma la integral doble en una forma más conveniente para su evaluación

\begin{align*}
    |\hat{\rho}_{totI}(\vec{\lambda},\vec{\chi},t)\rangle \rangle  & = \left( \textbf{1} + \epsilon \int_{0}^{t}\check{\mathcal{L}}'_{\lambda}(T)dT + \epsilon^{2}\int_{0}^{t}dT\int_{0}^{T}ds \check{\mathcal{L}}'_{\lambda}(T)\check{\mathcal{L}}'_{\lambda}(T-s) \right)|\hat{\rho}_{tot}(0)\rangle \rangle  \\
    & = \check{\mathcal{W}}(\vec{\lambda},\vec{\chi},t)|\hat{\rho}_{tot}(0)\rangle \rangle \\
    & =  [\check{\mathcal{W}}_{0}(\vec{\lambda},\vec{\chi},t) + \epsilon \check{\mathcal{W}}_{1}(\vec{\lambda},\vec{\chi},t) + \epsilon^{2}\check{\mathcal{W}}_{2}(\vec{\lambda},\vec{\chi},t)] |\hat{\rho}_{tot}(0)\rangle \rangle,
\end{align*}
con los superoperadores definidos por 

\begin{align*}
    \check{\mathcal{W}}_{0}(\vec{\lambda},\vec{\chi},t) &  = \textbf{1}\\
    \check{\mathcal{W}}_{1}(\vec{\lambda},\vec{\chi},t) & = \int_{0}^{t}dT \check{\mathcal{L}'}_{\lambda}(T) \\
    \check{\mathcal{W}}_{2}(\vec{\lambda},\vec{\chi},t) & = \int_{0}^{t}dT \int_{0}^{T}ds \check{\mathcal{L}'}_{\lambda}(T)\check{\mathcal{L}'}_{\lambda}(T-s).
\end{align*}

Considerando la expansión hasta segundo orden de la inversa del superoperador $\check{\mathcal{W}}$

\begin{equation}
    \check{\mathcal{W}}^{-1}(\vec{\lambda},\vec{\chi},t) =  \check{\mathcal{W}}_{0}(\vec{\lambda},\vec{\chi},t) - \epsilon  \check{\mathcal{W}}_{1}(\vec{\lambda},\vec{\chi},t) +  \epsilon^{2}[\check{\mathcal{W}}^{2}_{1}(\vec{\lambda},\vec{\chi},t) -  \check{\mathcal{W}}_{2}(\vec{\lambda},\vec{\chi},t) ],
\label{apendix2inverseW}
\end{equation}
se cumple la propiedad

\begin{equation}
    \dot{\check{\mathcal{W}}}(\vec{\lambda},\vec{\chi},t)\check{A}\check{\mathcal{W}}^{-1}(\vec{\lambda},\vec{\chi},t) = \epsilon \dot{\check{\mathcal{W}}}_{1}(\vec{\lambda},\vec{\chi},t)\check{A} + \epsilon^{2}[\dot{\check{\mathcal{W}}}_{2}(\vec{\lambda},\vec{\chi},t)\check{A} - \dot{\check{\mathcal{W}}}_{1}(\vec{\lambda},\vec{\chi},t)\check{A}\check{\mathcal{W}}_{1}(\vec{\lambda},\vec{\chi},t) ].
    \label{apendix2Wproperty}
\end{equation}

La obtención de la matriz densidad generalizada requiere proyectar la matriz de densidad total sobre el subespacio del sistema, lo que equivale a trazar los grados de libertad del reservorio. Para ello, se introduce el siguiente superoperador de proyección

\begin{equation*}
    \check{\mathcal{P}} = \sum_{r}|\rho_{R}^{eq} \rangle \rangle \langle \langle rr|,
\end{equation*}
donde \(\otimes_{\alpha} \hat{\tau}_{\alpha} \to |\hat{\rho}_{R}^{\mathrm{eq}} \rangle\rangle\) representa el estado de equilibrio del reservorio, expresado como un vector en el espacio de Liouville. Al aplicar este proyector sobre la matriz de densidad total, se obtiene

\begin{equation*}
    \check{\mathcal{P}}|\hat{\rho}(\vec{\lambda},\vec{\chi},t) \rangle \rangle = |\hat{\rho}_{S}(\vec{\lambda},\vec{\chi},t)\rangle \rangle \otimes |\hat{\rho}^{eq}_{R}\rangle \rangle .
\end{equation*}

La evolución de la matriz densidad generalizada, siguiendo el formalismo de Nakajima–Zwanzig, se expresa como

\begin{align}
    \check{\mathcal{P}}|\hat{\rho}_{totI}(\vec{\lambda},\vec{\chi},t)\rangle \rangle & =  \check{\mathcal{P}} \check{\mathcal{W}}(t)( \check{\mathcal{P}} +  \check{\mathcal{Q}})|\hat{\rho}_{tot}(0)\rangle \rangle  \label{apendix2proyectionev1} \\
    \check{\mathcal{Q}}|\hat{\rho}_{totI}(\vec{\lambda},\vec{\chi},t)\rangle \rangle & = \check{\mathcal{Q}} \check{\mathcal{W}}(t)( \check{\mathcal{P}} +  \check{\mathcal{Q}})|\hat{\rho}_{tot}(0)\rangle \rangle. 
\label{apendix2proyectionev}
\end{align}

Existen dos consideraciones fundamentales para continuar con el desarrollo. Primero, como se asume que la condición inicial del reservorio es diagonal, se cumple que
\[
\check{\mathcal{Q}}|\hat{\rho}_{\mathrm{tot}}(0)\rangle\rangle = 0.
\]

Segundo, se tiene que el estado total inicial puede expresarse como
\[
|\hat{\rho}_{\mathrm{tot}}(0)\rangle\rangle = \check{\mathcal{W}}^{-1}(\vec{\lambda}, \vec{\chi}, t)\, |\hat{\rho}_{\mathrm{tot}}(\vec{\lambda}, \vec{\chi}, t)\rangle\rangle,
\]
lo cual permite reformular la evolución proyectada en función del estado actual. Es decir, 
\begin{align*}
    |\hat{\rho}_{totI}(0)\rangle \rangle & = (\check{\mathcal{P}} + \check{\mathcal{Q}} )\check{\mathcal{W}}^{-1}(\vec{\lambda},\vec{\chi},t)(\check{\mathcal{P}} + \check{\mathcal{Q}})|\hat{\rho}_{totI}(\vec{\lambda},\vec{\chi},t)\rangle \rangle \\
        & = \check{\mathcal{P}}\check{\mathcal{W}}^{-1}(\vec{\lambda},\vec{\chi},t)(\check{\mathcal{P}} + \check{\mathcal{Q}})|\hat{\rho}_{totI}(\vec{\lambda},\vec{\chi},t)\rangle \rangle.
\end{align*}

Al tomar la derivada respecto al tiempo de las ecuaciones \ref{apendix2proyectionev1} y \ref{apendix2proyectionev}, se obtiene la siguiente expresión para la evolución proyectada

\begin{align*}
    \check{\mathcal{P}}|\dot{\hat{\rho}}_{totI}(\vec{\lambda},\vec{\chi},t) \rangle \rangle & = \check{\mathcal{P}}\dot{\check{\mathcal{W}}}(\vec{\lambda},\vec{\chi},t)\check{\mathcal{P}}\check{\mathcal{W}}^{-1}(\vec{\lambda},\vec{\chi},t)\check{\mathcal{P}}|\hat{\rho}_{totI}(\vec{\lambda},\vec{\chi},t)\rangle \rangle \\
     & = \check{\mathcal{P}}\dot{\check{\mathcal{W}}}(\vec{\lambda},\vec{\chi},t)\check{\mathcal{P}}\check{\mathcal{W}}^{-1}(\vec{\lambda},\vec{\chi},t)\check{\mathcal{Q}}|\hat{\rho}_{totI}(\vec{\lambda},\vec{\chi},t)\rangle \rangle 
\end{align*}

\begin{align*}
    \check{\mathcal{Q}}|\dot{\hat{\rho}}_{totI}(\vec{\lambda},\vec{\chi},t) \rangle \rangle & = \check{\mathcal{Q}}\dot{\check{\mathcal{W}}}(\vec{\lambda},\vec{\chi},t)\check{\mathcal{P}}\check{\mathcal{W}}^{-1}(\vec{\lambda},\vec{\chi},t)\check{\mathcal{P}}|\hat{\rho}_{totI}(\vec{\lambda},\vec{\chi},t)\rangle \rangle \\
     & = \check{\mathcal{Q}}\dot{\check{\mathcal{W}}}(\vec{\lambda},\vec{\chi},t)\check{\mathcal{P}}\check{\mathcal{W}}^{-1}(\vec{\lambda},\vec{\chi},t)\check{\mathcal{Q}}|\hat{\rho}_{totI}(\vec{\lambda},\vec{\chi},t)\rangle \rangle, 
\end{align*}
cabe destacar que las ecuaciones obtenidas son exactas. 

Para incorporar la hipótesis de acoplamiento débil entre el sistema y el reservorio, se comenzará utilizando la relación \ref{apendix2Wproperty}, con lo cual se obtiene

\begin{align*}
    \check{\mathcal{P}}\dot{\check{\mathcal{W}}}(\vec{\lambda},\vec{\chi},t)\check{\mathcal{P}}\check{\mathcal{W}}^{-1}(\vec{\lambda},\vec{\chi},t)\check{\mathcal{Q}} & = \epsilon \check{\mathcal{P}}\dot{\check{\mathcal{W}}}_{1}(\vec{\lambda},\vec{\chi},t) \check{\mathcal{P}}\check{\mathcal{Q}} \\
     & + \epsilon^{2} \check{\mathcal{P}}\dot{\check{\mathcal{W}}}_{2}(\vec{\lambda},\vec{\chi},t) \check{\mathcal{P}}\check{\mathcal{Q}} \\
     & - \epsilon^{2}\check{\mathcal{P}}\dot{\check{\mathcal{W}}}_{1}(\vec{\lambda},\vec{\chi},t)\check{\mathcal{P}} \check{\mathcal{W}}_{1}(\vec{\lambda},\vec{\chi},t)\check{\mathcal{Q}},
\end{align*}
tanto el primer como el segundo término se anulan, debido a que \(\check{\mathcal{P}} \check{\mathcal{Q}} = 0\). Por otro lado, 

\begin{align*}
    \check{\mathcal{P}}\dot{\check{\mathcal{W}}}_{1}(\vec{\lambda},\vec{\chi},t)\check{\mathcal{P}} = \sum_{r,r'}|\hat{\rho}^{eq}_{R} \rangle \rangle \langle \langle rr| \check{\mathcal{L}}'_{\lambda}(t)|\hat{\rho}_{R}^{eq}\rangle \rangle \langle \langle r'r'|,
\end{align*}
donde el término \(\langle\langle rr|\check{\mathcal{L}}'_{\lambda}(t)|\hat{\rho}_{R}^{\mathrm{eq}}\rangle\rangle\) equivale a

\begin{equation}
    \langle \langle rr|\check{\mathcal{L}}'_{\lambda}(t)|\hat{\rho}_{R}^{eq}\rangle \rangle  = \text{Tr}_{B}\{\hat{\rho}^{eq}_{R}\hat{V}_{\lambda}(t)\} - \text{Tr}_{B}\{\hat{V}_{-\lambda}(t)\hat{\rho}^{eq}_{R}\}.
    \label{sec3:ecrandom}
\end{equation}

Dado que \(\hat{\rho}_{R}^{\mathrm{eq}}\) conmuta con \(\hat{H}_{R}\), el estado de equilibrio también conmuta con \(\hat{A}(\lambda, \chi)\). Por lo tanto, el término correspondiente a la ecuación \ref{sec3:ecrandom} se anula, quedando únicamente el siguiente término

\begin{align*}
    \check{\mathcal{P}}\dot{\check{\mathcal{W}}}(\vec{\lambda},\vec{\chi},t)\check{\mathcal{P}}\check{\mathcal{W}}^{-1}(\vec{\lambda},\vec{\chi},t)\check{\mathcal{P}} & = \epsilon \check{\mathcal{P}}\dot{\check{\mathcal{W}}}_{1}(\vec{\lambda},\vec{\chi},t)\check{\mathcal{P}} \\
    & + \epsilon^{2}\check{\mathcal{P}}\dot{\check{\mathcal{W}}}_{2}(\vec{\lambda},\vec{\chi},t)\check{\mathcal{P}} \\
    & - \epsilon^{2}\check{\mathcal{P}}\dot{\check{\mathcal{W}}}_{1}(\vec{\lambda},\vec{\chi},t)\check{\mathcal{P}}  \check{\mathcal{W}}_{1}(\vec{\lambda},\vec{\chi},t) \check{\mathcal{P}}.
\end{align*}

De forma explícita, esto corresponde a 

\begin{equation*}
    \check{\mathcal{P}}\dot{\check{\mathcal{W}}}(\vec{\lambda},\vec{\chi},t)\check{\mathcal{P}}\check{\mathcal{W}}^{-1}(\vec{\lambda},\vec{\chi},t)\check{\mathcal{P}} = \epsilon^{2}\check{\mathcal{P}}\int_{0}^{t}ds \check{\mathcal{L}}'_{\lambda}(t)\check{\mathcal{L}}'_{\lambda}(t-s)\check{\mathcal{P}}. 
\end{equation*}

Si se utiliza la relación \(\check{\mathcal{P}}|\dot{\hat{\rho}}_{\mathrm{tot},I}(\vec{\lambda},\vec{\chi},t)\rangle\rangle = |\dot{\hat{\rho}}_{I,S}(\vec{\lambda},\vec{\chi},t)\rangle\rangle \otimes |\hat{\rho}_{R}^{\mathrm{eq}}\rangle\rangle\) y se aplica por la izquierda el operador \(\sum_{r} \langle\langle rr|\), con el objetivo de eliminar los grados de libertad del reservorio, se obtiene

\begin{equation*}
    \dot{\hat{\rho}}_{IS}(\vec{\lambda},\vec{\chi},t) = \epsilon^{2} \sum_{r}\langle \langle rr|\int_{0}^{t}ds \check{\mathcal{L}}'_{\lambda}(t)\check{\mathcal{L}}'_{\lambda}(t-s)|\hat{\rho}_{R}^{eq}\rangle \rangle \hat{\rho}_{IS}(\vec{\lambda},\vec{\chi},t),
\end{equation*}
a partir de esta expresión, es posible derivar las funciones de correlación mediante el cálculo del producto dado por:

\begin{align*}
    \sum_{r}\langle \langle rr| \check{\mathcal{L}}'_{\lambda}(t)\check{\mathcal{L}}'_{\lambda}(t-s)|\hat{\rho}_{R}^{eq}\rangle \rangle \hat{\rho}_{IS}(\vec{\lambda},\vec{\chi},t) = \text{Tr}_{B}\{\mathcal{L}'_{\lambda}(t)\mathcal{L}'_{\lambda}(t-s)\hat{\rho}_{R}^{eq}\hat{\rho}_{IS}(\vec{\lambda},\vec{\chi},t) \},
\end{align*}
desarrollando de manera explícita, los términos de la ecuación 

\begin{equation*}
    \mathcal{L}'_{\lambda}(t-s)\hat{\rho}_{IS}(\vec{\lambda},\vec{\chi},t)\hat{\rho}_{R}^{eq} = -i[\hat{V}_{\lambda}(t-s)\hat{\rho}_{IS}(\vec{\lambda},\vec{\chi},t)\hat{\rho}^{eq}_{R} -  \hat{\rho}_{IS}(\vec{\lambda},\vec{\chi},t)\hat{\rho}^{eq}_{R}\hat{V}_{-\lambda}(t-s)],
\end{equation*}
así, se obtiene 

\begin{align*}
    \mathcal{L}'_{\lambda}(t)\mathcal{L}'_{\lambda}(t-s)\hat{\rho}_{IS}(\vec{\lambda},\vec{\chi},t)\hat{\rho}_{R}^{eq}  = &  -\hat{V}_{\lambda}(t)\hat{V}_{\lambda}(t-s)\hat{\rho}_{IS}(\vec{\lambda},\vec{\chi},t)\hat{\rho}_{R}^{eq} + \hat{V}_{\lambda}(t)\hat{\rho}_{IS}(\vec{\lambda},\vec{\chi},t)\hat{\rho}_{R}^{eq}\hat{V}_{-\lambda}(t-s) \\
    & + \hat{V}_{\lambda}(t-s)\hat{\rho}_{IS}(\vec{\lambda},\vec{\chi},t)\hat{\rho}_{R}^{eq}\hat{V}_{-\lambda}(t) - \hat{\rho}_{IS}(\vec{\lambda},\vec{\chi},t)\hat{\rho}_{R}^{eq}\hat{V}_{-\lambda}(t-s)\hat{V}_{-\lambda}(t).
\end{align*}

Finalmente, al aplicar la traza parcial sobre los grados de libertad del reservorio en esta ecuación, se obtiene la ecuación \ref{ecmaestraVlambda}.


\label{apendixsubsectionmatriz}
    
\newpage
%%%%%%%%%%%%%%%%%%%%%%%%%%%%%%%%%%%%%%%%%%%%%%%%%%%%%%%%
%%%%%%%seccion%%%%%%%%%%%%%%%%%%%%%%%%%%%%%%%%%%%%%%%%%%
%%%%%%%%%%%%%%%%%%%%%%%%%%%%%%%%%%%%%%%%%%%%%%%%%%%%%%%%

\subsection{Funciones correlación}
Para derivar la ecuación maestra generalizada en términos de las funciones de correlación, se partirá de la siguiente expresión

\begin{align*}
& \text{Tr}_{B}\{ \hat{V}_{\lambda}(t)\hat{V}_{\lambda}(t-s) \hat{\rho}_{IS}(\vec{\lambda},\vec{\chi},t)\hat{\rho}^{eq}_{R} \}  =\\
&  - \sum_{\alpha,k,k';j,j'}e^{i(\omega_{j}-\omega_{j'})t}e^{i\omega_{j'}s}\hat{S}^{\dagger}_{\alpha k;j}\hat{S}_{\alpha,k';j'}\hat{\rho}_{IS}(\vec{\lambda},\vec{\chi},t)\text{Tr}_{B}\{e^{-(i/2)\hat{A}(\lambda,\chi)}\hat{B}^{\dagger}_{\alpha,k}(t)\hat{B}_{\alpha,k'}(t-s)e^{(i/2)\hat{A}(\lambda,\chi)}\hat{\rho}^{eq}_{R}\},
\end{align*}    
bajo homogeneidad temporal $\langle \hat{V}_{\lambda}(t)\hat{V}_{\lambda}(t-s) \rangle = \langle \hat{V}_{\lambda}(s)\hat{V}_{\lambda}(0) \rangle$, la ecuación se reduce a 
\begin{align*}
    \text{Tr}_{B}\{ \hat{V}_{\lambda}(t)\hat{V}_{\lambda}(t-s) \hat{\rho}_{IS}(\vec{\lambda},\vec{\chi},t)&\hat{\rho}^{eq}_{R} \}  = \\
    & - \sum_{\alpha,k,k';j,j'}e^{i(\omega_{j}-\omega_{j'})t}e^{i\omega_{j'}s}\hat{S}^{\dagger}_{\alpha k;j}\hat{S}_{\alpha,k';j'}\hat{\rho}_{IS}(\vec{\lambda},\vec{\chi},t)\text{Tr}_{B}\{\hat{B}^{\dagger}_{\alpha,k}(s)\hat{B}_{\alpha,k'}\hat{\tau}_{\alpha} \} \\
    \text{Tr}_{B}\{ \hat{\rho}_{IS}(\vec{\lambda},\vec{\chi},t)\hat{\rho}^{eq}_{R} \hat{V}_{-\lambda}(t-s)&\hat{V}_{-\lambda}(t) \}  = \\
    & - \sum_{\alpha,k,k';j,j'}e^{i(\omega_{j}-\omega_{j'})t}e^{-i\omega_{j}s}\hat{\rho}_{IS}(\vec{\lambda},\vec{\chi},t)\hat{S}^{\dagger}_{\alpha k;j}\hat{S}_{\alpha,k';j'} \text{Tr}_{B}\{ \hat{B}^{\dagger}_{\alpha,k}(s)\hat{B}_{\alpha,k'}\hat{\tau}_{\alpha} \}.   
\end{align*}    

Para analizar el término

\begin{equation*}
    \text{Tr}_{B}\{ \hat{V}_{\lambda}(t)\hat{\rho}_{IS}(\vec{\lambda},\vec{\chi},t)\hat{\rho}_{R}^{eq}\hat{V}_{-\lambda}(t-s) \},
\end{equation*}

se emplea la relación de conmutación (conservación global de partículas)

\begin{equation*}
    [\hat{B}_{\alpha,k},\hat{N}_{\alpha}] = n_{\alpha,k}\hat{B}_{\alpha,k},
\end{equation*}
que implica 

\begin{equation}
    e^{C\hat{N}_{\alpha}}\hat{B}_{\alpha,k}e^{-C\hat{N}_{\alpha}} = e^{-Cn_{\alpha,k}}\hat{B}_{\alpha,k}, 
    \label{apendix:conservationparticle}
\end{equation}
con $C$ alguna constante. Se podrá utilizar la relación \ref{apendix:conservationparticle} para obtener

\begin{align*}
   & \text{Tr}_{B}\{ \hat{V}_{\lambda}(t)\hat{\rho}_{IS}(\vec{\lambda},\vec{\chi},t)\hat{\rho}_{R}^{eq}\hat{V}_{-\lambda}(t-s) \} \\
   & = \sum_{\alpha,k,k';j,j'}e^{i(\omega_{j}-\omega_{j'})t}e^{-i\omega_{j}s}\hat{S}_{\alpha,k';j'}\hat{\rho}_{IS}(\vec{\lambda},\vec{\chi},t)\hat{S}^{\dagger}_{\alpha,k;j} \text{Tr}_{B}\{e^{-i(\hat{A}(\lambda,\chi))} \hat{B}_{\alpha,k}(t)e^{i\hat{A}(\lambda,\chi)}\hat{\rho}_{R}^{eq}\hat{B}^{\dagger}_{\alpha,k'}(t-s) \}.
\end{align*}

Por otro lado, para los operadores del baño se cumple

\begin{align}
    e^{-i(\hat{A}(\lambda,\chi))} \hat{B}_{\alpha,k}(t)e^{i\hat{A}(\lambda,\chi)} & = e^{-i\mu_{\alpha}n_{\alpha,k}(\lambda_{\alpha}-\chi_{\alpha})} \hat{B}_{\alpha,k}(t+\lambda_{\alpha}), 
    \label{apendix:correlationlambda}
\end{align}
utilizando \ref{apendix:correlationlambda}, se obtiene

\begin{align*}
 & \text{Tr}_{B}\{ \hat{V}_{\lambda}(t)\hat{\rho}_{IS}(\vec{\lambda},\vec{\chi},t)\hat{\rho}_{R}^{eq}\hat{V}_{-\lambda}(t-s) \} = \\
 & \sum_{\alpha,k,k';j,j'}e^{i(\omega_{j}-\omega_{j'})t}e^{-i\omega_{j}s}\hat{S}_{\alpha,k';j'}\hat{\rho}_{IS}(\vec{\lambda},\vec{\chi},t)\hat{S}^{\dagger}_{\alpha,k;j} C^{\alpha}_{k,k'}(-s-\lambda_{\alpha})e^{-i\mu_{\alpha}n_{\alpha,k}(\lambda_{\alpha}-\chi_{\alpha})}
 \end{align*}

 \begin{align*}
    & \text{Tr}_{B}\{ \hat{V}_{\lambda}(t-s)\hat{\rho}_{IS}(\vec{\lambda},\vec{\chi},t)\hat{\rho}_{R}^{eq}\hat{V}_{-\lambda}(t) \} = \\
    & \sum_{\alpha,k,k';j,j'}e^{i(\omega_{j}-\omega_{j'})t}e^{i\omega_{j'}s}\hat{S}_{\alpha,k';j'}\hat{\rho}_{IS}(\vec{\lambda},\vec{\chi},t)\hat{S}^{\dagger}_{\alpha,k;j} C^{\alpha}_{k,k'}(s-\lambda_{\alpha})e^{-i\mu_{\alpha}n_{\alpha,k}(\lambda_{\alpha}-\chi_{\alpha})}.
    \end{align*}
   
Finalmente, al sustituir las expresiones desarrolladas para los términos de correlación en la ecuación \ref{ecmaestraVlambda}, se obtiene la ecuación maestra generalizada para el sistema.
\label{finalequation}

\newpage
%%%%%%%%%%%%%%%%%%%%%%%%%%%%%%%%%%%%%%%%%%%%%%%%%%%%%%
%%%%%%%%%%%%%%%%%seccion%%%%%%%%%%%%%%%%%%%%%%%%%%%%%%
%%%%%%%%%%%%%%%%%%%%%%%%%%%%%%%%%%%%%%%%%%%%%%%%%%%%%%

\subsection{Ecuación de Lindblad generalizada}
Se puede desarrollar la ecuación \ref{ecmaestrafinal} para escribir

\begin{multline*}
    \frac{d}{dt}\hat{\rho}_{IS}(\vec{\lambda},\vec{\chi},t) = - \sum_{\alpha,k,k';q} \int_{0}^{\infty}ds \left[e^{i\omega_{q}s}C^{\alpha}_{k,k'}(s) \left(\sum_{j}e^{i\omega_{j}t}\hat{S}^{\dagger}_{\alpha,k;j} \right)\left(\sum_{j'}e^{-i\omega_{j'}t}\hat{S}_{\alpha,k';j'} \right)\hat{\rho}_{IS}(\vec{\lambda},\vec{\chi},t)  \right. \\
    \left. +  e^{-i\omega_{q}s}C^{\alpha}_{k,k'}(-s) \hat{\rho}_{IS}(\vec{\lambda},\vec{\chi},t) \left(\sum_{j}e^{i\omega_{j}t}\hat{S}^{\dagger}_{\alpha,k;j} \right)\left(\sum_{j'}e^{-i\omega_{j'}t}\hat{S}_{\alpha,k';j'} \right)\right. \\
    \left. - e^{-i\mu_{\alpha}n_{\alpha,k}(\lambda_{\alpha}-\chi_{\alpha})}(e^{i\omega_{q}s}C^{\alpha}_{k,k'}(s-\lambda_{\alpha}) + e^{-i\omega_{q}s}C^{\alpha}_{k,k'}(-s-\lambda_{\alpha}) ) \right.\\
    \left.\times \left(\sum_{j'}e^{-i\omega_{j'}t}\hat{S}_{\alpha,k';j'} \right)\hat{\rho}_{IS}(\vec{\lambda},\vec{\chi},t) \left(\sum_{j}e^{i\omega_{j}t}\hat{S}^{\dagger}_{\alpha,k;j} \right)    \right],   
\end{multline*}
Por simplicidad se asumirá que $C_{k,k'}^{\alpha} \propto \delta_{k,k'}$ (para fermiones y bosones los  términos del tipo $\langle \hat{c}^{\dagger}_{k}\hat{c}_{j}\rangle$ suelen ser cero en el baño)\cite{potts2021thermodynamically}. Así, se obtiene 

\begin{multline*}
    \frac{d}{dt}\hat{\rho}_{IS}(\vec{\lambda},\vec{\chi},t) = \\
     - \sum_{\alpha,k;q} \int_{0}^{\infty}ds \left[e^{i\omega_{q}s}C^{\alpha}_{k,k}(s) \hat{S}^{\dagger}_{\alpha,k;q}(t)\hat{S}_{\alpha,k;q}(t)\hat{\rho}_{IS}(\vec{\lambda},\vec{\chi},t)  +  e^{-i\omega_{q}s}C^{\alpha}_{k,k}(-s) \hat{\rho}_{IS}(\vec{\lambda},\vec{\chi},t) \hat{S}^{\dagger}_{\alpha,k;q}(t) \hat{S}_{\alpha,k;q}(t) \right. \\
    \left. - e^{-i\mu_{\alpha}n_{\alpha,k}(\lambda_{\alpha}-\chi_{\alpha})}(e^{i\omega_{q}s}C^{\alpha}_{k,k}(s-\lambda_{\alpha}) + e^{-i\omega_{q}s}C^{\alpha}_{k,k}(-s-\lambda_{\alpha}) ) \hat{S}_{\alpha,k;q}(t) \hat{\rho}_{IS}(\vec{\lambda},\vec{\chi},t) \hat{S}^{\dagger}_{\alpha,k;q}(t)    \right].  
\end{multline*}

Si se usan las identidades

\begin{align*}
    \int_{0}^{\infty}ds e^{i\omega_{q}s}C^{\alpha}_{k,k}(s) & = \int_{-\infty}^{\infty}ds e^{i\omega_{q}s}(1+\text{sgn}(s))C^{\alpha}_{k,k}(s)/2 \\
    & = \int_{-\infty}^{\infty}ds e^{i\omega_{q}s}C^{\alpha}_{k,k}(s)/2 + i \left(-\frac{i}{2} \right) \int_{-\infty}^{\infty}ds e^{i\omega_{q}s} \text{sgn}(s)C^{\alpha}_{k,k}(s) \\
    & = \frac{1}{2}\gamma_{k,k}^{\alpha}(\omega_{q}) + i \Delta^{\alpha}_{k}(\omega_{q}),
\end{align*}

y 

\begin{equation*}
    \int_{0}^{\infty}dse^{i\omega_{q}s}C^{\alpha}_{k,k}(s-\lambda_{\alpha}) = e^{i\lambda_{\alpha}\omega_{q}}\int_{0}^{\infty}e^{i(s-\lambda_{\alpha})\omega_{q}}C^{\alpha}_{k,k}(s-\lambda_{\alpha}), 
\end{equation*}
en la ecuación maestra, finalmente queda la ecuación 

\begin{multline*}
    \frac{d}{dt}\hat{\rho}_{IS}(\vec{\lambda},\vec{\chi},t) = - i \sum_{\alpha,k;q}\Delta^{\alpha}_{k,k}(\omega_{q})\left[\hat{S}^{\dagger}_{\alpha,k;q}(t)\hat{S}_{\alpha,k;q}(t)\hat{\rho}_{IS}(\vec{\lambda},\vec{\chi},t) - \hat{\rho}_{IS}(\vec{\lambda},\vec{\chi},t)\hat{S}^{\dagger}_{\alpha,k;q}(t)\hat{S}_{\alpha,k;q}(t) \right] \\
    + \sum_{\alpha,k;q} \gamma_{k,k}^{\alpha}(\omega_{q})\left[ e^{i\lambda_{\alpha}\omega_{q}+ i(\chi_{\alpha} - \lambda_{\alpha})\mu_{\alpha}n_{\alpha,k}}\hat{S}_{\alpha,k;q}(t)\hat{\rho}_{IS}(\vec{\lambda},\vec{\chi},t)\hat{S}^{\dagger}_{\alpha,k;q}(t) - \frac{1}{2}\{\hat{S}^{\dagger}_{\alpha,k;q}(t)\hat{S}_{\alpha,k;q}(t),\hat{\rho}_{IS}(\vec{\lambda},\vec{\chi},t) \} \right].
\end{multline*}

Que es la ecuación \ref{ecmaestrafinal} que se quiere demostrar.

\label{apendixGKLSgeneral}

\newpage
%%%%%%%%%%%%%%%%%%%%%%%%%%%%%%%%%%%%%%%%%%%%%%%%%%%%%%%
%%%%%%%%%%%%%%%%seccion%%%%%%%%%%%%%%%%%%%%%%%%%%%%%%%%
%%%%%%%%%%%%%%%%%%%%%%%%%%%%%%%%%%%%%%%%%%%%%%%%%%%%%%%


\subsection{Condición KMS y funciones correlación espectral}
Para empezar, se definirá una función correlación auxiliar que consiste en 

\begin{align*}
    C^{\alpha N}_{kk}(s) & =  \langle \hat{B}^{\dagger}_{\alpha,k}(s)\hat{B}_{\alpha,k} \rangle_{N} \\
   &  = \text{Tr}\{e^{is(\hat{H}_{\alpha} - \mu_{\alpha}\hat{N}_{\alpha})}\hat{B}^{\dagger}_{\alpha,k}e^{-is(\hat{H}_{\alpha} - \mu_{\alpha}\hat{N}_{\alpha})}\hat{B}_{\alpha,k}\hat{\tau}_{\alpha}  \} \\
   & = \frac{1}{Z}\text{Tr}\{\hat{B}_{\alpha,k} e^{-\beta_{\alpha}(\hat{H}_{\alpha} - \mu_{\alpha}\hat{N}_{\alpha})} e^{is(\hat{H}_{\alpha} - \mu_{\alpha}\hat{N}_{\alpha})}\hat{B}^{\dagger}_{\alpha,k}e^{-is(\hat{H}_{\alpha} - \mu_{\alpha}\hat{N}_{\alpha})}\} \\
   & = \text{Tr}\{\hat{B}_{\alpha,k}e^{i(s+i\beta_{\alpha})(\hat{H}_{\alpha} - \mu_{\alpha}\hat{N}_{\alpha})}\hat{B}^{\dagger}_{\alpha,k}e^{-i(s+i\beta_{\alpha})(\hat{H}_{\alpha} - \mu_{\alpha}\hat{N}_{\alpha})}\hat{\tau}_{\alpha} \} \\
   & = \langle \hat{B}_{\alpha,k} \hat{B}^{\dagger}_{\alpha,k}(s+i\beta_{\alpha})\rangle_{N},
\end{align*}
que se relaciona con la función correlación, de la siguiente manera 

\begin{align*}
    C^{\alpha N}_{kk}(s) & = \text{Tr}\{e^{-is\mu_{\alpha}\hat{N}_{\alpha} }\hat{B}^{\dagger}_{\alpha,k}(s)e^{is\mu_{\alpha}\hat{N}_{\alpha}} \hat{B}_{\alpha,k}e^{-is\mu_{\alpha}\hat{N}_{\alpha} }\hat{\tau}_{\alpha}   \} \\
 & = \text{Tr}\{\hat{B}^{\dagger}_{\alpha,k}(s)e^{is\mu_{\alpha}\hat{N}_{\alpha}} \hat{B}_{\alpha,k}e^{-is\mu_{\alpha}\hat{N}_{\alpha}}\hat{\tau}_{\alpha}   \}\\
 & = e^{-i\mu_{\alpha}n_{\alpha,k}s} C^{\alpha}_{kk}(s).
\end{align*}

Las funciones correlación espectral, pueden expresarse mediante las funciones correlación auxiliares

\begin{align*}
    \gamma^{\alpha}_{k,k}(\omega) & = \int_{-\infty}^{\infty}ds e^{i\mu_{\alpha}n_{\alpha,k}s} e^{i\omega s}C^{\alpha N}_{k,k}(s) \\
    & = \int_{-\infty}^{\infty}ds e^{i\mu_{\alpha}n_{\alpha,k}s} e^{i\omega s}\langle \hat{B}_{\alpha,k} \hat{B}^{\dagger}_{\alpha,k}(s+i\beta_{\alpha}) \rangle_{N} \\
    & = \int_{-\infty}^{\infty}ds e^{-i(i\beta \mu_{\alpha}n_{\alpha,k})} e^{i\omega s} \langle \hat{B}_{\alpha,k}\hat{B}^{\dagger}_{\alpha,k}(s+i\beta_{\alpha}) \rangle \\
    & = e^{\beta_{\alpha}\mu_{\alpha}n_{\alpha,k}}e^{\beta \omega} \int_{-\infty}^{\infty} ds e^{i\omega(s+i\beta_{\alpha})} \langle \hat{B}_{\alpha,k}\hat{B}^{\dagger}_{\alpha,k}(s+i\beta_{\alpha}) \rangle \\
    & = e^{\beta_{\alpha}(\omega - \mu_{\alpha}n_{\alpha,k})} \gamma^{\alpha}_{k,k}(-\omega).
\end{align*}


Esta relación es importante, pues garantiza el cumplimiento de la condición de balance detallado local. 

\label{apendixKMS}


\newpage
%%%%%%%%%%%%%%%%%%%%%%%%%%%%%%%%%%%%%%%%%%%%%%%%%%%%%%%%
%%%%%%%%%%%%%%%%%%%%%%%%%%%%Leyestermo%%%%%%%%%%%%%%%%%%
%%%%%%%%%%%%%%%%%%%%%%%%%%%%%%%%%%%%%%%%%%%%%%%%%%%%%%%%

\subsection{Ley cero}
Para demostrar la ley cero, se evaluarán los disipadores $\mathcal{D}[\hat{S}_{\alpha,k;q}]$ y $\mathcal{D}[\hat{S}^{\dagger}_{\alpha,k;q}]$ en el estado de Gibbs. Para ello, primero se debe considerar la relación de conmutación

\begin{align*}
    [\hat{S}_{\alpha,k;q},(\hat{H}_{TD} - \mu_{\alpha}\hat{N}_{S})] & = (\omega_{q} - \mu_{\alpha}n_{\alpha,k})\hat{S}_{\alpha,k;q}  \\
    [\hat{S}^{\dagger}_{\alpha,k;q},(\hat{H}_{TD} - \mu_{\alpha}\hat{N}_{S})] & = -(\omega_{q} - \mu_{\alpha}n_{\alpha,k})\hat{S}^{\dagger}_{\alpha,k;q},
\end{align*}
así, encontramos que 

 \begin{align*}
    e^{\beta_{\alpha}(\hat{H}_{TD} - \mu_{\alpha}\hat{N}_{S})}\hat{S}_{\alpha,k;q} e^{-\beta_{\alpha}(\hat{H}_{TD} - \mu_{\alpha}\hat{N}_{S})} & = \hat{S}_{\alpha,k;q}e^{-\beta_{\alpha}(\omega_{q} - \mu_{\alpha}n_{\alpha,k})} \\
    e^{\beta_{\alpha}(\hat{H}_{TD} - \mu_{\alpha}\hat{N}_{S})}\hat{S}^{\dagger}_{\alpha,k;q} e^{-\beta_{\alpha}(\hat{H}_{TD} - \mu_{\alpha}\hat{N}_{S})} & = \hat{S}^{\dagger}_{\alpha,k;q}e^{\beta_{\alpha}(\omega_{q} - \mu_{\alpha}n_{\alpha,k})}.
 \end{align*}

Por lo tanto, al aplicar el estado de Gibbs en los disipadores 

\begin{align*}
    \mathcal{D}[\hat{S}_{\alpha,k;q}]e^{-\beta_{\alpha}(\hat{H}_{TD} - \mu_{\alpha}\hat{N}_{S})} & =  e^{-\beta_{\alpha}(\hat{H}_{TD} - \mu_{\alpha}\hat{N}_{s})} \hat{S}_{\alpha,k;q}\hat{S}^{\dagger}_{\alpha,k;q} e^{-\beta_{\alpha(\omega_{q} - \mu_{\alpha}n_{\alpha,k})}} - e^{-\beta_{\alpha}(\hat{H}_{TD} - \mu_{\alpha}\hat{N}_{S})} \hat{S}^{\dagger}_{\alpha,k;q}\hat{S}_{\alpha,k;q} \\
    \mathcal{D}[\hat{S}^{\dagger}_{\alpha,k;q}]e^{-\beta_{\alpha}(\hat{H}_{TD} - \mu_{\alpha}\hat{N}_{S})} & = e^{-\beta_{\alpha}(\hat{H}_{TD} - \mu_{\alpha}\hat{N}_{s})} \hat{S}^{\dagger}_{\alpha,k;q}\hat{S}_{\alpha,k;q} e^{\beta_{\alpha(\omega_{q} - \mu_{\alpha}n_{\alpha,k})}} - e^{-\beta_{\alpha}(\hat{H}_{TD} - \mu_{\alpha}\hat{N}_{S})} \hat{S}_{\alpha,k;q}\hat{S}^{\dagger}_{\alpha,k;q}, 
\end{align*}
Al sumar las contribuciones de ambos disipadores en el superoperador de Lindblad $\mathcal{L}_{\alpha}$, se obtiene 

\begin{equation*}
    \mathcal{L}_{\alpha}e^{-\beta_{\alpha}(\hat{H}_{TD} - \mu_{\alpha}\hat{N}_{S})} \propto  \mathcal{D}[\hat{S}_{\alpha,k;q}]e^{-\beta_{\alpha}(\hat{H}_{TD} - \mu_{\alpha}\hat{N}_{S})} + e^{-\beta_{\alpha}(\omega_{q} - \mu_{\alpha}n_{\alpha,k})}\mathcal{D}[\hat{S}^{\dagger}_{\alpha,k;q}]e^{-\beta_{\alpha}(\hat{H}_{TD} - \mu_{\alpha}\hat{N}_{S})} = 0.
\end{equation*}

Con lo que se prueba la ley cero.

\newpage

%%%%%%%%%%%%%%%%%%%%%%%%%%%%%%%%%%%%%%%%%%%%%%%%%%%%%%%
%%%%%%%%%%%%%%%%%%%seccion%%%%%%%%%%%%%%%%%%%%%%%%%%%
%%%%%%%%%%%%%%%%%%%%%%%%%%%%%%%%%%%%%%%%%%%%%%%%%%%%%%%

\subsection{Segunda Ley}
Para derivar la Segunda Ley de la termodinámica, se empezará calculando la derivada temporal de la entropía de Von Neumann

\begin{align*}
    - \frac{d}{dt}\text{Tr}\{ \hat{\rho}_{S}(t)\ln \hat{\rho}_{S}(t) \} & =  -\text{Tr}\Big\{ \frac{d}{dt}\hat{\rho}_{S}(t)\ln \hat{\rho}_{S}(t) \Big\} - \frac{d}{dt}\text{Tr}\{\hat{\rho}_{S}(t) \}\\
  & = - i \text{Tr}\{[\hat{H}_{S}+\hat{H}_{LS},\hat{\rho}_{S}(t)]\ln \hat{\rho}_{S}(t)  \} - \sum_{\alpha} \text{Tr}\{(\mathcal{L}_{\alpha}\hat{\rho}_{S}(t)) \ln \hat{\rho}_{S}(t) \}  \\
  & = -\sum_{\alpha}\text{Tr}\{(\mathcal{L}_{\alpha}\hat{\rho}_{S}(t)) \ln \hat{\rho}_{S}(t) \},
\end{align*}
por otro lado, el flujo de calor consiste en 

\begin{align*}
    J_{\alpha} & = \text{Tr}\{ (\hat{H}_{TD} - \mu_{\alpha}\hat{N}_{s})\mathcal{L}_{\alpha}(\hat{\rho}_{S}(t)) \} \\
    & = -\frac{1}{\beta_{\alpha}} \text{Tr}\{(\mathcal{L}_{\alpha}\hat{\rho}_{S}(t)) \ln \hat{\rho}_{G}(\beta_{\alpha},\mu_{\alpha})  \} + \frac{\text{Tr}\{e^{-\beta_{\alpha}(\hat{H}_{TD} - \mu_{\alpha}\hat{N}_{S})} \} }{\beta_{\alpha}} \text{Tr}\{(\mathcal{L}_{\alpha}\hat{\rho}_{S}(t)) \} \\
    & = -\frac{1}{\beta_{\alpha}} \text{Tr}\{(\mathcal{L}_{\alpha}\hat{\rho}_{S}(t))\ln \hat{\rho}_{G}(\beta_{\alpha},\mu_{\alpha})  \},
\end{align*}
Así, utilizando la desigualdad de Spohn se redefine la segunda ley de la termodinámica como

\begin{equation*}
    \dot{\sigma} = - \sum_{\alpha} \text{Tr}\{(\mathcal{L}_{\alpha}\hat{\rho}_{S}(t)) [\ln \hat{\rho}_{S}(t) -\ln \hat{\rho}_{G}(\beta_{\alpha},\mu_{\alpha}) ] \} \geq 0.
\end{equation*}

Finalmente, se obtiene que la tasa de producción de entropía es siempre mayor igual a cero.

\label{apendix:thermolaws}
	% Imagen, se numerará automáticamente con la letra del anexo según
	% la configuración \appendixindepobjnum

\newpage 

    \section{Cálculos realizados sección 3}
    \subsection{Producción de entropía y información}
    Partiendo de la definición de la entropía conjunta para dos variables aleatorias discretas \( X \) e \( Y \)

\begin{equation}
    S^{XY} = - \sum_{x,y} p(x,y) \ln p(x,y),
    \label{apendixSxy}
\end{equation}

donde \( p(x,y) \) denota la distribución de probabilidad conjunta de \( X \) e \( Y \). Al derivar temporalmente la expresión \ref{apendixSxy} y utilizar la propiedad de antisimetría de la corriente de probabilidades, \( J_{x,x'}^{y,y'} = - J_{x',x}^{y',y} \), se obtiene


    \begin{align*}
        \partial_{t}S^{XY} & = - \sum_{x,y} \dot{p}(x,y) \ln p(x,y) - \sum_{x,y} \dot{p}(x,y) \\
                           & = - \sum_{x,x';y,y'} J_{x,x'}^{y,y'} \ln p(x,y)  \\
                           & = \sum_{x \geq x'; y\geq y'} J_{x,x'}^{y,y'} \ln \frac{p(x',y')}{p(x,y)} \\
                           & = \sum_{x \geq x'; y\geq y'} J_{x,x'}^{y,y'} \ln \frac{W_{x,x'}^{y,y'} p(x',y')}{W_{x',x}^{y',y} p(x,y)} +  \sum_{x \geq x'; y\geq y'} J_{x,x'}^{y,y'} \ln \frac{W_{x',x}^{y',y} }{W_{x,x'}^{y,y'} } \\
                           & = \dot{\sigma} - \dot{S}_{r}.
    \end{align*}

Para demostrar que la producción de entropía es no negativa, obsérvese que si \( J_{x,x'}^{y,y'} > 0 \), entonces se cumple que 
\[
W_{x,x'}^{y,y'} p(x',y') > W_{x',x}^{y',y} p(x,y),
\]
y por lo tanto, el cociente
\[
\frac{W_{x,x'}^{y,y'} p(x',y')}{W_{x',x}^{y',y} p(x,y)} > 1,
\]
lo que implica que su logaritmo es positivo. Como el término correspondiente a la producción de entropía es proporcional a \( J_{x,x'}^{y,y'} \ln \left( \frac{W_{x,x'}^{y,y'} p(x',y')}{W_{x',x}^{y',y} p(x,y)} \right) \), cada uno de estos términos contribuye positivamente, y por simetría de los índices se concluye que la suma total es mayor o igual a cero.

Por su parte, la derivada temporal de la información mutua se expresa como

\begin{equation*}
    \partial_{t} I_{XY} = \sum_{x,y} \dot{p}(x,y) \ln \frac{p(x,y)}{p(x)p(y)} + \sum_{x,y} p(x,y) \frac{\partial}{\partial t} \left( \ln \frac{1}{p(x)p(y)} \right),
\end{equation*}

donde, al derivar el logaritmo, se obtiene

\begin{align*}
    \frac{\partial}{\partial t} \left( \ln \frac{1}{p(x)p(y)} \right) 
    &= - \left( \frac{\dot{p}(x)}{p(x)} + \frac{\dot{p}(y)}{p(y)} \right),
\end{align*}

por lo que

\begin{align*}
    \partial_{t} I_{XY} &= \sum_{x,y} \dot{p}(x,y) \ln \frac{p(x,y)}{p(x)p(y)} 
    - \sum_{x,y} p(x,y) \left( \frac{\dot{p}(x)}{p(x)} + \frac{\dot{p}(y)}{p(y)} \right).
\end{align*}

Reescribiendo los últimos términos con ayuda de la identidad \(\sum_{y} p(x,y) = p(x)\) y la conservación de la probabilidad (\(\sum_{x} \dot{p}(x) = 0\)), se llega a

\begin{align}
    \partial_{t} I_{XY} 
    &= \sum_{x,y} \dot{p}(x,y) \ln \frac{p(x,y)}{p(x)p(y)}.
\end{align}

Utilizando la definición de corriente de probabilidad \(J_{x,x'}^{y,y'}\), se puede reescribir como

\begin{align}
    \partial_{t} I_{XY} &= \sum_{x,x';y,y'} J_{x,x'}^{y,y'} \ln \frac{p(x,y)}{p(x)p(y)}.
\end{align}

Separando la dinámica de \(X\) e \(Y\), se obtiene

\begin{align*}
    \partial_{t} I_{XY} &= \sum_{x,x';y} J_{x,x'}^{y} \ln \frac{p(x,y)}{p(x)p(y)} 
    + \sum_{x;y,y'} J_{x}^{y,y'} \ln \frac{p(x,y)}{p(x)p(y)}.
\end{align*}

Agrupando términos simétricos mediante pares ordenados \(x \geq x'\) e \(y \geq y'\), se llega a

\begin{align*}
    \partial_{t} I_{XY} &= \sum_{x \geq x';y} J_{x,x'}^{y} \left[ \ln \frac{p(x,y)}{p(x)p(y)} - \ln \frac{p(x',y)}{p(x')p(y)} \right] \nonumber \\
    &\quad + \sum_{x;y \geq y'} J_{x}^{y,y'} \left[ \ln \frac{p(x,y)}{p(x)p(y)} - \ln \frac{p(x,y')}{p(x)p(y')} \right].
\end{align*}

Utilizando la regla de Bayes \(p(x,y) = p(x)p(y|x) = p(y)p(x|y)\), se simplifica como

\begin{align*}
    \partial_{t} I_{XY} 
    &= \sum_{x \geq x';y} J_{x,x'}^{y} \ln \frac{p(y|x)}{p(y|x')} 
     + \sum_{x;y \geq y'} J_{x}^{y,y'} \ln \frac{p(x|y)}{p(x|y')}.
\end{align*}

Esta expresión permite interpretar la variación de la información mutua como la suma de dos flujos:

\begin{equation*}
    \partial_{t} I_{XY} = \dot{I}^{X} + \dot{I}^{Y},
\end{equation*}

donde \(\dot{I}^{X}\) e \(\dot{I}^{Y}\) estan dados por \ref{sec4:clasicalflows0} y \ref{sec4:clasicalflows}, y representan contribuciones individuales debido a la dinámica en los grados de libertad \(X\) e \(Y\), respectivamente.


    \label{apendix4:secondlaw}

	 % Desactiva el color de celda

\newpage

\section{Cálculos realizados seccion 4}
\subsection{Funciones de correlación para un baño de fermiones libres}
Para obtener las funciones de correlación espectral, se comenzará escribiendo las funciones correlación temporal 

\begin{equation*}
    C_{1}^{\alpha}(s) = \text{Tr}[e^{is\hat{H}_{\alpha}}\hat{B}^{\dagger}_{\alpha,1}e^{-is\hat{H}_{\alpha}}\hat{B}_{\alpha,1}\hat{\tau}_{\alpha}  ]  \hspace{10mm} C_{-1}^{\alpha}(s) = \text{Tr}[e^{is\hat{H}_{\alpha}}\hat{B}^{\dagger}_{\alpha,-1}e^{-is\hat{H}_{\alpha}}\hat{B}_{\alpha,-1}\hat{\tau}_{\alpha}  ], 
\end{equation*}
Reescribiendo las funciones correlación, en función de los operadores fermionicos

\begin{equation*}
    C_{1}^{\alpha}(s) = \sum_{l,l'}t_{\alpha,l}t_{\alpha,l'}e^{i\epsilon_{\alpha,l}s} \langle \hat{c}^{\dagger}_{\alpha,l}\hat{c}_{\alpha,l'} \rangle \hspace{10mm} C
    ^{\alpha}_{-1}(s) = \sum_{l,l'}t_{\alpha,l}t_{\alpha,l'}e^{-i\epsilon_{\alpha,l}s} \langle \hat{c}_{\alpha,l}\hat{c}^{\dagger}_{\alpha,l'} \rangle.
\end{equation*}

En el equilibrio gran canónico, el valor de expectación $\langle \hat{c}^{\dagger}_{\alpha,l}\hat{c}_{\alpha,l}\rangle$ es la distribución de Fermi $f_{\alpha}(\epsilon_{\alpha,l})$. Por lo tanto  

\begin{equation*}
    C_{1}^{\alpha}(s) = \sum_{l}t^{2}_{\alpha,l}e^{i\epsilon_{\alpha,l}s} f_{\alpha}(\epsilon_{\alpha,l}) \hspace{10mm} C
    ^{\alpha}_{-1}(s) = \sum_{l}t^{2}_{\alpha,l}e^{-i\epsilon_{\alpha,l}s} [1-f_{\alpha}(\epsilon_{\alpha,l})],
\end{equation*}
luego, si se define la tasa de túnel correspondiente al baño \( \alpha \) como

\begin{equation*}
    \kappa_{\alpha}(\omega) = 2\pi \sum_{l} |t_{\alpha,l}|^{2} \delta(\omega - \epsilon_{\alpha,l}),
\end{equation*}

se pueden reescribir las funciones de correlación en forma integral en términos de dicha tasa y de la distribución de ocupación del reservorio \( f_\alpha(\omega) \)

\begin{align*}
    C_{1}^{\alpha}(s) &= \frac{1}{2\pi} \int_{-\infty}^{\infty} d\omega\, e^{i\omega s}\, \kappa_{\alpha}(\omega) f_{\alpha}(\omega), \\
    C_{-1}^{\alpha}(s) &= \frac{1}{2\pi} \int_{-\infty}^{\infty} d\omega\, e^{-i\omega s}\, \kappa_{\alpha}(\omega)\big[1 - f_{\alpha}(\omega)\big].
\end{align*}

A partir de estas expresiones temporales, se puede obtener la forma espectral de las funciones de correlación a través de la transformada de Fourier

\begin{align*}
    \gamma_{1}^{\alpha}(\omega) &= \int_{-\infty}^{\infty} ds\, e^{i \omega s} C_{1}^{\alpha}(s) = \kappa_{\alpha}(-\omega) f_{\alpha}(-\omega), \\
    \gamma_{-1}^{\alpha}(\omega) &= \int_{-\infty}^{\infty} ds\, e^{i \omega s} C_{-1}^{\alpha}(s) = \kappa_{\alpha}(\omega)\big[1 - f_{\alpha}(\omega)\big].
\end{align*}

Estas funciones espectrales codifican tanto la densidad de estados accesibles del entorno como su grado de ocupación.

Más adelante, se abordará en detalle el régimen Markoviano, discutiendo bajo qué condiciones las funciones de correlación se aproximan por distribuciones localizadas temporalmente, y cómo esto permite simplificar el tratamiento dinámico del sistema.

\label{apendix5bathcorre}

\subsection{ Operadores de salto de sistema de 3 puntos cuánticos}
Para partir, se considerará un sistema de 3 puntos cuánticos, con Hamiltoniano $\hat{H}_{S}$

\begin{equation*}
    \hat{H}_{S} = \epsilon_{R}\hat{n}_{R} + \epsilon_{L}\hat{n}_{L} + \epsilon_{D}\hat{n}_{D}  + g(\hat{d}^{\dagger}_{R}\hat{d}_{L} + \hat{d}^{\dagger}_{L}\hat{d}_{R}) + U \hat{n}_{D}(\hat{n}_{R} + \hat{n}_{L}) + U_{f}\hat{n}_{R}\hat{n}_{L},
\end{equation*}

Para identificar los operadores de salto, primeramente se elimina el término de acoplamiento en el Hamiltoniano $\hat{H}_{S}$, mediante la transformación unitaria

\begin{align*}
    \hat{d}_{-} & = \cos(\theta/2)\hat{d}_{R} - \sin(\theta/2)\hat{d}_{L} \\
    \hat{d}_{+} & = \sin(\theta/2)\hat{d}_{R} + \cos(\theta/2)\hat{d}_{L},
\end{align*}
con 

\begin{equation*}
    \cos \theta = \Delta/\sqrt{ \Delta^{2} + g^{2} } \hspace{10mm} \Delta = \frac{(\epsilon_{L}-\epsilon_{R})}{2}.
\end{equation*}

Si se invierte la transformación unitaria, se pueden escribir los operadores locales en función de los operadores globales 

\begin{align*}
    \hat{d}_{R} & = \cos(\theta/2)\hat{d}_{-} + \sin(\theta/2)\hat{d}_{+} \\
    \hat{d}_{L} & = -\sin(\theta/2)\hat{d}_{-} + \cos(\theta/2)\hat{d}_{+}.
\end{align*}

Esto implica que los operadores de número locales se pueden expresar como

\begin{align*}
    \hat{d}^{\dagger}_{R}\hat{d}_{R} & = \cos^{2}(\theta/2) \hat{d}^{\dagger}_{-}\hat{d}_{-} + \sin^{2}(\theta/2) \hat{d}^{\dagger}_{+}\hat{d}_{+} + \cos(\theta/2)\sin(\theta/2)[\hat{d}^{\dagger}_{+}\hat{d}_{-} + \hat{d}^{\dagger}_{-}\hat{d}_{+} ] \\
    \hat{d}^{\dagger}_{L}\hat{d}_{L} & = \sin^{2}(\theta/2) \hat{d}^{\dagger}_{-}\hat{d}_{-} + \cos^{2}(\theta/2) \hat{d}^{\dagger}_{+}\hat{d}_{+} - \cos(\theta/2)\sin(\theta/2)[\hat{d}^{\dagger}_{+}\hat{d}_{-} + \hat{d}^{\dagger}_{-}\hat{d}_{+} ],
\end{align*}
por lo tanto 

\begin{equation}
    \hat{n}_{L} + \hat{n}_{R} = \hat{n}_{+} + \hat{n}_{-}.
    \label{apendixnumber}
\end{equation}

Usando la relación \ref{apendixnumber} y la cantidad definida por $\bar{\epsilon} = (\epsilon_{R} + \epsilon_{L})/2$, se obtiene 

\begin{equation}
    \epsilon_{R} \hat{n}_{R} + \epsilon_{L} \hat{n}_{L}  = \bar{\epsilon}( \hat{n}_{+} + \hat{n}_{-} ) - \Delta \sin(\theta) [\hat{d}^{\dagger}_{+}\hat{d}_{-} + \hat{d}^{\dagger}_{-}\hat{d}_{+}] - \Delta \cos(\theta) [\hat{d}^{\dagger}_{-}\hat{d}_{-} - \hat{d}^{\dagger}_{+}\hat{d}_{+}].
\label{apendix5:ec1}
\end{equation}

Complementariamente

\begin{equation}
    \hat{d}^{\dagger}_{R}\hat{d}_{L} = \cos^{2}(\theta/2)\hat{d}^{\dagger}_{+}\hat{d}_{-} - \sin^{2}(\theta/2) \hat{d}^{\dagger}_{-}\hat{d}_{+}  + \sin(\theta/2)\cos(\theta/2)[ \hat{d}^{\dagger}_{+}\hat{d}_{+} - \hat{d}^{\dagger}_{-}\hat{d}_{-} ]
    \label{apendix5:ec2}
\end{equation}

\begin{equation}
    \hat{d}^{\dagger}_{L}\hat{d}_{R} = \cos^{2}(\theta/2)\hat{d}^{\dagger}_{-}\hat{d}_{+} - \sin^{2}(\theta/2) \hat{d}^{\dagger}_{+}\hat{d}_{-}  + \sin(\theta/2)\cos(\theta/2)[ \hat{d}^{\dagger}_{+}\hat{d}_{+} - \hat{d}^{\dagger}_{-}\hat{d}_{-} ].
    \label{apendix5:ec3}
\end{equation}

De la combinación de \ref{apendix5:ec1}, \ref{apendix5:ec2} y \ref{apendix5:ec3} se puede mostrar que 

\begin{equation*}
    \epsilon_{R} \hat{n}_{R} + \epsilon_{L} \hat{n}_{L} = (\bar{\epsilon} + \sqrt{\Delta^{2} + g^{2}})\hat{n}_{+} +  (\bar{\epsilon} - \sqrt{\Delta^{2} + g^{2}})\hat{n}_{-}.
\end{equation*}

Para trabajar el término de Coulomb en el Hamiltoniano, se aplicará que los operadores de Fermi cumplen $\hat{n}^{2}_{i} = \hat{n}_{i}$, así el término de Coulomb 

\begin{align*}
    2 \hat{n}_{R}\hat{n}_{L} & = (\hat{n}_{R} +\hat{n}_{L})(\hat{n}_{R} +\hat{n}_{L}) - (\hat{n}_{R} + \hat{n}_{L}) \\
    & = (\hat{n}_{+} +\hat{n}_{-})(\hat{n}_{+} +\hat{n}_{-}) - (\hat{n}_{+} + \hat{n}_{-})  = 2\hat{n}_{+}\hat{n}_{-}.
\end{align*}

Finalmente, se definen las energías $\epsilon_{\pm} = \bar{\epsilon} \pm \sqrt{\Delta^{2}+g^{2}}$ para escribir el Hamiltoniano en la base global, dado por 

\begin{equation}
    \hat{H}_{S} = \epsilon_{D}\hat{n}_{D} + \epsilon_{+}\hat{n}_{+} + \epsilon_{-}\hat{n}_{-} + U\hat{n}_{D}(\hat{n}_{+} + \hat{n}_{-}) + U_{f}\hat{n}_{+}\hat{n}_{-}.
    \label{apendix5:ec4}
\end{equation}

El escribir el Hamiltoniano en función de la base global, permite calcular los operadores de salto, que están dados por 

\begin{align*}
    e^{i \hat{H}_{S}t}\hat{d}_{D}e^{-i\hat{H}_{S}t} & = \hat{d}_{D} + it[\hat{H}_{S},\hat{d}_{D}] + \frac{(it)^{2}}{2} [\hat{H}_{S},[\hat{H}_{S},\hat{d}_{D}]] +... \\
    e^{i \hat{H}_{S}t}\hat{d}_{+}e^{-i\hat{H}_{S}t} & = \hat{d}_{+} + it[\hat{H}_{S},\hat{d}_{+}] + \frac{(it)^{2}}{2} [\hat{H}_{S},[\hat{H}_{S},\hat{d}_{+}]] +... \\
    e^{i \hat{H}_{S}t}\hat{d}_{-}e^{-i\hat{H}_{S}t} & = \hat{d}_{-} + it[\hat{H}_{S},\hat{d}_{-}] + \frac{(it)^{2}}{2} [\hat{H}_{S},[\hat{H}_{S},\hat{d}_{-}]] +....
\end{align*}

Para calcular los operadores de salto correspondientes al operador $\hat{d}_{D}$, se usará la identidad

\begin{equation*}
    \textbf{1} = (\textbf{1} - \hat{n}_{+})(\textbf{1}-\hat{n}_{-}) + (\textbf{1} - \hat{n}_{+})\hat{n}_{-} + (\textbf{1} - \hat{n}_{-})\hat{n}_{+} + \hat{n}_{+}\hat{n}_{-},
\end{equation*}
al utilizar esta identidad y la relación $(\textbf{1} - \hat{n}_{i})\hat{n}_{i} = \textbf{0}$. El conmutador $[\hat{H}_{S},\hat{d}_{D}]$ se puede separar en los términos

\begin{align*}
    [\hat{H}_{S},\hat{d}_{D}(\textbf{1}-\hat{n}_{+})(\textbf{1} - \hat{n}_{-})] & = - \epsilon_{D}\hat{d}_{D}(\textbf{1}-\hat{n}_{+})(\textbf{1} - \hat{n}_{-}) \\
    [\hat{H}_{S},\hat{d}_{D}(\textbf{1}-\hat{n}_{+})\hat{n}_{-}] & = - (\epsilon_{D} + U)\hat{d}_{D}(\textbf{1} - \hat{n}_{+})\hat{n}_{-} \\
    [\hat{H}_{S},\hat{d}_{D}(\textbf{1}-\hat{n}_{-})\hat{n}_{+}] & = - (\epsilon_{D} + U)\hat{d}_{D}(\textbf{1} - \hat{n}_{-})\hat{n}_{+} \\
    [\hat{H}_{S},\hat{d}_{D}\hat{n}_{+}\hat{n}_{-}] & = - (\epsilon_{D} + 2U)\hat{d}_{D}\hat{n}_{+}\hat{n}_{-}, 
\end{align*}
al aplicar nuevamente el conmutador

\begin{align*}
    [\hat{H}_{S},[\hat{H}_{S},\hat{d}_{D}]] & = (\epsilon_{D})^{2}\hat{d}_{D}(\textbf{1} - \hat{n}_{+}) (\textbf{1} - \hat{n}_{-}) + (\epsilon_{D} + U)^{2}\hat{d}_{D}(\textbf{1} - \hat{n}_{+})\hat{n}_{-} \\
        & + (\epsilon_{D}+U)^{2}\hat{d}_{D}(\textbf{1} - \hat{n}_{-})\hat{n}_{+} + (\epsilon_{D} + 2U)^{2}\hat{d}_{D}\hat{n}_{+}\hat{n}_{-}.
\end{align*}

Finalmente 

\begin{align*}
    e^{i \hat{H}_{S}t}\hat{d}_{D}e^{-i\hat{H}_{S}t}  & = e^{-i\epsilon_{D}t} \hat{d}_{D}(\textbf{1} - \hat{n}_{+}) (\textbf{1} - \hat{n}_{-}) +  e^{-i(\epsilon_{D}+U)t} \hat{d}_{D}[(\textbf{1} - \hat{n}_{+})\hat{n}_{-} + (\textbf{1} - \hat{n}_{-})\hat{n}_{+}] \\
    & + e^{-i(\epsilon_{D} + 2U)t}\hat{d}_{D} \hat{n}_{+}\hat{n}_{-}.
\end{align*}

Para calcular los operadores de salto de $\hat{d}_{+}$, se usará la identidad 

\begin{equation*}
    \textbf{1} = (\textbf{1}-\hat{n}_{D})(\textbf{1}-\hat{n}_{-}) + (\textbf{1}-\hat{n}_{-})\hat{n}_{D} + (\textbf{1}-\hat{n}_{D})\hat{n}_{-} + \hat{n}_{D}\hat{n}_{-},
\end{equation*}
así, los conmutadores que incluyen $\hat{d}_{+}$ se podrán separar en 

\begin{align*}
    [\hat{H}_{S},\hat{d}_{+}(\textbf{1}-\hat{n}_{D})(\textbf{1} - \hat{n}_{-})] & = - \epsilon_{+}\hat{d}_{+}(\textbf{1}-\hat{n}_{D})(\textbf{1} - \hat{n}_{-}) \\
    [\hat{H}_{S},\hat{d}_{+}(\textbf{1}-\hat{n}_{D})\hat{n}_{-}] & = - (\epsilon_{+} + U)\hat{d}_{+}(\textbf{1} - \hat{n}_{-})\hat{n}_{D} \\
    [\hat{H}_{S},\hat{d}_{+}(\textbf{1}-\hat{n}_{-})\hat{n}_{D}] & = - (\epsilon_{+} + U_{f})\hat{d}_{+}(\textbf{1} - \hat{n}_{D})\hat{n}_{-} \\
    [\hat{H}_{S},\hat{d}_{+}\hat{n}_{D}\hat{n}_{-}] & = - (\epsilon_{+} + U + U_{f})\hat{d}_{+}\hat{n}_{D}\hat{n}_{-}, 
\end{align*}
aplicando recursivamente estos conmutadores, se obtiene la expresión para los operadores de salto 

\begin{align*}
    e^{i \hat{H}_{S}t}\hat{d}_{+}e^{-i\hat{H}_{S}t} & = e^{-i\epsilon_{+}t}\hat{d}_{+}(\textbf{1}-\hat{n}_{D})(\textbf{1}-\hat{n}_{-}) + e^{-i(\epsilon_{+}+U_{f})t} \hat{d}_{+}(\textbf{1}-\hat{n}_{D})\hat{n}_{-} \\
    & + e^{-i(\epsilon_{+}+U)t}\hat{d}_{+}(\textbf{1}-\hat{n}_{-})\hat{n}_{D} + e^{-i(\epsilon_{+}+U+U_{f})t}\hat{d}_{+}\hat{n}_{-}\hat{n}_{D},
\end{align*}
debido a la simetría entre los operadores de la base global $\hat{d}_{+}$ y $\hat{d}_{-}$, los operadores de salto de $\hat{d}_{-}$ son

\begin{align*}
    e^{i \hat{H}_{S}t}\hat{d}_{-}e^{-i\hat{H}_{S}t} & = e^{-i\epsilon_{-}t}\hat{d}_{-}(\textbf{1}-\hat{n}_{D})(\textbf{1}-\hat{n}_{+}) + e^{-i(\epsilon_{-}+U_{f})t} \hat{d}_{-}(\textbf{1}-\hat{n}_{D})\hat{n}_{+} \\
    & + e^{-i(\epsilon_{-}+U)t}\hat{d}_{-}(\textbf{1}-\hat{n}_{+})\hat{n}_{D} + e^{-i(\epsilon_{-}+U+U_{f})t}\hat{d}_{-}\hat{n}_{+}\hat{n}_{D}.
\end{align*}

Por su parte, se pueden expresar los operadores de salto en la base local $\hat{d}_{L}$ y $\hat{d}_{R}$ como 
\begin{align*}
    e^{i\hat{H}_{S}t}\hat{d}_{R}e^{-i\hat{H}_{S}t} & = \cos(\theta/2)\, e^{i\hat{H}_{S}t}\hat{d}_{-}e^{-i\hat{H}_{S}t} + \sin(\theta/2)\, e^{i\hat{H}_{S}t}\hat{d}_{+}e^{-i\hat{H}_{S}t}, \\
    e^{i\hat{H}_{S}t}\hat{d}_{L}e^{-i\hat{H}_{S}t} & = -\sin(\theta/2)\, e^{i\hat{H}_{S}t}\hat{d}_{-}e^{-i\hat{H}_{S}t} + \cos(\theta/2)\, e^{i\hat{H}_{S}t}\hat{d}_{+}e^{-i\hat{H}_{S}t}.
\end{align*}

A partir de esto, se identifican las frecuencias de Bohr asociadas a los operadores de creación y destrucción en el sistema

\begin{align*}
    \hat{d}_{D} & \rightarrow \{\epsilon_{D}, \epsilon_{D}+U, \epsilon_{D}+2U\}, \\
    \hat{d}_{L}, \hat{d}_{R} & \rightarrow \{\epsilon_{\pm}, \epsilon_{\pm}+U, \epsilon_{\pm}+U_{f}, \epsilon_{\pm}+U+U_{f}\}.
\end{align*}



\label{apendix5jumpop}

\subsection{Agrupación de frecuencias para 2 puntos cuánticos}
En el caso degenerado $\epsilon_{L} = \epsilon_{R}= \epsilon$ se tiene que $\epsilon_{\pm} = \epsilon \pm g$. En este régimen, si el parámetro de acoplamiento $g$ es pequeño en comparación con el inverso del tiempo de correlación del baño($2g\ll 1/\tau_{B}$), es posible agrupar estas dos frecuencias. La agrupación de frecuencias estará dada por 

\begin{align*}
    &(\epsilon_{+},\epsilon_{-})  \to \epsilon \\
    &(\epsilon_{+}+U,\epsilon_{-}+U)  \to \epsilon+U \\
    &(\epsilon_{+}+U_{f},\epsilon_{-}+U_{f})  \to \epsilon+U_{f} \\
    &(\epsilon_{+}+U+U_{f},\epsilon_{-}+U+U_{f})  \to \epsilon+U+U_{f}.
\end{align*}

Para determinar los operadores de salto asociados al operador $\hat{d}_{L}$ en el caso degenerado, se debe considerar la contribución de las transiciones con frecuencias $\epsilon_{+}$ y $\epsilon_{-}$. Es decir, se deben sumar las contribuciones correspondientes a los operadores $\hat{d}_{+}$ y $\hat{d}_{-}$, para formar un único operador efectivo de salto que posee la frecuencia agrupada. Por ende se debe calcular 

\begin{align*}
    \frac{\hat{d}_{+}(\textbf{1}-\hat{n}_{D})(\textbf{1}-\hat{n}_{-})}{\sqrt{2}} - \frac{\hat{d}_{-}(\textbf{1}-\hat{n}_{D})(\textbf{1}-\hat{n}_{+})}{\sqrt{2}} & = \frac{(\textbf{1}-\hat{n}_{D})}{\sqrt{2}} [\hat{d}_{+}(\textbf{1}-\hat{n}_{-}) - \hat{d}_{-}(\textbf{1}-\hat{n}_{+}) ],
\end{align*}
además 

\begin{align}
    \hat{d}_{+}(\textbf{1}-\hat{n}_{-}) - \hat{d}_{-}(\textbf{1}-\hat{n}_{+}) & = \frac{\hat{d}_{L} + \hat{d}_{R}}{\sqrt{2}} (\textbf{1} - \hat{n}_{-}) - \frac{\hat{d}_{R} - \hat{d}_{L}}{\sqrt{2}}(\textbf{1}-\hat{n}_{+}) \\
    & = \frac{\hat{d}_{L}}{\sqrt{2}}(\textbf{2} - \hat{n}_{+} -\hat{n}_{-} ) + \frac{\hat{d}_{R}}{\sqrt{2}}(\hat{n}_{+} -\hat{n}_{-}) \\
    & = \frac{\hat{d}_{L}}{\sqrt{2}}(\textbf{2}-\hat{n}_{L} -\hat{n}_{R}) + \frac{\hat{d}_{R}}{\sqrt{2}}(\hat{d}^{\dagger}_{R}\hat{d}_{L}+\hat{d}^{\dagger}_{L}\hat{d}_{R}),
    \label{apendix5:ec5x}
\end{align}
Utilizando $\hat{d}_{L}(\mathbf{1} - \hat{n}_{L}) = \hat{d}_{L}\hat{d}_{L}\hat{d}^{\dagger}_{L} = 0$ y 

\[
\hat{d}_{R}(\hat{d}^{\dagger}_{R}\hat{d}_{L} + \hat{d}^{\dagger}_{L}\hat{d}_{R}) = \hat{d}_{L}(\mathbf{1} - \hat{n}_{R}),
\]
en \ref{apendix5:ec5x}, se obtiene la identidad

\[
\hat{d}_{+}(\mathbf{1} - \hat{n}_{-}) - \hat{d}_{-}(\mathbf{1} - \hat{n}_{+}) = \hat{d}_{L}(\mathbf{1} - \hat{n}_{R}),
\]

lo que permite asociar la frecuencia $\epsilon$ al operador $\hat{d}_{L}(\mathbf{1} - \hat{n}_{D})(\mathbf{1} - \hat{n}_{R})$.

Del mismo modo, para las frecuencias agrupadas $(\epsilon_{-} + U,\, \epsilon_{+} + U) \to \epsilon + U$, el operador de salto efectivo queda determinado por

\[
\frac{\hat{d}_{+}\hat{n}_{D}(\mathbf{1} - \hat{n}_{-})}{\sqrt{2}} - \frac{\hat{d}_{-}\hat{n}_{D}(\mathbf{1} - \hat{n}_{+})}{\sqrt{2}} = \hat{d}_{L}\hat{n}_{D}(\mathbf{1} - \hat{n}_{R}).
\]

Para el caso de frecuencias $(\epsilon_{-} + U_{f},\, \epsilon_{+} + U_{f}) \to \epsilon + U_{f}$, los operadores de salto agrupados son

\[
\frac{\hat{d}_{+}\hat{n}_{-}(\mathbf{1} - \hat{n}_{D})}{\sqrt{2}} - \frac{\hat{d}_{-}\hat{n}_{+}(\mathbf{1} - \hat{n}_{D})}{\sqrt{2}} = \frac{(\mathbf{1} - \hat{n}_{D})}{\sqrt{2}}\left[\hat{d}_{+}\hat{n}_{-} - \hat{d}_{-}\hat{n}_{+}\right],
\]
donde la cantidad $\left[\hat{d}_{+}\hat{n}_{-} - \hat{d}_{-}\hat{n}_{+}\right]$ es 

\begin{align*}
    \hat{d}_{+}\hat{n}_{-} -\hat{d}_{-}\hat{n}_{+} & = \frac{\hat{d}_{R}+\hat{d}_{L}}{\sqrt{2}}\hat{n}_{-} - \frac{\hat{d}_{R}-\hat{d}_{L}}{\sqrt{2}}\hat{n}_{+} \\
    & = - \frac{\hat{d}_{R}}{\sqrt{2}}(\hat{d}^{\dagger}_{R}\hat{d}_{L}+ \hat{d}^{\dagger}_{L}\hat{d}_{R}) + \frac{\hat{d}_{L}}{\sqrt{2}}(\hat{n}_{L}+\hat{n}_{R}) \\
    & = \frac{2\hat{d}_{L}}{\sqrt{2}}\hat{n}_{R} - \frac{\hat{d}_{L}}{\sqrt{2}}(\textbf{1}-\hat{n}_{L}) \\
    & = \frac{2\hat{d}_{L}}{\sqrt{2}}\hat{n}_{R}.
\end{align*}

Lo que permite asociar de forma coherente la frecuencia $\epsilon + U_{f}$ al operador $\hat{d}_{L}\hat{n}_{R}(\mathbf{1}-\hat{n}_{D})$. 

Finalmente, para las frecuencias agrupadas $(\epsilon_{-} + U + U_{f},\, \epsilon_{+} + U + U_{f}) \to \epsilon + U + U_{f}$, el operador de salto efectivo es

\begin{equation*}
    \frac{\hat{d}_{+}\hat{n}_{D}\hat{n}_{-}}{\sqrt{2}} - \frac{\hat{d}_{-}\hat{n}_{D}\hat{n}_{+}}{\sqrt{2}} = \hat{d}_{L}\hat{n}_{D}\hat{n}_{R}.
\end{equation*}

En recapitulación, al agrupar las frecuencias los operadores de salto estan dados por 

\begin{align*}
    &\epsilon \to \hat{d}_{L}(\textbf{1}-\hat{n}_{R})(\textbf{1}-\hat{n}_{D})\\
   &\epsilon + U \to \hat{d}_{L}\hat{n}_{D}(\textbf{1}-\hat{n}_{R})\\
    &\epsilon +U_{f}\to \hat{d}_{L}\hat{n}_{R}(\textbf{1}-\hat{n}_{D})\\
    &\epsilon +U+U_{f}\to \hat{d}_{L}\hat{n}_{D}\hat{n}_{R},
\end{align*}
Análogamente, en el caso del operador local $\hat{d}_{R}$, se procede agrupando las energías $(\epsilon{+}, \epsilon_{-})$, así calcular  
\begin{align*}
    \frac{\hat{d}_{+}(\textbf{1}-\hat{n}_{D})(\textbf{1}-\hat{n}_{-}) }{\sqrt{2}} + \frac{\hat{d}_{-}(\textbf{1}-\hat{n}_{D})(\textbf{1}-\hat{n}_{+}) }{\sqrt{2}} & = \frac{(\textbf{1} - \hat{n}_{D})}{\sqrt{2}}[\hat{d}_{+}(\textbf{1}-\hat{n}_{-}) + \hat{d}_{-}(\textbf{1}-\hat{n}_{+})],
\end{align*}
además  

\begin{align*}
    \hat{d}_{+}(\textbf{1}-\hat{n}_{-}) + \hat{d}_{-}(\textbf{1}-\hat{n}_{+}) & =  \frac{\hat{d}_{R} + \hat{d}_{L}}{\sqrt{2}}(\textbf{1} - \hat{n}_{-}) + \frac{\hat{d}_{R} - \hat{d}_{L}}{\sqrt{2}}(\textbf{1} - \hat{n}_{+})   \\
    & = \frac{\hat{d}_{R}}{\sqrt{2}}(\textbf{2} - (\hat{n}_{L} + \hat{n}_{R})) + \frac{\hat{d}_{L}}{\sqrt{2}} (\hat{n}_{+}-\hat{n}_{-}) \\
    & = \frac{\hat{d}_{R}}{\sqrt{2}}(\textbf{1} -  \hat{n}_{R}) + \frac{\hat{d}_{L}}{\sqrt{2}} (\hat{d}^{\dagger}_{R}\hat{d}_{L} + \hat{d}^{\dagger}_{L}\hat{d}_{R}) \\
    & = \sqrt{2}\hat{d}_{R}(\textbf{1}-\hat{n}_{R}),
\end{align*}
por lo tanto, el operador de salto asociado a la agrupación $(\epsilon_{+},\epsilon_{-}) \to \epsilon$ es $\hat{d}_{R}(\textbf{1} - \hat{n}_{D})(\textbf{1} - \hat{n}_{L})$. De manera análoga, al calcular el operador correspondiente a la agrupación de frecuencias $(\epsilon_{+}+U, \epsilon_{-}+U)$

\begin{equation*}
    \frac{\hat{d}_{+}(\textbf{1}-\hat{n}_{-})\hat{n}_{D}}{\sqrt{2}} + \frac{\hat{d}_{-}(\textbf{1}-\hat{n}_{+})\hat{n}_{D}}{\sqrt{2}}  = \hat{d}_{R}\hat{n}_{D}(\textbf{1}-\hat{n}_{L}).
\end{equation*}

Para las frecuencias $(\epsilon_{+}+U_{f},\epsilon_{-}+U_{f})\to \epsilon + U_{f}$, se tendrán que sumar los operadores 

\begin{equation*}
    \frac{\hat{d}_{+}(\textbf{1}-\hat{n}_{D})\hat{n}_{-} }{\sqrt{2}} + \frac{\hat{d}_{-}(\textbf{1}-\hat{n}_{D})\hat{n}_{+} }{\sqrt{2}} = \frac{(\textbf{1}-\hat{n}_{D})}{\sqrt{2}} [\hat{d}_{+}\hat{n}_{-} + \hat{d}_{-}\hat{n}_{+}],
\end{equation*}
donde, el operador $\hat{d}_{+}\hat{n}_{-} + \hat{d}_{-}\hat{n}_{+}$ es 

\begin{align*}
    \hat{d}_{+}\hat{n}_{-} + \hat{d}_{-}\hat{n}_{+} & = \frac{\hat{d}_{R}+\hat{d}_{L}}{\sqrt{2}}\hat{n}_{-} + \frac{\hat{d}_{R}-\hat{d}_{L}}{\sqrt{2}}\hat{n}_{+}  \\
    & = \frac{\hat{d}_{R}}{\sqrt{2}}(\hat{n}_{L}+\hat{n}_{R}) - \frac{\hat{d}_{L}}{\sqrt{2}}(\hat{d}^{\dagger}_{L}\hat{d}_{R}+\hat{d}^{\dagger}_{R}\hat{d}_{L}) \\
    & = \sqrt{2}\hat{d}_{R}\hat{n}_{L},
\end{align*}
por lo tanto, el operador de salto de $\hat{d}_{R}$ al agrupar las frecuencias $(\epsilon_{+}+U_{f}, \epsilon_{-}+U_{f})$ es $\hat{d}_{R}(\textbf{1} - \hat{n}_{D})\hat{n}_{L}$; realizando el mismo procedimiento, se encuentra que para las frecuencias $(\epsilon_{+}+U+U_{f}, \epsilon_{-}+U+U_{f})$ el operador de salto corresponde a $\hat{d}_{R}\hat{n}_{D}\hat{n}_{L}$. Por ende, recapitulando, para el operador $\hat{d}_{R}$ se tiene

\begin{align*}
    &\epsilon \to \hat{d}_{R}(\textbf{1}-\hat{n}_{L})(\textbf{1}-\hat{n}_{D})\\
   &\epsilon + U \to \hat{d}_{R}\hat{n}_{D}(\textbf{1}-\hat{n}_{L})\\
    &\epsilon +U_{f}\to \hat{d}_{R}\hat{n}_{L}(\textbf{1}-\hat{n}_{D})\\
    &\epsilon +U+U_{f}\to \hat{d}_{R}\hat{n}_{D}\hat{n}_{L}.
\end{align*}

\label{apendix5frecuencygroup}

%%%%%%%%%%%%%%%%%%%%%%%%%ojooooaqui%%%%%%%%%%%%%
%\subsection{Límitaciones de la aproximación semilocal}

%\label{apendix5límites}

\subsection{Aspecto Markoviano de las funciones correlación}
Para que el sistema descrito en la sección \ref{sec5:modelo} exhiba un comportamiento análogo al de un Demonio de Maxwell autónomo, es necesario que las tasas de túnel $\kappa_{i}(\omega)$ respondan de manera diferenciada dependiendo de si el sitio $D$ se encuentra ocupado o desocupado. Sin embargo, no toda dependencia funcional de $\kappa_{i}(\omega)$ garantiza un comportamiento Markoviano del sistema. Esta condición puede evaluarse mediante el cálculo explícito de las funciones de correlación del reservorio, ya que el carácter Markoviano se encuentra fuertemente ligado al tiempo de correlación característico del entorno, denotado por $\tau_{B}$. Para analizar este aspecto, se requiere calcular las funciones de correlación correspondientes a un entorno fermiónico, las cuales, en general, adoptan la forma

\begin{equation*}
    C^{\sigma}_{i}(t) = \frac{1}{2\pi} \int_{-\infty}^{\infty} d\omega e^{i\sigma \omega t} \kappa_{i}(\omega) f_{F}(\sigma \beta(\omega-\mu)),
\end{equation*}
donde $f_{F}(x) = (\exp(x)+1)^{-1}$, $i= L,R$ y $\sigma = \pm$. 

La distribución de Fermi puede ser calculada mediante aproximantes de Padé \cite{hu2011pade,schinabeck2019hierarchical}, lo que permite expresarla como una sumatoria 

\begin{equation}
    f_{F}(x) \approx \frac{1}{2} - \sum_{l=0}^{N} \frac{ 2 K_{l}x }{ x^{2} + \xi^{2}_{l} },
    \label{apendix5:pade}
\end{equation}
los coeficientes $K_{l}$ y $\xi_{l}$ pueden ser calculados numéricamente y se encuentran tabulados en \cite{hu2011pade}. Para evaluar numéricamente la función de correlación, es necesario especificar una forma explícita para $\kappa_{i}(\omega)$, la cual debe presentar un máximo en cierta frecuencia característica $\epsilon_{0}$. Una elección común consiste en considerar un perfil lorentziano dado por

\begin{equation*}
    \kappa_{i}(\omega) = \kappa_{0} + \frac{\kappa_{f} W^{2}}{(\omega - \epsilon_{0})^{2} + W^{2}},
\end{equation*}
donde $W$ representa el ancho de la lorentziana, $\kappa_{0}$ y $\kappa_{f}$ son constantes que controlan la amplitud. En el estudio del Demonio de Maxwell autónomo, es fundamental distinguir dos casos relevantes: cuando el punto cuántico $D$ se encuentra desocupado o ocupado. Esta distinción se refleja en la evaluación de las tasas de túnel en dos frecuencias distintas, $\epsilon_{0}$ y $\epsilon_{0} + U$, tal como se ilustra en la Figura \ref{img:gammas} para las tasas de túnel correspondientes a los baños $L$ y $R$.

\insertimage[\label{img:gammas}]{ejemplos/gammmas.pdf}{scale=0.35}{Tasas de túnel en función de las frecuencias $\omega$.}

Para obtener tasas de túnel que satisfagan dichas condiciones, se puede plantear el siguiente sistema de ecuaciones 

\begin{align*}
    \kappa_{i}(\epsilon_{0}) & = \kappa_{0} + \kappa_{f} \\
    \kappa_{i}(\epsilon_{0} + U) & = \kappa_{0} + \frac{\kappa_{f} W^{2} }{U^{2} + W^{2}}.
\end{align*}

A partir de estas igualdades, es posible determinar los parámetros $\kappa_{0}$, $\kappa_{f}$ y $W$. Sustituyendo la expresión de $\kappa_{i}(\omega)$ en la función de correlación, se obtiene

\begin{align*}
    C^{\sigma}_{i}(t) & = \frac{\kappa_{0}}{2\pi} \int_{-\infty}^{\infty}d\omega e^{i\sigma \omega t} f_{F}(\sigma \beta (\omega-\mu)) + \frac{\kappa_{f}}{2\pi} \int_{-\infty}^{\infty}d\omega e^{i\sigma \omega t}\left[ \frac{W^{2}}{(\omega-\epsilon_{0})^{2} + W^{2}} \right] f_{F}(\sigma \beta (\omega-\mu)),
\end{align*}
la función de correlación puede descomponerse en dos contribuciones

\begin{equation}
    C^{\sigma}_{0}(t) = \frac{\kappa_{0}}{2\pi} \int_{-\infty}^{\infty}d\omega e^{i\sigma \omega t} f_{F}(\sigma \beta (\omega-\mu))
    \label{apendix5:correlacion0}
\end{equation}

\begin{equation}
    C^{\sigma}_{f}(t) = \frac{\kappa_{f}}{2\pi} \int_{-\infty}^{\infty}d\omega e^{i\sigma \omega t}\left[ \frac{W^{2}}{(\omega-\epsilon_{0})^{2} + W^{2}} \right] f_{F}(\sigma \beta (\omega-\mu)),
    \label{apendix5:correlacionf}
\end{equation}

y se podrá determinar el tiempo de correlación del baño como $\tau_{B} = \max\{\tau_{B0}, \tau_{Bf}\}$. Se comenzará calculando una expresión analítica aproximada para la función de correlación $C^{\sigma}_{f}(t)$. Para resolver la integral de \ref{apendix5:correlacionf}, será necesario localizar los polos en el plano complejo. Según \ref{apendix5:pade}, los polos asociados a la aproximación de Padé se encuentran en $\omega = \pm i \xi_{l}/\beta + \mu$, mientras que los polos correspondientes a la forma Lorentziana se ubican en $\omega = \pm i W + \omega_{0}$. Para realizar esta integración, se utilizará el teorema del residuo \cite{riley2006mathematical}, considerando un contorno de integración en forma de semicírculo en el hemisferio superior o inferior del plano complejo, dependiendo del signo de $\sigma$. El residuo asociado a la Lorentziana es

\begin{align*}
   \text{Res} \left[ \frac{e^{ i\sigma \omega t} f_{F}[\sigma \beta (\omega-\mu)] }{ (\omega-\omega_{0} + iW)(\omega -\omega_{0} -iW)}\right]_{\omega = \pm i W + \omega_{0}} = \frac{1}{\pm 2iW} ( e^{\mp \sigma Wt} f_{F}[\sigma \beta(\pm iW +\omega_{0}-\mu)]e^{i\sigma \omega_{0}t}). 
\end{align*}

Si se cumple que $\sigma > 0$, se toma como contorno de integración el hemisferio inferior del plano complejo; en cambio, si $\sigma < 0$, se considera el hemisferio superior. De esta manera, se obtiene

\begin{align*}
    \text{Res} \left[ \frac{e^{ i\sigma \omega t} f_{F}[\sigma \beta (\omega-\mu)] }{ (\omega-\omega_{0} + iW)(\omega -\omega_{0} -iW)}\right]_{\omega = \pm i W + \omega_{0}} = \frac{1}{ 2iW} ( e^{i\sigma \omega_{0}t}e^{- Wt} f_{F}[i\beta W + \sigma \beta(\omega_{0}-\mu)]). 
 \end{align*}

Para encontrar el residuo asociado a los polos restantes, se deberá calcular

\begin{align*}
    \text{Res} \left[ \frac{-e^{i\sigma \omega t}}{(\omega - \omega_{0})^{2} + W^{2} } \frac{ 2K_{l}[\sigma \beta (\omega-\mu)] }{ (\beta(\omega-\mu) + i \xi_{l} )(\beta(\omega-\mu) - i \xi_{l})} \right]_{\omega = \pm i \xi_{l}/\beta + \mu} & = \frac{-1}{\beta} \frac{e^{- \frac{\sigma \xi_{l}}{\beta}t} e^{i\sigma \mu t}(\pm K_{l}\sigma )}{ [ \frac{ \pm i\xi_{l}}{\beta} + (\mu - \omega_{0}) ]^{2} + W^{2} } ,
\end{align*}
seleccionando la región de integración en función del signo de $\sigma$ como se hizo anteriormente, se concluye que

\begin{align*}
    \text{Res} \left[ \frac{-e^{i\sigma \omega t}}{(\omega - \omega_{0})^{2} + W^{2} } \frac{ 2K_{l}[\sigma \beta (\omega-\mu)] }{ (\beta(\omega-\mu) + i \xi_{l} )(\beta(\omega-\mu) - i \xi_{l})} \right]_{\omega = \pm i \xi_{l}/\beta + \mu} & = \frac{-1}{\beta} \frac{e^{- \frac{ \xi_{l}}{\beta}t} e^{i\sigma \mu t} K_{l} }{ [ \frac{ \sigma i\xi_{l}}{\beta} + (\mu - \omega_{0}) ]^{2} + W^{2} }. 
\end{align*}

Al encontrar los residuos explícitamente, estos se podrán usar para escribir la función correlación 

\begin{equation}
    C^{\sigma}_{f}(t) \approx \sum_{l=0}^{N} \eta^{\sigma,l} e^{-\gamma_{\sigma,l} t},
    \label{correlationf}
\end{equation}
donde 

\begin{equation*}
    \eta^{\sigma,l} = \left\{ \begin{array}{lc} \frac{\kappa_{f}W}{2} f_{F}[i\beta W + \sigma \beta (\omega_{0}-\mu)]  & l = 0 \\ \\ - \frac{iK_{l}}{\beta} \left(\frac{\kappa_{f}W^{2}}{ (\frac{i\sigma \xi_{l}}{\beta} + (\mu-\omega_{0}))^{2} + W^{2} } \right) &  l \neq 0 \end{array} \right.
\end{equation*}

\begin{equation*}
    \gamma_{\sigma,l} =  \left\{ \begin{array}{lc} W- \sigma i \omega_{0}  & l = 0 \\ \\ \frac{\xi_{l}}{\beta} - \sigma i \mu &  l \neq 0 \end{array} \right..
\end{equation*}

Esto permite obtener una expresión numérica para la función de correlación $C^{\sigma}_{f}(t)$, y con ello evaluar su tiempo de correlación característico, $\tau_{Bf}$. Por otro lado, para la otra función de correlación es posible calcular la integral de forma analítica, es decir, evaluar

\begin{align*}
       C_{0}^{\sigma}(t) = \frac{\kappa_{0}}{2\pi} \int_{-\infty}^{\infty} d\omega e^{i\sigma \omega t }f_{F}(\sigma \beta (\omega -  \mu)).
\end{align*}    

Se puede  observar que este cálculo equivale a obtener la transformada de Fourier 

\[
\mathcal{F}\big(f[\sigma (\omega - \mu)]\big)(-\sigma t),
\]

de la función 

\[
f(\sigma(\omega-\mu)) = f_{F}(\sigma \beta (\omega-\mu)),
\]

empleando la propiedad

\begin{equation*}
\mathcal{F}(f[\sigma (\omega - \mu)])(-\sigma t) = e^{-i \sigma \mu t}\mathcal{F}(f[\sigma \omega])(-\sigma t),
\end{equation*}
es decir, podemos centrarnos en calcular la transformación

\begin{align*}
    \mathcal{F}(f[\sigma \omega])(-\sigma t) &= \frac{\kappa_{0}}{2\pi} \int_{-\infty}^{\infty} d\omega \frac{e^{i\sigma \omega t}}{e^{\sigma \beta \omega} +1 } \\
        & = \frac{\kappa_{0}}{4\pi} \left[\int_{-\infty}^{\infty}d \omega e^{i\sigma \omega t} - \int_{-\infty}^{\infty}d\omega e^{i\sigma \omega t} \tanh \left(\frac{ \sigma \beta \omega }{2} \right)   \right] \\
        & =  \frac{\kappa_{0}}{2} \left[\delta(t) - \frac{i}{\beta \sigma \sinh(\pi t/\beta \sigma)} \right],
\end{align*}
así

\begin{equation*}
    C^{\sigma}_{0}(t) = \frac{\kappa_{0}}{2}e^{-i\sigma \mu t} \left[\delta(t) - \frac{i}{\beta \sigma \sinh(\pi t/\beta \sigma)} \right].
\end{equation*}

Gracias a que se dispone de una expresión analítica para $C^{\sigma}_{0}(t)$, es posible determinar el tiempo de correlación $\tau_{B0}$. El análisis se centra principalmente en el término que involucra $\sinh(\pi t/\beta \sigma)$, ya que, dado que $\sigma = \pm 1$, el parámetro relevante es $\pi t/\beta$. Para tiempos $t > \beta$, se puede aproximar que 

\[
\frac{1}{\sinh(\pi t/\beta \sigma)} \propto e^{-t/\beta},
\]

lo que implica que el tiempo de correlación del baño está dado por el inverso de la temperatura, es decir, $\tau_{B0} = \beta$. Por consiguiente, el tiempo de correlación efectivo asociado a un único reservorio queda definido como

\[
\tau_{B} = \max \{\beta, \tau_{Bf}\}.
\]

\label{appendix5correlation}

\subsection{Cálculo numérico para los tiempos de correlación}
Dado que es posible reconstruir las tasas de túnel $\kappa_{L}(\omega)$ y $\kappa_{R}(\omega)$ mediante una función de tipo Lorentziana, cumpliendo las condiciones $\kappa_{L}(\epsilon) = \kappa_{R}(\epsilon+U) = 1/100$ y $\kappa_{L}(\epsilon+U) = \kappa_{R}(\epsilon) = 1/600$, se pueden calcular numéricamente las funciones correlación asociadas. 

Para determinar los tiempos de correlación $\tau_{Bf}$ correspondientes a los baños $L$ y $R$, se utilizará la expresión \ref{correlationf} de las funciones correlación. La estimación de $\tau_{Bf}$ se realiza identificando el punto donde la función correlación practicamente se anula. Posteriormente, para evaluar si el comportamiento del sistema es efectivamente markoviano, se debe verificar que $\tau_{R} \gg \tau_{Bf}$. Con este criterio, se procede al cálculo de las partes real e imaginaria de las funciones de correlación correspondientes a los baños $L$ y $R$.

\insertimage[\label{img:correlacionminus}]{ejemplos/cminusreal.pdf}{scale=0.5}{(a)Parte real de las funciones correlación $\sigma=-$ del baño $L$ en función del tiempo.(b) Parte real de las funciones correlación $\sigma=-$ del baño $R$ en función del tiempo.}
\insertimage[\label{img:correlacionminusimag}]{ejemplos/cminusimag.pdf}{scale=0.5}{(a)Parte imaginaria de las funciones correlación $\sigma=-$ del baño $L$ en función del tiempo.(b) Parte imaginaria de las funciones correlación $\sigma=-$ del baño $R$ en función del tiempo.}

\insertimage[\label{img:correlacionplus}]{ejemplos/cplusreal.pdf}{scale=0.5}{(a)Parte real de las funciones correlación $\sigma=+$ del baño $L$ en función del tiempo.(b) Parte real de las funciones correlación $\sigma=+$ del baño $R$ en función del tiempo.}
\insertimage[\label{img:correlacionplusimag}]{ejemplos/cplusimag.pdf}{scale=0.5}{(a)Parte imaginaria de las funciones correlación $\sigma=+$ del baño $L$ en función del tiempo.(b) Parte imaginaria de las funciones correlación $\sigma=+$ del baño $R$ en función del tiempo.}



A partir de las Figuras \ref{img:correlacionminus}, \ref{img:correlacionminusimag}, \ref{img:correlacionplus} y \ref{img:correlacionplusimag}, se observa que, para $t > 0.4$, las funciones de correlación correspondientes a ambos baños se anulan prácticamente. 

Para analizar el carácter markoviano del sistema, se puede comparar este tiempo de correlación con la dinámica de relajación de la matriz densidad hacia el estado estacionario. Para ello, en la Figura \ref{img:probas} se grafica la evolución temporal de las componentes de la matriz densidad, y se analiza el tiempo característico en que estas alcanzan su régimen estacionario.


\insertimage[\label{img:probas}]{ejemplos/rhotiempo.pdf}{scale=0.56}{Componentes $\rho_{100}$ y $\rho_{010}$ de la matriz densidad en función del tiempo, para distintos valores de $eV/T$.}

A partir de la Figura \ref{img:probas}, se observa que el tiempo de relajación del sistema es del orden de $\tau_{R} > 2000$. Esto permite corroborar que, para distintos valores de $eV/T$, se cumple la condición $\tau_{Bf}/\tau_{R} < 0.0002$. Lo que implica que es adecuado realizar la aproximación markoviana para las funciones correlación de los baños $L$ y $R$.

\label{appendix5tauf}

\newpage 

\subsection{Concurrencia}
En esta sección se presenta la demostración de la fórmula de la concurrencia. Para ello, se debe calcular la concurrencia en el subsistema $\hat{\rho}_{LR}$, lo que requiere, como primer paso, el cálculo de

\begin{equation*}
    \hat{\rho}_{LR} = \text{Tr}_{D}\{ \hat{\rho_{S}} \}.
\end{equation*}

Para escribir la matriz densidad $\hat{\rho}_{S}$, se utilizará la base  
\[
\left\{ |0,0,0\rangle,\ |1,0,0\rangle,\ |0,1,0\rangle,\ |0,0,1\rangle,\ |1,1,0\rangle,\ |1,0,1\rangle,\ |0,1,1\rangle,\ |1,1,1\rangle \right\}.
\]
En esta representación, la matriz densidad del sistema se expresa como

\begin{equation}
    \hat{\rho}_{S} = 
    \begin{bmatrix}
        \rho_{000} & 0 & 0 & 0 & 0 & 0 & 0 & 0 \\
        0 & \rho_{100} & \alpha & 0 & 0 & 0 & 0 & 0 \\
        0 & \alpha^{*} & \rho_{010} & 0 & 0 & 0 & 0 & 0 \\
        0 & 0 & 0 & \rho_{001} & 0 & 0 & 0 & 0 \\
        0 & 0 & 0 & 0 & \rho_{110} & 0 & 0 & 0 \\
        0 & 0 & 0 & 0 & 0 & \rho_{101} & \beta & 0 \\
        0 & 0 & 0 & 0 & 0 & \beta^{*} & \rho_{011} & 0 \\
        0 & 0 & 0 & 0 & 0 & 0 & 0 & \rho_{111} 
        \end{bmatrix}.
        \label{appendix5rhoconcu}
\end{equation}

En esta matriz densidad no se observan coherencias entre estados con distinto número total de partículas, lo cual se debe al principio de superselección de carga \cite{bartlett2007reference,wick1997intrinsic}. Además, dado que el punto cuántico $D$ sólo interactúa con los puntos $L$ y $R$ a través de una interacción de tipo Coulomb, no se generan coherencias del tipo $L$-$D$ ni $R$-$D$. 

Es posible obtener la matriz densidad reducida del subsistema $LR$ al proyectar sobre la base  $\{|00\rangle, |10\rangle, |01\rangle, |11\rangle\}$

\begin{equation*}
    \hat{\rho}_{LR} = 
    \begin{bmatrix}
        \rho_{000}+\rho_{001} & 0 & 0 & 0  \\
        0 & \rho_{100} + \rho_{101} & \alpha + \beta & 0  \\
        0 & \alpha^{*} +\beta^{*} & \rho_{010} + \rho_{011} & 0  \\
        0 & 0 & 0 & \rho_{110} + \rho_{111} 
        \end{bmatrix},
\end{equation*}
de este modo, se podrá calcular la concurrencia 

\begin{equation*}
    \mathcal{C}_{on} = \max \{ 0,\lambda_{1} - \lambda_{2} - \lambda_{3} - \lambda_{4} \},
\end{equation*}
donde los $\lambda_{i}$ son la raíz cuadrada de los autovalores ordenados en forma decreciente, de la matriz dada por 

\begin{equation}
    B = \hat{\rho}_{LR} \tilde{\rho}_{LR} \hspace{10mm} \tilde{\rho}_{LR} = (\sigma_{y} \otimes \sigma_{y}) \hat{\rho}_{LR} (\sigma_{y} \otimes \sigma_{y}),
    \label{apendixconcu}
\end{equation}
en la cual $\sigma_{y}$ pertenece a las matrices de Pauli. Al aplicar \ref{apendixconcu} se obtiene 

\begin{equation*}
    \tilde{\rho}_{LR} = 
    \begin{bmatrix}
        p_{D} & 0 & 0 & 0  \\
        0 & p_{R} & \alpha + \beta & 0  \\
        0 & \alpha^{*} +\beta^{*} & p_{L} & 0  \\
        0 & 0 & 0 & p_{0} 
        \end{bmatrix},
\end{equation*}
con  $p_{0}=\rho_{000}+\rho_{001}$, $p_{L}=\rho_{100} + \rho_{101}$, $p_{R}=\rho_{010}+\rho_{011}$, $p_{D}=\rho_{110}+\rho_{111}$  y la matriz $B$ es

\begin{equation*}
    B = 
    \begin{bmatrix}
        p_{0}p_{D} & 0 & 0 & 0  \\
        0 & p_{L}p_{R}+|\alpha+\beta|^{2} & 2(\alpha + \beta)p_{L} & 0  \\
        0 & 2(\alpha^{*} +\beta^{*})p_{R} & p_{L}p_{R}+|\alpha+\beta|^{2}  & 0  \\
        0 & 0 & 0 & p_{0}p_{D}
        \end{bmatrix}.
\end{equation*}

Suponiendo que $p_{D}$ y $p_{0}$ son pequeños, el orden decreciente de los autovalores corresponderá a

\begin{equation*}
    \lambda_{1} = p_{L}p_{R} + |\alpha + \beta| \hspace{10mm}  \lambda_{2} = p_{L}p_{R} - |\alpha + \beta| \hspace{10mm} \lambda_{3}=\lambda_{4} = \sqrt{p_{0}p_{D}},
\end{equation*}
entonces se obtiene

\begin{equation*}
    \mathcal{C}_{on} = \max\{ 2|\alpha+\beta| - 2\sqrt{p_{0}p_{D}},0\}.
\end{equation*}

\label{appendix5final}

\subsection{Modelo clásico}
Para comenzar con la derivación del modelo clásico que describe las componentes diagonales de la matriz densidad $\rho_{ijk}$, con $i,j,k \in {0,1}$, se parte considerando la ecuación de evolución para los operadores Nakajima-Zwanzig

\begin{equation*}
    \frac{d}{dt}\check{\mathcal{Q}}|\hat{\rho}(t)\rangle \rangle = \check{\mathcal{Q}}\check{\mathcal{L}}_{f}\check{\mathcal{Q}}|\hat{\rho}(t)\rangle \rangle + \check{\mathcal{Q}}\check{\mathcal{L}}_{f}\check{\mathcal{P}}|\hat{\rho}(t)\rangle \rangle \to \frac{d}{dt}\check{\mathcal{Q}}|\hat{\rho}(t)\rangle \rangle - \check{\mathcal{Q}}\check{\mathcal{L}}_{f}\check{\mathcal{Q}}|\hat{\rho}(t)\rangle \rangle = \check{\mathcal{Q}}\check{\mathcal{L}}_{f}\check{\mathcal{P}}|\hat{\rho}(t)\rangle \rangle ,
\end{equation*}
que es equivalente a una ecuación no homogenea lineal de la forma

\begin{equation*}
    \frac{dy}{dt} - A(t)y = B(t),
\end{equation*}
por lo tanto, se puede solucionar utilizando factor integrante

\begin{equation*}
    \check{\mu}(t) = \exp\left( - \int_{0}^{t}\check{\mathcal{Q}}\check{\mathcal{L}}_{f}ds \right),
\end{equation*}
multiplicando por el factor integrante la ecuacion diferencial para $\mu(t)$, se obtiene 

\begin{equation*}
    \check{\mu}(t)\frac{d}{dt}\check{\mathcal{Q}}|\hat{\rho}(t)\rangle \rangle - \check{\mu}(t)\check{\mathcal{Q}}\check{\mathcal{L}}_{f}\check{\mathcal{Q}}|\hat{\rho}(t)\rangle \rangle  = \check{\mu}(t)\check{\mathcal{Q}}\check{\mathcal{L}}_{f}\check{\mathcal{P}}|\hat{\rho}(t)\rangle \rangle  \to \frac{d}{dt}(\check{\mu}(t)\check{\mathcal{Q}}|\hat{\rho}(t)\rangle \rangle) = \check{\mu}(t)\check{\mathcal{Q}}\check{\mathcal{L}}_{f}\check{\mathcal{P}}|\hat{\rho}(t)\rangle \rangle ,
\end{equation*}
con solución

\begin{equation*}
    \check{\mu}(t)\check{\mathcal{Q}}|\hat{\rho}(t)\rangle \rangle  = \check{\mu}(0)\check{\mathcal{Q}}|\hat{\rho}(0)\rangle \rangle  + \int_{0}^{t}\check{\mu}(\tau)\check{\mathcal{Q}}\check{\mathcal{L}}_{f} \check{\mathcal{P}}|\hat{\rho}(\tau)\rangle \rangle d\tau,   
\end{equation*}
multiplicando por $\check{\mu}^{-1}(t)$ ambos lados de la ecuación, se obtiene

\begin{equation*}
    \check{\mathcal{Q}}|\hat{\rho}(t)\rangle \rangle  = \check{\mu}^{-1}(t)\check{\mathcal{Q}}|\hat{\rho}(0)\rangle \rangle  + \int_{0}^{t}\check{\mu}^{-1}(t)\check{\mu}(\tau)\check{\mathcal{Q}}\check{\mathcal{L}}_{f} \check{\mathcal{P}}|\hat{\rho}(\tau)\rangle \rangle d\tau,    
\end{equation*}
luego, se define el operador 
\begin{equation*}
    \check{\mathcal{G}}(t,s) = \exp\left(  \int_{s}^{t}\check{\mathcal{Q}}\check{\mathcal{L}}_{f}dt' \right),
\end{equation*}
y queda la ecuación 

\begin{equation*}
    \check{\mathcal{Q}}|\hat{\rho}(t)\rangle \rangle = \check{\mathcal{G}}(t,0)\check{\mathcal{Q}}|\hat{\rho}(0)\rangle \rangle  + \int_{0}^{t}\check{\mathcal{G}}(t,\tau)\check{\mathcal{Q}}\check{\mathcal{L}}_{f} \check{\mathcal{P}}|\hat{\rho}(\tau)\rangle \rangle d\tau.    
\end{equation*}

Esto se puede reinsertar en la ecuación principal, obteniéndose así

\begin{equation}
    \frac{d}{dt}\check{\mathcal{P}}|\hat{\rho}(t)\rangle \rangle  = \check{\mathcal{P}}\check{\mathcal{L}}_{f}\check{\mathcal{P}}|\hat{\rho}(t)\rangle \rangle  + \check{\mathcal{P}}\check{\mathcal{L}}_{f}\check{\mathcal{G}}(t,0)\check{\mathcal{Q}}|\hat{\rho}(0)\rangle \rangle  + \check{\mathcal{P}}\check{\mathcal{L}}_{f}\int_{0}^{t}\check{\mathcal{G}}(t,\tau)\check{\mathcal{Q}}\check{\mathcal{L}}_{f} \check{\mathcal{P}}|\hat{\rho}(\tau)\rangle \rangle d\tau.    
    \label{apendix5exactp}
\end{equation}

Estas ecuaciones son exactas, ya que hasta este punto no se ha introducido ninguna aproximación. El primer término del lado derecho de la ecuación \ref{apendix5exactp} representa la evolución markoviana del sistema, mientras que el segundo y tercer término corresponden, respectivamente, a un corrimiento inicial y a los efectos de memoria inducidos por el acoplamiento con el entorno. Para continuar con la deducción, se utilizarán las siguientes relaciones

\begin{equation*}
    \check{\mathcal{P}}\check{\mathcal{V}}\check{\mathcal{P}} = [\check{\mathcal{L}}_0,\check{\mathcal{P}}] = [\check{\mathcal{L}}_0,\check{\mathcal{Q}}] = 0, 
\end{equation*}
estas relaciones pueden entenderse de manera intuitiva; dado que $\check{\mathcal{V}}$ representa la perturbación responsable del túnel cuántico entre los puntos $L$ y $R$, su acción introduce coherencias en la matriz densidad. En particular, al aplicar $\check{\mathcal{V}}$ sobre un estado diagonal, se obtiene una componente no diagonal. Por lo tanto, al aplicar nuevamente el operador de proyección sobre la diagonal $\check{\mathcal{P}}$, esta contribución se anula, es decir, $\check{\mathcal{P}} \check{\mathcal{V}} \check{\mathcal{P}} = 0$.

Para continuar con el análisis de la ecuación \ref{apendix5exactp}, se harán dos suposiciones, en primer lugar, que el estado inicial no presenta coherencias, es decir, $\check{\mathcal{Q}}\hat{\rho}(0) = 0$; en segundo lugar, se sustituirá explícitamente el operador $\check{\mathcal{L}}_f$ en la ecuación \ref{apendix5exactp}. Con esto, se procederá a evaluar el término integral, utilizando las propiedades de los operadores de proyección de Nakajima-Zwanzig, comenzando por

\begin{align*}
    \check{\mathcal{G}}(t,\tau) & = \exp\left(  \int_{\tau}^{t}\check{\mathcal{Q}}(\check{\mathcal{L}}_{0} + \check{\mathcal{V}})dt' \right) \\
        & = \exp\left(  \int_{\tau}^{t}\check{\mathcal{Q}}(\check{\mathcal{L}}_{0}+\check{\mathcal{V}})(\check{\mathcal{P}}+\check{\mathcal{Q}}) dt' \right) \\
            & = \exp\left(  \int_{\tau}^{t}\left[\check{\mathcal{Q}}\check{\mathcal{L}}_{0}\check{\mathcal{Q}}+ \check{\mathcal{Q}}\check{\mathcal{V}}\check{\mathcal{P}}+\check{\mathcal{Q}}\check{\mathcal{V}}\check{\mathcal{Q}}\right] dt' \right),       
\end{align*}
donde se uso la igualdad

\begin{equation*}
    \check{\mathcal{Q}}\check{\mathcal{L}}_{0}\check{\mathcal{P}} = \check{\mathcal{Q}}\check{\mathcal{P}}\check{\mathcal{L}}_{0}= 0.
\end{equation*}

A continuación, la ecuación con la parte principal estará dada por 

\begin{equation*}
    \frac{d}{dt}\check{\mathcal{P}}|\hat{\rho}(t)\rangle \rangle = \check{\mathcal{P}}\check{\mathcal{L}}_{0}\check{\mathcal{P}}|\hat{\rho}(t)\rangle \rangle + \check{\mathcal{P}}(\check{\mathcal{L}}_{0} + \check{\mathcal{V}})(\check{\mathcal{P}}+\check{\mathcal{Q}})\int_{0}^{t}\check{\mathcal{G}}(t,\tau)\check{\mathcal{Q}}\check{\mathcal{L}}_{f} \check{\mathcal{P}}|\hat{\rho}(\tau)\rangle \rangle d\tau.       
\end{equation*}

Debido a la forma del exponencial en $\check{\mathcal{G}}$ al aplicar $\check{\mathcal{P}}\check{\mathcal{L}}_{0}\check{\mathcal{G}}(t,\tau)$ se obtiene $\check{\mathcal{P}}\check{\mathcal{L}}_{0}(t-\tau)$. Ya que, si extendemos el exponencial

\begin{equation*}
    \check{\mathcal{P}}\check{\mathcal{L}}_{0}\exp\left[\int_{\tau}^{t}ds\check{\mathcal{Q}}\check{\mathcal{L}}_{0}\check{\mathcal{Q}}+ \check{\mathcal{Q}}\check{\mathcal{V}}\check{\mathcal{P}}+\check{\mathcal{Q}}\check{\mathcal{V}}\check{\mathcal{Q}} \right] = \check{\mathcal{L}}_{0}\check{\mathcal{P}}\int_{\tau}^{t}\sum_{k=0}^{\infty}\left(\frac{1}{k!} \right)\left[\check{\mathcal{Q}}\check{\mathcal{L}}_{0}\check{\mathcal{Q}}+ \check{\mathcal{Q}}\check{\mathcal{V}}\check{\mathcal{P}}+\check{\mathcal{Q}}\check{\mathcal{V}}\check{\mathcal{Q}} \right]^{k}ds = \check{\mathcal{P}}\check{\mathcal{L}}_{0}(t-\tau).
\end{equation*}

Posteriormente, esta contribución se anula y la ecuación principal se reduce a

\begin{equation*}
    \frac{d}{dt}\check{\mathcal{P}}|\hat{\rho}(t)\rangle \rangle = \check{\mathcal{P}}\check{\mathcal{L}}_{0}\check{\mathcal{P}}|\hat{\rho}(t)\rangle \rangle + \check{\mathcal{P}}\check{\mathcal{V}}\check{\mathcal{Q}}\int_{0}^{t}\check{\mathcal{G}}(t,\tau)\check{\mathcal{Q}}\check{\mathcal{V}} \check{\mathcal{P}}|\hat{\rho}(\tau)\rangle \rangle d\tau.        
\end{equation*}

Igualmente, el término de la exponencial $\check{\mathcal{Q}}\check{\mathcal{V}}\check{\mathcal{P}}$ no actua en la ecuación, por ende queda  

\begin{equation*}
    \frac{d}{dt}\check{\mathcal{P}}|\hat{\rho}(t)\rangle \rangle  = \check{\mathcal{P}}\check{\mathcal{L}}_{0}\check{\mathcal{P}}|\hat{\rho}(t)\rangle \rangle  + \check{\mathcal{P}}\check{\mathcal{V}}\check{\mathcal{Q}}\int_{0}^{t}\exp \left[\left(\check{\mathcal{Q}}\check{\mathcal{V}}\check{\mathcal{Q}} + \check{\mathcal{Q}}\check{\mathcal{L}}_{0}\check{\mathcal{Q}}\right) (t-\tau) \right]\check{\mathcal{Q}}\check{\mathcal{V}} \check{\mathcal{P}}|\hat{\rho}(\tau)\rangle \rangle d\tau.        
\end{equation*}

Al realizar el cambio de variable $\tau \to t-\tau$

\begin{equation*}
    \frac{d}{dt}\check{\mathcal{P}}|\hat{\rho}(t)\rangle \rangle = \check{\mathcal{P}}\check{\mathcal{L}}_{0}\check{\mathcal{P}}|\hat{\rho}(t)\rangle \rangle  + \check{\mathcal{P}}\check{\mathcal{V}}\check{\mathcal{Q}}\int_{0}^{t}\exp \left[\left(\check{\mathcal{Q}}\check{\mathcal{V}}\check{\mathcal{Q}} + \check{\mathcal{Q}}\check{\mathcal{L}}_{0}\check{\mathcal{Q}}\right) (\tau) \right]\check{\mathcal{Q}}\check{\mathcal{V}} \check{\mathcal{P}}|\hat{\rho}(t-\tau)\rangle \rangle d\tau,        
\end{equation*}
es en este punto donde se introducen las aproximaciones perturbativas. Si se considera que el acoplamiento \( g \) es pequeño en comparación con las tasas de disipación \(\kappa_i\), es decir, \( g \ll \kappa_i \), se puede analizar la jerarquía de escalas entre los superoperadores involucrados. En particular, los autovalores del superoperador de interacción \(\check{\mathcal{V}}\) serán del orden de \(g\), mientras que los de \(\check{\mathcal{L}}_0\) serán del orden de \(\kappa_i\). Bajo esta suposición, la contribución de \(\check{\mathcal{V}}\) dentro del exponencial se puede despreciar,  y la evolución estará dada por

\begin{equation*}
    \frac{d}{dt}\check{\mathcal{P}}|\hat{\rho}(t)\rangle \rangle = \check{\mathcal{P}}\check{\mathcal{L}}_{0}\check{\mathcal{P}}|\hat{\rho}(t)\rangle \rangle + \check{\mathcal{P}}\check{\mathcal{V}}\check{\mathcal{Q}}\int_{0}^{t}\exp \left[\left( \check{\mathcal{Q}}\check{\mathcal{L}}_{0}\check{\mathcal{Q}}\right)\tau \right]\check{\mathcal{Q}}\check{\mathcal{V}} \check{\mathcal{P}}|\hat{\rho}(t-\tau)\rangle \rangle d\tau.        
\end{equation*}

Luego, para aplicar aproximación de Markov se supondrá que en escalas de tiempo $1/\kappa_{i}$ que $\check{\mathcal{P}}\hat{\rho}(t-\tau)$ no varia mucho, así poder considerar $\check{\mathcal{P}}\hat{\rho}(t-\tau) = \check{\mathcal{P}}\hat{\rho}(t)$. Además, en los límites del integral se reemplaza $t \to \infty$, así se obtiene 

\begin{equation*}
    \frac{d}{dt}\check{\mathcal{P}}|\hat{\rho}(t)\rangle \rangle = \check{\mathcal{P}}\check{\mathcal{L}}_{0}\check{\mathcal{P}}|\hat{\rho}(t)\rangle \rangle + \check{\mathcal{P}}\check{\mathcal{V}}\check{\mathcal{Q}}\int_{0}^{\infty}\exp \left[\left( \check{\mathcal{Q}}\check{\mathcal{L}}_{0}\check{\mathcal{Q}}\right)\tau \right]d\tau \check{\mathcal{Q}}\check{\mathcal{V}} \check{\mathcal{P}}|\hat{\rho}(t)\rangle \rangle.        
\end{equation*}

Finalmente, la ecuación para la evolución temporal será 

\begin{equation*}
    \frac{d}{dt}\check{\mathcal{P}}|\hat{\rho}(t)\rangle \rangle = (\check{\mathcal{L}}_{0} - \check{\mathcal{P}}\check{\mathcal{V}}\check{\mathcal{Q}}\check{\mathcal{L}}^{-1}_{0}\check{\mathcal{Q}}\check{\mathcal{V}}\check{\mathcal{P}})\check{\mathcal{P}}|\hat{\rho}(t)\rangle \rangle. 
\end{equation*}



\label{appendix5clasic}

\subsection{ Inverso de Drazin }
Dado la matriz $\check{\mathcal{L}}_{0}$, el inverso de Drazin es un tipo de inverso generalizado que consiste en la única matriz que cumple 

\begin{align*}
    \check{\mathcal{L}}_{0}\check{\mathcal{L}}_{0}^{+}\check{\mathcal{L}}_{0} & = \check{\mathcal{L}}_{0} \\
    \check{\mathcal{L}}_{0}^{+}\check{\mathcal{L}}_{0}\check{\mathcal{L}}_{0}^{+} & = \check{\mathcal{L}}_{0}^{+} \\
    \check{\mathcal{L}}_{0}^{+}\check{\mathcal{L}}_{0} & = \check{\mathcal{L}}_{0}\check{\mathcal{L}}_{0}^{+}.
\end{align*}

Si se considera la ecuación \ref{ec5:classicalmodel}, se puede probar que 

\begin{equation*}
    \check{\mathcal{L}}_{0}\check{\mathcal{L}}_{0}^{+} = - \int_{0}^{\infty}d\tau \check{\mathcal{L}}_{0}\exp[ \check{\mathcal{L}}_{0}\tau] \check{Q}= - \int_{0}^{\infty}d\tau \exp[ \check{\mathcal{L}}_{0}\tau]\check{\mathcal{L}}_{0} \check{Q} = - \int_{0}^{\infty}d\tau \exp[ \check{\mathcal{L}}_{0}\tau] \check{Q} \check{\mathcal{L}}_{0} = \check{\mathcal{L}}_{0}^{+}\check{\mathcal{L}}_{0},
\end{equation*}
de manera análoga, se pueden probar las otras 2 ecuaciones.

Finalmente, para poder utilizar el inverso de Drazin en la ecuación, este puede calcularse numéricamente mediante una descomposición espectral. En particular, se utilizan los autovalores y autovectores generalizados del superoperador, tanto por la izquierda como por la derecha

\begin{equation*}
    \check{\mathcal{L}}_{0}|x_{j}\rangle \rangle = \lambda_{j} |x_{j}\rangle \rangle \hspace{10mm}  \langle \langle y_{j}|\check{\mathcal{L}}_{0} = \langle \langle y_{j}|\lambda_{j},
\end{equation*}
donde $\langle \langle y_{i}|x_{j}\rangle \rangle = \delta_{ij}$. 

Si se toma $\lambda_{0}$ como el único autovalor cero del operador $\check{\mathcal{L}}_{0}$, se podrá escribir la descomposición espectral como

\begin{equation*}
    \check{\mathcal{L}}_{0} = \sum_{j\neq 0}\lambda_{j}|x_{j}\rangle \rangle \langle \langle y_{j}|.
\end{equation*}

Finalmente, el inverso de Drazin estará definido por 

\begin{equation*}
    \check{\mathcal{L}}^{+}_{0} = \sum_{j\neq 0}\frac{1}{\lambda_{j}}|x_{j}\rangle \rangle \langle \langle y_{j}|.
\end{equation*}

\label{appendix5drazin}

\subsection{Expresión para los flujos de información}
Para poder calcular los flujos de información, primero se debe ser capaz de determinar $\ln \hat{\rho}_{S}$. Se parte de la matriz densidad descrita en la ecuación \ref{appendix5rhoconcu}, la cual se encuentra diagonalizada, excepto por ciertos bloques no diagonales que deben ser tratados por separado, los cuales son 

\begin{equation*}
    \begin{bmatrix}
        \rho_{100} & \alpha \\
        \alpha^{*} & \rho_{010}
    \end{bmatrix}
    \hspace{10mm}
    \begin{bmatrix}
        \rho_{101} & \beta \\
        \beta^{*} & \rho_{011}
    \end{bmatrix}.
\end{equation*}

Con autovalores dados por

\begin{equation*}
    \lambda_{0\pm} = \frac{ \rho_{100} +\rho_{010} }{2} \pm \frac{( [\rho_{100} - \rho_{010}]^{2} + 4|\alpha|^{2} )^{1/2} }{ 2 }  \hspace{10mm}      \lambda_{1\pm} = \frac{ \rho_{101} +\rho_{011} }{2} \pm \frac{( [\rho_{101} - \rho_{011}]^{2} + 4|\beta|^{2} )^{1/2} }{ 2 }, 
\end{equation*}
si se considera $\alpha = |\alpha|e^{i\phi_{0}}$ y $\beta = |\beta|e^{i\phi_{1}}$, se podrán escribir los autovectores como

\begin{align*}
    v_{0+} &  = [0,0,\cos(\theta_{0}/2)e^{i\phi_{0}/2},\sen(\theta_{0}/2)e^{-i\phi_{0}/2},0,0,0,0]^{T} \\
    v_{0-} &  = [0,0,-\sen(\theta_{0}/2)e^{i\phi_{0}/2},\cos(\theta_{0}/2)e^{-i\phi_{0}/2},0,0,0,0]^{T} \\
    v_{1+} &  = [0,0,0,0,\cos(\theta_{1}/2)e^{i\phi_{1}/2},\sen(\theta_{1}/2)e^{-i\phi_{1}/2},0,0]^{T} \\
    v_{1-} &  = [0,0,0,0,-\sen(\theta_{1}/2)e^{i\phi_{1}/2},\cos(\theta_{1}/2)e^{-i\phi_{1}/2},0,0]^{T},                   
\end{align*}
con $\sen(\theta_{0}) = |\alpha|/\sqrt{\Delta^{2}_{0} + |\alpha|^{2}}$, $\sen(\theta_{1}) = |\beta|/\sqrt{\Delta^{2}_{1} + |\beta|^{2}}$ y

\begin{equation*}
    \Delta_{i} = \frac{(\rho_{10i} - \rho_{01i})}{2},
\end{equation*}
mediante esta expresión, se podrá obtener el logaritmo natural de la matriz 

\begin{equation*}
    \ln \hat{\rho}_{S} = \hat{V}\ln \hat{\rho}_{D} \hat{V}^{-1}.
\end{equation*}
    
Donde $\ln \hat{\rho}_{D}$ es la matriz diagonal 

\begin{equation*}
    \ln \hat{\rho}_{D} = 
    \begin{bmatrix}
        \ln \rho_{000} & 0 & 0 & 0 & 0 & 0 & 0 & 0 \\
        0 & \ln \lambda_{0+} & 0 & 0 & 0 & 0 & 0 & 0 \\
        0 & 0 & \ln \lambda_{0-} & 0 & 0 & 0 & 0 & 0 \\
        0 & 0 & 0 & \ln \rho_{001} & 0 & 0 & 0 & 0 \\
        0 & 0 & 0 & 0 & \ln \rho_{110} & 0 & 0 & 0 \\
        0 & 0 & 0 & 0 & 0 & \ln \lambda_{1+} & 0 & 0 \\
        0 & 0 & 0 & 0 & 0 & 0 & \ln \lambda_{1-} & 0 \\
        0 & 0 & 0 & 0 & 0 & 0 & 0 & \ln \rho_{111} 
        \end{bmatrix},
\end{equation*}
y 

\begin{equation*}
    \hat{V} = 
    \begin{bmatrix}
        1 & 0 & 0 & 0 & 0 & 0 & 0 & 0 \\
        0 & \cos(\theta_{0}/2)e^{i\phi_{0}/2} & -\sen(\theta_{0}/2)e^{i\phi_{0}/2} & 0 & 0 & 0 & 0 & 0 \\
        0 & \sen(\theta_{0}/2)e^{-i\phi_{0}/2} & \cos(\theta_{0}/2)e^{-i\phi_{0}/2} & 0 & 0 & 0 & 0 & 0 \\
        0 & 0 & 0 & 1 & 0 & 0 & 0 & 0 \\
        0 & 0 & 0 & 0 & 1 & 0 & 0 & 0 \\
        0 & 0 & 0 & 0 & 0 &  \cos(\theta_{1}/2)e^{i\phi_{1}/2} & -\sen(\theta_{1}/2)e^{i\phi_{1}/2} & 0 \\
        0 & 0 & 0 & 0 & 0 & \sen(\theta_{1}/2)e^{-i\phi_{1}/2} & \cos(\theta_{1}/2)e^{-i\phi_{1}/2} & 0 \\
        0 & 0 & 0 & 0 & 0 & 0 & 0 & 1 
        \end{bmatrix}.
\end{equation*}

Finalmente, se obtiene

\begin{equation*}
    \ln \hat{\rho}_{S} = 
    \begin{bmatrix}
        \ln \rho_{000} & 0 & 0 & 0 & 0 & 0 & 0 & 0 \\
        0 & a_{0} & c_{0} & 0 & 0 & 0 & 0 & 0 \\
        0 & c^{*}_{0} & b_{0} & 0 & 0 & 0 & 0 & 0 \\
        0 & 0 & 0 & \ln \rho_{001} & 0 & 0 & 0 & 0 \\
        0 & 0 & 0 & 0 & \ln \rho_{110} & 0 & 0 & 0 \\
        0 & 0 & 0 & 0 & 0 & a_{1} & c_{1} & 0 \\
        0 & 0 & 0 & 0 & 0 & c^{*}_{1} & b_{1} & 0 \\
        0 & 0 & 0 & 0 & 0 & 0 & 0 & \ln \rho_{111} 
        \end{bmatrix},
\end{equation*}
con las cantidades 

\begin{align*}
    a_{i} & = \cos^{2}(\theta_{i}/2) \ln \lambda_{i+} + \sen^{2}(\theta_{i}/2) \ln \lambda_{i-} \\
    b_{i} & = \sen^{2}(\theta_{i}/2) \ln \lambda_{i+} + \cos^{2}(\theta_{i}/2) \ln \lambda_{i-}  \\
    c_{i} & = \frac{\sin(\theta_{i})}{2}e^{i\phi_{i}} (\ln \lambda_{i+} - \ln \lambda_{i-}).    
\end{align*}

Así, el logaritmo de la matriz densidad es 

\begin{align*}
    \ln \hat{\rho}_{S} & = \ln \rho_{111}|111\rangle \langle 111| + \ln \rho_{110}|110\rangle \langle 110| + \ln \rho_{001}|001\rangle \langle 001| \\ 
     & + \ln \rho_{000}|000\rangle \langle 000| + a_{0}|100\rangle \langle 100| + b_{0}|010\rangle \langle 010|  \\  
     & + a_{1}|101\rangle \langle 101| + b_{1}|011\rangle\langle 011| + c_{0}|100\rangle \langle 010| \\ 
     & + c^{*}_{0}|010\rangle \langle 100| + c_{1}|101\rangle \langle 011| + c^{*}_{1}|011\rangle \langle 101|.
\end{align*}

Por otro lado, los disipadores que actuan en los operadores de Lindblad $\mathcal{L}_{L}$ y $\mathcal{L}_{R}$ se pueden expresar mediante los operadores 

\begin{align*}
    \mathcal{D}[\hat{d}^{\dagger}_{L}(\textbf{1}-\hat{n}_{D})(\textbf{1}-\hat{n}_{R})\hat{\rho}_{S}] & = \rho_{000}|100\rangle \langle 100| - \rho_{000}|000\rangle \langle 000| \\ 
    \mathcal{D}[\hat{d}^{\dagger}_{R}(\textbf{1}-\hat{n}_{D})(\textbf{1}-\hat{n}_{L})\hat{\rho}_{S}] & = \rho_{000}|010\rangle \langle 010| - \rho_{000}|000\rangle \langle 000| \\  
    \mathcal{D}[\hat{d}_{L}(\textbf{1}-\hat{n}_{D})(\textbf{1}-\hat{n}_{R})\hat{\rho}_{S}] & = \rho_{100}|000\rangle \langle 000| - \rho_{100}|100\rangle \langle 100|\\ 
    \mathcal{D}[\hat{d}_{R}(\textbf{1}-\hat{n}_{D})(\textbf{1}-\hat{n}_{L})\hat{\rho}_{S}] & = \rho_{010}|000\rangle \langle 000| - \rho_{010}|010\rangle \langle 010| \\ 
    \mathcal{D}[\hat{d}^{\dagger}_{L}\hat{n}_{D} \hat{n}_{R}\hat{\rho}_{S}] & = \rho_{011}|111\rangle \langle 111| - \rho_{011}|011\rangle \langle 011| \\  
    \mathcal{D}[\hat{d}^{\dagger}_{R} \hat{n}_{D}\hat{n}_{L}\hat{\rho}_{S}] & = \rho_{101}|111\rangle \langle 111| - \rho_{101}|101\rangle \langle 101| \\ 
     \mathcal{D}[\hat{d}_{L}\hat{n}_{D}\hat{n}_{R}\hat{\rho}_{S}] & = \rho_{111}|011\rangle \langle 011| - \rho_{111}|111\rangle \langle 111|  \\    
      \mathcal{D}[\hat{d}_{R}\hat{n}_{D}\hat{n}_{L}\hat{\rho}_{S}] & = \rho_{111}|101\rangle \langle 101| - \rho_{111}|111\rangle \langle 111|  \\ 
    \mathcal{D}[\hat{d}^{\dagger}_{L}(\textbf{1}-\hat{n}_{D})\hat{n}_{R}\hat{\rho}_{S}] & = \rho_{010}|110\rangle \langle 110| - \rho_{010}|010\rangle \langle 010| - \frac{1}{2}( \alpha|100\rangle \langle 010| + \alpha^{*}|010\rangle \langle 100|) \\ 
    \mathcal{D}[\hat{d}^{\dagger}_{R}(\textbf{1}-\hat{n}_{D})\hat{n}_{L}\hat{\rho}_{S}] & = \rho_{100}|110\rangle \langle 110| - \rho_{100}|100\rangle \langle 100| - \frac{1}{2}( \alpha |100\rangle \langle 010| + \alpha^{*}|010\rangle \langle 100| )  \\  
    \mathcal{D}[\hat{d}_{L}(\textbf{1}-\hat{n}_{D})\hat{n}_{R}\hat{\rho}_{S}] & = \rho_{110}|010\rangle \langle 010| - \rho_{110}|110\rangle \langle 110| \\ 
    \mathcal{D}[\hat{d}_{R}(\textbf{1}-\hat{n}_{D})\hat{n}_{L}\hat{\rho}_{S}] & = \rho_{110}|100\rangle \langle 100| - \rho_{110}|110\rangle \langle 110| \\  
    \mathcal{D}[\hat{d}^{\dagger}_{L}\hat{n}_{D}(\textbf{1}-\hat{n}_{R})\hat{\rho}_{S}] & = \rho_{001}|101\rangle \langle 101| - \rho_{001}|001\rangle \langle 001| \\ 
    \mathcal{D}[\hat{d}^{\dagger}_{R}\hat{n}_{D}(\textbf{1}-\hat{n}_{L})\hat{\rho}_{S}] & = \rho_{001}|011\rangle \langle 011| - \rho_{001}|001\rangle \langle 001| \\  
    \mathcal{D}[\hat{d}_{L}\hat{n}_{D}(\textbf{1}-\hat{n}_{R})\hat{\rho}_{S}] & = \rho_{101}|001\rangle \langle 001| - \rho_{101}|101\rangle \langle 101| - \frac{1}{2}(\beta |101\rangle \langle 011| + \beta^{*}|011\rangle \langle 101| ) \\ 
    \mathcal{D}[\hat{d}_{R}\hat{n}_{D}(\textbf{1}-\hat{n}_{L})\hat{\rho}_{S}] & = \rho_{011}|001\rangle \langle 001| - \rho_{011}|011\rangle \langle 011| - \frac{1}{2}(\beta^{*} |011\rangle \langle 101| + \beta|101\rangle \langle 011|  ).   
\end{align*}

Con estas cantidades, se podrán obtener los flujos de información en el estado estacionario mediante la expresión $\dot{I}_{i} = \text{Tr}[(\mathcal{L}_{i}\hat{\rho}_{S} \ln \hat{\rho_{S}}) ]$. De este modo, se obtiene 

\begin{align*}
    \dot{I}_{L} &  =  \kappa_{L}(f_{L}(\epsilon)[a_{0}\rho_{000} - \rho_{000}\ln \rho_{000}] + (1-f_{L}(\epsilon))[\rho_{100}\ln \rho_{000} - a_{0}\rho_{100} ] )  \\ 
      & + \kappa^{U}_{L}(f_{L}(\epsilon + U)[a_{1}\rho_{001} - \rho_{001}\ln \rho_{001}] + (1-f_{L}(\epsilon + U))[\rho_{101}\ln \rho_{001} -a_{1}\rho_{101} - Re(b^{*}_{1}\beta ) ]   ) \\  
      & + \kappa_{L}( f_{L}(\epsilon + U_{f})[\rho_{010}\ln \rho_{110} -b_{0}\rho_{010} - Re(\alpha c^{*}_{0}) ]  + (1-f_{L}(\epsilon + U_{f}))[b_{0}\rho_{110} - \rho_{110}\ln \rho_{110} ] ) \\  
      & + \kappa_{L}( f_{L}(\epsilon + U + U_{f})[\rho_{011}\ln \rho_{111}- b_{1}\rho_{011}] + (1-f_{L}(\epsilon+U+U_{f}) )[b_{1}\rho_{111} - \rho_{111}\ln \rho_{111}]  ) 
\end{align*}

\begin{align*}
    \dot{I}_{R} &  =  \kappa_{R}(f_{R}(\epsilon)[b_{0}\rho_{000} - \rho_{000}\ln \rho_{000}] + (1-f_{R}(\epsilon))[\rho_{010}\ln \rho_{000} - b_{0}\rho_{010} ] )  \\ 
      & + \kappa^{U}_{R}(f_{R}(\epsilon + U)[b_{1} \rho_{001} - \rho_{001}\ln \rho_{001}] + (1-f_{R}(\epsilon + U))[\rho_{011}\ln \rho_{001} -a_{1}\rho_{011} - Re(b^{*}_{1}\beta ) ]   ) \\  
      & + \kappa_{R}( f_{R}(\epsilon + U_{f})[\rho_{100}\ln \rho_{110} -a_{0}\rho_{100} - Re(\alpha c^{*}_{0}) ]  + (1-f_{R}(\epsilon + U_{f}))[a_{0}\rho_{110} - \rho_{110}\ln \rho_{110} ] ) \\  
      & + \kappa_{R}( f_{R}(\epsilon + U + U_{f})[\rho_{101}\ln \rho_{111}- a_{1}\rho_{101}] + (1-f_{R}(\epsilon+U+U_{f}) )[a_{1}\rho_{111} - \rho_{111}\ln \rho_{111}]  ). 
\end{align*}

Si bien el cálculo de los flujos de información puede realizarse numéricamente, la obtención de expresiones analíticas permite identificar claramente la contribución de las coherencias cuánticas a dichos flujos, manifestada tanto en los parámetros $a_i$, $b_i$ como en los términos $\mathrm{Re}(\alpha c_0^*)$ y $\mathrm{Re}(b_1^* \beta)$. 
Es importante destacar que en el límite en que las coherencias se anulan, se recupera el resultado correspondiente a un sistema clásico

\begin{align*}
 \lim_{\alpha,\beta \to 0}  & \dot{I}_{L}  = \dot{I}_{Lclassic}   =  \kappa_{L}(f_{L}(\epsilon)[\rho_{000}\ln \rho_{100} - \rho_{000}\ln \rho_{000}] + (1-f_{L}(\epsilon))[\rho_{100}\ln \rho_{000} - \rho_{100}\ln \rho_{100} ] )  \\ 
      & + \kappa^{U}_{L}(f_{L}(\epsilon + U)[\rho_{001}\ln \rho_{101} - \rho_{001}\ln \rho_{001}] + (1-f_{L}(\epsilon + U))[\rho_{101}\ln \rho_{001} -\rho_{101}\ln \rho_{101} ]   ) \\  
      & + \kappa_{L}( f_{L}(\epsilon + U_{f})[\rho_{010}\ln \rho_{110} -\rho_{010}\ln \rho_{010} ]  + (1-f_{L}(\epsilon + U_{f}))[\rho_{110}\ln \rho_{010} - \rho_{110}\ln \rho_{110} ] ) \\  
      & + \kappa_{L}( f_{L}(\epsilon + U + U_{f})[\rho_{011}\ln \rho_{111}- \rho_{011}\ln \rho_{011}]) \\   
      & + \kappa_{L}( (1-f_{L}(\epsilon+U+U_{f}) )[\rho_{111}\ln \rho_{011} - \rho_{111}\ln \rho_{111}])   
\end{align*}

\begin{align*}
 \lim_{\alpha,\beta \to 0}  & \dot{I}_{R}  = \dot{I}_{Rclassic}   =  \kappa_{R}(f_{R}(\epsilon)[\rho_{000}\ln \rho_{010} - \rho_{000}\ln \rho_{000}] + (1-f_{R}(\epsilon))[\rho_{010}\ln \rho_{000} - \rho_{010}\ln \rho_{010} ] )  \\ 
      & + \kappa^{U}_{R}(f_{R}(\epsilon + U)[\rho_{001}\ln \rho_{011} - \rho_{001}\ln \rho_{001}] + (1-f_{R}(\epsilon + U))[\rho_{011}\ln \rho_{001} -\rho_{011}\ln \rho_{011} ]   ) \\  
      & + \kappa_{R}( f_{R}(\epsilon + U_{f})[\rho_{100}\ln \rho_{110} -\rho_{100}\ln \rho_{100} ]  + (1-f_{R}(\epsilon + U_{f}))[\rho_{110}\ln \rho_{100} - \rho_{110}\ln \rho_{110} ] ) \\  
      & + \kappa_{R}( f_{R}(\epsilon + U + U_{f})[\rho_{101}\ln \rho_{111}- \rho_{101}\ln \rho_{101}]) \\   
      & + \kappa_{R}( (1-f_{R}(\epsilon+U+U_{f}) )[\rho_{111}\ln \rho_{101} - \rho_{111}\ln \rho_{111}]).   
\end{align*}

\label{apendix5infoflow}

\end{appendixs}