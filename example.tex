% Template:     Tesis LaTeX
% Documento:    Archivo de ejemplo
% Versión:      3.4.0 (23/08/2024)
% Codificación: UTF-8
%
% Autor: Pablo Pizarro R.
%        pablo@ppizarror.com
%
% Manual template: [https://latex.ppizarror.com/tesis]
% Licencia MIT:    [https://opensource.org/licenses/MIT]

% ------------------------------------------------------------------------------
% NUEVO CAPÍTULO
% ------------------------------------------------------------------------------
% A diferencia de Template-Informe, Template-Tesis requiere el uso de capítulos; las secciones, subsecciones, etc son parte de un capítulo. Se recomienda el uso de un capítulo en un archivo distinto
\chapteranum{Introducción}

Uno de los principales objetivos de la termodinámica contemporánea es esclarecer el carácter físico de la información. Este desafío fue anticipado por el célebre experimento mental propuesto por James Clerk Maxwell en 1871, el cual sugiere una aparente violación de la segunda ley de la termodinámica al introducir un "demonio" que manipula la dinámica de un gas ideal mediante información\cite{Maxwell_1871}. Esta paradoja motivó una profunda revisión conceptual, abordada en trabajos fundamentales de Szilard\cite{szilard1964decrease}, Brillouin\cite{brillouin1951maxwell}, Landauer\cite{Landauer_1961} y Bennett\cite{bennett1982thermodynamics}, quienes formalizaron el vínculo entre información y entropía, sentando las bases de la termodinámica de la información. \\

Los avances experimentales en nanotecnología han permitido explorar sistemas de pocas partículas donde las fluctuaciones térmicas no pueden ser despreciadas\cite{douarche2005experimental,wang2005experimental}. En este contexto, las formulaciones tradicionales de la termodinámica pierden validez, lo que exige un marco teórico más general. La termodinámica estocástica cumple este rol, al extender las leyes termodinámicas a regímenes fuera del equilibrio en sistemas clásicos de pequeña escala\cite{van2013stochastic,jarzynski1997nonequilibrium}. \\

Además del estudio de energía y entropía, resulta fundamental entender el rol de la información en estos sistemas. Se deben distinguir dos enfoques según el tipo de dinámica involucrada. El primero considera sistemas no autónomos, cuyo Hamiltoniano depende explícitamente del tiempo y es controlado por un agente externo. En estos casos, el Demonio de Maxwell puede modelarse mediante procesos de medición y retroalimentación, como ha sido mostrado en diversos trabajos\cite{cao2009thermodynamics,sagawa2010generalized}. El segundo enfoque aborda sistemas autónomos, de especial relevancia en contextos biológicos\cite{ehrich2023energy}, donde el sistema evoluciona sin intervención externa bajo gradientes constantes de energía (como diferencias de potencial o temperatura). Aquí, la estructura bipartita del sistema permite identificar un subsistema que actúa como "demonio", acoplado al sistema de interés y modulando su dinámica dependiendo del estado en que se encuentre. \\

En ausencia de un agente externo, cuantificar el flujo de información entre el demonio y el sistema presenta una dificultad adicional, ya que no se dispone del resultado explícito de la medición ni del mecanismo de retroalimentación. Este desafío fue abordado por Horowitz y Esposito mediante el formalismo de la termodinámica estocástica\cite{horowitz2014thermodynamics}, quienes propusieron una definición operacional del flujo de información en sistemas clásicos autónomos.\\

Posteriormente, Ptaszynski y Esposito extendieron este marco al contexto cuántico, cuantificando flujos de información en sistemas abiertos gobernados por dinámicas markovianas\cite{ptaszynski2019thermodynamics}. Para ello, emplearon una ecuación maestra de tipo Lindblad en su forma estándar (GKLS)\cite{gorini1976completely,lindblad1976generators,breuer2002theory}, compatible con una descripción termodinámica consistente cuando se cumple la condición de balance detallado local. Este enfoque permite incorporar coherencias cuánticas —elementos no diagonales de la matriz densidad—, las cuales pueden servir como recursos termodinámicos\cite{ptaszynski2023fermionic,streltsov2017colloquium}.\\

Sin embargo, una dificultad clave en la descripción cuántica es la elección de la base. Si se adopta una base global, como exige la aproximación secular, no siempre existe la posibilidad de distinguir entre flujos locales de información, como se realiza en su contraparte clásica\cite{horowitz2014thermodynamics}. Por otro lado, una formulación en la base local puede revelar la estructura bipartita del sistema, pero ha sido criticada por posibles inconsistencias termodinámicas\cite{levy2014local,novotny2002investigation}.\\

Al aplicar una aproximación secular-parcial a la ecuación de Lindblad, Potts et al.\ \cite{potts2021thermodynamically} propusieron una ecuación maestra termodinámicamente consistente para sistemas abiertos. 
Dicha formulación se expresa en la base local e incorpora términos no seculares, lo que representa una mejora respecto al enfoque estrictamente secular. \\

Trabajos previos han empleado el formalismo desarrollado por Potts para analizar los efectos cuánticos en modelos compuestos por dos puntos cuánticos, cada uno acoplado a un reservorio distinto. Asimismo, se ha investigado cómo dichos efectos influyen en el transporte de partículas~\cite{prech2023entanglement}. Por otro lado, el trabajo de Esposito y Horowitz investigó los flujos de información en un sistema clásico constituido por dos puntos cuánticos, donde uno de ellos está acoplado a dos reservorios y el otro a un único baño térmico. En dicho estudio, se demostró que el sistema actúa como un Demonio de Maxwell autónomo~\cite{horowitz2014thermodynamics}.\\

En este trabajo se estudia un Demonio de Maxwell autónomo en el régimen cuántico mediante la extensión del modelo de dos puntos cuánticos de \cite{horowitz2014thermodynamics} a un sistema de tres puntos cuánticos, denominados $L,R$ y $D$. Para ello se emplea el formalismo de Potts, a partir del cual se deriva una ecuación de Lindblad semilocal que describe la evolución temporal del sistema. Cada punto cuántico está acoplado a un reservorio distinto. Uno de ellos desempeña el rol de Demonio de Maxwell, mientras que los otros dos conforman el subsistema físico $LR$. \\

En este subsistema se introducen interacciones adicionales que permiten explorar la dinámica cuántica, en particular el acoplamiento entre los dos puntos que lo constituyen. Las simulaciones numéricas de la ecuación de Lindblad permiten obtener los flujos de información y diversas magnitudes termodinámicas relevantes. Dichos resultados se calculan en un régimen donde emergen efectos cuánticos como la coherencia y el entrelazamiento entre los sitios que componen el subsistema $LR$. \\ 

Estas magnitudes ---y en especial los flujos de información--- permiten caracterizar el comportamiento del sistema como un Demonio de Maxwell autónomo. 
Finalmente, se construye un modelo semiclásico basado en el formalismo de Nakajima--Zwanzig \cite{zwanzig1966statistical}, el cual considera únicamente las componentes diagonales de la matriz densidad. Este modelo permite comparar los flujos de información con los obtenidos en el tratamiento plenamente cuántico. \\

%%ojo aquí

La tesis está estructurada de la siguiente manera: en el Capítulo 1 se introducen los conceptos fundamentales necesarios para describir la evolución de un sistema cuántico, ya sea como sistema cerrado —mediante la ecuación de Liouville— o como sistema abierto —a través de la ecuación de Lindblad en la aproximación secular. El Capítulo 2 está dedicado al desarrollo del Formalismo de Estadística de Conteo, el cual será empleado para derivar la ecuación de Lindblad termodinámicamente consistente propuesta por Potts. En el Capítulo 3 se introduce el concepto de flujo de información, discutiendo su influencia en la formulación de la segunda ley de la termodinámica tanto en contextos clásicos como cuánticos. Finalmente, en el Capítulo 4 se describe la dinámica del sistema compuesto por tres puntos cuánticos utilizando la ecuación de Lindblad derivada en esta tesis. Además, se analizan los resultados más relevantes obtenidos para dos configuraciones distintas del modelo, en una de las cuales el sistema se comporta como un Demonio de Maxwell autónomo.





% SUB-SECCIÓN
% Las sub-secciones se inician con \subsection, si se quiere una sub-sección
% sin número se pueden usar las funciones \subsectionanum (nuevo subtítulo sin
% numeración) o la función \subsectionanumnoi para crear el mismo subtítulo sin
% numerar y sin aparecer en el índice


% ------------------------------------------------------------------------------
% NUEVO CAPÍTULO
% ------------------------------------------------------------------------------

\chapter{Sistemas cuánticos abiertos}

En este capítulo se presentan los conceptos fundamentales para describir la evolución de un sistema cuántico. 
Primero se aborda la dinámica unitaria de un sistema cerrado; luego se introduce la ecuación de Lindblad como una de las descripciones más utilizadas para sistemas abiertos bajo la aproximación secular; y finalmente se desarrolla una formulación general de la termodinámica de sistemas abiertos, donde el calor y el trabajo intercambiados con los reservorios permiten expresar las leyes de la termodinámica en términos de derivadas temporales de magnitudes físicas bien definidas, como la energía interna, el flujo de calor y la potencia.
\\

La sección \ref{sistemacerrado} se dedica a la evolución unitaria de sistemas cerrados. 
En la sección \ref{SEClindblad} se discute la ecuación de Lindblad y las aproximaciones necesarias para obtenerla, detalladas en la sección \ref{sec:lindblad}. 
Finalmente, en la sección \ref{termogeneral} se presenta la termodinámica de sistemas abiertos, donde se definen rigurosamente el calor y el trabajo, y se establece su papel central en la formulación de las leyes de la termodinámica.

\section{Sistemas cuánticos cerrados}
Un sistema cuántico cerrado puede describirse mediante la matriz densidad $\hat{\rho}$. La evolución de esta matriz está gobernada por el Hamiltoniano del sistema, $\hat{H}(t)$, y en el cuadro de Schrödinger se expresa a través de la ecuación de Liouville--von Neumann ($\hbar = 1$) \cite{breuer2002theory}:
\label{sistemacerrado}

\begin{equation*}
    \frac{d}{dt}\hat{\rho}(t) = -i[\hat{H}(t),\hat{\rho}(t)].
\end{equation*}
La solución está dada por la evolución unitaria

\begin{equation*}
    \hat{U}(t,t_{0}) = \hat{T}_{\leftarrow} \exp \left[ -i \int_{t_{0}}^{t} ds\, \hat{H}(s) \right] 
    \quad \implies \quad 
    \hat{\rho}(t) = \hat{U}(t,t_{0})\,\hat{\rho}(t_{0})\,\hat{U}^{\dagger}(t,t_{0}),
\end{equation*}
donde $\hat{T}_{\leftarrow}$ representa el operador de ordenación temporal cronológica. Este operador organiza los productos de operadores dependientes del tiempo de manera que los instantes crecen de derecha a izquierda.

\subsection{Cuadro de interacción}
Supongamos que el Hamiltoniano del sistema puede separarse en dos partes:

\begin{equation*}
    \hat{H}(t) = \hat{H}_{0} + \hat{H}_{I}(t).
\end{equation*}
En principio esta descomposición puede realizarse de distintas maneras. Sin embargo, en el caso de dos subsistemas, $\hat{H}_{0}$ corresponde a los Hamiltonianos de cada uno de ellos en ausencia de interacción, mientras que $\hat{H}_{I}(t)$ describe la interacción entre ambos.  \\

Si introducimos los operadores unitarios

\begin{equation*}
    \hat{U}_{0}(t,t_{0}) \equiv \exp[-i\hat{H}_{0}(t-t_{0})] 
    \hspace{15mm} 
    \hat{U}_{I}(t,t_0) \equiv \hat{U}^{\dagger}_{0}(t,t_{0})\hat{U}(t,t_{0}),
\end{equation*}
podemos describir la matriz densidad en el cuadro de interacción como

\begin{equation*}
    \hat{\rho}_{I}(t) \equiv \hat{U}_{I}(t,t_{0})\hat{\rho}(t_{0})\hat{U}^{\dagger}_{I}(t,t_{0}).
\end{equation*}
La evolución en este cuadro está gobernada por la ecuación

\begin{equation}
    \frac{d}{dt}\hat{\rho}_{I}(t) = -i[\tilde{H}_{I}(t), \hat{\rho}_{I}(t)],
    \label{sec11:interactionp}
\end{equation}
donde el Hamiltoniano de interacción transformado está dado por

\begin{equation*}
    \tilde{H}_{I}(t) = \hat{U}^{\dagger}_{0}(t,t_{0})\,\hat{H}_{I}(t)\,\hat{U}_{0}(t,t_{0}).
\end{equation*}

\label{sec:closedQM}



\section{Ecuación de Lindblad}
En esta sección se presenta la derivación de la ecuación de Lindblad, la cual describe la evolución de un sistema cuántico abierto. En primer lugar, se introduce el concepto de sistema abierto y se define la matriz densidad reducida. Posteriormente, se deriva la ecuación maestra en la aproximación de Born--Markov y se presenta la forma estándar de la ecuación de Lindblad bajo la aproximación secular.

\label{SEClindblad}
\subsection{Dinámica de un sistema abierto}

Un sistema abierto consiste en un sistema cuántico $S$, denominado sistema reducido, que está acoplado a un sistema $B$, denominado ambiente. Ambos constituyen subsistemas del sistema total $S+B$. Sea $\mathcal{H}_{S}$ el espacio de Hilbert del sistema y $\mathcal{H}_{B}$ el espacio de Hilbert del ambiente. Entonces, el espacio de Hilbert del sistema total $S+B$ se escribe como $\mathcal{H} = \mathcal{H}_{S} \otimes \mathcal{H}_{B}$. El Hamiltoniano total está compuesto por el Hamiltoniano del sistema $\hat{H}_{S}$, el Hamiltoniano del ambiente $\hat{H}_{B}$ y la interacción $\hat{H}_{I}(t)$:

\begin{equation}
    \hat{H}(t) = \hat{H}_{S} \otimes \mathbf{I}_{B} + \mathbf{I}_{S} \otimes \hat{H}_{B} + \hat{H}_{I}(t).
    \label{sec2:sistemabierto}
\end{equation}
Para utilizar la ecuación \eqref{sec2:sistemabierto} es necesario separar el sistema en dos subsistemas, $S$ y $B$, de modo que el espacio de Hilbert total pueda escribirse como $\mathcal{H} = \mathcal{H}_{S}\otimes \mathcal{H}_{B}$. Sin embargo, en el caso de partículas idénticas como los fermiones, la identificación de estos dos subsistemas no es inmediata. Aun así, en el formalismo de segunda cuantización el sistema puede describirse mediante una base del espacio de Fock, lo que permite encontrar subsistemas en los cuales el espacio de Hilbert total se construye como un producto tensorial antisimétrico entre ellos \cite{friis2013fermionic,vidal2021quantum}.
\\
Los observables asociados al sistema $S$ son de la forma $A\otimes \mathbf{I}_{B}$ y su valor de expectación se calcula como

\begin{equation*}
    \langle A \rangle = \text{Tr}_{S}\{A \hat{\rho}_{S} \},
\end{equation*}
donde la matriz densidad reducida está dada por

\begin{equation*}
    \hat{\rho}_{S} = \text{Tr}_{B}\{ \hat{\rho} \}.
\end{equation*}
Aquí, $\text{Tr}_{S}$ denota la traza sobre el espacio de Hilbert del sistema y $\text{Tr}_{B}$ la traza sobre el espacio del ambiente. El objeto principal de estudio es la matriz densidad reducida $\hat{\rho}_{S}$, cuya evolución será descrita mediante la ecuación de Lindblad.

\subsection{Ecuación maestra}
Sea un sistema cuántico $S$ débilmente acoplado a un reservorio $B$, cuyo Hamiltoniano total está dado por
\begin{equation*}
    \hat{H} = \hat{H}_{S} + \hat{H}_{B} + \hat{H}_{I}.
\end{equation*}
En el cuadro de interacción, la evolución del estado total se describe mediante
\begin{equation}
    \frac{d}{dt}\hat{\rho}_{I}(t) = -i[\hat{H}_{I}(t), \hat{\rho}_{I}(t)],
    \label{eqsec2:linbladpic}
\end{equation}
cuya forma integral es
\begin{equation*}
    \hat{\rho}_{I}(t) = \hat{\rho}(0) - i \int_{0}^{t} ds \,[\hat{H}_{I}(s), \hat{\rho}_{I}(s)].
\end{equation*}
Reinsertando esta expresión en la ecuación \eqref{eqsec2:linbladpic} y tomando la traza sobre los grados de libertad del reservorio, se obtiene
\begin{equation}
    \frac{d}{dt}\hat{\rho}_{IS}(t) = -i \,\text{Tr}_{B}\{[\hat{H}_{I}(t), \hat{\rho}(0)] \}  
    - \int_{0}^{t} ds \,\text{Tr}_{B}\{[\hat{H}_{I}(t), [\hat{H}_{I}(s), \hat{\rho}_{I}(s)]]\}.
\label{eqsec2:linbladpic2}
\end{equation}
A continuación se adoptan dos hipótesis estándar. En primer lugar,
\[
\text{Tr}_{B}\{[\hat{H}_{I}(t), \hat{\rho}(0)] \}=0,
\]
sin pérdida de generalidad \cite{breuer2002theory}. En segundo lugar, se introduce la aproximación de Born: para acoplamiento débil, la influencia del sistema sobre el reservorio es despreciable, por lo que el estado del baño se asume constante en la ecuación \eqref{eqsec2:linbladpic2} y el total se factoriza como
\begin{equation*}
    \hat{\rho}(t) \approx \hat{\rho}_{S}(t)\otimes \hat{\rho}_{B}.
\end{equation*}
Con esta suposición, la ecuación \eqref{eqsec2:linbladpic2} toma la forma
\begin{equation}
    \frac{d}{dt}\hat{\rho}_{IS}(t) = - \int_{0}^{t} ds \,\text{Tr}_{B}\{[\hat{H}_{I}(t), [\hat{H}_{I}(s), \hat{\rho}_{IS}(s) \otimes \hat{\rho}_{B}]]\}.
\end{equation}
La ecuación anterior es no markoviana, pues depende del historial $\hat{\rho}_{IS}(s)$. Para obtener una descripción local en el tiempo se aplica la aproximación de Markov, de modo que la evolución en el instante $t$ dependa únicamente de $\hat{\rho}_{IS}(t)$:
\begin{equation*}
    \frac{d}{dt}\hat{\rho}_{IS}(t) = - \int_{0}^{t} ds \,\text{Tr}_{B}\{[\hat{H}_{I}(t), [\hat{H}_{I}(s), \hat{\rho}_{IS}(t) \otimes \hat{\rho}_{B}]]\}.
\end{equation*}
Esta expresión corresponde a la ecuación de Redfield. Mediante el cambio de variable $s \mapsto t-s$, se obtiene
\begin{equation}
    \frac{d}{dt}\hat{\rho}_{IS}(t) = - \int_{0}^{t} ds \,\text{Tr}_{B}\{[\hat{H}_{I}(t), [\hat{H}_{I}(t-s), \hat{\rho}_{IS}(t) \otimes \hat{\rho}_{B}]]\}.
    \label{eq3sec2:markov}
\end{equation}

Las funciones de correlación del reservorio decaen en una escala de tiempo característica $\tau_{B}$, mientras que el sistema se relaja en una escala $\tau_{R}$. La aproximación markoviana requiere $\tau_{R}\gg \tau_{B}$, lo que implica que el sistema no resuelve la dinámica del baño. En este régimen, el límite superior del integral en la ecuación \eqref{eq3sec2:markov} puede extenderse a $\infty$, dado que el integrando se anula para tiempos largos:
\begin{equation}
    \frac{d}{dt}\hat{\rho}_{IS}(t) = - \int_{0}^{\infty} ds \,\text{Tr}_{B}\{[\hat{H}_{I}(t), [\hat{H}_{I}(t-s), \hat{\rho}_{IS}(t) \otimes \hat{\rho}_{B}]]\}.
    \label{eq3sec2:markov1}
\end{equation}
En el cuadro de Schrödinger, la interacción se escribe en la forma general
\begin{equation*}
    \hat{H}_{I} = \sum_{\alpha} A_{\alpha} \otimes B_{\alpha},
\end{equation*}
donde $A_{\alpha}=A_{\alpha}^{\dagger}$ actúa en el espacio de Hilbert del sistema, mientras que $B_{\alpha}=B_{\alpha}^{\dagger}$ actúa en el del reservorio. Es conveniente descomponer $A_{\alpha}$ en la base de autoestados de $\hat{H}_{S}$. Si $\{\Pi(\epsilon)=|\epsilon\rangle\langle\epsilon|\}$ son los proyectores sobre autovalores $\epsilon$, se definen los operadores
\begin{equation}
    A_{\alpha}(\omega) \equiv \sum_{\epsilon'-\epsilon=\omega} \Pi(\epsilon)A_{\alpha}\Pi(\epsilon'),
    \label{sec1:globalA}
\end{equation}
que cumplen
\[
\sum_{\omega} A_{\alpha}(\omega) = A_{\alpha}, \quad
[\hat{H}_{S}, A_{\alpha}(\omega)] = -\omega A_{\alpha}(\omega), \quad
[\hat{H}_{S}, A_{\alpha}^{\dagger}(\omega)] = \omega A_{\alpha}^{\dagger}(\omega).
\]
Aplicando la identidad de Baker--Campbell--Hausdorff,
\begin{equation}
    e^{A}Be^{-A} = B + [A,B] + \tfrac{1}{2}[A,[A,B]] + \cdots,
    \label{sec2lind:baker}
\end{equation}
se obtiene en el cuadro de interacción
\begin{equation}
    \hat{H}_{I}(t) = \sum_{\alpha,\omega} e^{-i\omega t} A_{\alpha}(\omega) \otimes B_{\alpha}(t)
    = \sum_{\alpha,\omega} e^{i\omega t} A_{\alpha}^{\dagger}(\omega) \otimes B_{\alpha}^{\dagger}(t),
    \label{seclindbladinteraction1}
\end{equation}
donde $B_{\alpha}(t)=e^{i\hat{H}_{B}t}B_{\alpha}e^{-i\hat{H}_{B}t}$.
\\

Sustituyendo la ecuación \eqref{seclindbladinteraction1} en la ecuación \eqref{eq3sec2:markov1} y evaluando la traza parcial sobre el reservorio, aparecen las funciones de correlación espectrales, que en equilibrio son homogéneas en el tiempo:
\begin{equation*}
    \Gamma_{\alpha\beta}(\omega) = \int_{0}^{\infty} ds \, e^{i\omega s}\, 
    \text{Tr}_{B}\{B_{\alpha}^{\dagger}(t)B_{\beta}(t-s)\hat{\rho}_{B}\}.
\end{equation*}
Bajo esta definición, la evolución del sistema se escribe como
\begin{align*}
    \frac{d}{dt}\hat{\rho}_{IS}(t) &=
    \sum_{\omega,\omega'}\sum_{\alpha,\beta} e^{i(\omega'-\omega)t}\, \Gamma_{\alpha\beta}(\omega)\,
    \Big(A_{\beta}(\omega)\hat{\rho}_{IS}(t)A_{\alpha}^{\dagger}(\omega')
        - A_{\alpha}^{\dagger}(\omega')A_{\beta}(\omega)\hat{\rho}_{IS}(t)\Big) \\
    &+ \sum_{\omega,\omega'}\sum_{\alpha,\beta} e^{-i(\omega'-\omega)t}\, \Gamma^{*}_{\beta\alpha}(\omega)\,
    \Big(A_{\alpha}(\omega')\hat{\rho}_{IS}(t)A_{\beta}^{\dagger}(\omega)
        - \hat{\rho}_{IS}(t)A_{\beta}^{\dagger}(\omega)A_{\alpha}(\omega')\Big).
\end{align*}

Sea $\tau_{S}=|\omega-\omega'|^{-1}$ la escala asociada a las oscilaciones entre frecuencias de Bohr. Si además se cumple $\tau_{R}\gg \tau_{S}$, los términos no seculares ($\omega\neq\omega'$) pueden despreciarse, ya que los factores altamente oscilantes $e^{\pm i(\omega-\omega')t}$ se promedian a cero en la escala $\tau_{R}$. Esta es la llamada aproximación secular \cite{breuer2002theory}. Bajo esta aproximación, la evolución se reduce a
\begin{align*}
    \frac{d}{dt}\hat{\rho}_{IS}(t) &=
    \sum_{\omega}\sum_{\alpha,\beta} \Gamma_{\alpha\beta}(\omega)\,
    \Big(A_{\beta}(\omega)\hat{\rho}_{IS}(t)A_{\alpha}^{\dagger}(\omega)
        - A_{\alpha}^{\dagger}(\omega)A_{\beta}(\omega)\hat{\rho}_{IS}(t)\Big) \\
    &+ \sum_{\omega}\sum_{\alpha,\beta} \Gamma^{*}_{\beta\alpha}(\omega)\,
    \Big(A_{\alpha}(\omega)\hat{\rho}_{IS}(t)A_{\beta}^{\dagger}(\omega)
        - \hat{\rho}_{IS}(t)A_{\beta}^{\dagger}(\omega)A_{\alpha}(\omega)\Big).
\end{align*}
Separando la función espectral en parte real e imaginaria,
\begin{equation*}
    \Gamma_{\alpha\beta}(\omega) = \tfrac{1}{2}\gamma_{\alpha\beta}(\omega) + iS_{\alpha\beta}(\omega),
\end{equation*}
se obtiene la ecuación de Lindblad en el cuadro de interacción:
\begin{equation}
    \frac{d}{dt}\hat{\rho}_{IS}(t) = -i[\hat{H}_{LS}, \hat{\rho}_{IS}(t)] + \mathcal{L}_{g}(\hat{\rho}_{IS}(t)),
    \label{seclindbladfinal}
\end{equation}
con Hamiltoniano de \emph{Lamb shift}
\begin{equation*}
    \hat{H}_{LS} = \sum_{\omega}\sum_{\alpha,\beta} S_{\alpha\beta}(\omega)A_{\alpha}^{\dagger}(\omega)A_{\beta}(\omega),
\end{equation*}
y disipador en forma GKLS
\begin{equation*}
    \mathcal{L}_{g}(\hat{\rho}_{IS}(t)) = \sum_{\omega}\sum_{\alpha,\beta} \gamma_{\alpha\beta}(\omega)\,
    \left[A_{\beta}(\omega)\hat{\rho}_{IS}(t)A_{\alpha}^{\dagger}(\omega) - \tfrac{1}{2}\{A_{\alpha}^{\dagger}(\omega)A_{\beta}(\omega), \hat{\rho}_{IS}(t)\}\right].
\end{equation*}
Esta forma garantiza la preservación de traza, hermiticidad y positividad de la matriz densidad \cite{manzano2020short}.
\\

Finalmente, en el cuadro de Schrödinger la ecuación de Lindblad se escribe como
\begin{equation}
    \frac{d}{dt}\hat{\rho}_{S}(t) = -i[\hat{H}_{S}+\hat{H}_{LS}, \hat{\rho}_{S}(t)] + \mathcal{L}_{g}(\hat{\rho}_{S}(t)).
    \label{eqsec2:lindbladS}
\end{equation}
Si el sistema $S$ interactúa con múltiples reservorios, el disipador se descompone como suma de contribuciones de cada reservorio $\alpha$:
\begin{equation*}
    \mathcal{L}_{g}(\hat{\rho}_{S}(t)) = \sum_{\alpha}\mathcal{L}_{\alpha,g}(\hat{\rho}_{S}(t)).
\end{equation*}

\label{sec:lindblad}

\section{Termodinámica de sistemas abiertos}
\label{termogeneral}
En esta sección se introduce la termodinámica de sistemas abiertos, siguiendo las referencias \cite{esposito2010entropy,potts2021thermodynamically}. Se definen las cantidades termodinámicas y se formulan las leyes de la termodinámica en un marco general válido tanto en equilibrio como fuera de equilibrio. 

\subsection{Cantidades termodinámicas}
\label{cantidadestermo}
Se considera un sistema descrito por el Hamiltoniano total
\begin{equation*}
    \hat{H}_{tot}(t) = \hat{H}_{S}(t) + \sum_{\alpha}\big(\hat{H}_{\alpha} + \hat{V}_{\alpha}\big)  
    = \hat{H}_{S}(t) + \hat{H}_{B} + \hat{V},
\end{equation*}
donde $\hat{H}_{S}(t)$ representa el Hamiltoniano del sistema, que en general puede depender explícitamente del tiempo; $\hat{H}_{B}=\sum_{\alpha}\hat{H}_{\alpha}$ corresponde a los Hamiltonianos de los reservorios térmicos; y $\hat{V}=\sum_{\alpha}\hat{V}_{\alpha}$ describe la interacción sistema–baño.
\\

El sistema intercambia energía y partículas con los reservorios. En consecuencia, el cambio de energía puede descomponerse en dos contribuciones: una asociada al calor y otra al trabajo. El calor liberado por el reservorio $\alpha$ en el intervalo de tiempo $[0,t]$ se define como
\begin{equation}
    \langle Q_{\alpha}\rangle = - \text{Tr}\{(\hat{H}_{\alpha} - \mu_{\alpha}\hat{N}_{\alpha})\hat{\rho}_{tot}(t) \} 
    + \text{Tr}\{(\hat{H}_{\alpha} - \mu_{\alpha}\hat{N}_{\alpha})\hat{\rho}_{tot}(0) \},
    \label{sec3:calor}
\end{equation}
donde $\hat{N}_{\alpha}$ es el operador número de partículas en el reservorio $\alpha$ y $\mu_{\alpha}$ su potencial químico. Por convención, $Q_{\alpha}>0$ corresponde al calor transferido hacia el sistema. El trabajo promedio asociado al reservorio $\alpha$ se define como
\begin{equation}
    \langle W_{\alpha}\rangle = - \mu_{\alpha} \Big( \text{Tr}\{\hat{N}_{\alpha} \hat{\rho}_{tot}(t) \} - \text{Tr}\{\hat{N}_{\alpha}\hat{\rho}_{tot}(0) \}\Big).
    \label{sec3:trabajo}
\end{equation}
Estas definiciones permiten establecer las leyes de la termodinámica para sistemas abiertos.
\label{Leyestermo}

\subsection{Ley cero}
Considérese un sistema total compuesto por un sistema reducido y un baño térmico, ambos en equilibrio caracterizado por una temperatura inversa $\beta$ y un potencial químico $\mu$. El estado de equilibrio del sistema total es el estado gran canónico
\begin{equation*}
    \hat{\rho}^{eq}_{tot} = \frac{e^{-\beta(\hat{H}_{tot} - \mu \hat{N}_{tot})}}{Z},  
    \hspace{10mm} 
    Z = \text{Tr}\{e^{-\beta(\hat{H}_{tot} - \mu \hat{N}_{tot})} \},
\end{equation*}
con $Z$ la función gran partición. El estado de equilibrio del sistema reducido corresponde a
\begin{equation*}
    \hat{\rho}_{S} = \frac{1}{Z}\text{Tr}_{B}\{ e^{-\beta(\hat{H}_{tot} - \mu \hat{N}_{tot})} \}.
\end{equation*}
En el límite de acoplamiento débil entre sistema y baño, este estado converge a la forma \cite{geva2000second}
\begin{equation*}
    \hat{\rho}_{S} = \frac{e^{-\beta(\hat{H}_{S} - \mu\hat{N}_{S})}}
    {\text{Tr}_{S}\{e^{-\beta(\hat{H}_{S} - \mu \hat{N}_{S})} \} }.
\end{equation*}

\subsection{Primera Ley}
En mecánica estadística fuera del equilibrio, el sistema suele alcanzar un estado estacionario en el cual se establecen flujos constantes de energía y partículas. 
En este contexto resulta conveniente distinguir las contribuciones asociadas al calor y al trabajo intercambiados con cada reservorio $\alpha$. 
Estas se introducen mediante la corriente de calor y la potencia
\begin{equation*}
    J_{\alpha}(t) = \partial_{t}\langle Q_{\alpha}\rangle, 
    \hspace{12mm}  
    \dot{W}_{\alpha}(t) = \partial_{t}\langle W_{\alpha}\rangle.
\end{equation*}
En el régimen de acoplamiento débil y para un Hamiltoniano independiente del tiempo, la primera ley adopta la forma
\begin{equation*}
    \partial_{t}E(t) = \sum_{\alpha}[J_{\alpha}(t) + \dot{W}_{\alpha}(t) ],
    \hspace{12mm}  
    E(t) = \text{Tr}\{\hat{H}_{S}\hat{\rho}_{tot}(t) \},
\end{equation*}
donde $E(t)$ es la energía interna del sistema, evaluada como el valor esperado del Hamiltoniano del sistema con respecto al estado total.

\subsection{Segunda Ley}
La formulación de la segunda ley requiere asumir una condición inicial no correlacionada entre el sistema y los reservorios:
\begin{equation*}
    \hat{\rho}_{tot}(0) = \hat{\rho}_{S}(0)\otimes_{\alpha}\hat{\tau}_{\alpha}, 
    \hspace{10mm}  
    \hat{\tau}_{\alpha} = \frac{e^{-\beta_{\alpha}(\hat{H}_{\alpha}-\mu_{\alpha}\hat{N}_{\alpha}) }}
    {\text{Tr}\{e^{-\beta_{\alpha}(\hat{H}_{\alpha} - \mu_{\alpha}\hat{N}_{\alpha})}\} }.
\end{equation*}
Donde cada reservorio $\alpha$ se encuentra en equilibrio gran canónico con temperatura inversa $\beta_{\alpha}$ y potencial químico $\mu_{\alpha}$. Con esta condición inicial, la segunda ley se expresa en términos de la producción de entropía $\sigma(t)$ \cite{esposito2010entropy}:
\begin{equation*}
    \sigma(t) \equiv \Delta S(t) - \sum_{\alpha}\beta_{\alpha}\langle Q_{\alpha}\rangle \geq 0,
\end{equation*}
donde $\Delta S(t)$ es el cambio en la entropía de von Neumann del sistema,
\begin{equation*}
    \Delta S(t) = -  \text{Tr}\{\hat{\rho}_{S}(t)\ln \hat{\rho}_{S}(t) \} 
    +  \text{Tr}\{ \hat{\rho}_{S}(0)\ln \hat{\rho}_{S}(0) \}.
\end{equation*}
La producción total de entropía se descompone en dos contribuciones: el cambio de entropía del sistema y la parte asociada al calor intercambiado con los reservorios. Para sistemas markovianos resulta conveniente estudiar la tasa de producción de entropía, ya que se cumple la desigualdad \cite{strasberg2019non}
\begin{equation*}
    \dot{\sigma}(t) \equiv \partial_{t}\Delta S(t) - \sum_{\alpha}\beta_{\alpha}J_{\alpha}(t) \geq 0.
\end{equation*}
La igualdad se cumple en el caso de procesos reversibles.  
\\

De este modo, las leyes de la termodinámica se expresan en términos de flujos continuos de energía y partículas, los cuales pueden calcularse a partir de la matriz densidad reducida. En esta tesis, tales cantidades serán analizadas principalmente en el régimen estacionario.
\label{sec3sub:leyestermo}



%\chapter{Estadística de conteo(\textit{Full Counting Statistics}) }

\chapter{Ecuación de Lindblad con aproximación secular-parcial}

En este capítulo se presenta el formalismo de \textit{full counting statistics} (FCS), que a partir de la matriz densidad reducida describe la estadística de los flujos de partículas y de energía en sistemas cuánticos, permitiendo cuantificar sus fluctuaciones \cite{esposito2009nonequilibrium,landi2024current}. Para ello, se introduce el esquema de medición en dos puntos, ampliamente utilizado en termodinámica cuántica para definir distribuciones de probabilidad de magnitudes estocásticas, tales como la energía y el número de partículas en los reservorios. A partir de dichas distribuciones es posible obtener momentos y cumulantes que caracterizan las propiedades estadísticas de las variables de interés \cite{esposito2009nonequilibrium}.
\\

A continuación, se introduce el formalismo de superoperadores, que permite expresar la evolución de la matriz densidad en un espacio de Hilbert aumentado. Posteriormente, se desarrolla el cálculo de la ecuación maestra generalizada, deducida en \cite{potts2021thermodynamically} mediante el formalismo FCS. Finalmente, se analiza la consistencia termodinámica, tanto en el caso de la ecuación maestra global como en el de la ecuación maestra secular parcial, a la que en lo sucesivo también denominaremos semilocal.
\\

El capítulo se organiza de la siguiente manera: en la sección \ref{sec2:estadistica2puntos} se introduce la estadística de medición en dos puntos; en la sección \ref{sec2:superop} se presenta el formalismo de superoperadores; en la sección \ref{sec2:master} se desarrolla el cálculo de la ecuación maestra generalizada; y en la sección \ref{sec2:thermolaws} se discute la consistencia termodinámica.

\label{sec2:estadistica}

\section{Estadística de medición en dos puntos}

En esta sección se introduce el esquema de medición en dos puntos, ampliamente utilizado en termodinámica cuántica para definir distribuciones de probabilidad de magnitudes estocásticas\cite{esposito2009nonequilibrium}. El objetivo es caracterizar rigurosamente el calor y el trabajo intercambiados con los reservorios como variables aleatorias, en lugar de limitarse únicamente a sus valores promedio. Aunque dichos promedios pueden calcularse mediante las expresiones \eqref{sec3:calor} y \eqref{sec3:trabajo}, la naturaleza cuántica de la evolución implica que los resultados de una medición sobre los reservorios pueden variar en diferentes repeticiones del experimento. Por ello, resulta natural describir el problema en términos de la distribución de probabilidad asociada al calor y al trabajo.

Consideremos un sistema total con condición inicial factorizada
\[
\hat{\rho}_{tot}(0) = \hat{\rho}_{S}(0)\otimes_{\alpha}\hat{\tau}_{\alpha},
\]
donde $\hat{\tau}_{\alpha}$ es el estado de equilibrio gran canónico del reservorio $\alpha$. Denotemos por $\mathbf{E}=(E_{\alpha})$ y $\mathbf{N}=(N_{\alpha})$ los vectores formados por las energías y los números de partículas de los reservorios, y por $\mathbf{Q}=(Q_{\alpha})$ y $\mathbf{W}=(W_{\alpha})$ los correspondientes calores y trabajos intercambiados. La distribución conjunta de calor y trabajo se escribe como
\begin{align*}
    P(\mathbf{Q},\mathbf{W}) & = \sum_{\mathbf{E},\mathbf{E}',\mathbf{N},\mathbf{N}'} 
    P_{t}(\mathbf{E}',\mathbf{N}'|\mathbf{E},\mathbf{N})\, P_{0}(\mathbf{E},\mathbf{N}) \\
    & \times \prod_{\alpha} \delta\!\left(W_{\alpha}-\mu_{\alpha}(N_{\alpha}-N'_{\alpha})\right)\,
    \delta\!\left(Q_{\alpha}+W_{\alpha}-(E_{\alpha}-E'_{\alpha})\right).
\end{align*}
Aquí, $W_\alpha$ representa el trabajo asociado al intercambio de partículas con el reservorio $\alpha$, mientras que $Q_\alpha$ corresponde al calor transferido desde dicho reservorio hacia el sistema.
\\

La probabilidad inicial de que cada reservorio $\alpha$ tenga energía $E_{\alpha}$ y número de partículas $N_{\alpha}$ está dada por
\[
P_{0}(\mathbf{E},\mathbf{N}) = 
\prod_{\alpha}  
\frac{e^{-\beta_{\alpha}(E_{\alpha}-\mu_{\alpha}N_{\alpha})}}
{\text{Tr}\left\{e^{-\beta_{\alpha}(\hat{H}_{\alpha}-\mu_{\alpha}\hat{N}_{\alpha})}\right\}}.
\]
La probabilidad condicional $P_{t}(\mathbf{E}',\mathbf{N}'|\mathbf{E},\mathbf{N})$ describe la probabilidad de encontrar los reservorios con energías $\mathbf{E}'$ y números de partículas $\mathbf{N}'$ en el tiempo $t$, dados valores iniciales $(\mathbf{E},\mathbf{N})$. Para describir este proceso, se parte del estado proyectado en $t=0$,
\[
\hat{\rho}'(0) =
\frac{\hat{P}_{\mathbf{E},\mathbf{N}}\,\hat{\rho}(0)\,\hat{P}_{\mathbf{E},\mathbf{N}}}
{\text{Tr}\{\hat{P}_{\mathbf{E},\mathbf{N}}\,\hat{\rho}(0)\,\hat{P}_{\mathbf{E},\mathbf{N}}\}},
\]
donde $\hat{P}_{\mathbf{E},\mathbf{N}} = I_{S}\otimes \hat{P}^{B}_{\mathbf{E},\mathbf{N}}$ es el proyector sobre los autoestados de energía y número de partículas de los reservorios, con 
$\hat{P}^{B}_{\mathbf{E},\mathbf{N}} = |\mathbf{E},\mathbf{N}\rangle\langle \mathbf{E},\mathbf{N}|$. Tras la evolución unitaria hasta un tiempo $t$,
\[
\hat{\rho}'(t) = \hat{U}(t)\,\hat{\rho}'(0)\,\hat{U}^{\dagger}(t),
\]
una segunda medición proyectiva sobre los reservorios entrega los valores $\mathbf{E}',\mathbf{N}'$, con probabilidad
\[
P_{t}(\mathbf{E}',\mathbf{N}'|\mathbf{E},\mathbf{N}) =
\text{Tr}\{\hat{P}_{\mathbf{E}',\mathbf{N}'}\,\hat{\rho}'(t)\,\hat{P}_{\mathbf{E}',\mathbf{N}'}\}.
\]
Al reescribir los proyectores explícitamente, se obtiene
\[
P_{t}(\mathbf{E}',\mathbf{N}'|\mathbf{E},\mathbf{N}) =
\text{Tr}_{S}\!\left\{\hat{M}\,\hat{\rho}_{S}(0)\,\hat{M}^{\dagger}\right\},
\qquad
\hat{M} = \text{Tr}_{B}\{|\mathbf{E},\mathbf{N}\rangle\langle \mathbf{E}',\mathbf{N}'|\,\hat{U}(t)\}.
\]

Con la distribución de probabilidad se construye la función generadora de momentos, a partir de la cual pueden obtenerse los momentos de calor y trabajo:
\begin{equation}
    \Lambda(\vec{\lambda},\vec{\chi}) = 
    \int d\mathbf{Q}\, d\mathbf{W}\,P(\mathbf{Q},\mathbf{W})\,
    e^{-i\vec{\lambda}\cdot \mathbf{Q} -i\vec{\chi}\cdot \mathbf{W}}.
    \label{sec2funciongeneradora}
\end{equation}
Esta función puede reescribirse en términos de una evolución modificada de la matriz densidad total \cite{esposito2009nonequilibrium},
\begin{equation}
    \Lambda(\vec{\lambda},\vec{\chi}) = 
    \text{Tr}\{\hat{\rho}_{tot}(\vec{\lambda},\vec{\chi};t)\}, 
    \qquad 
    \hat{\rho}_{tot}(\vec{\lambda},\vec{\chi};t) 
    = \hat{U}(\vec{\lambda},\vec{\chi};t)\,\hat{\rho}_{tot}(0)\,
    \hat{U}^{\dagger}(\vec{\lambda},\vec{\chi};t),
    \label{sec2:evolucionconteo}
\end{equation}
donde
\[
\hat{U}(\vec{\lambda},\vec{\chi};t) = 
e^{\tfrac{i}{2}\sum_{\alpha}[\lambda_{\alpha}(\hat{H}_{\alpha}-\mu_{\alpha}\hat{N}_{\alpha})
+ \chi_{\alpha}\mu_{\alpha}\hat{N}_{\alpha}]} 
\,\hat{U}(t)\,
e^{-\tfrac{i}{2}\sum_{\alpha}[\lambda_{\alpha}(\hat{H}_{\alpha}-\mu_{\alpha}\hat{N}_{\alpha})
+ \chi_{\alpha}\mu_{\alpha}\hat{N}_{\alpha}]}.
\]

Los parámetros $\vec{\lambda}$ y $\vec{\chi}$ se denominan \textit{counting fields}\cite{esposito2009nonequilibrium,landi2024current}, mientras que $\hat{\rho}_{tot}(\vec{\lambda},\vec{\chi};t)$ recibe el nombre de matriz densidad generalizada. El conocimiento de su evolución permite calcular de manera sistemática los momentos estadísticos del calor y del trabajo. Sin embargo, dado que esta matriz se construye a partir de la evolución unitaria de $\hat{\rho}_{tot}(t)$, resulta necesario derivar una ecuación maestra efectiva que describa únicamente los grados de libertad del sistema reducido, lo que se abordará en secciones futuras.


\label{sec2:estadistica2puntos}

\section{Formalismo de superoperadores y espacio de Liouville} 

Un operador en el espacio de Hilbert, representado por $\hat{\rho}$ y de dimensión $N\times N$, puede mapearse al espacio de Liouville mediante un vector columna $|\rho \rangle \rangle$ de dimensión $N^{2}\times 1$. De manera análoga, un superoperador $\mathcal{L}$ que actúa sobre $\hat{\rho}$ en el espacio de Hilbert se representa en el espacio de Liouville por una matriz $\check{\mathcal{L}}$ de dimensión $N^2 \times N^2$, la cual actúa sobre $|\rho\rangle\rangle$. La notación $|\cdot\rangle\rangle$ denota este mapeo, conocido como vectorización, que permite tratar operadores como vectores y superoperadores como matrices.
\\

En este espacio se definen las siguientes operaciones fundamentales:
\begin{equation*}
    \langle \langle A|B\rangle \rangle  \equiv \text{Tr}\{\hat{A}^{\dagger}\hat{B}\},
\end{equation*}
\begin{equation*}
    \check{1}  \equiv \sum_{n,n'}|nn'\rangle \rangle \langle \langle nn'|,
\end{equation*}
\begin{equation*}
     |nn'\rangle \rangle \;\longleftrightarrow\; |n\rangle \langle n'|,
     \hspace{12mm}  
     \langle \langle nn'| \;\longleftrightarrow\; |n'\rangle \langle n|,
\end{equation*}
y se verifican las relaciones
\begin{equation*}
     \langle \langle nn'|mm'\rangle \rangle  = \delta_{nm}\delta_{n'm'},
     \qquad
     \langle \langle nn'|A\rangle \rangle  = \langle n|\hat{A}|n'\rangle,
     \qquad
     \langle \langle 1|A\rangle \rangle  = \text{Tr}\{\hat{A}\}.
\end{equation*}

Si la evolución de la matriz densidad $|\hat{\rho}(t)\rangle \rangle$ está gobernada por el superoperador $\check{\mathcal{L}}$, la dinámica en el espacio de Liouville se describe como
\begin{equation*}
    \frac{d}{dt}|\hat{\rho}(t) \rangle \rangle = \check{\mathcal{L}}\,|\hat{\rho}(t) \rangle \rangle,
\end{equation*}
cuya solución formal es
\begin{equation}
    |\hat{\rho}(t)\rangle \rangle = e^{\check{\mathcal{L}}t}\,|\hat{\rho}(0)\rangle \rangle. 
    \label{sec2liouvilleformal}
\end{equation}
Asimismo, en este marco se definen los superoperadores de proyección de Nakajima–Zwanzig, que actúan sobre la matriz densidad total. En particular, el superoperador $\check{\mathcal{P}}$ proyecta sobre la parte relevante de la dinámica, mientras que $\check{\mathcal{Q}} = \check{1} - \check{\mathcal{P}}$ proyecta sobre la parte irrelevante \cite{zwanzig1966statistical}. Estos operadores cumplen las propiedades algebraicas
\begin{align*}
    & \check{\mathcal{P}} + \check{\mathcal{Q}} = \check{1}, \\
    & \check{\mathcal{P}}^{2} = \check{\mathcal{P}}, \\
    & \check{\mathcal{Q}}^{2} = \check{\mathcal{Q}}, \\
    & \check{\mathcal{P}}\check{\mathcal{Q}} = \check{\mathcal{Q}}\check{\mathcal{P}} = 0.
\end{align*}
La evolución temporal de la matriz densidad descompuesta en estas dos partes satisface
\begin{align*}
    \frac{d}{dt}\check{\mathcal{P}}|\hat{\rho}(t)\rangle \rangle 
    &= \check{\mathcal{P}}\check{\mathcal{L}}\check{\mathcal{P}}|\hat{\rho}(t)\rangle \rangle  
     + \check{\mathcal{P}}\check{\mathcal{L}}\check{\mathcal{Q}}|\hat{\rho}(t)\rangle \rangle, \\
    \frac{d}{dt}\check{\mathcal{Q}}|\hat{\rho}(t)\rangle \rangle  
    &= \check{\mathcal{Q}}\check{\mathcal{L}}\check{\mathcal{Q}}|\hat{\rho}(t)\rangle \rangle  
     + \check{\mathcal{Q}}\check{\mathcal{L}}\check{\mathcal{P}}|\hat{\rho}(t)\rangle \rangle.
\end{align*}


\label{sec2:superop}


\section{Ecuación maestra generalizada}

Considérese un Hamiltoniano total 
\(\hat{H} = \hat{H}_{S} + \hat{H}_{B} + \epsilon \hat{H}_{I} \equiv \hat{H}_{0} + \epsilon \hat{H}_{I}\),
donde \(\epsilon\) es un parámetro adimensional que permitirá implementar posteriormente el régimen de acoplamiento débil. Si este Hamiltoniano se emplea en la evolución unitaria \(\hat{U}(t)\) de la ecuación \eqref{sec2:evolucionconteo}, se obtiene
\begin{equation}
    \frac{d}{dt}\hat{\rho}_{tot}(\vec{\lambda},\vec{\chi},t) 
    = -i[\hat{H}_{0},\hat{\rho}_{tot}(\vec{\lambda},\vec{\chi},t)]
      - i\epsilon\!\left[\hat{V}_{\lambda}\,\hat{\rho}_{tot}(\vec{\lambda},\vec{\chi},t)
      - \hat{\rho}_{tot}(\vec{\lambda},\vec{\chi},t)\,\hat{V}_{-\lambda}\right],
    \label{sec3:ecgeneral}
\end{equation}
con
\begin{equation*}
    \hat{V}_{\lambda} 
    = e^{-\frac{i}{2}\hat{A}(\lambda,\chi)}\,\hat{H}_{I}\,e^{\frac{i}{2}\hat{A}(\lambda,\chi)},
    \qquad 
    \hat{A}(\lambda,\chi) 
    = -\sum_{\alpha}\!\left[\lambda_{\alpha}(\hat{H}_{\alpha}-\mu_{\alpha}\hat{N}_{\alpha})
      + \chi_{\alpha}\mu_{\alpha}\hat{N}_{\alpha}\right].
\end{equation*}
En el formalismo de superoperadores, la ecuación \eqref{sec3:ecgeneral} adopta la forma
\begin{align*}
    \frac{d}{dt}\,|\hat{\rho}_{tot}(\vec{\lambda},\vec{\chi},t)\rangle\rangle
    &= \check{\mathcal{L}}_{\lambda}\,|\hat{\rho}_{tot}(\vec{\lambda},\vec{\chi},t)\rangle\rangle \\
    &= \big(\check{\mathcal{L}}_{0} + \epsilon\,\check{\mathcal{L}}'_{\lambda}\big)\,
       |\hat{\rho}_{tot}(\vec{\lambda},\vec{\chi},t)\rangle\rangle .
\end{align*}

Al pasar al cuadro de interacción se define
\begin{align*}
    |\hat{\rho}_{totI}(\vec{\lambda},\vec{\chi},t)\rangle\rangle 
    &= e^{-\check{\mathcal{L}}_{0}t}\,|\hat{\rho}_{tot}(\vec{\lambda},\vec{\chi},t)\rangle\rangle \\
    &\longleftrightarrow\;
       e^{i\hat{H}_{0}t}\,\hat{\rho}_{tot}(\vec{\lambda},\vec{\chi},t)\,e^{-i\hat{H}_{0}t},
\end{align*}
y el superoperador de Liouville con parámetros de conteo se transforma como
\begin{equation*}
    \check{\mathcal{L}}_{\lambda}(t)
    = e^{-\check{\mathcal{L}}_{0}t}\,\check{\mathcal{L}}_{\lambda}\,e^{\check{\mathcal{L}}_{0}t}.
\end{equation*}
La evolución temporal en el cuadro de interacción queda entonces
\begin{equation}
    \frac{d}{dt}\,|\hat{\rho}_{totI}(\vec{\lambda},\vec{\chi},t)\rangle\rangle
    = \epsilon\,\check{\mathcal{L}}_{\lambda}(t)\,|\hat{\rho}_{totI}(\vec{\lambda},\vec{\chi},t)\rangle\rangle .
    \label{sec2FCS:evolution}
\end{equation}
De este modo, la evolución de los grados de libertad del sistema reducido (en el espacio de Hilbert) se describe por
\begin{multline}
    \dot{\hat{\rho}}_{IS}(\vec{\lambda},\vec{\chi},t) 
    = \epsilon^{2}\!\int_{0}^{t}\!ds \Big[
      - \textnormal{Tr}_{B}\!\left\{\hat{V}_{\lambda}(t)\hat{V}_{\lambda}(t-s)\,\hat{\rho}_{IS}(\vec{\lambda},\vec{\chi},t)\,\hat{\rho}^{eq}_{R}\right\}
      - \textnormal{Tr}_{B}\!\left\{\hat{\rho}_{IS}(\vec{\lambda},\vec{\chi},t)\,\hat{\rho}^{eq}_{R}\,\hat{V}_{-\lambda}(t-s)\hat{V}_{-\lambda}(t)\right\} \\
      + \textnormal{Tr}_{B}\!\left\{\hat{V}_{\lambda}(t)\,\hat{\rho}_{IS}(\vec{\lambda},\vec{\chi},t)\,\hat{\rho}^{eq}_{R}\,\hat{V}_{-\lambda}(t-s)\right\}
      + \textnormal{Tr}_{B}\!\left\{\hat{V}_{\lambda}(t-s)\,\hat{\rho}_{IS}(\vec{\lambda},\vec{\chi},t)\,\hat{\rho}^{eq}_{R}\,\hat{V}_{-\lambda}(t)\right\}
    \Big],
    \label{ecmaestraVlambda}
\end{multline}
donde \(\hat{\rho}^{eq}_{R}=\bigotimes_{\alpha}\hat{\tau}_{\alpha}\) es un estado producto entre los equilibrios de cada reservorio $\alpha$. La demostración de la ecuación \eqref{ecmaestraVlambda} se presenta en el apéndice \ref{apendixsubsectionmatriz}.
\\

Escribiendo la interacción como
\begin{align*}
    \hat{V} &= \sum_{\alpha,k}\hat{S}_{\alpha,k}\,\hat{B}_{\alpha,k}, \\
    \hat{V}_{\lambda} &= \sum_{\alpha,k}\hat{S}_{\alpha,k}\,\hat{B}_{\alpha,k,\lambda}, \\
    \hat{B}_{\alpha,k,\lambda} 
      &\equiv e^{\frac{i}{2}[\lambda_{\alpha}(\hat{H}_{\alpha} - \mu_{\alpha}\hat{N}_{\alpha}) + \chi_{\alpha}\mu_{\alpha}\hat{N}_{\alpha}]}\,
               \hat{B}_{\alpha,k}\,
               e^{-\frac{i}{2}[\lambda_{\alpha}(\hat{H}_{\alpha} - \mu_{\alpha}\hat{N}_{\alpha}) + \chi_{\alpha}\mu_{\alpha}\hat{N}_{\alpha}]},
\end{align*}
los operadores del baño satisfacen
\begin{equation*}
    [\hat{B}_{\alpha,k},\hat{N}_{\alpha}] = n_{\alpha,k}\,\hat{B}_{\alpha,k},
\end{equation*}
donde \(n_{\alpha,k}\) es el número de partículas intercambiado con el baño térmico \(\alpha\). Los operadores del sistema cumplen
\begin{equation*}
    \hat{U}^{\dagger}_{S}(t)\,\hat{S}_{\alpha,k}\,\hat{U}_{S}(t) 
    = \sum_{j} e^{-i\omega_{j}t}\,\hat{S}_{\alpha,k;j},
\end{equation*}
con \(\hat{S}_{\alpha,k;j}\) los operadores de salto y \(\omega_{j}\) las frecuencias de Bohr del Hamiltoniano del sistema. Obsérvese que \(\hat{S}_{\alpha,k}\) no tiene por qué ser autoadjunto (a diferencia del caso considerado en \ref{sec:lindblad}).
\\

Definiendo las funciones de correlación 
\(C^{\alpha}_{k,k'}(s)=\textnormal{Tr}\!\big\{e^{is\hat{H}_{\alpha}}\,\hat{B}^{\dagger}_{\alpha,k}\,e^{-is\hat{H}_{\alpha}}\,\hat{B}_{\alpha,k'}\,\hat{\tau}_{\alpha}\big\}\),
se obtiene la ecuación maestra generalizada
\begin{equation}
    \frac{d}{dt}\hat{\rho}_{IS}(\vec{\lambda},\vec{\chi},t)
    = - \sum_{\alpha,k,k';j,j'} e^{i(\omega_{j}-\omega_{j'})t}
      \int_{0}^{t}\!ds\; \mathcal{I}(s,t),
    \label{ecmaestrafinal}
\end{equation}
donde
\begin{multline}
    \mathcal{I}(s,t) 
    = e^{i\omega_{j'}s} C^{\alpha}_{k,k'}(s)\,
      \hat{S}^{\dagger}_{\alpha,k;j}\hat{S}_{\alpha,k';j'}\,
      \hat{\rho}_{IS}(\vec{\lambda},\vec{\chi},t)
      + e^{-i\omega_{j}s} C^{\alpha}_{k,k'}(-s)\,
        \hat{\rho}_{IS}(\vec{\lambda},\vec{\chi},t)\,
        \hat{S}^{\dagger}_{\alpha,k;j}\hat{S}_{\alpha,k';j'} \\
    - e^{-i\mu_{\alpha}n_{\alpha,k}(\lambda_{\alpha}-\chi_{\alpha})}
      \!\left[
        e^{i\omega_{j'}s} C^{\alpha}_{k,k'}(s-\lambda_{\alpha})
      + e^{-i\omega_{j}s} C^{\alpha}_{k,k'}(-s-\lambda_{\alpha})
      \right]
      \hat{S}_{\alpha,k';j'}\,
      \hat{\rho}_{IS}(\vec{\lambda},\vec{\chi},t)\,
      \hat{S}^{\dagger}_{\alpha,k;j}.
    \label{ecmaestrafinalI}
\end{multline}
Esta ecuación se deduce en el apéndice \ref{finalequation}.
\label{sec2:master}

\subsection{Resolución finita de energía}

De forma análoga a \ref{sec:lindblad}, se busca una dinámica markoviana. Para ello se extiende el límite superior de integración temporal en la ecuación \eqref{ecmaestrafinal} hacia el infinito, lo cual es válido si el tiempo de correlación del baño \(\tau_{B}\) es mucho menor que el tiempo de relajación del sistema \(\tau_{R}\). En presencia de parámetros de conteo \(\lambda_{\alpha}\), las funciones de correlación adquieren una dependencia modificada del tipo \(C^{\alpha}_{k,k'}(\pm \tau-\lambda_{\alpha})\). Por consiguiente, la aproximación de Markov requiere que
\[
    C^{\alpha}_{k,k'}(\pm \tau-\lambda_{\alpha}) \approx 0 
    \quad \text{para } \tau>\tau_{B},
\]
lo que delimita el régimen de validez a
\[
    \tau_{B}\ll \tau_{R},
    \qquad 
    |\lambda_{\alpha}|\ll \tau_{R}.
\]
Esta condición implica una resolución finita en las diferencias de energía que pueden inferirse al medir el calor: \(\lambda_{\alpha}\) y \(Q_{\alpha}\) son variables conjugadas en la función generadora, por lo que obedecen una relación de incertidumbre \cite{folland1997uncertainty}. En particular,
\[
    \langle \Delta \lambda^{2}_{\alpha} \rangle\,\langle (\Delta Q_{\alpha})^{2}\rangle \ge \gamma
    \;\;\Rightarrow\;\;
    \langle (\Delta Q_{\alpha})^{2}\rangle \ge \gamma/\tau_{R}^{2},
\]
con \(\gamma>0\). En consecuencia, diferencias de energía del orden de \(1/\tau_{R}\) dejan de ser resolubles y el promedio del calor puede ser comparable con su varianza. Para mantener la consistencia termodinámica, las leyes deben formularse considerando explícitamente esta resolución finita.
\label{sec2:finiteresol}

\subsection{Agrupación de frecuencias}

La ecuación de Redfield no garantiza en general la preservación de la positividad y puede producir probabilidades negativas en el estado reducido. La vía habitual para asegurar positividad es la aproximación secular (sección \ref{sec:lindblad}), que conduce a la forma GKLS, pero exige que las frecuencias de Bohr estén bien separadas respecto de \(1/\tau_{R}\). Ello excluye cuasi–degeneraciones y suprime términos no seculares \cite{trushechkin2021unified}.
\\

Un enfoque alternativo que conserva positividad sin descartar los términos no seculares parte de la condición markoviana \(\tau_{B}\ll \tau_{R}\), que garantiza que, para dos frecuencias de transición distintas, se cumple al menos una de las desigualdades
\[
|\omega_j-\omega_{j'}|\ll 1/\tau_B
\quad\text{o}\quad
|\omega_j-\omega_{j'}|\gg 1/\tau_R.
\]
De este modo, se agrupan las frecuencias en subconjuntos \(x_q\): si satisfacen la primera desigualdad pertenecen al mismo subconjunto; si satisfacen la segunda, pertenecen a subconjuntos distintos,
\begin{align*}
    |\omega_{j}-\omega_{j'}| \ll 1/\tau_{B}  &\;\Rightarrow\; \omega_{j},\omega_{j'} \in x_{q}, \\
    |\omega_{j}-\omega_{j'}| \gg 1/\tau_{R}  &\;\Rightarrow\; \omega_{j} \in x_{q},\; \omega_{j'} \in x_{q'}.
\end{align*}
Para frecuencias en subconjuntos distintos, la aproximación secular es válida (los factores \(e^{i(\omega_j-\omega_{j'})t}\) se promedian a cero). Dentro de un mismo subconjunto \(x_q\), la resolución finita impide distinguir transiciones cercanas; por lo tanto, en las funciones de correlación espectral
\(\gamma^{\alpha}(\omega)=\int_{-\infty}^{\infty} ds\,e^{i\omega s}C^{\alpha}(s)\)
se sustituyen las frecuencias \(\omega_{j}\in x_q\) por una frecuencia representativa \(\omega_{q}\), es decir,
\[
    e^{i\omega_{j}s},\;e^{i\omega_{j'}s} \to e^{i\omega_{q}s},
    \qquad 
    |\omega_{q}-\omega_{j}|\ll 1/\tau_{B}\;\;\forall\,\omega_{j}\in x_{q}.
\]
Esta sustitución implementa una descripción coarse-grained compatible con la positividad y la resolución energética finita \cite{chruscinski2017brief}.
\\

Aplicando este esquema en la ecuación \eqref{ecmaestrafinal} y extendiendo \(s\in[0,\infty)\), se obtiene la ecuación en forma GKLS
\begin{equation}
    \frac{d}{dt}\hat{\rho}_{IS}(\vec{\lambda},\vec{\chi},t) 
    = -i[\hat{H}_{LS},\hat{\rho}_{IS}(\vec{\lambda},\vec{\chi},t)]
      + \sum_{\alpha}\tilde{\mathcal{L}}^{\chi_{\alpha},\lambda_{\alpha}}_{\alpha}\,
        \hat{\rho}_{IS}(\vec{\lambda},\vec{\chi},t),
    \label{ecmaestracounting}
\end{equation}
con
\begin{equation*}
    \tilde{\mathcal{L}}^{\chi_{\alpha},\lambda_{\alpha}}_{\alpha}\hat{\rho} 
    = \sum_{k,q}\gamma^{\alpha}_{k}(\omega_{q})
      \left[
        e^{\,i\lambda_{\alpha}\omega_{q} + i(\chi_{\alpha}-\lambda_{\alpha})\mu_{\alpha}n_{\alpha,k}}
        \hat{S}_{\alpha,k;q}(t)\hat{\rho}\hat{S}^{\dagger}_{\alpha,k;q}(t)
        - \frac{1}{2}\{\hat{S}^{\dagger}_{\alpha,k;q}(t)\hat{S}_{\alpha,k;q}(t),\hat{\rho}\}
      \right],
\end{equation*}
donde
\begin{equation*}
    \hat{S}_{\alpha,k;q}(t) = \sum_{\{j:\,\omega_{j}\in x_{q}\}} e^{-i\omega_{j}t}\,\hat{S}_{\alpha,k;j}.
\end{equation*}

El Hamiltoniano de \textit{Lamb shift} es 

\begin{equation*}
    \hat{H}_{LS}(t) = \sum_{\alpha,k;q} S^{\alpha}_{k}(\omega_{q})\,
    \hat{S}^{\dagger}_{\alpha,k;q}(t)\hat{S}_{\alpha,k;q}(t),
\end{equation*}
con
\begin{equation}
    \gamma_{k}^{\alpha}(\omega) = \int_{-\infty}^{\infty}\!ds\, e^{i\omega s}\,C^{\alpha}_{k,k}(s),
    \qquad
    S^{\alpha}_{k}(\omega) = - \frac{i}{2} \int_{-\infty}^{\infty}\!ds\, e^{i\omega s}\,\textnormal{sign}(s)\,C^{\alpha}_{k,k}(s).
    \label{sec2:gammaanddelta}
\end{equation}
Aquí se asume, por simplicidad, \(C^{\alpha}_{k,k'}\propto \delta_{k,k'}\). La deducción completa se incluye en el apéndice \ref{apendixGKLSgeneral}.
\\

En el límite \(\vec{\lambda},\vec{\chi}\to 0\) se recupera
\begin{equation*}
    \frac{d}{dt}\hat{\rho}_{IS}(t) 
    = - i[\hat{H}_{LS}(t),\hat{\rho}_{IS}(t)] 
      + \sum_{\alpha}\tilde{\mathcal{L}}_{\alpha}\,\hat{\rho}_{IS}(t),
\end{equation*}
con
\begin{equation*}
    \tilde{\mathcal{L}}_{\alpha} 
    = \sum_{\{q:\,\omega_{q}>0\}} \sum_{k}\gamma^{\alpha}_{k}(\omega_{q})
      \left[
        \mathcal{D}[\hat{S}_{\alpha,k;q}(t)]
        + e^{-\beta_{\alpha}(\omega_{q}-\mu_{\alpha}n_{\alpha,k})}
          \mathcal{D}[\hat{S}^{\dagger}_{\alpha,k;q}(t)]
      \right],
\end{equation*}
y
\begin{equation*}
    \mathcal{D}[\hat{S}_{\alpha,k;q}]\,\hat{\rho}_{S}(t)
    = \hat{S}_{\alpha,k;q}\hat{\rho}_{S}(t)\hat{S}^{\dagger}_{\alpha,k;q}
      - \frac{1}{2}\{\hat{S}^{\dagger}_{\alpha,k;q}\hat{S}_{\alpha,k;q},\hat{\rho}_{S}(t)\}.
\end{equation*}
Para un Hamiltoniano independiente del tiempo, la ecuación en el cuadro de Schrödinger es
\begin{equation}
    \frac{d}{dt}\hat{\rho}_{S}(t) 
    = -i [\hat{H}_{S}+ \hat{H}_{LS},\hat{\rho}_{S}(t)]
      + \sum_{\alpha}\mathcal{L}_{\alpha}\big(\hat{\rho}_{S}(t)\big),
    \label{sec2schrodingerthermo}
\end{equation}
con
\begin{equation}
    \mathcal{L}_{\alpha} 
    = \sum_{\{q:\,\omega_{q}>0\}} \sum_{k}\gamma^{\alpha}_{k}(\omega_{q})
      \left[
        \mathcal{D}[\hat{S}_{\alpha,k;q}]
        + e^{-\beta_{\alpha}(\omega_{q}-\mu_{\alpha}n_{\alpha,k})}
          \mathcal{D}[\hat{S}^{\dagger}_{\alpha,k;q}]
      \right].
    \label{sec2lindbladconsistency}
\end{equation}

Existen dos límites de interés. (i) Si todas las frecuencias satisfacen 
\(|\omega_{j}-\omega_{j'}|\gg 1/\tau_{R}\),
entonces \(\mathcal{D}[\hat{S}_{\alpha,k;q}]=\mathcal{D}[\hat{S}_{\alpha,k;j}]\) y la aproximación secular es válida para todas las transiciones, recuperándose la ecuación de Lindblad \eqref{eqsec2:lindbladS}. (ii) Si 
\(|\omega_{j}-\omega_{j'}|\ll 1/\tau_{B}\),
todas las frecuencias pertenecen a un único subconjunto; en tal caso \(\hat{S}_{\alpha,k;q}=\hat{S}_{\alpha,k}\) y la ecuación maestra queda descrita por operadores locales (ecuación maestra local) \cite{wichterich2007modeling}. A partir de este punto, la ecuación \eqref{sec2schrodingerthermo} será denominada ecuación maestra \textit{semilocal}.



\section{Consistencia termodinámica}
\label{sec2:thermolaws}

Para derivar las leyes de la termodinámica en el marco de las ecuaciones maestras cuánticas es necesario distinguir entre dos escenarios: 
(i) la ecuación maestra global definida en la sección \ref{sec:lindblad}, y 
(ii) la ecuación maestra semilocal, que presenta un mayor desafío. 
En este último caso, la resolución finita de energía impuesta por la dinámica markoviana obliga a redefinir cuidadosamente las magnitudes termodinámicas para asegurar la consistencia. 
Con este fin, se introduce el Hamiltoniano termodinámico $\hat{H}_{TD}$, el cual satisface la relación de conmutación
\begin{equation*}
    [\hat{S}_{\alpha,k;j},\hat{H}_{TD}] = \omega_{q}\hat{S}_{\alpha,k;j},
\end{equation*}
para todas las frecuencias $\omega_{j} \in x_{q}$. 
Dicho Hamiltoniano se construye a partir de $\hat{H}_{S}$, modificando sus autovalores de manera que todas las transiciones dentro de un mismo conjunto $x_q$ compartan una frecuencia efectiva $\omega_{q}$.

\subsection*{Ley cero}

En el caso global, el estado de equilibrio para el baño $\alpha$ cumple
\begin{equation*}
    \mathcal{L}_{\alpha,g}\,e^{-\beta_{\alpha}(\hat{H}_{S} - \mu_{\alpha}\hat{N}_{S})} = 0,
\end{equation*}
y, si todos los reservorios poseen la misma temperatura inversa $\beta$ y el mismo potencial químico $\mu$, 
el estado de Gibbs corresponde a
\begin{equation*}
    \hat{\rho}_{G,g} = \frac{e^{-\beta(\hat{H}_{S} - \mu \hat{N}_{S})}}{\text{Tr}\{ e^{-\beta(\hat{H}_{S} - \mu \hat{N}_{S})}\}}.
\end{equation*}
Por el contrario, para la ecuación semilocal se cumple que
\begin{equation}
    \mathcal{L}_{\alpha}\,e^{-\beta_{\alpha}(\hat{H}_{TD} - \mu_{\alpha}\hat{N}_{S})} = 0,
    \label{sec2cerolaw}
\end{equation}
y, en condiciones de equilibrio, el estado de Gibbs está dado por
\begin{equation*}
    \hat{\rho}_{G} = \frac{e^{-\beta(\hat{H}_{TD} - \mu \hat{N}_{S})}}{\text{Tr}\{ e^{-\beta(\hat{H}_{TD} - \mu \hat{N}_{S})}\}}.
\end{equation*}


\subsection*{Primera ley}

En el caso global, la energía interna se define como
\begin{equation*}
    E_{g}(t) = \text{Tr}\{\hat{H}_{S}\hat{\rho}(t)\}.
\end{equation*}
En la ecuación maestra semilocal, la energía interna se redefine en función del Hamiltoniano termodinámico:
\begin{equation}
    E(t) = \text{Tr}\{\hat{H}_{TD}\hat{\rho}(t)\}.
    \label{sec2:energyeq}
\end{equation}
La corriente de calor y la potencia entregadas por el baño $\alpha$ en el caso global se expresan como
\begin{equation*}
    J_{\alpha,g}(t) = \text{Tr}\{(\hat{H}_{S} - \mu_{\alpha}\hat{N}_S)\,\mathcal{L}_{\alpha,g}\hat{\rho}_{S}(t)\}, 
    \hspace{12mm} 
    \dot{W}_{\alpha,g}(t) = \mu_{\alpha}\,\text{Tr}\{\hat{N}_{S}\,\mathcal{L}_{\alpha,g}\hat{\rho}_{S}(t)\},
\end{equation*}
donde $\mathcal{L}_{\alpha,g}$ es el superoperador de Lindblad asociado al baño $\alpha$, correspondiente a la ecuación maestra global \eqref{eqsec2:lindbladS}.
De manera análoga, para la ecuación semilocal se obtiene
\begin{equation}
    J_{\alpha}(t) = \text{Tr}\{(\hat{H}_{TD} - \mu_{\alpha}\hat{N}_S)\,\mathcal{L}_{\alpha}\hat{\rho}_{S}(t)\}, 
    \hspace{12mm} 
    \dot{W}_{\alpha}(t) = \mu_{\alpha}\,\text{Tr}\{\hat{N}_{S}\,\mathcal{L}_{\alpha}\hat{\rho}_{S}(t)\}.
    \label{sec2:heatworkeq}
\end{equation}

Estas definiciones cumplen automáticamente con la primera ley de la termodinámica. 
En efecto, al derivar la energía interna en el caso global se obtiene
\begin{equation*}
    \partial_{t}E_{g}(t) = \text{Tr}\{\hat{H}_{S}\,\partial_{t}\hat{\rho}_{S}(t)\} 
    = \sum_{\alpha}\big[J_{\alpha,g}(t) + \dot{W}_{\alpha,g}(t)\big].
\end{equation*}
De forma análoga, utilizando la relación de conmutación $[\hat{H}_{TD},\hat{H}_{S}+\hat{H}_{LS}] = 0$, 
se obtiene para el caso semilocal
\begin{equation*}
    \partial_{t}E(t) = \text{Tr}\{\hat{H}_{TD}\,\partial_{t}\hat{\rho}_{S}(t)\} 
    = \sum_{\alpha}\big[J_{\alpha}(t) + \dot{W}_{\alpha}(t)\big].
\end{equation*}


\subsection*{Segunda ley}

La tasa de producción de entropía se define a partir de la desigualdad de Spohn \cite{spohn2007irreversible}, 
la cual garantiza la no negatividad de la entropía producida en procesos irreversibles. 
De este modo, para la ecuación global se obtiene
\begin{equation*}
    \dot{\sigma}_{g}(t) = - \frac{d}{dt}\text{Tr}\{\hat{\rho}_{S}(t)\,\ln \hat{\rho}_{S}(t) \} 
    - \sum_{\alpha} \beta_{\alpha}\, J_{\alpha,g}(t) 
    = -\sum_{\alpha} \text{Tr}\Big\{(\mathcal{L}_{\alpha,g}\hat{\rho}_{S}(t))\,[\ln \hat{\rho}_{S}(t) - \ln \hat{\rho}_{G,g}(\beta_{\alpha},\mu_{\alpha})] \Big\} \geq 0.
\end{equation*}
De manera análoga, en la ecuación semilocal la tasa de producción de entropía es
\begin{equation}
    \dot{\sigma}(t) = - \frac{d}{dt}\text{Tr}\{\hat{\rho}_{S}(t)\,\ln \hat{\rho}_{S}(t) \} 
    - \sum_{\alpha} \beta_{\alpha}\, J_{\alpha}(t) 
    = -\sum_{\alpha} \text{Tr}\Big\{(\mathcal{L}_{\alpha}\hat{\rho}_{S}(t))\,[\ln \hat{\rho}_{S}(t) - \ln \hat{\rho}_{G}(\beta_{\alpha},\mu_{\alpha})] \Big\} \geq 0,
    \label{sec2secondlaw}
\end{equation}
donde la desigualdad asegura la validez de la segunda ley en ambos marcos \cite{potts2021thermodynamically}.
\\

En resumen, la formulación termodinámica basada en la ecuación maestra semilocal requiere redefinir las magnitudes termodinámicas utilizando el Hamiltoniano termodinámico $\hat{H}_{TD}$. Con estas definiciones, se garantiza la validez de las leyes de la termodinámica, incluso en presencia de cuasi–degeneraciones y sin necesidad de descartar términos no seculares.
\\

Contar con un marco termodinámico consistente para sistemas cuánticos abiertos constituye un paso esencial para abordar, en el capítulo siguiente, el estudio del rol de la información y de los flujos de información en estos sistemas.

% ------------------------------------------------------------------------------
% NUEVO CAPÍTULO
% ------------------------------------------------------------------------------
\chapter{Flujos de información} 

Los sistemas físicos no sólo intercambian energía entre sí, sino que también comparten información. Este intercambio de información puede tener consecuencias termodinámicas relevantes, como lo ejemplifica el paradigma del Demonio de Maxwell, que muestra cómo la información puede utilizarse para extraer trabajo de un sistema. Tradicionalmente, la paradoja se ha estudiado en el marco de demonios no autónomos, en los cuales es necesaria la intervención de un agente externo encargado de realizar mediciones y retroalimentación. No obstante, comprender cómo fluye la información en sistemas autónomos resulta igualmente fundamental, dado que en estos casos tanto la dinámica como la retroalimentación son generadas internamente, sin control externo. Este tipo de procesos aparece de manera natural en sistemas biológicos y en dispositivos tecnológicos modernos \cite{parrondo2023information,koski2015chip}, y su implementación experimental puede ser más accesible que la de los demonios no autónomos, donde se requiere una intervención activa del observador.
\\

En este capítulo se introduce el concepto de información termodinámica. En la sección \ref{sec4:Demon} se revisa el paradigma del Demonio de Maxwell como punto de partida. A continuación, en la sección \ref{sec4:autonomo} se presenta el Demonio de Maxwell autónomo, enfatizando su carácter intrínseco. Finalmente, en las secciones \ref{sec4:flujos} y \ref{sec4:flujos0} se formaliza el concepto de flujos de información, utilizando como herramientas la termodinámica estocástica en el marco clásico y la evolución de la matriz densidad en el marco cuántico\cite{horowitz2014thermodynamics,ptaszynski2019thermodynamics}.

\section{Demonio de Maxwell}

El Demonio de Maxwell es un experimento mental propuesto por James Clerk Maxwell en su obra \cite{Maxwell_1871}. Consiste en una caja dividida en dos compartimientos, A y B, cada uno de ellos lleno con un gas ideal a temperatura $T$ y presión $P$, como se muestra en la Figura \ref{img:demon}. Entre ambos compartimientos existe una compuerta sin masa, que permite el paso de partículas de un lado a otro. Esta puerta es controlada por un ente denominado ``Demonio'', el cual tiene la capacidad de manipularla sin costo energético. La característica esencial del Demonio es que conoce la velocidad de cada partícula en ambos compartimientos, de modo que puede discriminar entre partículas rápidas y lentas. Dado que la temperatura está directamente relacionada con la velocidad promedio de las partículas, esta diferencia influye en la temperatura del compartimiento hacia el cual se dirigen. En particular, el Demonio permite el paso del compartimiento A al B únicamente a las partículas más rápidas, mientras que deja pasar del compartimiento B al A las más lentas. Si el número de partículas en cada compartimiento permanece constante, este proceso conduce a un aumento de la temperatura en A y a una disminución en B, generando así una diferencia $\Delta T$ entre ambos lados.
\\

El cambio de entropía del sistema es
\begin{align*}
    \Delta S & = \Delta S_{A} + \Delta S_{B} = C_{V}\left( \log \frac{T-\Delta T}{T} + \log \frac{T+\Delta T}{T} \right) \\
       & =  C_{V} \log \left( 1 - \frac{\Delta T^{2}}{T^{2}}  \right) < 0,
\end{align*}
donde \( C_V \) representa la capacidad calorífica a volumen constante.
\\

De este cálculo se concluye que, sin realizar trabajo externo, el Demonio provoca una disminución de la entropía total, lo que constituye una aparente violación de la segunda ley de la termodinámica.
\\

La paradoja fue abordada a lo largo del siglo XX mediante distintos enfoques. En primer lugar, Szilard introdujo un modelo explícito de motor controlado por un Demonio \cite{szilard1964decrease}, mientras que Brillouin destacó la conexión entre información y entropía \cite{brillouin1951maxwell}. Posteriormente, Landauer formuló el principio que lleva su nombre, estableciendo que el borrado de información conlleva una disipación mínima de energía \cite{Landauer_1961}. Finalmente, Bennett mostró que el costo energético fundamental asociado al Demonio se encuentra en la operación de borrado de memoria \cite{bennett1982thermodynamics}. En conjunto, estos desarrollos restablecen la validez de la segunda ley en presencia de procesos de adquisición y procesamiento de información.
\\

En sistemas no autónomos, es decir, aquellos manipulados por un agente externo a través del control de variables macroscópicas, la paradoja de Maxwell ha sido estudiada en detalle. Un ejemplo clásico es el motor de Szilard \cite{szilard1964decrease}, en el cual se puede cuantificar tanto la energía requerida para realizar una medición como el costo energético asociado al borrado de memoria del Demonio, junto con el trabajo máximo extraíble mediante retroalimentación (\textit{feedback}) \cite{maruyama2009colloquium, sagawa2008second}. No obstante, el caso de sistemas autónomos, en los cuales no existe intervención externa y tanto la dinámica como la retroalimentación son generadas internamente, presenta características fundamentales que requieren un análisis más detallado.

\insertimage[\label{img:demon}]{ejemplos/Maxwelldemon1}{scale=0.9}{Esquema que representa al Demonio de Maxwell. En la primera etapa, ambos compartimientos contienen un gas ideal distribuido homogéneamente. Posteriormente, al permitir el paso selectivo de partículas, el compartimiento A queda enriquecido en partículas lentas (más frías) y el compartimiento B en partículas rápidas (más calientes). Figura adaptada de \cite{link1}.}

\label{sec4:Demon}

\section{Demonio de Maxwell autónomo}

Muchos procesos físicos requieren la interacción entre un conjunto de subsistemas que, en conjunto, conforman un sistema global. Esta interacción no se limita al intercambio de energía o partículas, sino que también implica un intercambio de información, a medida que los subsistemas se correlacionan entre sí de manera autónoma, es decir, sin la intervención de un agente externo que realice retroalimentación (\textit{feedback}). 
\\

Comprender cómo los flujos de información pueden ser utilizados para realizar tareas útiles es de gran relevancia. Un ejemplo claro se encuentra en sistemas biológicos, donde ocurre la adaptación sensorial: un organismo monitorea su entorno mientras ajusta su respuesta al mismo \cite{lan2012energy}. Para caracterizar este tipo de procesos, se considera el Demonio de Maxwell autónomo como un sistema bipartito. Una de sus partes corresponde al sistema controlado, mientras que la otra funciona como detector, el cual ejerce control mediante una interacción física que afecta al sistema controlado. 
\\

Un requisito fundamental es que el sistema sea verdaderamente autónomo, lo que significa que el Hamiltoniano total debe ser independiente del tiempo y que no se admiten intervenciones externas, tales como mediciones explícitas o protocolos de retroalimentación.

\label{sec4:autonomo}
\section{Contexto clásico de los flujos de información}

Para describir un Demonio de Maxwell autónomo es necesario caracterizar la evolución de un sistema acoplado a uno o más reservorios, y por lo tanto, sujeto a las leyes de la termodinámica en un contexto fuera del equilibrio. 
\\

Consideremos dos sistemas independientes, \( X \) e \( Y \), cada uno con estados discretos denotados por \( x \) e \( y \), respectivamente. Ambos sistemas presentan una dinámica interna gobernada por saltos aleatorios entre sus estados, cuyas tasas de transición están determinadas por los reservorios locales a los que se encuentran acoplados. Dichas tasas cumplen la condición de \textit{balance detallado local} \cite{van2015ensemble}. 
\\

La evolución entre estados se modela como un proceso de Markov \cite{van1992stochastic}. Para estudiar la evolución conjunta se introduce un acoplamiento bipartito: en este marco, si el sistema total se encuentra en el estado conjunto \((x,y)\), sólo son posibles transiciones del tipo \((x,y)\to(x,y')\) o \((x,y)\to(x',y)\), mientras que las transiciones simultáneas \((x,y)\to(x',y')\) están prohibidas.
\\

La probabilidad \( p(x,y) \) de encontrar al sistema en el estado \((x,y)\) está gobernada por la ecuación maestra
\[
d_{t}p(x,y) = \sum_{x',y'} \left[ W_{x,x'}^{y,y'}p(x',y') - W_{x',x}^{y',y}p(x,y) \right],
\]
donde la tasa de transición \( W_{x,x'}^{y,y'} \) describe un salto del estado \((x',y')\) al estado \((x,y)\). Para un sistema en contacto con un reservorio térmico, estas tasas obedecen la condición de balance detallado local
\[
\ln \left( \frac{W_{x,x'}^{y,y'}}{W_{x',x}^{y',y}} \right) = -\frac{\epsilon_{x,y}-\epsilon_{x',y'}}{T},
\]
la cual relaciona la razón de tasas de transición con el cambio de energía \(\Delta \epsilon = \epsilon_{x,y}-\epsilon_{x',y'}\), intercambiado con el reservorio en forma de calor. 
\\

Debido a la estructura bipartita, las tasas adoptan la forma
\[
W_{x,x'}^{y,y'} = 
\begin{cases}
w_{x,x'}^{y} & \text{si } x\neq x', \; y=y', \\[2mm]
w_{x}^{y,y'} & \text{si } x=x', \; y\neq y', \\[2mm]
0 & \text{si } x\neq x', \; y\neq y',
\end{cases}
\]
lo que refleja que en cada transición solo varía uno de los subsistemas.
\\

La corriente de probabilidad se define como
\[
J_{x,x'}^{y,y'} = W_{x,x'}^{y,y'}p(x',y') - W_{x',x}^{y',y}p(x,y).
\]
A partir de esta definición, la ecuación maestra puede reescribirse en la forma
\[
d_{t}p(x,y) = \sum_{x',y'}J_{x,x'}^{y,y'} = \sum_{x'}J_{x,x'}^{y} + \sum_{y'}J_{x}^{y,y'}.
\]
Aquí,
\[
J_{x}^{y,y'} = w_{x}^{y,y'}\,p(x,y') - w_{x}^{y',y}\,p(x,y),
\]
representa la corriente de probabilidad asociada a un salto \(y' \to y\) con \(x\) fijo, es decir, describe el flujo en la dirección \( Y \) condicionado al valor de \(x\). De manera análoga, \( J^{y}_{x,x'} \) describe el flujo en la dirección \( X \) condicionado a \(y\).  
\\

Esta descomposición es fundamental, ya que la estructura bipartita permite separar las corrientes en dos componentes: una asociada a transiciones en la dirección \(X\), y otra a transiciones en la dirección \(Y\). Como consecuencia, es posible estudiar funcionales de la corriente, definidos como
\[
\mathcal{A}(J) = \sum_{x,x',y,y'} J_{x,x'}^{y,y'}\, A_{x,x'}^{y,y'},
\]
los cuales se descomponen naturalmente en dos contribuciones:
\begin{equation}
\mathcal{A}(J) = \sum_{x\geq x';\,y\geq y'} J_{x,x'}^{y}A_{x,x'}^{y,y'} \;+\; \sum_{x\geq x';\,y\geq y'} J_{x}^{y,y'} A_{x,x'}^{y,y'}.
\label{sec4:functionalcurrent}
\end{equation}
De esta forma, la variación de \(\mathcal{A}\) puede interpretarse como la suma de dos contribuciones independientes: una correspondiente a transiciones en la dirección \(X\) y otra asociada a transiciones en la dirección \(Y\).

\label{sec4:flujos}
\section{Segunda ley de la termodinámica y flujos de información}

El sistema conjunto \(XY\), al estar en contacto con un baño térmico, constituye un sistema abierto cuya evolución se rige por las leyes de la termodinámica en condiciones fuera del equilibrio. Entre ellas, la segunda ley exige que la tasa de producción de entropía sea siempre no negativa, es decir,
\begin{equation*}
    \dot{\sigma} = \partial_{t}S_{XY} + \dot{S}_{r} \geq 0. 
\end{equation*}
Aquí, \( S_{XY} = - \sum_{x,y} p(x,y) \ln p(x,y) \) corresponde a la entropía de Shannon del sistema conjunto. Su derivada temporal puede expresarse como
\begin{equation*}
    \partial_{t}S_{XY} = \sum_{x\geq x'; y\geq y'} J_{x,x'}^{y,y'} \ln \frac{p(x',y')}{p(x,y)}.
\end{equation*}
El cambio de entropía en el ambiente viene dado por
\begin{equation*}
    \dot{S}_{r} = \sum_{x\geq x'; y\geq y'} J_{x,x'}^{y,y'} \ln \frac{W_{x,x'}^{y,y'}}{W_{x',x}^{y',y}},
\end{equation*}
y por lo tanto
\begin{equation*}
    \dot{\sigma} = \sum_{x\geq x'; y\geq y'} J_{x,x'}^{y,y'} \ln \frac{ W_{x,x'}^{y,y'}p(x',y')  }{ W_{x',x}^{y',y}p(x,y) } \geq 0.
\end{equation*}
La demostración rigurosa de estas expresiones se presenta en el apéndice \ref{apendix4:secondlaw}.  
\\

La entropía de Shannon del sistema \( XY \) mide la incertidumbre asociada a las variables aleatorias \( X \) e \( Y \). Una forma natural de cuantificar las correlaciones entre estas variables es mediante la \textit{información mutua}. Esta magnitud captura cuánto se reduce la incertidumbre en una variable cuando se conoce el valor de la otra, y constituye una medida de la información compartida entre los dos subsistemas \cite{parrondo2015thermodynamics,cover1999elements}. En términos de la probabilidad conjunta, la información mutua se define como
\begin{equation*}
    I_{XY} = \sum_{x,y} p(x,y) \ln \frac{p(x,y)}{p(x)p(y)} \geq 0. 
\end{equation*}
De este modo, un valor elevado de \( I_{XY} \) indica que los subsistemas están fuertemente correlacionados, mientras que \( I_{XY}=0 \) corresponde al caso de independencia estadística. Desde un punto de vista termodinámico, la información mutua permite cuantificar recursos informacionales que pueden emplearse para realizar trabajo o modificar flujos de energía en el sistema \cite{parrondo2015thermodynamics}.  
\\

Para analizar la dinámica de estas correlaciones, se considera la variación temporal de la información mutua. Dicha variación puede descomponerse en dos contribuciones asociadas a las dinámicas locales de cada subsistema:
\[
\partial_t I_{XY} = \dot{I}_{X} + \dot{I}_{Y},
\]
donde \( \dot{I}_{X} \) representa la contribución debida a la evolución del subsistema \( X \) (con \( Y \) fijo), y \( \dot{I}_{Y} \) la correspondiente al subsistema \( Y \)\cite{horowitz2014thermodynamics}. De manera explícita:
\begin{equation}
    \dot{I}_{X}  = \sum_{x\geq x'; y}J_{x,x'}^{y} \ln \frac{ p(y|x) }{p(y|x')},
    \label{sec4:clasicalflows0}
\end{equation}
\begin{equation}    
    \dot{I}_{Y}  = \sum_{x;y\geq y'} J_{x}^{y,y'} \ln \frac{p(x|y)}{ p(x|y') }.
    \label{sec4:clasicalflows}
\end{equation}
Estos términos cuantifican los \textit{flujos de información}. En particular, \( \dot{I}_{X} > 0 \) implica que, en promedio, las transiciones en la dirección \( X \) aumentan la información mutua, lo que equivale a que el subsistema \( X \) obtiene información acerca de \( Y \). Por el contrario, si \( \dot{I}_{X} < 0 \), las transiciones en \( X \) reducen las correlaciones, lo que puede interpretarse como un consumo de información para realizar trabajo o modificar flujos energéticos.
\\

Dado que la tasa de producción de entropía es un funcional de la corriente de probabilidad, se puede emplear la ecuación \eqref{sec4:functionalcurrent} para escribir
\[
\dot{\sigma} = \dot{\sigma}_{X} + \dot{\sigma}_{Y},
\]
con
\begin{align*}
    \dot{\sigma}_{X} & = \sum_{x \geq x';y} J_{x,x'}^{y} \ln  \frac{w_{x,x'}^{y} p(x',y) }{w_{x',x}^{y} p(x,y) } \geq 0, \\
    \dot{\sigma}_{Y} & = \sum_{x;y\geq y'}J_{x}^{y,y'} \ln \frac{w_{x}^{y,y'} p(x,y') }{ w_{x}^{y',y} p(x,y) } \geq 0,
\end{align*}
lo que permite identificar las tasas de producción de entropía locales:
\begin{align*}
    \dot{\sigma}_{X} &  = \sum_{x \geq x';y} J_{x,x'}^{y} \left[ \ln \frac{p(x')}{p(x)}  +\ln \frac{w_{x,x'}^{y}}{ w_{x',x}^{y} } + \ln \frac{p(y|x')}{p(y|x)} \right], \\
    \dot{\sigma}_{Y} &  = \sum_{x;y \geq y'} J_{x}^{y,y'} \left[ \ln \frac{p(y')}{p(y)}  + \ln \frac{w_{x}^{y,y'}}{ w_{x}^{y',y} } + \ln \frac{p(x|y')}{p(x|y)} \right].
\end{align*}

En términos de los flujos de información, estas ecuaciones toman la forma
\begin{align*}
    \dot{\sigma}_{X} & = \partial_{t}S_{X} + \dot{S}_{r}^{X} - \dot{I}_{X} \geq 0, \\
    \dot{\sigma}_{Y} & = \partial_{t}S_{Y} + \dot{S}_{r}^{Y} - \dot{I}_{Y} \geq 0.
\end{align*}
Estas expresiones muestran explícitamente cómo los flujos de información modifican la producción de entropía local en cada subsistema. En particular, si se ignora la interacción entre \( X \) e \( Y \) y sólo se monitorea \( X \), la tasa de producción de entropía asignada sería
\[
\dot{\sigma}_{X}^0 = \partial_{t} S_{X} + \dot{S}_{r}^{X},
\]
la cual puede ser negativa, reflejando una violación aparente de la segunda ley. Este fenómeno se interpreta como la acción efectiva de un Demonio de Maxwell, que aprovecha información oculta en las correlaciones.
\\

En el régimen estacionario, donde \( \partial_t I_{XY} = 0 \), se conserva el flujo de información:
\[
\dot{\mathcal{I}} = \dot{I}_{X} = -\dot{I}_{Y},
\]
y las producciones locales de entropía toman la forma
\begin{align*}
    \dot{\sigma}_{X} & = \dot{S}_{r}^{X} - \dot{\mathcal{I}} \geq 0, \\
    \dot{\sigma}_{Y} & = \dot{S}_{r}^{Y} + \dot{\mathcal{I}} \geq 0.
\end{align*}
En este contexto, si \( \dot{\mathcal{I}} > 0 \), el subsistema \( X \) actúa como un sensor al obtener información sobre \( Y \), lo que requiere disipar al menos una cantidad de entropía equivalente a
\[
\dot{S}_{r}^{X} \geq \dot{\mathcal{I}}.
\]
Por otro lado, el subsistema \( Y \) proporciona información que puede aprovecharse para extraer energía del entorno, cumpliéndose
\[
-\dot{S}_{r}^{Y} \leq \dot{\mathcal{I}}.
\]
De este modo, el flujo de información \( \dot{\mathcal{I}} \) establece una cota fundamental para el trabajo que puede extraerse a partir de correlaciones informacionales. Este principio constituye la base de la termodinámica de la información y será aplicado en el capítulo siguiente a un modelo específico donde el Demonio de Maxwell se manifiesta a través de corriente en contra de gradientes de potencial externos.

\section{Contexto cuántico de los flujos de información}

Para describir los flujos de información en un contexto cuántico es necesario caracterizar la dinámica de un sistema abierto acoplado a uno o varios reservorios. Dicha dinámica se describe mediante la matriz densidad reducida del sistema, mientras que el sistema total está gobernado por el Hamiltoniano
\[
\hat{H}=\hat{H}_{S}+\hat{H}_{B}+\hat{H}_{I}.
\]
En el régimen markoviano, la evolución temporal del sistema reducido se expresa mediante una ecuación maestra de la forma
\begin{equation*}
    \partial_{t}\hat{\rho}_{S}(t)
    = -\,i\,[\hat{H}_{S}+\hat{H}_{LS},\hat{\rho}_{S}(t)]
      + \mathcal{L}(\hat{\rho}_{S}(t)),
\end{equation*}
la cual, en el marco de esta tesis, corresponde a la ecuación \eqref{eqsec2:lindbladS} o a la ecuación \eqref{sec2schrodingerthermo}.
\\

El acoplamiento con múltiples reservorios $\alpha$ se modela mediante un generador de Lindblad descompuesto en contribuciones locales. En el caso global,
\begin{equation*}
    \mathcal{L}=\sum_{\alpha}\mathcal{L}_{\alpha,g},
\end{equation*}
mientras que, en el caso semilocal,
\begin{equation*}
    \mathcal{L}=\sum_{\alpha}\mathcal{L}_{\alpha}.
\end{equation*}
Aquí, cada superoperador $\mathcal{L}_{\alpha,g}$ o $\mathcal{L}_{\alpha}$ representa la interacción efectiva con el reservorio $\alpha$.
\\

Supondremos que cada reservorio induce un equilibrio local descrito por un estado de Gibbs. Para acoplamiento global y semilocal, respectivamente,
\[
\hat{\rho}_{\mathrm{eq},g}^{\alpha}
=\frac{1}{Z_{\beta_{\alpha},\mu_{\alpha};g}}
\,e^{-\beta_{\alpha}(\hat{H}_{S}-\mu_{\alpha}\hat{N}_{S})},
\qquad
\hat{\rho}_{\mathrm{eq}}^{\alpha}
=\frac{1}{Z_{\beta_{\alpha},\mu_{\alpha}}}
\,e^{-\beta_{\alpha}(\hat{H}_{TD}-\mu_{\alpha}\hat{N}_{S})}.
\]
A partir de estos estados y aplicando la desigualdad de Spohn \cite{spohn1978entropy} se obtiene una forma local de la segunda ley. Para el caso global,
\begin{equation}
    -\,\mathrm{Tr}\!\left[(\mathcal{L}_{\alpha,g}\hat{\rho}_{S})
      \big(\ln\hat{\rho}_{S}-\ln\hat{\rho}_{\mathrm{eq},g}^{\alpha}\big)\right]\ge 0,
    \label{spohninfo}
\end{equation}
mientras que, para el caso semilocal,
\begin{equation}
    -\,\mathrm{Tr}\!\left[(\mathcal{L}_{\alpha}\hat{\rho}_{S})
      \big(\ln\hat{\rho}_{S}-\ln\hat{\rho}_{\mathrm{eq}}^{\alpha}\big)\right]\ge 0.
    \label{spohninfolocal}
\end{equation}

Definimos la tasa de cambio de la entropía de von Neumann inducida por el reservorio $\alpha$ como
\[
\dot{S}^{\alpha,g}
= -\,\mathrm{Tr}\!\left\{(\mathcal{L}_{\alpha,g}\hat{\rho}_{S})\ln\hat{\rho}_{S}\right\},
\qquad
\dot{S}^{\alpha}
= -\,\mathrm{Tr}\!\left\{(\mathcal{L}_{\alpha}\hat{\rho}_{S})\ln\hat{\rho}_{S}\right\},
\]
donde la entropía de von Neumann del sistema es
\[
S=-\,\mathrm{Tr}\{\hat{\rho}_{S}\ln\hat{\rho}_{S}\}.
\]
Comparando las ecuaciones \eqref{spohninfo} y \eqref{spohninfolocal} se identifica el flujo de calor desde el reservorio $\alpha$ hacia el sistema como
\[
J_{\alpha,g}
=\mathrm{Tr}\!\left\{(\mathcal{L}_{\alpha,g}\hat{\rho}_{S})
(\hat{H}_{S}-\mu_{\alpha}\hat{N}_{S})\right\},
\qquad
J_{\alpha}
=\mathrm{Tr}\!\left\{(\mathcal{L}_{\alpha}\hat{\rho}_{S})
(\hat{H}_{TD}-\mu_{\alpha}\hat{N}_{S})\right\}.
\]
En consecuencia, las producciones locales de entropía cumplen
\begin{equation}
    \dot{\sigma}_{\alpha,g}=\dot{S}^{\alpha,g}-\beta_{\alpha}J_{\alpha,g}\ge 0,
    \qquad
    \dot{\sigma}_{\alpha}=\dot{S}^{\alpha}-\beta_{\alpha}J_{\alpha}\ge 0,
    \label{sec4:localentropy}
\end{equation}
lo que constituye la versión local de la segunda ley. La suma de estas contribuciones recupera la formulación global:
\[
\sum_{\alpha}\dot{\sigma}_{\alpha,g}
=\dot{\sigma}_{g}\ge 0,
\qquad
\sum_{\alpha}\dot{\sigma}_{\alpha}
=\dot{\sigma}\ge 0,
\]
en correspondencia con la inecuación de Clausius estándar. Obsérvese que, aun en estado estacionario
($\partial_{t}S_{g}=0$ o $\partial_{t}S=0$), los términos
$\dot{S}^{\alpha,g}$ y $\dot{S}^{\alpha}$ no tienen por qué anularse: dependen de los flujos locales que entran por cada disipador.
\\

Introducimos ahora las corrientes de energía y la potencia química asociadas a cada reservorio,
\begin{align*}
    \dot{E}_{\alpha,g}
    &= \mathrm{Tr}\!\left[(\mathcal{L}_{\alpha,g}\hat{\rho}_{S})\hat{H}_{S}\right],
    &\qquad
    \dot{E}_{\alpha}
    &= \mathrm{Tr}\!\left[(\mathcal{L}_{\alpha}\hat{\rho}_{S})\hat{H}_{TD}\right],\\[2pt]
    \dot{W}_{\alpha,g}
    &= \mu_{\alpha}\,\mathrm{Tr}\!\left[(\mathcal{L}_{\alpha,g}\hat{\rho}_{S})\hat{N}_{S}\right],
    &\qquad
    \dot{W}_{\alpha}
    &= \mu_{\alpha}\,\mathrm{Tr}\!\left[(\mathcal{L}_{\alpha}\hat{\rho}_{S})\hat{N}_{S}\right].
\end{align*}
Para ambas ecuaciones maestras (global y semilocal) se cumple que la suma de las corrientes de energía provenientes de los distintos reservorios coincide con la derivada temporal de la energía interna del sistema:
\[
\sum_{\alpha}\dot{E}_{\alpha,g}=\partial_{t}E_{g},
\qquad
\sum_{\alpha}\dot{E}_{\alpha}=\partial_{t}E.
\]
En ausencia de trabajo mecánico (Hamiltoniano independiente del tiempo), la energía intercambiada con cada reservorio se descompone como
\[
\dot{E}_{\alpha,g}=J_{\alpha,g}+\dot{W}_{\alpha,g},
\qquad
\dot{E}_{\alpha}=J_{\alpha}+\dot{W}_{\alpha}.
\]
Multiplicando la ecuación \eqref{sec4:localentropy} por $T_{\alpha}=\beta_{\alpha}^{-1}$ y utilizando la descomposición de energía dada en la expresión anterior, se obtienen las inecuaciones locales de energía libre:

\begin{equation}
    T_{\alpha}\dot{\sigma}_{\alpha,g}
    = \dot{W}_{\alpha,g}-\dot{\mathcal{F}}_{\alpha,g}\ge 0,
    \qquad
    T_{\alpha}\dot{\sigma}_{\alpha}
    = \dot{W}_{\alpha}-\dot{\mathcal{F}}_{\alpha}\ge 0,
    \label{sec4:localfreerate}
\end{equation}
donde las tasas de variación de energía libre asociadas al reservorio $\alpha$ se definen como
\[
\dot{\mathcal{F}}_{\alpha,g}
= \dot{E}_{\alpha,g}-T_{\alpha}\dot{S}^{\alpha,g},
\qquad
\dot{\mathcal{F}}_{\alpha}
= \dot{E}_{\alpha}-T_{\alpha}\dot{S}^{\alpha}.
\]
Al sumar sobre todos los reservorios se obtiene, para los casos global y semilocal,
\begin{equation}
    \sum_{\alpha}T_{\alpha}\dot{\sigma}_{\alpha,g}
    = \dot{W}_{g}-\dot{\mathcal{F}}_{g}\ge 0,
    \qquad
    \sum_{\alpha}T_{\alpha}\dot{\sigma}_{\alpha}
    = \dot{W}-\dot{\mathcal{F}}\ge 0,
    \label{sec4:freeratefinal}
\end{equation}
con $\dot{W}_{g}=\sum_{\alpha}\dot{W}_{\alpha,g}$,
$\dot{W}=\sum_{\alpha}\dot{W}_{\alpha}$ y
\[
\dot{\mathcal{F}}_{g}
=\partial_{t}E_{g}-\sum_{\alpha}T_{\alpha}\dot{S}^{\alpha,g},
\qquad
\dot{\mathcal{F}}
=\partial_{t}E-\sum_{\alpha}T_{\alpha}\dot{S}^{\alpha}.
\]

En el caso isotérmico ($T_{\alpha}=T$ para todo $\alpha$), las tasas de energía libre se reducen a
$\dot{\mathcal{F}}_{g}=\partial_{t}(E_{g}-TS_{g})$ y
$\dot{\mathcal{F}}=\partial_{t}(E-TS)$, por lo que las ecuaciones
\eqref{sec4:localentropy} y \eqref{sec4:freeratefinal} son equivalentes. En estado estacionario,
$\dot{\mathcal{F}}_{g}=\dot{\mathcal{F}}=0$; por lo tanto,
$\dot{W}_{g}\ge 0$ y $\dot{W}\ge 0$. En consecuencia, el sistema no puede entregar trabajo neto si únicamente intercambia calor con reservorios a la misma temperatura. La extracción de trabajo neto (signo opuesto en nuestras convenciones) requiere la presencia de reservorios a distintas temperaturas, caso en el cual $\dot{\mathcal{F}}_{g}\neq 0$ y $\dot{\mathcal{F}}\neq 0$.


\section{Flujos de información en sistemas cuánticos}

En esta sección, el análisis de los flujos de información se desarrollará en el marco de la evolución semilocal, ya que este será el esquema utilizado en el resto de la tesis. Para el caso de evolución global, los flujos de información pueden definirse de manera análoga, empleando las cantidades termodinámicas correspondientes a dicho marco.
\\

Consideremos un sistema $S$ compuesto por dos subsistemas acoplados, cuya dinámica está gobernada por un Hamiltoniano de la forma
\[
    \hat{H}_{S} = \hat{H}_{X} + \hat{H}_{Y} + \hat{H}_{XY}, 
\]
donde $\hat{H}_{i}$ representa el Hamiltoniano del subsistema $i = X,Y$, y $\hat{H}_{XY}$ corresponde al término de interacción entre ambos subsistemas. Se asume, además, que cada subsistema está acoplado a un conjunto distinto de reservorios, denotados por $\alpha_{i}$. Bajo esta hipótesis, la tasa de producción de entropía local de cada subsistema $i$ está dada por
\[
    \dot{\sigma}_{i} \equiv \sum_{\alpha_{i}} \dot{\sigma}_{\alpha_{i}} = \sum_{\alpha_{i}} \dot{S}^{\alpha_{i}} - \sum_{\alpha_{i}} \beta_{\alpha_{i}} J_{\alpha_{i}} \geq 0,
\]
de modo que la tasa de producción de entropía total del sistema se expresa como $\dot{\sigma} = \dot{\sigma}_{X} + \dot{\sigma}_{Y}$.
\\

De forma análoga a la descripción clásica, se busca establecer un vínculo entre las producciones locales de entropía y la información compartida entre subsistemas. Para ello, se emplea la información mutua cuántica, definida como
\[
    I_{XY} = S_{X} + S_{Y} - S_{XY},
\]
donde $S_{XY} = -\mathrm{Tr}\{ \hat{\rho}_{S} \ln \hat{\rho}_{S} \}$ es la entropía de von Neumann del sistema total, y $S_{i} = - \mathrm{Tr}\{ \hat{\rho}_{i} \ln \hat{\rho}_{i} \}$ corresponde a la entropía del subsistema $i = X, Y$, con $\hat{\rho}_{i}$ la matriz densidad reducida asociada.
\\

La derivada temporal de la información mutua puede descomponerse como
\begin{align*}
    \partial_{t}I_{XY} & = \partial_{t}S_{X} + \partial_{t}S_{Y} - \partial_{t}S_{XY} \\
        & = \partial_{t}S_{X} + \partial_{t}S_{Y} - \sum_{i=X,Y;\alpha_{i}}\dot{S}^{\alpha_{i}} \\
        & = \partial_{t}S_{X} - \sum_{\alpha_{X}} \dot{S}^{\alpha_{X}} + \partial_{t}S_{Y} - \sum_{\alpha_{Y}} \dot{S}^{\alpha_{Y}} \\
        & = \dot{I}_{X} + \dot{I}_{Y},
\end{align*}
donde el flujo de información asociado a cada subsistema $i$ se define como
\[
    \dot{I}_{i} = \partial_{t} S_{i} - \sum_{\alpha_{i}} \dot{S}^{\alpha_{i}}.
\]
Al sustituir $\sum_{\alpha_{i}} \dot{S}^{\alpha_{i}} = \partial_{t} S_{i} - \dot{I}_{i}$ en las expresiones para las producciones de entropía locales, se obtiene
\begin{align}
    \dot{\sigma}^{X} = \partial_{t}S_{X} - \sum_{\alpha_{X}} \beta_{\alpha_{X}} J_{\alpha_{X}} - \dot{I}_{X} \geq 0, \\
    \dot{\sigma}^{Y} = \partial_{t}S_{Y} - \sum_{\alpha_{Y}} \beta_{\alpha_{Y}} J_{\alpha_{Y}} - \dot{I}_{Y} \geq 0.
    \label{sec4:localentropyquantum}
\end{align}
Este resultado es análogo al obtenido en la descripción clásica; sin embargo, al estar formulado en términos de entropías de von Neumann permite incorporar de manera natural efectos cuánticos, como las correlaciones y coherencias, que resultan esenciales para la caracterización de los flujos de información en sistemas cuánticos.
\\

Si el subsistema $i$ está acoplado a un reservorio isotérmico de temperatura $T_i$, la tasa de cambio de energía libre se escribe como
\[
    \dot{\mathcal{F}}_{i} = \dot{E}_{i} - T_{i} \sum_{\alpha_{i}}\dot{S}^{\alpha_{i}}.
\]
En el estado estacionario se cumple
\[
    \dot{I}_{i} = - \sum_{\alpha_{i}}\dot{S}^{\alpha_{i}},
\]
por lo que la tasa de variación de energía libre del subsistema $i$ toma la forma
\[
    \dot{\mathcal{F}}_{i} = \dot{E}_{i} + T_{i} \dot{I}_{i}.
\]
Así, la variación de energía libre presenta dos contribuciones: una asociada al intercambio de energía y otra vinculada al flujo de información. En consecuencia, la potencia asociada al subsistema $i$ se encuentra acotada por
\[
    \dot{W}_{i} \geq \dot{E}_{i} + T_{i}\dot{I}_{i}.
\]
Esta cota muestra que la capacidad del subsistema $i$ para realizar trabajo externo depende no sólo del intercambio energético, sino también del flujo de información que reciba.
\\

En régimen estacionario, la energía total permanece constante,
\[
    \partial_{t}E = \dot{E}_{X} + \dot{E}_{Y} = 0,
\]
lo que permite interpretar los términos $\dot{E}_{i}$ como flujos de energía entre subsistemas. En este contexto, la variación de energía libre de cada subsistema se descompone en dos partes: una correspondiente al intercambio energético y otra asociada al flujo de información generado por sus correlaciones mutuas.
\\

En el caso particular $\dot{E}_{X} = \dot{E}_{Y} \approx 0$, la cota para la potencia se reduce a
\[
     \dot{W}_{i} \geq T_{i}\dot{I}_{i}.
\]
Cuando $\dot{I}_{i}<0$, esta relación permite que el sistema realice trabajo a partir de una contribución dominante de la información, incluso en ausencia de gradientes térmicos apreciables. Este mecanismo constituye lo que se denomina una \textit{máquina de información}, en la que los flujos informacionales entre subsistemas desempeñan un papel central, posibilitando la extracción de trabajo en un régimen puramente informacional. 
\\

Estos flujos de información y su influencia en la termodinámica de sistemas cuánticos serán explorados en detalle en el capítulo siguiente, donde se analizará un modelo específico que exhibe el comportamiento característico de un Demonio de Maxwell autónomo.

\label{sec4:flujos0}

\chapter{Dinámica de un sistema de tres puntos cuánticos conectado al ambiente}

En este capítulo se estudia un sistema compuesto por tres puntos cuánticos acoplados a tres reservorios térmicos que, bajo condiciones apropiadas, puede exhibir un comportamiento característico de un Demonio de Maxwell autónomo. La dinámica se describe mediante una ecuación maestra en forma de Lindblad (véase \ref{sec2lindbladconsistency}), y se analizan las condiciones en las que la información intercambiada entre los subsistemas impulsa transporte en contra del gradiente químico. Asimismo, se cuantifican magnitudes termodinámicas relevantes (corrientes de partículas, flujos de calor, potencia) y se discute el papel de las coherencias cuánticas.
\\

La organización es la siguiente. En la sección~\ref{puntoscuanticos} se presenta una breve reseña experimental de los puntos cuánticos. En la sección~\ref{modeloteorico} se introduce el modelo propuesto; en la sección~\ref{sec5:modelo} se detalla el Hamiltoniano del sistema reducido y su entorno, junto con el generador de Lindblad que se utilizará (con base en \ref{sec2lindbladconsistency}). A continuación, la sección~\ref{explicaciondem} ofrece una explicación heurística del Demonio de Maxwell autónomo. La sección~\ref{resultados} reúne los resultados y se divide en dos partes: primero, la sección~\ref{sec5:transporte} analiza el régimen de transporte cuántico con tasas de túnel independientes de la ocupación; luego, la sección~\ref{sec5:demonio1} estudia el caso en que las tasas de tunel dependen de la ocupación de $D$, con esto establece las condiciones bajo las cuales el sistema actúa como un Demonio de Maxwell autónomo. Finalmente, en la sección~\ref{sec5:demonio} se compara este comportamiento con su contraparte clásica, destacando las diferencias inducidas por efectos cuánticos.

\section{Puntos cuánticos}
\label{puntoscuanticos}
Los puntos cuánticos constituyen uno de los sistemas a nanoescala más estudiados en el ámbito de la termodinámica cuántica y el transporte electrónico. Estos dispositivos, fabricados mediante técnicas de litografía o autoensamblaje, consisten en regiones semiconductoras de tamaño reducido, típicamente del orden de $100\,\text{nm}$, donde los electrones quedan confinados en las tres dimensiones espaciales \cite{hanson2007spins}. Debido a este confinamiento, los puntos cuánticos exhiben niveles de energía discretos, análogos a los de los átomos; por ello se les denomina \textit{átomos artificiales} \cite{kouwenhoven2001few}. En un átomo real, los electrones son atraídos hacia un núcleo positivo; en cambio, en un punto cuántico la localización electrónica es consecuencia del potencial de confinamiento impuesto por la heteroestructura semiconductora \cite{ashoori1996electrons}. 
\\

De manera análoga a la energía de ionización de los átomos, los puntos cuánticos poseen una \textit{energía de carga}, que corresponde al costo energético de añadir un electrón adicional al sistema superando la repulsión de Coulomb y el confinamiento de las barreras \cite{hanson2007spins}. Esta magnitud no debe confundirse con la \textit{energía de sitio}, asociada a los niveles discretos originados por el confinamiento cuántico. Mientras la primera refleja el carácter colectivo de la interacción electrostática y depende de la capacitancia efectiva del punto cuántico, la segunda describe la estructura espectral propia del potencial de confinamiento. 
\\

Otro aspecto importante de los puntos cuánticos es que pueden acoplarse a reservorios electrónicos mediante barreras de túnel, lo que permite estudiar el transporte electrónico a través de ellos en distintas condiciones de acoplamiento y temperatura.
\\

En la Figura~\ref{img:dot} se muestra un esquema simplificado de un punto cuántico definido electrostáticamente en una heteroestructura de GaAs/AlGaAs. Aunque el análisis de esta tesis es de carácter teórico, esta representación ilustra de manera concreta cómo se implementa un confinamiento cuántico en sistemas semiconductores.

\insertimage[\label{img:dot}]{ejemplos/qd.pdf}{scale=0.5}{Esquema simplificado de un punto cuántico definido electrostáticamente en una heteroestructura de GaAs/AlGaAs. Los electrones quedan confinados en el pozo cuántico, dando lugar a niveles discretos de energía (adaptado de \cite{ashoori1996electrons}).}

\section{Modelo teórico}
\label{modeloteorico}
Con el fin de extender el modelo estudiado en \cite{horowitz2014thermodynamics} a un contexto cuántico, consideramos un sistema reducido compuesto por tres puntos cuánticos acoplados a tres reservorios térmicos, como se ilustra en la Figura~\ref{img:sistema3puntos}. El sistema incluye un punto central $D$, que actuará como ``demonio'', y dos puntos laterales $L$ y $R$, que constituyen el subsistema controlado. Cada punto se acopla a un baño térmico distinto, y las interacciones entre ellos permiten modelar dinámicas de transporte y correlaciones cuánticas.

\insertimage[\label{img:sistema3puntos}]{ejemplos/semilocal.pdf}{scale=0.36}{Esquema del sistema reducido compuesto por tres puntos cuánticos, modelados como tres niveles de energía. Estos niveles presentan interacción de Coulomb repulsiva entre sí y cada uno está acoplado a un baño térmico distinto.}


La elección de este sistema se justifica por su capacidad para exhibir un comportamiento característico de un Demonio de Maxwell autónomo, análogo al modelo clásico presentado en \cite{horowitz2014thermodynamics}. La interacción entre los tres puntos cuánticos hace posible que el Demonio influya en el flujo de electrones entre $L$ y $R$, posibilitando la extracción de trabajo. 
\\

Una manera de llevar a cabo dicha extracción consiste en aplicar un gradiente de potencial eléctrico entre los puntos $L$ y $R$, es decir, $\mu_{L} - \mu_{R} = eV$, lo que genera una corriente de electrones. Bajo ciertas condiciones, el Demonio puede invertir esta corriente, forzando a los electrones a fluir en contra del gradiente aplicado. Este efecto, que constituye una manifestación de trabajo extraído mediante información, se analiza con mayor detalle en la sección \ref{sec5:dem}. 
\\

En la siguiente sección se estudiará la dinámica del sistema mediante la ecuación maestra de Lindblad semilocal, donde se introducen los parámetros microscópicos que caracterizan al Hamiltoniano.

\section{Ecuación maestra de un sistema de tres puntos cuánticos}

El Hamiltoniano del sistema de tres puntos cuánticos, en segunda cuantización, se expresa como 

\begin{align*}
    \hat{H}_{S} & = \epsilon_{D}\hat{n}_{D} + \epsilon \hat{n}_{L} + \epsilon \hat{n}_{R} + g(\hat{d}^{\dagger}_{L}\hat{d}_{R} + \hat{d}^{\dagger}_{R}\hat{d}_{L} ) \\
          & \quad + U(\hat{n}_{D}\hat{n}_{L} + \hat{n}_{D}\hat{n}_{R} )  + U_{f}\hat{n}_{R}\hat{n}_{L},
\end{align*}
Aquí, $\hat{d}_{i}$ ($\hat{d}^{\dagger}_{i}$) denota el operador de aniquilación (creación) en el sitio $i\in\{L,R,D\}$, y $\hat{n}_{i}=\hat{d}^\dagger_i\hat{d}_i$ representa el operador de número. Los operadores $\hat{d}_{i}$ satisfacen las relaciones de anticonmutación

\begin{equation*}
    \{\hat{d}_{\alpha},\hat{d}_{\beta} \} = 0, \hspace{10mm} \{\hat{d}^{\dagger}_{\alpha}, \hat{d}_{\beta} \} = \delta_{\alpha \beta}.
\end{equation*}

El término proporcional a $\epsilon_i$ corresponde a la energía de sitio asociada al confinamiento cuántico en cada punto; $g$ describe el acoplamiento de \textit{hopping} coherente entre $L$ y $R$, análogo al acoplamiento entre orbitales atómicos \cite{simon2013oxford}, responsable de generar coherencias en la base de Fock; $U$ modela la interacción de Coulomb repulsiva entre el demonio $D$ y los sitios $L$ y $R$; mientras que $U_f$ representa la interacción de Coulomb repulsiva entre $L$ y $R$.
\\

En la base de Fock $|n_{L},n_{R},n_{D} \rangle = (\hat{d}^{\dagger}_{L})^{n_{L}}(\hat{d}^{\dagger}_{R})^{n_{R}}(\hat{d}^{\dagger}_{D})^{n_{D}}|0,0,0\rangle$, ordenada como

\begin{equation*}
     \{|0,0,0\rangle, |1,0,0\rangle, |0,1,0\rangle, |0,0,1\rangle, |1,1,0\rangle, |1,0,1\rangle, |0,1,1\rangle, |1,1,1\rangle\},
\end{equation*}
el Hamiltoniano adopta la forma matricial 

\begin{equation*}
    \hat{H}_{S} = 
    \begin{bmatrix}
        0 & 0 & 0 & 0 & 0 & 0 & 0 & 0 \\
        0 & \epsilon_{L} & g & 0 & 0 & 0 & 0 & 0 \\
        0 & g & \epsilon_{R} & 0 & 0 & 0 & 0 & 0 \\
        0 & 0 & 0 & \epsilon_{D} & 0 & 0 & 0 & 0 \\
        0 & 0 & 0 & 0 & \epsilon_{L} + \epsilon_{R}  + U_{f} & 0 & 0 & 0 \\
        0 & 0 & 0 & 0 & 0 & \epsilon_{L} + \epsilon_{D} + U & g & 0 \\
        0 & 0 & 0 & 0 & 0 & g & \epsilon_{R} + \epsilon_{D} + U & 0 \\
        0 & 0 & 0 & 0 & 0 & 0 & 0 & \epsilon_{L} + \epsilon_{R}  + \epsilon_{D} + 2U + U_{f} 
        \end{bmatrix}.
\end{equation*}
Esta representación muestra que el término de \textit{hopping} $g$ es el único que introduce coherencias en la base de Fock, al acoplar los estados $|1,0,0\rangle \leftrightarrow |0,1,0\rangle$ y $|1,0,1\rangle \leftrightarrow |0,1,1\rangle$. El resto de los términos son puramente diagonales y describen las energías de cada configuración de ocupación.
\\

El Hamiltoniano total incluye, además, los términos de los reservorios y la interacción sistema–baño: $\hat{H}_{B} = \hat{H}_{L}+\hat{H}_{R}+\hat{H}_{D}$ y $\hat{V} = \hat{V}_{L}+\hat{V}_{R}+\hat{V}_{D}$, con 

\begin{equation*}
    \hat{H}_{\alpha} = \sum_{l} \epsilon_{\alpha, l} \hat{c}^{\dagger}_{\alpha, l}\hat{c}_{\alpha, l}, \hspace{10mm} 
    \hat{V}_{\alpha} = \sum_{l} t_{\alpha, l} \left( \hat{d}^{\dagger}_{\alpha} \hat{c}_{\alpha, l} + \hat{c}^{\dagger}_{\alpha, l} \hat{d}_{\alpha} \right),
\end{equation*}
donde $\alpha = L,R,D$ denota los tres reservorios, $\hat{c}_{\alpha,l}$ son los operadores de aniquilación de los modos del reservorio $\alpha$, $\epsilon_{\alpha, l}$ sus energías, y $t_{\alpha, l}$ los parámetros reales de acoplamiento. 
\\

De $\hat{V}$ se identifican los operadores de sistema y baño como 

\begin{equation*}
    \hat{S}_{\alpha,1} = \hat{d}^{\dagger}_{\alpha}, \hspace{10mm} \hat{S}_{\alpha,-1} = \hat{d}_{\alpha},
\end{equation*}

\begin{equation*}
    \hat{B}_{\alpha,1} = \sum_{l}t_{\alpha,l} \hat{c}_{\alpha,l}, \hspace{10mm} \hat{B}_{\alpha,-1} = \sum_{l}t_{\alpha,l} \hat{c}^{\dagger}_{\alpha,l}.
\end{equation*}

La evolución no unitaria del sistema de tres puntos cuánticos, inducida por el acoplamiento a los reservorios, se describe mediante la ecuación maestra de Lindblad:  

\begin{equation}
    \mathcal{L} = \mathcal{L}_{R} + \mathcal{L}_{L} + \mathcal{L}_{D},
    \label{Lindbladsec5}
\end{equation}
donde cada disipador $\mathcal{L}_\alpha$ depende de los operadores locales de su sitio acoplado y adopta la forma explícita siguiente:

\begin{align*}
\mathcal{L}_R &= \kappa_R(\epsilon)\big( f_R(\epsilon)\mathcal{D}[\hat{d}^\dagger_R(1-\hat{n}_D)(1-\hat{n}_L)] + [1-f_R(\epsilon)]\mathcal{D}[\hat{d}_R(1-\hat{n}_D)(1-\hat{n}_L)] \big) \\
&+ \kappa_R(\epsilon+U)\big( f_R(\epsilon+U)\mathcal{D}[\hat{d}^\dagger_R\hat{n}_D(1-\hat{n}_L)] + [1-f_R(\epsilon+U)]\mathcal{D}[\hat{d}_R\hat{n}_D(1-\hat{n}_L)] \big) \\
&+ \kappa_R(\epsilon+U_f)\big( f_R(\epsilon+U_f)\mathcal{D}[\hat{d}^\dagger_R(1-\hat{n}_D)\hat{n}_L] + [1-f_R(\epsilon+U_f)]\mathcal{D}[\hat{d}_R(1-\hat{n}_D)\hat{n}_L] \big) \\
&+ \kappa_R(\epsilon+U+U_f)\big( f_R(\epsilon+U+U_f)\mathcal{D}[\hat{d}^\dagger_R\hat{n}_D\hat{n}_L] + [1-f_R(\epsilon+U+U_f)]\mathcal{D}[\hat{d}_R\hat{n}_D\hat{n}_L] \big),
\end{align*}

\begin{align*}
\mathcal{L}_L &= \kappa_L(\epsilon)\big( f_L(\epsilon)\mathcal{D}[\hat{d}^\dagger_L(1-\hat{n}_D)(1-\hat{n}_R)] + [1-f_L(\epsilon)]\mathcal{D}[\hat{d}_L(1-\hat{n}_D)(1-\hat{n}_R)] \big) \\
&+ \kappa_L(\epsilon+U)\big( f_L(\epsilon+U)\mathcal{D}[\hat{d}^\dagger_L\hat{n}_D(1-\hat{n}_R)] + [1-f_L(\epsilon+U)]\mathcal{D}[\hat{d}_L\hat{n}_D(1-\hat{n}_R)] \big) \\
&+ \kappa_L(\epsilon+U_f)\big( f_L(\epsilon+U_f)\mathcal{D}[\hat{d}^\dagger_L(1-\hat{n}_D)\hat{n}_R] + [1-f_L(\epsilon+U_f)]\mathcal{D}[\hat{d}_L(1-\hat{n}_D)\hat{n}_R] \big) \\
&+ \kappa_L(\epsilon+U+U_f)\big( f_L(\epsilon+U+U_f)\mathcal{D}[\hat{d}^\dagger_L\hat{n}_D\hat{n}_R] + [1-f_L(\epsilon+U+U_f)]\mathcal{D}[\hat{d}_L\hat{n}_D\hat{n}_R] \big),
\end{align*}

\begin{align*}
\mathcal{L}_D &= \kappa_D(\epsilon_D)\big( f_D(\epsilon_D)\mathcal{D}[\hat{d}^\dagger_D(1-\hat{n}_L)(1-\hat{n}_R)] + [1-f_D(\epsilon_D)]\mathcal{D}[\hat{d}_D(1-\hat{n}_L)(1-\hat{n}_R)] \big) \\ 
&+ \kappa_D(\epsilon_D+U) f_D(\epsilon_D+U)\mathcal{D}[\hat{d}^\dagger_D(\hat{n}_L(1-\hat{n}_R) + \hat{n}_R(1-\hat{n}_L) )] \\
&+ \kappa_D(\epsilon_D+U) [1-f_D(\epsilon_D+U)]\mathcal{D}[\hat{d}_D(\hat{n}_L(1-\hat{n}_R) + \hat{n}_R(1-\hat{n}_L))]  \\
&+ \kappa_D(\epsilon_D+2U)\big( f_D(\epsilon_D+2U)\mathcal{D}[\hat{d}^\dagger_D\hat{n}_R\hat{n}_L] + [1-f_D(\epsilon_D+2U)]\mathcal{D}[\hat{d}_D\hat{n}_R\hat{n}_L] \big).
\end{align*}
Aquí, $f_\alpha(\omega) = \left(e^{\beta_\alpha(\omega-\mu_\alpha)}+1\right)^{-1}$ corresponde a la función de Fermi, mientras que $\kappa_\alpha(\omega) = 2\pi\sum_l t^2_{\alpha,l}\delta(\omega-\epsilon_{\alpha,l})$ define las tasas de túnel. Los términos $\kappa_\alpha(\omega)[1-f_\alpha(\omega)]$ y $\kappa_\alpha(\omega)f_{\alpha}(\omega)$ describen, respectivamente, la transición de un electrón desde el sitio hacia el baño térmico y el proceso inverso, del baño hacia el sitio. Además, es importante notar que los disipadores dependen no sólo de los operadores $\hat{d}_{i}$, sino también de los operadores de número asociados a los otros sitios, lo que refleja la naturaleza semilocal del enfoque empleado. La ecuación \eqref{Lindbladsec5} se obtiene a partir de los operadores de salto calculados en \ref{apendix5jumpop} y \ref{apendix5frecuencygroup}. 
\\

Despreciando el término de \textit{Lamb Shift} $\hat{H}_{LS}$, como se realiza en~\cite{prech2023entanglement}, la evolución del sistema está gobernada por  

\begin{equation}
    \frac{d}{dt}\hat{\rho}_{S}(t) = -i[\hat{H}_{S},\hat{\rho}_{S}(t)] + \mathcal{L}(\hat{\rho}_{S}(t)),
\end{equation}
donde $\hat{\rho}_{S}(t)$ es la matriz densidad del sistema reducido. Esta ecuación permite calcular numéricamente la dinámica y extraer las magnitudes termodinámicas de interés.  
\\

Un aspecto relevante es que, en la descripción semilocal, los disipadores se construyen a partir de operadores locales, lo cual facilita asignar cantidades termodinámicas a cada subsistema, en particular los flujos de información. En contraste, en la ecuación maestra global los disipadores dependen de operadores colectivos, lo que dificulta la interpretación local. Es crucial determinar bajo qué condiciones un Hamiltoniano $\hat{H}_S$ conserva la localidad de los operadores tras la agrupación de frecuencias. Un ejemplo en que la localidad no se cumple ocurre cuando todas las interacciones de Coulomb entre los sitios toman valores distintos.

\label{sec5:modelo}


\subsection{Explicación del Demonio de Maxwell autónomo}
\label{explicaciondem}
Para que el sistema de tres puntos cuánticos se comporte como un Demonio de Maxwell autónomo, es necesario tratarlo como un sistema bipartito compuesto por dos subsistemas: el subsistema $LR$, que consta de dos niveles de energía $\epsilon$, y el subsistema $D$, cuyo nivel de energía $\epsilon_D$ desempeña el rol de demonio.
\\

El demonio debe ser capaz de detectar el estado de ocupación del subsistema $LR$ y, en respuesta, modificar su dinámica, actuando de manera análoga a una medición. Gracias a la interacción de Coulomb $U$, el demonio distingue entre las configuraciones en las que $LR$ está vacío, ocupado o doblemente ocupado. A su vez, la dinámica del subsistema $LR$ depende del estado de ocupación del demonio, estableciéndose así una retroalimentación.
\\

Esta retroalimentación se implementa matemáticamente imponiendo que las tasas de túnel cumplan la condición $\kappa_{i}(\epsilon + U) \neq \kappa_{i}(\epsilon)$, con $i = L, R$. En adelante, se utilizará la notación $\kappa^{U}_{i} \equiv \kappa_{i}(\epsilon+U)$ y $\kappa_{i} \equiv \kappa_{i}(\epsilon)=\kappa_{i}(\epsilon+U_{f})=\kappa_{i}(\epsilon+U+U_{f})$. Por el contrario, las tasas de túnel del demonio no dependen de la ocupación del sistema, de modo que se escriben simplemente como $\kappa_{D}\equiv \kappa_{D}(\epsilon_D)$.
\\

Para que el demonio detecte de manera efectiva el estado del subsistema $LR$, deben cumplirse dos condiciones físicas. Primero, la energía de interacción debe superar ampliamente las fluctuaciones térmicas de su reservorio asociado, es decir, $\beta_{D} U \gg 1$. Segundo, la dinámica disipativa del demonio debe ser más rápida que la del subsistema $LR$, lo que requiere que $\kappa_{D} > \max\{\kappa_{i}, \kappa^{U}_{i}\}$ con $i = L, R$.
\\

La dinámica del demonio puede descomponerse en tres etapas, ilustradas en las Figuras~\ref{img:dinamica1}–\ref{img:dinamica3}.  

\insertimage[\label{img:dinamica1}]{ejemplos/diseño1}{scale=0.6}{Primera etapa: los sitios $D$ y $R$ presentan una mayor probabilidad de estar ocupados. La ocupación se indica mediante círculos negros.}

En la primera etapa, la elección $\epsilon_D = \mu_D - U/2$ favorece la ocupación del nivel de energía del demonio. Para inducir una corriente en dirección opuesta al gradiente de potencial, es conveniente imponer que $\kappa^{U}_{R} > \kappa^{U}_{L}$, de modo que las transiciones inducidas por el reservorio derecho sean más frecuentes, aumentando así la probabilidad de ocupación del demonio.  

\insertimage[\label{img:dinamica2}]{ejemplos/diseño2}{scale=0.6}{Segunda etapa: el \textit{hopping} $g$ genera una superposición coherente entre los sitios $L$ y $R$, indicada mediante círculos grises.}

En la segunda etapa, el acoplamiento $g$ permite el intercambio coherente de electrones entre los niveles $L$ y $R$. Este proceso introduce coherencias cuánticas que son importantes para la dinámica del sistema.  

\insertimage[\label{img:dinamica3}]{ejemplos/diseño3}{scale=0.6}{Tercera etapa: el electrón se transporta preferentemente hacia el reservorio $L$.}

Finalmente, en la tercera etapa, con el objetivo de generar un flujo de corriente en contra del gradiente de potencial, resulta conveniente imponer que $\kappa_{L} > \kappa_{R}$. Esta condición garantiza que, cuando el demonio está desocupado, la probabilidad de transición hacia el reservorio $L$ sea mayor que hacia el reservorio $R$. Como consecuencia, se favorece el transporte de electrones desde el reservorio $R$ (de menor potencial) hacia el reservorio $L$ (de mayor potencial), lo que constituye un comportamiento característico de un Demonio de Maxwell autónomo.


\label{sec5:dem}

\newpage

\section{Resultados}
\label{resultados}
En esta sección se presentan los resultados numéricos obtenidos a partir de la resolución de la ecuación maestra de Lindblad semilocal para el sistema de tres puntos cuánticos. Se analizan dos escenarios: en primer lugar, un caso típico de transporte cuántico en el que las tasas de túnel son independientes de la ocupación del sitio $D$; en segundo lugar, un caso en el que dichas tasas dependen de la ocupación, bajo condiciones en las que el sistema puede exhibir un comportamiento característico de un Demonio de Maxwell autónomo.  

\subsection{Estudio del transporte sin efectos de demonio}

Se aborda inicialmente el escenario en que las tasas de túnel no dependen de la ocupación y permanecen constantes, es decir, $\kappa_{i}(\omega) = \kappa_{i}$. En este caso se estudia el flujo de corriente
\[
\dot{N}_{i} = \mathrm{Tr}\{(\mathcal{L}_{i}\hat{\rho}_{S})\hat{N}_{S}\},
\]
generado por una diferencia de potencial $\mu_{L} - \mu_{R} = eV$, con $\mu_{L}=-\mu_{R}$. Para los cálculos se fijaron las tasas de túnel en $\kappa_{L} = \kappa_{R} = 1/100$ y $\kappa_{D} = 1/50$. Los resultados se muestran en la Figura~\ref{img:transporte1}.  

\insertimage[\label{img:transporte1}]{ejemplos/particletp.pdf}{scale=0.40}{Corriente de partículas en función de la razón \( eV/T \). Las temperaturas de los reservorios son \( T_{L} = T_{R} = T = 100 \) y \( T_{D} = 2 \). El parámetro de acoplamiento se fija en \( g = \frac{5}{1000} \). Las energías de sitio corresponden a \( \epsilon = 0 \) y \( \epsilon_{D} = \mu_{D} - \frac{U}{2} \), con $\mu_{D}=2$. Las interacciones de Coulomb se fijan en \( U_{f} = 500 \) y \( U = 40 \).}

La figura muestra que, a medida que aumenta la diferencia de potencial $eV$, la corriente de partículas $\dot{N}_L = -\dot{N}_R$ que fluye del baño $L$ al baño $R$ también crece. Este comportamiento es consistente con lo esperado, ya que un mayor potencial químico en el reservorio $L$ impulsa el transporte hacia $R$.  
\\

La potencia suministrada al sistema, dada por la ecuación~\eqref{sec2:heatworkeq}, se presenta en la Figura~\ref{img:transporte2}.  

\insertimage[\label{img:transporte2}]{ejemplos/worktp.pdf}{scale=0.40}{Potencia entregada por los reservorios en función de $eV/T$.}
    
En la Figura~\ref{img:transporte2} se observa que los reservorios $L$ y $R$ entregan trabajo al sistema para mantener la corriente de partículas. Este resultado concuerda con la interpretación física, ya que el transporte de electrones desde un potencial químico mayor hacia uno menor implica una transferencia de energía al sistema en forma de trabajo. Por su parte, dado que el reservorio $D$ no intercambia partículas en promedio, la potencia asociada a este es nula.  
\\

Los flujos de calor definidos en la ecuación~\eqref{sec2:heatworkeq} se muestran en la Figura~\ref{img:transporte3}.  

\insertimage[\label{img:transporte3}]{ejemplos/heatlrdtp.pdf}{scale=0.40}{Flujo de calor recibido por los sistemas $LR$ y $D$ en función de $eV/T$.}
    
Se observa que, para $eV/T \lessapprox 2$, el sistema $LR$ recibe calor de los reservorios $L$ y $R$, mientras que para $eV/T \gtrapprox 2$ este sistema entrega calor a dichos reservorios.  
\\

Para completar el análisis termodinámico, se utilizan las relaciones de la primera ley $\dot{E}_{\alpha} = J_{\alpha} + \dot{W}_{\alpha}$ (ec.~\eqref{sec2:energyeq}) y se grafican los flujos de energía en la Figura~\ref{img:transporte4}.  

\insertimage[\label{img:transporte4}]{ejemplos/energytp.pdf}{scale=0.43}{Flujo de energía entregado por los reservorios en función de $eV/T$.}

En el estado estacionario, la condición $\partial_t E = 0$ implica que $\dot{E}_{LR} = -\dot{E}_D$. Esto se corrobora en la Figura~\ref{img:transporte4}, donde el flujo de energía $\dot{E}_{LR}$ resulta positivo y $\dot{E}_D$ negativo. Como $\dot{W}_{D} = 0$, se deduce que el flujo de calor hacia el demonio está dado por $J_D = \dot{E}_D < 0$, es decir, el reservorio $D$ absorbe calor. En consecuencia, el sistema $LR$ recibe energía de los reservorios $L$ y $R$, la transfiere al subsistema $D$, y esta es finalmente disipada en su respectivo baño térmico. Este comportamiento concuerda con la primera ley de la termodinámica, ya que la energía suministrada por los reservorios de mayor temperatura ($T_L = T_R = T > T_D$) se conserva y fluye hacia el reservorio de menor temperatura.  
\\

Con el fin de analizar los efectos cuánticos en el estado estacionario, se calculan dos cantidades relevantes. La primera es la norma-$l_{1}$, $\mathcal{C}_{l_{1}}$, que cuantifica las coherencias en el sistema $LR$:  

\[
\mathcal{C}_{l_{1}} = \sum_{i> j} |\hat{\rho}_{S;i,j}|.
\]
La segunda cantidad corresponde al entrelazamiento entre modos específicos. En particular, se estudia el entrelazamiento entre los pares de estados $\{|1,0,0\rangle, |0,1,0\rangle\}$ y $\{|1,0,1\rangle, |0,1,1\rangle\}$. Para cuantificar estas correlaciones no clásicas se utiliza la concurrencia $\mathcal{C}_{\text{on}}$ \cite{hill1997entanglement, wootters1998entanglement}, definida como  

\[
\mathcal{C}_{\text{on}} = \max \left\{ 2|\alpha + \beta| - 2\sqrt{p_0 p_D},\, 0 \right\},
\]
donde $p_0$ y $p_D$ representan las probabilidades de que el sistema $LR$ se encuentre vacío o doblemente ocupado, respectivamente. El término $\alpha$ corresponde a la coherencia entre los estados $|1,0,0\rangle$ y $|0,1,0\rangle$, mientras que $\beta$ representa la coherencia entre los estados $|1,0,1\rangle$ y $|0,1,1\rangle$. La deducción de esta expresión se presenta en el Apéndice~\ref{appendix5final}.  
\\

En la Figura~\ref{img:transporte5} se ilustra la variación de la coherencia y la concurrencia en función de la diferencia de potencial.  

\insertimage[\label{img:transporte5}]{ejemplos/quantumtp.pdf}{scale=0.4}{Coherencia (línea azul) y concurrencia (línea roja) en función de $eV/T$.}
    
Los resultados muestran que la coherencia comienza en cero y aumenta de forma progresiva con la diferencia de potencial. En cuanto al entrelazamiento, este también es nulo al inicio y permanece así hasta superar un umbral, a partir del cual la concurrencia crece con $eV/T$. Esto indica la aparición de entrelazamiento en el sistema $LR$. Este resultado es relevante, ya que sugiere que la presencia de una corriente favorece el desarrollo de coherencia y entrelazamiento entre los sitios $L$ y $R$.  

\label{sec5:transporte}



\subsection{Efectos del Demonio de Maxwell autónomo en sistema de 3 puntos cuánticos}
En esta subsección se eligen distintos valores para las tasas de túnel, dependiendo de la ocupación del sitio $D$. El objetivo es inducir un comportamiento tipo Demonio de Maxwell. Para ello, se estudiará el comportamiento de la matriz densidad, así como los flujos de corriente de partículas, calor, información y potencia, en función de la diferencia de potencial $\mu_{L} - \mu_{R} = eV$.
\\

Se comenzará analizando las probabilidades de ocupación, extraídas de la matriz densidad en el estado estacionario. Como se muestra en la Figura~\ref{img:densitymatrix}, las tres probabilidades más altas corresponden a los estados $\rho_{001}$, $\rho_{100}$ y $\rho_{010}$, lo cual es coherente con la presencia de interacción de Coulomb, que penaliza las configuraciones con ocupación múltiple.
\\

Para valores de $eV/T \lessapprox 3.6$, la probabilidad más alta corresponde al estado $\rho_{001}$, lo que indica que el sitio $D$ es, en promedio, el más ocupado. Sin embargo, a medida que la diferencia de potencial aumenta ($eV/T \gtrapprox 3.6$), las probabilidades asociadas a los estados $\rho_{100}$ y $\rho_{010}$ —correspondientes a los sitios $L$ y $R$— superan a $\rho_{001}$, lo que refleja un cambio en la ocupación predominante del sistema.
\\

Debido a que la interacción de Coulomb penaliza los estados con más de un electrón, cabría esperar que la probabilidad de vacío $\rho_{000}$ fuese del mismo orden de magnitud que $\rho_{100},\rho_{010},\rho_{001}$. Sin embargo, debido a la constante inyección de partículas inducida por la diferencia de potencial, esta probabilidad resulta ser significativamente menor. 
\\

Se observa además un cruce entre las curvas de $\rho_{101}$ y $\rho_{011}$ en $eV/T \approx 2.4$. Para valores $eV/T \lessapprox 2.4$, la configuración con interacción Coulomb entre los puntos $R$ y $D$ (representada por $\rho_{011}$) es más probable que $\rho_{101}$, esto favorece a que el sistema se encuentre en la configuración de la Figura \ref{img:dinamica1}, y por lo tanto es esperable que favorezca el comportamiento tipo Demonio. En cambio, para $eV/T \gtrapprox 2.4$, la ocupación $\rho_{011}$ se hace menos probable, indicando una menor interacción entre los sitios $R$ y $D$, por lo tanto se suprime el comportamiento tipo Demonio.


\insertimage[\label{img:densitymatrix}]{ejemplos/matrixdemon.pdf}{scale=0.45}{Elementos diagonales de la matriz densidad en función de $eV/T$. Los resultados se obtuvieron considerando temperaturas $T = 100$ y $T_D = 2$, niveles de energía $\epsilon = 0$ y $\epsilon_D = \mu_D - U/2$ con potencial químico $\mu_D = 2$, tasas de túnel $\kappa_L = \kappa^{U}_{R} = 1/100$, $\kappa_R = \kappa^{U}_{L} = 1/600$, $\kappa_D =  1/50$, acoplamiento entre los sitios $g = 5/1000$ e interacciones de Coulomb $U = 40$ y $U_f = 500$.}

El comportamiento de un Demonio de Maxwell autónomo se puede corrobar analizando las cantidades termodinámicas del sistema. En la Figura~\ref{img:fig1resultado} se muestran: (a) los flujos de calor correspondientes a cada reservorio, y (b) las corrientes de partículas asociadas. Cabe destacar que, dado que no existe un término de hopping entre el sistema y el demonio, la corriente de partículas en el demonio es nula, $\dot{N}_{D} = 0$. Además, por conservación de partículas, se cumple que $\dot{N}_{L} = -\dot{N}_{R}$, lo cual implica que la influencia del demonio sobre el sistema ocurre exclusivamente a través de flujos de energía o información.
\\

Existe un régimen \( eV/T \lessapprox 2.4 \) donde el sistema \( LR \) recibe calor de los reservorios \( L \) y \( R \), es decir, \( J_L, J_R > 0 \). En este mismo régimen, el demonio disipa calor en su respectivo reservorio, \( J_D < 0 \). 
\\

Dado que $\mu_L - \mu_R = eV$, se esperaría que $\dot{N}_L > 0$ para todo valor de $eV$. No obstante, en el régimen $eV/T \lessapprox 2.4$, se observa que $\dot{N}_L < 0$, lo que indica un flujo de partículas en contra del gradiente de potencial. Este cambio de signo en la corriente coincide con el cruce entre las probabilidades $\rho_{101}$ y $\rho_{011}$, lo que es consistente con el comportamiento de Demonio de Maxwell. En particular, como se ilustró en la Figura~\ref{img:dinamica1}, dicho régimen requiere una interacción simultánea entre los puntos $R$ y $D$.
\\

Si se analiza únicamente el subsistema $LR$, la existencia de una corriente contra el gradiente químico representa una violación aparente de la segunda ley de la termodinámica. Para este análisis, se considera la tasa de producción de entropía efectiva del subsistema en el estado estacionario, dada por $\dot{\sigma}^{o}_{LR} = - \sum_{i=L,R} J_i/T$, cuya evolución en función de $eV/T$ se muestra en la Figura~\ref{img:fig2resultado}.


\insertimage[\label{img:fig1resultado}]{ejemplos/heatparticledemon.pdf}{scale=0.54}{(a) Flujos de calor liberados por cada reservorio $\alpha$. (b) Flujos de corriente de partículas correspondientes a cada reservorio.}

\insertimage[\label{img:fig2resultado}]{ejemplos/entropyapparentdemon.pdf}{scale=0.45}{Tasa de producción de entropía aparente del sistema $LR$($\dot{\sigma}^{o}_{LR}$) en función de $eV/T$.}

Donde se observa que para $eV/T \lessapprox 3$, la tasa de producción de entropía aparente $\dot{\sigma}^{o}_{LR}$ es menor a cero, lo que implica una violación aparente de la segunda ley de la termodinámica. Sin embargo, al incluir la contribución del flujo de información en el cálculo de la tasa de producción de entropía, dada por la ecuación~\eqref{sec4:localentropyquantum}, se obtiene la Figura~\ref{img:fig3resultado}, en la que se verifica que la producción total de entropía en el sistema $LR$ permanece siempre positiva.

\insertimage[\label{img:fig3resultado}]{ejemplos/entropyrialdemon.pdf}{scale=0.47}{Tasa de producción de entropía del sistema $LR$($\dot{\sigma}_{LR}$) en función de $eV/T$.}

En el régimen de parámetros en que \( eV/T \lessapprox 2.4 \), el subsistema \( LR \) realiza trabajo. Para determinar si este trabajo es impulsado principalmente por el flujo de información, es necesario analizar el intercambio energético entre el sistema y el demonio. En el estado estacionario, los flujos de energía cumplen $\dot{E}_{LR}= -\dot{E}_{D}$. Para que el sistema pueda interpretarse como un Demonio de Maxwell, el trabajo realizado por el subsistema \( LR \) debe estar dominado por la contribución informacional. Esta condición se puede expresar como:

\[
\dot{\mathcal{F}}_{LR} = \dot{E}_{LR} + T\dot{I}_{LR} = -\dot{E}_{D} + T\dot{I}_{LR} \approx T\dot{I}_{LR}, \hspace{10mm} \dot{W}_{LR} \geq \dot{\mathcal{F}}_{LR},
\]
donde \( \dot{\mathcal{F}}_{LR} \) representa la tasa de cambio de energía libre del subsistema \( LR \), incluyendo una contribución informacional \( T\dot{I}_{LR} \).
\\

A fin de verificar este comportamiento, se grafican las cantidades termodinámicas relevantes tanto del sistema como del demonio.


\insertimage[\label{img:fig4resultado}]{ejemplos/thermoquantitiesdemon}{scale=0.60}{ Flujo de energía, tasa de cambio de energía libre, flujo de información, potencia y flujo de calor del sistema $LR$. (b)  Flujo de energía, tasa de cambio de energía libre, flujo de información y potencia del Demonio. }

La Figura \ref{img:fig4resultado} muestra que, en el régimen en el que el subsistema $LR$ realiza trabajo ($\dot{W}_{LR} < 0$ y $\dot{\mathcal{F}}_{LR} < 0$), la contribución informacional domina sobre el flujo energético, es decir, $\dot{\mathcal{F}}_{LR} \approx T \dot{I}_{LR}$ mientras que $\dot{E}_{LR} \approx 0$. Este resultado respalda la interpretación del dispositivo como un Demonio de Maxwell autónomo.
\\

La potencia negativa $\dot{W}_{LR} < 0$ indica que el sistema está realizando trabajo en contra del gradiente de potencial, lo cual requiere absorber calor del entorno. En efecto, se verifica que $J_{LR} > 0$, señalando que el subsistema $LR$ extrae calor de sus respectivos reservorios.
\\

Dado que no hay transporte neto de partículas a través del Demonio ($\dot{N}_{D} = 0$), el flujo de energía asociado se reduce a $J_{D} = \dot{E}_{D}$. Como se observa en la figura, $J_{D} < 0$, lo que implica que el Demonio disipa calor hacia su baño, caracterizado por una temperatura menor a los reservorios $L$ y $R$ ($T_D < T$). 
\\

Si se adopta una perspectiva global del sistema, es decir, considerando la termodinámica de los tres puntos cuánticos como un conjunto, este comportamiento puede interpretarse como el de una máquina térmica. En este escenario, el sistema total extrae calor del reservorio caliente (\( J_{LR} \)), convierte parte de esa energía en trabajo útil (\( \dot{W}_{LR} \)) y disipa el resto hacia el reservorio frío (\( J_D \)), cumpliendo así la relación \( J_{LR} + J_D = -\dot{W}_{LR} \).
\\

Al estar cada punto cuántico acoplado a un reservorio distinto, es posible descomponer el flujo de información total como $\dot{I}_{LR} = \dot{I}_{L} + \dot{I}_{R}$, lo que permite analizar la contribución informacional de cada punto cuántico por separado. Esta descomposición se ilustra en la Figura \ref{img:figinforesultado}.


\insertimage[\label{img:figinforesultado}]{ejemplos/infolrddemon.pdf}{scale=0.45}{Flujos de información de cada sitio, $L$,$R$ y $D$. }

Es interesante notar que, cuando \( eV/T \gtrapprox 3.6 \), el menor flujo de información corresponde al punto cuántico \( R \), lo que implica que este entrega más información al sistema. En cambio, para \( 1 \lessapprox eV/T \lessapprox 3.6 \), el flujo de información menor proviene del sitio \( L \), de modo que este es el que aporta más información en ese rango. Esta transición se correlaciona con lo observado en la Figura~\ref{img:densitymatrix}: para \( eV/T \gtrapprox 3.6 \) se cumple que \(\rho_{001} < \rho_{010} < \rho_{100}\), lo que indica una mayor probabilidad de ocupación del sitio \( L \). Además, en este mismo rango de parámetros  ($eV/T \gtrapprox 3.6$) la probabilidad del estado \( \rho_{110} \) supera a la de \( \rho_{011} \), lo que implica que la ocupación simultánea de \( L \) y \( R \) ---y por ende su monitoreo mutuo--- es más probable que la de \( R \) y \( D \). Lo anterior puede interpretarse como que, en este régimen, las correlaciones entre \( L \) y \( R \) se intensifican. Pues, en el estado \( \rho_{110} \) los sitios \( L \) y \( R \) interactúan mediante energía de Coulomb, siendo está una interacción mayor al \textit{hopping} ($U_{f}>g$), lo que facilita que \( L \) obtenga información de \( R \) y, en consecuencia, contribuya positivamente a su flujo de información, de modo que \( \dot{I}_{L} > \dot{I}_{R} \). Para diferencias de potencial pequeñas (\( eV/T \lessapprox 1 \)), ambos flujos de información son prácticamente idénticos (\( \dot{I}_{L} \approx \dot{I}_{R} \)).
\\

Un aspecto relevante de esta dinámica es la posible presencia de efectos cuánticos. Para examinar este fenómeno, se representan dos cantidades: la norma-$l_{1}$ de coherencia \( \mathcal{C}_{l_{1}} \), que permite cuantificar la coherencia cuántica en el subsistema \( LR \), y la concurrencia, que sirve como medida del entrelazamiento entre estados del subsistema $LR$ $(|1,0>,|0,1>)$.

\insertimage[\label{img:fig5resultado}]{ejemplos/quantumdemon.pdf}{scale=0.40}{En línea azul la norma-$l_{1}$ de la coherencia $\mathcal{C}_{l_{1}}$ y en línea rojas la concurrencia $\mathcal{C}_{on}$.}

Se observan dos comportamientos relevantes. Primero, para \( eV/T \lessapprox 1 \), la concurrencia \( \mathcal{C}_{\text{on}} > 0 \), lo que indica la presencia de entrelazamiento entre los sitios \( L \) y \( R \). Segundo, al analizar la coherencia, se aprecia que para \( eV/T \lessapprox 2.4 \) esta disminuye progresivamente hasta anularse, y luego comienza a incrementarse, a partir de \( eV/T \gtrapprox 2.4 \). Este comportamiento es particularmente significativo, ya que la coherencia se anula en un punto cercano al valor en el que la corriente de partículas invierte su dirección. 
\\

Dado el comportamiento tipo Demonio de Maxwell del sistema, resulta pertinente examinar cómo las cantidades termodinámicas se ven influenciadas por la coherencia, el parámetro del Hamiltoniano $\hat{H}_{S}$ que se encarga de inducir coherencia en el sistema, es el acoplamiento \( g \). Por ende, en la Figura \ref{img:fig6resultado} se muestran las cantidades termodinámicas en función del parámetro \( g/\kappa_{L} \) 

\insertimage[\label{img:fig6resultado}]{ejemplos/thermoquantum.pdf}{scale=0.57}{Cantidades termodinámicas para el sistema $LR$ y el Demonio en función de $g/\kappa_{L}$, considerando $eV/T=1$ y el resto de parámetros corresponden a los de la Figura \ref{img:densitymatrix}. }

Se observa que cuando el acoplamiento se encuentra en un régimen \( g/\kappa_{L} \gtrapprox 10^{-1}\), la potencia generada del subsistema $LR$ ($\dot{W}_{LR}$) y la contribución informacional ($T\dot{I}_{LR}$) decrecen, por ende, el subsistema $LR$ realiza más trabajo, y entrega más información al sitio $D$. Este resultado sugiere que un acoplamiento suficientemente fuerte entre los sitios $L$ y $R$ potencia la conversión de información en trabajo. 
Además, bajo esta misma condición \( g/\kappa_{L} \gtrapprox 10^{-1} \), el flujo de energía del demonio \( \dot{E}_{D} \) se vuelve más negativo, lo que indica mayor transporte energético entre el demonio y el sistema $LR$.
\\

Por otra parte, el análisis de las contribuciones individuales a los flujos de información revela cómo el acoplamiento modifica la cantidad de información entregada por cada punto cuántico. Estos efectos se visualizan en la Figura \ref{img:fig7resultado}, donde se aprecia la dependencia de los flujos informacionales con respecto al parámetro \( g \).

\insertimage[\label{img:fig7resultado}]{ejemplos/concuinfodemon.pdf}{scale=0.56}{(a)Flujos de información de cada sitio. (b)Concurrencia y coherencia en función de $g/\kappa_{L}$. }

Un comportamiento notable se presenta cuando el acoplamiento alcanza valores comparables a la tasa de túnel, es decir, \( 10^{-1}< g/\kappa_{L}< 10^{0} \). En este régimen, tanto la concurrencia como la coherencia alcanzan valores cercanos a sus máximos, lo que indica una presencia significativa de efectos cuánticos entre los puntos \( L \) y \( R \), resultado que se asemeja a \cite{prech2023entanglement}. 
\\

Simultáneamente, los flujos de información individuales tienden a igualarse, es decir, \( \dot{I}_{L} \approx \dot{I}_{R} \), lo que sugiere que, en condiciones de entrelazamiento y coherencia elevadas, ambos puntos cuánticos contribuyen de manera equilibrada en el flujo de información del sistema $LR$. 

\label{sec5:demonio1}

\newpage 

\subsection{Comparación con el caso clásico}

Una de las ventajas del formalismo presentado en \cite{potts2021thermodynamically} es que permite incluir términos no seculares en la dinámica del sistema. Esto posibilita la aparición de coherencias y de entrelazamiento no nulos en el régimen \( g \sim \kappa_{L} \). Surge entonces la pregunta natural de si estos efectos cuánticos representan una ventaja para el funcionamiento del demonio. Para abordar esta cuestión, se considera una evolución semiclásica restringida a las componentes diagonales de la matriz densidad del sistema. Inicialmente, se introduce el superoperador

\begin{equation*}
\check{\mathcal{L}}_{f}|\hat{\rho}\rangle\rangle = -i[\hat{H}_{S}, \hat{\rho}] + \mathcal{L}(\hat{\rho}),
\end{equation*}
que combina la dinámica coherente del sistema con la disipación inducida por los baños, como se discute en la ecuación~\eqref{sec2schrodingerthermo}. Este superoperador se descompone como $\check{\mathcal{L}}_{f} = \check{\mathcal{L}}_{0} + \check{\mathcal{V}}$, donde el término $\check{\mathcal{V}}$ representa el acoplamiento entre los sitios y actúa de acuerdo con

\begin{equation*}
\check{\mathcal{V}}|\hat{\rho}\rangle\rangle = -ig[\hat{d}_{L}^{\dagger} \hat{d}_{R} + \hat{d}_{R}^{\dagger} \hat{d}_{L}, \hat{\rho}].
\end{equation*}

Para obtener una ecuación maestra que describa únicamente la dinámica de las componentes diagonales de la matriz densidad, se aplica el formalismo de Nakajima–Zwanzig \cite{zwanzig1966statistical}, con los operadores $\check{\mathcal{P}}$ y $\check{\mathcal{Q}}$ que distinguen la parte relevante (diagonal) y la no relevante (no diagonal) de $\hat{\rho}_{S}$. Este procedimiento conduce a la ecuación maestra

\begin{equation}
    \frac{d}{dt}\check{\mathcal{P}}|\hat{\rho}_{S}(t)\rangle \rangle = \left(\check{\mathcal{L}}_{0} - \check{\mathcal{P}}\check{\mathcal{V}}\check{\mathcal{Q}}\check{\mathcal{L}}^{+}_{0}\check{\mathcal{Q}}\check{\mathcal{V}}\check{\mathcal{P}}\right)\check{\mathcal{P}}|\hat{\rho}_{S}(t)\rangle \rangle. 
\label{ec5:classicalmodel}
\end{equation}
El operador $\check{\mathcal{L}}_{0}^{+}$ corresponde al inverso de Drazin de $\check{\mathcal{L}}_{0}$, definido como
\begin{equation*}
\check{\mathcal{L}}_{0}^{+} = -\int_{0}^{\infty} d\tau\, e^{\check{\mathcal{L}}_{0} \tau} \check{\mathcal{Q}},
\end{equation*}
según lo discutido en~\cite{landi2024current}. La demostración detallada de la ecuación~\eqref{ec5:classicalmodel} se presenta en el Apéndice~\ref{appendix5clasic}.
\\

Disponer de una ecuación maestra que describe únicamente la evolución de las componentes diagonales permite modelar el sistema de manera clásica. Para ello se define el vector de probabilidades de ocupación
\begin{equation*}
\mathbf{P} = [\rho_{000}, \rho_{100}, \rho_{010}, \rho_{001}, \rho_{110}, \rho_{101}, \rho_{011}, \rho_{111}]^{T},
\end{equation*}
cuya dinámica está gobernada por la ecuación maestra

\begin{equation}
    \frac{d}{dt}\textbf{P} = \textbf{W}\textbf{P},
    \label{sec5:classicsemi}
\end{equation}
donde la matriz de transición $\mathbf{W}$ se obtiene a partir de la ecuación~\eqref{ec5:classicalmodel}.
\\

A continuación, se estudia el estado estacionario de la ecuación~\eqref{sec5:classicsemi}. En particular, se comparan las probabilidades de ocupación obtenidas en este modelo clásico con las resultantes del tratamiento cuántico, como se muestra en las Figuras~\ref{img:rho100} y~\ref{img:rho011}. 

\insertimage[\label{img:rho100}]{ejemplos/rho1004.pdf}{scale=0.46}{Elementos diagonales $\rho_{100}$, $\rho_{010}$, $\rho_{111}$ y $\rho_{110}$ en función de $eV/T$. El caso clásico se representa en línea roja y el cuántico en línea azul. Los parámetros corresponden a los de la Figura \ref{img:densitymatrix}.}
\insertimage[\label{img:rho011}]{ejemplos/rho1014.pdf}{scale=0.46}{Elementos diagonales $\rho_{101}$, $\rho_{000}$, $\rho_{011}$ y $\rho_{001}$ en función de $eV/T$.}

De las Figuras \ref{img:rho100} y \ref{img:rho011} se observa que, en ambos casos, las probabilidades de ocupación dominantes corresponden a $\rho_{100}$, $\rho_{010}$ y $\rho_{001}$. Por el contrario, la menor probabilidad de ocupación es $\rho_{111}$.
\\

En torno a $eV/T \approx 2.4$, donde la coherencia es prácticamente nula, todas las probabilidades de ocupación clásicas se cruzan con sus contrapartes cuánticas. En cambio, para $eV/T \lessapprox 1$, los elementos $\rho_{100}$, $\rho_{010}$ y $\rho_{101}$ presentan diferencias apreciables entre los comportamientos clásico y cuántico, lo que puede interpretarse como un efecto de la coherencia sobre dichos elementos.
\\

A partir de la solución estacionaria, se calculan los flujos de partículas e información asociados a cada baño $\alpha$ mediante

\begin{equation}
    \dot{N}_{\alpha} = -\mathbf{N}^{T} \mathbf{W}_{\alpha}\mathbf{P},
    \label{sec5:currentsemi}
\end{equation}
\begin{equation}
    \dot{I}_{\alpha} = -(\ln \mathbf{P})^{T} \mathbf{W}_{\alpha}\mathbf{P},
    \label{sec5:infosemi}
\end{equation}
donde $\mathbf{P}$ es el vector de probabilidades en la base de Fock, $\mathbf{W}_{\alpha}$ es la contribución del baño $\alpha$ en la evolución $\mathbf{W}$, $\mathbf{N}$ contiene el número de partículas en cada estado, y $\ln \mathbf{P}$ denota el vector cuyos elementos son los logaritmos de cada componente de $\mathbf{P}$, es decir, $(\ln \mathbf{P})_i = \ln P_i$\cite{monsel2025autonomous}.
\\

En el modelo clásico, la expresión analítica para el flujo de partículas correspondiente al baño $L$ coincide con la versión cuántica. Usando la ecuación~\eqref{sec5:currentsemi}, la corriente de partículas resulta

\begin{align*}
    \dot{N}_{L} & = \kappa_{L}\big(f_{L}(\epsilon)\rho_{000} - [1-f_{L}(\epsilon)]\rho_{100} \big) \\
        & + \kappa^{U}_{L} \big(f_{L}(\epsilon+U)\rho_{001} - [1-f_{L}(\epsilon+U)]\rho_{101} \big)  \\
        & + \kappa_{L} \big(f_{L}(\epsilon+U_{f})\rho_{010} - [1-f_{L}(\epsilon+U_{f})]\rho_{110} \big)  \\  
        & + \kappa_{L} \big(f_{L}(\epsilon+U + U_{f})\rho_{011} - [1-f_{L}(\epsilon+U+U_{f})]\rho_{111} \big).
\end{align*}

Por su parte, los flujos de información clásicos asociados a los baños $L$ y $R$ se calculan con la ecuación~\eqref{sec5:infosemi}, y se detallan explícitamente en las siguientes expresiones:

% [Aquí van las ecuaciones largas de \dot{I}_L y \dot{I}_R sin cambios]
\begin{align*} \dot{I}_{L} & = \kappa_{L}(f_{L}(\epsilon)[\rho_{000}\ln \rho_{100} - \rho_{000}\ln \rho_{000}] + (1-f_{L}(\epsilon))[\rho_{100}\ln \rho_{000} - \rho_{100}\ln \rho_{100} ] ) \\ & + \kappa^{U}_{L}(f_{L}(\epsilon + U)[\rho_{001}\ln \rho_{101} - \rho_{001}\ln \rho_{001}] + (1-f_{L}(\epsilon + U))[\rho_{101}\ln \rho_{001} -\rho_{101}\ln \rho_{101} ] ) \\ & + \kappa_{L}( f_{L}(\epsilon + U_{f})[\rho_{010}\ln \rho_{110} -\rho_{010}\ln \rho_{010} ] + (1-f_{L}(\epsilon + U_{f}))[\rho_{110}\ln \rho_{010} - \rho_{110}\ln \rho_{110} ] ) \\ & + \kappa_{L}( f_{L}(\epsilon + U + U_{f})[\rho_{011}\ln \rho_{111}- \rho_{011}\ln \rho_{011}]) \\ & + \kappa_{L}( (1-f_{L}(\epsilon+U+U_{f}) )[\rho_{111}\ln \rho_{011} - \rho_{111}\ln \rho_{111}]), \end{align*} 

\begin{align*} \dot{I}_{R} & = \kappa_{R}(f_{R}(\epsilon)[\rho_{000}\ln \rho_{010} - \rho_{000}\ln \rho_{000}] + (1-f_{R}(\epsilon))[\rho_{010}\ln \rho_{000} - \rho_{010}\ln \rho_{010} ] ) \\ & + \kappa^{U}_{R}(f_{R}(\epsilon + U)[\rho_{001}\ln \rho_{011} - \rho_{001}\ln \rho_{001}] + (1-f_{R}(\epsilon + U))[\rho_{011}\ln \rho_{001} -\rho_{011}\ln \rho_{011} ] ) \\ & + \kappa_{R}( f_{R}(\epsilon + U_{f})[\rho_{100}\ln \rho_{110} - \rho_{100}\ln \rho_{100} ] + (1-f_{R}(\epsilon + U_{f}))[\rho_{110}\ln \rho_{100} - \rho_{110}\ln \rho_{110} ] ) \\ & + \kappa_{R}( f_{R}(\epsilon + U + U_{f})[\rho_{101}\ln \rho_{111}- \rho_{101}\ln \rho_{101}]) \\ & + \kappa_{R}( (1-f_{R}(\epsilon+U+U_{f}) )[\rho_{111}\ln \rho_{101} - \rho_{111}\ln \rho_{111}]). \end{align*}


La versión cuántica de los flujos de información se presenta en el Apéndice~\ref{apendix5infoflow}. En la Figura~\ref{img:P4clqm} se comparan gráficamente ambos casos.

\insertimage[\label{img:P4clqm}]{ejemplos/currentinfosemi.pdf}{scale=0.54}{Corriente de partículas y flujos de información en el caso clásico (rojo) y cuántico (azul) en función de $eV/T$.}

Como se aprecia en la Figura~\ref{img:P4clqm}, en el régimen $eV/T\lessapprox 2.4$ la corriente de partículas calculada en el caso cuántico es menor que en el caso clásico, lo cual representa una ventaja: una corriente más baja implica un transporte más eficiente en contra del gradiente de potencial. Por el contrario, para $eV/T \gtrapprox 2.4$ la corriente cuántica supera a la clásica. 
\\

En cuanto al flujo de información, el asociado al sitio $D$ resulta mayor en el caso cuántico, con diferencias apreciables en los rangos $eV/T \lessapprox 1$ y $eV/T \gtrapprox 6$. Este resultado indica que la dinámica cuántica mejora el intercambio de información entre el demonio y el subsistema $LR$. La explicación radica en que el sitio $D$ no solo establece correlaciones clásicas a través de los elementos diagonales, sino que también genera correlaciones cuánticas con el subsistema $LR$, gracias a la presencia de coherencia. Este comportamiento se confirma con los flujos cuánticos obtenidos analíticamente, los cuales incluyen contribuciones provenientes de los elementos no diagonales de la matriz densidad (véase Apéndice~\ref{apendix5infoflow}).
\\

Finalmente, este aumento del flujo de información en el régimen cuántico reduce la tasa de cambio de la energía libre del subsistema $LR$, dada por $\dot{\mathcal{F}}_{LR} \approx T\dot{I}_{LR}$. Como consecuencia, se incrementa la capacidad del sistema para realizar trabajo, lo que evidencia el papel beneficioso de la coherencia cuántica como recurso termodinámico.

\label{sec5:demonio}


% ------------------------------------------------------------------------------
% NUEVO CAPÍTULO
% ------------------------------------------------------------------------------
\chapteranum{Conclusiones}
En esta tesis se derivó una ecuación maestra en la base local para un sistema compuesto por tres puntos cuánticos, la cual permite incorporar y analizar términos no seculares en su evolución. Se observó que, dependiendo de las tasas de túnel asociadas a los reservorios $L$ y $R$, existen regiones del parámetro $eV/T$ en las cuales el sistema exhibe un comportamiento análogo al de un Demonio de Maxwell autónomo. En dichos regímenes, los flujos de información entre el punto cuántico $D$ y el subsistema $LR$ permiten la realización de trabajo, que se manifiesta como un transporte neto de partículas en contra del gradiente de potencial. Este fenómeno corresponde a una aparente violación de la segunda ley de la termodinámica, en concordancia con resultados previos obtenidos para sistemas de dos puntos cuánticos, donde también se identifica un comportamiento tipo Demonio de Maxwell \cite{horowitz2014thermodynamics}.
\\

Se encontró que los flujos de información asociados a los sitios $L$ y $R$ dependen explícitamente del parámetro de acoplamiento $g$. En particular, dichos flujos se igualan en torno a $g \sim \kappa_L$, lo cual coincide con un régimen en el que tanto la coherencia cuántica como la concurrencia alcanzan valores cercanos a su máximo. Este resultado evidencia que la coherencia y el entrelazamiento influyen directamente en la dinámica de los flujos de información.
\\

Asimismo, se comparó la evolución cuántica completa del sistema con una descripción semiclásica que considera únicamente las componentes diagonales de la matriz densidad. Se observó que, para valores de $eV/T \lesssim 2.4$, la corriente de partículas es menor en el caso cuántico. Además, los flujos de información que recibe el sitio $D$ (el Demonio) resultan mayores en la descripción cuántica que en la semiclásica. Esta observación tiene dos implicancias relevantes. Primero, confirma que la coherencia cuántica puede actuar como un recurso termodinámico: al aumentar el flujo de información hacia el demonio, se incrementa la cota superior para el trabajo que el subsistema $LR$ puede realizar. Segundo, muestra que, incluso cuando la interacción entre el demonio y el sistema es puramente clásica (de tipo Coulomb), los efectos cuánticos —en particular la coherencia— continúan desempeñando un rol en la dinámica de la transferencia de información.
\\

Finalmente, el modelo propuesto admite diversas extensiones. Una posible dirección consiste en derivar una ecuación maestra para el caso en que la interacción de Coulomb entre el sitio $D$ y los sitios $L$ y $R$ sea asimétrica. Este tipo de asimetría podría inducir un comportamiento de tipo Demonio de Maxwell autónomo, análogo al reportado recientemente en modelos clásicos \cite{monsel2025autonomous}. Por otra parte, desde el punto de vista del transporte, sería interesante incorporar más puntos cuánticos al sistema, lo que permitiría introducir acoplamientos adicionales y estudiar efectos de coherencia cuántica más complejos, como los discutidos en \cite{khandelwal2025maximal}.

%%%se obtuvo una ecuación maestra para un sistema de 3 puntos cuánticos, a través de la aproximación semilocal
%%%se obtuvo que bajo ciertas condiciones el sistema se comporta como demonio de Maxwell
%%%los flujos de información por separados poseen una dependecia de g y la coherencia y concurrencia
%%%se comparo el caso clásico y el cuántico, en donde en el caso cuántico se presenta una mejoría de los flujos de información  




% ------------------------------------------------------------------------------
% REFERENCIAS, revisar configuración \stylecitereferences
% ------------------------------------------------------------------------------
\bibliography{library}


% ------------------------------------------------------------------------------
% ANEXO
% Existe adicionalmente el entorno \begin{appendixd} que permite insertar
% \chapter y el entorno \begin{appendixdtitle}[style1] (4 estilos diferentes),
% el cual acepta \chapter y escribe el título de anexos encima
% ------------------------------------------------------------------------------
\begin{appendixs}
	
	\section{Cálculos complementarios sección 2}

    \subsection{Matriz de densidad en función del campo de conteo}
\label{apendix:fcs1}

Al sustituir la distribución de probabilidad en la función generadora \eqref{sec2funciongeneradora}, se obtiene

\begin{align*}
    \Lambda(\vec{\lambda},\vec{\chi}) & = \sum_{\textbf{E},\textbf{E}',\textbf{N},\textbf{N}'} \int d\textbf{Q}d\textbf{W}\, P_{t}(\textbf{E}',\textbf{N}'|\textbf{E},\textbf{N}) P_{0}(\textbf{E},\textbf{N})  \\
    & \quad \times \Pi_{\alpha} \delta\!\big(W_{\alpha} - \mu_{\alpha}(N_{\alpha} - N'_{\alpha})\big) \,
      \delta\!\big(Q_{\alpha} + W_{\alpha} -(E_{\alpha} - E'_{\alpha})\big)\,
      e^{-i\vec{\lambda}\cdot \textbf{Q}} e^{-i\vec{\chi}\cdot \textbf{W}} \\
    & = \sum_{\textbf{E},\textbf{E}',\textbf{N},\textbf{N}'} P_{t}(\textbf{E}',\textbf{N}'|\textbf{E},\textbf{N}) P_{0}(\textbf{E},\textbf{N})\, 
        \Pi_{\alpha} e^{-i\chi_{\alpha}\mu_{\alpha}(N_{\alpha} - N'_{\alpha})}
        e^{-i\lambda_{\alpha}((E_{\alpha} -\mu_{\alpha}N_{\alpha}) - (E'_{\alpha} - \mu_{\alpha}N'_{\alpha})) }  \\
    & = \sum_{\textbf{E},\textbf{E}',\textbf{N},\textbf{N}'} \mathrm{Tr}\!\left\{\hat{P}_{\textbf{E}',\textbf{N}'}\hat{U}(t)\hat{P}_{\textbf{E},\textbf{N}}\big(\hat{\rho}_{S}(0)\otimes \Pi_{\alpha}\hat{\tau}_{\alpha}\big)\hat{P}_{\textbf{E},\textbf{N}}\hat{U}^{\dagger}(t)\hat{P}_{\textbf{E}',\textbf{N}'}\right\} \\
    & \quad \times \Pi_{\alpha} e^{-i\chi_{\alpha}\mu_{\alpha}(N_{\alpha} - N'_{\alpha})}
        e^{-i\lambda_{\alpha}((E_{\alpha} -\mu_{\alpha}N_{\alpha}) - (E'_{\alpha} - \mu_{\alpha}N'_{\alpha})) }.
\end{align*}

Sea $\hat{A}(0)$ un observable con descomposición espectral asociada a proyectores $\{\hat{P}_{a_0}\}$. Si estos proyectores actúan sobre un estado diagonal $\hat{\rho}_{\text{diag}}$, se cumple la identidad \cite{esposito2009nonequilibrium}
\begin{equation}
    \sum_{a_0} e^{-i\lambda a_0}\, \hat{P}_{a_0}\, \hat{\rho}_{\text{diag}}\, \hat{P}_{a_0} 
    = e^{-i(\lambda/2)\hat{A}(0)}\, \hat{\rho}_{\text{diag}}\, e^{-i(\lambda/2)\hat{A}(0)}.
    \label{apendixobservable}
\end{equation}
Esta relación permite reescribir la función generadora en términos del observable
\(
\hat{A}(0) = \sum_{\alpha}\big[\lambda_{\alpha}(\hat{H}_{\alpha}-\mu_{\alpha}\hat{N}_{\alpha}) + \chi_{\alpha}\mu_{\alpha}\hat{N}_{\alpha}\big]
\)
y del estado inicial diagonal. En particular,

\begin{align*}
    \Lambda(\vec{\lambda},\vec{\chi}) 
    & = \sum_{\textbf{E}',\textbf{N}'}\mathrm{Tr}\!\left\{ \hat{P}_{\textbf{E}',\textbf{N}'} \hat{U}(t)\,
    e^{-\frac{i}{2}\sum_{\alpha}[\lambda_{\alpha}(\hat{H}_{\alpha} - \mu_{\alpha}\hat{N}_{\alpha}) + \chi_{\alpha}\mu_{\alpha}\hat{N}_{\alpha}]}\,
    \hat{\rho}_{tot}(0)\,
    e^{-\frac{i}{2}\sum_{\alpha}[\lambda_{\alpha}(\hat{H}_{\alpha} - \mu_{\alpha}\hat{N}_{\alpha}) + \chi_{\alpha}\mu_{\alpha}\hat{N}_{\alpha}]}\,
    \hat{U}^{\dagger}(t)\, \hat{P}_{\textbf{E}',\textbf{N}'} \right\} \\
    & \qquad \times \Pi_{\alpha} e^{i\chi_{\alpha}\mu_{\alpha}\hat{N}_{\alpha}} e^{i\lambda_{\alpha}(\hat{H}_{\alpha} -\mu_{\alpha}\hat{N}_{\alpha})} \\
    & = \mathrm{Tr}\!\left\{ \hat{U}(\vec{\lambda},\vec{\chi};t)\,\hat{\rho}_{tot}(0)\,\hat{U}^{\dagger}(-\vec{\lambda},-\vec{\chi};t) \right\}.
\end{align*}

En la última línea, la traza se toma sobre el espacio total sistema+reservorios y
\(\hat{U}(\vec{\lambda},\vec{\chi};t)\) denota el propagador modificado por los campos de conteo.

\newpage 

    \subsection{Matriz densidad generalizada}

La ecuación \eqref{sec2FCS:evolution} se puede expandir en serie de Dyson hasta segundo orden para obtener una expresión aproximada de la evolución de la matriz densidad generalizada. En la representación de interacción,

\begin{equation*}
|\hat{\rho}_{totI}(\vec{\lambda},\vec{\chi},t)\rangle \rangle  = \left( \mathbf{1} + \epsilon \int_{0}^{t}\check{\mathcal{L}}'_{\lambda}(t_{1})dt_{1} + \epsilon^{2}\int_{0}^{t}dt_{1}\int_{0}^{t_{1}}\check{\mathcal{L}}'_{\lambda}(t_{1})\check{\mathcal{L}}'_{\lambda}(t_{2})dt_{2} \right) |\hat{\rho}_{tot}(0)\rangle \rangle .
\end{equation*}
Realizando el cambio de variables \(t_1=T\) y \(t_2=T-s\), la integral doble adopta una forma más conveniente:

\begin{align*}
|\hat{\rho}_{totI}(\vec{\lambda},\vec{\chi},t)\rangle \rangle  
&= \left( \mathbf{1} + \epsilon \int_{0}^{t}\check{\mathcal{L}}'_{\lambda}(T)\,dT + \epsilon^{2}\int_{0}^{t}dT\int_{0}^{T}ds\, \check{\mathcal{L}}'_{\lambda}(T)\check{\mathcal{L}}'_{\lambda}(T-s) \right)|\hat{\rho}_{tot}(0)\rangle \rangle  \\
&= \check{\mathcal{W}}(\vec{\lambda},\vec{\chi},t)\,|\hat{\rho}_{tot}(0)\rangle \rangle \\
&=  \big[\check{\mathcal{W}}_{0}(\vec{\lambda},\vec{\chi},t) + \epsilon \check{\mathcal{W}}_{1}(\vec{\lambda},\vec{\chi},t) + \epsilon^{2}\check{\mathcal{W}}_{2}(\vec{\lambda},\vec{\chi},t)\big] |\hat{\rho}_{tot}(0)\rangle \rangle ,
\end{align*}
donde los superoperadores se definen por

\begin{align*}
\check{\mathcal{W}}_{0}(\vec{\lambda},\vec{\chi},t) &= \mathbf{1},\\
\check{\mathcal{W}}_{1}(\vec{\lambda},\vec{\chi},t) &= \int_{0}^{t}dT \,\check{\mathcal{L}}'_{\lambda}(T),\\
\check{\mathcal{W}}_{2}(\vec{\lambda},\vec{\chi},t) &= \int_{0}^{t}dT \int_{0}^{T}ds \,\check{\mathcal{L}}'_{\lambda}(T)\check{\mathcal{L}}'_{\lambda}(T-s).
\end{align*}
Al segundo orden en \(\epsilon\), la inversa de \(\check{\mathcal{W}}\) queda

\begin{equation}
\check{\mathcal{W}}^{-1}(\vec{\lambda},\vec{\chi},t) =  \check{\mathcal{W}}_{0}(\vec{\lambda},\vec{\chi},t) - \epsilon  \check{\mathcal{W}}_{1}(\vec{\lambda},\vec{\chi},t) +  \epsilon^{2}\big[\check{\mathcal{W}}^{2}_{1}(\vec{\lambda},\vec{\chi},t) -  \check{\mathcal{W}}_{2}(\vec{\lambda},\vec{\chi},t) \big],
\label{apendix2inverseW}
\end{equation}
y satisface la identidad

\begin{equation}
\dot{\check{\mathcal{W}}}(\vec{\lambda},\vec{\chi},t)\check{A}\check{\mathcal{W}}^{-1}(\vec{\lambda},\vec{\chi},t) = \epsilon \dot{\check{\mathcal{W}}}_{1}(\vec{\lambda},\vec{\chi},t)\check{A} + \epsilon^{2}\big[\dot{\check{\mathcal{W}}}_{2}(\vec{\lambda},\vec{\chi},t)\check{A} - \dot{\check{\mathcal{W}}}_{1}(\vec{\lambda},\vec{\chi},t)\check{A}\check{\mathcal{W}}_{1}(\vec{\lambda},\vec{\chi},t) \big].
\label{apendix2Wproperty}
\end{equation}

Para obtener la matriz densidad generalizada del sistema reducido, se proyecta la matriz de densidad total sobre el subespacio del sistema, equivalente a trazar los grados de libertad del reservorio. Definimos el proyector

\begin{equation*}
\check{\mathcal{P}} = \sum_{r}|\rho_{R}^{eq} \rangle \rangle \langle \langle rr|,
\end{equation*}
donde \(\otimes_{\alpha} \hat{\tau}_{\alpha} \to |\hat{\rho}_{R}^{\mathrm{eq}} \rangle\rangle\) es el estado producto de los equilibrios de cada reservorio, escrito como vector en el espacio de Liouville. Aplicando \(\check{\mathcal{P}}\) se obtiene

\begin{equation*}
\check{\mathcal{P}}|\hat{\rho}(\vec{\lambda},\vec{\chi},t) \rangle \rangle = |\hat{\rho}_{S}(\vec{\lambda},\vec{\chi},t)\rangle \rangle \otimes |\hat{\rho}^{eq}_{R}\rangle \rangle .
\end{equation*}
En el marco de Nakajima–Zwanzig, la evolución proyectada viene dada por

\begin{align}
\check{\mathcal{P}}|\hat{\rho}_{totI}(\vec{\lambda},\vec{\chi},t)\rangle \rangle &=  \check{\mathcal{P}} \check{\mathcal{W}}(t)( \check{\mathcal{P}} +  \check{\mathcal{Q}})|\hat{\rho}_{tot}(0)\rangle \rangle , \label{apendix2proyectionev1}\\
\check{\mathcal{Q}}|\hat{\rho}_{totI}(\vec{\lambda},\vec{\chi},t)\rangle \rangle &= \check{\mathcal{Q}} \check{\mathcal{W}}(t)( \check{\mathcal{P}} +  \check{\mathcal{Q}})|\hat{\rho}_{tot}(0)\rangle \rangle .
\label{apendix2proyectionev}
\end{align}
A continuación, se emplean dos hechos. (i) Como el estado inicial del reservorio es diagonal, se cumple
\[
\check{\mathcal{Q}}|\hat{\rho}_{\mathrm{tot}}(0)\rangle\rangle = 0.
\]
(ii) El estado total inicial puede reescribirse como
\[
|\hat{\rho}_{\mathrm{tot}}(0)\rangle\rangle = \check{\mathcal{W}}^{-1}(\vec{\lambda}, \vec{\chi}, t)\, |\hat{\rho}_{\mathrm{tot}}(\vec{\lambda}, \vec{\chi}, t)\rangle\rangle,
\]
lo que permite expresar la evolución proyectada en función del estado actual:
\begin{align*}
|\hat{\rho}_{totI}(0)\rangle \rangle &= (\check{\mathcal{P}} + \check{\mathcal{Q}} )\check{\mathcal{W}}^{-1}(\vec{\lambda},\vec{\chi},t)(\check{\mathcal{P}} + \check{\mathcal{Q}})|\hat{\rho}_{totI}(\vec{\lambda},\vec{\chi},t)\rangle \rangle \\
&= \check{\mathcal{P}}\check{\mathcal{W}}^{-1}(\vec{\lambda},\vec{\chi},t)(\check{\mathcal{P}} + \check{\mathcal{Q}})|\hat{\rho}_{totI}(\vec{\lambda},\vec{\chi},t)\rangle \rangle .
\end{align*}
Al derivar temporalmente las ecuaciones \eqref{apendix2proyectionev1}–\eqref{apendix2proyectionev} se obtiene la evolución proyectada exacta:

\begin{align*}
\check{\mathcal{P}}|\dot{\hat{\rho}}_{totI}(\vec{\lambda},\vec{\chi},t) \rangle \rangle &= \check{\mathcal{P}}\dot{\check{\mathcal{W}}}(\vec{\lambda},\vec{\chi},t)\check{\mathcal{P}}\check{\mathcal{W}}^{-1}(\vec{\lambda},\vec{\chi},t)\check{\mathcal{P}}|\hat{\rho}_{totI}(\vec{\lambda},\vec{\chi},t)\rangle \rangle \\
&= \check{\mathcal{P}}\dot{\check{\mathcal{W}}}(\vec{\lambda},\vec{\chi},t)\check{\mathcal{P}}\check{\mathcal{W}}^{-1}(\vec{\lambda},\vec{\chi},t)\check{\mathcal{Q}}|\hat{\rho}_{totI}(\vec{\lambda},\vec{\chi},t)\rangle \rangle ,
\end{align*}

\begin{align*}
\check{\mathcal{Q}}|\dot{\hat{\rho}}_{totI}(\vec{\lambda},\vec{\chi},t) \rangle \rangle &= \check{\mathcal{Q}}\dot{\check{\mathcal{W}}}(\vec{\lambda},\vec{\chi},t)\check{\mathcal{P}}\check{\mathcal{W}}^{-1}(\vec{\lambda},\vec{\chi},t)\check{\mathcal{P}}|\hat{\rho}_{totI}(\vec{\lambda},\vec{\chi},t)\rangle \rangle \\
&= \check{\mathcal{Q}}\dot{\check{\mathcal{W}}}(\vec{\lambda},\vec{\chi},t)\check{\mathcal{P}}\check{\mathcal{W}}^{-1}(\vec{\lambda},\vec{\chi},t)\check{\mathcal{Q}}|\hat{\rho}_{totI}(\vec{\lambda},\vec{\chi},t)\rangle \rangle .
\end{align*}

Para incorporar la hipótesis de acoplamiento débil (expansión en \(\epsilon\)), se usa la ecuación \eqref{apendix2Wproperty}, con lo cual

\begin{align*}
\check{\mathcal{P}}\dot{\check{\mathcal{W}}}(\vec{\lambda},\vec{\chi},t)\check{\mathcal{P}}\check{\mathcal{W}}^{-1}(\vec{\lambda},\vec{\chi},t)\check{\mathcal{Q}} &= \epsilon \check{\mathcal{P}}\dot{\check{\mathcal{W}}}_{1}(\vec{\lambda},\vec{\chi},t) \check{\mathcal{P}}\check{\mathcal{Q}} \\
&\quad + \epsilon^{2} \check{\mathcal{P}}\dot{\check{\mathcal{W}}}_{2}(\vec{\lambda},\vec{\chi},t) \check{\mathcal{P}}\check{\mathcal{Q}} \\
&\quad - \epsilon^{2}\check{\mathcal{P}}\dot{\check{\mathcal{W}}}_{1}(\vec{\lambda},\vec{\chi},t)\check{\mathcal{P}} \check{\mathcal{W}}_{1}(\vec{\lambda},\vec{\chi},t)\check{\mathcal{Q}},
\end{align*}
y los dos primeros términos se anulan porque \(\check{\mathcal{P}}\check{\mathcal{Q}}=0\). Por otro lado,

\begin{align*}
\check{\mathcal{P}}\dot{\check{\mathcal{W}}}_{1}(\vec{\lambda},\vec{\chi},t)\check{\mathcal{P}} = \sum_{r,r'}|\hat{\rho}^{eq}_{R} \rangle \rangle \langle \langle rr| \check{\mathcal{L}}'_{\lambda}(t)|\hat{\rho}_{R}^{eq}\rangle \rangle \langle \langle r'r'|,
\end{align*}
donde

\begin{equation}
\langle \langle rr|\check{\mathcal{L}}'_{\lambda}(t)|\hat{\rho}_{R}^{eq}\rangle \rangle  = \text{Tr}_{B}\{\hat{\rho}^{eq}_{R}\hat{V}_{\lambda}(t)\} - \text{Tr}_{B}\{\hat{V}_{-\lambda}(t)\hat{\rho}^{eq}_{R}\}.
\label{sec3:ecrandom}
\end{equation}

Como \(\hat{\rho}_{R}^{eq}\) conmuta con \(\hat{H}_{R}\), también conmuta con \(\hat{A}(\lambda,\chi)\); por lo tanto, el término \eqref{sec3:ecrandom} se anula y queda

\begin{align*}
\check{\mathcal{P}}\dot{\check{\mathcal{W}}}(\vec{\lambda},\vec{\chi},t)\check{\mathcal{P}}\check{\mathcal{W}}^{-1}(\vec{\lambda},\vec{\chi},t)\check{\mathcal{P}} &= \epsilon \check{\mathcal{P}}\dot{\check{\mathcal{W}}}_{1}(\vec{\lambda},\vec{\chi},t)\check{\mathcal{P}} \\
&\quad + \epsilon^{2}\check{\mathcal{P}}\dot{\check{\mathcal{W}}}_{2}(\vec{\lambda},\vec{\chi},t)\check{\mathcal{P}} \\
&\quad - \epsilon^{2}\check{\mathcal{P}}\dot{\check{\mathcal{W}}}_{1}(\vec{\lambda},\vec{\chi},t)\check{\mathcal{P}}  \check{\mathcal{W}}_{1}(\vec{\lambda},\vec{\chi},t) \check{\mathcal{P}} .
\end{align*}
Explícitamente,

\begin{equation*}
\check{\mathcal{P}}\dot{\check{\mathcal{W}}}(\vec{\lambda},\vec{\chi},t)\check{\mathcal{P}}\check{\mathcal{W}}^{-1}(\vec{\lambda},\vec{\chi},t)\check{\mathcal{P}} = \epsilon^{2}\check{\mathcal{P}}\int_{0}^{t}ds\, \check{\mathcal{L}}'_{\lambda}(t)\check{\mathcal{L}}'_{\lambda}(t-s)\check{\mathcal{P}} .
\end{equation*}
Usando \(\check{\mathcal{P}}|\dot{\hat{\rho}}_{tot,I}(\vec{\lambda},\vec{\chi},t)\rangle\rangle = |\dot{\hat{\rho}}_{IS}(\vec{\lambda},\vec{\chi},t)\rangle\rangle \otimes |\hat{\rho}_{R}^{eq}\rangle\rangle\) y multiplicando por la izquierda por \(\sum_{r}\langle\langle rr|\) para trazar el reservorio, se obtiene

\begin{equation*}
\dot{\hat{\rho}}_{IS}(\vec{\lambda},\vec{\chi},t) = \epsilon^{2} \sum_{r}\langle \langle rr|\int_{0}^{t}ds \check{\mathcal{L}}'_{\lambda}(t)\check{\mathcal{L}}'_{\lambda}(t-s)|\hat{\rho}_{R}^{eq}\rangle \rangle \hat{\rho}_{IS}(\vec{\lambda},\vec{\chi},t),
\end{equation*}
y, por lo tanto,

\begin{align*}
\sum_{r}\langle \langle rr| \check{\mathcal{L}}'_{\lambda}(t)\check{\mathcal{L}}'_{\lambda}(t-s)|\hat{\rho}_{R}^{eq}\rangle \rangle \hat{\rho}_{IS}(\vec{\lambda},\vec{\chi},t)
= \text{Tr}_{B}\big\{\mathcal{L}'_{\lambda}(t)\mathcal{L}'_{\lambda}(t-s)\hat{\rho}_{R}^{eq}\hat{\rho}_{IS}(\vec{\lambda},\vec{\chi},t) \big\}.
\end{align*}
Que al desarrollar explícitamente,

\begin{equation*}
\mathcal{L}'_{\lambda}(t-s)\hat{\rho}_{IS}(\vec{\lambda},\vec{\chi},t)\hat{\rho}_{R}^{eq} = -i\big[\hat{V}_{\lambda}(t-s)\hat{\rho}_{IS}(\vec{\lambda},\vec{\chi},t)\hat{\rho}^{eq}_{R} -  \hat{\rho}_{IS}(\vec{\lambda},\vec{\chi},t)\hat{\rho}^{eq}_{R}\hat{V}_{-\lambda}(t-s)\big],
\end{equation*}
y, por consiguiente,

\begin{align*}
\mathcal{L}'_{\lambda}(t)\mathcal{L}'_{\lambda}(t-s)\hat{\rho}_{IS}(\vec{\lambda},\vec{\chi},t)\hat{\rho}_{R}^{eq}  
&=  -\hat{V}_{\lambda}(t)\hat{V}_{\lambda}(t-s)\hat{\rho}_{IS}(\vec{\lambda},\vec{\chi},t)\hat{\rho}_{R}^{eq}
+ \hat{V}_{\lambda}(t)\hat{\rho}_{IS}(\vec{\lambda},\vec{\chi},t)\hat{\rho}_{R}^{eq}\hat{V}_{-\lambda}(t-s) \\
&\quad + \hat{V}_{\lambda}(t-s)\hat{\rho}_{IS}(\vec{\lambda},\vec{\chi},t)\hat{\rho}_{R}^{eq}\hat{V}_{-\lambda}(t)
- \hat{\rho}_{IS}(\vec{\lambda},\vec{\chi},t)\hat{\rho}_{R}^{eq}\hat{V}_{-\lambda}(t-s)\hat{V}_{-\lambda}(t).
\end{align*}
Trazando sobre los grados de libertad del reservorio en la última expresión se recupera la ecuación \eqref{ecmaestraVlambda}.


\label{apendixsubsectionmatriz}
    

%%%%%%%%%%%%%%%%%%%%%%%%%%%%%%%%%%%%%%%%%%%%%%%%%%%%%%%%
%%%%%%%seccion%%%%%%%%%%%%%%%%%%%%%%%%%%%%%%%%%%%%%%%%%%
%%%%%%%%%%%%%%%%%%%%%%%%%%%%%%%%%%%%%%%%%%%%%%%%%%%%%%%%

\subsection{Funciones de correlación}

Para derivar la ecuación maestra generalizada en términos de funciones de correlación, se parte de la expresión

\begin{align*}
& \mathrm{Tr}_{B}\{ \hat{V}_{\lambda}(t)\hat{V}_{\lambda}(t-s)\, \hat{\rho}_{IS}(\vec{\lambda},\vec{\chi},t)\hat{\rho}^{eq}_{R} \}  =\\
&\quad - \sum_{\alpha,k,k';j,j'} e^{i(\omega_{j}-\omega_{j'})t}e^{i\omega_{j'}s}\, 
\hat{S}^{\dagger}_{\alpha k;j}\hat{S}_{\alpha,k';j'}\, \hat{\rho}_{IS}(\vec{\lambda},\vec{\chi},t)\,
\mathrm{Tr}_{B}\{ e^{-(i/2)\hat{A}(\lambda,\chi)}\hat{B}^{\dagger}_{\alpha,k}(t)\hat{B}_{\alpha,k'}(t-s)e^{(i/2)\hat{A}(\lambda,\chi)}\hat{\rho}^{eq}_{R}\}.
\end{align*}    
Bajo la hipótesis de homogeneidad temporal, $\langle \hat{V}_{\lambda}(t)\hat{V}_{\lambda}(t-s) \rangle = \langle \hat{V}_{\lambda}(s)\hat{V}_{\lambda}(0) \rangle$, la ecuación se simplifica a  

\begin{align*} 
    \text{Tr}_{B}\{ \hat{V}_{\lambda}(t)\hat{V}_{\lambda}(t-s) \hat{\rho}_{IS}(\vec{\lambda},\vec{\chi},t)&\hat{\rho}^{eq}_{R} \} = \\ & - \sum_{\alpha,k,k';j,j'}e^{i(\omega_{j}-\omega_{j'})t}e^{i\omega_{j'}s}\hat{S}^{\dagger}_{\alpha k;j}\hat{S}_{\alpha,k';j'}\hat{\rho}_{IS}(\vec{\lambda},\vec{\chi},t)\text{Tr}_{B}\{\hat{B}^{\dagger}_{\alpha,k}(s)\hat{B}_{\alpha,k'}\hat{\tau}_{\alpha} \}, \\ \text{Tr}_{B}\{ \hat{\rho}_{IS}(\vec{\lambda},\vec{\chi},t)\hat{\rho}^{eq}_{R} \hat{V}_{-\lambda}(t-s)&\hat{V}_{-\lambda}(t) \} = \\ & - \sum_{\alpha,k,k';j,j'}e^{i(\omega_{j}-\omega_{j'})t}e^{-i\omega_{j}s}\hat{\rho}_{IS}(\vec{\lambda},\vec{\chi},t)\hat{S}^{\dagger}_{\alpha k;j}\hat{S}_{\alpha,k';j'} \text{Tr}_{B}\{ \hat{B}^{\dagger}_{\alpha,k}(s)\hat{B}_{\alpha,k'}\hat{\tau}_{\alpha} \}. 
\end{align*} 
Para analizar el término

\begin{equation*}
    \mathrm{Tr}_{B}\{ \hat{V}_{\lambda}(t)\hat{\rho}_{IS}(\vec{\lambda},\vec{\chi},t)\hat{\rho}_{R}^{eq}\hat{V}_{-\lambda}(t-s) \},
\end{equation*}
se emplea la relación de conmutación asociada a la conservación global de partículas,  

\begin{equation*}
    [\hat{B}_{\alpha,k},\hat{N}_{\alpha}] = n_{\alpha,k}\hat{B}_{\alpha,k},
\end{equation*}
que implica 

\begin{equation}
    e^{C\hat{N}_{\alpha}}\hat{B}_{\alpha,k}e^{-C\hat{N}_{\alpha}} = e^{-Cn_{\alpha,k}}\hat{B}_{\alpha,k}, 
    \label{apendix:conservationparticle}
\end{equation}
con $C$ una constante arbitraria. Esta identidad permite escribir

\begin{align*}
   & \mathrm{Tr}_{B}\{ \hat{V}_{\lambda}(t)\hat{\rho}_{IS}(\vec{\lambda},\vec{\chi},t)\hat{\rho}_{R}^{eq}\hat{V}_{-\lambda}(t-s) \} \\
   &\quad = \sum_{\alpha,k,k';j,j'} e^{i(\omega_{j}-\omega_{j'})t} e^{-i\omega_{j}s}\, 
   \hat{S}_{\alpha,k';j'}\, \hat{\rho}_{IS}(\vec{\lambda},\vec{\chi},t)\, \hat{S}^{\dagger}_{\alpha,k;j}\,
   \mathrm{Tr}_{B}\{e^{-i\hat{A}(\lambda,\chi)} \hat{B}_{\alpha,k}(t)e^{i\hat{A}(\lambda,\chi)}\hat{\rho}_{R}^{eq}\hat{B}^{\dagger}_{\alpha,k'}(t-s) \}.
\end{align*}
Además, para los operadores del baño se cumple la transformación

\begin{align}
    e^{-i\hat{A}(\lambda,\chi)} \hat{B}_{\alpha,k}(t)e^{i\hat{A}(\lambda,\chi)} 
    &= e^{-i\mu_{\alpha}n_{\alpha,k}(\lambda_{\alpha}-\chi_{\alpha})}\, \hat{B}_{\alpha,k}(t+\lambda_{\alpha}), 
    \label{apendix:correlationlambda}
\end{align}
de modo que, al sustituir la ecuación \eqref{apendix:correlationlambda}, se obtiene  

\begin{align*}
 & \mathrm{Tr}_{B}\{ \hat{V}_{\lambda}(t)\hat{\rho}_{IS}(\vec{\lambda},\vec{\chi},t)\hat{\rho}_{R}^{eq}\hat{V}_{-\lambda}(t-s) \} = \\
 & \quad \sum_{\alpha,k,k';j,j'} e^{i(\omega_{j}-\omega_{j'})t} e^{-i\omega_{j}s}\, 
 \hat{S}_{\alpha,k';j'}\, \hat{\rho}_{IS}(\vec{\lambda},\vec{\chi},t)\, \hat{S}^{\dagger}_{\alpha,k;j}\,
 C^{\alpha}_{k,k'}(-s-\lambda_{\alpha})\, e^{-i\mu_{\alpha}n_{\alpha,k}(\lambda_{\alpha}-\chi_{\alpha})},
 \end{align*}

 \begin{align*}
    & \mathrm{Tr}_{B}\{ \hat{V}_{\lambda}(t-s)\hat{\rho}_{IS}(\vec{\lambda},\vec{\chi},t)\hat{\rho}_{R}^{eq}\hat{V}_{-\lambda}(t) \} = \\
    & \quad \sum_{\alpha,k,k';j,j'} e^{i(\omega_{j}-\omega_{j'})t} e^{i\omega_{j'}s}\, 
    \hat{S}_{\alpha,k';j'}\, \hat{\rho}_{IS}(\vec{\lambda},\vec{\chi},t)\, \hat{S}^{\dagger}_{\alpha,k;j}\,
    C^{\alpha}_{k,k'}(s-\lambda_{\alpha})\, e^{-i\mu_{\alpha}n_{\alpha,k}(\lambda_{\alpha}-\chi_{\alpha})}.
 \end{align*}
Finalmente, se obtienen las expresiones de funciones correlación que aparecen en la ecuación \eqref{ecmaestrafinalI}.
\label{finalequation}


%%%%%%%%%%%%%%%%%%%%%%%%%%%%%%%%%%%%%%%%%%%%%%%%%%%%%%
%%%%%%%%%%%%%%%%%seccion%%%%%%%%%%%%%%%%%%%%%%%%%%%%%%
%%%%%%%%%%%%%%%%%%%%%%%%%%%%%%%%%%%%%%%%%%%%%%%%%%%%%%
\subsection{Ecuación de Lindblad generalizada}

A partir de la ecuación \eqref{ecmaestrafinal} se puede desarrollar la expresión

\begin{multline*}
    \frac{d}{dt}\hat{\rho}_{IS}(\vec{\lambda},\vec{\chi},t) = - \sum_{\alpha,k,k';q} \int_{0}^{\infty}ds \left[e^{i\omega_{q}s}C^{\alpha}_{k,k'}(s) \left(\sum_{j}e^{i\omega_{j}t}\hat{S}^{\dagger}_{\alpha,k;j} \right)\left(\sum_{j'}e^{-i\omega_{j'}t}\hat{S}_{\alpha,k';j'} \right)\hat{\rho}_{IS}(\vec{\lambda},\vec{\chi},t)  \right. \\
    \left. +  e^{-i\omega_{q}s}C^{\alpha}_{k,k'}(-s) \hat{\rho}_{IS}(\vec{\lambda},\vec{\chi},t) \left(\sum_{j}e^{i\omega_{j}t}\hat{S}^{\dagger}_{\alpha,k;j} \right)\left(\sum_{j'}e^{-i\omega_{j'}t}\hat{S}_{\alpha,k';j'} \right)\right. \\
    \left. - e^{-i\mu_{\alpha}n_{\alpha,k}(\lambda_{\alpha}-\chi_{\alpha})}(e^{i\omega_{q}s}C^{\alpha}_{k,k'}(s-\lambda_{\alpha}) + e^{-i\omega_{q}s}C^{\alpha}_{k,k'}(-s-\lambda_{\alpha}) ) \right.\\
    \left.\times \left(\sum_{j'}e^{-i\omega_{j'}t}\hat{S}_{\alpha,k';j'} \right)\hat{\rho}_{IS}(\vec{\lambda},\vec{\chi},t) \left(\sum_{j}e^{i\omega_{j}t}\hat{S}^{\dagger}_{\alpha,k;j} \right)    \right],   
\end{multline*}
donde, por simplicidad, se asumirá que $C_{k,k'}^{\alpha} \propto \delta_{k,k'}$. Esta hipótesis se justifica porque, tanto para fermiones como para bosones, los términos del tipo $\langle \hat{c}^{\dagger}_{k}\hat{c}_{j}\rangle$ suelen anularse al promediar en el baño \cite{potts2021thermodynamically}. Bajo esta consideración, la ecuación se reduce a

\begin{multline*}
    \frac{d}{dt}\hat{\rho}_{IS}(\vec{\lambda},\vec{\chi},t) = \\
     - \sum_{\alpha,k;q} \int_{0}^{\infty}ds \left[e^{i\omega_{q}s}C^{\alpha}_{k,k}(s) \hat{S}^{\dagger}_{\alpha,k;q}(t)\hat{S}_{\alpha,k;q}(t)\hat{\rho}_{IS}(\vec{\lambda},\vec{\chi},t)  +  e^{-i\omega_{q}s}C^{\alpha}_{k,k}(-s) \hat{\rho}_{IS}(\vec{\lambda},\vec{\chi},t) \hat{S}^{\dagger}_{\alpha,k;q}(t) \hat{S}_{\alpha,k;q}(t) \right. \\
    \left. - e^{-i\mu_{\alpha}n_{\alpha,k}(\lambda_{\alpha}-\chi_{\alpha})}(e^{i\omega_{q}s}C^{\alpha}_{k,k}(s-\lambda_{\alpha}) + e^{-i\omega_{q}s}C^{\alpha}_{k,k}(-s-\lambda_{\alpha}) ) \hat{S}_{\alpha,k;q}(t) \hat{\rho}_{IS}(\vec{\lambda},\vec{\chi},t) \hat{S}^{\dagger}_{\alpha,k;q}(t)    \right].  
\end{multline*}
Al utilizar las identidades

\begin{align*}
    \int_{0}^{\infty}ds e^{i\omega_{q}s}C^{\alpha}_{k,k}(s) & = \int_{-\infty}^{\infty}ds e^{i\omega_{q}s}(1+\text{sgn}(s))C^{\alpha}_{k,k}(s)/2 \\
    & = \int_{-\infty}^{\infty}ds e^{i\omega_{q}s}C^{\alpha}_{k,k}(s)/2 + i \left(-\frac{i}{2} \right) \int_{-\infty}^{\infty}ds e^{i\omega_{q}s} \text{sgn}(s)C^{\alpha}_{k,k}(s) \\
    & = \frac{1}{2}\gamma_{k}^{\alpha}(\omega_{q}) + i S^{\alpha}_{k}(\omega_{q}),
\end{align*}
y 

\begin{equation*}
    \int_{0}^{\infty}dse^{i\omega_{q}s}C^{\alpha}_{k,k}(s-\lambda_{\alpha}) = e^{i\lambda_{\alpha}\omega_{q}}\int_{0}^{\infty}e^{i(s-\lambda_{\alpha})\omega_{q}}C^{\alpha}_{k,k}(s-\lambda_{\alpha}), 
\end{equation*}
la ecuación maestra toma la forma final

\begin{multline*}
    \frac{d}{dt}\hat{\rho}_{IS}(\vec{\lambda},\vec{\chi},t) = - i \sum_{\alpha,k;q}S^{\alpha}_{k}(\omega_{q})\left[\hat{S}^{\dagger}_{\alpha,k;q}(t)\hat{S}_{\alpha,k;q}(t)\hat{\rho}_{IS}(\vec{\lambda},\vec{\chi},t) - \hat{\rho}_{IS}(\vec{\lambda},\vec{\chi},t)\hat{S}^{\dagger}_{\alpha,k;q}(t)\hat{S}_{\alpha,k;q}(t) \right] \\
    + \sum_{\alpha,k;q} \gamma_{k}^{\alpha}(\omega_{q})\left[ e^{i\lambda_{\alpha}\omega_{q}+ i(\chi_{\alpha} - \lambda_{\alpha})\mu_{\alpha}n_{\alpha,k}}\hat{S}_{\alpha,k;q}(t)\hat{\rho}_{IS}(\vec{\lambda},\vec{\chi},t)\hat{S}^{\dagger}_{\alpha,k;q}(t) - \frac{1}{2}\{\hat{S}^{\dagger}_{\alpha,k;q}(t)\hat{S}_{\alpha,k;q}(t),\hat{\rho}_{IS}(\vec{\lambda},\vec{\chi},t) \} \right],
\end{multline*}
la cual corresponde a la ecuación \eqref{ecmaestracounting}.

\label{apendixGKLSgeneral}



%%%%%%%%%%%%%%%%%%%%%%%%%%%%%%%%%%%%%%%%%%%%%%%%%%%%%%%
%%%%%%%%%%%%%%%%seccion%%%%%%%%%%%%%%%%%%%%%%%%%%%%%%%%
%%%%%%%%%%%%%%%%%%%%%%%%%%%%%%%%%%%%%%%%%%%%%%%%%%%%%%%


\subsection{Condición KMS y funciones correlación espectral}

Para obtener la condición KMS, se define una función de correlación auxiliar como

\begin{align*}
    C^{\alpha N}_{kk}(s) & =  \langle \hat{B}^{\dagger}_{\alpha,k}(s)\hat{B}_{\alpha,k} \rangle_{N} \\
   &  = \text{Tr}\{e^{is(\hat{H}_{\alpha} - \mu_{\alpha}\hat{N}_{\alpha})}\hat{B}^{\dagger}_{\alpha,k}e^{-is(\hat{H}_{\alpha} - \mu_{\alpha}\hat{N}_{\alpha})}\hat{B}_{\alpha,k}\hat{\tau}_{\alpha}  \} \\
   & = \frac{1}{Z}\text{Tr}\{\hat{B}_{\alpha,k} e^{-\beta_{\alpha}(\hat{H}_{\alpha} - \mu_{\alpha}\hat{N}_{\alpha})} e^{is(\hat{H}_{\alpha} - \mu_{\alpha}\hat{N}_{\alpha})}\hat{B}^{\dagger}_{\alpha,k}e^{-is(\hat{H}_{\alpha} - \mu_{\alpha}\hat{N}_{\alpha})}\} \\
   & = \text{Tr}\{\hat{B}_{\alpha,k}e^{i(s+i\beta_{\alpha})(\hat{H}_{\alpha} - \mu_{\alpha}\hat{N}_{\alpha})}\hat{B}^{\dagger}_{\alpha,k}e^{-i(s+i\beta_{\alpha})(\hat{H}_{\alpha} - \mu_{\alpha}\hat{N}_{\alpha})}\hat{\tau}_{\alpha} \} \\
   & = \langle \hat{B}_{\alpha,k} \hat{B}^{\dagger}_{\alpha,k}(s+i\beta_{\alpha})\rangle_{N},
\end{align*}
la cual se relaciona con la función de correlación estándar de la forma

\begin{align*}
    C^{\alpha N}_{kk}(s) & = \text{Tr}\{e^{-is\mu_{\alpha}\hat{N}_{\alpha} }\hat{B}^{\dagger}_{\alpha,k}(s)e^{is\mu_{\alpha}\hat{N}_{\alpha}} \hat{B}_{\alpha,k}e^{-is\mu_{\alpha}\hat{N}_{\alpha} }\hat{\tau}_{\alpha}   \} \\
 & = \text{Tr}\{\hat{B}^{\dagger}_{\alpha,k}(s)e^{is\mu_{\alpha}\hat{N}_{\alpha}} \hat{B}_{\alpha,k}e^{-is\mu_{\alpha}\hat{N}_{\alpha}}\hat{\tau}_{\alpha}   \}\\
 & = e^{-i\mu_{\alpha}n_{\alpha,k}s} C^{\alpha}_{kk}(s).
\end{align*}
Las funciones de correlación espectral pueden expresarse en términos de estas funciones auxiliares como

\begin{align*}
    \gamma^{\alpha}_{k}(\omega) & = \int_{-\infty}^{\infty}ds e^{i\mu_{\alpha}n_{\alpha,k}s} e^{i\omega s}C^{\alpha N}_{k,k}(s) \\
    & = \int_{-\infty}^{\infty}ds e^{i\mu_{\alpha}n_{\alpha,k}s} e^{i\omega s}\langle \hat{B}_{\alpha,k} \hat{B}^{\dagger}_{\alpha,k}(s+i\beta_{\alpha}) \rangle_{N} \\
    & = \int_{-\infty}^{\infty}ds e^{-i(i\beta \mu_{\alpha}n_{\alpha,k})} e^{i\omega s} \langle \hat{B}_{\alpha,k}\hat{B}^{\dagger}_{\alpha,k}(s+i\beta_{\alpha}) \rangle \\
    & = e^{\beta_{\alpha}\mu_{\alpha}n_{\alpha,k}}e^{\beta \omega} \int_{-\infty}^{\infty} ds e^{i\omega(s+i\beta_{\alpha})} \langle \hat{B}_{\alpha,k}\hat{B}^{\dagger}_{\alpha,k}(s+i\beta_{\alpha}) \rangle \\
    & = e^{\beta_{\alpha}(\omega - \mu_{\alpha}n_{\alpha,k})} \gamma^{\alpha}_{k}(-\omega).
\end{align*}

Esta relación constituye la condición de Kubo–Martin–Schwinger (KMS) y asegura el cumplimiento del balance detallado local.

\label{apendixKMS}




%%%%%%%%%%%%%%%%%%%%%%%%%%%%%%%%%%%%%%%%%%%%%%%%%%%%%%%%
%%%%%%%%%%%%%%%%%%%%%%%%%%%%Leyestermo%%%%%%%%%%%%%%%%%%
%%%%%%%%%%%%%%%%%%%%%%%%%%%%%%%%%%%%%%%%%%%%%%%%%%%%%%%%

\subsection{Ley cero}

Para demostrar la validez de la ley cero, se evalúan los disipadores $\mathcal{D}[\hat{S}_{\alpha,k;q}]$ y $\mathcal{D}[\hat{S}^{\dagger}_{\alpha,k;q}]$ actuando sobre el estado de Gibbs. El punto de partida es la siguiente relación de conmutación:

\begin{align*}
    [\hat{S}_{\alpha,k;q},(\hat{H}_{TD} - \mu_{\alpha}\hat{N}_{S})] & = (\omega_{q} - \mu_{\alpha}n_{\alpha,k})\hat{S}_{\alpha,k;q},  \\
    [\hat{S}^{\dagger}_{\alpha,k;q},(\hat{H}_{TD} - \mu_{\alpha}\hat{N}_{S})] & = -(\omega_{q} - \mu_{\alpha}n_{\alpha,k})\hat{S}^{\dagger}_{\alpha,k;q}.
\end{align*}
De aquí se obtiene que

\begin{align*}
    e^{\beta_{\alpha}(\hat{H}_{TD} - \mu_{\alpha}\hat{N}_{S})}\hat{S}_{\alpha,k;q} e^{-\beta_{\alpha}(\hat{H}_{TD} - \mu_{\alpha}\hat{N}_{S})} & = \hat{S}_{\alpha,k;q}e^{-\beta_{\alpha}(\omega_{q} - \mu_{\alpha}n_{\alpha,k})}, \\
    e^{\beta_{\alpha}(\hat{H}_{TD} - \mu_{\alpha}\hat{N}_{S})}\hat{S}^{\dagger}_{\alpha,k;q} e^{-\beta_{\alpha}(\hat{H}_{TD} - \mu_{\alpha}\hat{N}_{S})} & = \hat{S}^{\dagger}_{\alpha,k;q}e^{\beta_{\alpha}(\omega_{q} - \mu_{\alpha}n_{\alpha,k})}.
\end{align*}
Al aplicar el estado de Gibbs en los disipadores, se encuentra

\begin{align*}
    \mathcal{D}[\hat{S}_{\alpha,k;q}]\,e^{-\beta_{\alpha}(\hat{H}_{TD} - \mu_{\alpha}\hat{N}_{S})} & =  e^{-\beta_{\alpha}(\hat{H}_{TD} - \mu_{\alpha}\hat{N}_{S})} \hat{S}_{\alpha,k;q}\hat{S}^{\dagger}_{\alpha,k;q} e^{-\beta_{\alpha}(\omega_{q} - \mu_{\alpha}n_{\alpha,k})} - e^{-\beta_{\alpha}(\hat{H}_{TD} - \mu_{\alpha}\hat{N}_{S})} \hat{S}^{\dagger}_{\alpha,k;q}\hat{S}_{\alpha,k;q}, \\
    \mathcal{D}[\hat{S}^{\dagger}_{\alpha,k;q}]\,e^{-\beta_{\alpha}(\hat{H}_{TD} - \mu_{\alpha}\hat{N}_{S})} & = e^{-\beta_{\alpha}(\hat{H}_{TD} - \mu_{\alpha}\hat{N}_{S})} \hat{S}^{\dagger}_{\alpha,k;q}\hat{S}_{\alpha,k;q} e^{\beta_{\alpha}(\omega_{q} - \mu_{\alpha}n_{\alpha,k})} - e^{-\beta_{\alpha}(\hat{H}_{TD} - \mu_{\alpha}\hat{N}_{S})} \hat{S}_{\alpha,k;q}\hat{S}^{\dagger}_{\alpha,k;q}. 
\end{align*}
Sumando las contribuciones de ambos disipadores en el superoperador de Lindblad $\mathcal{L}_{\alpha}$, se obtiene

\begin{equation*}
    \mathcal{L}_{\alpha}\,e^{-\beta_{\alpha}(\hat{H}_{TD} - \mu_{\alpha}\hat{N}_{S})} 
    \propto \mathcal{D}[\hat{S}_{\alpha,k;q}]\,e^{-\beta_{\alpha}(\hat{H}_{TD} - \mu_{\alpha}\hat{N}_{S})} 
    + e^{-\beta_{\alpha}(\omega_{q} - \mu_{\alpha}n_{\alpha,k})}\,\mathcal{D}[\hat{S}^{\dagger}_{\alpha,k;q}]\,e^{-\beta_{\alpha}(\hat{H}_{TD} - \mu_{\alpha}\hat{N}_{S})} = 0.
\end{equation*}
Con ello se demuestra que el estado de Gibbs es estacionario bajo la dinámica de Lindblad, lo que constituye una manifestación de la ley cero de la termodinámica.



%%%%%%%%%%%%%%%%%%%%%%%%%%%%%%%%%%%%%%%%%%%%%%%%%%%%%%%
%%%%%%%%%%%%%%%%%%%seccion%%%%%%%%%%%%%%%%%%%%%%%%%%%
%%%%%%%%%%%%%%%%%%%%%%%%%%%%%%%%%%%%%%%%%%%%%%%%%%%%%%%

\subsection{Segunda ley}

Para derivar la segunda ley de la termodinámica en este contexto, se comienza calculando la derivada temporal de la entropía de von Neumann:

\begin{align*}
    - \frac{d}{dt}\text{Tr}\{\hat{\rho}_{S}(t)\ln \hat{\rho}_{S}(t)\} 
    &=  -\text{Tr}\Big\{ \frac{d}{dt}\hat{\rho}_{S}(t)\ln \hat{\rho}_{S}(t) \Big\} - \frac{d}{dt}\text{Tr}\{\hat{\rho}_{S}(t)\} \\
    &= - i \,\text{Tr}\{[\hat{H}_{S}+\hat{H}_{LS},\hat{\rho}_{S}(t)]\ln \hat{\rho}_{S}(t)\} 
    - \sum_{\alpha} \text{Tr}\{(\mathcal{L}_{\alpha}\hat{\rho}_{S}(t)) \ln \hat{\rho}_{S}(t)\}  \\
    &= -\sum_{\alpha}\text{Tr}\{(\mathcal{L}_{\alpha}\hat{\rho}_{S}(t)) \ln \hat{\rho}_{S}(t)\}.
\end{align*}
Por otra parte, el flujo de calor hacia el sistema desde el reservorio $\alpha$ está dado por

\begin{align*}
    J_{\alpha} 
    &= \text{Tr}\{ (\hat{H}_{TD} - \mu_{\alpha}\hat{N}_{S})\,\mathcal{L}_{\alpha}(\hat{\rho}_{S}(t)) \} \\
    &= -\frac{1}{\beta_{\alpha}} \text{Tr}\{(\mathcal{L}_{\alpha}\hat{\rho}_{S}(t)) \ln \hat{\rho}_{G}(\beta_{\alpha},\mu_{\alpha}) \} 
    + \frac{\text{Tr}\{e^{-\beta_{\alpha}(\hat{H}_{TD} - \mu_{\alpha}\hat{N}_{S})}\}}{\beta_{\alpha}} \,\text{Tr}\{(\mathcal{L}_{\alpha}\hat{\rho}_{S}(t))\} \\
    &= -\frac{1}{\beta_{\alpha}} \text{Tr}\{(\mathcal{L}_{\alpha}\hat{\rho}_{S}(t))\ln \hat{\rho}_{G}(\beta_{\alpha},\mu_{\alpha}) \},
\end{align*}
donde $\hat{\rho}_{G}(\beta_{\alpha},\mu_{\alpha})$ es el estado de Gibbs correspondiente al reservorio $\alpha$.
\\

Aplicando la desigualdad de Spohn, se obtiene una forma generalizada de la segunda ley de la termodinámica:

\begin{equation*}
    \dot{\sigma} = - \sum_{\alpha} \text{Tr}\{(\mathcal{L}_{\alpha}\hat{\rho}_{S}(t)) [\ln \hat{\rho}_{S}(t) - \ln \hat{\rho}_{G}(\beta_{\alpha},\mu_{\alpha})] \} \geq 0.
\end{equation*}
En consecuencia, la tasa de producción de entropía es no negativa, lo cual establece rigurosamente la validez de la segunda ley en este marco.
\label{apendix:thermolaws}
	% Imagen, se numerará automáticamente con la letra del anexo según
	% la configuración \appendixindepobjnum

\newpage 

    \section{Cálculos complementarios sección 3}
    \subsection{Producción de entropía e información}

Consideremos la definición de la entropía conjunta para dos variables aleatorias discretas \(X\) e \(Y\):

\begin{equation}
    S_{XY} = - \sum_{x,y} p(x,y) \ln p(x,y),
    \label{apendixSxy}
\end{equation}
donde \(p(x,y)\) denota la distribución de probabilidad conjunta de \(X\) e \(Y\).  
Al derivar temporalmente la expresión \eqref{apendixSxy} y emplear la propiedad de antisimetría de la corriente de probabilidades, \(J_{x,x'}^{y,y'} = - J_{x',x}^{y',y}\), se obtiene

\begin{align*}
    \partial_{t}S_{XY} 
    &= - \sum_{x,y} \dot{p}(x,y) \ln p(x,y) - \sum_{x,y} \dot{p}(x,y) \\
    &= - \sum_{x,x';y,y'} J_{x,x'}^{y,y'} \ln p(x,y)  \\
    &= \sum_{x \geq x'; y\geq y'} J_{x,x'}^{y,y'} \ln \frac{p(x',y')}{p(x,y)} \\
    &= \sum_{x \geq x'; y\geq y'} J_{x,x'}^{y,y'} \ln \frac{W_{x,x'}^{y,y'} p(x',y')}{W_{x',x}^{y',y} p(x,y)} 
     +  \sum_{x \geq x'; y\geq y'} J_{x,x'}^{y,y'} \ln \frac{W_{x',x}^{y',y}}{W_{x,x'}^{y,y'}} \\
    &= \dot{\sigma} - \dot{S}_{r}.
\end{align*}

Para demostrar que la producción de entropía es no negativa, obsérvese que si \(J_{x,x'}^{y,y'} > 0\), entonces se cumple que
\[
W_{x,x'}^{y,y'} p(x',y') > W_{x',x}^{y',y} p(x,y),
\]
lo que implica que
\[
\frac{W_{x,x'}^{y,y'} p(x',y')}{W_{x',x}^{y',y} p(x,y)} > 1,
\]
y, en consecuencia, su logaritmo es positivo. Dado que el término correspondiente a la producción de entropía es proporcional a 
\[
J_{x,x'}^{y,y'} \ln \left( \frac{W_{x,x'}^{y,y'} p(x',y')}{W_{x',x}^{y',y} p(x,y)} \right),
\]
cada contribución es no negativa, y por simetría de los índices la suma total es mayor o igual a cero.
\\

Por otro lado, la derivada temporal de la información mutua se expresa como

\begin{equation*}
    \partial_{t} I_{XY} = \sum_{x,y} \dot{p}(x,y) \ln \frac{p(x,y)}{p(x)p(y)} 
    + \sum_{x,y} p(x,y) \frac{\partial}{\partial t} \left( \ln \frac{1}{p(x)p(y)} \right).
\end{equation*}
Al derivar el logaritmo, se obtiene

\begin{align*}
    \frac{\partial}{\partial t} \left( \ln \frac{1}{p(x)p(y)} \right) 
    &= - \left( \frac{\dot{p}(x)}{p(x)} + \frac{\dot{p}(y)}{p(y)} \right),
\end{align*}
y, en consecuencia,

\begin{align*}
    \partial_{t} I_{XY} 
    &= \sum_{x,y} \dot{p}(x,y) \ln \frac{p(x,y)}{p(x)p(y)} 
    - \sum_{x,y} p(x,y) \left( \frac{\dot{p}(x)}{p(x)} + \frac{\dot{p}(y)}{p(y)} \right).
\end{align*}
Reescribiendo los últimos términos con ayuda de la identidad \(\sum_{y} p(x,y) = p(x)\) y de la conservación de la probabilidad (\(\sum_{x} \dot{p}(x) = 0\)), se obtiene

\begin{align}
    \partial_{t} I_{XY} 
    &= \sum_{x,y} \dot{p}(x,y) \ln \frac{p(x,y)}{p(x)p(y)}.
\end{align}
Utilizando la definición de corriente de probabilidad \(J_{x,x'}^{y,y'}\), se puede reescribir como

\begin{align}
    \partial_{t} I_{XY} 
    &= \sum_{x,x';y,y'} J_{x,x'}^{y,y'} \ln \frac{p(x,y)}{p(x)p(y)}.
\end{align}
Separando la dinámica de \(X\) e \(Y\), se llega a

\begin{align*}
    \partial_{t} I_{XY} 
    &= \sum_{x,x';y} J_{x,x'}^{y} \ln \frac{p(x,y)}{p(x)p(y)} 
    + \sum_{x;y,y'} J_{x}^{y,y'} \ln \frac{p(x,y)}{p(x)p(y)}.
\end{align*}
Agrupando términos simétricos mediante pares ordenados \(x \geq x'\) y \(y \geq y'\), se obtiene

\begin{align*}
    \partial_{t} I_{XY} 
    &= \sum_{x \geq x';y} J_{x,x'}^{y} \left[ \ln \frac{p(x,y)}{p(x)p(y)} - \ln \frac{p(x',y)}{p(x')p(y)} \right] \\
    &\quad + \sum_{x;y \geq y'} J_{x}^{y,y'} \left[ \ln \frac{p(x,y)}{p(x)p(y)} - \ln \frac{p(x,y')}{p(x)p(y')} \right].
\end{align*}
Finalmente, aplicando la regla de Bayes, \(p(x,y) = p(x)p(y|x) = p(y)p(x|y)\), se simplifica como

\begin{align*}
    \partial_{t} I_{XY} 
    &= \sum_{x \geq x';y} J_{x,x'}^{y} \ln \frac{p(y|x)}{p(y|x')} 
     + \sum_{x;y \geq y'} J_{x}^{y,y'} \ln \frac{p(x|y)}{p(x|y')}.
\end{align*}

    \label{apendix4:secondlaw}

	 % Desactiva el color de celda

\newpage

\section{Cálculos complemetarios seccion 4}
\label{apendixsection4}


\subsection{Funciones de correlación para un baño de fermiones libres}

Para obtener las funciones de correlación espectral, se comienza escribiendo las funciones de correlación temporales:

\begin{equation*}
    C_{1}^{\alpha}(s) = \text{Tr}\!\left[e^{is\hat{H}_{\alpha}}\hat{B}^{\dagger}_{\alpha,1}e^{-is\hat{H}_{\alpha}}\hat{B}_{\alpha,1}\hat{\tau}_{\alpha}  \right], 
    \hspace{10mm} 
    C_{-1}^{\alpha}(s) = \text{Tr}\!\left[e^{is\hat{H}_{\alpha}}\hat{B}^{\dagger}_{\alpha,-1}e^{-is\hat{H}_{\alpha}}\hat{B}_{\alpha,-1}\hat{\tau}_{\alpha}  \right].
\end{equation*}
Al reescribir estas funciones en términos de los operadores fermiónicos se obtiene:

\begin{equation*}
    C_{1}^{\alpha}(s) = \sum_{l,l'} t_{\alpha,l}t_{\alpha,l'} e^{i\epsilon_{\alpha,l}s} 
    \langle \hat{c}^{\dagger}_{\alpha,l}\hat{c}_{\alpha,l'} \rangle, 
    \hspace{10mm} 
    C^{\alpha}_{-1}(s) = \sum_{l,l'} t_{\alpha,l}t_{\alpha,l'} e^{-i\epsilon_{\alpha,l}s} 
    \langle \hat{c}_{\alpha,l}\hat{c}^{\dagger}_{\alpha,l'} \rangle.
\end{equation*}
En el equilibrio gran canónico, el valor de expectación $\langle \hat{c}^{\dagger}_{\alpha,l}\hat{c}_{\alpha,l}\rangle$ corresponde a la distribución de Fermi $f_{\alpha}(\epsilon_{\alpha,l})$. Así,

\begin{equation*}
    C_{1}^{\alpha}(s) = \sum_{l} t^{2}_{\alpha,l} e^{i\epsilon_{\alpha,l}s} f_{\alpha}(\epsilon_{\alpha,l}), 
    \hspace{10mm} 
    C^{\alpha}_{-1}(s) = \sum_{l} t^{2}_{\alpha,l} e^{-i\epsilon_{\alpha,l}s} \left[1-f_{\alpha}(\epsilon_{\alpha,l})\right].
\end{equation*}
\\

Definimos la tasa de túnel correspondiente al baño $\alpha$ como

\begin{equation*}
    \kappa_{\alpha}(\omega) = 2\pi \sum_{l} t_{\alpha,l}^{2} \delta(\omega - \epsilon_{\alpha,l}),
\end{equation*}
las funciones de correlación pueden expresarse en forma integral:

\begin{align*}
    C_{1}^{\alpha}(s) &= \frac{1}{2\pi} \int_{-\infty}^{\infty} d\omega\, e^{i\omega s}\, \kappa_{\alpha}(\omega) f_{\alpha}(\omega), \\
    C_{-1}^{\alpha}(s) &= \frac{1}{2\pi} \int_{-\infty}^{\infty} d\omega\, e^{-i\omega s}\, \kappa_{\alpha}(\omega)\left[1 - f_{\alpha}(\omega)\right].
\end{align*}
A partir de estas expresiones temporales, se obtiene la representación espectral mediante la transformada de Fourier:

\begin{align*}
    \gamma_{1}^{\alpha}(\omega) &= \int_{-\infty}^{\infty} ds\, e^{i \omega s} C_{1}^{\alpha}(s) 
    = \kappa_{\alpha}(-\omega)\, f_{\alpha}(-\omega), \\
    \gamma_{-1}^{\alpha}(\omega) &= \int_{-\infty}^{\infty} ds\, e^{i \omega s} C_{-1}^{\alpha}(s) 
    = \kappa_{\alpha}(\omega)\left[1 - f_{\alpha}(\omega)\right].
\end{align*}
Estas funciones espectrales codifican tanto la densidad de estados accesibles del entorno como su grado de ocupación.  
Más adelante, se analizará con detalle el régimen markoviano, discutiendo bajo qué condiciones estas funciones de correlación pueden aproximarse por distribuciones localizadas en el tiempo, lo que permite simplificar la descripción dinámica del sistema.

\label{apendix5bathcorre}


\subsection{Operadores de salto en un sistema de tres puntos cuánticos}
\label{apendix5jumpop}

Consideremos un sistema de tres puntos cuánticos descrito por el Hamiltoniano reducido
\begin{equation*}
    \hat{H}_{S} = \epsilon_{R}\hat{n}_{R} + \epsilon_{L}\hat{n}_{L} + \epsilon_{D}\hat{n}_{D}
    + g(\hat{d}^{\dagger}_{R}\hat{d}_{L} + \hat{d}^{\dagger}_{L}\hat{d}_{R})
    + U \hat{n}_{D}(\hat{n}_{R} + \hat{n}_{L}) + U_{f}\hat{n}_{R}\hat{n}_{L}.
\end{equation*}
Para identificar los operadores de salto, diagonalizamos primero el término de acoplamiento mediante la transformación unitaria
\begin{align*}
    \hat{d}_{-} & = \cos(\tfrac{\theta}{2})\hat{d}_{R} - \sin(\tfrac{\theta}{2})\hat{d}_{L}, \\
    \hat{d}_{+} & = \sin(\tfrac{\theta}{2})\hat{d}_{R} + \cos(\tfrac{\theta}{2})\hat{d}_{L},
\end{align*}
con
\begin{equation*}
    \cos \theta = \frac{\Delta}{\sqrt{ \Delta^{2} + g^{2} }}, 
    \qquad 
    \Delta = \frac{\epsilon_{L}-\epsilon_{R}}{2}.
\end{equation*}
La transformación inversa permite escribir los operadores locales en función de los globales:
\begin{align*}
    \hat{d}_{R} & = \cos(\tfrac{\theta}{2})\hat{d}_{-} + \sin(\tfrac{\theta}{2})\hat{d}_{+}, \\
    \hat{d}_{L} & = -\sin(\tfrac{\theta}{2})\hat{d}_{-} + \cos(\tfrac{\theta}{2})\hat{d}_{+}.
\end{align*}
Por lo tanto, los operadores de número locales se expresan como
\begin{align*}
    \hat{d}^{\dagger}_{R}\hat{d}_{R} & =
    \cos^{2}(\tfrac{\theta}{2}) \hat{d}^{\dagger}_{-}\hat{d}_{-} + \sin^{2}(\tfrac{\theta}{2}) \hat{d}^{\dagger}_{+}\hat{d}_{+}
    + \cos(\tfrac{\theta}{2})\sin(\tfrac{\theta}{2})\big[\hat{d}^{\dagger}_{+}\hat{d}_{-} + \hat{d}^{\dagger}_{-}\hat{d}_{+} \big], \\
    \hat{d}^{\dagger}_{L}\hat{d}_{L} & =
    \sin^{2}(\tfrac{\theta}{2}) \hat{d}^{\dagger}_{-}\hat{d}_{-} + \cos^{2}(\tfrac{\theta}{2}) \hat{d}^{\dagger}_{+}\hat{d}_{+}
    - \cos(\tfrac{\theta}{2})\sin(\tfrac{\theta}{2})\big[\hat{d}^{\dagger}_{+}\hat{d}_{-} + \hat{d}^{\dagger}_{-}\hat{d}_{+} \big].
\end{align*}
Por ende,
\begin{equation}
    \hat{n}_{L} + \hat{n}_{R} = \hat{n}_{+} + \hat{n}_{-}.
    \label{apendixnumber}
\end{equation}
Usando la ecuación \eqref{apendixnumber} y definiendo $\bar{\epsilon} = (\epsilon_{R} + \epsilon_{L})/2$, se obtiene
\begin{equation}
    \epsilon_{R} \hat{n}_{R} + \epsilon_{L} \hat{n}_{L}
    = \bar{\epsilon}( \hat{n}_{+} + \hat{n}_{-} )
    - \Delta \sin\theta \, (\hat{d}^{\dagger}_{+}\hat{d}_{-} + \hat{d}^{\dagger}_{-}\hat{d}_{+})
    - \Delta \cos\theta \, (\hat{d}^{\dagger}_{-}\hat{d}_{-} - \hat{d}^{\dagger}_{+}\hat{d}_{+}).
\label{apendix5:ec1}
\end{equation}
Además,
\begin{equation}
    \hat{d}^{\dagger}_{R}\hat{d}_{L} =
    \cos^{2}(\tfrac{\theta}{2})\hat{d}^{\dagger}_{+}\hat{d}_{-}
    - \sin^{2}(\tfrac{\theta}{2}) \hat{d}^{\dagger}_{-}\hat{d}_{+}
    + \sin(\tfrac{\theta}{2})\cos(\tfrac{\theta}{2})\big( \hat{d}^{\dagger}_{+}\hat{d}_{+} - \hat{d}^{\dagger}_{-}\hat{d}_{-} \big),
    \label{apendix5:ec2}
\end{equation}

\begin{equation}
    \hat{d}^{\dagger}_{L}\hat{d}_{R} =
    \cos^{2}(\tfrac{\theta}{2})\hat{d}^{\dagger}_{-}\hat{d}_{+}
    - \sin^{2}(\tfrac{\theta}{2}) \hat{d}^{\dagger}_{+}\hat{d}_{-}
    + \sin(\tfrac{\theta}{2})\cos(\tfrac{\theta}{2})\big( \hat{d}^{\dagger}_{+}\hat{d}_{+} - \hat{d}^{\dagger}_{-}\hat{d}_{-} \big).
    \label{apendix5:ec3}
\end{equation}
Combinando las ecuaciones \eqref{apendix5:ec1}–\eqref{apendix5:ec3} se obtiene
\begin{equation*}
    \epsilon_{R} \hat{n}_{R} + \epsilon_{L} \hat{n}_{L}
    + g( \hat{d}^{\dagger}_{L}\hat{d}_{R} + \hat{d}^{\dagger}_{R}\hat{d}_{L} )
    = (\bar{\epsilon} + \sqrt{\Delta^{2} + g^{2}})\hat{n}_{+}
    +  (\bar{\epsilon} - \sqrt{\Delta^{2} + g^{2}})\hat{n}_{-}.
\end{equation*}
Para el término de Coulomb, usando $\hat{n}^{2}_{i} = \hat{n}_{i}$, se tiene
\begin{align*}
    2 \hat{n}_{R}\hat{n}_{L}
    &= (\hat{n}_{R} +\hat{n}_{L})^{2} - (\hat{n}_{R} + \hat{n}_{L})
     = (\hat{n}_{+} +\hat{n}_{-})^{2} - (\hat{n}_{+} + \hat{n}_{-})
     = 2\hat{n}_{+}\hat{n}_{-}.
\end{align*}
Al definir $\epsilon_{\pm} = \bar{\epsilon} \pm \sqrt{\Delta^{2}+g^{2}}$, el Hamiltoniano en la base global queda
\begin{equation}
    \hat{H}_{S} = \epsilon_{D}\hat{n}_{D} + \epsilon_{+}\hat{n}_{+} + \epsilon_{-}\hat{n}_{-}
    + U\hat{n}_{D}(\hat{n}_{+} + \hat{n}_{-}) + U_{f}\hat{n}_{+}\hat{n}_{-}.
    \label{apendix5:ec4}
\end{equation}
La escritura en la base global permite calcular los operadores de salto a partir de
\begin{align*}
    e^{i \hat{H}_{S}t}\hat{d}_{D}e^{-i\hat{H}_{S}t} & = \hat{d}_{D} + it[\hat{H}_{S},\hat{d}_{D}] + \frac{(it)^{2}}{2} [\hat{H}_{S},[\hat{H}_{S},\hat{d}_{D}]] + \dots, \\
    e^{i \hat{H}_{S}t}\hat{d}_{+}e^{-i\hat{H}_{S}t} & = \hat{d}_{+} + it[\hat{H}_{S},\hat{d}_{+}] + \frac{(it)^{2}}{2} [\hat{H}_{S},[\hat{H}_{S},\hat{d}_{+}]] + \dots, \\
    e^{i \hat{H}_{S}t}\hat{d}_{-}e^{-i\hat{H}_{S}t} & = \hat{d}_{-} + it[\hat{H}_{S},\hat{d}_{-}] + \frac{(it)^{2}}{2} [\hat{H}_{S},[\hat{H}_{S},\hat{d}_{-}]] + \dots.
\end{align*}
Para $\hat{d}_{D}$, usando la identidad
\[
\mathbf{1} = (\mathbf{1} - \hat{n}_{+})(\mathbf{1}-\hat{n}_{-}) + (\mathbf{1} - \hat{n}_{+})\hat{n}_{-} + (\mathbf{1} - \hat{n}_{-})\hat{n}_{+} + \hat{n}_{+}\hat{n}_{-},
\]
y $(\mathbf{1}-\hat{n}_{i})\hat{n}_{i}=\mathbf{0}$, los conmutadores se separan en
\begin{align*}
    [\hat{H}_{S},\hat{d}_{D}(\mathbf{1}-\hat{n}_{+})(\mathbf{1} - \hat{n}_{-})] & = - \epsilon_{D}\hat{d}_{D}(\mathbf{1}-\hat{n}_{+})(\mathbf{1} - \hat{n}_{-}), \\
    [\hat{H}_{S},\hat{d}_{D}(\mathbf{1}-\hat{n}_{+})\hat{n}_{-}] & = - (\epsilon_{D} + U)\hat{d}_{D}(\mathbf{1} - \hat{n}_{+})\hat{n}_{-}, \\
    [\hat{H}_{S},\hat{d}_{D}(\mathbf{1}-\hat{n}_{-})\hat{n}_{+}] & = - (\epsilon_{D} + U)\hat{d}_{D}(\mathbf{1} - \hat{n}_{-})\hat{n}_{+}, \\
    [\hat{H}_{S},\hat{d}_{D}\hat{n}_{+}\hat{n}_{-}] & = - (\epsilon_{D} + 2U)\hat{d}_{D}\hat{n}_{+}\hat{n}_{-},
\end{align*}
y, aplicando de nuevo el conmutador,
\begin{align*}
    [\hat{H}_{S},[\hat{H}_{S},\hat{d}_{D}]] & =
    \epsilon_{D}^{2}\hat{d}_{D}(\mathbf{1} - \hat{n}_{+}) (\mathbf{1} - \hat{n}_{-})
    + (\epsilon_{D} + U)^{2}\hat{d}_{D}(\mathbf{1} - \hat{n}_{+})\hat{n}_{-} \\
    & \quad + (\epsilon_{D}+U)^{2}\hat{d}_{D}(\mathbf{1} - \hat{n}_{-})\hat{n}_{+}
    + (\epsilon_{D} + 2U)^{2}\hat{d}_{D}\hat{n}_{+}\hat{n}_{-}.
\end{align*}
De este modo,
\begin{align*}
    e^{i \hat{H}_{S}t}\hat{d}_{D}e^{-i\hat{H}_{S}t}  & =
    e^{-i\epsilon_{D}t} \hat{d}_{D}(\mathbf{1} - \hat{n}_{+}) (\mathbf{1} - \hat{n}_{-})
    +  e^{-i(\epsilon_{D}+U)t} \hat{d}_{D}\big[(\mathbf{1} - \hat{n}_{+})\hat{n}_{-} + (\mathbf{1} - \hat{n}_{-})\hat{n}_{+}\big] \\
    & \quad + e^{-i(\epsilon_{D} + 2U)t}\hat{d}_{D} \hat{n}_{+}\hat{n}_{-}.
\end{align*}
Para $\hat{d}_{+}$, con la identidad
\[
\mathbf{1} = (\mathbf{1}-\hat{n}_{D})(\mathbf{1}-\hat{n}_{-}) + (\mathbf{1}-\hat{n}_{-})\hat{n}_{D} + (\mathbf{1}-\hat{n}_{D})\hat{n}_{-} + \hat{n}_{D}\hat{n}_{-},
\]
los conmutadores se separan en
\begin{align*}
    [\hat{H}_{S},\hat{d}_{+}(\mathbf{1}-\hat{n}_{D})(\mathbf{1} - \hat{n}_{-})] & = - \epsilon_{+}\hat{d}_{+}(\mathbf{1}-\hat{n}_{D})(\mathbf{1} - \hat{n}_{-}), \\
    [\hat{H}_{S},\hat{d}_{+}(\mathbf{1}-\hat{n}_{D})\hat{n}_{-}] & = - (\epsilon_{+} + U_{f})\hat{d}_{+}(\mathbf{1}-\hat{n}_{D})\hat{n}_{-}, \\
    [\hat{H}_{S},\hat{d}_{+}(\mathbf{1}-\hat{n}_{-})\hat{n}_{D}] & = - (\epsilon_{+} + U)\hat{d}_{+}(\mathbf{1} - \hat{n}_{-})\hat{n}_{D}, \\
    [\hat{H}_{S},\hat{d}_{+}\hat{n}_{D}\hat{n}_{-}] & = - (\epsilon_{+} + U + U_{f})\hat{d}_{+}\hat{n}_{D}\hat{n}_{-},
\end{align*}
y, por lo tanto,
\begin{align*}
    e^{i \hat{H}_{S}t}\hat{d}_{+}e^{-i\hat{H}_{S}t} & =
    e^{-i\epsilon_{+}t}\hat{d}_{+}(\mathbf{1}-\hat{n}_{D})(\mathbf{1}-\hat{n}_{-})
    + e^{-i(\epsilon_{+}+U_{f})t} \hat{d}_{+}(\mathbf{1}-\hat{n}_{D})\hat{n}_{-} \\
    & \quad + e^{-i(\epsilon_{+}+U)t}\hat{d}_{+}(\mathbf{1}-\hat{n}_{-})\hat{n}_{D}
    + e^{-i(\epsilon_{+}+U+U_{f})t}\hat{d}_{+}\hat{n}_{-}\hat{n}_{D}.
\end{align*}
Por simetría, para $\hat{d}_{-}$ se obtiene
\begin{align*}
    e^{i \hat{H}_{S}t}\hat{d}_{-}e^{-i\hat{H}_{S}t} & =
    e^{-i\epsilon_{-}t}\hat{d}_{-}(\mathbf{1}-\hat{n}_{D})(\mathbf{1}-\hat{n}_{+})
    + e^{-i(\epsilon_{-}+U_{f})t} \hat{d}_{-}(\mathbf{1}-\hat{n}_{D})\hat{n}_{+} \\
    & \quad + e^{-i(\epsilon_{-}+U)t}\hat{d}_{-}(\mathbf{1}-\hat{n}_{+})\hat{n}_{D}
    + e^{-i(\epsilon_{-}+U+U_{f})t}\hat{d}_{-}\hat{n}_{+}\hat{n}_{D}.
\end{align*}
Finalmente, los operadores de salto en la base local se expresan como
\begin{align*}
    e^{i\hat{H}_{S}t}\hat{d}_{R}e^{-i\hat{H}_{S}t} & =
    \cos(\tfrac{\theta}{2})\, e^{i\hat{H}_{S}t}\hat{d}_{-}e^{-i\hat{H}_{S}t}
    + \sin(\tfrac{\theta}{2})\, e^{i\hat{H}_{S}t}\hat{d}_{+}e^{-i\hat{H}_{S}t}, \\
    e^{i\hat{H}_{S}t}\hat{d}_{L}e^{-i\hat{H}_{S}t} & =
    -\sin(\tfrac{\theta}{2})\, e^{i\hat{H}_{S}t}\hat{d}_{-}e^{-i\hat{H}_{S}t}
    + \cos(\tfrac{\theta}{2})\, e^{i\hat{H}_{S}t}\hat{d}_{+}e^{-i\hat{H}_{S}t}.
\end{align*}
En consecuencia, las frecuencias de Bohr asociadas a los operadores de creación/aniquilación del sistema son
\begin{align*}
    \hat{d}_{D} & \rightarrow \{\epsilon_{D}, \epsilon_{D}+U, \epsilon_{D}+2U\}, \\
    \hat{d}_{L}, \hat{d}_{R} & \rightarrow \{\epsilon_{\pm}, \epsilon_{\pm}+U, \epsilon_{\pm}+U_{f}, \epsilon_{\pm}+U+U_{f}\}.
\end{align*}

\subsection{Agrupación de frecuencias para dos puntos cuánticos}
\label{apendix5frecuencygroup}

En el caso degenerado \(\epsilon_{L}=\epsilon_{R}=\epsilon\) se obtiene \(\epsilon_{\pm}=\epsilon\pm g\). En este régimen, si el acoplamiento \(g\) es pequeño frente al inverso del tiempo de correlación del baño, es decir,
\[
2g\,\tau_{B}\ll 1,
\]
es lícito agrupar las dos frecuencias cercanas en una sola. La agrupación queda
\begin{align*}
&(\epsilon_{+},\epsilon_{-}) \to \epsilon,\\
&(\epsilon_{+}+U,\epsilon_{-}+U) \to \epsilon+U,\\
&(\epsilon_{+}+U_{f},\epsilon_{-}+U_{f}) \to \epsilon+U_{f},\\
&(\epsilon_{+}+U+U_{f},\epsilon_{-}+U+U_{f}) \to \epsilon+U+U_{f}.
\end{align*}

Para determinar los operadores de salto asociados a \(\hat{d}_{L}\) en el caso degenerado, se suman las contribuciones de las transiciones a frecuencias \(\epsilon_{+}\) y \(\epsilon_{-}\), de modo de formar un único operador efectivo a la frecuencia agrupada. Comenzamos con la frecuencia \(\epsilon\), para lo cual calculamos
\begin{align*}
\frac{\hat{d}_{+}(\mathbf{1}-\hat{n}_{D})(\mathbf{1}-\hat{n}_{-})}{\sqrt{2}}
-\frac{\hat{d}_{-}(\mathbf{1}-\hat{n}_{D})(\mathbf{1}-\hat{n}_{+})}{\sqrt{2}}
&=\frac{(\mathbf{1}-\hat{n}_{D})}{\sqrt{2}}\Big[\hat{d}_{+}(\mathbf{1}-\hat{n}_{-})
- \hat{d}_{-}(\mathbf{1}-\hat{n}_{+})\Big],
\end{align*}
y además
\begin{align}
\hat{d}_{+}(\mathbf{1}-\hat{n}_{-})-\hat{d}_{-}(\mathbf{1}-\hat{n}_{+})
&=\frac{\hat{d}_{L}+\hat{d}_{R}}{\sqrt{2}}(\mathbf{1}-\hat{n}_{-})
-\frac{\hat{d}_{R}-\hat{d}_{L}}{\sqrt{2}}(\mathbf{1}-\hat{n}_{+})\nonumber\\
&=\frac{\hat{d}_{L}}{\sqrt{2}}(2-\hat{n}_{+}-\hat{n}_{-})
+\frac{\hat{d}_{R}}{\sqrt{2}}(\hat{n}_{+}-\hat{n}_{-})\nonumber\\
&=\frac{\hat{d}_{L}}{\sqrt{2}}(2-\hat{n}_{L}-\hat{n}_{R})
+\frac{\hat{d}_{R}}{\sqrt{2}}(\hat{d}^{\dagger}_{R}\hat{d}_{L}+\hat{d}^{\dagger}_{L}\hat{d}_{R}),
\label{apendix5:ec5x}
\end{align}
donde se empleó \(\hat{d}_{L}(\mathbf{1}-\hat{n}_{L})=\hat{d}_{L}\hat{d}_{L}\hat{d}^{\dagger}_{L}=0\) y
\[
\hat{d}_{R}(\hat{d}^{\dagger}_{R}\hat{d}_{L}+\hat{d}^{\dagger}_{L}\hat{d}_{R})
=\hat{d}_{L}(\mathbf{1}-\hat{n}_{R}).
\]
Sustituyendo en la ecuación \eqref{apendix5:ec5x}, se obtiene la identidad
\[
\hat{d}_{+}(\mathbf{1}-\hat{n}_{-})-\hat{d}_{-}(\mathbf{1}-\hat{n}_{+})
=\hat{d}_{L}(\mathbf{1}-\hat{n}_{R}),
\]
lo que asocia la frecuencia agrupada \(\epsilon\) con el operador
\(\hat{d}_{L}(\mathbf{1}-\hat{n}_{D})(\mathbf{1}-\hat{n}_{R})\).
\\

De forma análoga, para \((\epsilon_{-}+U,\epsilon_{+}+U)\to \epsilon+U\), el operador efectivo resulta
\[
\frac{\hat{d}_{+}\hat{n}_{D}(\mathbf{1}-\hat{n}_{-})}{\sqrt{2}}
-\frac{\hat{d}_{-}\hat{n}_{D}(\mathbf{1}-\hat{n}_{+})}{\sqrt{2}}
=\hat{d}_{L}\hat{n}_{D}(\mathbf{1}-\hat{n}_{R}).
\]
Para \((\epsilon_{-}+U_{f},\epsilon_{+}+U_{f})\to \epsilon+U_{f}\), se agrupan
\[
\frac{\hat{d}_{+}\hat{n}_{-}(\mathbf{1}-\hat{n}_{D})}{\sqrt{2}}
-\frac{\hat{d}_{-}\hat{n}_{+}(\mathbf{1}-\hat{n}_{D})}{\sqrt{2}}
=\frac{(\mathbf{1}-\hat{n}_{D})}{\sqrt{2}}\big[\hat{d}_{+}\hat{n}_{-}-\hat{d}_{-}\hat{n}_{+}\big],
\]
donde
\begin{align*}
\hat{d}_{+}\hat{n}_{-}-\hat{d}_{-}\hat{n}_{+}
&=\frac{\hat{d}_{R}+\hat{d}_{L}}{\sqrt{2}}\hat{n}_{-}
-\frac{\hat{d}_{R}-\hat{d}_{L}}{\sqrt{2}}\hat{n}_{+}\\
&=-\frac{\hat{d}_{R}}{\sqrt{2}}(\hat{d}^{\dagger}_{R}\hat{d}_{L}+\hat{d}^{\dagger}_{L}\hat{d}_{R})
+\frac{\hat{d}_{L}}{\sqrt{2}}(\hat{n}_{L}+\hat{n}_{R})\\
&=\frac{2\hat{d}_{L}}{\sqrt{2}}\hat{n}_{R},
\end{align*}
lo que asocia la frecuencia \(\epsilon+U_{f}\) al operador
\(\hat{d}_{L}\hat{n}_{R}(\mathbf{1}-\hat{n}_{D})\).
\\

Finalmente, para \((\epsilon_{-}+U+U_{f},\epsilon_{+}+U+U_{f})\to \epsilon+U+U_{f}\), el operador efectivo queda
\[
\frac{\hat{d}_{+}\hat{n}_{D}\hat{n}_{-}}{\sqrt{2}}
-\frac{\hat{d}_{-}\hat{n}_{D}\hat{n}_{+}}{\sqrt{2}}
=\hat{d}_{L}\hat{n}_{D}\hat{n}_{R}.
\]
En resumen, tras la agrupación de frecuencias, los operadores de salto asociados a \(\hat{d}_{L}\) son
\begin{align*}
&\epsilon \ \ \to\ \ \hat{d}_{L}(\mathbf{1}-\hat{n}_{R})(\mathbf{1}-\hat{n}_{D}),\\
&\epsilon+U \ \ \to\ \ \hat{d}_{L}\hat{n}_{D}(\mathbf{1}-\hat{n}_{R}),\\
&\epsilon+U_{f} \ \ \to\ \ \hat{d}_{L}\hat{n}_{R}(\mathbf{1}-\hat{n}_{D}),\\
&\epsilon+U+U_{f} \ \ \to\ \ \hat{d}_{L}\hat{n}_{D}\hat{n}_{R}.
\end{align*}

El mismo procedimiento se aplica al operador local \(\hat{d}_{R}\). Para la agrupación \((\epsilon_{+},\epsilon_{-})\to \epsilon\) se considera
\begin{align*}
\frac{\hat{d}_{+}(\mathbf{1}-\hat{n}_{D})(\mathbf{1}-\hat{n}_{-})}{\sqrt{2}}
+\frac{\hat{d}_{-}(\mathbf{1}-\hat{n}_{D})(\mathbf{1}-\hat{n}_{+})}{\sqrt{2}}
&=\frac{(\mathbf{1}-\hat{n}_{D})}{\sqrt{2}}\big[\hat{d}_{+}(\mathbf{1}-\hat{n}_{-})
+\hat{d}_{-}(\mathbf{1}-\hat{n}_{+})\big],
\end{align*}
con
\begin{align*}
\hat{d}_{+}(\mathbf{1}-\hat{n}_{-})+\hat{d}_{-}(\mathbf{1}-\hat{n}_{+})
&=\frac{\hat{d}_{R}+\hat{d}_{L}}{\sqrt{2}}(\mathbf{1}-\hat{n}_{-})
+\frac{\hat{d}_{R}-\hat{d}_{L}}{\sqrt{2}}(\mathbf{1}-\hat{n}_{+})\\
&=\frac{\hat{d}_{R}}{\sqrt{2}}\big(2-(\hat{n}_{L}+\hat{n}_{R})\big)
+\frac{\hat{d}_{L}}{\sqrt{2}}(\hat{n}_{+}-\hat{n}_{-})\\
&=\frac{\hat{d}_{R}}{\sqrt{2}}(\mathbf{1}-\hat{n}_{R})
+\frac{\hat{d}_{L}}{\sqrt{2}}(\hat{d}^{\dagger}_{R}\hat{d}_{L}+\hat{d}^{\dagger}_{L}\hat{d}_{R})\\
&=\sqrt{2}\,\hat{d}_{R}(\mathbf{1}-\hat{n}_{R}),
\end{align*}
por lo cual el operador efectivo para \(\epsilon\) es
\(\hat{d}_{R}(\mathbf{1}-\hat{n}_{D})(\mathbf{1}-\hat{n}_{L})\).
\\

Para \((\epsilon_{+}+U,\epsilon_{-}+U)\to \epsilon+U\) se obtiene
\[
\frac{\hat{d}_{+}(\mathbf{1}-\hat{n}_{-})\hat{n}_{D}}{\sqrt{2}}
+\frac{\hat{d}_{-}(\mathbf{1}-\hat{n}_{+})\hat{n}_{D}}{\sqrt{2}}
=\hat{d}_{R}\hat{n}_{D}(\mathbf{1}-\hat{n}_{L}).
\]
Mientras que, para \((\epsilon_{+}+U_{f},\epsilon_{-}+U_{f})\to \epsilon+U_{f}\),
\[
\frac{\hat{d}_{+}(\mathbf{1}-\hat{n}_{D})\hat{n}_{-}}{\sqrt{2}}
+\frac{\hat{d}_{-}(\mathbf{1}-\hat{n}_{D})\hat{n}_{+}}{\sqrt{2}}
=\frac{(\mathbf{1}-\hat{n}_{D})}{\sqrt{2}}\big[\hat{d}_{+}\hat{n}_{-}+\hat{d}_{-}\hat{n}_{+}\big],
\]
y
\begin{align*}
\hat{d}_{+}\hat{n}_{-}+\hat{d}_{-}\hat{n}_{+}
&=\frac{\hat{d}_{R}+\hat{d}_{L}}{\sqrt{2}}\hat{n}_{-}
+\frac{\hat{d}_{R}-\hat{d}_{L}}{\sqrt{2}}\hat{n}_{+}\\
&=\frac{\hat{d}_{R}}{\sqrt{2}}(\hat{n}_{L}+\hat{n}_{R})
-\frac{\hat{d}_{L}}{\sqrt{2}}(\hat{d}^{\dagger}_{L}\hat{d}_{R}+\hat{d}^{\dagger}_{R}\hat{d}_{L})\\
&=\sqrt{2}\,\hat{d}_{R}\hat{n}_{L},
\end{align*}
de donde el operador efectivo es \(\hat{d}_{R}(\mathbf{1}-\hat{n}_{D})\hat{n}_{L}\). Finalmente, para
\((\epsilon_{+}+U+U_{f},\epsilon_{-}+U+U_{f})\to \epsilon+U+U_{f}\), resulta
\(\hat{d}_{R}\hat{n}_{D}\hat{n}_{L}\).
\\

En resumen, los operadores de salto asociados a \(\hat{d}_{R}\) tras la agrupación son
\begin{align*}
&\epsilon \ \ \to\ \ \hat{d}_{R}(\mathbf{1}-\hat{n}_{L})(\mathbf{1}-\hat{n}_{D}),\\
&\epsilon+U \ \ \to\ \ \hat{d}_{R}\hat{n}_{D}(\mathbf{1}-\hat{n}_{L}),\\
&\epsilon+U_{f} \ \ \to\ \ \hat{d}_{R}\hat{n}_{L}(\mathbf{1}-\hat{n}_{D}),\\
&\epsilon+U+U_{f} \ \ \to\ \ \hat{d}_{R}\hat{n}_{D}\hat{n}_{L}.
\end{align*}


%%%%%%%%%%%%%%%%%%%%%%%%%ojooooaqui%%%%%%%%%%%%%
%\subsection{Límitaciones de la aproximación semilocal}

%\label{apendix5límites}

\subsection{Aspecto Markoviano de las funciones correlación}
Para que el sistema descrito en la sección \ref{sec5:modelo} exhiba un comportamiento análogo al de un Demonio de Maxwell autónomo, es necesario que las tasas de túnel $\kappa_{j}(\omega)$ respondan de manera diferenciada dependiendo de si el sitio $D$ se encuentra ocupado o desocupado. Sin embargo, no toda dependencia funcional de $\kappa_{j}(\omega)$ garantiza un comportamiento Markoviano del sistema. Esta condición puede evaluarse mediante el cálculo explícito de las funciones de correlación del reservorio, ya que el carácter Markoviano se encuentra fuertemente ligado al tiempo de correlación característico del entorno, denotado por $\tau_{B}$. Para analizar este aspecto, se requiere calcular las funciones de correlación correspondientes a un entorno fermiónico, las cuales, en general, adoptan la forma

\begin{equation*}
    C^{\sigma}_{j}(t) = \frac{1}{2\pi} \int_{-\infty}^{\infty} d\omega e^{i\sigma \omega t} \kappa_{j}(\omega) f_{F}(\sigma \beta(\omega-\mu)),
\end{equation*}
donde $f_{F}(x) = (\exp(x)+1)^{-1}$, \(j\in\{L,R\}\) y $\sigma = \pm$. 
\\

La distribución de Fermi puede ser calculada mediante aproximantes de Padé \cite{hu2011pade,schinabeck2019hierarchical}. Esto permite expresar la ditribución como una sumatoria 

\begin{equation}
    f_{F}(x) \approx \frac{1}{2} - \sum_{l=0}^{N} \frac{ 2 K_{l}x }{ x^{2} + \xi^{2}_{l} },
    \label{apendix5:pade}
\end{equation}
los coeficientes $K_{l}$ y $\xi_{l}$ pueden ser calculados numéricamente y se encuentran tabulados en \cite{hu2011pade}. Para evaluar numéricamente la función de correlación, es necesario especificar una forma explícita para $\kappa_{j}(\omega)$, la cual debe presentar un máximo en cierta frecuencia característica $\epsilon_{0}$. Una elección común consiste en considerar un perfil lorentziano dado por

\begin{equation*}
    \kappa_{j}(\omega) = \kappa_{0} + \frac{\kappa_{f} W^{2}}{(\omega - \epsilon_{0})^{2} + W^{2}},
\end{equation*}
donde $W$ representa el ancho de la lorentziana, $\kappa_{0}$ y $\kappa_{f}$ son constantes que controlan la amplitud. En el estudio del Demonio de Maxwell autónomo, es fundamental distinguir dos casos relevantes: cuando el punto cuántico $D$ se encuentra desocupado o ocupado. Esta distinción se refleja en la evaluación de las tasas de túnel en dos frecuencias distintas, $\epsilon_{0}$ y $\epsilon_{0} + U$, tal como se ilustra en la Figura \ref{img:gammas} para las tasas de túnel correspondientes a los baños $L$ y $R$.

\insertimage[\label{img:gammas}]{ejemplos/gammmas.pdf}{scale=0.35}{Tasas de túnel en función de las frecuencias $\omega$.}

Para obtener tasas de túnel que satisfagan dichas condiciones, se puede plantear el siguiente sistema de ecuaciones 

\begin{align*}
    \kappa_{j}(\epsilon_{0}) & = \kappa_{0} + \kappa_{f} \\
    \kappa_{j}(\epsilon_{0} + U) & = \kappa_{0} + \frac{\kappa_{f} W^{2} }{U^{2} + W^{2}}.
\end{align*}
A partir de estas igualdades, es posible determinar los parámetros $\kappa_{0}$, $\kappa_{f}$ y $W$. 
\\

Al sustituir la expresión de $\kappa_{j}(\omega)$ en la función correlación, se obtiene

\begin{align*}
    C^{\sigma}_{j}(t) & = \frac{\kappa_{0}}{2\pi} \int_{-\infty}^{\infty}d\omega e^{i\sigma \omega t} f_{F}(\sigma \beta (\omega-\mu)) + \frac{\kappa_{f}}{2\pi} \int_{-\infty}^{\infty}d\omega e^{i\sigma \omega t}\left[ \frac{W^{2}}{(\omega-\epsilon_{0})^{2} + W^{2}} \right] f_{F}(\sigma \beta (\omega-\mu)).
\end{align*}
La función correlación puede descomponerse en dos contribuciones

\begin{equation}
    C^{\sigma}_{j0}(t) = \frac{\kappa_{0}}{2\pi} \int_{-\infty}^{\infty}d\omega e^{i\sigma \omega t} f_{F}(\sigma \beta (\omega-\mu))
    \label{apendix5:correlacion0}
\end{equation}

\begin{equation}
    C^{\sigma}_{jf}(t) = \frac{\kappa_{f}}{2\pi} \int_{-\infty}^{\infty}d\omega e^{i\sigma \omega t}\left[ \frac{W^{2}}{(\omega-\epsilon_{0})^{2} + W^{2}} \right] f_{F}(\sigma \beta (\omega-\mu)),
    \label{apendix5:correlacionf}
\end{equation}
y se podrá determinar el tiempo de correlación del baño como $\tau_{B} = \max\{\tau_{B0}, \tau_{Bf}\}$, donde $\tau_{B0}$ es el tiempo de correlación de la parte $C^{\sigma}_{j0}(t)$ y $\tau_{Bf}$ es el tiempo de correlación de la parte $C^{\sigma}_{jf}(t)$. 
\\

A continuación, se calculará una expresión analítica aproximada para la función de correlación $C^{\sigma}_{jf}(t)$. Para resolver la integral de la ecuación \eqref{apendix5:correlacionf}, será necesario localizar los polos en el plano complejo. Según la ecuación \eqref{apendix5:pade}, los polos asociados a la aproximación de Padé se encuentran en $\omega = \pm i \xi_{l}/\beta + \mu$, mientras que los polos correspondientes a la forma Lorentziana se ubican en $\omega = \pm i W + \epsilon_{0}$. Para realizar esta integración, se utilizará el teorema del residuo \cite{riley2006mathematical}, considerando un contorno de integración en forma de semicírculo en el hemisferio superior o inferior del plano complejo, dependiendo del signo de $\sigma$. Así, el residuo asociado a la Lorentziana es

\begin{align*}
   \text{Res} \left[ \frac{e^{ i\sigma \omega t} f_{F}[\sigma \beta (\omega-\mu)] }{ (\omega-\epsilon_{0} + iW)(\omega -\epsilon_{0} -iW)}\right]_{\omega = \pm i W + \epsilon_{0}} = \frac{1}{\pm 2iW} ( e^{\mp \sigma Wt} f_{F}[\sigma \beta(\pm iW +\epsilon_{0}-\mu)]e^{i\sigma \epsilon_{0}t}). 
\end{align*}
Si se cumple que $\sigma > 0$, se toma como contorno de integración el hemisferio inferior del plano complejo; en cambio, si $\sigma < 0$, se considera el hemisferio superior. De esta manera, se obtiene

\begin{align*}
    \text{Res} \left[ \frac{e^{ i\sigma \omega t} f_{F}[\sigma \beta (\omega-\mu)] }{ (\omega-\epsilon_{0} + iW)(\omega -\epsilon_{0} -iW)}\right]_{\omega = \pm i W + \epsilon_{0}} = \frac{1}{ 2iW} ( e^{i\sigma \epsilon_{0}t}e^{- Wt} f_{F}[i\beta W + \sigma \beta(\epsilon_{0}-\mu)]). 
 \end{align*}
Para encontrar el residuo asociado a los polos restantes, se deberá calcular

\begin{align*}
    \text{Res} \left[ \frac{-e^{i\sigma \omega t}}{(\omega - \epsilon_{0})^{2} + W^{2} } \frac{ 2K_{l}[\sigma \beta (\omega-\mu)] }{ (\beta(\omega-\mu) + i \xi_{l} )(\beta(\omega-\mu) - i \xi_{l})} \right]_{\omega = \pm i \xi_{l}/\beta + \mu} & = \frac{-1}{\beta} \frac{e^{- \frac{\sigma \xi_{l}}{\beta}t} e^{i\sigma \mu t}(\pm K_{l}\sigma )}{ [ \frac{ \pm i\xi_{l}}{\beta} + (\mu - \epsilon_{0}) ]^{2} + W^{2} }.
\end{align*}
Al seleccionar la región de integración en función del signo de $\sigma$ como se hizo anteriormente, se concluye que

\begin{align*}
    \text{Res} \left[ \frac{-e^{i\sigma \omega t}}{(\omega - \epsilon_{0})^{2} + W^{2} } \frac{ 2K_{l}[\sigma \beta (\omega-\mu)] }{ (\beta(\omega-\mu) + i \xi_{l} )(\beta(\omega-\mu) - i \xi_{l})} \right]_{\omega = \pm i \xi_{l}/\beta + \mu} & = \frac{-1}{\beta} \frac{e^{- \frac{ \xi_{l}}{\beta}t} e^{i\sigma \mu t} K_{l} }{ [ \frac{ \sigma i\xi_{l}}{\beta} + (\mu - \epsilon_{0}) ]^{2} + W^{2} }. 
\end{align*}

Al encontrar los residuos explícitamente, estos se podrán usar para escribir la función correlación 

\begin{equation}
    C^{\sigma}_{jf}(t) \approx \sum_{l=0}^{N} \eta^{\sigma,l} e^{-\gamma_{\sigma,l} t},
    \label{correlationf}
\end{equation}
donde 

\begin{equation*}
    \eta^{\sigma,l} = \left\{ \begin{array}{lc} \frac{\kappa_{f}W}{2} f_{F}[i\beta W + \sigma \beta (\epsilon_{0}-\mu)]  & l = 0 \\ \\ - \frac{iK_{l}}{\beta} \left(\frac{\kappa_{f}W^{2}}{ (\frac{i\sigma \xi_{l}}{\beta} + (\mu-\epsilon_{0}))^{2} + W^{2} } \right) &  l \neq 0 \end{array} \right.
\end{equation*}

\begin{equation*}
    \gamma_{\sigma,l} =  \left\{ \begin{array}{lc} W- \sigma i \epsilon_{0}  & l = 0 \\ \\ \frac{\xi_{l}}{\beta} - \sigma i \mu &  l \neq 0 \end{array} \right..
\end{equation*}
Esto permite obtener una expresión numérica para la función de correlación $C^{\sigma}_{jf}(t)$, y con ello evaluar su tiempo de correlación característico, $\tau_{Bf}$. Por otro lado, para la otra función de correlación es posible calcular la integral de forma analítica, es decir, evaluar

\begin{align*}
       C_{j0}^{\sigma}(t) = \frac{\kappa_{0}}{2\pi} \int_{-\infty}^{\infty} d\omega e^{i\sigma \omega t }f_{F}(\sigma \beta (\omega -  \mu)).
\end{align*}    
Se puede  observar que este cálculo equivale a obtener la transformada de Fourier 

\[
\mathcal{F}\big(f[\sigma (\omega - \mu)]\big)(-\sigma t),
\]
de la función 

\[
f(\sigma(\omega-\mu)) = f_{F}(\sigma \beta (\omega-\mu)),
\]
empleando la propiedad

\begin{equation}
\mathcal{F}(f[\sigma (\omega - \mu)])(-\sigma t) = e^{-i \sigma \mu t}\mathcal{F}(f[\sigma \omega])(-\sigma t).
\label{apendiceprop}
\end{equation}
Usando la ecuación \eqref{apendiceprop} podemos calcular la transformación

\begin{align*}
    \mathcal{F}(f[\sigma \omega])(-\sigma t) &= \frac{\kappa_{0}}{2\pi} \int_{-\infty}^{\infty} d\omega \frac{e^{i\sigma \omega t}}{e^{\sigma \beta \omega} +1 } \\
        & = \frac{\kappa_{0}}{4\pi} \left[\int_{-\infty}^{\infty}d \omega e^{i\sigma \omega t} - \int_{-\infty}^{\infty}d\omega e^{i\sigma \omega t} \tanh \left(\frac{ \sigma \beta \omega }{2} \right)   \right] \\
        & =  \frac{\kappa_{0}}{2} \left[\delta(t) - \frac{i}{\beta \sigma \sinh(\pi t/\beta \sigma)} \right],
\end{align*}
así

\begin{equation*}
    C^{\sigma}_{j0}(t) = \frac{\kappa_{0}}{2}e^{-i\sigma \mu t} \left[\delta(t) - \frac{i}{\beta \sigma \sinh(\pi t/\beta \sigma)} \right].
\end{equation*}
Gracias a que se dispone de una expresión analítica para $C^{\sigma}_{j0}(t)$, es posible determinar el tiempo de correlación $\tau_{B0}$. El análisis se centra principalmente en el término que involucra $\sinh(\pi t/\beta \sigma)$, ya que, dado que $\sigma = \pm 1$, el parámetro relevante es $\pi t/\beta$. Para tiempos $t > \beta$, se puede aproximar que 

\[
\frac{1}{\sinh(\pi t/\beta \sigma)} \propto e^{-t/\beta},
\]
lo que implica que el tiempo de correlación del baño está dado por el inverso de la temperatura, es decir, $\tau_{B0} = \beta$. Por consiguiente, el tiempo de correlación efectivo asociado a un único reservorio queda definido como

\[
\tau_{B} = \max \{\beta, \tau_{Bf}\}.
\]

\label{appendix5correlation}

\subsection{Cálculo numérico para los tiempos de correlación}
\label{appendix5tauf}
Dado que es posible reconstruir las tasas de túnel $\kappa_{L}(\omega)$ y $\kappa_{R}(\omega)$ mediante una función de tipo Lorentziana, se pueden construir perfiles que satisfacen las condiciones $\kappa_{L}(\epsilon) = \kappa_{R}(\epsilon+U) = 1/100$ y $\kappa_{L}(\epsilon+U) = \kappa_{R}(\epsilon) = 1/600$. Con ello es posible calcular numéricamente las funciones de correlación asociadas.  
\\

Para determinar los tiempos de correlación $\tau_{Bf}$ correspondientes a los baños $L$ y $R$, se emplea la ecuación \eqref{correlationf}. La estimación de $\tau_{Bf}$ se realiza de manera numérica, considerando como referencia el instante en el que las funciones de correlación se vuelven despreciables. Posteriormente, para verificar si el sistema puede describirse mediante una aproximación markoviana, se contrasta con el tiempo característico de relajación $\tau_{R}$, exigiendo que se cumpla $\tau_{R} \gg \tau_{Bf}$.  

\insertimage[\label{img:correlacionminus}]{ejemplos/cminusreal.pdf}{scale=0.5}{(a) Parte real de las funciones de correlación $\sigma=-$ del baño $L$ en función del tiempo. (b) Parte real de las funciones de correlación $\sigma=-$ del baño $R$ en función del tiempo.}
\insertimage[\label{img:correlacionminusimag}]{ejemplos/cminusimag.pdf}{scale=0.5}{(a) Parte imaginaria de las funciones de correlación $\sigma=-$ del baño $L$ en función del tiempo. (b) Parte imaginaria de las funciones de correlación $\sigma=-$ del baño $R$ en función del tiempo.}

\insertimage[\label{img:correlacionplus}]{ejemplos/cplusreal.pdf}{scale=0.5}{(a) Parte real de las funciones de correlación $\sigma=+$ del baño $L$ en función del tiempo. (b) Parte real de las funciones de correlación $\sigma=+$ del baño $R$ en función del tiempo.}
\insertimage[\label{img:correlacionplusimag}]{ejemplos/cplusimag.pdf}{scale=0.5}{(a) Parte imaginaria de las funciones de correlación $\sigma=+$ del baño $L$ en función del tiempo. (b) Parte imaginaria de las funciones de correlación $\sigma=+$ del baño $R$ en función del tiempo.}

A partir de las Figuras \ref{img:correlacionminus}, \ref{img:correlacionminusimag}, \ref{img:correlacionplus} y \ref{img:correlacionplusimag}, se observa que, para $t > 0.4$, las funciones de correlación correspondientes a ambos baños se vuelven prácticamente nulas.  
\\

Con el fin de contrastar esta escala temporal con la dinámica del sistema, en la Figura \ref{img:probas} se muestra la evolución temporal de ciertas componentes de la matriz densidad, lo que permite estimar el tiempo de relajación hacia el estado estacionario.  

\insertimage[\label{img:probas}]{ejemplos/rhotiempo.pdf}{scale=0.56}{Componentes $\rho_{100}$ y $\rho_{010}$ de la matriz densidad en función del tiempo, para distintos valores de $eV/T$.}

De la Figura \ref{img:probas} se obtiene que el tiempo de relajación es del orden de $\tau_{R} > 2000$. Así, para distintos valores de $eV/T$, se cumple que $\tau_{Bf}/\tau_{R} < 0.0002$. Esto demuestra que la separación de escalas es suficiente y, por lo tanto, queda justificado el uso de la aproximación markoviana en la descripción de las funciones de correlación de los baños $L$ y $R$.  


\subsection{Concurrencia}
En esta subsección se presenta la deducción de la fórmula de la concurrencia. Para ello, se calcula la concurrencia en el subsistema $\hat{\rho}_{LR}$, lo que requiere, como primer paso, obtener  

\begin{equation*}
    \hat{\rho}_{LR} = \text{Tr}_{D}\{ \hat{\rho}_{S} \}.
\end{equation*}
La matriz densidad $\hat{\rho}_{S}$ se expresará en la base  
\[
\left\{ |0,0,0\rangle,\ |1,0,0\rangle,\ |0,1,0\rangle,\ |0,0,1\rangle,\ |1,1,0\rangle,\ |1,0,1\rangle,\ |0,1,1\rangle,\ |1,1,1\rangle \right\}.
\]
En esta representación, $\hat{\rho}_{S}$ toma la forma

\begin{equation}
    \hat{\rho}_{S} = 
    \begin{bmatrix}
        \rho_{000} & 0 & 0 & 0 & 0 & 0 & 0 & 0 \\
        0 & \rho_{100} & \alpha & 0 & 0 & 0 & 0 & 0 \\
        0 & \alpha^{*} & \rho_{010} & 0 & 0 & 0 & 0 & 0 \\
        0 & 0 & 0 & \rho_{001} & 0 & 0 & 0 & 0 \\
        0 & 0 & 0 & 0 & \rho_{110} & 0 & 0 & 0 \\
        0 & 0 & 0 & 0 & 0 & \rho_{101} & \beta & 0 \\
        0 & 0 & 0 & 0 & 0 & \beta^{*} & \rho_{011} & 0 \\
        0 & 0 & 0 & 0 & 0 & 0 & 0 & \rho_{111} 
        \end{bmatrix}.
        \label{appendix5rhoconcu}
\end{equation}

En esta matriz densidad no aparecen coherencias entre estados con distinto número total de partículas, lo que se explica por el principio de superselección de carga \cite{bartlett2007reference,wick1997intrinsic}. Asimismo, dado que el punto cuántico $D$ sólo interactúa con los puntos $L$ y $R$ mediante una interacción de Coulomb, no se generan coherencias del tipo $L$-$D$ ni $R$-$D$. 
\\

Al proyectar sobre la base $\{|00\rangle, |10\rangle, |01\rangle, |11\rangle\}$ se obtiene la matriz reducida del subsistema $LR$,

\begin{equation*}
    \hat{\rho}_{LR} = 
    \begin{bmatrix}
        \rho_{000}+\rho_{001} & 0 & 0 & 0  \\
        0 & \rho_{100} + \rho_{101} & \alpha + \beta & 0  \\
        0 & \alpha^{*} + \beta^{*} & \rho_{010} + \rho_{011} & 0  \\
        0 & 0 & 0 & \rho_{110} + \rho_{111} 
        \end{bmatrix}.
\end{equation*}
Con esta reducción, la concurrencia se define como

\begin{equation*}
    \mathcal{C}_{on} = \max \{ 0,\lambda_{1} - \lambda_{2} - \lambda_{3} - \lambda_{4} \},
\end{equation*}
donde los $\lambda_{i}$ corresponden a las raíces cuadradas de los autovalores de la matriz

\begin{equation}
    B = \hat{\rho}_{LR} \tilde{\rho}_{LR}, 
    \hspace{10mm} 
    \tilde{\rho}_{LR} = (\sigma_{y} \otimes \sigma_{y}) \hat{\rho}_{LR} (\sigma_{y} \otimes \sigma_{y}),
    \label{apendixconcu}
\end{equation}
y $\sigma_{y}$ es la matriz de Pauli correspondiente.  
\\

Al aplicar la ecuación \eqref{apendixconcu} se obtiene

\begin{equation*}
    \tilde{\rho}_{LR} = 
    \begin{bmatrix}
        p_{D} & 0 & 0 & 0  \\
        0 & p_{R} & \alpha + \beta & 0  \\
        0 & \alpha^{*} + \beta^{*} & p_{L} & 0  \\
        0 & 0 & 0 & p_{0} 
        \end{bmatrix},
\end{equation*}
con  $p_{0}=\rho_{000}+\rho_{001}$, $p_{L}=\rho_{100} + \rho_{101}$, $p_{R}=\rho_{010}+\rho_{011}$ y $p_{D}=\rho_{110}+\rho_{111}$. Así, la matriz $B$ resulta

\begin{equation*}
    B = 
    \begin{bmatrix}
        p_{0}p_{D} & 0 & 0 & 0  \\
        0 & p_{L}p_{R}+|\alpha+\beta|^{2} & 2(\alpha + \beta)p_{L} & 0  \\
        0 & 2(\alpha^{*} + \beta^{*})p_{R} & p_{L}p_{R}+|\alpha+\beta|^{2}  & 0  \\
        0 & 0 & 0 & p_{0}p_{D}
        \end{bmatrix}.
\end{equation*}
En el régimen en que $p_{D}$ y $p_{0}$ son pequeños, el orden decreciente de los autovalores queda dado por

\begin{equation*}
    \lambda_{1} = p_{L}p_{R} + |\alpha + \beta|, 
    \hspace{10mm}  
    \lambda_{2} = p_{L}p_{R} - |\alpha + \beta|, 
    \hspace{10mm}  
    \lambda_{3}=\lambda_{4} = \sqrt{p_{0}p_{D}}.
\end{equation*}
Finalmente, la concurrencia del subsistema $LR$ se expresa como

\begin{equation*}
    \mathcal{C}_{on} = \max\{ 2|\alpha+\beta| - 2\sqrt{p_{0}p_{D}},0\}.
\end{equation*}

\label{appendix5final}

\subsection{Modelo clásico}
Para derivar el modelo clásico que gobierna las componentes diagonales de la matriz densidad
$\rho_{ijk}$, con $i,j,k\in\{0,1\}$, partimos de la ecuación de evolución proyectada de
Nakajima–Zwanzig para el complemento $\check{\mathcal{Q}}=\mathbf{1}-\check{\mathcal{P}}$:
\begin{equation*}
    \frac{d}{dt}\check{\mathcal{Q}}|\hat{\rho}(t)\rangle\rangle
    = \check{\mathcal{Q}}\check{\mathcal{L}}_{f}\check{\mathcal{Q}}|\hat{\rho}(t)\rangle\rangle
      + \check{\mathcal{Q}}\check{\mathcal{L}}_{f}\check{\mathcal{P}}|\hat{\rho}(t)\rangle\rangle
    \;\;\Longrightarrow\;\;
    \frac{d}{dt}\check{\mathcal{Q}}|\hat{\rho}(t)\rangle\rangle
      - \check{\mathcal{Q}}\check{\mathcal{L}}_{f}\check{\mathcal{Q}}|\hat{\rho}(t)\rangle\rangle
    = \check{\mathcal{Q}}\check{\mathcal{L}}_{f}\check{\mathcal{P}}|\hat{\rho}(t)\rangle\rangle,
\end{equation*}
que es una ecuación lineal no homogénea del tipo
\(
\frac{dy}{dt}-A(t)y=B(t)
\).
La cual puede resolverse mediante un factor integrante
\begin{equation*}
    \check{\mu}(t)=\exp\!\left(-\int_{0}^{t}\check{\mathcal{Q}}\check{\mathcal{L}}_{f}\,ds\right).
\end{equation*}
Multiplicando por $\check{\mu}(t)$ y usando la regla del producto:
\begin{equation*}
    \check{\mu}(t)\frac{d}{dt}\check{\mathcal{Q}}|\hat{\rho}(t)\rangle\rangle
    -\check{\mu}(t)\check{\mathcal{Q}}\check{\mathcal{L}}_{f}\check{\mathcal{Q}}|\hat{\rho}(t)\rangle\rangle
    =\check{\mu}(t)\check{\mathcal{Q}}\check{\mathcal{L}}_{f}\check{\mathcal{P}}|\hat{\rho}(t)\rangle\rangle
    \;\;\Longrightarrow\;\;
    \frac{d}{dt}\!\left[\check{\mu}(t)\check{\mathcal{Q}}|\hat{\rho}(t)\rangle\rangle\right]
    =\check{\mu}(t)\check{\mathcal{Q}}\check{\mathcal{L}}_{f}\check{\mathcal{P}}|\hat{\rho}(t)\rangle\rangle,
\end{equation*}
con solución
\begin{equation}
    \check{\mu}(t)\check{\mathcal{Q}}|\hat{\rho}(t)\rangle\rangle
    =\check{\mu}(0)\check{\mathcal{Q}}|\hat{\rho}(0)\rangle\rangle
     +\int_{0}^{t}\check{\mu}(\tau)\check{\mathcal{Q}}\check{\mathcal{L}}_{f}\check{\mathcal{P}}
        |\hat{\rho}(\tau)\rangle\rangle\,d\tau .
    \label{apendixclasic1}
\end{equation}
A continuación, multiplicamos la ecuación \eqref{apendixclasic1} por $\check{\mu}^{-1}(t)$ y se obtiene
\begin{equation}
    \check{\mathcal{Q}}|\hat{\rho}(t)\rangle\rangle
    =\check{\mu}^{-1}(t)\check{\mathcal{Q}}|\hat{\rho}(0)\rangle\rangle
     +\int_{0}^{t}\check{\mu}^{-1}(t)\check{\mu}(\tau)\check{\mathcal{Q}}\check{\mathcal{L}}_{f}\check{\mathcal{P}}
     |\hat{\rho}(\tau)\rangle\rangle\,d\tau .
    \label{apendixclasic2}
\end{equation}


Definimos el propagador restringido
\begin{equation*}
    \check{\mathcal{G}}(t,s)=\exp\!\left(\int_{s}^{t}\check{\mathcal{Q}}\check{\mathcal{L}}_{f}\,dt'\right),
\end{equation*}
con lo cual la ecuación \eqref{apendixclasic2} se reescribe como
\begin{equation*}
    \check{\mathcal{Q}}|\hat{\rho}(t)\rangle\rangle
    =\check{\mathcal{G}}(t,0)\check{\mathcal{Q}}|\hat{\rho}(0)\rangle\rangle
     +\int_{0}^{t}\check{\mathcal{G}}(t,\tau)\check{\mathcal{Q}}\check{\mathcal{L}}_{f}\check{\mathcal{P}}
        |\hat{\rho}(\tau)\rangle\rangle\,d\tau .
\end{equation*}
Insertando esta expresión en la ecuación para el subespacio proyectado obtenemos
\begin{equation}
    \frac{d}{dt}\check{\mathcal{P}}|\hat{\rho}(t)\rangle\rangle
    =\check{\mathcal{P}}\check{\mathcal{L}}_{f}\check{\mathcal{P}}|\hat{\rho}(t)\rangle\rangle
     +\check{\mathcal{P}}\check{\mathcal{L}}_{f}\check{\mathcal{G}}(t,0)\check{\mathcal{Q}}|\hat{\rho}(0)\rangle\rangle
     +\check{\mathcal{P}}\check{\mathcal{L}}_{f}\!\int_{0}^{t}\!\check{\mathcal{G}}(t,\tau)\check{\mathcal{Q}}\check{\mathcal{L}}_{f}\check{\mathcal{P}}
       |\hat{\rho}(\tau)\rangle\rangle\,d\tau .
    \label{apendix5exactp}
\end{equation}
La identidad anterior es exacta: el primer término describe la parte markoviana, el segundo un
corrimiento inicial y el tercero los efectos de memoria.
\\

Usaremos ahora las propiedades
\begin{equation*}
    \check{\mathcal{P}}\check{\mathcal{V}}\check{\mathcal{P}}=0,
    \qquad
    [\check{\mathcal{L}}_{0},\check{\mathcal{P}}]=[\check{\mathcal{L}}_{0},\check{\mathcal{Q}}]=0,
\end{equation*}
donde $\check{\mathcal{L}}_{f}=\check{\mathcal{L}}_{0}+\check{\mathcal{V}}$. Intuitivamente,
$\check{\mathcal{V}}$ (el acoplamiento que genera túnel coherente) lleva estados diagonales fuera
del subespacio diagonal; por eso, $\check{\mathcal{P}}\check{\mathcal{V}}\check{\mathcal{P}}=0$.
\\

Supondremos además un estado inicial sin coherencias:
$\check{\mathcal{Q}}|\hat{\rho}(0)\rangle\rangle=0$.
Escribimos entonces
\begin{align*}
    \check{\mathcal{G}}(t,\tau)
    &=\exp\!\left(\int_{\tau}^{t}\check{\mathcal{Q}}(\check{\mathcal{L}}_{0}+\check{\mathcal{V}})\,dt'\right)
      =\exp\!\left(\int_{\tau}^{t}\big[\check{\mathcal{Q}}\check{\mathcal{L}}_{0}\check{\mathcal{Q}}
      +\check{\mathcal{Q}}\check{\mathcal{V}}\check{\mathcal{P}}
      +\check{\mathcal{Q}}\check{\mathcal{V}}\check{\mathcal{Q}}\big]\,dt'\right),
\end{align*}
usando que $\check{\mathcal{Q}}\check{\mathcal{L}}_{0}\check{\mathcal{P}}=\check{\mathcal{Q}}\check{\mathcal{P}}\check{\mathcal{L}}_{0}=0$.
\\

La ecuación proyectada principal toma la forma
\begin{equation*}
    \frac{d}{dt}\check{\mathcal{P}}|\hat{\rho}(t)\rangle\rangle
    =\check{\mathcal{P}}\check{\mathcal{L}}_{0}\check{\mathcal{P}}|\hat{\rho}(t)\rangle\rangle
     +\check{\mathcal{P}}(\check{\mathcal{L}}_{0}+\check{\mathcal{V}})(\check{\mathcal{P}}+\check{\mathcal{Q}})
      \int_{0}^{t}\!\check{\mathcal{G}}(t,\tau)\check{\mathcal{Q}}\check{\mathcal{L}}_{f}\check{\mathcal{P}}
      |\hat{\rho}(\tau)\rangle\rangle\,d\tau .
\end{equation*}
Como el exponencial en $\check{\mathcal{G}}$ está restringido a $\check{\mathcal{Q}}$, al actuar
$\check{\mathcal{P}}\check{\mathcal{L}}_{0}$ se reduce a una contribución proporcional a $(t-\tau)$
que se anula por la idempotencia de los proyectores en la cadena resultante; por lo tanto,
permanece
\begin{equation*}
    \frac{d}{dt}\check{\mathcal{P}}|\hat{\rho}(t)\rangle\rangle
    =\check{\mathcal{P}}\check{\mathcal{L}}_{0}\check{\mathcal{P}}|\hat{\rho}(t)\rangle\rangle
     +\check{\mathcal{P}}\check{\mathcal{V}}\check{\mathcal{Q}}
      \int_{0}^{t}\!\check{\mathcal{G}}(t,\tau)\check{\mathcal{Q}}\check{\mathcal{V}}\check{\mathcal{P}}
      |\hat{\rho}(\tau)\rangle\rangle\,d\tau .
\end{equation*}
Además, el término $\check{\mathcal{Q}}\check{\mathcal{V}}\check{\mathcal{P}}$ dentro del
exponencial no contribuye en esta cadena, con lo cual
\begin{equation*}
    \frac{d}{dt}\check{\mathcal{P}}|\hat{\rho}(t)\rangle\rangle
    =\check{\mathcal{P}}\check{\mathcal{L}}_{0}\check{\mathcal{P}}|\hat{\rho}(t)\rangle\rangle
     +\check{\mathcal{P}}\check{\mathcal{V}}\check{\mathcal{Q}}
      \int_{0}^{t}\!\exp\!\Big[(\check{\mathcal{Q}}\check{\mathcal{V}}\check{\mathcal{Q}}
      +\check{\mathcal{Q}}\check{\mathcal{L}}_{0}\check{\mathcal{Q}})(t-\tau)\Big]
      \check{\mathcal{Q}}\check{\mathcal{V}}\check{\mathcal{P}}
      |\hat{\rho}(\tau)\rangle\rangle\,d\tau .
\end{equation*}
Haciendo el cambio $\tau\mapsto t-\tau$, vemos que 
\begin{equation*}
    \frac{d}{dt}\check{\mathcal{P}}|\hat{\rho}(t)\rangle\rangle
    =\check{\mathcal{P}}\check{\mathcal{L}}_{0}\check{\mathcal{P}}|\hat{\rho}(t)\rangle\rangle
     +\check{\mathcal{P}}\check{\mathcal{V}}\check{\mathcal{Q}}
      \int_{0}^{t}\!\exp\!\Big[(\check{\mathcal{Q}}\check{\mathcal{V}}\check{\mathcal{Q}}
      +\check{\mathcal{Q}}\check{\mathcal{L}}_{0}\check{\mathcal{Q}})\tau\Big]
      \check{\mathcal{Q}}\check{\mathcal{V}}\check{\mathcal{P}}
      |\hat{\rho}(t-\tau)\rangle\rangle\,d\tau .
\end{equation*}


Supongamos acoplamiento coherente débil frente a la disipación,
$g\ll \kappa_i$, de modo que los autovalores de
$\check{\mathcal{V}}$ son $\mathcal{O}(g)$ mientras que los de
$\check{\mathcal{L}}_{0}$ son $\mathcal{O}(\kappa_i)$. En esa jerarquía,
se desprecia $\check{\mathcal{V}}$ en el exponencial,
\[
\exp\!\big[(\check{\mathcal{Q}}\check{\mathcal{V}}\check{\mathcal{Q}}
+\check{\mathcal{Q}}\check{\mathcal{L}}_{0}\check{\mathcal{Q}})\tau\big]
\;\approx\;
\exp\!\big[(\check{\mathcal{Q}}\check{\mathcal{L}}_{0}\check{\mathcal{Q}})\tau\big].
\]
Además, al realizar la aproximación de Markov se aplican dos pasos. Primero,
$\check{\mathcal{P}}|\hat{\rho}(t-\tau)\rangle\rangle\simeq
\check{\mathcal{P}}|\hat{\rho}(t)\rangle\rangle$ para $\tau\lesssim 1/\kappa_i$,
y segundo, extiende el límite superior de integración a $\infty$:
\begin{equation*}
    \frac{d}{dt}\check{\mathcal{P}}|\hat{\rho}(t)\rangle\rangle
    =\check{\mathcal{P}}\check{\mathcal{L}}_{0}\check{\mathcal{P}}|\hat{\rho}(t)\rangle\rangle
     +\check{\mathcal{P}}\check{\mathcal{V}}\check{\mathcal{Q}}
      \int_{0}^{\infty}\!e^{(\check{\mathcal{Q}}\check{\mathcal{L}}_{0}\check{\mathcal{Q}})\tau}\,d\tau\;
      \check{\mathcal{Q}}\check{\mathcal{V}}\check{\mathcal{P}}
      |\hat{\rho}(t)\rangle\rangle .
\end{equation*}

Así, se encuentra el inverso de Drazin del superoperador $\check{\mathcal{L}}_{0}$ como
\[
\check{\mathcal{L}}_{0}^{+}
=\int_{0}^{\infty}\!e^{(\check{\mathcal{Q}}\check{\mathcal{L}}_{0}\check{\mathcal{Q}})\tau}\,d\tau,
\]
y obtenemos finalmente la ecuación para la diagonal de la matriz densidad:
\begin{equation*}
    \frac{d}{dt}\check{\mathcal{P}}|\hat{\rho}(t)\rangle\rangle
    =\big(\check{\mathcal{L}}_{0}
      -\check{\mathcal{P}}\check{\mathcal{V}}\check{\mathcal{Q}}
       \check{\mathcal{L}}_{0}^{+}\check{\mathcal{Q}}\check{\mathcal{V}}\check{\mathcal{P}}\big)\;
      \check{\mathcal{P}}|\hat{\rho}(t)\rangle\rangle .
\end{equation*}

\label{appendix5clasic}

\subsection{Inverso de Drazin}
Consideremos $\check{\mathcal{L}}_{0}$ como un generador lineal (tipo Liouvilliano) asociado a la evolución
``libre'' del sistema reducido. Su espectro contiene un autovalor nulo, correspondiente al estado
estacionario, y el resto de los autovalores tienen parte real no positiva. Denotemos por
\[
\check{\mathcal{P}} = |x_{0}\rangle\rangle\langle\langle y_{0}|, \qquad
\check{\mathcal{Q}} = \mathbf{1}-\check{\mathcal{P}},
\]
el proyector sobre el subespacio estacionario (núcleo de $\check{\mathcal{L}}_{0}$) y su complemento,
respectivamente, donde $|x_{0}\rangle\rangle$ es el estado estacionario (por la derecha) y
$\langle\langle y_{0}|$ es el autoestado por la izquierda, de modo que cumplen
$\langle\langle y_{0}|x_{0}\rangle\rangle=1$.
\\

En estas condiciones, el \emph{inverso de Drazin} $\check{\mathcal{L}}_{0}^{+}$ es el único operador que satisface
\begin{align*}
    \check{\mathcal{L}}_{0}\check{\mathcal{L}}_{0}^{+}\check{\mathcal{L}}_{0} &= \check{\mathcal{L}}_{0},\\
    \check{\mathcal{L}}_{0}^{+}\check{\mathcal{L}}_{0}\check{\mathcal{L}}_{0}^{+} &= \check{\mathcal{L}}_{0}^{+},\\
    \check{\mathcal{L}}_{0}\check{\mathcal{L}}_{0}^{+} &= \check{\mathcal{L}}_{0}^{+}\check{\mathcal{L}}_{0} = \check{\mathcal{Q}}.
\end{align*}
Bajo las hipótesis anteriores, una forma útil  de definir
$\check{\mathcal{L}}_{0}^{+}$ es
\begin{equation}
    \check{\mathcal{L}}_{0}^{+} \;=\; \int_{0}^{\infty} e^{\,\check{\mathcal{L}}_{0}\tau}\,\check{\mathcal{Q}}\, d\tau .
    \label{eq:DrazinIntegral}
\end{equation}
Esta integral converge porque $e^{\check{\mathcal{L}}_{0}\tau}$ decae exponencialmente en el
subespacio $\check{\mathcal{Q}}$ (todas las autofrecuencias allí tienen parte real negativa o nula
estrictamente menor que la del modo estacionario). De la ecuación \eqref{eq:DrazinIntegral} se prueba directamente que
\[
\check{\mathcal{L}}_{0}\check{\mathcal{L}}_{0}^{+}
= \int_{0}^{\infty} \check{\mathcal{L}}_{0} e^{\check{\mathcal{L}}_{0}\tau}\,\check{\mathcal{Q}}\, d\tau
= \Big[ e^{\check{\mathcal{L}}_{0}\tau}\,\check{\mathcal{Q}}\Big]_{0}^{\infty}
= \mathbf{0}-\check{\mathcal{Q}} = \check{\mathcal{Q}},
\]
y análogamente $\check{\mathcal{L}}_{0}^{+}\check{\mathcal{L}}_{0}=\check{\mathcal{Q}}$.
\\

Si $\check{\mathcal{L}}_{0}$ es diagonalizable (o al menos tiene base de vectores generalizados) con
autovectores (generalizados) derechos $|x_{j}\rangle\rangle$ e izquierdos $\langle\langle y_{j}|$:
\[
\check{\mathcal{L}}_{0}|x_{j}\rangle\rangle=\lambda_{j}|x_{j}\rangle\rangle,\qquad
\langle\langle y_{j}|\check{\mathcal{L}}_{0}=\lambda_{j}\langle\langle y_{j}|,\qquad
\langle\langle y_{i}|x_{j}\rangle\rangle=\delta_{ij},
\]
con $\lambda_{0}=0$, entonces
\begin{equation*}
    \check{\mathcal{L}}_{0}
    = \sum_{j\neq 0}\lambda_{j}\,|x_{j}\rangle\rangle\langle\langle y_{j}|, \qquad
    \check{\mathcal{L}}_{0}^{+}
    = \sum_{j\neq 0}\frac{1}{\lambda_{j}}\,|x_{j}\rangle\rangle\langle\langle y_{j}|.
\end{equation*}

\label{appendix5drazin}


\subsection{Expresión para los flujos de información}
Para calcular los flujos de información es necesario, en primer lugar, determinar $\ln \hat{\rho}_{S}$. Partimos de la matriz densidad dada por la ecuación \eqref{appendix5rhoconcu}, la cual es diagonal salvo por dos bloques no diagonales que deben tratarse por separado:
\[
\begin{bmatrix}
    \rho_{100} & \alpha \\
    \alpha^{*} & \rho_{010}
\end{bmatrix}
\hspace{10mm}
\begin{bmatrix}
    \rho_{101} & \beta \\
    \beta^{*} & \rho_{011}
\end{bmatrix}.
\]
Sus autovalores son
\[
\lambda_{0\pm} = \frac{ \rho_{100} +\rho_{010} }{2} \pm \frac{( [\rho_{100} - \rho_{010}]^{2} + 4|\alpha|^{2} )^{1/2} }{ 2 }, 
\qquad
\lambda_{1\pm} = \frac{ \rho_{101} +\rho_{011} }{2} \pm \frac{( [\rho_{101} - \rho_{011}]^{2} + 4|\beta|^{2} )^{1/2} }{ 2 }.
\]
Escribiendo $\alpha = |\alpha|e^{i\phi_{0}}$ y $\beta = |\beta|e^{i\phi_{1}}$, los autovectores se pueden parametrizar como
\begin{align*}
    v_{0+} &  = [0,0,\cos(\theta_{0}/2)e^{i\phi_{0}/2},\sen(\theta_{0}/2)e^{-i\phi_{0}/2},0,0,0,0]^{T}, \\
    v_{0-} &  = [0,0,-\sen(\theta_{0}/2)e^{i\phi_{0}/2},\cos(\theta_{0}/2)e^{-i\phi_{0}/2},0,0,0,0]^{T}, \\
    v_{1+} &  = [0,0,0,0,\cos(\theta_{1}/2)e^{i\phi_{1}/2},\sen(\theta_{1}/2)e^{-i\phi_{1}/2},0,0]^{T}, \\
    v_{1-} &  = [0,0,0,0,-\sen(\theta_{1}/2)e^{i\phi_{1}/2},\cos(\theta_{1}/2)e^{-i\phi_{1}/2},0,0]^{T},
\end{align*}
con $\sen(\theta_{0}) = |\alpha|/\sqrt{\Delta^{2}_{0} + |\alpha|^{2}}$, $\sen(\theta_{1}) = |\beta|/\sqrt{\Delta^{2}_{1} + |\beta|^{2}}$ y
\[
\Delta_{i} = \frac{(\rho_{10i} - \rho_{01i})}{2}.
\]
En estas condiciones, el logaritmo de la matriz densidad se obtiene mediante la descomposición 
\[
\ln \hat{\rho}_{S} = \hat{V}\,\ln \hat{\rho}_{D}\, \hat{V}^{-1},
\]
donde la matriz diagonal es
\[
\ln \hat{\rho}_{D} =
\begin{bmatrix}
    \ln \rho_{000} & 0 & 0 & 0 & 0 & 0 & 0 & 0 \\
    0 & \ln \lambda_{0+} & 0 & 0 & 0 & 0 & 0 & 0 \\
    0 & 0 & \ln \lambda_{0-} & 0 & 0 & 0 & 0 & 0 \\
    0 & 0 & 0 & \ln \rho_{001} & 0 & 0 & 0 & 0 \\
    0 & 0 & 0 & 0 & \ln \rho_{110} & 0 & 0 & 0 \\
    0 & 0 & 0 & 0 & 0 & \ln \lambda_{1+} & 0 & 0 \\
    0 & 0 & 0 & 0 & 0 & 0 & \ln \lambda_{1-} & 0 \\
    0 & 0 & 0 & 0 & 0 & 0 & 0 & \ln \rho_{111}
\end{bmatrix},
\]
y la matriz de autovectores
\[
\hat{V} =
\begin{bmatrix}
    1 & 0 & 0 & 0 & 0 & 0 & 0 & 0 \\
    0 & \cos(\theta_{0}/2)e^{i\phi_{0}/2} & -\sen(\theta_{0}/2)e^{i\phi_{0}/2} & 0 & 0 & 0 & 0 & 0 \\
    0 & \sen(\theta_{0}/2)e^{-i\phi_{0}/2} & \cos(\theta_{0}/2)e^{-i\phi_{0}/2} & 0 & 0 & 0 & 0 & 0 \\
    0 & 0 & 0 & 1 & 0 & 0 & 0 & 0 \\
    0 & 0 & 0 & 0 & 1 & 0 & 0 & 0 \\
    0 & 0 & 0 & 0 & 0 &  \cos(\theta_{1}/2)e^{i\phi_{1}/2} & -\sen(\theta_{1}/2)e^{i\phi_{1}/2} & 0 \\
    0 & 0 & 0 & 0 & 0 & \sen(\theta_{1}/2)e^{-i\phi_{1}/2} & \cos(\theta_{1}/2)e^{-i\phi_{1}/2} & 0 \\
    0 & 0 & 0 & 0 & 0 & 0 & 0 & 1
\end{bmatrix}.
\]
Por lo tanto,
\[
\ln \hat{\rho}_{S} =
\begin{bmatrix}
    \ln \rho_{000} & 0 & 0 & 0 & 0 & 0 & 0 & 0 \\
    0 & a_{0} & c_{0} & 0 & 0 & 0 & 0 & 0 \\
    0 & c^{*}_{0} & b_{0} & 0 & 0 & 0 & 0 & 0 \\
    0 & 0 & 0 & \ln \rho_{001} & 0 & 0 & 0 & 0 \\
    0 & 0 & 0 & 0 & \ln \rho_{110} & 0 & 0 & 0 \\
    0 & 0 & 0 & 0 & 0 & a_{1} & c_{1} & 0 \\
    0 & 0 & 0 & 0 & 0 & c^{*}_{1} & b_{1} & 0 \\
    0 & 0 & 0 & 0 & 0 & 0 & 0 & \ln \rho_{111}
\end{bmatrix},
\]
con
\begin{align*}
    a_{i} & = \cos^{2}(\theta_{i}/2)\,\ln \lambda_{i+} + \sen^{2}(\theta_{i}/2)\,\ln \lambda_{i-}, \\
    b_{i} & = \sen^{2}(\theta_{i}/2)\,\ln \lambda_{i+} + \cos^{2}(\theta_{i}/2)\,\ln \lambda_{i-}, \\
    c_{i} & = \frac{\sin(\theta_{i})}{2}e^{i\phi_{i}} \big(\ln \lambda_{i+} - \ln \lambda_{i-}\big).
\end{align*}
Equivalente y de forma explícita,
\begin{align*}
    \ln \hat{\rho}_{S} & =
    \ln \rho_{111}|111\rangle \langle 111| + \ln \rho_{110}|110\rangle \langle 110| + \ln \rho_{001}|001\rangle \langle 001| \\
     &\quad + \ln \rho_{000}|000\rangle \langle 000|
     + a_{0}|100\rangle \langle 100| + b_{0}|010\rangle \langle 010|  \\
     &\quad + a_{1}|101\rangle \langle 101| + b_{1}|011\rangle\langle 011|
     + c_{0}|100\rangle \langle 010| + c^{*}_{0}|010\rangle \langle 100| \\
     &\quad + c_{1}|101\rangle \langle 011| + c^{*}_{1}|011\rangle \langle 101|.
\end{align*}

Las contribuciones de los disipadores en $\mathcal{L}_{L}$ y $\mathcal{L}_{R}$ se expresan, actuando sobre $\hat{\rho}_{S}$, mediante
\begin{align*}
    \mathcal{D}[\hat{d}^{\dagger}_{L}(\textbf{1}-\hat{n}_{D})(\textbf{1}-\hat{n}_{R})\hat{\rho}_{S}] & = \rho_{000}|100\rangle \langle 100| - \rho_{000}|000\rangle \langle 000|, \\ 
    \mathcal{D}[\hat{d}^{\dagger}_{R}(\textbf{1}-\hat{n}_{D})(\textbf{1}-\hat{n}_{L})\hat{\rho}_{S}] & = \rho_{000}|010\rangle \langle 010| - \rho_{000}|000\rangle \langle 000|, \\  
    \mathcal{D}[\hat{d}_{L}(\textbf{1}-\hat{n}_{D})(\textbf{1}-\hat{n}_{R})\hat{\rho}_{S}] & = \rho_{100}|000\rangle \langle 000| - \rho_{100}|100\rangle \langle 100|, \\ 
    \mathcal{D}[\hat{d}_{R}(\textbf{1}-\hat{n}_{D})(\textbf{1}-\hat{n}_{L})\hat{\rho}_{S}] & = \rho_{010}|000\rangle \langle 000| - \rho_{010}|010\rangle \langle 010|, \\ 
    \mathcal{D}[\hat{d}^{\dagger}_{L}\hat{n}_{D} \hat{n}_{R}\hat{\rho}_{S}] & = \rho_{011}|111\rangle \langle 111| - \rho_{011}|011\rangle \langle 011|, \\  
    \mathcal{D}[\hat{d}^{\dagger}_{R} \hat{n}_{D}\hat{n}_{L}\hat{\rho}_{S}] & = \rho_{101}|111\rangle \langle 111| - \rho_{101}|101\rangle \langle 101|, \\ 
     \mathcal{D}[\hat{d}_{L}\hat{n}_{D}\hat{n}_{R}\hat{\rho}_{S}] & = \rho_{111}|011\rangle \langle 011| - \rho_{111}|111\rangle \langle 111|, \\    
      \mathcal{D}[\hat{d}_{R}\hat{n}_{D}\hat{n}_{L}\hat{\rho}_{S}] & = \rho_{111}|101\rangle \langle 101| - \rho_{111}|111\rangle \langle 111|, \\ 
    \mathcal{D}[\hat{d}^{\dagger}_{L}(\textbf{1}-\hat{n}_{D})\hat{n}_{R}\hat{\rho}_{S}] & = \rho_{010}|110\rangle \langle 110| - \rho_{010}|010\rangle \langle 010| - \frac{1}{2}( \alpha|100\rangle \langle 010| + \alpha^{*}|010\rangle \langle 100|), \\ 
    \mathcal{D}[\hat{d}^{\dagger}_{R}(\textbf{1}-\hat{n}_{D})\hat{n}_{L}\hat{\rho}_{S}] & = \rho_{100}|110\rangle \langle 110| - \rho_{100}|100\rangle \langle 100| - \frac{1}{2}( \alpha |100\rangle \langle 010| + \alpha^{*}|010\rangle \langle 100| ), \\  
    \mathcal{D}[\hat{d}_{L}(\textbf{1}-\hat{n}_{D})\hat{n}_{R}\hat{\rho}_{S}] & = \rho_{110}|010\rangle \langle 010| - \rho_{110}|110\rangle \langle 110|, \\ 
    \mathcal{D}[\hat{d}_{R}(\textbf{1}-\hat{n}_{D})\hat{n}_{L}\hat{\rho}_{S}] & = \rho_{110}|100\rangle \langle 100| - \rho_{110}|110\rangle \langle 110|, \\  
    \mathcal{D}[\hat{d}^{\dagger}_{L}\hat{n}_{D}(\textbf{1}-\hat{n}_{R})\hat{\rho}_{S}] & = \rho_{001}|101\rangle \langle 101| - \rho_{001}|001\rangle \langle 001|, \\ 
    \mathcal{D}[\hat{d}^{\dagger}_{R}\hat{n}_{D}(\textbf{1}-\hat{n}_{L})\hat{\rho}_{S}] & = \rho_{001}|011\rangle \langle 011| - \rho_{001}|001\rangle \langle 001|, \\  
    \mathcal{D}[\hat{d}_{L}\hat{n}_{D}(\textbf{1}-\hat{n}_{R})\hat{\rho}_{S}] & = \rho_{101}|001\rangle \langle 001| - \rho_{101}|101\rangle \langle 101| - \frac{1}{2}(\beta |101\rangle \langle 011| + \beta^{*}|011\rangle \langle 101|), \\ 
    \mathcal{D}[\hat{d}_{R}\hat{n}_{D}(\textbf{1}-\hat{n}_{L})\hat{\rho}_{S}] & = \rho_{011}|001\rangle \langle 001| - \rho_{011}|011\rangle \langle 011| - \frac{1}{2}(\beta^{*} |011\rangle \langle 101| + \beta|101\rangle \langle 011|).
\end{align*}
Con estas expresiones, los flujos de información estacionarios se obtienen evaluando
$\dot{I}_{i} = \mathrm{Tr}\!\big[(\mathcal{L}_{i}\hat{\rho}_{S}) \ln \hat{\rho}_{S}\big]$. En particular,
\begin{align*}
    \dot{I}_{L} &  =
      \kappa_{L}\!\left(f_{L}(\epsilon)\,[a_{0}\rho_{000} - \rho_{000}\ln \rho_{000}] + (1-f_{L}(\epsilon))[\rho_{100}\ln \rho_{000} - a_{0}\rho_{100}] \right) \\ 
      &\quad + \kappa^{U}_{L}\!\left(f_{L}(\epsilon + U)\,[a_{1}\rho_{001} - \rho_{001}\ln \rho_{001}] + (1-f_{L}(\epsilon + U))[\rho_{101}\ln \rho_{001} -a_{1}\rho_{101} - \mathrm{Re}(b^{*}_{1}\beta )] \right) \\  
      &\quad + \kappa_{L}\!\left( f_{L}(\epsilon + U_{f})[\rho_{010}\ln \rho_{110} -b_{0}\rho_{010} - \mathrm{Re}(\alpha c^{*}_{0})]  + (1-f_{L}(\epsilon + U_{f}))[b_{0}\rho_{110} - \rho_{110}\ln \rho_{110}] \right) \\  
      &\quad + \kappa_{L}\!\left( f_{L}(\epsilon + U + U_{f})[\rho_{011}\ln \rho_{111}- b_{1}\rho_{011}] + (1-f_{L}(\epsilon+U+U_{f}) )[b_{1}\rho_{111} - \rho_{111}\ln \rho_{111}] \right),
\end{align*}
\begin{align*}
    \dot{I}_{R} &  =
      \kappa_{R}\!\left(f_{R}(\epsilon)\,[b_{0}\rho_{000} - \rho_{000}\ln \rho_{000}] + (1-f_{R}(\epsilon))[\rho_{010}\ln \rho_{000} - b_{0}\rho_{010}] \right) \\ 
      &\quad + \kappa^{U}_{R}\!\left(f_{R}(\epsilon + U)\,[b_{1} \rho_{001} - \rho_{001}\ln \rho_{001}] + (1-f_{R}(\epsilon + U))[\rho_{011}\ln \rho_{001} -a_{1}\rho_{011} - \mathrm{Re}(b^{*}_{1}\beta )] \right) \\  
      &\quad + \kappa_{R}\!\left( f_{R}(\epsilon + U_{f})[\rho_{100}\ln \rho_{110} -a_{0}\rho_{100} - \mathrm{Re}(\alpha c^{*}_{0})]  + (1-f_{R}(\epsilon + U_{f}))[a_{0}\rho_{110} - \rho_{110}\ln \rho_{110}] \right) \\  
      &\quad + \kappa_{R}\!\left( f_{R}(\epsilon + U + U_{f})[\rho_{101}\ln \rho_{111}- a_{1}\rho_{101}] + (1-f_{R}(\epsilon+U+U_{f}) )[a_{1}\rho_{111} - \rho_{111}\ln \rho_{111}] \right). \\
\end{align*}
Si bien la evaluación numérica de $\dot{I}_{i}$ es directa, disponer de expresiones analíticas permite aislar con claridad la contribución de las coherencias cuánticas, que aparece encapsulada en los parámetros $a_i$, $b_i$ y en los términos $\mathrm{Re}(\alpha c_0^*)$ y $\mathrm{Re}(b_1^* \beta)$. En el límite en que las coherencias se anulan se recupera el caso clásico, es decir
\begin{align*}
 \lim_{\alpha,\beta \to 0}\dot{I}_{L}  &= \dot{I}_{L\text{classic}}   =
 \kappa_{L}\!\left(f_{L}(\epsilon)[\rho_{000}\ln \rho_{100} - \rho_{000}\ln \rho_{000}] + (1-f_{L}(\epsilon))[\rho_{100}\ln \rho_{000} - \rho_{100}\ln \rho_{100}] \right)  \\ 
      &\quad + \kappa^{U}_{L}\!\left(f_{L}(\epsilon + U)[\rho_{001}\ln \rho_{101} - \rho_{001}\ln \rho_{001}] + (1-f_{L}(\epsilon + U))[\rho_{101}\ln \rho_{001} -\rho_{101}\ln \rho_{101}] \right) \\  
      &\quad + \kappa_{L}\!\left( f_{L}(\epsilon + U_{f})[\rho_{010}\ln \rho_{110} -\rho_{010}\ln \rho_{010}] + (1-f_{L}(\epsilon + U_{f}))[\rho_{110}\ln \rho_{010} - \rho_{110}\ln \rho_{110}] \right) \\  
      &\quad + \kappa_{L}\!\left( f_{L}(\epsilon + U + U_{f})[\rho_{011}\ln \rho_{111}- \rho_{011}\ln \rho_{011}] \right) \\
      & \quad + \kappa_{L}\!\left( (1-f_{L}(\epsilon+U+U_{f}))[\rho_{111}\ln \rho_{011} - \rho_{111}\ln \rho_{111}] \right),
    \end{align*}

\begin{align*}
 \lim_{\alpha,\beta \to 0}\dot{I}_{R}  &= \dot{I}_{R\text{classic}}   =
 \kappa_{R}\!\left(f_{R}(\epsilon)[\rho_{000}\ln \rho_{010} - \rho_{000}\ln \rho_{000}] + (1-f_{R}(\epsilon))[\rho_{010}\ln \rho_{000} - \rho_{010}\ln \rho_{010}] \right)  \\ 
      &\quad + \kappa^{U}_{R}\!\left(f_{R}(\epsilon + U)[\rho_{001}\ln \rho_{011} - \rho_{001}\ln \rho_{001}] + (1-f_{R}(\epsilon + U))[\rho_{011}\ln \rho_{001} -\rho_{011}\ln \rho_{011}] \right) \\  
      &\quad + \kappa_{R}\!\left( f_{R}(\epsilon + U_{f})[\rho_{100}\ln \rho_{110} -\rho_{100}\ln \rho_{100}] + (1-f_{R}(\epsilon + U_{f}))[\rho_{110}\ln \rho_{100} - \rho_{110}\ln \rho_{110}] \right) \\  
      &\quad + \kappa_{R}\!\left( f_{R}(\epsilon + U + U_{f})[\rho_{101}\ln \rho_{111}- \rho_{101}\ln \rho_{101}] \right) \\
      & \quad
      + \kappa_{R}\!\left( (1-f_{R}(\epsilon+U+U_{f}))[\rho_{111}\ln \rho_{101} - \rho_{111}\ln \rho_{111}] \right).
\end{align*}

\label{apendix5infoflow}


\end{appendixs}