% Template:     Tesis LaTeX
% Documento:    Archivo de ejemplo
% Versión:      3.4.0 (23/08/2024)
% Codificación: UTF-8
%
% Autor: Pablo Pizarro R.
%        pablo@ppizarror.com
%
% Manual template: [https://latex.ppizarror.com/tesis]
% Licencia MIT:    [https://opensource.org/licenses/MIT]

% ------------------------------------------------------------------------------
% NUEVO CAPÍTULO
% ------------------------------------------------------------------------------
% A diferencia de Template-Informe, Template-Tesis requiere el uso de capítulos; las secciones, subsecciones, etc son parte de un capítulo. Se recomienda el uso de un capítulo en un archivo distinto
\chapteranum{Introducción}

Uno de los principales objetivos de la termodinámica contemporánea es esclarecer el carácter físico de la información. Este desafío fue anticipado por el célebre experimento mental propuesto por James Clerk Maxwell en 1871, el cual sugiere una aparente violación de la segunda ley de la termodinámica al introducir un "demonio" que manipula la dinámica de un gas ideal mediante información\cite{Maxwell_1871}. Esta paradoja motivó una profunda revisión conceptual, abordada en trabajos fundamentales de Szilard\cite{szilard1964decrease}, Brillouin\cite{brillouin1951maxwell}, Landauer\cite{Landauer_1961} y Bennett\cite{bennett1982thermodynamics}, quienes formalizaron el vínculo entre información y entropía, sentando las bases de la termodinámica de la información.

Los avances experimentales en nanotecnología han permitido explorar sistemas de pocas partículas donde las fluctuaciones térmicas no pueden ser despreciadas\cite{douarche2005experimental,wang2005experimental}. En este contexto, las formulaciones tradicionales de la termodinámica pierden validez, lo que exige un marco teórico más general. La termodinámica estocástica cumple este rol, al extender las leyes termodinámicas a regímenes fuera del equilibrio en sistemas clásicos de pequeña escala\cite{van2013stochastic,jarzynski1997nonequilibrium}.

Además del estudio de energía y entropía, resulta fundamental entender el rol de la información en estos sistemas. Se deben distinguir dos enfoques según el tipo de dinámica involucrada. El primero considera sistemas no autónomos, cuyo Hamiltoniano depende explícitamente del tiempo y es controlado por un agente externo. En estos casos, el demonio de Maxwell puede modelarse mediante procesos de medición y retroalimentación, como ha sido mostrado en diversos trabajos\cite{cao2009thermodynamics,sagawa2010generalized}. El segundo enfoque aborda sistemas autónomos, de especial relevancia en contextos biológicos\cite{ehrich2023energy}, donde el sistema evoluciona sin intervención externa bajo gradientes constantes de energía (como diferencias de potencial o temperatura). Aquí, la estructura bipartita del sistema permite identificar un subsistema que actúa como "demonio", acoplado al sistema de interés y modulando su dinámica dependiendo del estado en que se encuentre.

En ausencia de un agente externo, cuantificar el flujo de información entre el demonio y el sistema presenta una dificultad adicional, ya que no se dispone del resultado explícito de la medición ni del mecanismo de retroalimentación. Este desafío fue abordado por Horowitz y Esposito mediante el formalismo de la termodinámica estocástica\cite{horowitz2014thermodynamics}, quienes propusieron una definición operacional del flujo de información en sistemas clásicos autónomos.

Posteriormente, Ptaszynski y Esposito extendieron este marco al contexto cuántico, cuantificando flujos de información en sistemas abiertos gobernados por dinámicas markovianas\cite{ptaszynski2019thermodynamics}. Para ello, emplearon una ecuación maestra de tipo Lindblad en su forma estándar (GKLS)\cite{gorini1976completely,lindblad1976generators,breuer2002theory}, compatible con una descripción termodinámica consistente cuando se cumple la condición de balance detallado. Este enfoque permite incorporar coherencias cuánticas —elementos no diagonales de la matriz densidad—, las cuales pueden servir como recursos termodinámicos\cite{ptaszynski2023fermionic,streltsov2017colloquium}.

Sin embargo, una dificultad clave en la descripción cuántica es la elección de la base. Si se adopta una base global, como exige la aproximación secular, se pierde la posibilidad de distinguir entre flujos locales de información. Por otro lado, una formulación en la base local puede revelar la estructura bipartita del sistema, pero ha sido criticada por posibles inconsistencias termodinámicas\cite{levy2014local,novotny2002investigation}, dado que en general no respeta las leyes de la termodinámica.

Para resolver este dilema, Potts et al.\ propusieron una ecuación maestra termodinámicamente consistente que permite trabajar en la base local e incluir términos no seculares\cite{potts2021thermodynamically}. Su enfoque reconoce que una descripción markoviana implica una resolución energética finita, y por tanto introduce un Hamiltoniano rescalado en la definición de las cantidades termodinámicas.

Trabajos previos han empleado el formalismo desarrollado por Potts para analizar los efectos cuánticos en modelos compuestos por dos puntos cuánticos, cada uno acoplado a un reservorio distinto, y cómo estos efectos influyen en el flujo de partículas~\cite{prech2023entanglement}. Por otro lado, el trabajo de Esposito y Horowitz investigó los flujos de información en un sistema clásico constituido por dos puntos cuánticos, donde uno de ellos está acoplado a dos reservorios y el otro a un único baño térmico. En dicho estudio, se demostró que el sistema actúa como un Demonio de Maxwell autónomo~\cite{horowitz2014thermodynamics}.

En el presente trabajo se emplea el formalismo de Potts para describir la evolución temporal de un sistema cuántico formado por tres puntos cuánticos, cada uno acoplado a un reservorio distinto. Uno de estos puntos cuánticos desempeña el rol de Demonio de Maxwell, mientras que los otros dos constituyen el sistema físico, denotado como $LR$. Esta configuración permite estudiar los flujos de información y las demás magnitudes termodinámicas relevantes, en un régimen donde se manifiestan efectos cuánticos tales como la coherencia y el entrelazamiento entre los distintos sitios del sistema.

La tesis está estructurada de la siguiente manera: en el Capítulo 1 se introducen los conceptos fundamentales necesarios para describir la evolución de un sistema cuántico, ya sea como sistema cerrado —mediante la ecuación de Liouville— o como sistema abierto —a través de la ecuación de Lindblad en la aproximación secular. El Capítulo 2 está dedicado al desarrollo del Formalismo de Estadística de Conteo, el cual será empleado para derivar la ecuación de Lindblad termodinámicamente consistente propuesta por Potts. En el Capítulo 3 se introduce el concepto de flujo de información, discutiendo su influencia en la formulación de la segunda ley de la termodinámica tanto en contextos clásicos como cuánticos. Finalmente, en el Capítulo 4 se presenta la evolución dinámica de un sistema compuesto por tres puntos cuánticos, y se analizan los principales resultados obtenidos para dos configuraciones distintas del modelo.




% SUB-SECCIÓN
% Las sub-secciones se inician con \subsection, si se quiere una sub-sección
% sin número se pueden usar las funciones \subsectionanum (nuevo subtítulo sin
% numeración) o la función \subsectionanumnoi para crear el mismo subtítulo sin
% numerar y sin aparecer en el índice


% ------------------------------------------------------------------------------
% NUEVO CAPÍTULO
% ------------------------------------------------------------------------------

\chapter{Sistemas cuánticos abiertos}

En este capítulo se describen los conceptos básicos para poder describir la evolución de un sistema cuántico, en la sección \ref{sec:closedQM} se describe la evolución unitaria para un sistema cerrado.  Por otro lado en la sección \ref{sec:lindblad} se describe una de las ecuaciones utilizadas para describir el comportamiento de un sistema cuántico acoplado a un reservorio con infinitos grados de libertad.


\section{Sistemas cuánticos cerrados}
Un sistema cuántico cerrado puede ser descrito por la matriz densidad $\hat{\rho}$, la evolución de la matriz densidad dependerá del Hamiltoniano del sistema $\hat{H}(t)$, la cuál en el cuadro de Schrodinger consiste en la ecuación de Liouville-Von Neumman($\hbar = 1$)\cite{breuer2002theory}:

\begin{equation*}
    \frac{d}{dt}\hat{\rho}(t) = -i[\hat{H}(t),\hat{\rho}(t)].
\end{equation*}

Y la solución descrita por la evolución unitaria 

\begin{equation*}
    \hat{U}(t,t_{0}) = \hat{T}_{\leftarrow} \exp \left[ -i \int_{t_{0}}^{t}ds \hat{H}(s) \right] \implies \hat{\rho}(t) = \hat{U}(t,t_{0})\hat{\rho}(t_{0})\hat{U}^{\dagger}(t,t_{0}),
\end{equation*}
en donde $\hat{T}_{\leftarrow}$ consiste en el operador ordenación temporal cronológico que ordena los productos de operadores dependientes del tiempo, de tal manera que el tiempo en el que son evaluados los operadores va creciendo de derecha a izquierda.

\subsection{Cuadro de interacción}
Supongamos que el Hamiltoniano del sistema se puede separar en dos partes

\begin{equation*}
    \hat{H}(t) = \hat{H}_{0} + \hat{H}_{I}(t),
\end{equation*}
en teoría esto se puede hacer de varias formas, pero por lo general, si tenemos el caso de dos subsistemas, $\hat{H}_{0}$ contiene los Hamiltonianos de cada uno de ellos cuando no hay interacción, mientras que $\hat{H}_{I}$ representa la interacción entre ellos. Si introducimos los operadores unitarios

\begin{equation*}
    \hat{U}_{0}(t,t_{0}) \equiv \exp[-i\hat{H}_{0}(t-t_{0}) ]  \hspace{15mm} \hat{U}_{I}(t,t_0) \equiv \hat{U}^{\dagger}_{0}(t,t_{0})\hat{U}(t,t_{0}),
\end{equation*}
se puede describir la matriz densidad en el cuadro de interacción

\begin{equation*}
    \hat{\rho}_{I}(t) \equiv \hat{U}_{I}(t,t_{0})\hat{\rho}(t_{0})\hat{U}^{\dagger}_{I}(t,t_{0}).
\end{equation*}

Y la evolución en este cuadro de interacción es 

\begin{equation}
    \frac{d}{dt}\hat{\rho}_{I}(t) = -i[\tilde{H}_{I}(t), \hat{\rho}_{I}(t)],
    \label{sec11:interactionp}
\end{equation}
con 

\begin{equation*}
    \tilde{H}_{I}(t) = \hat{U}^{\dagger}_{0}(t,t_{0})\hat{H}_{I}(t)\hat{U}_{0}(t,t_{0}).
\end{equation*}

\label{sec:closedQM}



\section{Ecuación de Lindblad}
\label{SEClindblad}
\subsection{Dinámica de un sistema abierto}

Un sistema abierto consiste en un sistema cuántico $S$ denominado el sistema reducido, el cuál está acoplado a un sistema $B$ denominado el ambiente. Estos representan  subsistemas del sistema total $S+B$. Sea $\mathcal{H}_{s}$ el espacio de Hilbert del sistema y $\mathcal{H}_{B}$ el espacio de Hilbert del ambiente, el espacio de Hilbert del sistema total $S+B$ es  $\mathcal{H} = \mathcal{H}_{s} \otimes \mathcal{H}_{B}$, el Hamiltoniano total se constituye por el Hamiltoniano del sistema $\hat{H}_{S}$, el Hamiltoniano $\hat{H}_{B}$ y la interacción $\hat{H}_{I}(t)$

\begin{equation}
    \hat{H}(t) = \hat{H}_{S} \otimes \mathbf{I}_{B} + \mathbf{I}_{s} \otimes \hat{H}_{B} + \hat{H}_{I}(t).
    \label{sec2:sistemabierto}
\end{equation}

Para describir el sistema con la ecuación \ref{sec2:sistemabierto} es necesario separarlo en dos subsistemas, $S$ y $B$, permitiendo así que el espacio de Hilbert total sea descrito por $\mathcal{H} = \mathcal{H}_{s}\otimes \mathcal{H}_{B}$. Sin embargo, en el caso de partículas idénticas como los fermiones el identificar estos dos subsistemas no es obvio. Aún así, en el formalismo de segunda cuantización el sistema se puede describir mediante una base del espacio de Fock, que sí permite encontrar subsistemas en los cuáles el espacio de Hilbert total es un producto tensorial entre los dos subsistemas\cite{friis2013fermionic,vidal2021quantum}.

Los observables que se estudian en el sistema $S$ son de la forma $A\otimes \mathbf{I}_{B}$ y su valor de expectación puede ser calculado mediante la ecuación

\begin{equation*}
    \langle A \rangle = \text{Tr}_{S}\{A \hat{\rho}_{S} \},
\end{equation*}
donde 

\begin{equation*}
    \hat{\rho}_{S} = \text{Tr}_{B}\{ \hat{\rho} \},
\end{equation*}
es la matriz densidad del sistema reducido. La expresión $\text{Tr}_{S}$ hace referencia a la traza con respecto al espacio de Hilbert del sistema, mientras que $\text{Tr}_{B}$  
es la traza con respecto al espacio de Hilbert del ambiente. El objeto de estudio principalmente es $\hat{\rho}_{S}$ y la ecuación de Lindblad será utilizada para describir su evolución.

\subsection{Ecuación maestra}
Partiendo de un sistema $S$ acoplado débilmente a un reservorio $B$, en donde el Hamiltoniano total es descrito por 

\begin{equation*}
    \hat{H} = \hat{H}_{S} + \hat{H}_{B} + \hat{H}_{I},
\end{equation*}
la evolución en el cuadro de interacción del sistema total será

\begin{equation}
    \frac{d}{dt}\hat{\rho}_{I}(t) = -i[\hat{H}_{I}(t), \hat{\rho}_{I}(t)],
    \label{eqsec2:linbladpic}
\end{equation}
o en su forma integral

\begin{equation*}
    \hat{\rho}_{I}(t) = \hat{\rho}(0) - i \int_{0}^{t} ds[\hat{H}_{I}(s), \hat{\rho}_{I}(s)].
\end{equation*}

Reinsertando la forma integral en la ecuación \ref{eqsec2:linbladpic} y trazando los grados de libertad del reservorio, nos queda

\begin{equation}
    \frac{d}{dt}\hat{\rho}_{IS}(t) = -i \text{Tr}_{B}\{[\hat{H}_{I}(t),\hat{\rho}(0)] \}  -  \int_{0}^{t}ds \text{Tr}_{B}\{[\hat{H}_{I}(t), [\hat{H}_{I}(s),\hat{\rho}_{I}(s)]]\}.
\end{equation}

Ahora se asumirán dos cosas, primero que $\text{Tr}_{B}\{[\hat{H}_{I}(t),\rho(0)] \} = 0$. Segundo, debido a que todavía la ecuación depende de $\hat{\rho}(t)$ correspondiente al sistema total, tendremos que realizar la aproximación de Born, que consiste en que si se tiene acoplamiento débil, la influencia del sistema en el reservorio será pequeña, por ende se puede asumir que el estado del reservorio $\hat{\rho}_{B}$ permanece constante, y sólo el que se ve alterado es la evolución del sistema, es decir

\begin{equation*}
    \hat{\rho}(t) = \hat{\rho}_{S}(t)\otimes \hat{\rho}_{B},
\end{equation*}
con lo cuál la ecuación \ref{eqsec2:linbladpic} se vuelve

\begin{equation}
    \frac{d}{dt}\hat{\rho}_{IS}(t) = -  \int_{0}^{t}ds \text{Tr}_{B}\{[\hat{H}_{I}(t), [\hat{H}_{I}(s),\hat{\rho}_{IS}(s) \otimes \hat{\rho}_{B}]]\}.
\end{equation} 

Esta ecuación es lo que se denomina no Markoviana, ya que requiere conocer todo el pasado de $\hat{\rho}_{IS}(s)$. Para simplificar el problema, se aplicará la aproximación de Markov, con el objetivo de que la evolución de $\hat{\rho}_{IS}(t)$ dependa sólo del estado en que se encuentra en el instante de tiempo, es decir

\begin{equation*}
    \frac{d}{dt}\hat{\rho}_{IS}(t) = -  \int_{0}^{t}ds \text{Tr}_{B}\{[\hat{H}_{I}(t), [\hat{H}_{I}(s),\hat{\rho}_{IS}(t) \otimes \hat{\rho}_{B}]]\},
\end{equation*} 
esta ecuación es denominada la ecuación de Redfield. Se puede hacer el cambio de variable $s= x = t-s$, así el integral queda

\begin{equation}
    \frac{d}{dt}\hat{\rho}_{IS}(t) = -  \int_{0}^{t}ds \text{Tr}_{B}\{[\hat{H}_{I}(t), [\hat{H}_{I}(t-s),\hat{\rho}_{IS}(t) \otimes \hat{\rho}_{B}]]\}.
    \label{eq3sec2:markov}
\end{equation} 

Por último, se podrá hacer otra aproximación, ya que en esta expresión aparecen las funciones correlación del baño, las cuáles decaen en el tiempo con una escala característica dada por el denominado tiempo de correlación $\tau_{B}$, a partir del cual dichas funciones se vuelven despreciables. Mientras que el sistema tendrá su tiempo de relajación $\tau_{R}$ que consiste en el tiempo que demora en llegar a su estado estacionario. La aproximación markoviana requiere que $\tau_{R}\gg \tau_{B}$, ya que él sistema no debe ser capaz de percibir la dinámica del baño, esto permite tomar el límite del integral en \ref{eq3sec2:markov} a infinito, ya que para tiempos muy largos se anula el integral debido a que las funciones correlación se vuelven practicamente nulas. Y finalmente quedará la evolución

\begin{equation}
    \frac{d}{dt}\hat{\rho}_{IS}(t) = -  \int_{0}^{\infty}ds \text{Tr}_{B}\{[\hat{H}_{I}(t), [\hat{H}_{I}(t-s),\hat{\rho}_{IS}(t) \otimes \hat{\rho}_{B}]]\}.
    \label{eq3sec2:markov1}
\end{equation} 

La interacción en el cuadro de Schrodinger $\hat{H}_{I}$ será de la forma general

\begin{equation*}
    \hat{H}_{I} = \sum_{\alpha}A_{\alpha} \otimes B_{\alpha},
\end{equation*}
donde los operadores $A_{\alpha} = A^{\dagger}_{\alpha}$ actuan en el espacio de Hilbert del sistema, mientras que $B_{\alpha}=B^{\dagger}_{\alpha}$ actua en el espacio de Hilbert del reservorio. Es conveniente escribir la interacción en función de los autoestados de $\hat{H}_{S}$. Para ello, si tenemos los autovalores $\epsilon$ y sus respectivos operadores de proyección $\Pi(\epsilon) = |\epsilon\rangle \langle \epsilon|$, se podrán definir los operadores

\begin{equation*}
    A_{\alpha}(\omega) \equiv \sum_{\epsilon' - \epsilon}\Pi(\epsilon)A_{\alpha}\Pi(\epsilon'), 
\end{equation*}
estos operadores se denominan operadores globales, ya que utilizan los autoestados del sistema. Debido a esta definición, se cumplen las relaciones

\begin{align*}
    [\hat{H}_{S},A_{\alpha}(\omega)] & = - \omega A_{\alpha}(\omega) \\
    [\hat{H}_{S},A^{\dagger}_{\alpha}(\omega)] & = \omega A^{\dagger}_{\alpha}(\omega),
\end{align*}
posteriormente se deberá pasar al cuadro de interacción y calcular $U^{\dagger}_{s}(t)\hat{H}_{I}U_{s}(t)$. Para ello, se utilizará la relación de Baker Campbell

\begin{equation}
    e^{A}Be^{-A} = B  + [A,B] + \frac{1}{2}[A,[A,B]] +..,
    \label{sec2lind:baker}
\end{equation}
y se puede derivar las relaciones 

\begin{align*}
    e^{i\hat{H}_{S}t}A_{\alpha}(\omega) e^{-i\hat{H}_{S}t} & = e^{-i\omega t} A_{\alpha}(\omega) \\
    e^{i\hat{H}_{S}t}A^{\dagger}_{\alpha}(\omega) e^{-i\hat{H}_{S}t} & = e^{i\omega t} A^{\dagger}_{\alpha}(\omega).
\end{align*}

De la relación de completitud $\sum_{\epsilon}\Pi(\epsilon) = \mathbf{I}$ podremos notar que

\begin{align*}
   \sum_{\omega}A_{\alpha}(\omega) & =  \sum_{\omega,\epsilon'-\epsilon  = \omega} \Pi(\epsilon) A_{\alpha} \Pi(\epsilon') \\
   & = \sum_{\omega,\epsilon} \Pi(\epsilon) A_{\alpha} \Pi(\epsilon + \omega) \\
   & = A_{\alpha} = \sum_{\omega}A^{\dagger}_{\alpha}(\omega),
\end{align*}

con esto el Hamiltoniano de interacción en el cuadro de Schrodinger

\begin{equation*}
    \hat{H}_{I} = \sum_{\alpha,\omega}A_{\alpha}(\omega) \otimes B_{\alpha} = \sum_{\alpha,\omega}A^{\dagger}_{\alpha}(\omega) \otimes B^{\dagger}_{\alpha}.
\end{equation*}

Aplicando las relaciones en el cuadro de interacción

\begin{equation}
    \hat{H}_{I}(t) = \sum_{\alpha,\omega}e^{-i\omega t}A_{\alpha}(\omega) \otimes B_{\alpha}(t) = \sum_{\alpha,\omega}e^{i\omega t}A^{\dagger}_{\alpha}(\omega) \otimes B^{\dagger}_{\alpha}(t),
    \label{seclindbladinteraction1}
\end{equation}
donde $B_{\alpha}(t) = e^{i\hat{H}_{B}t}B_{\alpha}e^{-i\hat{H}_{B}t}$. La ecuación \ref{seclindbladinteraction1} se puede introducir en la ecuación \ref{eq3sec2:markov1} y se obtiene 

\begin{align*}
    \frac{d}{dt}\hat{\rho}_{IS}(t) = \int_{0}^{\infty} ds \text{Tr}_{B}\left[  \hat{H}_{I}(t-s)\hat{\rho}_{IS}(t)\hat{\rho}_{B}\hat{H}_{I}(t) - \hat{\rho}_{IS}(t)\hat{\rho}_{B}\hat{H}_{I}(t-s)\hat{H}_{I}(t)    \right.\\
    \left. + \hat{H}_{I}(t)\hat{\rho}_{IS}(t)\hat{\rho}_{B}\hat{H}_{I}(t-s) -  \hat{H}_{I}(t)\hat{H}_{I}(t-s)\hat{\rho}_{IS}(t)\hat{\rho}_{B}  \right],
\end{align*}
así, reemplazando la forma explícita de la interacción  

\begin{align*}
    \frac{d}{dt}\hat{\rho}_{IS}(t) & = \sum_{\omega,\omega'}\sum_{\alpha,\beta} \int_{0}^{\infty} ds  \text{Tr}_{B}[e^{i\omega s}B_{\beta}(t-s)\hat{\rho}_{B}B^{\dagger}_{\alpha}(t)]e^{i(\omega'- \omega)t}\left(A_{\beta}(\omega)\hat{\rho}_{IS}(t)A^{\dagger}_{\alpha}(\omega') - A^{\dagger}_{\alpha}(\omega')A_{\beta}(\omega) \hat{\rho}_{IS}(t) \right) \\
     + & \sum_{\omega,\omega'}\sum_{\alpha,\beta} \int_{0}^{\infty} ds  \text{Tr}_{B}[e^{-i\omega s}B_{\alpha}(t)\hat{\rho}_{B}B^{\dagger}_{\beta}(t-s)]e^{-i(\omega'- \omega)t}\left(A_{\alpha}(\omega')\hat{\rho}_{IS}(t)A^{\dagger}_{\beta}(\omega) - \hat{\rho}_{IS}(t)A^{\dagger}_{\beta}(\omega)A_{\alpha}(\omega') \right).
\end{align*}

Se pueden definir las funciones correlación del baño por

\begin{equation*}
    \Gamma_{\alpha\beta}(\omega) = \int_{0}^{\infty}ds e^{i\omega s}\text{Tr}_{B}[B^{\dagger}_{\alpha}(t)B_{\beta}(t-s)\hat{\rho}_{B}],
\end{equation*}
y aplicando estas funciones queda 

\begin{align*}
    \frac{d}{dt}\hat{\rho}_{IS}(t) & = \sum_{\omega,\omega'}\sum_{\alpha,\beta} e^{i(\omega'- \omega)t}\Gamma_{\alpha \beta}(\omega)\left(A_{\beta}(\omega)\hat{\rho}_{IS}(t)A^{\dagger}_{\alpha}(\omega') - A^{\dagger}_{\alpha}(\omega')A_{\beta}(\omega) \hat{\rho}_{IS}(t) \right) \\
    & + \sum_{\omega,\omega'}\sum_{\alpha,\beta} e^{-i(\omega'- \omega)t}\Gamma^{*}_{\beta \alpha}(\omega) \left(A_{\alpha}(\omega')\hat{\rho}_{IS}(t)A^{\dagger}_{\beta}(\omega) - \hat{\rho}_{IS}(t)A^{\dagger}_{\beta}(\omega)A_{\alpha}(\omega') \right).
\end{align*}

Si se considera el tiempo característico de evolución del sistema $S$ como $\tau_S = |\omega - \omega'|^{-1}$, y además se cumple que el tiempo de relajación $\tau_R \gg \tau_S$, entonces los términos no seculares (aquellos con $\omega \neq \omega'$) pueden despreciarse. Esto se debe a que los factores oscilatorios $e^{i(\omega - \omega')t}$ varían rápidamente en la escala de tiempo $\tau_R$, promediándose a cero.

Al tomar esta aproximación, la evolución del sistema queda
\begin{align*}
    \frac{d}{dt}\hat{\rho}_{IS}(t) & = \sum_{\omega}\sum_{\alpha,\beta} \Gamma_{\alpha \beta}(\omega)\left(A_{\beta}(\omega)\hat{\rho}_{IS}(t)A^{\dagger}_{\alpha}(\omega) - A^{\dagger}_{\alpha}(\omega)A_{\beta}(\omega) \hat{\rho}_{IS}(t) \right) \\
    & + \sum_{\omega}\sum_{\alpha,\beta} \Gamma^{*}_{\beta \alpha}(\omega) \left(A_{\alpha}(\omega)\hat{\rho}_{IS}(t)A^{\dagger}_{\beta}(\omega) - \hat{\rho}_{IS}(t)A^{\dagger}_{\beta}(\omega)A_{\alpha}(\omega) \right),
\end{align*}
finalmente, al separar la función correlación espectral en su parte real e imaginaria como

\begin{equation*}
\Gamma_{\alpha \beta}(\omega) = \gamma_{\alpha \beta}(\omega)/2 + iS_{\alpha \beta}(\omega).
\end{equation*}

Se obtiene la ecuación de Lindblad 

\begin{equation}
    \frac{d}{dt} \hat{\rho}_{IS}(t) = -i[\hat{H}_{LS},\hat{\rho}_{IS}(t)] + \mathcal{D}(\hat{\rho}_{IS}(t)),
    \label{seclindbladfinal}
\end{equation}
con $\hat{H}_{LS}$ el Hamiltoniano \textit{Lamb Shift}

\begin{equation*}
    \hat{H}_{LS} = \sum_{\omega} \sum_{\alpha,\beta} S_{\alpha,\beta}(\omega)A^{\dagger}_{\alpha}(\omega)A_{\beta}(\omega), 
\end{equation*}
y el Disipador

\begin{equation*}
    \mathcal{D}(\hat{\rho}_{IS}(t)) = \sum_{\omega}\sum_{\alpha,\beta} \gamma_{\alpha \beta}(\omega) \left[ A_{\beta}(\omega)\hat{\rho}_{IS}(t)A^{\dagger}_{\alpha}(\omega) - \frac{1}{2}\{A^{\dagger}_{\alpha}(\omega)A_{\beta}(\omega), \hat{\rho}_{IS}(t)  \} \right].
\end{equation*}

El escribir el disipador de esta manera es importante, ya que esta evolución se encuentra en la forma GKLS que permite preservar traza, hermiticidad y positividad de la matriz densidad\cite{manzano2020short}. 

\label{sec:lindblad}


\chapter{Estadística de conteo(\textit{Full Counting Statistics}) }
En este capítulo se presenta el formalismo de \textit{Full Counting Statistics}(FCS). La sección \ref{sec3workheat} esta dedicada a la  descripción de las cantidades termodinámicas de interés, y en la sección \ref{Leyestermo} se establecen las leyes de la termodinámica en el contexto del modelo. En la sección \ref{sec2:estadistica2puntos} se desarrolla la estadística de medición en 2 puntos, mientras que en la sección \ref{sec2:superop} se describe brevemente el formalismo de superoperadores. Finalmente, en la sección \ref{sec2:master} se desarrolla el cálculo de la ecuación maestra generalizada deducida en \cite{potts2021thermodynamically}. 

\section{Cantidades termodinámicas}
Se considera el sistema descrito por el Hamiltoniano

\begin{equation*}
    \hat{H}_{tot}(t) = \hat{H}_{S}(t) + \sum_{\alpha}(\hat{H}_{\alpha} + \hat{V}_{\alpha})  = \hat{H}_{S}(t) + \hat{H}_{B} + \hat{V},
\end{equation*}
donde el primer término describe el Hamiltoniano del sistema el cuál puede ser tiempo dependiente, la segunda parte describe los reservorios térmicos y por último la tercera parte constituye el acoplamiento entre sistema-baño.\\
El sistema intercambia energía y partículas con el reservorio, por ende el cambio de energía puede ser dividido por una contribución correspondiente al calor y otra correspondiente al trabajo. Así, se define el calor que libera el baño $\alpha$ durante el intervalo de tiempo $[0,t]$ como

\begin{equation}
    \langle Q_{\alpha}\rangle = - \text{Tr}\{(\hat{H}_{\alpha} - \mu_{\alpha}\hat{N}_{\alpha})\hat{\rho}_{tot}(t) \} + \text{Tr}\{(\hat{H}_{\alpha} - \mu_{\alpha}\hat{N}_{\alpha})\hat{\rho}_{tot}(0) \},
\label{sec3:calor}
\end{equation}
en la cuál $\hat{N}_{\alpha}$ corresponde al operador de número que cuantifica el número de partículas en el baño $\alpha$ y $\mu_{\alpha}$ es su potencial químico. El trabajo promedio que entrega el reservorio $\alpha$ se define como

\begin{equation}
    \langle W_{\alpha}\rangle = - \mu_{\alpha} (\text{Tr}\{\hat{N}_{\alpha} \hat{\rho}_{tot}(t) \} - \text{Tr}\{\hat{N}_{\alpha}\hat{\rho}_{tot}(0) \}  ).
    \label{sec3:trabajo}
\end{equation}
\label{sec3workheat}

\newpage

\section{Leyes de la termodinámica}
\label{Leyestermo}
\subsection{Ley cero}
Si se tiene un sistema total descrito por un sistema reducido y un baño, que se encuentran en equilibrio a temperatura inversa $\beta$ y potencial químico $\mu$. De tal manera que el estado del sistema en equilibrio es el equilibrio gran canónico

\begin{equation*}
    \hat{\rho}^{eq}_{tot} = \frac{e^{-\beta(\hat{H}_{tot} - \mu \hat{N}_{tot})}}{Z}  \hspace{28mm} Z = \text{Tr}\{e^{-\beta(\hat{H}_{tot} - \mu \hat{N}_{tot})} \},
\end{equation*}
y por lo tanto el estado de equilibrio del sistema reducido es

\begin{equation*}
    \hat{\rho}_{S} = \frac{1}{Z}\text{Tr}_{B}\{ e^{-\beta(\hat{H}_{tot} - \mu \hat{N}_{tot})} \},
\end{equation*}
que en el límite de acoplamiento débil entre el sistema y el baño, este equilibrio se convierte en \cite{geva2000second}
\begin{equation*}
    \hat{\rho}_{S} = \frac{e^{-\beta(\hat{H}_{S} - \mu\hat{N}_{S})}}{\text{Tr}_{S}\{e^{-\beta(\hat{H}_{S} - \mu \hat{N}_{S})} \} }.
\end{equation*}

\subsection{Primera Ley}
Para escribir la primera ley, primero se debe introducir la corriente de calor y la potencia entregadas por el baño $\alpha$

\begin{equation*}
    J_{\alpha}(t) = \partial_{t}\langle Q_{\alpha}\rangle \hspace{12mm}  P_{\alpha}(t) = \partial_{t}\langle W_{\alpha}\rangle.
\end{equation*}

En el límite de acoplamiento débil para un Hamiltoniano tiempo independiente la primera ley consiste en 
\begin{equation*}
    \partial_{t}E(t) = \sum_{\alpha}[J_{\alpha}(t) + P_{\alpha}(t) ] \hspace{12mm} E = \text{Tr}\{\hat{H}_{S}\hat{\rho}_{tot}(t) \},
\end{equation*}
donde $E(t)$ consiste en la energía interna del sistema, mientras que la corriente de calor y la potencia entregada por el baño $\alpha$ en función de la matriz densidad del sistema será 

\begin{equation*}
    J_{\alpha}(t) = \text{Tr}\{(\hat{H}_{S} - \mu_{\alpha}\hat{N}_S)\mathcal{L}_{\alpha}\hat{\rho}_{S}(t) \} \hspace{12mm} P_{\alpha} = \mu_{\alpha} \text{Tr}\{\hat{N}_{S}\mathcal{L}_{\alpha}\hat{\rho}_{S}(t) \}. 
\end{equation*}

\subsection{Segunda Ley}
Para introducir la segunda ley se parte de la condición inicial en la que el sistema y el reservorio son sistemas no correlacionados, es decir 

\begin{equation*}
    \hat{\rho}_{tot}(0) = \hat{\rho}_{S}(0)\otimes_{\alpha}\hat{\tau}_{\alpha} \hspace{12mm} \hat{\tau}_{\alpha} = \frac{e^{-\beta_{\alpha}(\hat{H}_{\alpha}-\mu_{\alpha}\hat{N}_{\alpha}) }}{ \text{Tr}\{e^{-\beta_{\alpha}(\hat{H}_{\alpha} - \mu_{\alpha}\hat{N}_{\alpha})}\} },
\end{equation*}
donde cada reservorio se encuentra en equilibrio gran canónico con inverso de la temperatura $\beta_{\alpha}$ y potencial químico $\mu_{\alpha}$. Con esta condición inicial, la segunda ley puede ser escrita en función de la producción de entropía $\sigma(t)$ como \cite{esposito2010entropy}

\begin{equation*}
    \sigma(t) \equiv \Delta S(t) - \sum_{\alpha}\beta_{\alpha}\langle Q_{\alpha}\rangle \geq 0.
\end{equation*}

$\Delta S$ denota el cambio en la entropía de Von Neumann del sistema

\begin{equation*}
    \Delta S(t) = -  \text{Tr}\{\hat{\rho}_{S}(t)\ln \hat{\rho}_{S}(t) \} +  \text{Tr}\{ \hat{\rho}_{S}(0)\ln \hat{\rho}_{S}(0) \},
\end{equation*}
es decir, la producción de entropía se separa en una parte que corresponde al cambio de entropía del sistema, mientras que la segunda parte corresponde a la contribución debido al calor que entrega el ambiente.\\
Como en este trabajo se hará análisis de flujos continuos tanto de energía como de partículas, es importante estudiar la tasa de producción de entropía. Si bien, la producción de entropía siempre es positiva, la taza de producción de entropía no siempre lo es, aún así para sistemas markovianos se cumple que \cite{strasberg2019non}

\begin{equation*}
    \dot{\sigma}(t) \equiv \partial_{t}\Delta S(t) - \sum_{\alpha}\beta_{\alpha}J_{\alpha} \geq 0.
\end{equation*}

Y la igualdad se cumple para procesos reversibles. Con esto se han definido las leyes de la termodinámica en función de flujos continuos, que se podrán calcular a través de la matriz densidad reducida. En esta tesis estas cantidades serán estudiadas principalmente en el estado estacionario.

\label{sec3sub:leyestermo}

\section{Estadística de medición en dos puntos}
A lo largo de este trabajo, se pretende calcular diversas cantidades termodinámicas asociadas a la dinámica del sistema, tales como la energía y el trabajo. Si bien los valores promedio de estas cantidades pueden obtenerse con las expresiones \ref{sec3:calor} y \ref{sec3:trabajo}, tanto el calor y el trabajo intercambiados con los reservorios deben considerarse como variables aleatorias, ya que su determinación implica medir sobre el estado de los reservorios. Debido a la naturaleza cuántica de la evolución, dichas mediciones pueden arrojar distintos resultados en diferentes repeticiones del experimento. Por lo tanto, para acceder a los valores promedio, es útil calcular la distribución de probabilidad del calor y el trabajo intercambiados por el reservorio.

Para  un sistema total descrito por la condición inicial $\hat{\rho}_{tot}(0) = \hat{\rho}_{S}(0) \otimes_{\alpha} \hat{\tau}_{\alpha}$ la distribución de probabilidad es   

\begin{align*}
    P(\textbf{Q},\textbf{W}) & = \sum_{\textbf{E},\textbf{E}',\textbf{N},\textbf{N}'} P_{t}(\textbf{E}',\textbf{N}'|\textbf{E},\textbf{N}) P_{0}(\textbf{E},\textbf{N})\\
                             & \times \Pi_{\alpha} \delta(W_{\alpha} - \mu_{\alpha}(N_{\alpha} - N'_{\alpha})) \delta(Q_{\alpha} + W_{\alpha}  - (E_{\alpha} - E'_{\alpha})).   
\end{align*}

La probabilidad conjunta de que cada baño $\alpha$ tenga energía $E_{\alpha}$ y número de partículas $N_{\alpha}$ a tiempo $t=0$ es

\begin{equation*}
    P_{0}(\textbf{E},\textbf{N}) = \Pi_{\alpha}  \frac{e^{-\beta_{\alpha}(E_{\alpha} - \mu_{\alpha}N_{\alpha} )  }}{ \text{Tr}\{e^{-\beta_{\alpha}(E_{\alpha} - \mu_{\alpha}N_{\alpha} )  }\} }.
\end{equation*}

Para obtener $P_{t}(\textbf{E}',\textbf{N}'|\textbf{E},\textbf{N})$ que corresponde a la probabilidad condicional de que los reservorios tengan energías $\textbf{E}'$ y números de partículas $\textbf{N}'$ a tiempo $t$, dado que inicialmente se midieron los valores $\textbf{E}$ y $\textbf{N}$, se debe partir del estado proyectado a $t=0$ correspondiente a dicha medición inicial, es decir

\begin{equation*}
    \hat{\rho}'(0) = \frac{\hat{P}_{\textbf{E}, \textbf{N} }\hat{\rho}(0) \hat{P}_{\textbf{E}, \textbf{N} } }{\text{Tr}\{\hat{P}_{\textbf{E}, \textbf{N} }\hat{\rho}(0) \hat{P}_{\textbf{E}, \textbf{N} } \} },
\end{equation*}
donde $\hat{P}_{\textbf{E}, \textbf{N} } = I_{S} \otimes \hat{P}^{B}_{\textbf{E}, \textbf{N} }$ es el proyector asociado a una medición de las energías y los números de partículas en los reservorios. A continuación, el sistema evoluciona de forma unitaria hasta un tiempo $t$

\begin{equation*}
    \hat{\rho}'(t) = \hat{U}(t)\hat{\rho}'(0)\hat{U}^{\dagger}(t),
\end{equation*}
finalmente se realiza una segunda medición a tiempo $t$, obteniéndose los valores $\textbf{E}'$ y $\textbf{N}'$. Así, la probabilidad condicional correspondiente es

\begin{equation*}
    P_{t}(\textbf{E}',\textbf{N}'|\textbf{E},\textbf{N}) = \text{Tr}\{\hat{P}_{\textbf{E}', \textbf{N}' }\hat{\rho}'(t) \hat{P}_{\textbf{E}', \textbf{N}' } \},
\end{equation*}
escribiendo los proyectores de manera explícita $\hat{P}^{B}_{\textbf{E}, \textbf{N} } = |\textbf{E}, \textbf{N} \rangle\langle \textbf{E}, \textbf{N}|$, se puede obtener 

\begin{align*} 
    P_{t}(\textbf{E}',\textbf{N}'|\textbf{E},\textbf{N}) & =  \text{Tr}\{\hat{U}(t)(\hat{\rho}_{S}(0)\otimes |\textbf{E}, \textbf{N}\rangle  \langle \textbf{E}, \textbf{N}| ) \hat{U}^{\dagger}(t)  |\textbf{E}', \textbf{N}'\rangle  \langle \textbf{E}', \textbf{N}'| \} \\
        & =  \text{Tr}\{|\textbf{E},\textbf{N} \rangle \langle \textbf{E}', \textbf{N}'| \hat{U}(t)\hat{\rho}_{S}(0) \langle \textbf{E}, \textbf{N}|\hat{U}^{\dagger}(t)|\textbf{E}', \textbf{N}'\rangle \}      \\ 
        & = \text{Tr}_{S}\{ \text{Tr}_{B}\{|\textbf{E},\textbf{N} \rangle \langle \textbf{E}', \textbf{N}'|\hat{U}(t) \}\hat{\rho}_{S}(0)\langle \textbf{E}, \textbf{N}|\hat{U}^{\dagger}(t)|\textbf{E}', \textbf{N}'\rangle      \} \\
        & = \text{Tr}_{s}\{ \text{Tr}_{B}\{|\textbf{E},\textbf{N}\rangle \langle \textbf{E}',\textbf{N}'|\hat{U}(t)  \} \hat{\rho}_{S}(0) \text{Tr}_{B}\{\hat{U}^{\dagger}(t) |\textbf{E}',\textbf{N}' \rangle \langle \textbf{E},\textbf{N}| \}     \}.
    \end{align*}    

Por ende, $P_{t}(\textbf{E}',\textbf{N}'|\textbf{E},\textbf{N})$ se puede expresar como
\begin{equation*}
    P_{t}(\textbf{E}',\textbf{N}'|\textbf{E},\textbf{N}) = \text{Tr}_{S}\{\hat{M} \hat{\rho}_{s}(0)\hat{M}^{\dagger} \}  \hspace{10mm} \hat{M} = \text{Tr}_{B}\{|\textbf{E},\textbf{N} \rangle \langle \textbf{E}',\textbf{N}' | \hat{U}(t)\}.
\end{equation*}

Al contar con una distribución de probabilidad, se puede construir la función generadora de momentos, a partir de la cuál pueden obtenerse cantidades como el promedio o la varianza. Esta función es

\begin{equation}
    \Lambda(\vec{\lambda},\vec{\chi}) \equiv \int d\textbf{Q} d\textbf{W}P(\textbf{Q},\textbf{W}) e^{-i\vec{\lambda}\cdot \textbf{Q} -i\vec{\chi}\cdot \textbf{W} },
\label{sec2funciongeneradora}
\end{equation}
que se puede escribir en función de la evolución de una matriz densidad auxiliar \ref{apendix:fcs1}

\begin{equation}
    \Lambda(\vec{\lambda},\vec{\chi}) = \text{Tr}\{\hat{\rho}_{tot}(\vec{\lambda},\vec{\chi};t) \}    \hspace{14mm} \hat{\rho}_{tot}(\vec{\lambda},\vec{\chi};t) = \hat{U}(\vec{\lambda},\vec{\chi};t) \hat{\rho}_{tot}(0) \hat{U}^{\dagger}(\vec{\lambda},\vec{\chi};t),
    \label{sec2:evolucionconteo}
\end{equation}
y 

\begin{equation*}
    \hat{U}(\vec{\lambda},\vec{\chi};t) = e^{\frac{i}{2}\sum_{\alpha}[\lambda_{\alpha}(\hat{H}_{\alpha} - \mu_{\alpha}\hat{N}_{\alpha} ) + \chi_{\alpha}\mu_{\alpha}\hat{N}_{\alpha} ]  } \hat{U}(t) e^{-\frac{i}{2}\sum_{\alpha}[ \lambda_{\alpha}(\hat{H}_{\alpha} - \mu_{\alpha}\hat{N}_{\alpha}) + \chi_{\alpha}\mu_{\alpha}\hat{N}_{\alpha} ]},
\end{equation*}
$\vec{\lambda}$ y $\vec{\chi}$ se denominan los parámetros de conteo o \textit{Counting Fields}, y $\hat{\rho}(\vec{\lambda},\vec{\chi};t)$ se denomina la matriz densidad generalizada. Conocer la evolución de esta matriz permite acceder a los momentos del calor, el trabajo y en general, del observable que se quiera estudiar. Sin embargo, dado que esta matriz se define a partir de la matriz densidad total, el siguiente paso consiste en derivar una ecuación maestra efectiva para los grados de libertad del sistema reducido.

\label{sec2:estadistica2puntos}



\section{Formalismo de Superoperadores y Espacio de Liouville}
Un operador en el espacio de Hilbert, representado por $\hat{\rho}$ y de dimensión $N\times N$, puede mapearse al espacio de Liouville mediante  un vector columna $|\rho \rangle \rangle$ de dimension $N^{2}\times 1$. De forma análoga, un superoperador \( L \), que actúa sobre \( \hat{\rho} \) en el espacio de Hilbert, se convierte en una matriz \( \check{L} \) de dimensión \( N^2 \times N^2 \) que actúa sobre \( |\rho\rangle\rangle \) en el espacio de Liouville. En este espacio se definen las siguientes operaciones

\begin{equation*}
    \langle \langle A|B\rangle \rangle  \equiv \text{Tr}\{\hat{A}^{\dagger}\hat{B}\}
\end{equation*}

\begin{equation*}
    \check{1}  \equiv \sum_{n,n'}|nn'\rangle \rangle \langle \langle nn'|
\end{equation*}

\begin{equation*}
     |nn'\rangle \rangle   \to |n\rangle \langle n'|  \hspace{10mm}  \langle \langle nn'| \to |n'\rangle \langle n|,
\end{equation*}
además, se cumple que 

\begin{equation*}
     \langle \langle nn'|mm'\rangle \rangle  = \delta_{nm}\delta_{n'm'}
\end{equation*}

\begin{equation*}
    \langle \langle nn'|A\rangle \rangle  = \langle n|\hat{A}|n'\rangle
\end{equation*}

\begin{equation*}
     \langle \langle 1|A\rangle \rangle  = \text{Tr}\{\hat{A}\}.
\end{equation*}


Si la evolución de la matriz densidad $|\hat{\rho}(t)\rangle \rangle$ esta gobernada por el superoperador $\check{\mathcal{L}}$, entonces la evolución temporal en el espacio de Liouville es 

\begin{equation*}
    \frac{d|\hat{\rho}(t) \rangle \rangle}{dt} = \check{\mathcal{L}}|\hat{\rho}(t) \rangle \rangle,
\end{equation*}
la solución formal de esta ecuación consiste en

\begin{equation}
    |\hat{\rho}(t)\rangle \rangle = e^{\check{\mathcal{L}}t}|\hat{\rho}(0)\rangle \rangle. 
    \label{sec2liouvilleformal}
\end{equation}

Además, existen los superoperadores de proyección de Nakajima-Zwanzig, que actúan sobre la matriz densidad total. En este contexto, se define el superoperador \( \check{\mathcal{P}} \), que proyecta sobre la parte relevante de la dinámica, mientras que \( \check{\mathcal{Q}} = 1 - \check{\mathcal{P}} \) proyecta sobre la parte irrelevante\cite{zwanzig1966statistical}. Que cumplen con las propiedades

\begin{align*}
    & \check{\mathcal{P}} + \check{\mathcal{Q}} = \check{1} \\
    & \check{\mathcal{P}}^{2} = \check{\mathcal{P}} \\
    & \check{\mathcal{Q}}^{2} = \check{\mathcal{Q}} \\
    & \check{\mathcal{P}}\check{\mathcal{Q}} = \check{\mathcal{Q}}\check{\mathcal{P}} = 0.
\end{align*}    

Finalmente, la evolución temporal de la matriz densidad a través de estos proyectores es 

\begin{align*}
    \frac{d}{dt}\check{\mathcal{P}}|\hat{\rho}(t)\rangle \rangle & = \check{\mathcal{P}}\check{\mathcal{L}}\check{\mathcal{P}}|\hat{\rho}(t)\rangle \rangle  + \check{\mathcal{P}}\check{\mathcal{L}}\check{\mathcal{Q}}|\hat{\rho}(t)\rangle \rangle \\
    \frac{d}{dt}\check{\mathcal{Q}}|\hat{\rho}(t)\rangle \rangle  & = \check{\mathcal{Q}}\check{\mathcal{L}}\check{\mathcal{Q}}|\hat{\rho}(t)\rangle \rangle  + \check{\mathcal{Q}}\check{\mathcal{L}}\check{\mathcal{P}}|\hat{\rho}(t)\rangle \rangle.
\end{align*}

\label{sec2:superop}

\newpage

\section{Ecuación maestra generalizada}
Si se considera un Hamiltoniano total $\hat{H} = \hat{H}_{S} + \hat{H}_{B} + \epsilon \hat{H}_{I} = \hat{H}_{0} + \epsilon \hat{H}_{I}$, con $\epsilon$ un parámetro adimensional que posteriormente permite aplicar acoplamiento débil. Y se introduce este Hamiltoniano en la ecuación \ref{sec2:evolucionconteo}, se obtiene 

\begin{equation}
    \frac{d}{dt}\hat{\rho}_{tot}(\vec{\lambda},\vec{\chi},t) = -i[\hat{H}_{0},\hat{\rho}_{tot}(\vec{\lambda},\vec{\chi},t)] - i\epsilon[\hat{V}_{\lambda} \hat{\rho}_{tot}(\vec{\lambda},\vec{\chi},t) - \hat{\rho}_{tot}(\vec{\lambda},\vec{\chi},t)\hat{V}_{-\lambda}],
    \label{sec3:ecgeneral}
\end{equation}
con el operador $\hat{V}_{\lambda}$ descrito por

\begin{equation*}
    \hat{V}_{\lambda} = e^{-\frac{i}{2}\hat{A}(\lambda,\chi)}\hat{H}_{I}e^{\frac{i}{2}\hat{A}(\lambda,\chi)}, \hspace{14mm}  
\end{equation*}
y $\hat{A}(\lambda,\chi) = -\sum_{\alpha}[\lambda_{\alpha}(\hat{H}_{\alpha} - \mu_{\alpha}\hat{N}_{\alpha}) + \chi_{\alpha}\mu_{\alpha}\hat{N}_{\alpha} ]$. 

En el formalismo de superoperadores, \ref{sec3:ecgeneral} se transforma en 

\begin{align*}
    \frac{d}{dt}|\hat{\rho}_{tot}(\vec{\lambda},\vec{\chi},t)\rangle \rangle  & = \check{\mathcal{L}}_{\lambda}|\hat{\rho}_{tot}(\vec{\lambda},\vec{\chi},t)\rangle \rangle  \\  
        & = (\check{\mathcal{L}}_{0} + \epsilon \check{\mathcal{L}}'_{\lambda} )|\hat{\rho}_{tot}(\vec{\lambda},\vec{\chi},t)\rangle \rangle. 
\end{align*}

Si se utiliza el cuadro de interacción, esta ecuación se convierte en  

\begin{align*}
    \hat{\rho}_{totI}(\vec{\lambda},\vec{\chi},t) & = e^{-\mathcal{L}_{0}t}\hat{\rho}_{tot}(\vec{\lambda},\vec{\chi},t) \\
    & = e^{i\hat{H}_{0}t}\hat{\rho}_{tot}(\vec{\lambda},\vec{\chi},t)e^{-i\hat{H}_{0}t}.
\end{align*}

En el cuadro de interacción, el superoperador de Liouville que incorpora el parámetro de conteo se transforma en 

\begin{equation*}
    \check{\mathcal{L}}_{\lambda}(t) = e^{-\check{\mathcal{L}}_{0}t}\check{\mathcal{L}}_{\lambda}e^{\check{\mathcal{L}}_{0}t},
\end{equation*}
y la evolución temporal en el cuadro de interacción es 

\begin{equation}
    \frac{d}{dt}|\hat{\rho}_{totI}(\vec{\lambda},\vec{\chi},t)\rangle \rangle  = \epsilon \check{\mathcal{L}}_{\lambda}(t)|\hat{\rho}_{totI}(\vec{\lambda},\vec{\chi},t)\rangle \rangle,
 \label{sec2FCS:evolution}
\end{equation}
de este modo, la evolución de los grados de libertad del sistema reducido en el espacio de Hilbert queda descrita por 

\begin{multline}
    \dot{\hat{\rho}}_{IS}(\vec{\lambda},\vec{\chi},t) =  \epsilon^{2}\int_{0}^{t}ds \left[- \text{Tr}_{B}\{\hat{V}_{\lambda}(t)\hat{V}_{\lambda}(t-s)\hat{\rho}_{IS}(\vec{\lambda},\vec{\chi},t)\hat{\rho}^{eq}_{R} \} - \text{Tr}_{B}\{\hat{\rho}_{IS}(\vec{\lambda},\vec{\chi},t)\hat{\rho}^{eq}_{R}\hat{V}_{-\lambda}(t-s)\hat{V}_{-\lambda}(t) \} \right.\\
    \left. + \text{Tr}_{B}\{\hat{V}_{\lambda}(t)\hat{\rho}_{IS}(\vec{\lambda},\vec{\chi},t)\hat{\rho}^{eq}_{R}\hat{V}_{-\lambda}(t-s) \} + \text{Tr}_{B}\{ \hat{V}_{\lambda}(t-s)\hat{\rho}_{IS}(\vec{\lambda},\vec{\chi},t)\hat{\rho}^{eq}_{R}\hat{V}_{-\lambda}(t) \}  \right],
\label{ecmaestraVlambda}
\end{multline}
la demostración de esta ecuación se encuentra en el apéndice \ref{apendixsubsectionmatriz}.

Escribiendo una interacción de la forma

\begin{align*}
    \hat{V} & = \sum_{\alpha,k}\hat{S}_{\alpha,k}\hat{B}_{\alpha,k} \\
    \hat{V}_{\lambda} & = \sum_{\alpha,k}\hat{S}_{\alpha,k}\hat{B}_{\alpha,k,\lambda} \\
    \hat{B}_{\alpha,k,\lambda} & \equiv e^{(i/2)[\lambda_{\alpha}(\hat{H}_{\alpha} - \mu_{\alpha}\hat{N}_{\alpha}) + \chi_{\alpha}\mu_{\alpha}\hat{N}_{\alpha}]}\hat{B}_{\alpha,k}e^{-(i/2)[\lambda_{\alpha}(\hat{H}_{\alpha} - \mu_{\alpha}\hat{N}_{\alpha}) + \chi_{\alpha}\mu_{\alpha}\hat{N}_{\alpha}]},  
\end{align*}
donde los operadores del baño cumplen  
\begin{equation*}
    [\hat{B}_{\alpha,k},\hat{N}_{\alpha}] = n_{\alpha,k}\hat{B}_{\alpha,k},
\end{equation*}
con $n_{\alpha,k}$ el número de part\'iculas que se intercambia en la interacción $k$.
 
Y los operadores del sistema cumplen

\begin{equation*}
    \hat{U}^{\dagger}_{S}(t)\hat{S}_{\alpha,k}\hat{U}_{S}(t) = \sum_{j}e^{-i\omega_{j}t}\hat{S}_{\alpha,k;j},
\end{equation*}
con $\hat{S}_{\alpha,k;j}$ los operadores de salto y $\omega_{j}$ las frecuencias de Bohr del Hamiltoniano del sistema. 

Cabe mencionar que los operadores $\hat{S}_{\alpha,k}$ no es necesitan ser autoadjuntos, a diferencia de los  operadores de interacción considerados en \ref{sec:lindblad}.

Si se definen las funciones correlación como $C^{\alpha}_{k,k'}(s) = \text{Tr}\{e^{is\hat{H}_{\alpha} }\hat{B}^{\dagger}_{\alpha,k}e^{-is\hat{H}_{\alpha} }\hat{B}_{\alpha,t}\hat{\tau}_{\alpha}\}$, se obtiene la ecuación maestra generalizada

\begin{equation}
    \frac{d}{dt}\hat{\rho}_{IS}(\vec{\lambda},\vec{\chi},t) = - \sum_{\alpha,k,k';j,j'}e^{i(\omega_{j}-\omega_{j'})t}\int_{0}^{t}ds \mathcal{I}(s,t) ,
\label{ecmaestrafinal}
\end{equation}
donde

\begin{multline*}
    \mathcal{I}(s,t) = e^{i\omega_{j'}s} C^{\alpha}_{k,k'}(s)\hat{S}^{\dagger}_{\alpha,k;j}\hat{S}_{\alpha,k',j'}\hat{\rho}_{IS}(\vec{\lambda},\vec{\chi},t) + e^{-i\omega_{j}s}C^{\alpha}_{k,k'}(-s)\hat{\rho}_{IS}(\vec{\lambda},\vec{\chi},t)\hat{S}^{\dagger}_{\alpha,k;j}\hat{S}_{\alpha,k';j'} \\
    - e^{-i\mu_{\alpha}n_{\alpha,k}(\lambda_{\alpha} - \chi_{\alpha})}\left[e^{i\omega_{j'}s}C^{\alpha}_{k,k'}(s-\lambda_{\alpha}) + e^{-i\omega_{j}s}C^{\alpha}_{k,k'}(-s-\lambda_{\alpha})  \right]  \hat{S}_{\alpha,k';j'}\hat{\rho}_{IS}(\vec{\lambda},\vec{\chi},t)\hat{S}^{\dagger}_{\alpha,k;j}.
\end{multline*}

Está ecuación fue deducida en el Apéndice \ref{finalequation}.

\label{sec2:master}

\subsection{Resolución finita de energía}
De forma análoga a lo presentado en la sección \ref{sec:lindblad}, uno de los requisitos que se busca en la evolución de la ecuación maestra es que sea Markoviana. Para ello, se debe tomar el límite superior del integral en el tiempo, presente en la ecuación \ref{ecmaestrafinal}, hacia infinito. Esta aproximación es válida siempre que el tiempo de correlación del baño $\tau_{B}$ sea mucho menor al tiempo de relajación del sistema $\tau_{R}$. Sin embargo, en presencia de parámetros de conteo $\lambda_{\alpha}$, las funciones de correlación del baño adquieren una dependencia modificada en el tiempo, del tipo $C^{\alpha}_{k,k'}(\pm \tau - \lambda_\alpha)$. Por lo tanto, para que la aproximación markoviana sea válida en presencia de parámetros de conteo, se requiere que 

\begin{equation*}
    C^{\alpha}_{k,k'}(\pm \tau - \lambda_\alpha) \approx 0 \quad \text{para } \tau > \tau_{R}.
\end{equation*}

Esto implica que el régimen de validez de la aproximación markoviana es 

\begin{equation*}
    \tau_{B} \ll \tau_{R} \hspace{10mm} |\lambda_{\alpha}| \ll \tau_{R}.
\end{equation*}

Esto tiene repercusiones importantes, ya que implica que la resolución de diferencias de energía en la medición del calor sea finita. Esta limitación se debe a que el parámetro $\lambda_{\alpha}$ y el calor medido en el baño $Q_{\alpha}$ son variables conjugadas en la distribución de probabilidad de calor y trabajo. Esto implica que ambas obedecen una relación de incertidumbre \cite{folland1997uncertainty}. Como consecuencia las diferencias de energía del orden de $1/\tau_{R}$ dejan de ser confiables, ya que en ese régimen el valor promedio del calor es comparable con su varianza, es decir

\begin{equation*}
    \langle \Delta \lambda^{2}_{\alpha} \rangle \langle (\Delta Q_{\alpha})^{2}\rangle \geq \gamma \implies \langle (\Delta Q_{\alpha})^{2}\rangle  \geq \frac{\gamma}{\tau^{2}_{R}},
\end{equation*}
con $\gamma$ alguna constante positiva. La profundidad de este resultado radica en que al aplicar una aproximación markoviana, la evolución del sistema sufre, de forma inherente, una resolución limitada respecto al calor intercambiado con los reservorios. Esta limitación puede conducir a inconsistencias termodinámicas, ya que se pierde información sobre fluctuaciones relevantes a escalas energéticas del orden $1/\tau_{R}$. Por lo tanto, para garantizar una evolución termodinámicamente consistente, es necesario redefinir las leyes de la termodinámica considerando explícitamente la resolución finita del calor impuesta por la dinámica.

\label{sec2:finiteresol}

\subsection{Agrupación de frecuencias}
%La ecuación de Redfield no siempre preserva positividad, lo que puede generar la aparición de probabilidades negativas en la matriz densidad del sistema reducido. La forma más común de asegurar la positividad es usar la aproximación secular vista en la sección \ref{sec:lindblad} para obtener la forma GKLS, el problema de aplicar esta aproximación es que requiere que las frecuencias de Bohr estén bien separadas con respecto a $1/\tau_{R}$, por lo tanto el aplicar esta aproximación necesita que no hayan frecuencias de Bohr casi degeneradas, eliminando una parte de los efectos cuánticos, ya que se pierden las coherencias entre niveles de energías cercanos\cite{trushechkin2021unified}. Se puede considerar un esquema diferente que asegure positividad, partiendo del punto que la aproximación de Markov asegura que para dos frecuencias de transición distintas, se cumple que $|\omega_{j} - \omega_{j'}|\ll 1/\tau_{B}$ o $|\omega_{j}-\omega_{j'}|\gg 1/\tau_{R}$. Incluso, se pueden cumplir las dos opciones. Dependiendo de cuál se cumpla, podemos agrupar las frecuencias de transición en conjuntos $x_{q}$, tal que si se cumple la primera o la segunda inecuación, estan en el mismo o en diferentes grupos, matemáticamente se traduce en
La ecuación de Redfield no garantiza en general la preservación de la positividad, lo que puede llevar a la aparición de probabilidades negativas en la matriz densidad del sistema reducido. La forma más común de asegurar dicha positividad es aplicar la aproximación secular, vista en la sección \ref{sec:lindblad}, la cual conduce a una ecuación maestra en forma (GKLS).

Sin embargo, esta aproximación requiere que las frecuencias de Bohr estén bien separadas en comparación con \( 1/\tau_R \), lo cual excluye casos con frecuencias casi degeneradas. Esta condición impone una pérdida de efectos cuánticos significativos, ya que elimina los términos no seculares en la evolución temporal\cite{trushechkin2021unified}. Para enfrentar esta limitación, puede considerarse un enfoque alternativo que conserve la positividad sin eliminar los términos no seculares. 

Partimos de que la aproximación de Markov garantiza que para dos frecuencias de transición distintas, se cumple al menos una de las siguientes condiciones:
\[
|\omega_j - \omega_{j'}| \ll 1/\tau_B \quad \text{o} \quad |\omega_j - \omega_{j'}| \gg 1/\tau_R.
\]
Incluso, ambas pueden cumplirse simultáneamente. Es posible agrupar las frecuencias de transición en subconjuntos \( x_q \), tal que frecuencias que cumplen la primera desigualdad pertenecen al mismo conjunto, mientras que aquellas que cumplen la segunda pertenecen a conjuntos distintos. Esto se traduce matemáticamente en


\begin{align*}
    |\omega_{j}-\omega_{j'}| \ll 1/\tau_{B}  &\hspace{10mm} \omega_{j} \in x_{q}, \omega_{j'} \in x_{q} \\
    |\omega_{j}-\omega_{j'}| \gg 1/\tau_{R}  &\hspace{10mm} \omega_{j} \in x_{q}, \omega_{j'} \in x_{q'}.
\end{align*}

Notemos que para frecuencias $\omega_{j}$, $\omega_{j'}$ que están en distintos grupos, se cumple la aproximación secular, es decir los términos $e^{i(\omega_{j} - \omega_{j'})t}$ oscilan rápidamente, por ende en  promedio se anulan.

La funcion correlación  espectral $\Gamma^{\alpha}(\omega) = \int_{-\infty}^{\infty} ds e^{i \omega s} C^{\alpha}(s)$, representa la razón de transición del sistema entre estados con diferencia de energía $\omega$, inducida por el baño $\alpha$. Dentro de un mismo conjunto $x_{q}$, los términos $e^{i\omega_{j}s},e^{i\omega_{j'}s}$ contribuyen, respectivamente, a  $\Gamma^{\alpha}(\omega_{j})$ y $\Gamma^{\alpha}(\omega_{j'})$, es decir,  describen transiciones inducidas por el baño con energías cercanas $\omega_{j}$ y $\omega_{j'}$. Sin embargo, debido a la resolución finita del calor intercambiado con el reservorio, no es posible distinguir entre transiciones con frecuencias dentro del mismo conjunto $x_q$. Por lo tanto, en las funciones correlación espectral se deben sustituir las frecuencias individuales $\omega_{j} \in x_{q}$ por una frecuencia auxiliar $\omega_{q}$, es decir 

\begin{equation*}
    e^{i\omega_{j}s},e^{i\omega_{j'}s} \to e^{i\omega_{q}s} \hspace{10mm} |\omega_{q} - \omega_{j}| \ll 1/\tau_{B} \hspace{10mm} \forall \omega_{j} \in x_{q}.
\end{equation*}

Esta sustitución refleja la incapacidad de distinguir transiciones con energías dentro del mismo conjunto \( x_q \), y permite construir una descripción coarse-grained de la dinámica que respeta tanto la positividad como las limitaciones de resolución energética\cite{chruscinski2017brief}. \\
Usando este esquema en la ecuación \ref{ecmaestrafinal}, se obtiene la ecuación en la forma GKLS

\begin{equation*}
    \frac{d}{dt}\hat{\rho}_{IS}(\vec{\lambda},\vec{\chi},t) = -i[\hat{H}_{LS},\hat{\rho}_{IS}(\vec{\lambda},\vec{\chi},t)] + \sum_{\alpha}\tilde{\mathcal{L}}^{\chi_{\alpha},\lambda_{\alpha}}_{\alpha} \hat{\rho}_{IS}(\vec{\lambda},\vec{\chi},t),
\end{equation*}
con

\begin{equation*}
    \tilde{\mathcal{L}}^{\chi_{\alpha},\lambda_{\alpha}}_{\alpha}\hat{\rho} = \sum_{k,q}\Gamma^{\alpha}_{k}(\omega_{q}) \left[e^{i\lambda_{\alpha}\omega_{q} + i(\chi_{\alpha}-\lambda_{\alpha})\mu_{\alpha}n_{\alpha,k}}\hat{S}_{\alpha,k;q}(t)\hat{\rho}\hat{S}^{\dagger}_{\alpha,k;q}(t) - \frac{1}{2}\{\hat{S}^{\dagger}_{\alpha,k;q}(t)\hat{S}_{\alpha,k;q}(t),\hat{\rho} \} \right].
\end{equation*}

Los operadores de salto consisten en

\begin{equation*}
    \hat{S}_{\alpha,k;q}(t) = \sum_{\{j|\omega_{j}\in x_{q} \} } e^{-i\omega_{j}t}\hat{S}_{\alpha,k;j}.
\end{equation*}

Y el Hamiltoniano de \textit{Lamb Shift}

\begin{equation*}
    \hat{H}_{LS} = \sum_{\alpha,k;q} \Delta^{\alpha}_{k}(\omega_{q}) \hat{S}^{\dagger}_{\alpha,k;q}(t)\hat{S}_{\alpha,k;q}(t),
\end{equation*}
con las cantidades

\begin{equation*}
    \Gamma_{k}^{\alpha}(\omega) = \int_{-\infty}^{\infty}ds e^{i\omega s}C^{\alpha}_{k,k}(s) \hspace{10mm} \Delta^{\alpha}_{k}(\omega) = - \frac{i}{2} \int^{\infty}_{-\infty}ds e^{i\omega s} \text{sign}(s)C^{\alpha}_{k,k}(s).
\end{equation*}

En donde se asume por simplicidad $C^{\alpha}_{k,k'} \propto \delta_{k,k'}$. La demostración de esta ecuación está incluida en el apéndice \ref{apendixGKLSgeneral}. 

En el límite en que los parámetros de conteo tienden a cero, se obtiene  

\begin{equation*}
    \frac{d}{dt}\hat{\rho}_{IS}(t) = - i[\hat{H}_{LS}(t),\hat{\rho}_{IS}(t)] + \sum_{\alpha}\tilde{\mathcal{L}}_{\alpha} \hat{\rho}_{IS}(t),
\end{equation*}
con
\begin{equation*}
    \tilde{\mathcal{L}}_{\alpha} = \sum_{\{q|\omega_{q}>0\}} \sum_{k}\Gamma^{\alpha}_{k,k}(\omega_{q}) \left[ \mathcal{D}[\hat{S}_{\alpha,k,q}(t)] + e^{-\beta_{\alpha}(\omega_{q} - \mu_{\alpha}n_{\alpha,k})}\mathcal{D}[\hat{S}^{\dagger}_{\alpha,k,q}(t)]  \right].
\end{equation*}

La demostración de esta ecuación se encuentra en \ref{apendixKMS}.

Para un Hamiltoniano tiempo independiente, la ecuación maestra en el cuadro de Schrodinger es

\begin{equation}
    \frac{d}{dt}\hat{\rho}_{S} = -i [\hat{H}_{S}+ \hat{H}_{LS},\hat{\rho}_{S}(t)] + \sum_{\alpha}\mathcal{L}_{\alpha}(\hat{\rho}_{S}(t)),
\label{sec2schrodingerthermo}
\end{equation}
con

\begin{equation}
    \mathcal{L}_{\alpha} = \sum_{\{q|\omega_{q}>0\}} \sum_{k}\Gamma^{\alpha}_{k,k}(\omega_{q}) \left[ \mathcal{D}[\hat{S}_{\alpha,k,q}] + e^{-\beta_{\alpha}(\omega_{q} - \mu_{\alpha}n_{\alpha,k})}\mathcal{D}[\hat{S}^{\dagger}_{\alpha,k,q}]  \right].
\label{sec2lindbladconsistency}
\end{equation}

Hay dos límites importantes a considerar. El primero corresponde al caso en que todas las frecuencias satisfacen $|\omega_{j}-\omega_{j'}| \gg 1/\tau_{S}$, ya que se cumple que $\mathcal{D}[\hat{S}_{\alpha,k,q}] = \mathcal{D}[\hat{S}_{\alpha,k,j}]$, lo que permite realizar la aproximación secular entre todas las frecuencias de Bohr y recuperar la ecuación de Lindblad. En el caso de que se cumpla que $|\omega_{j}-\omega_{j'}| \ll 1/\tau_{B}$, todas las frecuencias se agrupan en un sólo grupo. Por lo tanto, los operadores del sistema cumplen que $\hat{S}_{\alpha,k;q} = \hat{S}_{\alpha,k}$, y la ecuación maestra queda descrita por operadores locales, lo que corresponde a la ecuación maestra local\cite{wichterich2007modeling}.
\section{Consistencia termodinámica}
Debido a la resolución finita de energía impuesta por la dinámica markoviana, para asegurar la consistencia termodinámica es necesario redefinir las leyes de la termodinámica. Como primer paso, se introduce el Hamiltoniano termodinámico $\hat{H}_{TD}$, el cual satisface la relación de conmutación

\begin{equation*}
    [\hat{S}_{\alpha,k,j},\hat{H}_{TD}] = \omega_{q}\hat{S}_{\alpha,k,j},
\end{equation*}
para todas las frecuencias $\omega_{j} \in x_{q}$. Este Hamiltoniano puede construirse mediante el Hamiltoniano $\hat{H}_{S}$, modificando sus autovalores de modo que todas las transiciones dentro de un mismo conjunto $x_q$ compartan una frecuencia $\omega_{q}$ para $\omega_{j} \in x_{q}$.

Para las leyes de la termodinámica, se redefine la energía interna

\begin{equation*}
    E(t) = \text{Tr}\{\hat{H}_{TD}\hat{\rho}(t) \}.
\end{equation*}

Finalmente, la corriente de calor y el trabajo entregado por el baño $\alpha$ se redefinen por

\begin{equation*}
    J_{\alpha}(t) = \text{Tr}\{(\hat{H}_{TD} - \mu_{\alpha}\hat{N}_S)\mathcal{L}_{\alpha}\hat{\rho}_{S}(t) \} \hspace{12mm} P_{\alpha} = \mu_{\alpha} \text{Tr}\{\hat{N}_{S}\mathcal{L}_{\alpha}\hat{\rho}_{S}(t) \}.
\end{equation*}

Está definición automáticamente cumple con la primera ley de la termodinámica, ya que al derivar la energía interna $\partial_{t}E(t) = \text{Tr}\{ \hat{H}_{TD}\partial_{t}\hat{\rho}_{S}(t) \}$ y utilizar la relación de conmutación $[\hat{H}_{TD},\hat{H}_{S} + \hat{H}_{LS}] = 0$. Se obtiene la primera ley

\begin{equation*}
    \partial_{t}E(t) = \sum_{\alpha}[J_{\alpha} + P_{\alpha}].
\end{equation*}

\subsection{Ley cero}
Usando la ecuación maestra, con los superoperadores \ref{sec2lindbladconsistency} se cumple que

\begin{equation}
    \mathcal{L}_{\alpha}e^{-\beta_{\alpha}(\hat{H}_{TD} - \mu_{\alpha}\hat{N}_{S})} = 0,
\label{sec2cerolaw}
\end{equation}
además, si los reservorios tienen la misma temperatura inversa $\beta$ y el mismo potencial químico $\mu$, el estado de Gibbs corresponde a 

\begin{equation*}
    \hat{\rho}_G = \frac{e^{-\beta(\hat{H}_{TD} - \mu \hat{N}_{S})}}{\text{Tr}\{ e^{-\beta(\hat{H}_{TD} - \mu \hat{N}_{S})}\}}.
\end{equation*}

\subsection{Segunda ley}
La tasa de producción de entropía será 

\begin{equation}
    \dot{\sigma} = - \frac{d}{dt}\text{Tr}\{\hat{\rho}_{S}(t) \ln \hat{\rho}_{S}(t) \} - \sum_{\alpha} \beta_{\alpha} J_{\alpha}(t) = -\sum_{\alpha} \text{Tr}\{(\mathcal{L}_{\alpha}\hat{\rho}_{S}(t))[\ln \hat{\rho}_{S}(t) - \ln \hat{\rho}_{G}(\beta_{\alpha},\mu_{\alpha})] \} \geq 0.
\label{sec2secondlaw}
\end{equation}

Donde en la última parte se considera el hecho de que $\rho_{G}(\beta_{\alpha},\mu_{\alpha})$ es estado estacionario de $\mathcal{L}_{\alpha}$, para utilizar la desigualdad de Spohn\cite{spohn2007irreversible}. La demostración de \ref{sec2cerolaw} y de \ref{sec2secondlaw} se encuentra en el apéndice \ref{apendix:thermolaws}. 

Con esto finalmente se logra definir las leyes de la termodinámica usando la ecuación maestra \ref{sec2schrodingerthermo}.


% ------------------------------------------------------------------------------
% NUEVO CAPÍTULO
% ------------------------------------------------------------------------------
\chapter{Flujos de información}
En este capítulo se introduce el concepto de información termodinámica, comenzando en la sección \ref{sec4:Demon} con una revisión del paradigma del Demonio de Maxwell. En la sección \ref{sec4:autonomo} se presenta el Demonio de Maxwell autónomo. Finalmente, en las secciones \ref{sec4:flujos} y \ref{sec4:flujos0} se presenta el concepto de flujos de información, utilizando herramientas como la termodinámica estocástica en contexto clásico y la matriz densidad en contexto cuántico\cite{horowitz2014thermodynamics,ptaszynski2019thermodynamics}.


\section{Demonio de Maxwell}
El demonio de Maxwell es un experimento mental esbozado por James Clerk Maxwell en su obra\cite{Maxwell_1871}. Consiste en una caja dividida en dos compartimientos, A y B, cada uno de ellos lleno con un gas ideal a temperatura $T$ y presión $P$, como se muestra en la Figura \ref{img:demon}. Entre ambos compartimientos existe una compuerta sin masa, que permite el paso de partículas de un lado a otro. Esta puerta es controlada por una entidad denominada "Demonio", la cual tiene la capacidad de abrirla y cerrarla sin costo energético. La característica esencial del Demonio es que conoce la velocidad de cada partícula en ambos compartimientos. 

Dado que la temperatura se relaciona con la velocidad promedio de las partículas, el Demonio puede discriminar entre partículas rápidas y lentas. De este modo, permite pasar del compartimiento A al B solo aquellas partículas cuya velocidad es mayor que la velocidad promedio, mientras que deja pasar del compartimiento B al A las partículas más lentas. Si se mantiene constante el número de partículas en cada compartimiento, este proceso da lugar a un aumento de la temperatura en A y una disminución en B, es decir, una diferencia \( \Delta T \) entre ambos lados.

El cambio de entropía del sistema es
 
\begin{align*}
    \Delta S & = \Delta S_{A} + \Delta S_{B} = C_{V}\left( \log \frac{T-\Delta T}{T} + \log \frac{T+\Delta T}{T} \right) \\
       & =  C_{V} \log \left( 1 - \frac{\Delta T^{2}}{T^{2}}  \right) < 0,
\end{align*}
Donde \( C_V \) representa la capacidad calorífica a volumen constante.

Del cálculo de la entropía total se concluye que, sin realizar trabajo externo, el Demonio logra una disminución de la entropía, lo cual representa una aparente violación de la segunda ley de la termodinámica.

Sin embargo, esta paradoja fue resuelta posteriormente por Rolf Landauer, quien ``exorcizó'' al Demonio al señalar que, para que este pueda conocer la velocidad de las partículas, debe realizar mediciones. El proceso de medición, y en particular el borrado de la información asociada a dichas mediciones, implica una disipación de energía. Este resultado, conocido como el principio de Landauer, restablece la validez de la segunda ley\cite{Landauer_1961}.

En sistemas no autónomos, es decir sistemas manipulados por un agente externo a través del control de cantidades macroscópicas, la paradoja de Maxwell ha sido abordada extensamente. Un ejemplo clásico es el motor de Szilard\cite{szilard1964decrease}, donde se ha logrado describir y ``exorcizar'' al Demonio cuantificando tanto la energía requerida para realizar una medición,  como el costo energético asociado al borrado de la memoria del Demonio, y el trabajo máximo que puede extraerse mediante retroalimentación (feedback) \cite{maruyama2009colloquium, sagawa2008second}. No obstante, el caso de sistemas autónomos, en los cuales no existe intervención externa y tanto la dinámica como la retroalimentación son generadas internamente, presenta características fundamentales que merecen un análisis más detallado.

\insertimage[\label{img:demon}]{ejemplos/Maxwelldemon1}{scale=0.9}{Esquema que representa al Demonio de Maxwell. Primero, consiste en los dos compartimientos que poseen el gas ideal distribuido de manera homogénea. Por último, al trasladar las partículas de un lado a otro, queda el compartimiento A con partículas frías mientras que el compartimiento B con partículas calientes. Esta figura fue tomada de \cite{link1} .}

\label{sec4:Demon}

\section{Demonio de Maxwell autónomo}
Muchos procesos físicos requieren la interacción entre un conjunto de subsistemas que forman un sistema global. Esta interacción entre los subsistemas no sólo incluye un intercambio de energía o partículas, sino que también incluye un intercambio de información a medida que se correlacionan estos subsistemas entre sí de manera autónoma, es decir, sin un factor externo el cuál realize un feedback en él.

Comprender como son utilizados estos flujos de información para hacer tareas útiles es de gran relevancia. Un ejemplo claro se encuentra en los sistemas biólogicos, donde ocurre adaptación sensorial, que corresponde al monitoreo que realiza un organismo a su ambiente mientras cambia su respuesta a él\cite{lan2012energy}. Para caracterizar este tipo de procesos,se considera un Demonio de Maxwell autónomo como un sistema bipartito. Donde una de sus partes consiste en el sistema controlado, mientras que la otra parte en el sistema que actúa como detector, el cuál ejerce control mediante una interacción física que afecta al sistema controlado.

Es fundamental que el sistema sea autónomo, es decir que el Hamiltoniano total sea tiempo independiente e intervenciones externas, tales como mediciones y feedback no son consideradas.

\label{sec4:autonomo}
\section{Descripción clásica}
Para describir un demonio de Maxwell autónomo, es necesario poder calcular la evolución de un sistema acoplado a uno o más reservorios, y por lo tanto, sujeto a las leyes de la termodinámica. 

Consideremos el caso de dos sistemas independientes, \( X \) e \( Y \), cada uno con estados discretos denotados por \( x \) e \( y \), respectivamente. Ambos sistemas presentan una dinámica interna caracterizada por saltos aleatorios entre sus respectivos estados, cuya tasa de transición está determinada por los reservorios locales a los que cada subsistema se encuentra acoplado. Estas tasas cumplen la condición de \textit{balance detallado local} \cite{van2015ensemble}.

La evolución entre estados se modela como un proceso de Markov\cite{van1992stochastic}. Para estudiar la evolución conjunta del sistema, se asume un acoplamiento bipartito: en este contexto, esto significa que si el sistema total se encuentra en un estado conjunto $(x,y)$, sólo se permiten transiciones del tipo $(x,y) \to (x,y')$ o $(x,y) \to (x',y)$, pero no transiciones simultáneas del tipo $(x,y) \to (x',y')$.

Esta estructura garantiza que la dinámica conjunta del sistema \( XY \) siga siendo markoviana, y que la probabilidad \( p(x,y) \) de encontrar al sistema en el estado \( (x,y) \) esté gobernada por la siguiente ecuación maestra:

\begin{equation*}
    d_{t}p(x,y) = \sum_{x',y'} \left[ W_{x,x'}^{y,y'}p(x',y')  - W_{x',x}^{y',y}p(x,y) \right].
\end{equation*}

Donde \( W_{x,x'}^{y,y'} \) representa la razón de transición para un salto del estado \( (x',y') \) al estado \( (x,y) \). Esta razón obedece la condición de balance detallado local:

\[
\ln \left( \frac{W_{x,x'}^{y,y'}}{W_{x',x}^{y',y}} \right) = -\frac{\epsilon_{x,y} - \epsilon_{x',y'}}{T},
\]

la cual relaciona las razones de transición con el cambio de energía \( \Delta \epsilon = \epsilon_{x,y} - \epsilon_{x',y'} \) asociado al salto. Esta energía es intercambiada con el reservorio térmico local en forma de calor.

Debido a que el sistema es bipartito, la forma de las razones de transición corresponde a

\begin{equation*}
    W_{x,x'}^{y,y'} = \left\{ \begin{array}{lcc} w_{x,x'}^{y} & si & x \neq x'; y=y' \\ \\ w_{x}^{y,y'} & si & x=x';y\neq y'\\ \\ 0 & si & x \neq x'; y \neq y' \end{array} \right..
\end{equation*}

En términos de la corriente de probabilidad 

\begin{equation*}
    J_{x,x'}^{y,y'} = W_{x,x'}^{y,y'}p(x',y') - W_{x',x}^{y',y}p(x,y),
\end{equation*}
la ecuación maestra puede reescribirse como 

\begin{equation*}
    d_{t}p(x,y) = \sum_{x',y'}J_{x,x'}^{y,y'} = \sum_{x'}J_{x,x'}^{y} + \sum_{y'}J_{x}^{y,y'}.
\end{equation*}

Donde  
\[
J_{x}^{y,y'} = w_{x}^{y,y'}\,p(x,y') - w_{x}^{y',y}\,p(x,y)
\]
representa la corriente de probabilidad correspondiente a un salto entre los estados \( y' \to y \), manteniendo fijo el estado \( x \). Es decir, describe el flujo en la dirección \( Y \) para un valor dado de \( x \). De manera análoga, $J^{y}_{x,x'}$ describe el flujo en la dirección \( X\) para un valor fijo de \(y\).  

Este hecho es importante, ya que la estructura bipartita permite separar las corrientes de probabilidad en dos componentes: una correspondiente a transiciones en la dirección \( X \), y otra en la dirección \( Y \). Esta propiedad puede ser aprovechada para estudiar funcionales de la corriente, es decir, cantidades de la forma
\[
\mathcal{A}(J) = \sum_{x,x',y,y'} J_{x,x'}^{y,y'}\, A_{x,x'}^{y,y'},
\]
que pueden ser separados en dos contribuciones

\begin{equation}
    \mathcal{A}(J) = \sum_{x\geq x';y \geq y'} J_{x,x'}^{y}A_{x,x'}^{y,y'} + \sum_{x \geq x'; y \geq y'}J_{x}^{y,y'} A_{x,x'}^{y,y'}.
\label{sec4:functionalcurrent}
\end{equation}

De este modo, la variación de \( \mathcal{A} \) se puede descomponer en dos contribuciones: una asociada a las transiciones en la dirección \( X \), y otra correspondiente a las transiciones en la dirección \( Y \).


\label{sec4:flujos}
\section{Segunda ley de la termodinámica y flujos de información}
El sistema conjunto \( XY \), al estar en contacto con un baño, constituye un sistema abierto que debe satisfacer la segunda ley de la termodinámica. Esta ley exige que la tasa de producción de entropía sea siempre mayor o igual que cero, es decir 

\begin{equation*}
    \dot{\sigma} = \partial_{t}S^{XY} + \dot{S}_{r} \geq 0. 
\end{equation*}

Donde $\partial_{t}S^{XY}$ corresponde a la derivada temporal de la entropía del sistema, que se expresa como  

\begin{equation*}
    \partial_{t}S^{XY} = \sum_{x\geq x'; y\geq y'} J_{x,x'}^{y,y'} \ln \frac{p(x',y')}{p(x,y)}.
\end{equation*}

Y el cambio de entropía en el ambiente corresponde a

\begin{equation*}
    \dot{S}_{r} = \sum_{x\geq x'; y\geq y'} J_{x,x'}^{y,y'} \ln \frac{W_{x,x'}^{y,y'}}{W_{x',x}^{y',y}},
\end{equation*}
por ende

\begin{equation*}
    \dot{\sigma} = \sum_{x\geq x'; y\geq y'} J_{x,x'}^{y,y'} \ln \frac{ W_{x,x'}^{y,y'}p(x',y')  }{ W_{x',x}^{y',y}p(x,y) } \geq 0.
\end{equation*}

Estos resultados se demuestran en el apendice \ref{apendix4:secondlaw}.  

Una vez definida la tasa de producción de entropía, es necesario cuantificar la información compartida entre los dos subsistemas. La magnitud que permite medir las correlaciones entre ellos es la información mutua, la cual captura las correlaciones estadísticas entre las variables \( X \) e \( Y \). Se define por

\begin{equation*}
    I_{xy} = \sum_{x,y} p(x,y) \ln \frac{p(x,y)}{p(x)p(y)} \geq 0. 
\end{equation*}

De este modo, cuando \( I_{XY} \) es grande, los dos subsistemas están fuertemente correlacionados, mientras que $I_{xy}$ pequeña implica que los dos sistemas se conocen poco entre ellos, \( I_{XY} = 0 \) indica que son estadísticamente independientes.

Para definir los flujos de información, es necesario calcular la variación temporal de la información mutua. Esta puede descomponerse en dos contribuciones asociadas a las dinámicas de cada subsistema:
\[
\partial_t I_{XY} = \dot{I}^{X} + \dot{I}^{Y},
\]
donde \( \dot{I}^{X} \) representa la contribución debida a la evolución del subsistema \( X \) (con \( Y \) fijo), y \( \dot{I}^{Y} \) la correspondiente al subsistema \( Y \). Que explícitamente son

\begin{align*}
    \dot{I}^{X} & = \sum_{x\geq x'; y}J_{x,x'}^{y} \ln \frac{ p(y|x) }{p(y|x')} \\
    \dot{I}^{Y} & = \sum_{x;y\geq y'} J_{x}^{y,y'} \ln \frac{p(x|y)}{ p(x|y') }.
\end{align*}

 La demostración de este resultado también está en el apendice \ref{apendix4:secondlaw}.

Los términos \( \dot{I}^{X} \) y \( \dot{I}^{Y} \) cuantifican cómo fluye la información entre los dos subsistemas. Si \( \dot{I}^{X} > 0 \), significa que un salto en la dirección \( X \), en promedio, aumenta la información mutua \( I_{xy} \); es decir, el subsistema \( X \) está ``aprendiendo'' o midiendo al subsistema \( Y \).

En cambio, si \( \dot{I}^{X} < 0 \), los saltos en la dirección \( X \) reducen las correlaciones, lo que puede interpretarse como un consumo de información con el fin de extraer trabajo o energía.

Como la tasa de producción de entropía es un funcional de la corriente de probabilidad, se puede usar la ecuación \ref{sec4:functionalcurrent} para escribir:


 \begin{equation*}
    \dot{\sigma} = \dot{\sigma}^{X} + \dot{\sigma}^{Y},
 \end{equation*}
así

\begin{align*}
    \dot{\sigma}^{X} & = \sum_{x \geq x';y} J_{x,x'}^{y} \ln  \frac{w_{x,x'}^{y} p(x',y) }{w_{x',x}^{y} p(x,y) } \geq 0 \\
    \dot{\sigma}^{Y} & = \sum_{x;y\geq y'}J_{x}^{y,y'} \ln \frac{w_{x}^{y,y'} p(x,y') }{ w_{x}^{y',y} p(x,y) } \geq 0, 
\end{align*}
donde se pueden identificar las tasas de producción de entropía locales

\begin{align*}
    \dot{\sigma}^{X} &  = \sum_{x \geq x';y} J_{x,x'}^{y} \left[ \ln \frac{p(x')}{p(x)}  +\ln \frac{w_{x,x'}^{y}}{ w_{x',x}^{y} } + \ln \frac{p(y|x')}{p(y|x)} \right] \\
    \dot{\sigma}^{Y} &  = \sum_{x;y \geq y'} J_{x}^{y,y'} \left[ \ln \frac{p(y')}{p(y)}  + \ln \frac{w_{x}^{y,y'}}{ w_{x}^{y',y} } + \ln \frac{p(x|y')}{p(x|y)} \right].
\end{align*}

Y en función de los flujos de información son  

\begin{align*}
    \dot{\sigma}^{X} & = \partial_{t}S^{X} + \dot{S}_{r}^{X} - \dot{I}^{X} \geq 0 \\
    \dot{\sigma}^{Y} & = \partial_{t}S^{Y} + \dot{S}_{r}^{Y} - \dot{I}^{Y} \geq 0.
\end{align*}

Estas dos ecuaciones permiten visualizar cómo la contribución de la información influye en la producción de entropía local de cada subsistema.

Supongamos que se desconoce la existencia de la interacción entre el subsistema \( Y \) y el subsistema \( X \), y que únicamente se monitoriza la dinámica de \( X \). En ese caso, se le asignaría al sistema una tasa de producción de entropía dada por:
\[
\dot{\sigma}^{X}_0 = \partial_t S^X + \dot{S}_r^X.
\]

Si el subsistema \( X \) estuviera efectivamente aislado, se cumpliría \( \dot{\sigma}^{X}_0 \geq 0 \), en concordancia con la segunda ley de la termodinámica. Sin embargo, debido a su interacción con \( Y \), pueden darse situaciones en las que \( \dot{\sigma}^{X}_0 < 0 \), lo cual representa una violación aparente de la segunda ley. Esta puede interpretarse como el efecto de un Demonio de Maxwell, que utiliza la información que proviene de monitorear \(X \).

En la evolución de sistemas autónomos que alcanzan un estado estacionario, se cumple que \( \partial_t I_{xy} = 0 \). En ese régimen, el flujo de información se conserva:
\[
\dot{\mathcal{I}} = \dot{I}^{X} = -\dot{I}^{Y},
\]
y la producción de entropía local de cada subsistema se puede escribir como:


\begin{align*}
    \dot{\sigma}^{X} & = \dot{S}_{r}^{X} - \dot{\mathcal{I}} \geq 0 \\
    \dot{\sigma}^{Y} & =  \dot{S}_{r}^{Y} + \dot{\mathcal{I}} \geq 0 . 
\end{align*}

Supongamos el caso en que \( \dot{\mathcal{I}} > 0 \). En esta situación, se puede interpretar que el subsistema \( X \) está actuando como un sensor, al monitorear activamente al subsistema \( Y \). Para realizar esta función de monitoreo, el subsistema \( X \) debe disipar una cantidad mínima de energía tal que:
\[
\dot{S}_{r}^{X} \geq \dot{\mathcal{I}}.
\]

Por otro lado, el subsistema \( Y \) está proporcionando información, la cual puede ser aprovechada para extraer energía del entorno. En este caso, se cumple que:
\[
-\dot{S}_{r}^{Y} \leq \dot{\mathcal{I}}.
\]

Esta relación es importante, ya que permite utilizar información para realizar trabajo útil en sistemas fuera del equilibrio, o incluso para extraer calor de un reservorio caliente.

Un ejemplo característico de este fenómeno se encuentra en sistemas sometidos a un gradiente de potencial, donde el Demonio de Maxwell se manifiesta a través del surgimiento de una corriente de partículas en contra del gradiente. Este efecto será analizado en detalle en las secciones posteriores.


\section{Descripción cuántica}
Para describir los flujos de información en un contexto cuántico, es necesario conocer la dinámica de un sistema cuántico abierto acoplado a uno o varios reservorios. Para ello, se utilizará la matriz densidad, la cual describe la evolución del sistema reducido. El sistema total estará descrito por el Hamiltoniano $\hat{H} = \hat{H}_{S} + \hat{H}_{B} + \hat{H}_{I}$. Y la evolución markoviana de este sistema conduce a una ecuación maestra

\begin{equation*}
    \partial_{t}\hat{\rho}_{S}(t) = - i[\hat{H}_{S} + \hat{H}_{LS},\hat{\rho}_{S}] + \mathcal{L}(\hat{\rho}_{S}),
\end{equation*}

El acoplamiento del sistema con múltiples reservorios $\alpha$ se modela a través de un generador de Lindblad, que se descompone como
\begin{equation*}
    \mathcal{L} = \sum_{\alpha} \mathcal{L}_{\alpha},
\end{equation*}
donde cada superoperador $\mathcal{L}_{\alpha}$ representa la interacción del sistema con el reservorio $\alpha$.

Si se asume que cada reservorio induce un equilibrio local descrito por un estado de Gibbs
\begin{equation*}
    \hat{\rho}_{\text{eq}}^{\alpha} = \frac{1}{Z_{\beta_\alpha, \mu_\alpha}} e^{-\beta_\alpha (\hat{H}_S - \mu_\alpha \hat{N})},
\end{equation*}
entonces es posible aplicar la desigualdad de Spohn \cite{spohn1978entropy}, para obtener una formulación generalizada de la segunda ley de la termodinámica, en términos de una desigualdad de Clausius para la tasa de producción de entropía local asociada al reservorio $\alpha$, es decir 

\begin{equation}
    - \text{Tr}[ (\mathcal{L}_{\alpha} \hat{\rho}_{S})(\ln \hat{\rho}_{S} - \ln \hat{\rho}^{\alpha}_{eq} )  ] \geq 0.
\label{spohninfo}
\end{equation}

Cabe señalar, que este resultado se mantiene válido si el estado estacionario esta descrito por el Hamiltoniano termodinámico $\hat{H}_{\mathrm{TD}}$, en lugar del Hamiltoniano físico $\hat{H}_S$.

Si se define la cantidad
\begin{equation}
    \dot{S}^{\alpha} = - \mathrm{Tr} \{ (\mathcal{L}_{\alpha} \hat{\rho}_S) \ln \hat{\rho}_S \},
\end{equation}
como la tasa de cambio de la entropía de von Neumann del sistema inducida por el acoplamiento al reservorio $\alpha$, con la entropía definida por
\begin{equation}
    S = - \mathrm{Tr} \{ \hat{\rho}_S \ln \hat{\rho}_S \},
\end{equation}
y se identifica el flujo de calor en el segundo término de la desigualdad de Spohn (ver ecuación~\ref{spohninfo}), se obtiene una expresión para la tasa de producción de entropía local asociada al reservorio $\alpha$:
\begin{equation}
    \dot{\sigma}^{\alpha} = \dot{S}^{\alpha} - \beta_{\alpha} \dot{Q}^{\alpha} \geq 0.
\end{equation}
Esta desigualdad corresponde a una forma local de la segunda ley de la termodinámica, donde $\dot{Q}^{\alpha}$ es el flujo de calor hacia el sistema desde el reservorio $\alpha$, y $\beta_{\alpha}$ su inversa de temperatura. El término $\dot{\sigma}^{\alpha}$ representa la producción neta de entropía asociada a dicho contacto, y es 

\begin{equation}
    \dot{\sigma}^{\alpha} = \dot{S}^{\alpha} - \beta_{\alpha} J_{\alpha} \geq 0,
\label{sec4:localentropy}
\end{equation}
que actúa como una inecuación de Clausius parcial. Esto es importante, ya que localmente se puede separar la producción de entropía en cantidades mayor a cero, similar a como se hizo en la descripción clásica. De hecho si se suman todas las tasas de cambio de entropía, se obtiene la derivada total $\partial_{t} S = \sum_{\alpha} \dot{S}^{\alpha}$, por lo tanto al sumar todas las producciones de entropía locales

\begin{equation*}
    \sum_{\alpha}\dot{\sigma}^{\alpha} = \partial_{t}S - \sum_{\alpha}\beta_{\alpha}J_{\alpha} = \dot{\sigma} \geq 0,
\end{equation*}
que corresponde a la inecuación de Clausius estándar.

Se puede notar que para el caso estacionario $\partial_{t}S = 0$, no necesariamente $\dot{S}^{\alpha}$ es cero. Sino que depende del flujo de calor que entra localmente por el disipador $\alpha$. También se puede tratar la inecuación para la energía libre, primero definiendo la corriente de energía y calor correspondientes al baño $\alpha$

\begin{align*}
    \dot{E}_{\alpha} & = \text{Tr}[ (\mathcal{L}_{\alpha} \hat{\rho}_{S}) \hat{H}_{S}] \\
    \dot{W}_{\alpha} & = \mu_{\alpha}\text{Tr}[ (\mathcal{L}_{\alpha} \hat{\rho}_{S}) \hat{N}].
\end{align*}

En este caso se usa el Hamiltoniano $\hat{H}_{S}$ para escribir la tasa de cambio de la energía interna, sin embargo si el estado estacionario contiene $\hat{H}_{TD}$, entonces el flujo de energía es $ \dot{E}_{\alpha} = \text{Tr}[ (\mathcal{L}_{\alpha} \hat{\rho}_{S}) \hat{H}_{TD}]$.

Se cumple que la suma de las corrientes de energía provenientes de los distintos reservorios corresponde a la derivada temporal de la energía interna del sistema

\begin{equation*}
    \sum_{\alpha}\dot{E}_{\alpha} = \partial_{t}E.
\end{equation*}

En el caso de un Hamiltoniano independiente del tiempo, no se realiza trabajo mecánico, por lo que sólo se considera trabajo químico. En consecuencia, la energía intercambiada con cada reservorio se descompone como

\begin{equation*}
    \dot{E}_{\alpha} = J_{\alpha} + \dot{W}_{\alpha}.
\end{equation*}

Multiplicando la ecuación \ref{sec4:localentropy} por la temperatura del reservorio $T_{\alpha}$ y reemplazando $J_{\alpha}$ mediante la expresión anterior, se obtiene

\begin{equation}
    T_{\alpha} \dot{\sigma}^{\alpha} = \dot{W}_{\alpha} - \dot{\mathcal{F}}_{\alpha} \geq 0,
\label{sec4:localfreerate}
\end{equation}
donde se ha definido la tasa de variación de la energía libre asociado al reservorio $\alpha$ como 

\begin{equation*}
    \dot{\mathcal{F}}_{\alpha} = \dot{E}_{\alpha} - T_{\alpha}\dot{S}^{\alpha}.
\end{equation*}

La inecuación \ref{sec4:localfreerate} consiste en una inecuación local de energía libre. Si se suman las inecuaciones de todos los reservorios, se tendrá 

\begin{equation}
    \sum_{\alpha}T_{\alpha} \dot{\sigma}^{\alpha} = \dot{W} - \dot{\mathcal{F}} \geq 0,
\label{sec4:freeratefinal}
\end{equation}
donde $\dot{W} = \sum_{\alpha}\dot{W}_{\alpha}$ y la tasa de cambio de energía libre total es

\begin{equation*}
    \dot{\mathcal{F}} = \partial_{t}E - \sum_{\alpha}T_{\alpha} \dot{S}^{\alpha}.
\end{equation*}

Para el caso particular, en qué todas las temperaturas son iguales, la tasa de cambio de energía libre $\dot{\mathcal{F}}$ se vuelve equivalente a la derivada temporal de la energía libre $\partial_{t}F = \partial_{t}(E-TS)$. Sólo en este caso \ref{sec4:localentropy} y \ref{sec4:freeratefinal} son equivalentes.

En el caso isotérmico, se cumple en el estado estacionario que $\dot{\mathcal{F}} = \partial_{t}F = 0$, lo que implica $\dot{W}>0$. Por lo tanto, el sistema no puede realizar trabajo neto si sólo va a intercambiar  calor con reservorios a la misma temperatura. La única forma de obtener trabajo neto del sistema, o sea, lograr que $\dot{W}<0$, es mediante la presencia de reservorios con distinta temperatura.  En tal caso, la tasa de cambio de energía libre no se anula y es $\dot{\mathcal{F}} = - \sum_{\alpha} T_{\alpha} \dot{S}^{\alpha} \neq 0$.   

\section{Flujos de información en contexto cuántico}
Suponga un sistema compuesto por dos subsistemas acoplados, descrito por el Hamiltoniano total

\begin{equation*}
    \hat{H}_{S} = \hat{H}_{X} + \hat{H}_{Y} + \hat{H}_{XY}, 
\end{equation*}
donde $\hat{H}_{i}$ representa el Hamiltoniano del subsistema $i=X,Y$, y $\hat{H}_{XY}$ corresponde al término de interacción entre los dos subsistemas. Se asume, además, que cada subsistema está acoplado a un conjunto distinto de reservorios. En particular, los reservorios asociados al subsistema $i$ serán denotados por $\alpha_{i}$. Esto permite describir la tasa de producción de entropía local de cada subsistema como

\begin{equation*}
    \dot{\sigma}^{i} \equiv \sum_{\alpha_{i}} \dot{\sigma}^{\alpha_{i}} = \sum_{\alpha_{i}} \dot{S}^{\alpha_{i}} - \sum_{\alpha_{i}} \beta_{\alpha_{i}} J_{\alpha_{i}} \geq 0,
\end{equation*}
por lo tanto, la tasa de producción de entropía total del sistema corresponde a $\dot{\sigma} = \dot{\sigma}^{X} + \dot{\sigma}^{Y}$. 

De forma análoga a lo realizado en la descripción clásica, se busca establecer una relación entre las producciones locales de entropía y la información compartida entre subsistemas. Para ello, se utilizará la información mutua entre los dos subsistemas, definida como
 
\begin{equation*}
    I_{xy} = S_{X} + S_{Y} - S_{XY},
\end{equation*}
donde $S_{i} = - \text{Tr}\{\hat{\rho}_{i}\ln \hat{\rho_{i}}\}$ es la entropía de Von Neumann del subsistema $i=X,Y$, y $\hat{\rho}_{i}$ consiste en la matriz densidad del subsistema $i$. 

Se puede separar la derivada temporal de la información mutua

\begin{align*}
    \partial_{t}I_{xy} & = \partial_{t}S_{X} + \partial_{t}S_{Y} - \partial_{t}S_{XY} \\
        & = \partial_{t}S_{X} + \partial_{t}S_{Y} - \sum_{i=X,Y;\alpha_{i}}\dot{S}^{\alpha_{i}} \\
        & =  \partial_{t}S_{X} - \sum_{\alpha_{X}} \dot{S}^{\alpha_{X}} + \partial_{t}S_{Y} - \sum_{\alpha_{Y}} \dot{S}^{\alpha_{Y}} \\
        & = \dot{I}^{X} + \dot{I}^{Y},
\end{align*}
donde aparece el flujo de información correspondiente a cada subsistema $i=X,Y$, como la cantidad $\dot{I}^{i} = \partial_{t}S^{i} - \sum_{\alpha_{i}} \dot{S}^{\alpha_{i}} $. Luego, si se reemplaza $\sum_{\alpha_{i}}\dot{S}^{\alpha_{i}} \to \partial_{t}S_{i} - \dot{I}^{i}$, en las tasas de producción de entropía locales, quedan las inecuaciones  

\begin{align*}
    \dot{\sigma}^{X} = \partial_{t}S^{X} - \sum_{\alpha_{X}} \beta_{\alpha_{X}} J_{\alpha_{X}} - \dot{I}^{X} \geq 0 \\
    \dot{\sigma}^{Y} = \partial_{t}S^{Y} - \sum_{\alpha_{Y}} \beta_{\alpha_{Y}} J_{\alpha_{Y}} - \dot{I}^{Y} \geq 0,
\end{align*}
este resultado coincide con el obtenido en la descripción clásica. No obstante, su formulación en términos de entropía de von Neumann permite incorporar de forma natural los efectos cuánticos presentes en el sistema, lo cual resulta fundamental para caracterizar los flujos de información en regímenes donde las correlaciones cuánticas y las coherencias desempeñan un rol relevante.

Si se considera que el subsistema $i$ esta conectado a a un reservorio isotermico de temperatura $T_i$, la tasa de cambio de la energía libre será
\begin{align*}
    \dot{\mathcal{F}}_{i} & = \dot{E}_{i} - T_{i} \sum_{\alpha_{i}}\dot{S}^{\alpha_{i}}. 
\end{align*}

En el estado estacionario se cumple la relación

\begin{equation*}
    \dot{I}^{i} = - \sum_{\alpha_{i}}\dot{S}^{\alpha_{i}},
\end{equation*}
y la tasa de cambio de energía libre del subsistema $i$ es 

\begin{equation*}
    \dot{\mathcal{F}}_{i} = \dot{E}_{i} + T_{i} \dot{I}^{i}.
\end{equation*}

Así, la tasa de cambio de la energía libre posee dos contribuciones: una correspondiente al intercambio de energía y otra asociada al flujo de información. Si se considera el régimen estacionario, donde la energía total permanece constante, es decir
\begin{equation*}
    \partial_{t}E = \dot{E}^{X} + \dot{E}^{Y} = 0,
\end{equation*}
entonces los términos $\dot{E}^{i}$ pueden interpretarse como flujos de energía entre los dos subsistemas. En este contexto, la tasa de cambio de energía libre para cada subsistema se descompone en una parte asociada al intercambio energético y otra vinculada al flujo de información generado por las correlaciones entre los subsistemas.

En el caso $\dot{E}^{X} = \dot{E}^{Y} \approx 0 $, esta descomposición permite que el sistema realice trabajo a partir de una contribución dominante de la información, incluso en ausencia de diferencias térmicas significativas. Tal mecanismo constituye lo que se conoce como una máquina de la información, en la que los flujos de información entre subsistemas juegan un rol principal, permitiendo la extracción de trabajo en un régimen puramente informacional.


\label{sec4:flujos0}

\chapter{Dinámica de un sistema de 3 puntos cuánticos conectado al ambiente}
En este capítulo se estudia un sistema compuesto por tres puntos cuánticos acoplados a múltiples reservorios térmicos. La evolución del sistema se analiza mediante la ecuación de Lindblad desarrollada en la sección \ref{sec2lindbladconsistency}. En la sección \ref{sec5:modelo} se introduce el Hamiltoniano del sistema reducido y su entorno. Posteriormente, en la sección \ref{sec5:transporte}, se analiza el comportamiento del sistema en un escenario típico de transporte cuántico. En la sección \ref{sec5:demonio1}, se establecen las condiciones bajo las cuales la dinámica del sistema puede interpretarse como la de un Demonio de Maxwell autónomo. Finalmente, en la sección \ref{sec5:demonio}, se compara este comportamiento con su contraparte clásica, destacando las diferencias inducidas por los efectos cuánticos.

\section{Modelo teórico}
El sistema reducido está compuesto por tres puntos cuánticos. Uno de ellos posee un nivel de energía $\epsilon_d$ y se encuentra acoplado, mediante una interacción de Coulomb $U$, a los otros dos puntos cuánticos. Estos últimos tienen una energía de sitio $\epsilon$ y están conectados entre sí por un término de hopping $g$, además de una interacción de Coulomb $U_f$. Cada uno de los niveles de energía se encuentra acoplado a un baño distinto. 

La configuración del sistema reducido se ilustra en la Figura~\ref{img:sistema3puntos}.


\insertimage[\label{img:sistema3puntos}]{ejemplos/semilocal1}{scale=0.36}{Esquema que representa el sistema reducido compuesto por tres puntos cuánticos, modelados como tres niveles de energía. Estos niveles presentan interacción de Coulomb repulsiva entre ellos, y cada uno está acoplado a un baño térmico distinto.}

El Hamiltoniano del sistema reducido esta dado por 

\begin{align*}
    \hat{H}_{S} & = \epsilon_{d}\hat{d}^{\dagger}_{D}\hat{d}_{D} + \epsilon \hat{d}^{\dagger}_{L}\hat{d}_{L} + \epsilon \hat{d}^{\dagger}_{R}\hat{d}_{R} + g(\hat{d}^{\dagger}_{L}\hat{d}_{R} + \hat{d}^{\dagger}_{R}\hat{d}_{L} ) \\
          & + U(\hat{n}_{D}\hat{n}_{L} + \hat{n}_{D}\hat{n}_{R} )  + U_{f}\hat{n}_{R}\hat{n}_{L},
\end{align*}
donde $\hat{n}_{i} = \hat{d}^{\dagger}_{i} \hat{d}_{i}$ es el operador de número del sitio $i$, y los operadores $\hat{d}_{i}$ satisfacen las relaciones de anticonmutación

\begin{equation*}
    \{\hat{d}_{\alpha},\hat{d}_{\beta} \} = 0  \hspace{10mm} \{\hat{d}^{\dagger}_{\alpha}, \hat{d}_{\beta} \} = \delta_{\alpha \beta}.
\end{equation*}

Usando la base de Fock $|n_{L},n_{R},n_{D} \rangle = (\hat{d}^{\dagger}_{L})^{n_{L}}(\hat{d}^{\dagger}_{R})^{n_{R}}(\hat{d}^{\dagger}_{D})^{n_{D}}|0,0,0\rangle$, y escribiendo el Hamiltoniano en el orden de la base 

\begin{equation*}
     \{|0,0,0\rangle, |1,0,0\rangle, |0,1,0\rangle, |0,0,1\rangle, |1,1,0\rangle, |1,0,1\rangle, |0,1,1\rangle, |1,1,1\rangle\},
\end{equation*}
se puede representar el Hamiltoniano del sistema en forma matricial 

\begin{equation*}
    \hat{H}_{S} = 
    \begin{bmatrix}
        0 & 0 & 0 & 0 & 0 & 0 & 0 & 0 \\
        0 & \epsilon_{L} & g & 0 & 0 & 0 & 0 & 0 \\
        0 & g & \epsilon_{R} & 0 & 0 & 0 & 0 & 0 \\
        0 & 0 & 0 & \epsilon_{d} & 0 & 0 & 0 & 0 \\
        0 & 0 & 0 & 0 & \epsilon_{L} + \epsilon_{R}  + U_{f} & 0 & 0 & 0 \\
        0 & 0 & 0 & 0 & 0 & \epsilon_{L} + \epsilon_{d} + U & g & 0 \\
        0 & 0 & 0 & 0 & 0 & g & \epsilon_{R} + \epsilon_{d} + U & 0 \\
        0 & 0 & 0 & 0 & 0 & 0 & 0 & \epsilon_{L} + \epsilon_{R}  + \epsilon_{d} + 2U + U_{f} 
        \end{bmatrix}.
\end{equation*}

Por otro lado, el Hamiltoniano del baño y la interacción con el sistema se descomponen como $\hat{H}_{B} = \hat{H}_{L}+\hat{H}_{R}+\hat{H}_{D}$ y $\hat{V} = \hat{V}_{L}+\hat{V}_{R}+\hat{V}_{D}$ respectivamente. Cada término esta dado por 

\begin{equation*}
    \hat{H}_{\alpha} = \sum_{l} \epsilon_{\alpha l} \hat{c}^{\dagger}_{\alpha l}\hat{c}_{\alpha l} \hspace{10mm} \hat{V}_{\alpha} = \sum_{l} t_{\alpha l} ( \hat{d}^{\dagger}_{\alpha} \hat{c}_{\alpha l} + \hat{c}^{\dagger}_{\alpha l} \hat{d}_{\alpha} ),
\end{equation*}
donde $\alpha = L,R,D$ denota los tres reservorios acoplados a cada punto cuántico, $\hat{c}_{\alpha,l}$ son los operadores de aniquilación de los modos del reservorio $\alpha$, $\epsilon_{\alpha l}$ sus energías, y $t_{\alpha l}$ los parámetros de acoplamiento entre el sistema y baño. 

A partir del término de interacción $\hat{V}$, es posible identificar los operadores asociados al sistema y al baño

\begin{equation*}
    \hat{S}_{\alpha,1} = \hat{d}^{\dagger}_{\alpha} \hspace{10mm} \hat{S}_{\alpha,-1} = \hat{d}_{\alpha}
\end{equation*}

\begin{equation*}
    \hat{B}_{\alpha,1} = \sum_{l}t_{\alpha l} \hat{c}_{\alpha l} \hspace{10mm} \hat{B}_{\alpha,-1} = \sum_{l}t_{\alpha l} \hat{c}^{\dagger}_{\alpha l}.
\end{equation*}

La evolución no unitaria del sistema de tres puntos cuánticos, inducida por su acoplamiento a los baños, se describe mediante una ecuación maestra descrita por el operador de Lindblad

\begin{equation}
    \mathcal{L} = \mathcal{L}_{R} + \mathcal{L}_{L} + \mathcal{L}_{D},
    \label{Lindbladsec5}
\end{equation}
donde el operador de Lindblad $\mathcal{L}$ se descompone como una suma de contribuciones individuales, una por cada baño acoplado al sistema. Explícitamente cada contribución $\mathcal{L}_{i}$ corresponde a 

\begin{align*}
    \mathcal{L}_{R} & = \gamma_{R}(\epsilon)(f_{R}(\epsilon)\mathcal{D}_{0}[\hat{d}^{\dagger}_{R}(\textbf{1}-\hat{n}_{D})(\textbf{1}-\hat{n}_{L}) ]  + [1-f_{R}(\epsilon)]\mathcal{D}_{0}[\hat{d}_{R}(\textbf{1}-\hat{n}_{D})(\textbf{1}-\hat{n}_{L}) ]  )  \\
                    & + \gamma_{R}(\epsilon+U)(f_{R}(\epsilon+U)\mathcal{D}_{0}[\hat{d}^{\dagger}_{R}\hat{n}_{D}(\textbf{1}-\hat{n}_{L}) ]  + [1-f_{R}(\epsilon+U)]\mathcal{D}_{0}[\hat{d}_{R}\hat{n}_{D}(\textbf{1}-\hat{n}_{L}) ]  ) \\
                   & + \gamma_{R}(\epsilon+U_{f})(f_{R}(\epsilon+U_{f})\mathcal{D}_{0}[\hat{d}^{\dagger}_{R}(\textbf{1}-\hat{n}_{D})\hat{n}_{L} ]  + [1-f_{R}(\epsilon+U_{f})]\mathcal{D}_{0}[\hat{d}_{R}(\textbf{1}-\hat{n}_{D})\hat{n}_{L} ]  ) \\
                  & + \gamma_{R}(\epsilon+U+U_{f})(f_{R}(\epsilon+U+U_{f})\mathcal{D}_{0}[\hat{d}^{\dagger}_{R}\hat{n}_{D}\hat{n}_{L} ]  + [1-f_{R}(\epsilon+U+U_{f})]\mathcal{D}_{0}[\hat{d}_{R}\hat{n}_{D}\hat{n}_{L} ]  ) 
\end{align*}

\begin{align*}
    \mathcal{L}_{L} & = \gamma_{L}(\epsilon)(f_{L}(\epsilon)\mathcal{D}_{0}[\hat{d}^{\dagger}_{L}(\textbf{1}-\hat{n}_{D})(\textbf{1}-\hat{n}_{R}) ]  + [1-f_{L}(\epsilon)]\mathcal{D}_{0}[\hat{d}_{L}(\textbf{1}-\hat{n}_{D})(\textbf{1}-\hat{n}_{R}) ]  )  \\
                    & + \gamma_{L}(\epsilon+U)(f_{L}(\epsilon+U)\mathcal{D}_{0}[\hat{d}^{\dagger}_{L}\hat{n}_{D}(\textbf{1}-\hat{n}_{R}) ]  + [1-f_{L}(\epsilon+U)]\mathcal{D}_{0}[\hat{d}_{L}\hat{n}_{D}(\textbf{1}-\hat{n}_{R}) ]  ) \\
                   & + \gamma_{L}(\epsilon+U_{f})(f_{L}(\epsilon+U_{f})\mathcal{D}_{0}[\hat{d}^{\dagger}_{L}(\textbf{1}-\hat{n}_{D})\hat{n}_{R} ]  + [1-f_{L}(\epsilon+U_{f})]\mathcal{D}_{0}[\hat{d}_{L}(\textbf{1}-\hat{n}_{D})\hat{n}_{R} ]  ) \\
                  & + \gamma_{L}(\epsilon+U+U_{f})(f_{L}(\epsilon+U+U_{f})\mathcal{D}_{0}[\hat{d}^{\dagger}_{L}\hat{n}_{D}\hat{n}_{R} ]  + [1-f_{L}(\epsilon+U+U_{f})]\mathcal{D}_{0}[\hat{d}_{L}\hat{n}_{D}\hat{n}_{R} ]  ) 
\end{align*}

\begin{align*}
    \mathcal{L}_{D} & = \gamma_{D}(\epsilon_{d})(f_{D}(\epsilon_{d})\mathcal{D}_{0}[\hat{d}^{\dagger}_{D}(\textbf{1}-\hat{n}_{R})(\textbf{1}-\hat{n}_{L}) ]  + [1-f_{D}(\epsilon_{d})]\mathcal{D}_{0}[\hat{d}_{R}(\textbf{1}-\hat{n}_{D})(\textbf{1}-\hat{n}_{L}) ]  )  \\
                    & + \gamma_{D}(\epsilon_{d}+U)f_{D}(\epsilon_{d}+U)\mathcal{D}_{0}[\hat{d}^{\dagger}_{D}(\hat{n}_{R}(\textbf{1}-\hat{n}_{L}) + \hat{n}_{L}(\textbf{1}-\hat{n}_{R})) ]  \\
                    & + \gamma_{D}(\epsilon_{d}+U)[1-f_{D}(\epsilon_{d}+U)]\mathcal{D}_{0}[\hat{d}_{D}(\hat{n}_{R}(\textbf{1}-\hat{n}_{L}) + \hat{n}_{L}(\textbf{1}-\hat{n}_{R}))]   \\
                   & + \gamma_{D}(\epsilon_{d}+2U)(f_{D}(\epsilon_{d}+2U)\mathcal{D}_{0}[\hat{d}^{\dagger}_{D}\hat{n}_{R}\hat{n}_{L} ]  + [1-f_{D}(\epsilon_{d}+2U)]\mathcal{D}_{0}[\hat{d}_{D}\hat{n}_{R}\hat{n}_{L} ]  ).
\end{align*}

Con $f_{i}(\omega) = (\exp(\beta_{i}(\omega - \mu_{i})) + 1)^{-1}$ y el operador $\mathcal{D}_{0}$  

\begin{equation*}
    \mathcal{D}_{0}[\hat{A}]\hat{\rho} = \hat{A}\hat{\rho}\hat{A}^{\dagger}- \frac{1}{2} \{\hat{A }^{\dagger}\hat{A},\hat{\rho} \}.
\end{equation*}

La demostración de la ecuación \ref{Lindbladsec5} se presenta en los apéndices \ref{apendix5jumpop} y \ref{apendix5frecuencygroup}.

Despreciando el término de \textit{Lamb Shift} $\hat{H}_{LS}$, como se discute en~\cite{prech2023entanglement}, la evolución del sistema de tres puntos cuánticos está dada por
\begin{equation}
    \frac{d}{dt}\hat{\rho}_{S}(t) = -i[\hat{H}_{S},\hat{\rho}_{S}(t)] + \mathcal{L}(\hat{\rho}_{S}(t)),
\end{equation}
donde $\hat{\rho}_{S}(t)$ es la matriz densidad del sistema reducido. A partir de esta ecuación, es posible calcular numéricamente la evolución del sistema y obtener las magnitudes termodinámicas relevantes.

\label{sec5:modelo}

\newpage

\section{Estudio del transporte}
Se aborda inicialmente el caso en que las razones de túnel no dependen de la frecuencia, es decir, $\gamma_{i}(\omega) = \gamma_{i}$. Y se estudia el flujo de corriente generado por una diferencia de potencial $\mu_{L} - \mu_{R} = eV$, empleando las razones de túnel $\gamma_{L} = \gamma_{R} = 1/100$ y $\gamma_{d} = 1/50$. Los resultados se muestran en la Figura~\ref{img:transporte1}.

\insertimage[\label{img:transporte1}]{ejemplos/particletp.pdf}{scale=0.35}{Corriente de partículas en función de la razón \( eV/T \). Las temperaturas de los reservorios son \( T_{L} = T_{R} = 100 \) y \( T_{d} = 2 \). El parámetro de acoplamiento se fija en \( g = \frac{5}{1000} \). Las energías de sitio corresponden a \( \epsilon = 0 \) y \( \epsilon_{d} = \mu_{d} - \frac{U}{2} \), mientras que las interacciones de Coulomb se fijan en \( U_{F} = 500 \) y \( U = 40 \).}

La figura muestra que, a medida que aumenta la diferencia de potencial \( eV \), la corriente de partículas \( \dot{N}_L = -\dot{N}_R \) que fluye del baño \( L \) al baño \( R \) también aumenta. Este comportamiento es consistente con lo esperado, ya que un mayor potencial químico en el reservorio \( L \) impulsa el transporte hacia \( R \). Asimismo, se puede analizar la potencia suministrada al sistema en función de \( eV \).

\insertimage[\label{img:transporte2}]{ejemplos/worktp.pdf}{scale =0.35}{Potencia entregada por los reservorios en función de $eV/T$.}
    
La Figura \ref{img:transporte2} muestra que el conjunto de reservorios \( L \) y \( R \) entrega trabajo al sistema para mantener la corriente de partículas. Este comportamiento es consistente desde el punto de vista físico, ya que el paso de electrones desde un potencial químico mayor hacia uno menor implica una transferencia de energía al sistema, que se manifiesta como trabajo. Por otro lado, dado que el reservorio \( D \) no intercambia partículas en promedio, la potencia que este entrega es nula.

 Asimismo, pueden examinarse los flujos de calor que reciben los sistemas $LR$ y $D$


\insertimage[\label{img:transporte3}]{ejemplos/heatlrdtp.pdf}{scale =0.35}{Flujo de calor recibido por los sistemas $LR$ y $D$ en función de $eV/T$.}
    
Como uno puede ver, para $eV/T \lessapprox 2$ , el sistema $LR$ en total recibe calor de los reservorios $L$ y $R$. Para $eV/T \gtrapprox 2$ el sistema $LR$ en total entrega calor a los reservorios $L$ y $R$. 

Para completar el análisis termodinámico a partir de la primera ley $\dot{E}_{\alpha} = J_{\alpha} + \dot{W}_{\alpha}$, se grafican a continuación los flujos de energía involucrados

\insertimage[\label{img:transporte4}]{ejemplos/energytp.pdf}{scale =0.4}{Flujo de energía entregado por los reservorios en función de $eV/T$.}

En el estado estacionario, la condición \( \partial_t E = 0 \) implica que \( \dot{E}_{LR} = -\dot{E}_d \), lo cual se observa en la Figura \ref{img:transporte4}. Allí, el flujo de energía hacia el sistema \( LR \) es positivo, mientras que el flujo desde el subsistema \( D \) es negativo. Esto indica que el sistema \( LR \) recibe energía proveniente de los reservorios \( L \) y \( R \), la cual es posteriormente transferida al subsistema \( D \) y desde allí al reservorio \( D \). Este comportamiento es coherente con el hecho de que la temperatura del reservorio \( D \) es menor (\( T_{d} < T_{L} = T_{R} \)), y por lo tanto, el flujo de calor se dirige desde los reservorios más calientes hacia el más frío.

Para analizar la presencia de efectos cuánticos en el estado estacionario, se calcularán dos cantidades relevantes. La primera de ellas es la norma-\( l_{1} \), \( \mathcal{C}_{l_{1}} \), que permite cuantificar las coherencias en el sistema \( LR \).


\begin{equation*}
    \mathcal{C}_{l_{1}} = \sum_{i> j} |\hat{\rho}_{S;i,j}|,
\end{equation*}
otro efecto cuántico relevante de estudiar es el entrelazamiento entre modos específicos del sistema. En particular, se analiza el entrelazamiento entre los pares de estados \(|1,0,0\rangle, |0,1,0\rangle\) y \(|1,0,1\rangle, |0,1,1\rangle\). Para cuantificar este tipo de correlaciones no clásicas se utiliza la concurrencia \( \mathcal{C}_{\text{on}} \) \cite{hill1997entanglement, wootters1998entanglement}, definida como

\begin{equation*}
    \mathcal{C}_{\text{on}} = \max \left\{ 2|\alpha + \beta| - 2\sqrt{p_0 p_D},\, 0 \right\},
\end{equation*}
donde \( p_0 \) y \( p_D \) representan las probabilidades de que el sistema \( LR \) se encuentre vacío o doblemente ocupado, respectivamente. El término \( \alpha \) corresponde a la coherencia entre los estados \(|1,0,0\rangle\) y \(|0,1,0\rangle\), mientras que \( \beta \) representa la coherencia entre los estados \(|1,0,1\rangle\) y \(|0,1,1\rangle\). La deducción de esta fórmula se presenta en el Apéndice \ref{appendix5final}. 

Finalmente, en la Figura \ref{img:transporte5} se analiza cómo varían la coherencia y la concurrencia en función de la diferencia de potencial

\insertimage[\label{img:transporte5}]{ejemplos/quantumtp.pdf}{scale =0.4}{Coherencia(línea azul) y concurrencia(linea roja) en función de $eV/T$.}
    
En este caso, la coherencia comienza en cero y aumenta progresivamente a medida que se incrementa la diferencia de potencial. En cuanto al entrelazamiento, también parte desde cero —coincidiendo con el valor nulo de la coherencia— y se mantiene nulo hasta alcanzar un cierto umbral, a partir del cual la concurrencia comienza a crecer con el aumento de \( eV/T \). Esto indica la aparición de entrelazamiento en el sistema \( LR \). Este comportamiento es relevante, ya que sugiere que la presencia de una corriente en el sistema contribuye a que se desarrollen coherencia y entrelazamiento entre los sitios $L$ y $R$.  


\label{sec5:transporte}

\newpage

\section{Demonio de Maxwell en 3 puntos cuánticos}
Para que el sistema de tres puntos cuánticos se comporte como un Demonio de Maxwell autónomo, es fundamental tratarlo como un sistema bipartito compuesto por dos subsistemas: por un lado, el subsistema $LR$, que consiste en dos niveles de energía $\epsilon$; por otro, el nivel $\epsilon_D$, que desempeña el rol de demonio.

Para que este último actúe como demonio, debe ser capaz de detectar cambios en el subsistema $LR$, ejecutando una acción análoga a una medición. Gracias a la interacción de Coulomb $U$, el demonio puede distinguir si el subsistema $LR$ está vacío, ocupado o doblemente ocupado. A su vez, el subsistema $LR$ modifica su dinámica dependiendo del estado de ocupación del demonio.

Esta retroalimentación se implementa matemáticamente al modificar las razones de túnel $\gamma_{i}(\epsilon + U) \neq \gamma_{i}(\epsilon)$, con $i = L, R$. 

Para que el demonio pueda detectar eficazmente el estado del subsistema $LR$, deben cumplirse dos condiciones. Primero, que la energía de interacción supere las fluctuaciones térmicas del reservorio asociado al demonio, es decir, $\beta_{D} U \gg 1$. Segundo, que su dinámica disipativa sea más rápida que la del subsistema $LR$, lo que implica $\gamma_{D} > \max\{\gamma_{i}, \gamma^{U}_{i}\}$ con $i = L, R$.

La dinámica específica del demonio se describirá con mayor detalle en la Figura~\ref{img:dinamica1}.

\insertimage[\label{img:dinamica1}]{ejemplos/diseño1}{scale=0.6}{Primer paso en la dinámica, el sitio $D$ y el sitio $R$ poseen una mayor probabilidad de estar ocupados.}

La elección $\epsilon_D = \mu_D - U/2$ favorece la ocupación del nivel de energía del demonio. Para inducir una corriente en dirección opuesta al gradiente de potencial, es conveniente imponer que $\gamma^{U}_{R} > \gamma^{U}_{L}$, de modo que las transiciones sean más frecuentes hacia el baño derecho, aumentando así la probabilidad de que este nivel se pueble. Posteriormente, como se ilustra en la Figura \ref{img:dinamica2}, el acoplamiento $g$ permite el intercambio coherente del electrón entre los niveles $L$ y $R$, introduciendo así una dinámica esencialmente cuántica.


\insertimage[\label{img:dinamica2}]{ejemplos/diseño2}{scale=0.6}{Segundo paso en la dinámica, el hopping $g$ genera superposición entre los sitios $L$ y $R$.}

Finalmente, se produce una tercera etapa, ilustrada en la Figura \ref{img:dinamica3}. En esta fase, con el objetivo de generar un flujo de corriente en contra del gradiente de potencial, resulta conveniente imponer que $\gamma_{L} > \gamma_{R}$. Esta condición asegura que, cuando el demonio se encuentra desocupado, la probabilidad de que ocurra una transición hacia el nivel $L$ sea mayor que hacia el nivel $R$. Como consecuencia, se favorece el transporte de electrones desde el reservorio $R$ (de menor potencial) hacia el reservorio $L$ (de mayor potencial), evidenciando así un comportamiento característico de un Demonio de Maxwell autónomo.


\insertimage[\label{img:dinamica3}]{ejemplos/diseño3}{scale=0.6}{Tercer paso en la dinámica, el electrón se transporta al baño $L$.}

Teniendo definida la dinámica del sistema, es posible elegir de forma adecuada distintos candidatos para las razones de túnel con el objetivo de inducir un comportamiento tipo Demonio de Maxwell. Para ello, se estudiará el comportamiento de la matriz densidad, así como los flujos de corriente de partículas, calor, información y potencia, en función de la diferencia de potencial $\mu_{L} - \mu_{R} = eV$.

Se comenzará analizando las probabilidades de ocupación extraídas de la matriz densidad. Como se muestra en la Figura~\ref{img:densitymatrix}, las tres probabilidades más altas corresponden a los estados $\rho_{001}$, $\rho_{100}$ y $\rho_{010}$, lo cual es coherente con la presencia de interacción de Coulomb, que penaliza las configuraciones con ocupación múltiple.

Para valores de $eV/T \lessapprox 3.6$, la probabilidad más alta corresponde al estado $\rho_{001}$, lo que indica que el sitio $D$ es, en promedio, el más ocupado. Sin embargo, a medida que la diferencia de potencial aumenta ($eV/T \gtrapprox 3.6$), las probabilidades asociadas a los estados $\rho_{100}$ y $\rho_{010}$ —correspondientes a los sitios $L$ y $R$— superan a $\rho_{001}$, lo que refleja un cambio en la ocupación predominante del sistema.

Debido a que la interacción de Coulomb penaliza los estados con más de un electrón, cabría esperar que la probabilidad de vacío $\rho_{000}$ fuese del mismo orden de magnitud que $\rho_{100},\rho_{010},\rho_{001}$. Sin embargo, debido a la constante inyección de partículas inducida por la diferencia de potencial, esta probabilidad resulta ser significativamente menor. 

Se observa además un cruce entre las curvas de $\rho_{101}$ y $\rho_{011}$ en $eV/T \approx 2.4$. Para valores $eV/T \lessapprox 2.4$, la configuración con interacción Coulomb entre los puntos $R$ y $D$ (representada por $\rho_{011}$) es más probable. En cambio, para $eV/T \gtrapprox 2.4$, la ocupación $\rho_{101}$ domina, indicando una mayor interacción entre los puntos $L$ y $D$.


\insertimage[\label{img:densitymatrix}]{ejemplos/matrixdemon.pdf}{scale=0.4}{Elementos diagonales de la matriz densidad en función de $eV/T$.}

El comportamiento de un Demonio de Maxwell autónomo se refleja en las cantidades termodinámicas del sistema. En la Figura~\ref{img:fig1resultado} se muestran: (a) los flujos de calor correspondientes a cada reservorio, y (b) las corrientes de partículas asociadas. Cabe destacar que, dado que no existe un término de hopping entre el sistema y el demonio, la corriente de partículas en el demonio es nula, $\dot{N}_{d} = 0$. Además, por conservación de partículas, se cumple que $\dot{N}_{L} = -\dot{N}_{R}$, lo cual implica que la influencia del demonio sobre el sistema ocurre exclusivamente a través de flujos de energía o información.

Dado que $\mu_L - \mu_R = eV$, se esperaría que $\dot{N}_L > 0$ para todo valor de $eV$. No obstante, en el régimen $eV/T \lessapprox 2.4$, se observa que $\dot{N}_L < 0$, lo que indica un flujo de partículas en contra del gradiente de potencial. Este cambio de signo en la corriente coincide con el cruce entre las probabilidades $\rho_{101}$ y $\rho_{011}$, lo que es consistente con el comportamiento tipo Maxwell. En particular, como se ilustró en la Figura~\ref{img:dinamica1}, dicho régimen requiere una interacción simultánea entre los puntos $R$ y $D$. Sin embargo, al dominar $\rho_{101}$, esta dinámica se ve suprimida.

Si se analiza únicamente el subsistema $LR$, la existencia de una corriente contra el gradiente químico representa una violación aparente de la segunda ley de la termodinámica. Para este análisis, se considera la tasa de producción de entropía efectiva del subsistema, dada por $\dot{\sigma}^{o}_{LR} = d_{t}S_{LR} - \sum_{i=L,R} J_i/T$, cuya evolución en función de $eV/T$ se muestra en la Figura~\ref{img:fig2resultado}.


\insertimage[\label{img:fig1resultado}]{ejemplos/heatparticledemon.pdf}{scale=0.47}{(a) Flujos de calor liberados por cada reservorio $\alpha$. (b) Flujos de corriente de partículas correspondientes a cada reservorio. Los resultados se obtuvieron considerando temperaturas $T = 100$ y $T_d = 2$, niveles de energía $\epsilon = 0$ y $\epsilon_d = \mu_d - U/2$ con potencial químico $\mu_d = 2$, razones de túnel $\gamma_L = \gamma^{U}_{R} = 1/100$, $\gamma_R = \gamma^{U}_{L} = 1/600$, $\gamma_D = \gamma^{U}_{D} = 1/50$, acoplamiento entre los sitios $g = 5/1000$ e interacciones de Coulomb $U = 40$ y $U_F = 500$.}

\insertimage[\label{img:fig2resultado}]{ejemplos/entropyapparentdemon.pdf}{scale=0.45}{Tasa de producción de entropía aparente del sistema $LR$($\dot{\sigma}^{o}_{LR}$) en función de $eV/T$.}

En la Figura \ref{img:fig2resultado} se grafica la tasa de producción de entropía aparente, donde se observa que para $eV/T \lessapprox 2.4$, la tasa de producción de entropía aparente $\dot{\sigma}^{o}_{LR}$ es menor a cero, lo que implica una violación aparente de la segunda ley de la termodinámica. Sin embargo, al incorporar la contribución del flujo de información en el cálculo de la tasa de producción de entropía, se obtiene la Figura \ref{img:fig3resultado}, en la cual se verifica que la producción total de entropía en el sistema $LR$ permanece siempre positiva.

\insertimage[\label{img:fig3resultado}]{ejemplos/entropyrialdemon.pdf}{scale=0.45}{Tasa de producción de entropía del sistema $LR$($\dot{\sigma}_{LR}$) en función de $eV/T$.}

En el régimen de parámetros en que \( eV/T \lessapprox 2.4 \), el subsistema \( LR \) realiza trabajo. Para determinar si este trabajo es impulsado principalmente por el flujo de información, es necesario analizar el intercambio energético entre el sistema y el demonio. En el estado estacionario se cumple:

\[
\dot{E}_{L} + \dot{E}_{R} + \dot{E}_{d} = \partial_{t}E = \dot{E}_{LR} + \dot{E}_{d} = 0.
\]

Para que el sistema pueda interpretarse como un Demonio de Maxwell, el trabajo realizado por el subsistema \( LR \) debe estar dominado por la contribución informacional. Esta condición se puede expresar como:

\[
\dot{\mathcal{F}}_{LR} = \dot{E}_{LR} + T\dot{I}_{LR} = -\dot{E}_{d} + T\dot{I}_{LR} \approx T\dot{I}_{LR}, \hspace{10mm} \dot{W}_{LR} \geq \dot{\mathcal{F}}_{LR},
\]

donde \( \dot{\mathcal{F}}_{LR} \) representa la tasa de cambio de energía libre del subsistema \( LR \), incluyendo una contribución informacional \( T\dot{I}_{LR} \).

A fin de verificar este comportamiento, se grafican las cantidades termodinámicas relevantes tanto del sistema como del demonio.


\insertimage[\label{img:fig4resultado}]{ejemplos/thermoquantitiesdemon}{scale=0.53}{ Flujo de energía, tasa de cambio de energía libre, flujo de información, potencia y flujo de calor del sistema $LR$. (b)  Flujo de energía, tasa de cambio de energía libre, flujo de información y potencia del Demonio. }

La Figura \ref{img:fig4resultado} muestra que, en el régimen en el que el subsistema $LR$ realiza trabajo ($\dot{W}_{LR} < 0$ y $\dot{\mathcal{F}}_{LR} < 0$), la contribución informacional domina sobre el flujo energético, es decir, $\dot{\mathcal{F}}_{LR} \approx T \dot{I}_{LR}$ mientras que $\dot{E}_{LR} \approx 0$. Este resultado respalda la interpretación del dispositivo como un Demonio de Maxwell autónomo.

La potencia negativa $\dot{W}_{LR} < 0$ indica que el sistema está realizando trabajo en contra del gradiente de potencial, lo cual requiere absorber calor del entorno. En efecto, se verifica que $J_{LR} > 0$, señalando que el subsistema $LR$ extrae calor de sus respectivos reservorios.

Dado que no hay transporte neto de partículas a través del Demonio ($\dot{N}_{d} = 0$), el flujo de energía asociado se reduce a $J_{d} = \dot{E}_{d}$. Como se observa en la figura, $J_{d} < 0$, lo que implica que el Demonio disipa calor hacia su baño, caracterizado por una temperatura menor a los reservorios $L$ y $R$ ($T_d < T$). 

Este comportamiento puede entenderse también como el de una máquina térmica, donde el sistema total extrae calor del reservorio caliente ($J_{LR}$), transforma parte de él en trabajo útil ($\dot{W}_{LR}$), y disipa el resto en el baño frío ($J_{d}$), cumpliendo así $J_{LR} + J_{d} = -\dot{W}_{LR}$.

Al estar cada punto cuántico acoplado a un reservorio distinto, es posible descomponer el flujo de información total como $\dot{I}_{LR} = \dot{I}_{L} + \dot{I}_{R}$, lo que permite analizar la contribución informacional de cada punto cuántico por separado. Esta descomposición se ilustra en la Figura \ref{img:figinforesultado}.


\insertimage[\label{img:figinforesultado}]{ejemplos/infolrddemon.pdf}{scale=0.45}{Flujos de información de cada sitio, $L$,$R$ y $D$. }

Es interesante notar que, cuando \( eV/T \gtrapprox 3.6 \), el menor flujo de información proviene del punto cuántico \( R \), mientras que para \( eV/T \lessapprox 3.6 \), el sitio \( L \) es el menor, y por ende entrega más información a $D$. Esta transición se correlaciona con los resultados observados en la Figura~\ref{img:densitymatrix}, donde se muestra que, a medida que aumenta \( eV/T \), la probabilidad de ocupación del estado \( \rho_{110} \) supera a la de \( \rho_{011} \).

Este comportamiento es consistente con la intuición física: un mayor potencial favorece la ocupación del punto cuántico \( L \), mientras que la asimetría en las razones de túnel está diseñada para promover la ocupación de \( R \). Esta competencia reduce la eficacia de la dinámica asociada al mecanismo tipo Demonio de Maxwell, descrita en las Figuras~\ref{img:dinamica1},~\ref{img:dinamica2} y~\ref{img:dinamica3}.

La clave de este mecanismo es que el subsistema \( D \) debe ser capaz de monitorear el estado de ocupación del punto \( R \), para luego permitir una transición hacia \( L \) cuando se encuentra desocupado. Sin embargo, al aumentar la probabilidad de ocupación de \( L \), se interrumpe esta dinámica, lo que reduce el flujo de información desde ese sitio. Como consecuencia, el flujo informacional asociado a \( L \) crece con \( eV/T \), llegando a superar el de \( R \).

Un aspecto relevante de esta dinámica es la posible presencia de efectos cuánticos. Para examinar este fenómeno, se representan dos cantidades: la norma-$l_{1}$ de coherencia \( \mathcal{C}_{l_{1}} \), que permite cuantificar la coherencia cuántica en el subsistema \( LR \), y la concurrencia, que sirve como medida del entrelazamiento entre estados del subsistema $LR$ $(|1,0>,|0,1>)$.

\insertimage[\label{img:fig5resultado}]{ejemplos/quantumdemon.pdf}{scale=0.40}{En línea azul la norma-$l_{1}$ de la coherencia $\mathcal{C}_{l_{1}}$ y en línea rojas la concurrencia $\mathcal{C}_{on}$.}

Se observan dos comportamientos relevantes. Primero, para \( eV/T \lessapprox 1 \), la concurrencia \( \mathcal{C}_{\text{on}} > 0 \), lo que indica la presencia de entrelazamiento entre los sitios \( L \) y \( R \). Segundo, al analizar la coherencia, se aprecia que para \( eV/T \lessapprox 2.4 \) esta disminuye progresivamente hasta anularse, y luego comienza a incrementarse nuevamente a medida que crece \( eV/T \). Este comportamiento es particularmente significativo, ya que la coherencia se anula en un punto cercano al valor en el que la corriente de partículas invierte su dirección.

Dado el comportamiento tipo Demonio de Maxwell del sistema, resulta pertinente examinar cómo estas cantidades cuánticas se ven influenciadas por el parámetro de acoplamiento \( g \). Para ello, se comenzará analizando las cantidades termodinámicas.

\insertimage[\label{img:fig6resultado}]{ejemplos/thermoquantum.pdf}{scale=0.47}{Cantidades termodinámicas para el sistema $LR$ y el Demonio en función de $g/\gamma_{L}$, considerando $eV/T=1$ y el resto de parámetros corresponden a los de la Figura \ref{img:fig1resultado}. }

Se observa que cuando el acoplamiento \( g \) alcanza valores comparables con la tasa de túnel \( \gamma_{L} \), tanto la potencia generada como la contribución informacional del subsistema \( LR \) aumentan. Este resultado sugiere que un acoplamiento suficientemente fuerte entre el sistema y el demonio potencia la conversión de información en trabajo. 

Además, bajo esta misma condición \( g \sim \gamma_{L} \), el valor absoluto del flujo de energía del demonio \( \dot{E}_{d} \) también se incrementa, lo que indica mayor transporte energético entre el demonio y el sistema $LR$.

Por otra parte, el análisis de las contribuciones individuales a los flujos de información revela cómo el acoplamiento modifica la cantidad de información entregada por cada punto cuántico. Estos efectos se visualizan en la Figura \ref{img:fig7resultado}, donde se aprecia la dependencia de los flujos informacionales con respecto al parámetro \( g \).

\insertimage[\label{img:fig7resultado}]{ejemplos/concuinfodemon.pdf}{scale=0.51}{(a)Flujos de información de cada sitio. (b)Concurrencia y coherencia en función de $g/\gamma_{L}$. }

Un comportamiento notable se presenta cuando el acoplamiento alcanza valores comparables a la tasa de túnel, es decir, \( g \sim \gamma_{L} \). En este régimen, tanto la concurrencia como la coherencia alcanzan valores cercanos a sus máximos, lo que indica una presencia significativa de efectos cuánticos entre los puntos \( L \) y \( R \), resultado que se asemeja a \cite{prech2023entanglement}. 

Simultáneamente, los flujos de información individuales tienden a igualarse, es decir, \( \dot{I}_{L} \approx \dot{I}_{R} \), lo que sugiere que, en condiciones de entrelazamiento y coherencia elevadas, ambos puntos cuánticos contribuyen de manera equilibrada en el flujo de información del sistema $LR$. 


\label{sec5:demonio1}

\newpage 

\section{Comparación con caso clásico}
Una de las ventajas del formalismo presentado en \cite{potts2021thermodynamically} es que permite incluir términos no seculares en la dinámica del sistema. Esto posibilita la aparición de coherencias y entrelazamiento no nulos en el régimen \( g \sim \gamma_{L} \). La pregunta natural que surge es si estos efectos cuánticos representan una ventaja para el funcionamiento del demonio. Para abordar esta cuestión, se puede considerar una evolución puramente(semi) clásica, limitada a las componentes diagonales de la matriz de densidad del sistema, la cual está gobernada por

\begin{equation}
    \frac{d}{dt}\check{\mathcal{P}}|\hat{\rho}_{S}(t)\rangle \rangle = (\check{\mathcal{L}}_{0} - \check{\mathcal{P}}\check{\mathcal{V}}\check{\mathcal{Q}}\check{\mathcal{L}}^{-1}_{0}\check{\mathcal{Q}}\check{\mathcal{V}}\check{\mathcal{P}})\check{\mathcal{P}}|\hat{\rho}_{S}(t)\rangle \rangle, 
\label{ec5:classicalmodel}
\end{equation}

En este contexto, la evolución del sistema está gobernada por el superoperador total $\check{\mathcal{L}}_{f}$, definido como
\begin{equation*}
\check{\mathcal{L}}_{f}|\hat{\rho}\rangle\rangle = -i[\hat{H}_{S}, \hat{\rho}] + \mathcal{L}(\hat{\rho}),
\end{equation*}
que combina la dinámica coherente del sistema con la disipación inducida por los baños. Este superoperador puede descomponerse como $\check{\mathcal{L}}_{f} = \check{\mathcal{L}}_{0} + \check{\mathcal{V}}$, donde el término $\check{\mathcal{V}}$ representa el acoplamiento entre los sitios y actúa como
\begin{equation*}
\check{\mathcal{V}}|\hat{\rho}\rangle\rangle = -ig[\hat{d}_{L}^{\dagger} \hat{d}_{R} + \hat{d}_{R}^{\dagger} \hat{d}_{L}, \hat{\rho}].
\end{equation*}
Por otro lado, el operador inverso $\check{\mathcal{L}}_{0}^{-1}$ se define a través del inverso de Drazin del operador $\check{\mathcal{L}}_{0}$ como:
\begin{equation*}
\check{\mathcal{L}}_{0}^{-1} = -\int_{0}^{\infty} d\tau\, e^{\check{\mathcal{L}}_{0} \tau} \check{\mathcal{Q}},
\end{equation*}
según lo discutido en~\cite{landi2024current}. Finalmente, $\check{\mathcal{P}}$ y $\check{\mathcal{Q}}$ son los operadores de proyección de Nakajima-Zwanzig sobre las partes diagonal y no diagonal de la matriz densidad del sistema, respectivamente. La demostración detallada de esta ecuación se encuentra en el Apéndice~\ref{appendix5clasic}.

El hecho de disponer de una ecuación maestra que describe únicamente la evolución de las partes diagonales de la matriz densidad permite modelar el sistema de manera estocástica. Esto se logra mediante un vector de probabilidades de ocupación definido como:
\begin{equation*}
\mathbf{P} = [\rho_{000}, \rho_{100}, \rho_{010}, \rho_{001}, \rho_{110}, \rho_{101}, \rho_{011}, \rho_{111}]^{T},
\end{equation*}
cuya dinámica está gobernada por la ecuación maestra

\begin{equation}
    \frac{d}{dt}\textbf{P} = \textbf{W}\textbf{P},
    \label{sec5:classicsemi}
\end{equation}
donde la matriz de transición $\mathbf{W}$ puede obtenerse a partir de la ecuación~(\ref{ec5:classicalmodel}), y se descompone como $\mathbf{W} = \sum_{\alpha} \mathbf{W}_{\alpha}$, donde $\mathbf{W}_{\alpha}$ representa la contribución del baño $\alpha$. 

En lo que sigue, se estudiará el estado estacionario de la ecuación maestra presentada en la ecuación~\ref{sec5:classicsemi}. En particular, se compararán las probabilidades de ocupación obtenidas en este modelo clásico con aquellas resultantes del tratamiento cuántico, lo cual se muestra en las Figuras~\ref{img:rho100} y~\ref{img:rho011}. 

\insertimage[\label{img:rho100}]{ejemplos/rho1004.pdf}{scale=0.46}{Elementos diagonales $\rho_{100}$, $\rho_{010}$, $\rho_{111}$ y $\rho_{110}$ en función de $eV/T$. Donde el caso clásico corresponde a la línea roja y el caso cuántico a la línea azul.}
\insertimage[\label{img:rho011}]{ejemplos/rho1014.pdf}{scale=0.46}{Elementos diagonales $\rho_{101}$, $\rho_{000}$, $\rho_{011}$ y $\rho_{001}$ en función de $eV/T$.}

De las Figuras \ref{img:rho100} y \ref{img:rho011} se observa que en los dos casos, las probabilidades de ocupación dominantes corresponden a $\rho_{100},\rho_{010}$ y $\rho_{001}$. Por el contrario, la probabilidad de ocupación más pequeña es $\rho_{111}$.
 
Se puede notar que, cerca del punto $eV/T \approx 2.4$ donde la coherencia es casi nula, todas las probabilidades de ocupación clásicas se cruzan con sus contrapartes cuánticas. En cambio, para $eV/T \lessapprox 1$, los elementos $\rho_{100}$, $\rho_{010}$ y $\rho_{101}$ presentan una diferencia apreciable entre los comportamientos cuántico y clásico, lo cual se debe al efecto de la coherencia sobre estos elementos.


A partir de la solución estacionaria del sistema, es posible calcular los flujos de corriente de partículas e información asociados al baño $\alpha$ mediante las expresiones

\begin{equation}
    \dot{N}_{\alpha} = -\mathbf{N}^{T} \mathbf{W}_{\alpha}\mathbf{P},
    \label{sec5:currentsemi}
\end{equation}
\begin{equation}
    \dot{I}_{\alpha} = -(\log \mathbf{P})^{T} \mathbf{W}_{\alpha}\mathbf{P},
    \label{sec5:infosemi}
\end{equation}

donde $\mathbf{P}$ es el vector de probabilidades en la base de Fock utilizada, $\mathbf{N}$ contiene el número de partículas en cada estado, y $\log \mathbf{P}$ se refiere al logaritmo aplicado componente a componente.

En el modelo clásico, la expresión analítica para el flujo de corriente de partículas correspondiente al baño $L$ es idéntica al cuántico. Usando la expresión \ref{sec5:currentsemi}, la corriente de partículas esta dada por  

\begin{align*}
    \dot{N}_{L} & = \gamma_{L}(\epsilon)\big(f_{L}(\epsilon)\rho_{000} - [1-f_{L}(\epsilon)]\rho_{100} \big) \\
        & + \gamma_{L}(\epsilon + U) \big(f_{L}(\epsilon+U)\rho_{001} - [1-f_{L}(\epsilon+U)]\rho_{101} \big)  \\
        & + \gamma_{L}(\epsilon + U_{f}) \big(f_{L}(\epsilon+U_{f})\rho_{010} - [1-f_{L}(\epsilon+U_{f})]\rho_{110} \big)  \\  
        & + \gamma_{L}(\epsilon + U + U_{f}) \big(f_{L}(\epsilon+U + U_{f})\rho_{011} - [1-f_{L}(\epsilon+U+U_{f})]\rho_{111} \big).
\end{align*}

Por otro lado, mediante la expresión \ref{sec5:infosemi} se obtienen los flujos de información clásicos asociados a los baños $L$ y $R$, los cuales están descritos por

\begin{align*}
  \dot{I}_{L}  &  =  \gamma_{L}(\epsilon)(f_{L}(\epsilon)[\rho_{000}\ln \rho_{100} - \rho_{000}\ln \rho_{000}] + (1-f_{L}(\epsilon))[\rho_{100}\ln \rho_{000} - \rho_{100}\ln \rho_{100} ] )  \\ 
      & + \gamma_{L}(\epsilon + U)(f_{L}(\epsilon + U)[\rho_{001}\ln \rho_{101} - \rho_{001}\ln \rho_{001}] + (1-f_{L}(\epsilon + U))[\rho_{101}\ln \rho_{001} -\rho_{101}\ln \rho_{101} ]   ) \\  
      & + \gamma_{L}(\epsilon + U_{f})( f_{L}(\epsilon + U_{f})[\rho_{010}\ln \rho_{110} -\rho_{010}\ln \rho_{010} ]  + (1-f_{L}(\epsilon + U_{f}))[\rho_{110}\ln \rho_{010} - \rho_{110}\ln \rho_{110} ] ) \\  
      & + \gamma_{L}(\epsilon + U + U_{f})( f_{L}(\epsilon + U + U_{f})[\rho_{011}\ln \rho_{111}- \rho_{011}\ln \rho_{011}]) \\   
      & + \gamma_{L}(\epsilon + U + U_{f})( (1-f_{L}(\epsilon+U+U_{f}) )[\rho_{111}\ln \rho_{011} - \rho_{111}\ln \rho_{111}])   
\end{align*}

\begin{align*}
 \dot{I}_{R} & =  \gamma_{R}(\epsilon)(f_{R}(\epsilon)[\rho_{000}\ln \rho_{010} - \rho_{000}\ln \rho_{000}] + (1-f_{R}(\epsilon))[\rho_{010}\ln \rho_{000} - \rho_{010}\ln \rho_{010} ] )  \\ 
      & + \gamma_{R}(\epsilon + U)(f_{R}(\epsilon + U)[\rho_{001}\ln \rho_{011} - \rho_{001}\ln \rho_{001}] + (1-f_{R}(\epsilon + U))[\rho_{011}\ln \rho_{001} -\rho_{011}\ln \rho_{011} ]   ) \\  
      & + \gamma_{R}(\epsilon + U_{f})( f_{R}(\epsilon + U_{f})[\rho_{100}\ln \rho_{110} - \rho_{100}\ln \rho_{100} ]  + (1-f_{R}(\epsilon + U_{f}))[\rho_{110}\ln \rho_{100} - \rho_{110}\ln \rho_{110} ] ) \\  
      & + \gamma_{R}(\epsilon + U + U_{f})( f_{R}(\epsilon + U + U_{f})[\rho_{101}\ln \rho_{111}- \rho_{101}\ln \rho_{101}]) \\   
      & + \gamma_{R}(\epsilon + U + U_{f})( (1-f_{R}(\epsilon+U+U_{f}) )[\rho_{111}\ln \rho_{101} - \rho_{111}\ln \rho_{111}]).   
\end{align*}

La versión cuántica de los flujos de información se encuentra detallada en el apéndice~\ref{apendix5infoflow}. En la Figura~\ref{img:P4clqm} se muestra una comparación gráfica entre los casos clásico y cuántico para estas cantidades.

\insertimage[\label{img:P4clqm}]{ejemplos/currentinfosemi.pdf}{scale=0.54}{En línea azul el cálculo realizado con el modelo cuántico que permite coherencias no nulas, mientras que la línea roja corresponde al cálculo semiclásico.}

Como se muestra en la Figura \ref{img:P4clqm}, en el régimen $eV/T\lessapprox 2.4$ la corriente de partículas calculada en el caso cuántico es menor a la del caso clásico. Por otro lado, para $eV/T \gtrapprox 2.4$ la corriente de partículas en el caso cuántico es mayor que en el caso clásico. Es decir, el comportamiento cuántico mejora el transporte de partículas, ya sea en contra o a favor del gradiente de potencial. 

Como se observa en la Figura \ref{img:P4clqm}, el flujo de información asociado al sitio $D$ es mayor en el caso cuántico, mostrando una diferencia apreciable en las regiones $eV/T \lessapprox 1$ y $eV/T \gtrapprox 6$. Este resultado indica que la dinámica cuántica permite una mejora en el intercambio de información entre el sistema $D$ y el subsistema $LR$. Dicha mejora se explica porque el sitio $D$ no solo establece correlaciones clásicas a través de los elementos diagonales del estado, sino que, debido a la presencia de coherencia, también genera correlaciones cuánticas con el sistema $LR$, incluso cuando la interacción entre ellos es exclusivamente de tipo clásico, mediada por la energía de Coulomb. Este comportamiento se corrobora con los flujos de información cuánticos obtenidos analíticamente, los cuales presentan contribuciones provenientes de los elementos no diagonales de la matriz densidad (ver Apéndice~\ref{apendix5infoflow}).
 

Finalmente, debido a este aumento en el flujo de información en el régimen cuántico, la tasa de cambio de la energía libre del subsistema $LR$, dada por $\dot{\mathcal{F}}_{LR} \approx T\dot{I}_{LR}$, disminuye. Como consecuencia, se incrementa la capacidad del sistema para realizar trabajo, lo cual evidencia el rol beneficioso de la coherencia cuántica como recurso termodinámico.


\label{sec5:demonio}


% ------------------------------------------------------------------------------
% NUEVO CAPÍTULO
% ------------------------------------------------------------------------------
\chapteranum{Conclusiones}



% ------------------------------------------------------------------------------
% REFERENCIAS, revisar configuración \stylecitereferences
% ------------------------------------------------------------------------------
\bibliography{library}


% ------------------------------------------------------------------------------
% ANEXO
% Existe adicionalmente el entorno \begin{appendixd} que permite insertar
% \chapter y el entorno \begin{appendixdtitle}[style1] (4 estilos diferentes),
% el cual acepta \chapter y escribe el título de anexos encima
% ------------------------------------------------------------------------------
\begin{appendixs}
	
	\section{Cálculos realizados sección 3}

	\subsection{Matriz densidad en función del campo de conteo}
    \label{apendix:fcs1}
    Reemplazando la distribución de probabilidad en la función generadora \ref{sec2funciongeneradora}

    \begin{align*}
        \Lambda(\vec{\lambda},\vec{\chi}) & = \sum_{\textbf{E},\textbf{E}',\textbf{N},\textbf{N}'} \int d\textbf{Q}d\textbf{W} P_{t}(\textbf{E}',\textbf{N}'|\textbf{E},\textbf{N}) P_{0}(\textbf{E},\textbf{N})  \\
        & \times \Pi_{\alpha} \delta(W_{\alpha} - \mu_{\alpha}(N_{\alpha} - N'_{\alpha}) ) \delta(Q_{\alpha} + W_{\alpha} -(E_{\alpha} - E'_{\alpha})) e^{-i\vec{\lambda}\cdot \textbf{Q}} e^{-i\vec{\chi}\cdot \textbf{W}} \\
        & = \sum_{\textbf{E},\textbf{E}',\textbf{N},\textbf{N}'}P_{t}(\textbf{E}',\textbf{N}'|\textbf{E},\textbf{N})P_{0}(\textbf{E},\textbf{N}) \Pi_{\alpha}e^{-i\chi_{\alpha}\mu_{\alpha}(N_{\alpha} - N'_{\alpha})}e^{-i\lambda_{\alpha}((E_{\alpha} -\mu_{\alpha}N_{\alpha}) - (E'_{\alpha} - \mu_{\alpha}N'_{\alpha}) ) }  \\
        & = \sum_{\textbf{E},\textbf{E}',\textbf{N},\textbf{N}'} \text{Tr}\{\hat{P}_{\textbf{E}',\textbf{N}'}\hat{U}(t)\hat{P}_{\textbf{E},\textbf{N}}(\hat{\rho}_{s}(0) \otimes \Pi_{\alpha}\hat{\tau}_{\alpha} )\hat{P}_{\textbf{E},\textbf{N}}\hat{U}^{\dagger}(t)\hat{P}_{\textbf{E}',\textbf{N}'}   \} \\
        & \times \Pi_{\alpha} e^{-i\chi_{\alpha}\mu_{\alpha}(N_{\alpha} - N'_{\alpha})}e^{-i\lambda_{\alpha}((E_{\alpha} -\mu_{\alpha}N_{\alpha}) - (E'_{\alpha} - \mu_{\alpha}N'_{\alpha}) ) }.
    \end{align*}

Si se considera un observable $\hat{A}(0)$ y proyectores del tipo $\hat{P}_{a_{0}}$ y se aplican para un estado diagonal $\hat{\rho}_{diag}$\cite{esposito2009nonequilibrium}, se cumple que

\begin{equation}
    \sum_{a_{0}}e^{-i\lambda a_{0}}\hat{P}_{a_{0}}\hat{\rho}_{diag}\hat{P}_{a_{0}} = e^{-i(\lambda/2)\hat{A}(0)}\hat{\rho}_{diag}e^{-i(\lambda/2)\hat{A}(0)}.
\label{apendixobservable}
\end{equation}

Usando la relación \ref{apendixobservable} en la función generadora

\begin{align*}
    \Lambda(\vec{\lambda},\vec{\chi}) & = \sum_{\textbf{E}',\textbf{N}'}\text{Tr}\{ \hat{P}_{\textbf{E}',\textbf{N}'} \hat{U}(t) e^{-\frac{i}{2}\sum_{\alpha}[\lambda_{\alpha}(\hat{H}_{\alpha} - \mu_{\alpha}\hat{N}_{\alpha}) + \chi_{\alpha}\mu_{\alpha}\hat{N}_{\alpha}  ]}  \}\hat{\rho}_{tot}(0) e^{-\frac{i}{2}\sum_{\alpha}[\lambda_{\alpha}(\hat{H}_{\alpha} - \mu_{\alpha}\hat{N}_{\alpha}) + \chi_{\alpha}\mu_{\alpha}\hat{N}_{\alpha} ] }\hat{U}^{\dagger}(t) \hat{P}_{\textbf{E}',\textbf{N}'} \} \\
    & \times \Pi_{\alpha}e^{i\chi_{\alpha}\mu_{\alpha}\hat{N}_{\alpha}}e^{i\lambda_{\alpha}(\hat{H}_{\alpha} -\mu_{\alpha}\hat{N}_{\alpha})} \\
    & = \text{Tr}\{ \hat{U}(\vec{\lambda},\vec{\chi};t)\hat{\rho}_{tot}(0)\hat{U}^{\dagger}(-\vec{\lambda},-\vec{\chi};t)  \}.
\end{align*}

\newpage 

    \subsection{Matriz densidad generalizada}
    Se puede realizar una expansión a segundo orden de la ecuación \ref{sec2FCS:evolution} y tener la evolución

    \begin{equation*}
        |\hat{\rho}_{totI}(\vec{\lambda},\vec{\chi},t)\rangle \rangle  = \left( \textbf{1} + \epsilon \int_{0}^{t}\check{\mathcal{L}}'_{\lambda}(t_{1})dt_{1} + \epsilon^{2}\int_{0}^{t}dt_{1}\int_{0}^{t_{1}}\check{\mathcal{L}}'_{\lambda}(t_{1})\check{\mathcal{L}}'_{\lambda}(t_{2})dt_{2} \right) |\hat{\rho}_{tot}(0)\rangle \rangle, 
    \end{equation*}
al hacer el cambio de variable $t_{1}=T$ y $t_{2}=T-s$ 

\begin{align*}
    |\hat{\rho}_{totI}(\vec{\lambda},\vec{\chi},t)\rangle \rangle  & = \left( \textbf{1} + \epsilon \int_{0}^{t}\check{\mathcal{L}}'_{\lambda}(T)dT + \epsilon^{2}\int_{0}^{t}dT\int_{0}^{T}ds \check{\mathcal{L}}'_{\lambda}(T)\check{\mathcal{L}}'_{\lambda}(T-s) \right)|\hat{\rho}_{tot}(0)\rangle \rangle  \\
    & = \check{\mathcal{W}}(\vec{\lambda},\vec{\chi},t)|\hat{\rho}_{tot}(0)\rangle \rangle \\
    & =  [\check{\mathcal{W}}_{0}(\vec{\lambda},\vec{\chi},t) + \epsilon \check{\mathcal{W}}_{1}(\vec{\lambda},\vec{\chi},t) + \epsilon^{2}\check{\mathcal{W}}_{2}(\vec{\lambda},\vec{\chi},t)] |\hat{\rho}_{tot}(0)\rangle \rangle,
\end{align*}
con los superoperadores definidos por 

\begin{align*}
    \check{\mathcal{W}}_{0}(\vec{\lambda},\vec{\chi},t) &  = \textbf{1}\\
    \check{\mathcal{W}}_{1}(\vec{\lambda},\vec{\chi},t) & = \int_{0}^{t}dT \check{\mathcal{L}'}_{\lambda}(T) \\
    \check{\mathcal{W}}_{2}(\vec{\lambda},\vec{\chi},t) & = \int_{0}^{t}dT \int_{0}^{T}ds \check{\mathcal{L}'}_{\lambda}(T)\check{\mathcal{L}'}_{\lambda}(T-s).
\end{align*}

Tomando hasta segundo orden la inversa del superoperador $\check{\mathcal{W}}$

\begin{equation}
    \check{\mathcal{W}}^{-1}(\vec{\lambda},\vec{\chi},t) =  \check{\mathcal{W}}_{0}(\vec{\lambda},\vec{\chi},t) - \epsilon  \check{\mathcal{W}}_{1}(\vec{\lambda},\vec{\chi},t) +  \epsilon^{2}[\check{\mathcal{W}}^{2}_{1}(\vec{\lambda},\vec{\chi},t) -  \check{\mathcal{W}}_{2}(\vec{\lambda},\vec{\chi},t) ],
\label{apendix2inverseW}
\end{equation}
se cumple la propiedad

\begin{equation}
    \dot{\check{\mathcal{W}}}(\vec{\lambda},\vec{\chi},t)\check{A}\check{\mathcal{W}}^{-1}(\vec{\lambda},\vec{\chi},t) = \epsilon \dot{\check{\mathcal{W}}}_{1}(\vec{\lambda},\vec{\chi},t)\check{A} + \epsilon^{2}[\dot{\check{\mathcal{W}}}_{2}(\vec{\lambda},\vec{\chi},t)\check{A} - \dot{\check{\mathcal{W}}}_{1}(\vec{\lambda},\vec{\chi},t)\check{A}\check{\mathcal{W}}_{1}(\vec{\lambda},\vec{\chi},t) ].
    \label{apendix2Wproperty}
\end{equation}

Para encontrar la matriz densidad generalizada se quiere rescatar la parte relevante de la matriz densidad total, es decir, trazar los grados de libertad del reservorio, por ende el proyector que se usará es 

\begin{equation*}
    \check{\mathcal{P}} = \sum_{r}|\rho_{R}^{eq} \rangle \rangle \langle \langle rr|,
\end{equation*}
en donde $\otimes_{\alpha} \hat{\tau}_{\alpha} \to  |\hat{\rho}_{R}^{eq} \rangle \rangle$ es el vector en el espacio de Liouville que representa el equilibrio del reservorio. Al aplicar esto a la matriz densidad total

\begin{equation*}
    \check{\mathcal{P}}|\hat{\rho}(\vec{\lambda},\vec{\chi},t) \rangle \rangle = |\hat{\rho}_{S}(\vec{\lambda},\vec{\chi},t)\rangle \rangle \otimes |\hat{\rho}^{eq}_{R}\rangle \rangle ,
\end{equation*}
la evolución de la matriz densidad generalizada usando los operadores Nakajima-Zwanzig se vuelve 

\begin{align}
    \check{\mathcal{P}}|\hat{\rho}_{totI}(\vec{\lambda},\vec{\chi},t)\rangle \rangle & =  \check{\mathcal{P}} \check{\mathcal{W}}(t)( \check{\mathcal{P}} +  \check{\mathcal{Q}})|\hat{\rho}_{tot}(0)\rangle \rangle  \label{apendix2proyectionev1} \\
    \check{\mathcal{Q}}|\hat{\rho}_{totI}(\vec{\lambda},\vec{\chi},t)\rangle \rangle & = \check{\mathcal{Q}} \check{\mathcal{W}}(t)( \check{\mathcal{P}} +  \check{\mathcal{Q}})|\hat{\rho}_{tot}(0)\rangle \rangle. 
\label{apendix2proyectionev}
\end{align}

Dos indicaciones son importantes para seguir. Primero, debido a que se asume que la condición inicial del reservorio es diagonal se tendrá que $\check{\mathcal{Q}}|\hat{\rho}_{tot}(0)\rangle\rangle = 0 $. Segundo 
 que $|\hat{\rho}_{tot}(0)\rangle \rangle = \check{\mathcal{W}}^{-1}(\vec{\lambda},\vec{\chi},t)|\hat{\rho}_{tot}(\vec{\lambda},\vec{\chi},t)\rangle \rangle$ y

\begin{align*}
    |\hat{\rho}_{totI}(0)\rangle \rangle & = (\check{\mathcal{P}} + \check{\mathcal{Q}} )\check{\mathcal{W}}^{-1}(\vec{\lambda},\vec{\chi},t)(\check{\mathcal{P}} + \check{\mathcal{Q}})|\hat{\rho}_{totI}(\vec{\lambda},\vec{\chi},t)\rangle \rangle \\
        & = \check{\mathcal{P}}\check{\mathcal{W}}^{-1}(\vec{\lambda},\vec{\chi},t)(\check{\mathcal{P}} + \check{\mathcal{Q}})|\hat{\rho}_{totI}(\vec{\lambda},\vec{\chi},t)\rangle \rangle.
\end{align*}

Al derivar temporalmente las ecuaciones \ref{apendix2proyectionev1} y \ref{apendix2proyectionev} 

\begin{align*}
    \check{\mathcal{P}}|\dot{\hat{\rho}}_{totI}(\vec{\lambda},\vec{\chi},t) \rangle \rangle & = \check{\mathcal{P}}\dot{\check{\mathcal{W}}}(\vec{\lambda},\vec{\chi},t)\check{\mathcal{P}}\check{\mathcal{W}}^{-1}(\vec{\lambda},\vec{\chi},t)\check{\mathcal{P}}|\hat{\rho}_{totI}(\vec{\lambda},\vec{\chi},t)\rangle \rangle \\
     & = \check{\mathcal{P}}\dot{\check{\mathcal{W}}}(\vec{\lambda},\vec{\chi},t)\check{\mathcal{P}}\check{\mathcal{W}}^{-1}(\vec{\lambda},\vec{\chi},t)\check{\mathcal{Q}}|\hat{\rho}_{totI}(\vec{\lambda},\vec{\chi},t)\rangle \rangle 
\end{align*}

\begin{align*}
    \check{\mathcal{Q}}|\dot{\hat{\rho}}_{totI}(\vec{\lambda},\vec{\chi},t) \rangle \rangle & = \check{\mathcal{Q}}\dot{\check{\mathcal{W}}}(\vec{\lambda},\vec{\chi},t)\check{\mathcal{P}}\check{\mathcal{W}}^{-1}(\vec{\lambda},\vec{\chi},t)\check{\mathcal{P}}|\hat{\rho}_{totI}(\vec{\lambda},\vec{\chi},t)\rangle \rangle \\
     & = \check{\mathcal{Q}}\dot{\check{\mathcal{W}}}(\vec{\lambda},\vec{\chi},t)\check{\mathcal{P}}\check{\mathcal{W}}^{-1}(\vec{\lambda},\vec{\chi},t)\check{\mathcal{Q}}|\hat{\rho}_{totI}(\vec{\lambda},\vec{\chi},t)\rangle \rangle, 
\end{align*}
estas ecuaciones son exactas. Para utilizar la suposición de acoplamiento débil, se partirá usando la relación \ref{apendix2Wproperty}, así

\begin{align*}
    \check{\mathcal{P}}\dot{\check{\mathcal{W}}}(\vec{\lambda},\vec{\chi},t)\check{\mathcal{P}}\check{\mathcal{W}}^{-1}(\vec{\lambda},\vec{\chi},t)\check{\mathcal{Q}} & = \epsilon \check{\mathcal{P}}\dot{\check{\mathcal{W}}}_{1}(\vec{\lambda},\vec{\chi},t) \check{\mathcal{P}}\check{\mathcal{Q}} \\
     & + \epsilon^{2} \check{\mathcal{P}}\dot{\check{\mathcal{W}}}_{2}(\vec{\lambda},\vec{\chi},t) \check{\mathcal{P}}\check{\mathcal{Q}} \\
     & - \epsilon^{2}\check{\mathcal{P}}\dot{\check{\mathcal{W}}}_{1}(\vec{\lambda},\vec{\chi},t)\check{\mathcal{P}} \check{\mathcal{W}}_{1}(\vec{\lambda},\vec{\chi},t)\check{\mathcal{Q}},
\end{align*}
el primer y segundo término se hacen cero debido a que $\check{\mathcal{P}}\check{\mathcal{Q}}=0$, por otro lado

\begin{align*}
    \check{\mathcal{P}}\dot{\check{\mathcal{W}}}_{1}(\vec{\lambda},\vec{\chi},t)\check{\mathcal{P}} = \sum_{r,r'}|\hat{\rho}^{eq}_{R} \rangle \rangle \langle \langle rr| \check{\mathcal{L}}'_{\lambda}(t)|\hat{\rho}_{R}^{eq}\rangle \rangle \langle \langle r'r'|,
\end{align*}
en donde el término $\langle \langle rr|\check{\mathcal{L}}'_{\lambda}(t)|\hat{\rho}_{R}^{eq}\rangle \rangle$ equivale a 

\begin{equation}
    \langle \langle rr|\check{\mathcal{L}}'_{\lambda}(t)|\hat{\rho}_{R}^{eq}\rangle \rangle  = \text{Tr}_{B}\{\hat{\rho}^{eq}_{R}\hat{V}_{\lambda}(t)\} - \text{Tr}_{B}\{\hat{V}_{-\lambda}(t)\hat{\rho}^{eq}_{R}\}.
    \label{sec3:ecrandom}
\end{equation}

Debido a que $\hat{\rho}^{eq}_{R}$ conmuta con $\hat{H}_{R}$ tendremos que el estado de equilibrio conmuta con $\hat{A}(\lambda,\chi)$, por lo tanto se anula \ref{sec3:ecrandom}. Y el término restante queda como 

\begin{align*}
    \check{\mathcal{P}}\dot{\check{\mathcal{W}}}(\vec{\lambda},\vec{\chi},t)\check{\mathcal{P}}\check{\mathcal{W}}^{-1}(\vec{\lambda},\vec{\chi},t)\check{\mathcal{P}} & = \epsilon \check{\mathcal{P}}\dot{\check{\mathcal{W}}}_{1}(\vec{\lambda},\vec{\chi},t)\check{\mathcal{P}} \\
    & + \epsilon^{2}\check{\mathcal{P}}\dot{\check{\mathcal{W}}}_{2}(\vec{\lambda},\vec{\chi},t)\check{\mathcal{P}} \\
    & - \epsilon^{2}\check{\mathcal{P}}\dot{\check{\mathcal{W}}}_{1}(\vec{\lambda},\vec{\chi},t)\check{\mathcal{P}}  \check{\mathcal{W}}_{1}(\vec{\lambda},\vec{\chi},t) \check{\mathcal{P}}.
\end{align*}

De manera explícita esto es

\begin{equation*}
    \check{\mathcal{P}}\dot{\check{\mathcal{W}}}(\vec{\lambda},\vec{\chi},t)\check{\mathcal{P}}\check{\mathcal{W}}^{-1}(\vec{\lambda},\vec{\chi},t)\check{\mathcal{P}} = \epsilon^{2}\check{\mathcal{P}}\int_{0}^{t}ds \check{\mathcal{L}}'_{\lambda}(t)\check{\mathcal{L}}'_{\lambda}(t-s)\check{\mathcal{P}}, 
\end{equation*}
aplicando $\check{\mathcal{P}}|\dot{\hat{\rho}}_{totI}(\vec{\lambda},\vec{\chi},t)\rangle \rangle = |\dot{\hat{\rho}}_{IS}(\vec{\lambda},\vec{\chi},t)\rangle \rangle \otimes |\hat{\rho}^{eq}_{R}\rangle \rangle$ y multiplicando a la izquierda por $\sum_{r}\langle \langle rr|$ para eliminar los grados de libertad del reservorio

\begin{equation*}
    \dot{\hat{\rho}}_{IS}(\vec{\lambda},\vec{\chi},t) = \epsilon^{2} \sum_{r}\langle \langle rr|\int_{0}^{t}ds \check{\mathcal{L}}'_{\lambda}(t)\check{\mathcal{L}}'_{\lambda}(t-s)|\hat{\rho}_{R}^{eq}\rangle \rangle \hat{\rho}_{IS}(\vec{\lambda},\vec{\chi},t),
\end{equation*}
de aquí se podrán obtener las funciones correlación calculando el producto

\begin{align*}
    \sum_{r}\langle \langle rr| \check{\mathcal{L}}'_{\lambda}(t)\check{\mathcal{L}}'_{\lambda}(t-s)|\hat{\rho}_{R}^{eq}\rangle \rangle \hat{\rho}_{IS}(\vec{\lambda},\vec{\chi},t) = \text{Tr}_{B}\{\mathcal{L}'_{\lambda}(t)\mathcal{L}'_{\lambda}(t-s)\hat{\rho}_{R}^{eq}\hat{\rho}_{IS}(\vec{\lambda},\vec{\chi},t) \},
\end{align*}
desarrollando paso por paso

\begin{equation*}
    \mathcal{L}'_{\lambda}(t-s)\hat{\rho}_{IS}(\vec{\lambda},\vec{\chi},t)\hat{\rho}_{R}^{eq} = -i[\hat{V}_{\lambda}(t-s)\hat{\rho}_{IS}(\vec{\lambda},\vec{\chi},t)\hat{\rho}^{eq}_{R} -  \hat{\rho}_{IS}(\vec{\lambda},\vec{\chi},t)\hat{\rho}^{eq}_{R}\hat{V}_{-\lambda}(t-s)],
\end{equation*}
y así de manera explícita
\begin{align*}
    \mathcal{L}'_{\lambda}(t)\mathcal{L}'_{\lambda}(t-s)\hat{\rho}_{IS}(\vec{\lambda},\vec{\chi},t)\hat{\rho}_{R}^{eq}  = &  -\hat{V}_{\lambda}(t)\hat{V}_{\lambda}(t-s)\hat{\rho}_{IS}(\vec{\lambda},\vec{\chi},t)\hat{\rho}_{R}^{eq} + \hat{V}_{\lambda}(t)\hat{\rho}_{IS}(\vec{\lambda},\vec{\chi},t)\hat{\rho}_{R}^{eq}\hat{V}_{-\lambda}(t-s) \\
    & + \hat{V}_{\lambda}(t-s)\hat{\rho}_{IS}(\vec{\lambda},\vec{\chi},t)\hat{\rho}_{R}^{eq}\hat{V}_{-\lambda}(t) - \hat{\rho}_{IS}(\vec{\lambda},\vec{\chi},t)\hat{\rho}_{R}^{eq}\hat{V}_{-\lambda}(t-s)\hat{V}_{-\lambda}(t).
\end{align*}

Finalmente, aplicando la traza en el reservorio a esta ecuación se obtiene \ref{ecmaestraVlambda}.

    \label{apendixsubsectionmatriz}
    
\newpage
%%%%%%%%%%%%%%%%%%%%%%%%%%%%%%%%%%%%%%%%%%%%%%%%%%%%%%%%
%%%%%%%seccion%%%%%%%%%%%%%%%%%%%%%%%%%%%%%%%%%%%%%%%%%%
%%%%%%%%%%%%%%%%%%%%%%%%%%%%%%%%%%%%%%%%%%%%%%%%%%%%%%%%

\subsection{Funciones correlación}
Para escribir la ecuación maestra generalizada a través de las funciones correlación, se partirá de la expresión
\begin{align*}
& \text{Tr}_{B}\{ \hat{V}_{\lambda}(t)\hat{V}_{\lambda}(t-s) \hat{\rho}_{IS}(\vec{\lambda},\vec{\chi},t)\hat{\rho}^{eq}_{R} \}  =\\
&  - \sum_{\alpha,k,k';j,j'}e^{i(\omega_{j}-\omega_{j'})t}e^{i\omega_{j'}s}\hat{S}^{\dagger}_{\alpha k;j}\hat{S}_{\alpha,k';j'}\hat{\rho}_{IS}(\vec{\lambda},\vec{\chi},t)\text{Tr}_{B}\{e^{-(i/2)\hat{A}(\lambda,\chi)}\hat{V}_{\lambda}(t)\hat{V}_{\lambda}(t-s)e^{(i/2)\hat{A}(\lambda,\chi)}\hat{\rho}^{eq}_{R}\},
\end{align*}    
suponiendo que las funciones de correlación son homogéneas, esto se vuelve

\begin{align*}
    \text{Tr}_{B}\{ \hat{V}_{\lambda}(t)\hat{V}_{\lambda}(t-s) \hat{\rho}_{IS}(\vec{\lambda},\vec{\chi},t)\hat{\rho}^{eq}_{R} \} & = \\
    & - \sum_{\alpha,k,k';j,j'}e^{i(\omega_{j}-\omega_{j'})t}e^{i\omega_{j'}s}\hat{S}^{\dagger}_{\alpha k;j}\hat{S}_{\alpha,k';j'}\hat{\rho}_{IS}(\vec{\lambda},\vec{\chi},t)\text{Tr}_{B}\{\hat{B}^{\dagger}_{\alpha,k}(s)\hat{B}_{\alpha,k}\hat{\tau}_{\alpha} \} \\
    \text{Tr}_{B}\{ \hat{\rho}_{IS}(\vec{\lambda},\vec{\chi},t)\hat{\rho}^{eq}_{R} \hat{V}_{-\lambda}(t-s)\hat{V}_{-\lambda}(t) \} & = \\
    & - \sum_{\alpha,k,k';j,j'}e^{i(\omega_{j}-\omega_{j'})t}e^{-i\omega_{j}s}\hat{\rho}_{IS}(\vec{\lambda},\vec{\chi},t)\hat{S}^{\dagger}_{\alpha k;j}\hat{S}_{\alpha,k';j'} \text{Tr}_{B}\{ \hat{B}^{\dagger}_{\alpha,k}(s)\hat{B}_{\alpha,k}\hat{\tau}_{\alpha} \}.   
\end{align*}    

Analizaremos el término $\text{Tr}_{B}\{ \hat{V}_{\lambda}(t)\hat{\rho}_{IS}(\vec{\lambda},\vec{\chi},t)\hat{\rho}_{R}^{eq}\hat{V}_{-\lambda}(t-s) \}$ para ello se utilizará la relación de conmutación $[\hat{B}_{\alpha,k},\hat{N}_{\alpha}] = n_{\alpha,k}\hat{B}_{\alpha,k}$ que conlleva que

\begin{equation*}
    e^{C\hat{N}_{\alpha}}\hat{B}_{\alpha,k}e^{-C\hat{N}_{\alpha}} = e^{-Cn_{\alpha,k}}\hat{B}_{\alpha,k},
\end{equation*}
con $C$ una constante. Se podrá utilizar esto para obtener

\begin{align*}
   & \text{Tr}_{B}\{ \hat{V}_{\lambda}(t)\hat{\rho}_{IS}(\vec{\lambda},\vec{\chi},t)\hat{\rho}_{R}^{eq}\hat{V}_{-\lambda}(t-s) \} \\
   & = \sum_{\alpha,k,k';j,j'}e^{i(\omega_{j}-\omega_{j'})t}e^{-i\omega_{j}s}\hat{S}_{\alpha,k';j'}\hat{\rho}_{IS}(\vec{\lambda},\vec{\chi},t)\hat{S}^{\dagger}_{\alpha,k;j} \text{Tr}_{B}\{e^{-i(\hat{A}(\lambda,\chi))} \hat{B}_{\alpha,k}(t)e^{i\hat{A}(\lambda,\chi)}\hat{\rho}_{R}^{eq}\hat{B}^{\dagger}_{\alpha,k}(t-s) \}.
\end{align*}

Por último, se puede notar que 

\begin{align*}
    e^{-i(\hat{A}(\lambda,\chi))} \hat{B}_{\alpha,k}(t)e^{i\hat{A}(\lambda,\chi)} & = e^{-i\mu_{\alpha}n_{\alpha,k}(\lambda_{\alpha}-\chi_{\alpha})} \hat{B}_{\alpha,k}(t+\lambda_{\alpha}), 
\end{align*}
con esto se tiene

\begin{align*}
 & \text{Tr}_{B}\{ \hat{V}_{\lambda}(t)\hat{\rho}_{IS}(\vec{\lambda},\vec{\chi},t)\hat{\rho}_{R}^{eq}\hat{V}_{-\lambda}(t-s) \} = \\
 & \sum_{\alpha,k,k';j,j'}e^{i(\omega_{j}-\omega_{j'})t}e^{-i\omega_{j}s}\hat{S}_{\alpha,k';j'}\hat{\rho}_{IS}(\vec{\lambda},\vec{\chi},t)\hat{S}^{\dagger}_{\alpha,k;j} C^{\alpha}_{k,k'}(-s-\lambda_{\alpha})e^{-i\mu_{\alpha}n_{\alpha,k}(\lambda_{\alpha}-\chi_{\alpha})}
 \end{align*}

 \begin{align*}
    & \text{Tr}_{B}\{ \hat{V}_{\lambda}(t-s)\hat{\rho}_{IS}(\vec{\lambda},\vec{\chi},t)\hat{\rho}_{R}^{eq}\hat{V}_{-\lambda}(t) \} = \\
    & \sum_{\alpha,k,k';j,j'}e^{i(\omega_{j}-\omega_{j'})t}e^{i\omega_{j'}s}\hat{S}_{\alpha,k';j'}\hat{\rho}_{IS}(\vec{\lambda},\vec{\chi},t)\hat{S}^{\dagger}_{\alpha,k;j} C^{\alpha}_{k,k'}(s-\lambda_{\alpha})e^{-i\mu_{\alpha}n_{\alpha,k}(\lambda_{\alpha}-\chi_{\alpha})}.
    \end{align*}
   
Reemplazando estos términos en \ref{ecmaestraVlambda} se obtiene la ecuación maestra generalizada.   

\label{finalequation}

\newpage
%%%%%%%%%%%%%%%%%%%%%%%%%%%%%%%%%%%%%%%%%%%%%%%%%%%%%%
%%%%%%%%%%%%%%%%%seccion%%%%%%%%%%%%%%%%%%%%%%%%%%%%%%
%%%%%%%%%%%%%%%%%%%%%%%%%%%%%%%%%%%%%%%%%%%%%%%%%%%%%%

\subsection{Ecuación de Lindblad generalizada}
Desarrollando la ecuación \ref{ecmaestrafinal} podremos escribir

\begin{multline*}
    \frac{d}{dt}\hat{\rho}_{IS}(\vec{\lambda},\vec{\chi},t) = - \sum_{\alpha,k,k';q} \int_{0}^{\infty}ds \left[e^{i\omega_{q}s}C^{\alpha}_{k,k'}(s) \left(\sum_{j}e^{i\omega_{j}t}\hat{S}^{\dagger}_{\alpha,k;j} \right)\left(\sum_{j'}e^{-i\omega_{j'}t}\hat{S}_{\alpha,k';j'} \right)\hat{\rho}_{IS}(\vec{\lambda},\vec{\chi},t)  \right. \\
    \left. +  e^{-i\omega_{q}s}C^{\alpha}_{k,k'}(-s) \hat{\rho}_{IS}(\vec{\lambda},\vec{\chi},t) \left(\sum_{j}e^{i\omega_{j}t}\hat{S}^{\dagger}_{\alpha,k;j} \right)\left(\sum_{j'}e^{-i\omega_{j'}t}\hat{S}_{\alpha,k';j'} \right)\right. \\
    \left. - e^{-i\mu_{\alpha}n_{\alpha,k}(\lambda_{\alpha}-\chi_{\alpha})}(e^{i\omega_{q}s}C^{\alpha}_{k,k'}(s-\lambda_{\alpha}) + e^{-i\omega_{q}s}C^{\alpha}_{k,k'}(-s-\lambda_{\alpha}) ) \right.\\
    \left.\times \left(\sum_{j'}e^{-i\omega_{j'}t}\hat{S}_{\alpha,k';j'} \right)\hat{\rho}_{IS}(\vec{\lambda},\vec{\chi},t) \left(\sum_{j}e^{i\omega_{j}t}\hat{S}^{\dagger}_{\alpha,k;j} \right)    \right],   
\end{multline*}
tomando la condición que $C_{k,k'}^{\alpha} \propto \delta_{k,k'}$

\begin{multline*}
    \frac{d}{dt}\hat{\rho}_{IS}(\vec{\lambda},\vec{\chi},t) = \\
     - \sum_{\alpha,k;q} \int_{0}^{\infty}ds \left[e^{i\omega_{q}s}C^{\alpha}_{k,k}(s) \hat{S}^{\dagger}_{\alpha,k;q}(t)\hat{S}_{\alpha,k;q}(t)\hat{\rho}_{IS}(\vec{\lambda},\vec{\chi},t)  +  e^{-i\omega_{q}s}C^{\alpha}_{k,k}(-s) \hat{\rho}_{IS}(\vec{\lambda},\vec{\chi},t) \hat{S}^{\dagger}_{\alpha,k;q}(t) \hat{S}_{\alpha,k;q}(t) \right. \\
    \left. - e^{-i\mu_{\alpha}n_{\alpha,k}(\lambda_{\alpha}-\chi_{\alpha})}(e^{i\omega_{q}s}C^{\alpha}_{k,k}(s-\lambda_{\alpha}) + e^{-i\omega_{q}s}C^{\alpha}_{k,k}(-s-\lambda_{\alpha}) ) \hat{S}_{\alpha,k;q}(t) \hat{\rho}_{IS}(\vec{\lambda},\vec{\chi},t) \hat{S}^{\dagger}_{\alpha,k;q}(t)    \right].  
\end{multline*}

Usando la identidad 

\begin{align*}
    \int_{0}^{\infty}ds e^{i\omega_{q}s}C^{\alpha}_{k,k}(s) & = \int_{-\infty}^{\infty}ds e^{i\omega_{q}s}(1+\text{sgn}(s))C^{\alpha}_{k,k}(s)/2 \\
    & = \int_{-\infty}^{\infty}ds e^{i\omega_{q}s}C^{\alpha}_{k,k}(s)/2 + i \left(-\frac{i}{2} \right) \int_{-\infty}^{\infty}ds e^{i\omega_{q}s} \text{sgn}(s)C^{\alpha}_{k,k}(s) \\
    & = \frac{1}{2}\Gamma_{k,k}^{\alpha}(\omega_{q}) + i \Delta^{\alpha}_{k}(\omega_{q}),
\end{align*}
y usando que $\int_{0}^{\infty}dse^{i\omega_{q}s}C^{\alpha}_{k,k}(s-\lambda_{\alpha}) = e^{i\lambda_{\alpha}\omega_{q}}\int_{0}^{\infty}e^{i(s-\lambda_{\alpha})\omega_{q}}C^{\alpha}_{k,k}(s-\lambda_{\alpha}) $, nos queda finalmente

\begin{multline*}
    \frac{d}{dt}\hat{\rho}_{IS}(\vec{\lambda},\vec{\chi},t) = - i \sum_{\alpha,k;q}\Delta^{\alpha}_{k,k}(\omega_{q})\left[\hat{S}^{\dagger}_{\alpha,k;q}(t)\hat{S}_{\alpha,k;q}(t)\hat{\rho}_{IS}(\vec{\lambda},\vec{\chi},t) - \hat{\rho}_{IS}(\vec{\lambda},\vec{\chi},t)\hat{S}^{\dagger}_{\alpha,k;q}(t)\hat{S}_{\alpha,k;q}(t) \right] \\
    + \sum_{\alpha,k;q} \Gamma_{k,k}^{\alpha}(\omega_{q})\left[ e^{i\lambda_{\alpha}\omega_{q}+ i(\chi_{\alpha} - \lambda_{\alpha})\mu_{\alpha}n_{\alpha,k}}\hat{S}_{\alpha,k;q}(t)\hat{\rho}_{IS}(\vec{\lambda},\vec{\chi},t)\hat{S}^{\dagger}_{\alpha,k;q}(t) - \frac{1}{2}\{\hat{S}^{\dagger}_{\alpha,k;q}(t)\hat{S}_{\alpha,k;q}(t),\hat{\rho}_{IS}(\vec{\lambda},\vec{\chi},t) \} \right].
\end{multline*}

Que es la ecuación que se quiere demostrar.

\label{apendixGKLSgeneral}

\newpage
%%%%%%%%%%%%%%%%%%%%%%%%%%%%%%%%%%%%%%%%%%%%%%%%%%%%%%%
%%%%%%%%%%%%%%%%seccion%%%%%%%%%%%%%%%%%%%%%%%%%%%%%%%%
%%%%%%%%%%%%%%%%%%%%%%%%%%%%%%%%%%%%%%%%%%%%%%%%%%%%%%%


\subsection{Condición KMS y funciones correlación espectral}
Definiremos una función correlación auxiliar de la forma

\begin{align*}
    C^{\alpha N}_{kk}(s) & =  \langle \hat{B}^{\dagger}_{\alpha,k}(s)\hat{B}_{\alpha,k} \rangle_{N} \\
   &  = \text{Tr}\{e^{is(\hat{H}_{\alpha} - \mu_{\alpha}\hat{N}_{\alpha})}\hat{B}^{\dagger}_{\alpha,k}e^{-is(\hat{H}_{\alpha} - \mu_{\alpha}\hat{N}_{\alpha})}\hat{B}_{\alpha,k}\hat{\tau}_{\alpha}  \} \\
   & = \frac{1}{Z}\text{Tr}\{\hat{B}_{\alpha,k} e^{-\beta_{\alpha}(\hat{H}_{\alpha} - \mu_{\alpha}\hat{N}_{\alpha})} e^{is(\hat{H}_{\alpha} - \mu_{\alpha}\hat{N}_{\alpha})}\hat{B}^{\dagger}_{\alpha,k}e^{-is(\hat{H}_{\alpha} - \mu_{\alpha}\hat{N}_{\alpha})}\} \\
   & = \text{Tr}\{\hat{B}_{\alpha,k}e^{i(s+i\beta_{\alpha})(\hat{H}_{\alpha} - \mu_{\alpha}\hat{N}_{\alpha})}\hat{B}^{\dagger}_{\alpha,k}e^{-i(s+i\beta_{\alpha})(\hat{H}_{\alpha} - \mu_{\alpha}\hat{N}_{\alpha})}\hat{\tau}_{\alpha} \} \\
   & = \langle \hat{B}_{\alpha,k} \hat{B}^{\dagger}_{\alpha,k}(s+i\beta_{\alpha})\rangle_{N},
\end{align*}
además, se puede notar que

\begin{align*}
    C^{\alpha N}_{kk}(s) & = \text{Tr}\{e^{-is\mu_{\alpha}\hat{N}_{\alpha} }\hat{B}^{\dagger}_{\alpha,k}(s)e^{is\mu_{\alpha}\hat{N}_{\alpha}} \hat{B}_{\alpha,k}e^{-is\mu_{\alpha}\hat{N}_{\alpha} }\hat{\tau}_{\alpha}   \} \\
 & = \text{Tr}\{\hat{B}^{\dagger}_{\alpha,k}(s)e^{is\mu_{\alpha}\hat{N}_{\alpha}} \hat{B}_{\alpha,k}e^{-is\mu_{\alpha}\hat{N}_{\alpha}}\hat{\tau}_{\alpha}   \}\\
 & = e^{-i\mu_{\alpha}n_{\alpha,k}s} C^{\alpha}_{kk}(s).
\end{align*}

Ahora podremos estudiar como se comportan las funciones correlación espectral

\begin{align*}
    \Gamma^{\alpha}_{k,k}(\omega) & = \int_{-\infty}^{\infty}ds e^{i\mu_{\alpha}n_{\alpha,k}s} e^{i\omega s}C^{\alpha N}_{k,k}(s) \\
    & = \int_{-\infty}^{\infty}ds e^{i\mu_{\alpha}n_{\alpha,k}s} e^{i\omega s}\langle \hat{B}_{\alpha,k} \hat{B}^{\dagger}_{\alpha,k}(s+i\beta_{\alpha}) \rangle_{N} \\
    & = \int_{-\infty}^{\infty}ds e^{-i(i\beta \mu_{\alpha}n_{\alpha,k})} e^{i\omega s} \langle \hat{B}_{\alpha,k}\hat{B}^{\dagger}_{\alpha,k}(s+i\beta_{\alpha}) \rangle \\
    & = e^{\beta_{\alpha}\mu_{\alpha}n_{\alpha,k}}e^{\beta \omega} \int_{-\infty}^{\infty} ds e^{i\omega(s+i\beta_{\alpha})} \langle \hat{B}_{\alpha,k}\hat{B}^{\dagger}_{\alpha,k}(s+i\beta_{\alpha}) \rangle \\
    & = e^{\beta_{\alpha}(\omega - \mu_{\alpha}n_{\alpha,k})} \Gamma^{\alpha}_{k,k}(-\omega).
\end{align*}


Esta condición es importante, ya que permite que se cumpla la condición de balance detallado local. 

\label{apendixKMS}


\newpage
%%%%%%%%%%%%%%%%%%%%%%%%%%%%%%%%%%%%%%%%%%%%%%%%%%%%%%%%
%%%%%%%%%%%%%%%%%%%%%%%%%%%%Leyestermo%%%%%%%%%%%%%%%%%%
%%%%%%%%%%%%%%%%%%%%%%%%%%%%%%%%%%%%%%%%%%%%%%%%%%%%%%%%

\subsection{Redefinición leyes de la termodinámica}
Para obtener la ley cero calcularemos $\mathcal{D}[\hat{S}_{\alpha,k;q}]e^{-\beta_{\alpha}(\hat{H}_{TD} - \mu_{\alpha}\hat{N}_{S})}$ y $\mathcal{D}[\hat{S}^{\dagger}_{\alpha,k;q}]e^{-\beta_{\alpha}(\hat{H}_{TD} - \mu_{\alpha}\hat{N}_{S})}$, considerando la relación de conmutación

\begin{align*}
    [\hat{S}_{\alpha,k;q},(\hat{H}_{TD} - \mu_{\alpha}\hat{N}_{S})] & = (\omega_{q} - \mu_{\alpha}n_{\alpha,k})\hat{S}_{\alpha,k;q}  \\
    [\hat{S}^{\dagger}_{\alpha,k;q},(\hat{H}_{TD} - \mu_{\alpha}\hat{N}_{S})] & = -(\omega_{q} - \mu_{\alpha}n_{\alpha,k})\hat{S}^{\dagger}_{\alpha,k;q},
\end{align*}
por ende se tendrá

 \begin{align*}
    e^{\beta_{\alpha}(\hat{H}_{TD} - \mu_{\alpha}\hat{N}_{S})}\hat{S}_{\alpha,k;q} e^{-\beta_{\alpha}(\hat{H}_{TD} - \mu_{\alpha}\hat{N}_{S})} & = \hat{S}_{\alpha,k;q}e^{-\beta_{\alpha}(\omega_{q} - \mu_{\alpha}n_{\alpha,k})} \\
    e^{\beta_{\alpha}(\hat{H}_{TD} - \mu_{\alpha}\hat{N}_{S})}\hat{S}^{\dagger}_{\alpha,k;q} e^{-\beta_{\alpha}(\hat{H}_{TD} - \mu_{\alpha}\hat{N}_{S})} & = \hat{S}^{\dagger}_{\alpha,k;q}e^{\beta_{\alpha}(\omega_{q} - \mu_{\alpha}n_{\alpha,k})}.
 \end{align*}

Podremos aplicar esto en el disipador

\begin{align*}
    \mathcal{D}[\hat{S}_{\alpha,k;q}]e^{-\beta_{\alpha}(\hat{H}_{TD} - \mu_{\alpha}\hat{N}_{S})} & =  e^{-\beta_{\alpha}(\hat{H}_{TD} - \mu_{\alpha}\hat{N}_{s})} \hat{S}_{\alpha,k;q}\hat{S}^{\dagger}_{\alpha,k;q} e^{-\beta_{\alpha(\omega_{q} - \mu_{\alpha}n_{\alpha,k})}} - e^{-\beta_{\alpha}(\hat{H}_{TD} - \mu_{\alpha}\hat{N}_{S})} \hat{S}^{\dagger}_{\alpha,k;q}\hat{S}_{\alpha,k;q} \\
    \mathcal{D}[\hat{S}^{\dagger}_{\alpha,k;q}]e^{-\beta_{\alpha}(\hat{H}_{TD} - \mu_{\alpha}\hat{N}_{S})} & = e^{-\beta_{\alpha}(\hat{H}_{TD} - \mu_{\alpha}\hat{N}_{s})} \hat{S}^{\dagger}_{\alpha,k;q}\hat{S}_{\alpha,k;q} e^{\beta_{\alpha(\omega_{q} - \mu_{\alpha}n_{\alpha,k})}} - e^{-\beta_{\alpha}(\hat{H}_{TD} - \mu_{\alpha}\hat{N}_{S})} \hat{S}_{\alpha,k;q}\hat{S}^{\dagger}_{\alpha,k;q}, 
\end{align*}
finalmente al sumar los dos disipadores en el superoperador $\mathcal{L}_{\alpha}$ se demuestra que

\begin{equation*}
    \mathcal{L}_{\alpha}e^{-\beta_{\alpha}(\hat{H}_{TD} - \mu_{\alpha}\hat{N}_{S})} \propto  \mathcal{D}[\hat{S}_{\alpha,k;q}]e^{-\beta_{\alpha}(\hat{H}_{TD} - \mu_{\alpha}\hat{N}_{S})} + e^{-\beta_{\alpha}(\omega_{q} - \mu_{\alpha}n_{\alpha,k})}\mathcal{D}[\hat{S}^{\dagger}_{\alpha,k;q}]e^{-\beta_{\alpha}(\hat{H}_{TD} - \mu_{\alpha}\hat{N}_{S})} = 0.
\end{equation*}

Con lo que se prueba la ley cero.

\newpage

%%%%%%%%%%%%%%%%%%%%%%%%%%%%%%%%%%%%%%%%%%%%%%%%%%%%%%%
%%%%%%%%%%%%%%%%%%%seccion%%%%%%%%%%%%%%%%%%%%%%%%%%%
%%%%%%%%%%%%%%%%%%%%%%%%%%%%%%%%%%%%%%%%%%%%%%%%%%%%%%%

\subsection{Segunda Ley}
Para desarrollar el cálculo de la segunda Ley, se partirá de

\begin{align*}
    - \frac{d}{dt}\text{Tr}\{ \hat{\rho}_{S}(t)\ln \hat{\rho}_{S}(t) \} & =  -\text{Tr}\Big\{ \frac{d}{dt}\hat{\rho}_{S}(t)\ln \hat{\rho}_{S}(t) \Big\} - \frac{d}{dt}\text{Tr}\{\hat{\rho}_{S}(t) \}\\
  & = - i \text{Tr}\{[\hat{H}_{S}+\hat{H}_{LS},\hat{\rho}_{S}(t)]\ln \hat{\rho}_{S}(t)  \} - \sum_{\alpha} \text{Tr}\{(\mathcal{L}_{\alpha}\hat{\rho}_{S}(t)) \ln \hat{\rho}_{S}(t) \}  \\
  & = -\text{Tr}\{(\mathcal{L}_{\alpha}\hat{\rho}_{S}(t)) \ln \hat{\rho}_{S}(t) \},
\end{align*}
por otro lado

\begin{align*}
    J_{\alpha} & = \text{Tr}\{ (\hat{H}_{TD} - \mu_{\alpha}\hat{N}_{s})\mathcal{L}_{\alpha}(\hat{\rho}_{S}(t)) \} \\
    & = -\frac{1}{\beta_{\alpha}} \text{Tr}\{(\mathcal{L}_{\alpha}\hat{\rho}_{S}(t)) \ln \hat{\rho}_{G}(\beta_{\alpha},\mu_{\alpha})  \} + \frac{\text{Tr}\{e^{-\beta_{\alpha}(\hat{H}_{TD} - \mu_{\alpha}\hat{N}_{S})} \} }{\beta_{\alpha}} \text{Tr}\{(\mathcal{L}_{\alpha}\hat{\rho}_{S}(t)) \} \\
    & = -\frac{1}{\beta_{\alpha}} \text{Tr}\{(\mathcal{L}_{\alpha}\hat{\rho}_{S}(t))\ln \hat{\rho}_{G}(\beta_{\alpha},\mu_{\alpha})  \},
\end{align*}
con lo que finalmente, redefinimos la segunda ley de la termodinámica como

\begin{equation*}
    \dot{\sigma} = - \sum_{\alpha} \text{Tr}\{(\mathcal{L}_{\alpha}\hat{\rho}_{S}(t)) [\ln \hat{\rho}_{S}(t) -\ln \hat{\rho}_{G}(\beta_{\alpha},\mu_{\alpha}) ] \} \geq 0.
\end{equation*}

Así se obtiene que la razón de producción de entropía es mayor igual a cero.

\label{apendix:thermolaws}
	% Imagen, se numerará automáticamente con la letra del anexo según
	% la configuración \appendixindepobjnum

\newpage 

    \section{Cálculos realizados sección 4}
    \subsection{Producción de entropía y información}
    Partiendo de la definición de la entropía conjunta

    \begin{equation*}
        S^{XY} = - \sum_{x,y}p(x,y) \ln p(x,y),
    \end{equation*}
al derivar y utilizar $J_{x,x'}^{y,y'} = - J_{x',x}^{y',y}$ se obtiene

    \begin{align*}
        \partial_{t}S^{XY} & = - \sum_{x,y} \dot{p}(x,y) \ln p(x,y) - \sum_{x,y} \dot{p}(x,y) \\
                           & = - \sum_{x,x';y,y'} J_{x,x'}^{y,y'} \ln p(x,y)  \\
                           & = \sum_{x \geq x'; y\geq y'} J_{x,x'}^{y,y'} \ln \frac{p(x',y')}{p(x,y)} \\
                           & = \sum_{x \geq x'; y\geq y'} J_{x,x'}^{y,y'} \ln \frac{W_{x,x'}^{y,y'} p(x',y')}{W_{x',x}^{y',y} p(x,y)} +  \sum_{x \geq x'; y\geq y'} J_{x,x'}^{y,y'} \ln \frac{W_{x',x}^{y',y} }{W_{x,x'}^{y,y'} } \\
                           & = \dot{\sigma} - \dot{S}_{r}.
    \end{align*}

Para notar que la producción de entropía es mayor a cero, notemos que si $J_{x,x'}^{y,y'} > 0$, entonces $W_{x,x'}^{y,y'}p(x',y') > W_{x',x}^{y',y}p(x,y) $  y por ende

\begin{equation*}
    J_{x,x'}^{y,y'} \ln \frac{ W_{x,x'}^{y,y'}p(x',y') }{ W_{x',x}^{y',y}p(x,y) } > 0,
\end{equation*}
sucede de manera similar si $J_{x,x'}^{y,y'}<0$.\\
 Para la derivada temporal de la información mutua 

\begin{align*}
    \partial_{t} I_{xy} = \sum_{x,y}\dot{p}(x,y) \ln \frac{p(x,y) }{ p(x)p(y) } + \sum_{x,y}p(x)p(y) \frac{\partial}{\partial t} \left( \frac{p(x,y)}{p(x)p(y)} \right),
 \end{align*}
calculando la derivada

\begin{align*}
    \frac{\partial }{\partial t} \left( \frac{p(x,y)}{p(x)p(y)} \right) & = \frac{ \dot{p}(x,y)p(x)p(y) - (p(x)\dot{p}(y) + p(y)\dot{p}(x))p(x,y)   }{ (p(x)p(y))^{2} } \\
    & = \frac{\dot{p}(x,y)}{p(x)p(y) } - \frac{\dot{p}(y)p(x,y)  }{ p(x)p(y)^{2} } - \frac{\dot{p}(x)p(x,y)  }{ p(x)^{2}p(y) },  
\end{align*}
tendremos que

\begin{equation*}
    \partial_{t}I_{xy} = \sum_{x,y}\dot{p}(x,y) \ln \frac{p(x,y)}{ p(x)p(y) } + \sum_{x,y} \dot{p}(x,y) - \sum_{x,y} \dot{p}(y) \frac{p(x,y)}{p(y)} - \sum_{x,y} \dot{p}(x) \frac{p(x,y)}{p(x)}. 
\end{equation*}

Utilizando de las probabilidades conjuntas $\sum_{y}p(x,y) = p(x)$, $\sum_{x}p(x,y)  = p(y)$ y la conservación de la probabilidad

\begin{align*}
    \partial_{t}I_{xy} & = \sum_{x,y}\dot{p}(x,y) \ln \frac{p(x,y)}{ p(x)p(y) } \\
    & = \sum_{x,x';y,y'}J_{x,x'}^{y,y'} \ln \frac{p(x,y)}{ p(x)p(y) } \\
    & = \sum_{x,x';y}J_{x,x'}^{y} \ln \frac{p(x,y)}{p(x)p(y)} + \sum_{x;y,y'}J_{x}^{y,y'} \ln \frac{p(x,y)}{p(x)p(y)} \\
    & = \sum_{x\geq x';y}J_{x,x'}^{y} \left[ \ln \frac{p(x,y)}{p(x)p(y)} - \ln \frac{p(x',y)}{p(x')p(y)}  \right] \\
    & + \sum_{x;y\geq y'} J_{x}^{y,y'} \left[ \ln \frac{p(x,y)}{p(x)p(y)} - \ln \frac{p(x,y')}{p(x)p(y')} \right],
\end{align*}
y del teorema de bayes $p(x,y) = p(x)p(y|x)$, $p(x,y) = p(y)p(x|y)$ se tendrá

\begin{align*}
    \partial_{t}I_{xy} & = \sum_{x \geq x';y} J_{x,x'}^{y} \ln \frac{p(y|x) }{p(y|x')} + \sum_{x;y\geq y'} J_{x}^{y,y'} \ln \frac{p(x|y) }{p(x|y')} \\
                       & = \dot{I}^{X} + \dot{I}^{Y}. 
\end{align*}

    \label{apendix4:secondlaw}

	 % Desactiva el color de celda

\newpage

\section{Cálculos realizados seccion 5}
\subsection{Funciones de correlación para un baño de fermiones libres}
Para partir obteniendo las funciones correlación espectral, se partirá calculando

\begin{equation*}
    C_{1}^{\alpha}(s) = \text{Tr}[e^{is\hat{H}_{\alpha}}\hat{B}^{\dagger}_{\alpha,1}e^{-is\hat{H}_{\alpha}}\hat{B}_{\alpha,1}\hat{\tau}_{\alpha}  ]  \hspace{10mm} C_{-1}^{\alpha}(s) = \text{Tr}[e^{is\hat{H}_{\alpha}}\hat{B}^{\dagger}_{\alpha,-1}e^{-is\hat{H}_{\alpha}}\hat{B}_{\alpha,-1}\hat{\tau}_{\alpha}  ], 
\end{equation*}
tendremos que

\begin{equation*}
    C_{1}^{\alpha}(s) = \sum_{l,l'}t_{\alpha,l}t_{\alpha,l'}e^{i\epsilon_{\alpha,l}s} \langle \hat{c}^{\dagger}_{\alpha,l}\hat{c}_{\alpha,l'} \rangle \hspace{10mm} C
    ^{\alpha}_{-1}(s) = \sum_{l,l'}t_{\alpha,l}t_{\alpha,l'}e^{-i\epsilon_{\alpha,l}s} \langle \hat{c}_{\alpha,l}\hat{c}^{\dagger}_{\alpha,l'} \rangle,
\end{equation*}
debido a que el reservorio está en equilibrio gran canónico, tendremos que el valor de expectación $\langle \hat{c}^{\dagger}_{\alpha,l}\hat{c}_{\alpha,l}\rangle$ es la distribución de Fermi $f_{\alpha}(\epsilon_{\alpha,l})$ y por ende  

\begin{equation*}
    C_{1}^{\alpha}(s) = \sum_{l}t^{2}_{\alpha,l}e^{i\epsilon_{\alpha,l}s} f_{\alpha}(\epsilon_{\alpha,l}) \hspace{10mm} C
    ^{\alpha}_{-1}(s) = \sum_{l}t^{2}_{\alpha,l}e^{-i\epsilon_{\alpha,l}s} [1-f_{\alpha}(\epsilon_{\alpha,l})].
\end{equation*}

Definiendo la razón de tuneleo correspondiente al baño $\alpha$

\begin{equation*}
    \gamma_{\alpha}(\omega) = 2\pi \sum_{l}t^{2}_{\alpha,l}\delta(\omega-\epsilon_{\alpha,l}),
\end{equation*}
podremos escribir de manera integral las funciones correlación usando la razón de tuneleo

\begin{equation*}
    C_{1}^{\alpha}(s) = \frac{1}{2\pi} \int_{-\infty}^{\infty}d\omega e^{i\omega s} \gamma_{\alpha}(\omega) f_{\alpha}(\omega)  \hspace{10mm} C_{-1}^{\alpha}(s) = \frac{1}{2\pi} \int_{-\infty}^{\infty}d\omega e^{-i\omega s} \gamma_{\alpha}(\omega) [1-f_{\alpha}(\omega)],
\end{equation*}
así podemos encontrar las funciones correlación espectral usando la transformada de Fourier

\begin{equation*}
    \Gamma_{1}^{\alpha}(\omega) = \gamma_{\alpha}(-\omega)f_{\alpha}(-\omega)  \hspace{10mm} \Gamma_{-1}^{\alpha}(\omega) = \gamma_{\alpha}(\omega)[1-f_{\alpha}(\omega)].
\end{equation*}

Más adelante en el apéndice se profundizará en el aspecto Markoviano de la función correlación correspondiente a las funciones correlación espectral.
\label{apendix5bathcorre}

\subsection{ Operadores de salto de sistema de 3 puntos cuánticos}
Para poder obtener los operadores de salto, primero se buscará eliminar el término de acoplamiento en el Hamiltoniano del sistema, para ello se hará la transformación

\begin{align*}
    \hat{d}_{-} & = \cos(\theta/2)\hat{d}_{R} - \sin(\theta/2)\hat{d}_{L} \\
    \hat{d}_{+} & = \sin(\theta/2)\hat{d}_{R} + \cos(\theta/2)\hat{d}_{L},
\end{align*}
con $\cos \theta = \Delta/\sqrt{ \Delta^{2} + g^{2} }$ y $\Delta = (\epsilon_{L} - \epsilon_{R})/2$, además  

\begin{align*}
    \hat{d}_{R} & = \cos(\theta/2)\hat{d}_{-} + \sin(\theta/2)\hat{d}_{+} \\
    \hat{d}_{L} & = -\sin(\theta/2)\hat{d}_{-} + \cos(\theta/2)\hat{d}_{+},
\end{align*}
bajo estas definiciones, notemos que se cumplen las relaciones de anticonmutación entre los operadores $\hat{d}_{+}$ y $\hat{d}_{-}$. Por otro lado

\begin{align*}
    \hat{d}^{\dagger}_{R}\hat{d}_{R} & = \cos^{2}(\theta/2) \hat{d}^{\dagger}_{-}\hat{d}_{-} + \sin^{2}(\theta/2) \hat{d}^{\dagger}_{+}\hat{d}_{+} + \cos(\theta/2)\sin(\theta/2)[\hat{d}^{\dagger}_{+}\hat{d}_{-} + \hat{d}^{\dagger}_{-}\hat{d}_{+} ] \\
    \hat{d}^{\dagger}_{L}\hat{d}_{L} & = \sin^{2}(\theta/2) \hat{d}^{\dagger}_{-}\hat{d}_{-} + \cos^{2}(\theta/2) \hat{d}^{\dagger}_{+}\hat{d}_{+} - \cos(\theta/2)\sin(\theta/2)[\hat{d}^{\dagger}_{+}\hat{d}_{-} + \hat{d}^{\dagger}_{-}\hat{d}_{+} ],
\end{align*}
así se tendrá que $\hat{n}_{L} + \hat{n}_{R} = \hat{n}_{+} + \hat{n}_{-}$. Usando esta relación y la cantidad definida por $\bar{\epsilon} = (\epsilon_{R} + \epsilon_{L})/2$ 

\begin{equation}
    \epsilon_{R} \hat{n}_{R} + \epsilon_{L} \hat{n}_{L}  = \bar{\epsilon}( \hat{n}_{+} + \hat{n}_{-} ) - \Delta \sin(\theta) [\hat{d}^{\dagger}_{+}\hat{d}_{-} + \hat{d}^{\dagger}_{-}\hat{d}_{+}] - \Delta \cos(\theta) [\hat{d}^{\dagger}_{-}\hat{d}_{-} - \hat{d}^{\dagger}_{+}\hat{d}_{+}].
\label{apendix5:ec1}
\end{equation}

Por otro lado 

\begin{equation}
    \hat{d}^{\dagger}_{R}\hat{d}_{L} = \cos^{2}(\theta/2)\hat{d}^{\dagger}_{+}\hat{d}_{-} - \sin^{2}(\theta/2) \hat{d}^{\dagger}_{-}\hat{d}_{+}  + \sin(\theta/2)\cos(\theta/2)[ \hat{d}^{\dagger}_{+}\hat{d}_{+} - \hat{d}^{\dagger}_{-}\hat{d}_{-} ]
    \label{apendix5:ec2}
\end{equation}

\begin{equation}
    \hat{d}^{\dagger}_{L}\hat{d}_{R} = \cos^{2}(\theta/2)\hat{d}^{\dagger}_{-}\hat{d}_{+} - \sin^{2}(\theta/2) \hat{d}^{\dagger}_{+}\hat{d}_{-}  + \sin(\theta/2)\cos(\theta/2)[ \hat{d}^{\dagger}_{+}\hat{d}_{+} - \hat{d}^{\dagger}_{-}\hat{d}_{-} ],
    \label{apendix5:ec3}
\end{equation}
de la combinación de \ref{apendix5:ec1}, \ref{apendix5:ec2} y \ref{apendix5:ec3} obtenemos

\begin{equation*}
    \epsilon_{R} \hat{n}_{R} + \epsilon_{L} \hat{n}_{L} = (\bar{\epsilon} + \sqrt{\Delta^{2} + g^{2}})\hat{n}_{+} +  (\bar{\epsilon} - \sqrt{\Delta^{2} + g^{2}})\hat{n}_{-},
\end{equation*}
además de $\hat{n}^{2}_{i} = \hat{n}_{i}$ se obtiene 

\begin{align*}
    2 \hat{n}_{R}\hat{n}_{L} & = (\hat{n}_{R} +\hat{n}_{L})(\hat{n}_{R} +\hat{n}_{L}) - (\hat{n}_{R} + \hat{n}_{L}) \\
    & = (\hat{n}_{+} +\hat{n}_{-})(\hat{n}_{+} +\hat{n}_{-}) - (\hat{n}_{+} + \hat{n}_{-})  = 2\hat{n}_{+}\hat{n}_{-}.
\end{align*}

Definiendo $\epsilon_{\pm} = \bar{\epsilon} \pm \sqrt{\Delta^{2}+g^{2}}$, se obtendrá para el Hamiltoniano

\begin{equation}
    \hat{H}_{S} = \epsilon_{D}\hat{n}_{D} + \epsilon_{+}\hat{n}_{+} + \epsilon_{-}\hat{n}_{-} + U\hat{n}_{D}(\hat{n}_{+} + \hat{n}_{-}) + U_{f}\hat{n}_{+}\hat{n}_{-}.
    \label{apendix5:ec4}
\end{equation}

Con el Hamiltoniano escrito en función de los operadores $\hat{d}_{+}$ y $\hat{d}_{-}$ podremos calcular los operadores de salto, es decir

\begin{align*}
    e^{i \hat{H}_{S}t}\hat{d}_{D}e^{-i\hat{H}_{S}t} & = \hat{d}_{D} + it[\hat{H}_{S},\hat{d}_{D}] + \frac{(it)^{2}}{2} [\hat{H}_{S},[\hat{H}_{S},\hat{d}_{D}]] +... \\
    e^{i \hat{H}_{S}t}\hat{d}_{+}e^{-i\hat{H}_{S}t} & = \hat{d}_{+} + it[\hat{H}_{S},\hat{d}_{+}] + \frac{(it)^{2}}{2} [\hat{H}_{S},[\hat{H}_{S},\hat{d}_{+}]] +... \\
    e^{i \hat{H}_{S}t}\hat{d}_{-}e^{-i\hat{H}_{S}t} & = \hat{d}_{-} + it[\hat{H}_{S},\hat{d}_{-}] + \frac{(it)^{2}}{2} [\hat{H}_{S},[\hat{H}_{S},\hat{d}_{-}]] +...,
\end{align*}
partiremos primero por $\hat{d}_{D}$ para ello se usará la identidad

\begin{equation*}
    \textbf{1} = (\textbf{1} - \hat{n}_{+})(\textbf{1}-\hat{n}_{-}) + (\textbf{1} - \hat{n}_{+})\hat{n}_{-} + (\textbf{1} - \hat{n}_{-})\hat{n}_{+} + \hat{n}_{+}\hat{n}_{-},
\end{equation*}
utilizando que $(\textbf{1} - \hat{n}_{i})\hat{n}_{i} = \textbf{0}$ se tendrá que

\begin{align*}
    [\hat{H}_{S},\hat{d}_{D}(\textbf{1}-\hat{n}_{+})(\textbf{1} - \hat{n}_{-})] & = - \epsilon_{D}\hat{d}_{D}(\textbf{1}-\hat{n}_{+})(\textbf{1} - \hat{n}_{-}) \\
    [\hat{H}_{S},\hat{d}_{D}(\textbf{1}-\hat{n}_{+})\hat{n}_{-}] & = - (\epsilon_{D} + U)\hat{d}_{D}(\textbf{1} - \hat{n}_{+})\hat{n}_{-} \\
    [\hat{H}_{S},\hat{d}_{D}(\textbf{1}-\hat{n}_{-})\hat{n}_{+}] & = - (\epsilon_{D} + U)\hat{d}_{D}(\textbf{1} - \hat{n}_{-})\hat{n}_{+} \\
    [\hat{H}_{S},\hat{d}_{D}\hat{n}_{+}\hat{n}_{-}] & = - (\epsilon_{D} + 2U)\hat{d}_{D}\hat{n}_{+}\hat{n}_{-}, 
\end{align*}
lo que permite notar que

\begin{align*}
    [\hat{H}_{S},[\hat{H}_{S},\hat{d}_{D}]] & = (\epsilon_{D})^{2}\hat{d}_{D}(\textbf{1} - \hat{n}_{+}) (\textbf{1} - \hat{n}_{-}) + (\epsilon_{D} + U)^{2}\hat{d}_{D}(\textbf{1} - \hat{n}_{+})\hat{n}_{-} \\
        & + (\epsilon_{D}+U)^{2}\hat{d}_{D}(\textbf{1} - \hat{n}_{-})\hat{n}_{+} + (\epsilon_{D} + 2U)^{2}\hat{d}_{D}\hat{n}_{+}\hat{n}_{-}.
\end{align*}

Y así obtener de manera recursiva

\begin{align*}
    e^{i \hat{H}_{S}t}\hat{d}_{D}e^{-i\hat{H}_{S}t}  & = e^{-i\epsilon_{D}t} \hat{d}_{D}(\textbf{1} - \hat{n}_{+}) (\textbf{1} - \hat{n}_{-}) +  e^{-i(\epsilon_{D}+U)t} \hat{d}_{D}[(\textbf{1} - \hat{n}_{+})\hat{n}_{-} + (\textbf{1} - \hat{n}_{-})\hat{n}_{+}] \\
    & + e^{-i(\epsilon_{D} + 2U)t}\hat{d}_{D} \hat{n}_{+}\hat{n}_{-}.
\end{align*}

Para poder describir los operadores de salto de $\hat{d}_{+}$ se usará el 1 conveniente

\begin{equation*}
    \textbf{1} = (\textbf{1}-\hat{n}_{D})(\textbf{1}-\hat{n}_{-}) + (\textbf{1}-\hat{n}_{-})\hat{n}_{D} + (\textbf{1}-\hat{n}_{D})\hat{n}_{-} + \hat{n}_{D}\hat{n}_{-},
\end{equation*}
los conmutadores quedarán

\begin{align*}
    [\hat{H}_{S},\hat{d}_{+}(\textbf{1}-\hat{n}_{D})(\textbf{1} - \hat{n}_{-})] & = - \epsilon_{+}\hat{d}_{+}(\textbf{1}-\hat{n}_{D})(\textbf{1} - \hat{n}_{-}) \\
    [\hat{H}_{S},\hat{d}_{+}(\textbf{1}-\hat{n}_{D})\hat{n}_{-}] & = - (\epsilon_{+} + U)\hat{d}_{+}(\textbf{1} - \hat{n}_{-})\hat{n}_{D} \\
    [\hat{H}_{S},\hat{d}_{+}(\textbf{1}-\hat{n}_{-})\hat{n}_{D}] & = - (\epsilon_{+} + U_{f})\hat{d}_{+}(\textbf{1} - \hat{n}_{D})\hat{n}_{-} \\
    [\hat{H}_{S},\hat{d}_{+}\hat{n}_{D}\hat{n}_{-}] & = - (\epsilon_{+} + U + U_{f})\hat{d}_{+}\hat{n}_{D}\hat{n}_{-}, 
\end{align*}
de manera recursiva se obtendrá que los operadores de salto

\begin{align*}
    e^{i \hat{H}_{S}t}\hat{d}_{+}e^{-i\hat{H}_{S}t} & = e^{-i\epsilon_{+}t}\hat{d}_{+}(\textbf{1}-\hat{n}_{D})(\textbf{1}-\hat{n}_{-}) + e^{-i(\epsilon_{+}+U_{f})t} \hat{d}_{+}(\textbf{1}-\hat{n}_{D})\hat{n}_{-} \\
    & + e^{-i(\epsilon_{+}+U)t}\hat{d}_{+}(\textbf{1}-\hat{n}_{-})\hat{n}_{D} + e^{-i(\epsilon_{+}+U+U_{f})t}\hat{d}_{+}\hat{n}_{-}\hat{n}_{D},
\end{align*}
debido a la simetría, tendremos que

\begin{align*}
    e^{i \hat{H}_{S}t}\hat{d}_{-}e^{-i\hat{H}_{S}t} & = e^{-i\epsilon_{-}t}\hat{d}_{-}(\textbf{1}-\hat{n}_{D})(\textbf{1}-\hat{n}_{+}) + e^{-i(\epsilon_{-}+U_{f})t} \hat{d}_{-}(\textbf{1}-\hat{n}_{D})\hat{n}_{+} \\
    & + e^{-i(\epsilon_{-}+U)t}\hat{d}_{-}(\textbf{1}-\hat{n}_{+})\hat{n}_{D} + e^{-i(\epsilon_{-}+U+U_{f})t}\hat{d}_{-}\hat{n}_{+}\hat{n}_{D}.
\end{align*}

Ahora se puede regresar a los operadores locales del sistema $\hat{d}_{L}$ y $\hat{d}_{R}$

\begin{align*}
    e^{i\hat{H}_{S}t}\hat{d}_{R}e^{-i\hat{H}_{S}t} & = \cos(\theta/2)e^{i\hat{H}_{S}t}\hat{d}_{-}e^{-i\hat{H}_{S}t} + \sin(\theta/2)e^{i\hat{H}_{S}t}\hat{d}_{+}e^{-i\hat{H}_{S}t}  \\
    e^{i\hat{H}_{S}t}\hat{d}_{L}e^{-i\hat{H}_{S}t} & = -\sin(\theta/2)e^{i\hat{H}_{S}t}\hat{d}_{-}e^{-i\hat{H}_{S}t} + \cos(\theta/2)e^{i\hat{H}_{S}t}\hat{d}_{+}e^{-i\hat{H}_{S}t},
\end{align*}
con lo que encontramos las frecuencias de Bohr correspondientes a

\begin{align*}
    \hat{d}_{D} & \to (\epsilon_{D}, \epsilon_{D}+U,\epsilon_{D}+2U) \\
    \hat{d}_{L} & \to (\epsilon_{+},\epsilon_{-},\epsilon_{+}+U,\epsilon_{-}+U,\epsilon_{+}+U_{f},\epsilon_{-}+U_{f},\epsilon_{+}+U+U_{f},\epsilon_{-}+U+U_{f}) \\
    \hat{d}_{R} & \to (\epsilon_{+},\epsilon_{-},\epsilon_{+}+U,\epsilon_{-}+U,\epsilon_{+}+U_{f},\epsilon_{-}+U_{f},\epsilon_{+}+U+U_{f},\epsilon_{-}+U+U_{f}).
\end{align*}

\label{apendix5jumpop}

\subsection{Agrupación de frecuencias para 2 puntos cuánticos}
En el caso en que $\epsilon_{L} = \epsilon_{R}$ se cumple que $\epsilon_{\pm} = \epsilon \pm g$. Aquí es que se invoca la agrupación de frecuencias, ya que si el parámetro $g$ es pequeño en relación al inverso del tiempo de correlación del baño, podremos agrupar las frecuencias tal que

\begin{align*}
    &(\epsilon_{+},\epsilon_{-})  \to \epsilon \\
    &(\epsilon_{+}+U,\epsilon_{-}+U)  \to \epsilon+U \\
    &(\epsilon_{+}+U_{f},\epsilon_{-}+U_{f})  \to \epsilon+U_{f} \\
    &(\epsilon_{+}+U+U_{f},\epsilon_{-}+U+U_{f})  \to \epsilon+U+U_{f}.
\end{align*}

Para encontrar los nuevos operadores de salto de $\hat{d}_{L}$, para el primer caso se deberá sumar los operadores de salto respectivos a las frecuencias $\epsilon_{+},\epsilon_{-}$ 

\begin{align*}
    \frac{\hat{d}_{+}(\textbf{1}-\hat{n}_{D})(\textbf{1}-\hat{n}_{-})}{\sqrt{2}} - \frac{\hat{d}_{-}(\textbf{1}-\hat{n}_{D})(\textbf{1}-\hat{n}_{+})}{\sqrt{2}} & = \frac{(\textbf{1}-\hat{n}_{D})}{\sqrt{2}} [\hat{d}_{+}(\textbf{1}-\hat{n}_{-}) - \hat{d}_{-}(\textbf{1}-\hat{n}_{+}) ],
\end{align*}
por otro lado

\begin{align*}
    \hat{d}_{+}(\textbf{1}-\hat{n}_{-}) - \hat{d}_{-}(\textbf{1}-\hat{n}_{+}) & = \frac{\hat{d}_{L} + \hat{d}_{R}}{\sqrt{2}} (\textbf{1} - \hat{n}_{-}) - \frac{\hat{d}_{R} - \hat{d}_{L}}{\sqrt{2}}(\textbf{1}-\hat{n}_{+}) \\
    & = \frac{\hat{d}_{L}}{\sqrt{2}}(\textbf{2} - \hat{n}_{+} -\hat{n}_{-} ) + \frac{\hat{d}_{R}}{\sqrt{2}}(\hat{n}_{+} -\hat{n}_{-}) \\
    & = \frac{\hat{d}_{L}}{\sqrt{2}}(\textbf{2}-\hat{n}_{L} -\hat{n}_{R}) + \frac{\hat{d}_{R}}{\sqrt{2}}(\hat{d}^{\dagger}_{R}\hat{d}_{L}+\hat{d}^{\dagger}_{L}\hat{d}_{R}),
\end{align*}
de $\hat{d}_{L}(\textbf{1} - \hat{n}_{L}) = \hat{d}_{L}\hat{d}_{L}\hat{d}^{\dagger}_{L} = 0$ y de $\hat{d}_{R}(\hat{d}^{\dagger}_{R}\hat{d}_{L} + \hat{d}^{\dagger}_{L}\hat{d}_{R}) = \hat{d}_{L}(\textbf{1}-\hat{n}_{R})$ nos queda

\begin{equation*}
    \hat{d}_{+}(\textbf{1}-\hat{n}_{-}) - \hat{d}_{-}(\textbf{1}-\hat{n}_{+})  = \hat{d}_{L}(\textbf{1} - \hat{n}_{R}),
\end{equation*}
con lo que podremos asignar a la frecuencia $\epsilon$ el operador $\hat{d}_{L}(\textbf{1}-\hat{n}_{D})(\textbf{1}-\hat{n}_{R})$. Siguiendo con las frecuencias $(\epsilon_{-}+U,\epsilon_{+}+U) \to \epsilon + U$ se obtiene

\begin{align*}
    \frac{\hat{d}_{+}\hat{n}_{D}(\textbf{1}-\hat{n}_{-})}{\sqrt{2}} - \frac{\hat{d}_{-}\hat{n}_{D}(\textbf{1}-\hat{n}_{+})}{\sqrt{2}} = \hat{d}_{L}\hat{n}_{D}(\textbf{1}-\hat{n}_{R}). 
\end{align*}

Para las frecuencias $(\epsilon_{-}+U_{f},\epsilon_{+}+U_{f}) \to \epsilon+U_{f}$ tendremos que sumar

\begin{align*}
    \frac{\hat{d}_{+}\hat{n}_{-}(\textbf{1}-\hat{n}_{D})}{\sqrt{2}} - \frac{\hat{d}_{-}\hat{n}_{+}(\textbf{1}-\hat{n}_{D})}{\sqrt{2}} & = \frac{(\textbf{1}-\hat{n}_{D})}{\sqrt{2}}[\hat{d}_{+}\hat{n}_{-} -\hat{d}_{-}\hat{n}_{+}],
\end{align*}
es así que

\begin{align*}
    \hat{d}_{+}\hat{n}_{-} -\hat{d}_{-}\hat{n}_{+} & = \frac{\hat{d}_{R}+\hat{d}_{L}}{\sqrt{2}}\hat{n}_{-} - \frac{\hat{d}_{R}-\hat{d}_{L}}{\sqrt{2}}\hat{n}_{+} \\
    & = - \frac{\hat{d}_{R}}{\sqrt{2}}(\hat{d}^{\dagger}_{R}\hat{d}_{L}+ \hat{d}^{\dagger}_{L}\hat{d}_{R}) + \frac{\hat{d}_{L}}{\sqrt{2}}(\hat{n}_{L}+\hat{n}_{R}) \\
    & = \frac{2\hat{d}_{L}}{\sqrt{2}}\hat{n}_{R} - \frac{\hat{d}_{L}}{\sqrt{2}}(\textbf{1}-\hat{n}_{L}) \\
    & = \frac{2\hat{d}_{L}}{\sqrt{2}}\hat{n}_{R}.
\end{align*}

Con lo que el operador de salto respectivo a la frecuencia $(\epsilon+U_{f})$ es $\hat{d}_{L}\hat{n}_{R}(\textbf{1}-\hat{n}_{D})$, usando este mismo análisis se obtiene que el operador de salto de la frecuencia $\epsilon+U+U_{f}$ corresponde a $\hat{d}_{L}\hat{n}_{D}\hat{n}_{R}$. Finalmente recapitulando

\begin{align*}
    &\epsilon \to \hat{d}_{L}(\textbf{1}-\hat{n}_{R})(\textbf{1}-\hat{n}_{D})\\
   &\epsilon + U \to \hat{d}_{L}\hat{n}_{D}(\textbf{1}-\hat{n}_{R})\\
    &\epsilon +U_{f}\to \hat{d}_{L}\hat{n}_{R}(\textbf{1}-\hat{n}_{D})\\
    &\epsilon +U+U_{f}\to \hat{d}_{L}\hat{n}_{D}\hat{n}_{R},
\end{align*}
de manera similar, para el operador local $\hat{d}_{R}$ se puede partir agrupando las energías $(\epsilon_{+},\epsilon_{-})$ en donde

\begin{align*}
    \frac{\hat{d}_{+}(\textbf{1}-\hat{n}_{D})(\textbf{1}-\hat{n}_{-}) }{\sqrt{2}} + \frac{\hat{d}_{-}(\textbf{1}-\hat{n}_{D})(\textbf{1}-\hat{n}_{+}) }{\sqrt{2}} & = \frac{(\textbf{1} - \hat{n}_{D})}{\sqrt{2}}[\hat{d}_{+}(\textbf{1}-\hat{n}_{-}) + \hat{d}_{-}(\textbf{1}-\hat{n}_{+})],
\end{align*}
se deberá calcular 

\begin{align*}
    \hat{d}_{+}(\textbf{1}-\hat{n}_{-}) + \hat{d}_{-}(\textbf{1}-\hat{n}_{+}) & =  \frac{\hat{d}_{R} + \hat{d}_{L}}{\sqrt{2}}(\textbf{1} - \hat{n}_{-}) + \frac{\hat{d}_{R} - \hat{d}_{L}}{\sqrt{2}}(\textbf{1} - \hat{n}_{+})   \\
    & = \frac{\hat{d}_{R}}{\sqrt{2}}(\textbf{2} - (\hat{n}_{L} + \hat{n}_{R})) + \frac{\hat{d}_{L}}{\sqrt{2}} (\hat{n}_{+}-\hat{n}_{-}) \\
    & = \frac{\hat{d}_{R}}{\sqrt{2}}(\textbf{1} -  \hat{n}_{R}) + \frac{\hat{d}_{L}}{\sqrt{2}} (\hat{d}^{\dagger}_{R}\hat{d}_{L} + \hat{d}^{\dagger}_{L}\hat{d}_{R}) \\
    & = \sqrt{2}\hat{d}_{R}(\textbf{1}-\hat{n}_{R}),
\end{align*}
por lo tanto el operador de salto asociado a la agrupación $(\epsilon_{+},\epsilon_{-})\to \epsilon$ es $\hat{d}_{R}(\textbf{1}-\hat{n}_{D})(\textbf{1}-\hat{n}_{L})$. Por el mismo motivo al calcular el operador que corresponde a agrupar las frecuencias $(\epsilon_{+}+U,\epsilon_{-}+U)$ 

\begin{equation*}
    \frac{\hat{d}_{+}(\textbf{1}-\hat{n}_{-})\hat{n}_{D}}{\sqrt{2}} + \frac{\hat{d}_{-}(\textbf{1}-\hat{n}_{+})\hat{n}_{D}}{\sqrt{2}}  = \hat{d}_{R}\hat{n}_{D}(\textbf{1}-\hat{n}_{L}).
\end{equation*}

Para las frecuencias $(\epsilon_{+}+U_{f},\epsilon_{-}+U_{f})\to \epsilon + U_{f}$ tendremos que sumar los operadores

\begin{equation*}
    \frac{\hat{d}_{+}(\textbf{1}-\hat{n}_{D})\hat{n}_{-} }{\sqrt{2}} + \frac{\hat{d}_{-}(\textbf{1}-\hat{n}_{D})\hat{n}_{+} }{\sqrt{2}} = \frac{(\textbf{1}-\hat{n}_{D})}{\sqrt{2}} [\hat{d}_{+}\hat{n}_{-} + \hat{d}_{-}\hat{n}_{+}],
\end{equation*}
calculando la suma

\begin{align*}
    \hat{d}_{+}\hat{n}_{-} + \hat{d}_{-}\hat{n}_{+} & = \frac{\hat{d}_{R}+\hat{d}_{L}}{\sqrt{2}}\hat{n}_{-} + \frac{\hat{d}_{R}-\hat{d}_{L}}{\sqrt{2}}\hat{n}_{+}  \\
    & = \frac{\hat{d}_{R}}{\sqrt{2}}(\hat{n}_{L}+\hat{n}_{R}) - \frac{\hat{d}_{L}}{\sqrt{2}}(\hat{d}^{\dagger}_{L}\hat{d}_{R}+\hat{d}^{\dagger}_{R}\hat{d}_{L}) \\
    & = \sqrt{2}\hat{d}_{R}\hat{n}_{L},
\end{align*}
por lo tanto el operador de salto de $\hat{d}_{R}$ al agrupar las frecuencias $(\epsilon_{+}+U_{f},\epsilon_{-}+U_{f})$ es $\hat{d}_{R}(\textbf{1}-\hat{d}_{D})\hat{n}_{L}$, realizando el mismo cálculo se encuentra que para las frecuencias $(\epsilon_{+}+U+U_{f},\epsilon_{-}+U+U_{f})$ el operador de salto corresponde a $\hat{d}_{R}\hat{n}_{D}\hat{n}_{L}$. Por ende recapitulando, para el operador $\hat{d}_{R}$ se tendrá

\begin{align*}
    &\epsilon \to \hat{d}_{R}(\textbf{1}-\hat{n}_{L})(\textbf{1}-\hat{n}_{D})\\
   &\epsilon + U \to \hat{d}_{R}\hat{n}_{D}(\textbf{1}-\hat{n}_{L})\\
    &\epsilon +U_{f}\to \hat{d}_{R}\hat{n}_{L}(\textbf{1}-\hat{n}_{D})\\
    &\epsilon +U+U_{f}\to \hat{d}_{R}\hat{n}_{D}\hat{n}_{L}.
\end{align*}

\label{apendix5frecuencygroup}

%%%%%%%%%%%%%%%%%%%%%%%%%ojooooaqui%%%%%%%%%%%%%
\subsection{Límitaciones de la aproximación semilocal}

\label{apendix5límites}

\subsection{Aspecto Markoviano de las funciones correlación }
Para que el sistema descrito en la sección \ref{sec5:modelo} tenga un comportamiento cercano a un Demonio de Maxwell autónomo, se necesita que las razones de tuneleo $\gamma_{i}(\omega)$ modifiquen su respuesta dependiendo de si el punto cuántico del nivel $D$ este ocupado o desocupado. No todo comportamiento de la razón de tuneleo aseguran que el comportamiento del sistema sea Markoviano, esto se puede determinar calculando la función correlación del reservorio y analizando el orden del tiempo de correlación $\tau_{B}$ que presenta este mismo. Para poder lograr hacer esto se deben calcular las funciones correlación que para el caso de un ambiente fermiónico son de la forma

\begin{equation*}
    C^{\sigma}(t) = \frac{1}{2\pi} \int_{-\infty}^{\infty} d\omega e^{i\sigma \omega t} \gamma(\omega) f_{F}(\sigma \beta(\omega-\mu)),
\end{equation*}
en donde $f_{F}(x) = (\exp(x)+1)^{-1}$ y $\sigma = \pm$. La distribución de Fermi puede ser aproximada utilizando aproximantes de Padé \cite{hu2011pade,schinabeck2019hierarchical}. Esto permite escribir la distribución como sumatoria

\begin{equation}
    f_{F}(x) \approx \frac{1}{2} - \sum_{l=0}^{N} \frac{ 2 \kappa_{l}x }{ x^{2} + \xi^{2}_{l} },
    \label{apendix5:pade}
\end{equation}
los coeficientes $\kappa_{l}$ y $\xi_{l}$ se pueden calcular numéricamente y se encuentran en \cite{hu2011pade}. Para poder calcular numéricamente la función correlación, se necesita una forma explícita para $\gamma(\omega)$ el cuál tiene que tener un máximo en alguna frecuencia $\epsilon_{0}$. Se puede elegir un comportamiento lorentziano de la forma

\begin{equation*}
    \gamma(\omega) = \gamma_{0} + \frac{\gamma_{f} W^{2}}{(\omega - \epsilon_{0})^{2} + W^{2}},
\end{equation*}
con $W$ el ancho de la lorentziana y $\gamma_{0}$, $\gamma_{f}$ constantes. Para el estudio del Demonio de Maxwell hay dos casos importantes que hay que diferenciar, es decir cuándo el punto cuántico $D$ esta desocupado o ocupado, lo que se traduce en evaluar las razones de tuneleo en dos frecuencias $\epsilon_{0}$ y $\epsilon_{0}+U$ como aparece en la Figura \ref{img:gammas}

\insertimage[\label{img:gammas}]{ejemplos/gammmas.pdf}{scale=0.65}{Descripción gráfica de las razones de tuneleo en función de las frecuencias $\omega$.}

Para obtener razones de tuneleo que cumplan estos valores, se puede realizar el sistema de ecuaciones 

\begin{align*}
    \gamma(\epsilon_{0}) & = \gamma_{0} + \gamma_{f} \\
    \gamma(\epsilon_{0} + U) & = \gamma_{0} + \frac{\gamma_{f} W^{2} }{U^{2} + W^{2}}.
\end{align*}

Con estas igualdades se pueden determinar los valores de $\gamma_{0}$, $\gamma_{f}$ y $W$. Reemplazando $\gamma(\omega)$ en la función correlación se tendrá

\begin{align*}
    C^{\sigma}(t) & = \frac{\gamma_{0}}{2\pi} \int_{-\infty}^{\infty}d\omega e^{i\sigma \omega t} f_{F}(\sigma \beta (\omega-\mu)) + \frac{\gamma_{f}}{2\pi} \int_{-\infty}^{\infty}d\omega e^{i\sigma \omega t}\left[ \frac{W^{2}}{(\omega-\epsilon_{0})^{2} + W^{2}} \right] f_{F}(\sigma \beta (\omega-\mu)),
\end{align*}
la función correlación se puede separar en dos partes

\begin{equation*}
    C^{\sigma}_{0}(t) = \frac{\gamma_{0}}{2\pi} \int_{-\infty}^{\infty}d\omega e^{i\sigma \omega t} f_{F}(\sigma \beta (\omega-\mu)) \hspace{12mm} C^{\sigma}_{f}(t) = \frac{\gamma_{f}}{2\pi} \int_{-\infty}^{\infty}d\omega e^{i\sigma \omega t}\left[ \frac{W^{2}}{(\omega-\epsilon_{0})^{2} + W^{2}} \right] f_{F}(\sigma \beta (\omega-\mu)),
\end{equation*}
y podremos determinar el tiempo de correlación del baño como $\tau_{B} = \max\{\tau_{B0},\tau_{Bf}\}$. Se partirá calculando una expresión analítica aproximada de la función correlación $C^{\sigma}_{f}(t)$, para realizar este integral se deberá localizar los polos en el plano complejo, para ello notemos que para \ref{apendix5:pade} los polos se ubican en $\omega = \pm i \xi_{l}/\beta + \mu $. Mientras que para la Lorentziana los polos se ubican en $\omega = \pm i W + \omega_{0}$. Para poder realizar este integral se usará el teorema del residuo \cite{riley2006mathematical}, tomando como zona de integración un semicírculo en el hemisferio positivo o negativo dependiendo del signo de $\sigma$. Primero el residuo de la Lorentziana consiste en 

\begin{align*}
   \text{Res} \left[ \frac{e^{ i\sigma \omega t} f_{F}[\sigma \beta (\omega-\mu)] }{ (\omega-\omega_{0} + iW)(\omega -\omega_{0} -iW)}\right]_{\omega = \pm i W + \omega_{0}} = \frac{1}{\pm 2iW} ( e^{\mp \sigma Wt} f_{F}[\sigma \beta(\pm iW +\omega_{0}-\mu)]e^{i\sigma \omega_{0}t}). 
\end{align*}

Si se cumple que $\sigma > 0$ se toma el hemisferio inferior para la integración, mientras que para $\sigma<0$ se toma el hemosferio superior, para así obtener 

\begin{align*}
    \text{Res} \left[ \frac{e^{ i\sigma \omega t} f_{F}[\sigma \beta (\omega-\mu)] }{ (\omega-\omega_{0} + iW)(\omega -\omega_{0} -iW)}\right]_{\omega = \pm i W + \omega_{0}} = \frac{1}{ 2iW} ( e^{i\sigma \omega_{0}t}e^{- Wt} f_{F}[i\beta W + \sigma \beta(\omega_{0}-\mu)]), 
 \end{align*}
para encontrar el residuo de los polos restantes, deberemos calcular 

\begin{align*}
    \text{Res} \left[ \frac{-e^{i\sigma \omega t}}{(\omega - \omega_{0})^{2} + W^{2} } \frac{ 2\kappa_{l}[\sigma \beta (\omega-\mu)] }{ (\beta(\omega-\mu) + i \xi_{l} )(\beta(\omega-\mu) - i \xi_{l})} \right]_{\omega = \pm i \xi_{l}/\beta + \mu} & = \frac{-1}{\beta} \frac{e^{- \frac{\sigma \xi_{l}}{\beta}t} e^{i\sigma \mu t}(\pm \kappa_{l}\sigma )}{ [ \frac{ \pm i\xi_{l}}{\beta} + (\mu - \omega_{0}) ]^{2} + W^{2} } ,
\end{align*}
eligiendo la zona de integración dependiendo del signo de $\sigma$ obtendremos que 

\begin{align*}
    \text{Res} \left[ \frac{-e^{i\sigma \omega t}}{(\omega - \omega_{0})^{2} + W^{2} } \frac{ 2\kappa_{l}[\sigma \beta (\omega-\mu)] }{ (\beta(\omega-\mu) + i \xi_{l} )(\beta(\omega-\mu) - i \xi_{l})} \right]_{\omega = \pm i \xi_{l}/\beta + \mu} & = \frac{-1}{\beta} \frac{e^{- \frac{ \xi_{l}}{\beta}t} e^{i\sigma \mu t} \kappa_{l} }{ [ \frac{ \sigma i\xi_{l}}{\beta} + (\mu - \omega_{0}) ]^{2} + W^{2} }. 
\end{align*}

Encontrando los residuos explícitamente se podrá escribir la función correlación como

\begin{equation}
    C^{\sigma}_{f}(t) \approx \sum_{l=0}^{N} \eta^{\sigma,l} e^{-\gamma_{\sigma,l} t},
    \label{correlationf}
\end{equation}
en donde 

\begin{equation*}
    \eta^{\sigma,l} = \left\{ \begin{array}{lc} \frac{\gamma_{f}W}{2} f_{F}[i\beta W + \sigma \beta (\omega_{0}-\mu)]  & l = 0 \\ \\ - \frac{i\kappa_{l}}{\beta} \left(\frac{\gamma_{f}W^{2}}{ (\frac{i\sigma \xi_{l}}{\beta} + (\mu-\omega_{0}))^{2} + W^{2} } \right) &  l \neq 0 \end{array} \right.
\end{equation*}

\begin{equation*}
    \gamma_{\sigma,l} =  \left\{ \begin{array}{lc} W- \sigma i \omega_{0}  & l = 0 \\ \\ \frac{\xi_{l}}{\beta} - \sigma i \mu &  l \neq 0 \end{array} \right..
\end{equation*}

Esto permite una expresión numérica para la función correlación $C^{\sigma}_{f}(t)$ y así evaluar su tiempo de correlación $\tau_{Bf}$. Para la otra función correlación se podrá encontrar el integral de manera analítica, es decir calcular

\begin{align*}
       C_{0}^{\sigma}(t) = \frac{\gamma_{0}}{2\pi} \int_{-\infty}^{\infty} d\omega e^{i\sigma \omega t }f_{F}(\sigma \beta (\omega -  \mu)).
\end{align*}    

Se puede partir notando que esto consiste en calcular la Transformada de Fourier de $\mathcal{F}(f[\sigma (\omega - \mu)])(-\sigma t)$ de la función $f(\sigma(\omega-\mu))= f_{F}(\sigma \beta (\omega-\mu) )$, utilizando la propiedad

\begin{equation*}
\mathcal{F}(f[\sigma (\omega - \mu)])(-\sigma t) = e^{-i \sigma \mu t}\mathcal{F}(f[\sigma \omega])(-\sigma t),
\end{equation*}
es decir, podemos centrarnos en calcular la transformación

\begin{align*}
    \mathcal{F}(f[\sigma \omega])(-\sigma t) &= \frac{\gamma_{0}}{2\pi} \int_{-\infty}^{\infty} d\omega \frac{e^{i\sigma \omega t}}{e^{\sigma \beta \omega} +1 } \\
        & = \frac{\gamma_{0}}{4\pi} \left[\int_{-\infty}^{\infty}d \omega e^{i\sigma \omega t} - \int_{-\infty}^{\infty}d\omega e^{i\sigma \omega t} \tanh \left(\frac{ \sigma \beta \omega }{2} \right)   \right] \\
        & =  \frac{\gamma_{0}}{2} \left[\delta(t) - \frac{i}{\beta \sigma \sinh(\pi t/\beta \sigma)} \right],
\end{align*}
y así finalmente

\begin{equation*}
    C^{\sigma}_{0}(t) = \frac{\gamma_{0}}{2}e^{-i\sigma \mu t} \left[\delta(t) - \frac{i}{\beta \sigma \sinh(\pi t/\beta \sigma)} \right].
\end{equation*}

Gracias a que se tiene una expresión analítica para $C^{\sigma}_{0}(t)$ se puede determinar el $\tau_{B0}$. Se debe concentrar principalmente en la parte con $\sinh(\pi t/\beta \sigma)$ debido a que $\sigma=\pm 1$ sólo nos interesa analizar la cantidad $\pi t/\beta$, ya que para tiempos $t > \beta$ se podrá considerar que $1/\sinh(\pi t/\beta \sigma) \propto \exp(-t/\beta)$, por lo tanto para está función correlación el tiempo de correlación del baño consiste en el inverso de la temperatura $\tau_{B0} = \beta$. Con esto finalmente se podrá obtener el tiempo de correlación asociado a un sólo reservorio como $\tau_{B} = \max \{\beta,\tau_{Bf} \}$.  

\label{appendix5correlation}

\subsection{Cálculo numérico para $\tau_{Bf}$}
Para poder calcular los tiempos de correlación $\tau_{Bf}$ correspondiente a los baños $L$ y $R$ se podrá usar la expresión \ref{correlationf}. Para poder determinar este $\tau_{Bf}$ se busca el punto en que la función correlación se anula, y luego para poder determinar si sigue un comportamiento markoviano se debe tener que $\tau_{R}\gg \tau_{Bf}$, con esto dicho se procede a calcular las partes imaginarias y reales de las funciones correlación correspondientes a los baños $L$ y $R$ 

\insertimage[\label{img:correlacion}]{ejemplos/corre.pdf}{scale=0.7}{Gráficos de la parte real e imaginaria de las funciones correlación en función del tiempo para los baños $L$ y $R$ con una diferencia de potencial correspondiente a $eV=20$ y $\sigma=+$.}
\insertimage[\label{img:correlacionmin}]{ejemplos/correminus.pdf}{scale=0.7}{Gráficos de la parte real e imaginaria de las funciones correlación en función del tiempo para los baños $L$ y $R$ con una diferencia de potencial correspondiente a $eV=20$ y $\sigma=-$.}

De la Figura \ref{img:correlacion} y \ref{img:correlacionmin} se puede notar que para $t>0.4$ las funciones correlación para ambos baños ya son practicamente cero. Ahora para poder analizar la Markovianeidad, se puede graficar la evolución del sistema y ver en que tiempo las componentes de la matriz densidad relajan al estado estacionario. Para ello se grafica 


\insertimage[\label{img:probas}]{ejemplos/probas.png}{scale=0.4}{Gráficos de las partes diagonales de la matriz densidad del sistema $LRD$ en función del tiempo.}

Con la Figura \ref{img:probas} se puede visualizar que el tiempo de relajación del sistema es $\tau_{R}>1000$ y por lo tanto permite corrobar que con respecto a los tiempos $\tau_{Bf}$, se cumple que $\tau_{Bf}/\tau_{R} < 0.0004 $. Este análisis se probo para distintos valores de $eV$ los cuáles dieron valores similares para $\tau_{Bf}$ y el tiempo de relajación $\tau_{R}$. 

\label{appendix5tauf}

\newpage 

\subsection{Concurrencia}
En está sección se incluye la demostración de la fórmula de concurrencia. Para ello, se requiere calcular la concurrencia en el subsistema $\hat{\rho}_{LR}$, por lo tanto se tiene que partir calculando 

\begin{equation*}
    \hat{\rho}_{LR} = \text{Tr}_{D}\{ \hat{\rho_{S}} \},
\end{equation*}
usando la base $\{|0,0,0\rangle, |1,0,0\rangle, |0,1,0\rangle, |0,0,1\rangle, |1,1,0\rangle, |1,0,1\rangle, |0,1,1\rangle, |1,1,1\rangle\}$, la matriz densidad del sistema queda

\begin{equation}
    \hat{\rho}_{S} = 
    \begin{bmatrix}
        \rho_{000} & 0 & 0 & 0 & 0 & 0 & 0 & 0 \\
        0 & \rho_{100} & \alpha & 0 & 0 & 0 & 0 & 0 \\
        0 & \alpha^{*} & \rho_{010} & 0 & 0 & 0 & 0 & 0 \\
        0 & 0 & 0 & \rho_{001} & 0 & 0 & 0 & 0 \\
        0 & 0 & 0 & 0 & \rho_{110} & 0 & 0 & 0 \\
        0 & 0 & 0 & 0 & 0 & \rho_{101} & \beta & 0 \\
        0 & 0 & 0 & 0 & 0 & \beta^{*} & \rho_{011} & 0 \\
        0 & 0 & 0 & 0 & 0 & 0 & 0 & \rho_{111} 
        \end{bmatrix}.
        \label{appendix5rhoconcu}
\end{equation}

En esta matriz no aparecen coherencias entre estados con distinto número de partículas por el principio de superselección de carga \cite{bartlett2007reference,wick1997intrinsic}. Además debido a que el punto cuántico $D$ sólo interactua mediante interacción de Coulomb con los puntos $L$ y $R$, no aparecen coherencias del tipo $L-D$ y $R-D$. Así se puede obtener la matriz densidad del sistema $LR$ en la base $\{|00\rangle, |10\rangle, |01\rangle, |11\rangle \}$

\begin{equation*}
    \hat{\rho}_{LR} = 
    \begin{bmatrix}
        \rho_{000}+\rho_{001} & 0 & 0 & 0  \\
        0 & \rho_{100} + \rho_{101} & \alpha + \beta & 0  \\
        0 & \alpha^{*} +\beta^{*} & \rho_{010} + \rho_{011} & 0  \\
        0 & 0 & 0 & \rho_{110} + \rho_{111} 
        \end{bmatrix},
\end{equation*}
así se podrá calcular la concurrencia 

\begin{equation*}
    \mathcal{C}_{on} = \max \{ 0,\lambda_{1} - \lambda_{2} - \lambda_{3} - \lambda_{4} \},
\end{equation*}
en donde $\lambda_{i}$ son la raíz cuadrada de los autovalores ordenados en forma decreciente de la matriz

\begin{equation*}
    B = \hat{\rho}_{LR} \tilde{\rho}_{LR} \hspace{10mm} \tilde{\rho}_{LR} = (\sigma_{y} \otimes \sigma_{y}) \hat{\rho}_{LR} (\sigma_{y} \otimes \sigma_{y}),
\end{equation*}
en la cuál $\sigma_{y}$ pertenece a las matrices de Pauli, así se obtiene

\begin{equation*}
    \tilde{\rho}_{LR} = 
    \begin{bmatrix}
        p_{D} & 0 & 0 & 0  \\
        0 & p_{R} & \alpha + \beta & 0  \\
        0 & \alpha^{*} +\beta^{*} & p_{L} & 0  \\
        0 & 0 & 0 & p_{0} 
        \end{bmatrix}.
\end{equation*}

Con  $p_{0}=\rho_{000}+\rho_{001}$, $p_{L}=\rho_{100} + \rho_{101}$, $p_{R}=\rho_{010}+\rho_{011}$, $p_{D}=\rho_{110}+\rho_{111}$ la matriz $B$ quedará

\begin{equation*}
    B = 
    \begin{bmatrix}
        p_{0}p_{D} & 0 & 0 & 0  \\
        0 & p_{L}p_{R}+|\alpha+\beta|^{2} & 2(\alpha + \beta)p_{L} & 0  \\
        0 & 2(\alpha^{*} +\beta^{*})p_{R} & p_{L}p_{R}+|\alpha+\beta|^{2}  & 0  \\
        0 & 0 & 0 & p_{0}p_{D}
        \end{bmatrix}.
\end{equation*}

Suponiendo que $p_{D}$ y $p_{0}$ son pequeños, el orden decreciente de los autovalores corresponderá a

\begin{equation*}
    \lambda_{1} = p_{L}p_{R} + |\alpha + \beta| \hspace{10mm}  \lambda_{2} = p_{L}p_{R} - |\alpha + \beta| \hspace{10mm} \lambda_{3}=\lambda_{4} = \sqrt{p_{0}p_{D}},
\end{equation*}
finalmente se obtiene

\begin{equation*}
    \mathcal{C}_{on} = \max\{ 2|\alpha+\beta| - 2\sqrt{p_{0}p_{D}},0\}.
\end{equation*}

\label{appendix5final}

\subsection{Modelo clásico}
Para partir derivando el modelo clásico para las partes diagonales de la matriz densidad $\rho_{ijk}$ con $i,j,k \in \{0,1\}$, primero se notará que la ecuación de operadores

\begin{equation*}
    \frac{d}{dt}\check{\mathcal{Q}}|\hat{\rho}(t)\rangle \rangle = \check{\mathcal{Q}}\check{\mathcal{L}}_{f}\check{\mathcal{Q}}|\hat{\rho}(t)\rangle \rangle + \check{\mathcal{Q}}\check{\mathcal{L}}_{f}\check{\mathcal{P}}|\hat{\rho}(t)\rangle \rangle \to \frac{d}{dt}\check{\mathcal{Q}}|\hat{\rho}(t)\rangle \rangle - \check{\mathcal{Q}}\check{\mathcal{L}}_{f}\check{\mathcal{Q}}|\hat{\rho}(t)\rangle \rangle = \check{\mathcal{Q}}\check{\mathcal{L}}_{f}\check{\mathcal{P}}|\hat{\rho}(t)\rangle \rangle ,
\end{equation*}
es equivalente a una ecuación no homogenea lineal de la forma

\begin{equation*}
    \frac{dy}{dt} - A(t)y = B(t),
\end{equation*}
por lo tanto se puede solucionar utilizando factor integrante

\begin{equation*}
    \check{\mu}(t) = \exp\left( - \int_{0}^{t}\check{\mathcal{Q}}\check{\mathcal{L}}_{f}ds \right),
\end{equation*}
multiplicando por el factor integrante la ecuacion diferencial para $\mu(t)$, se tendrá

\begin{equation*}
    \check{\mu}(t)\frac{d}{dt}\check{\mathcal{Q}}|\hat{\rho}(t)\rangle \rangle - \check{\mu}(t)\check{\mathcal{Q}}\check{\mathcal{L}}_{f}\check{\mathcal{Q}}|\hat{\rho}(t)\rangle \rangle  = \check{\mu}(t)\check{\mathcal{Q}}\check{\mathcal{L}}_{f}\check{\mathcal{P}}|\hat{\rho}(t)\rangle \rangle  \to \frac{d}{dt}(\check{\mu}(t)\check{\mathcal{Q}}|\hat{\rho}(t))\rangle \rangle = \check{\mu}(t)\check{\mathcal{Q}}\check{\mathcal{L}}_{f}\check{\mathcal{P}}|\hat{\rho}(t)\rangle \rangle ,
\end{equation*}
así queda la solución

\begin{equation*}
    \check{\mu}(t)\check{\mathcal{Q}}|\hat{\rho}(t)\rangle \rangle  = \check{\mu}(0)\check{\mathcal{Q}}|\hat{\rho}(0)\rangle \rangle  + \int_{0}^{t}\check{\mu}(\tau)\check{\mathcal{Q}}\check{\mathcal{L}}_{f} \check{\mathcal{P}}|\hat{\rho}(\tau)\rangle \rangle d\tau,   
\end{equation*}
multiplicando por $\check{\mu}^{-1}(t)$ 

\begin{equation*}
    \check{\mathcal{Q}}|\hat{\rho}(t)\rangle \rangle  = \check{\mu}^{-1}(t)\check{\mathcal{Q}}|\hat{\rho}(0)\rangle \rangle  + \int_{0}^{t}\check{\mu}^{-1}(t)\check{\mu}(\tau)\check{\mathcal{Q}}\check{\mathcal{L}}_{f} \check{\mathcal{P}}|\hat{\rho}(\tau)\rangle \rangle d\tau,    
\end{equation*}
para asi definir el operador

\begin{equation*}
    \check{\mathcal{G}}(t,s) = \exp\left(  \int_{s}^{t}\check{\mathcal{Q}}\check{\mathcal{L}}_{f}dt' \right),
\end{equation*}
finalmente

\begin{equation*}
    \check{\mathcal{Q}}|\hat{\rho}(t)\rangle \rangle = \check{\mathcal{G}}(t,0)\check{\mathcal{Q}}|\hat{\rho}(0)\rangle \rangle  + \int_{0}^{t}\check{\mathcal{G}}(t,\tau)\check{\mathcal{Q}}\check{\mathcal{L}}_{f} \check{\mathcal{P}}|\hat{\rho}(\tau)\rangle \rangle d\tau.    
\end{equation*}

Podremos reinsertar esto en la parte principal y quedará

\begin{equation}
    \frac{d}{dt}\check{\mathcal{P}}|\hat{\rho}(t)\rangle \rangle  = \check{\mathcal{P}}\check{\mathcal{L}}_{f}\check{\mathcal{P}}|\hat{\rho}(t)\rangle \rangle  + \check{\mathcal{P}}\check{\mathcal{L}}_{f}\check{\mathcal{G}}(t,0)\check{\mathcal{Q}}|\hat{\rho}(0)\rangle \rangle  + \check{\mathcal{P}}\check{\mathcal{L}}_{f}\int_{0}^{t}\check{\mathcal{G}}(t,\tau)\check{\mathcal{Q}}\check{\mathcal{L}}_{f} \check{\mathcal{P}}|\hat{\rho}(\tau)\rangle \rangle d\tau.    
    \label{apendix5exactp}
\end{equation}

Estas ecuaciones son exactas, ya que hasta ahora no se ha hecho ninguna aproximación. La primera parte del lado derecho de la ecuación \ref{apendix5exactp} consiste en la evolución markoviana, mientras que la segunda y  tercera parte corresponden a un corrimiento inicial y a los efectos de memoria respectivamente. Para seguir la deducción se utilizarán las relaciones

\begin{equation*}
    \check{\mathcal{P}}\check{\mathcal{V}}\check{\mathcal{P}} = [\check{\mathcal{L}}_0,\check{\mathcal{P}}] = [\check{\mathcal{L}}_0,\check{\mathcal{Q}}] = 0, 
\end{equation*}
estas relaciones se pueden explicar intuitivamente, ya que al ser $\check{\mathcal{V}}$ la perturbación que genera el tuneleo entre los puntos $L$ y $R$, está contribución es la que produce efectos de coherencias. Así se puede notar que $\check{\mathcal{V}}\check{\mathcal{P}}\hat{\rho}$ es no diagonal y por lo tanto al aplicar nuevamente el operador que proyecta en la diagonal este se anula.
\\
Para seguir con \ref{apendix5exactp}, primero supondremos que inicialmente no hay coherencias, es decir $\check{\mathcal{Q}}\hat{\rho}(0) = 0$. Segundo se sustituirá $\check{\mathcal{L}}_{f}$ en la ecuación \ref{apendix5exactp} y se usarán las propiedades de los operadores Nakajima-Zwanzig, partiendo por el integral

\begin{align*}
    \check{\mathcal{G}}(t,\tau) & = \exp\left(  \int_{\tau}^{t}\check{\mathcal{Q}}(\check{\mathcal{L}}_{0} + \check{\mathcal{V}})dt' \right) \\
        & = \exp\left(  \int_{\tau}^{t}\check{\mathcal{Q}}(\check{\mathcal{L}}_{0}+\check{\mathcal{V}})(\check{\mathcal{P}}+\check{\mathcal{Q}}) dt' \right) \\
            & = \exp\left(  \int_{\tau}^{t}\left[\check{\mathcal{Q}}\check{\mathcal{L}}_{0}\check{\mathcal{Q}}+ \check{\mathcal{Q}}\check{\mathcal{V}}\check{\mathcal{P}}+\check{\mathcal{Q}}\check{\mathcal{V}}\check{\mathcal{Q}}\right] dt' \right),       
\end{align*}
en donde se uso que $\check{\mathcal{Q}}\check{\mathcal{L}}_{0}\check{\mathcal{P}} = \check{\mathcal{Q}}\check{\mathcal{P}}\check{\mathcal{L}}_{0}=0$. Continuando con la ecuación principal

\begin{equation*}
    \frac{d}{dt}\check{\mathcal{P}}|\hat{\rho}(t)\rangle \rangle = \check{\mathcal{P}}\check{\mathcal{L}}_{0}\check{\mathcal{P}}|\hat{\rho}(t)\rangle \rangle + \check{\mathcal{P}}(\check{\mathcal{L}}_{0} + \check{\mathcal{V}})(\check{\mathcal{P}}+\check{\mathcal{Q}})\int_{0}^{t}\check{\mathcal{G}}(t,\tau)\check{\mathcal{Q}}\check{\mathcal{L}}_{f} \check{\mathcal{P}}|\hat{\rho}(\tau)\rangle \rangle d\tau.       
\end{equation*}

Debido a la forma del exponencial en $\check{\mathcal{G}}$ al aplicar $\check{\mathcal{P}}\check{\mathcal{L}}_{0}\check{\mathcal{G}}(t,\tau)$ se obtiene $\check{\mathcal{P}}\check{\mathcal{L}}_{0}(t-\tau)$, ya que si extendemos el exponencial

\begin{equation*}
    \check{\mathcal{P}}\check{\mathcal{L}}_{0}\exp\left[\int_{\tau}^{t}ds\check{\mathcal{Q}}\check{\mathcal{L}}_{0}\check{\mathcal{Q}}+ \check{\mathcal{Q}}\check{\mathcal{V}}\check{\mathcal{P}}+\check{\mathcal{Q}}\check{\mathcal{V}}\check{\mathcal{Q}} \right] = \check{\mathcal{L}}_{0}\check{\mathcal{P}}\int_{\tau}^{t}\sum_{k=0}^{\infty}\left(\frac{1}{k!} \right)\left[\check{\mathcal{Q}}\check{\mathcal{L}}_{0}\check{\mathcal{Q}}+ \check{\mathcal{Q}}\check{\mathcal{V}}\check{\mathcal{P}}+\check{\mathcal{Q}}\check{\mathcal{V}}\check{\mathcal{Q}} \right]^{k}ds = \check{\mathcal{P}}\check{\mathcal{L}}_{0}(t-\tau), 
\end{equation*}
posteriormente esta contribución se anula y la ecuación principal se reduce a

\begin{equation*}
    \frac{d}{dt}\check{\mathcal{P}}|\hat{\rho}(t)\rangle \rangle = \check{\mathcal{P}}\check{\mathcal{L}}_{0}\check{\mathcal{P}}|\hat{\rho}(t)\rangle \rangle + \check{\mathcal{P}}\check{\mathcal{V}}\check{\mathcal{Q}}\int_{0}^{t}\check{\mathcal{G}}(t,\tau)\check{\mathcal{Q}}\check{\mathcal{V}} \check{\mathcal{P}}|\hat{\rho}(\tau)\rangle \rangle d\tau,        
\end{equation*}
de manera similar se puede notar que el término de la exponencial $\check{\mathcal{Q}}\check{\mathcal{V}}\check{\mathcal{P}}$ no actua en la ecuación, por ende queda que 

\begin{equation*}
    \frac{d}{dt}\check{\mathcal{P}}|\hat{\rho}(t)\rangle \rangle  = \check{\mathcal{P}}\check{\mathcal{L}}_{0}\check{\mathcal{P}}|\hat{\rho}(t)\rangle \rangle  + \check{\mathcal{P}}\check{\mathcal{V}}\check{\mathcal{Q}}\int_{0}^{t}\exp \left[\left(\check{\mathcal{Q}}\check{\mathcal{V}}\check{\mathcal{Q}} + \check{\mathcal{Q}}\check{\mathcal{L}}_{0}\check{\mathcal{Q}}\right) (t-\tau) \right]\check{\mathcal{Q}}\check{\mathcal{V}} \check{\mathcal{P}}|\hat{\rho}(\tau)\rangle \rangle d\tau,        
\end{equation*}
haciendo el cambio de variable $\tau \to t-\tau$

\begin{equation*}
    \frac{d}{dt}\check{\mathcal{P}}|\hat{\rho}(t)\rangle \rangle = \check{\mathcal{P}}\check{\mathcal{L}}_{0}\check{\mathcal{P}}|\hat{\rho}(t)\rangle \rangle  + \check{\mathcal{P}}\check{\mathcal{V}}\check{\mathcal{Q}}\int_{0}^{t}\exp \left[\left(\check{\mathcal{Q}}\check{\mathcal{V}}\check{\mathcal{Q}} + \check{\mathcal{Q}}\check{\mathcal{L}}_{0}\check{\mathcal{Q}}\right) (\tau) \right]\check{\mathcal{Q}}\check{\mathcal{V}} \check{\mathcal{P}}|\hat{\rho}(t-\tau)\rangle \rangle d\tau,        
\end{equation*}
en este momento es donde se hacen aproximaciones, ya que si consideramos el acoplamiento $g$ pequeño, es decir $g \ll \gamma_{i}$. Sabiendo que los autovalores de los superoperadores $\check{\mathcal{V}}$ y $\check{\mathcal{L}}_{0}$ serán proporcionales a $g$ y $\gamma_{i}$ respectivamente,  podremos despreciar la contribución de $\check{\mathcal{V}}$ en la exponencial, así queda

\begin{equation*}
    \frac{d}{dt}\check{\mathcal{P}}|\hat{\rho}(t)\rangle \rangle = \check{\mathcal{P}}\check{\mathcal{L}}_{0}\check{\mathcal{P}}|\hat{\rho}(t)\rangle \rangle + \check{\mathcal{P}}\check{\mathcal{V}}\check{\mathcal{Q}}\int_{0}^{t}\exp \left[\left( \check{\mathcal{Q}}\check{\mathcal{L}}_{0}\check{\mathcal{Q}}\right)\tau \right]\check{\mathcal{Q}}\check{\mathcal{V}} \check{\mathcal{P}}|\hat{\rho}(t-\tau)\rangle \rangle d\tau,        
\end{equation*}
finalmente, para aplicar aproximación de Markov suponemos que $\check{\mathcal{P}}\hat{\rho}(t-\tau)$ no varia mucho en escalas de tiempo de $1/\gamma_{i}$, así poder considerar $\check{\mathcal{P}}\hat{\rho}(t-\tau) = \check{\mathcal{P}}\hat{\rho}(t)$ y el integral de $0$ a el infinito con lo que se obtiene

\begin{equation*}
    \frac{d}{dt}\check{\mathcal{P}}|\hat{\rho}(t)\rangle \rangle = \check{\mathcal{P}}\check{\mathcal{L}}_{0}\check{\mathcal{P}}|\hat{\rho}(t)\rangle \rangle + \check{\mathcal{P}}\check{\mathcal{V}}\check{\mathcal{Q}}\int_{0}^{\infty}\exp \left[\left( \check{\mathcal{Q}}\check{\mathcal{L}}_{0}\check{\mathcal{Q}}\right)\tau \right]d\tau \check{\mathcal{Q}}\check{\mathcal{V}} \check{\mathcal{P}}|\hat{\rho}(t)\rangle \rangle ,        
\end{equation*}
y finalmente se recupera la ecuación en formalismo de superoperadores  

\begin{equation*}
    \frac{d}{dt}\check{\mathcal{P}}|\hat{\rho}_{s}(t)\rangle \rangle = (\check{\mathcal{L}}_{0} - \check{\mathcal{P}}\check{\mathcal{V}}\check{\mathcal{Q}}\check{\mathcal{L}}^{-1}_{0}\check{\mathcal{Q}}\check{\mathcal{V}}\check{\mathcal{P}})\check{\mathcal{P}}|\hat{\rho}_{s}(t)\rangle \rangle. 
\end{equation*}



\label{appendix5clasic}

\subsection{ Inverso de Drazin }
El inverso de Drazin es un tipo de inverso generalizado que consiste en la única matriz que  dado la matriz $\check{\mathcal{L}}_{0}$ cumple que 

\begin{align*}
    \check{\mathcal{L}}_{0}\check{\mathcal{L}}_{0}^{+}\check{\mathcal{L}}_{0} & = \check{\mathcal{L}}_{0} \\
    \check{\mathcal{L}}_{0}^{+}\check{\mathcal{L}}_{0}\check{\mathcal{L}}_{0}^{+} & = \check{\mathcal{L}}_{0}^{+} \\
    \check{\mathcal{L}}_{0}^{+}\check{\mathcal{L}}_{0} & = \check{\mathcal{L}}_{0}\check{\mathcal{L}}_{0}^{+}.
\end{align*}

Considerando la ecuación \ref{ec5:classicalmodel} se puede probar que 

\begin{equation*}
    \check{\mathcal{L}}_{0}\check{\mathcal{L}}_{0}^{+} = - \int_{0}^{\infty}d\tau \check{\mathcal{L}}_{0}\exp[ \check{\mathcal{L}}_{0}\tau] \check{Q}= - \int_{0}^{\infty}d\tau \exp[ \check{\mathcal{L}}_{0}\tau]\check{\mathcal{L}}_{0} \check{Q} = - \int_{0}^{\infty}d\tau \exp[ \check{\mathcal{L}}_{0}\tau] \check{Q} \check{\mathcal{L}}_{0} = \check{\mathcal{L}}_{0}^{+}\check{\mathcal{L}}_{0},
\end{equation*}
de manera similar se pueden probar las otras 2 ecuaciones.\\
 Finalmente para poder ocupar el inverso de Drazin, este se puede calcular numéricamente mediante descomposición espectral usando los autovalores por la izquierda y derecha dados por 

\begin{equation*}
    \check{\mathcal{L}}_{0}|x_{j}\rangle \rangle = \lambda_{j} |x_{j}\rangle \rangle \hspace{10mm}  \langle \langle y_{j}|\check{\mathcal{L}}_{0} = \langle \langle y_{j}|\lambda_{j},
\end{equation*}
en donde $\langle \langle y_{i}|x_{j}\rangle \rangle = \delta_{ij}$. Si se toma $\lambda_{0}$ como el único autovalor cero del operador $\check{\mathcal{L}}_{0}$, se podrá escribir la descomposición espectral como

\begin{equation*}
    \check{\mathcal{L}}_{0} = \sum_{j\neq 0}\lambda_{j}|x_{j}\rangle \rangle \langle \langle y_{j}|,
\end{equation*}
y así poder calcular el inverso de Drazin usando

\begin{equation*}
    \check{\mathcal{L}}^{+}_{0} = \sum_{j\neq 0}\frac{1}{\lambda_{j}}|x_{j}\rangle \rangle \langle \langle y_{j}|.
\end{equation*}

\label{appendix5drazin}

\subsection{Expresión para los flujos de información}
Para poder calcular los flujos de información, primero se debe ser capaces de calcular $\ln \hat{\rho}_{s}$. Se puede partir de la matriz densidad descrita en la ecuación \ref{appendix5rhoconcu} la cuál ya viene diagonal, excepto por los bloques

\begin{equation*}
    \begin{bmatrix}
        \rho_{100} & \alpha \\
        \alpha^{*} & \rho_{010}
    \end{bmatrix}
    \hspace{10mm}
    \begin{bmatrix}
        \rho_{101} & \beta \\
        \beta^{*} & \rho_{011}
    \end{bmatrix}.
\end{equation*}

Los autovalores estarán dados por

\begin{equation*}
    \lambda_{0\pm} = \frac{ \rho_{100} +\rho_{010} }{2} \pm \frac{( [\rho_{100} - \rho_{010}]^{2} + 4|\alpha|^{2} )^{1/2} }{ 2 }  \hspace{10mm}      \lambda_{1\pm} = \frac{ \rho_{101} +\rho_{011} }{2} \pm \frac{( [\rho_{101} - \rho_{011}]^{2} + 4|\beta|^{2} )^{1/2} }{ 2 }, 
\end{equation*}
si se considera $\alpha = |\alpha|e^{i\phi_{0}}$ y $\beta = |\beta|e^{i\phi_{1}}$ podremos escribir los autovectores como

\begin{align*}
    v_{0+} &  = [0,0,\cos(\theta_{0}/2)e^{i\phi_{0}/2},\sen(\theta_{0}/2)e^{-i\phi_{0}/2},0,0,0,0]^{T} \\
    v_{0-} &  = [0,0,-\sen(\theta_{0}/2)e^{i\phi_{0}/2},\cos(\theta_{0}/2)e^{-i\phi_{0}/2},0,0,0,0]^{T} \\
    v_{1+} &  = [0,0,0,0,\cos(\theta_{1}/2)e^{i\phi_{1}/2},\sen(\theta_{1}/2)e^{-i\phi_{1}/2},0,0]^{T} \\
    v_{1-} &  = [0,0,0,0,-\sen(\theta_{1}/2)e^{i\phi_{1}/2},\cos(\theta_{1}/2)e^{-i\phi_{1}/2},0,0]^{T},                   
\end{align*}
con $\sen(\theta_{0}) = |\alpha|/\sqrt{\Delta^{2}_{0} + |\alpha|^{2}}$, $\sen(\theta_{1}) = |\beta|/\sqrt{\Delta^{2}_{1} + |\beta|^{2}}$ y

\begin{equation*}
    \Delta_{i} = \frac{(\rho_{10i} - \rho_{01i})}{2},
\end{equation*}
con esto se podrá obtener el logaritmo natural de la matriz mediante

\begin{equation*}
    \ln \hat{\rho}_{S} = \hat{V}\ln \hat{\rho}_{D} \hat{V}^{-1}.
\end{equation*}
    
En donde $\ln \hat{\rho}_{D}$ es la matriz diagonal 

\begin{equation*}
    \ln \hat{\rho}_{D} = 
    \begin{bmatrix}
        \ln \rho_{000} & 0 & 0 & 0 & 0 & 0 & 0 & 0 \\
        0 & \ln \lambda_{0+} & 0 & 0 & 0 & 0 & 0 & 0 \\
        0 & 0 & \ln \lambda_{0-} & 0 & 0 & 0 & 0 & 0 \\
        0 & 0 & 0 & \ln \rho_{001} & 0 & 0 & 0 & 0 \\
        0 & 0 & 0 & 0 & \ln \rho_{110} & 0 & 0 & 0 \\
        0 & 0 & 0 & 0 & 0 & \ln \lambda_{1+} & 0 & 0 \\
        0 & 0 & 0 & 0 & 0 & 0 & \ln \lambda_{1-} & 0 \\
        0 & 0 & 0 & 0 & 0 & 0 & 0 & \ln \rho_{111} 
        \end{bmatrix},
\end{equation*}
y 

\begin{equation*}
    \hat{V} = 
    \begin{bmatrix}
        1 & 0 & 0 & 0 & 0 & 0 & 0 & 0 \\
        0 & \cos(\theta_{0}/2)e^{i\phi_{0}/2} & -\sen(\theta_{0}/2)e^{i\phi_{0}/2} & 0 & 0 & 0 & 0 & 0 \\
        0 & \sen(\theta_{0}/2)e^{-i\phi_{0}/2} & \cos(\theta_{0}/2)e^{-i\phi_{0}/2} & 0 & 0 & 0 & 0 & 0 \\
        0 & 0 & 0 & 1 & 0 & 0 & 0 & 0 \\
        0 & 0 & 0 & 0 & 1 & 0 & 0 & 0 \\
        0 & 0 & 0 & 0 & 0 &  \cos(\theta_{1}/2)e^{i\phi_{1}/2} & -\sen(\theta_{1}/2)e^{i\phi_{1}/2} & 0 \\
        0 & 0 & 0 & 0 & 0 & \sen(\theta_{1}/2)e^{-i\phi_{1}/2} & \cos(\theta_{1}/2)e^{-i\phi_{1}/2} & 0 \\
        0 & 0 & 0 & 0 & 0 & 0 & 0 & 1 
        \end{bmatrix}.
\end{equation*}

Finalmente se obtiene

\begin{equation*}
    \ln \hat{\rho}_{S} = 
    \begin{bmatrix}
        \ln \rho_{000} & 0 & 0 & 0 & 0 & 0 & 0 & 0 \\
        0 & a_{0} & c_{0} & 0 & 0 & 0 & 0 & 0 \\
        0 & c^{*}_{0} & b_{0} & 0 & 0 & 0 & 0 & 0 \\
        0 & 0 & 0 & \ln \rho_{001} & 0 & 0 & 0 & 0 \\
        0 & 0 & 0 & 0 & \ln \rho_{110} & 0 & 0 & 0 \\
        0 & 0 & 0 & 0 & 0 & a_{1} & c_{1} & 0 \\
        0 & 0 & 0 & 0 & 0 & c^{*}_{1} & b_{1} & 0 \\
        0 & 0 & 0 & 0 & 0 & 0 & 0 & \ln \rho_{111} 
        \end{bmatrix},
\end{equation*}
con las cantidades 

\begin{align*}
    a_{i} & = \cos^{2}(\theta_{i}/2) \ln \lambda_{i+} + \sen^{2}(\theta_{i}/2) \ln \lambda_{i-} \\
    b_{i} & = \sen^{2}(\theta_{i}/2) \ln \lambda_{i+} + \cos^{2}(\theta_{i}/2) \ln \lambda_{i-}  \\
    c_{i} & = \frac{\sin(\theta_{i})}{2}e^{i\phi_{i}} (\ln \lambda_{i+} - \ln \lambda_{i-}),    
\end{align*}
así el logaritmo de la matriz densidad es 

\begin{align*}
    \ln \hat{\rho}_{S} & = \ln \rho_{111}|111\rangle \langle 111| + \ln \rho_{110}|110\rangle \langle 110| + \ln \rho_{001}|001\rangle \langle 001| \\ 
     & + \ln \rho_{000}|000\rangle \langle 000| + a_{0}|100\rangle \langle 100| + b_{0}|010\rangle \langle 010|  \\  
     & + a_{1}|101\rangle \langle 101| + b_{1}|011\rangle\langle 011| + c_{0}|100\rangle \langle 010| \\ 
     & + c^{*}_{0}|010\rangle \langle 100| + c_{1}|101\rangle \langle 011| + c^{*}_{1}|011\rangle \langle 101|.
\end{align*}

Por otro lado, los disipadores que actuan en los operadores de Lindblad $\mathcal{L}_{L}$ y $\mathcal{L}_{R}$ se pueden expresar 

\begin{align*}
    \mathcal{D}[\hat{d}^{\dagger}_{L}(\textbf{1}-\hat{n}_{D})(\textbf{1}-\hat{n}_{R})\hat{\rho}_{S}] & = \rho_{000}|100\rangle \langle 100| - \rho_{000}|000\rangle \langle 000| \\ 
    \mathcal{D}[\hat{d}^{\dagger}_{R}(\textbf{1}-\hat{n}_{D})(\textbf{1}-\hat{n}_{L})\hat{\rho}_{S}] & = \rho_{000}|010\rangle \langle 010| - \rho_{000}|000\rangle \langle 000| \\  
    \mathcal{D}[\hat{d}_{L}(\textbf{1}-\hat{n}_{D})(\textbf{1}-\hat{n}_{R})\hat{\rho}_{S}] & = \rho_{100}|000\rangle \langle 000| - \rho_{100}|100\rangle \langle 100|\\ 
    \mathcal{D}[\hat{d}_{R}(\textbf{1}-\hat{n}_{D})(\textbf{1}-\hat{n}_{L})\hat{\rho}_{S}] & = \rho_{010}|000\rangle \langle 000| - \rho_{010}|010\rangle \langle 010| \\ 
    \mathcal{D}[\hat{d}^{\dagger}_{L}\hat{n}_{D} \hat{n}_{R}\hat{\rho}_{S}] & = \rho_{011}|111\rangle \langle 111| - \rho_{011}|011\rangle \langle 011| \\  
    \mathcal{D}[\hat{d}^{\dagger}_{R} \hat{n}_{D}\hat{n}_{L}\hat{\rho}_{S}] & = \rho_{101}|111\rangle \langle 111| - \rho_{101}|101\rangle \langle 101| \\ 
     \mathcal{D}[\hat{d}_{L}\hat{n}_{D}\hat{n}_{R}\hat{\rho}_{S}] & = \rho_{111}|011\rangle \langle 011| - \rho_{111}|111\rangle \langle 111|  \\    
      \mathcal{D}[\hat{d}_{R}\hat{n}_{D}\hat{n}_{L}\hat{\rho}_{S}] & = \rho_{111}|101\rangle \langle 101| - \rho_{111}|111\rangle \langle 111|  \\ 
    \mathcal{D}[\hat{d}^{\dagger}_{L}(\textbf{1}-\hat{n}_{D})\hat{n}_{R}\hat{\rho}_{S}] & = \rho_{010}|110\rangle \langle 110| - \rho_{010}|010\rangle \langle 010| - \frac{1}{2}( \alpha|100\rangle \langle 010| + \alpha^{*}|010\rangle \langle 100|) \\ 
    \mathcal{D}[\hat{d}^{\dagger}_{R}(\textbf{1}-\hat{n}_{D})\hat{n}_{L}\hat{\rho}_{S}] & = \rho_{100}|110\rangle \langle 110| - \rho_{100}|100\rangle \langle 100| - \frac{1}{2}( \alpha |100\rangle \langle 010| + \alpha^{*}|010\rangle \langle 100| )  \\  
    \mathcal{D}[\hat{d}_{L}(\textbf{1}-\hat{n}_{D})\hat{n}_{R}\hat{\rho}_{S}] & = \rho_{110}|010\rangle \langle 010| - \rho_{110}|110\rangle \langle 110| \\ 
    \mathcal{D}[\hat{d}_{R}(\textbf{1}-\hat{n}_{D})\hat{n}_{L}\hat{\rho}_{S}] & = \rho_{110}|100\rangle \langle 100| - \rho_{110}|110\rangle \langle 110| \\  
    \mathcal{D}[\hat{d}^{\dagger}_{L}\hat{n}_{D}(\textbf{1}-\hat{n}_{R})\hat{\rho}_{S}] & = \rho_{001}|101\rangle \langle 101| - \rho_{001}|001\rangle \langle 001| \\ 
    \mathcal{D}[\hat{d}^{\dagger}_{R}\hat{n}_{D}(\textbf{1}-\hat{n}_{L})\hat{\rho}_{S}] & = \rho_{001}|011\rangle \langle 011| - \rho_{001}|001\rangle \langle 001| \\  
    \mathcal{D}[\hat{d}_{L}\hat{n}_{D}(\textbf{1}-\hat{n}_{R})\hat{\rho}_{S}] & = \rho_{101}|001\rangle \langle 001| - \rho_{101}|101\rangle \langle 101| - \frac{1}{2}(\beta |101\rangle \langle 011| + \beta^{*}|011\rangle \langle 101| ) \\ 
    \mathcal{D}[\hat{d}_{R}\hat{n}_{D}(\textbf{1}-\hat{n}_{L})\hat{\rho}_{S}] & = \rho_{011}|001\rangle \langle 001| - \rho_{011}|011\rangle \langle 011| - \frac{1}{2}(\beta^{*} |011\rangle \langle 101| + \beta|101\rangle \langle 011|  ),   
\end{align*}
con estas cantidades se podrán obtener los flujos de información en el estado estacionario $\dot{I}_{i} = \text{Tr}[(\mathcal{L}_{i}\hat{\rho}_{S} \ln \hat{\rho_{S}}) ]$ lo que permite obtener

\begin{align*}
    \dot{I}_{L} &  =  \gamma_{L}(\epsilon)(f_{L}(\epsilon)[a_{0}\rho_{000} - \rho_{000}\ln \rho_{000}] + (1-f_{L}(\epsilon))[\rho_{100}\ln \rho_{000} - a_{0}\rho_{100} ] )  \\ 
      & + \gamma_{L}(\epsilon + U)(f_{L}(\epsilon + U)[a_{1}\rho_{001} - \rho_{001}\ln \rho_{001}] + (1-f_{L}(\epsilon + U))[\rho_{101}\ln \rho_{001} -a_{1}\rho_{101} - Re(b^{*}_{1}\beta ) ]   ) \\  
      & + \gamma_{L}(\epsilon + U_{f})( f_{L}(\epsilon + U_{f})[\rho_{010}\ln \rho_{110} -b_{0}\rho_{010} - Re(\alpha c^{*}_{0}) ]  + (1-f_{L}(\epsilon + U_{f}))[b_{0}\rho_{110} - \rho_{110}\ln \rho_{110} ] ) \\  
      & + \gamma_{L}(\epsilon + U + U_{f})( f_{L}(\epsilon + U + U_{f})[\rho_{011}\ln \rho_{111}- b_{1}\rho_{011}] + (1-f_{L}(\epsilon+U+U_{f}) )[b_{1}\rho_{111} - \rho_{111}\ln \rho_{111}]  ) 
\end{align*}

\begin{align*}
    \dot{I}_{R} &  =  \gamma_{R}(\epsilon)(f_{L}(\epsilon)[b_{0}\rho_{000} - \rho_{000}\ln \rho_{000}] + (1-f_{R}(\epsilon))[\rho_{010}\ln \rho_{000} - b_{0}\rho_{010} ] )  \\ 
      & + \gamma_{R}(\epsilon + U)(f_{R}(\epsilon + U)[b_{1} \rho_{001} - \rho_{001}\ln \rho_{001}] + (1-f_{R}(\epsilon + U))[\rho_{011}\ln \rho_{001} -a_{1}\rho_{011} - Re(b^{*}_{1}\beta ) ]   ) \\  
      & + \gamma_{R}(\epsilon + U_{f})( f_{R}(\epsilon + U_{f})[\rho_{100}\ln \rho_{110} -a_{0}\rho_{100} - Re(\alpha c^{*}_{0}) ]  + (1-f_{R}(\epsilon + U_{f}))[a_{0}\rho_{110} - \rho_{110}\ln \rho_{110} ] ) \\  
      & + \gamma_{R}(\epsilon + U + U_{f})( f_{R}(\epsilon + U + U_{f})[\rho_{101}\ln \rho_{111}- a_{1}\rho_{101}] + (1-f_{R}(\epsilon+U+U_{f}) )[a_{1}\rho_{111} - \rho_{111}\ln \rho_{111}]  ), 
\end{align*}
si bien el cálculo de los flujos de información se puede hacer númerico, escribir estas expresiones analíticas nos permite determinar que existe contribución de las coherencias en la información, tanto en los parámetros $a_{i},b_{i}$ como en los términos $Re(\alpha c^{*}_{0}),Re(b^{*}_{1} \beta)$ . Es importante destacar que en el  límite en que las coherencias se anulan se recupera el resultado para un sistema clásico

\begin{align*}
 \lim_{\alpha,\beta \to 0}  & \dot{I}_{L}  = \dot{I}_{Lclassic}   =  \gamma_{L}(\epsilon)(f_{L}(\epsilon)[\rho_{000}\ln \rho_{100} - \rho_{000}\ln \rho_{000}] + (1-f_{L}(\epsilon))[\rho_{100}\ln \rho_{000} - \rho_{100}\ln \rho_{100} ] )  \\ 
      & + \gamma_{L}(\epsilon + U)(f_{L}(\epsilon + U)[\rho_{001}\ln \rho_{101} - \rho_{001}\ln \rho_{001}] + (1-f_{L}(\epsilon + U))[\rho_{101}\ln \rho_{001} -\rho_{101}\ln \rho_{101} ]   ) \\  
      & + \gamma_{L}(\epsilon + U_{f})( f_{L}(\epsilon + U_{f})[\rho_{010}\ln \rho_{110} -\rho_{010}\ln \rho_{010} ]  + (1-f_{L}(\epsilon + U_{f}))[\rho_{110}\ln \rho_{010} - \rho_{110}\ln \rho_{110} ] ) \\  
      & + \gamma_{L}(\epsilon + U + U_{f})( f_{L}(\epsilon + U + U_{f})[\rho_{011}\ln \rho_{111}- \rho_{011}\ln \rho_{011}]) \\   
      & + \gamma_{L}(\epsilon + U + U_{f})( (1-f_{L}(\epsilon+U+U_{f}) )[\rho_{111}\ln \rho_{011} - \rho_{111}\ln \rho_{111}])   
\end{align*}

\begin{align*}
 \lim_{\alpha,\beta \to 0}  & \dot{I}_{R}  = \dot{I}_{Rclassic}   =  \gamma_{R}(\epsilon)(f_{R}(\epsilon)[\rho_{000}\ln \rho_{010} - \rho_{000}\ln \rho_{000}] + (1-f_{R}(\epsilon))[\rho_{010}\ln \rho_{000} - \rho_{010}\ln \rho_{010} ] )  \\ 
      & + \gamma_{R}(\epsilon + U)(f_{R}(\epsilon + U)[\rho_{001}\ln \rho_{011} - \rho_{001}\ln \rho_{001}] + (1-f_{R}(\epsilon + U))[\rho_{011}\ln \rho_{001} -\rho_{011}\ln \rho_{011} ]   ) \\  
      & + \gamma_{R}(\epsilon + U_{f})( f_{R}(\epsilon + U_{f})[\rho_{100}\ln \rho_{110} -\rho_{100}\ln \rho_{100} ]  + (1-f_{R}(\epsilon + U_{f}))[\rho_{110}\ln \rho_{100} - \rho_{110}\ln \rho_{110} ] ) \\  
      & + \gamma_{R}(\epsilon + U + U_{f})( f_{R}(\epsilon + U + U_{f})[\rho_{101}\ln \rho_{111}- \rho_{101}\ln \rho_{101}]) \\   
      & + \gamma_{R}(\epsilon + U + U_{f})( (1-f_{R}(\epsilon+U+U_{f}) )[\rho_{111}\ln \rho_{101} - \rho_{111}\ln \rho_{111}]).   
\end{align*}

\label{apendix5infoflow}

\end{appendixs}